%% Generated by Sphinx.
\def\sphinxdocclass{report}
\documentclass[a4paper,10pt,english]{sphinxmanual}
\ifdefined\pdfpxdimen
   \let\sphinxpxdimen\pdfpxdimen\else\newdimen\sphinxpxdimen
\fi \sphinxpxdimen=.75bp\relax

\PassOptionsToPackage{warn}{textcomp}
\usepackage[utf8]{inputenc}
\ifdefined\DeclareUnicodeCharacter
 \ifdefined\DeclareUnicodeCharacterAsOptional
  \DeclareUnicodeCharacter{"00A0}{\nobreakspace}
  \DeclareUnicodeCharacter{"2500}{\sphinxunichar{2500}}
  \DeclareUnicodeCharacter{"2502}{\sphinxunichar{2502}}
  \DeclareUnicodeCharacter{"2514}{\sphinxunichar{2514}}
  \DeclareUnicodeCharacter{"251C}{\sphinxunichar{251C}}
  \DeclareUnicodeCharacter{"2572}{\textbackslash}
 \else
  \DeclareUnicodeCharacter{00A0}{\nobreakspace}
  \DeclareUnicodeCharacter{2500}{\sphinxunichar{2500}}
  \DeclareUnicodeCharacter{2502}{\sphinxunichar{2502}}
  \DeclareUnicodeCharacter{2514}{\sphinxunichar{2514}}
  \DeclareUnicodeCharacter{251C}{\sphinxunichar{251C}}
  \DeclareUnicodeCharacter{2572}{\textbackslash}
 \fi
\fi
\usepackage{cmap}
\usepackage[T1]{fontenc}
\usepackage{amsmath,amssymb,amstext}
\usepackage{babel}
\usepackage{times}
\usepackage[Bjarne]{fncychap}
\usepackage{sphinx}

\usepackage{geometry}

% Include hyperref last.
\usepackage{hyperref}
% Fix anchor placement for figures with captions.
\usepackage{hypcap}% it must be loaded after hyperref.
% Set up styles of URL: it should be placed after hyperref.
\urlstyle{same}

\addto\captionsenglish{\renewcommand{\figurename}{Fig.}}
\addto\captionsenglish{\renewcommand{\tablename}{Table}}
\addto\captionsenglish{\renewcommand{\literalblockname}{Listing}}

\addto\captionsenglish{\renewcommand{\literalblockcontinuedname}{continued from previous page}}
\addto\captionsenglish{\renewcommand{\literalblockcontinuesname}{continues on next page}}

\addto\extrasenglish{\def\pageautorefname{page}}

\setcounter{tocdepth}{1}

\setcounter{tocdepth}{2}


\title{odoo}
\date{May 17, 2018}
\release{11.0}
\author{}
\newcommand{\sphinxlogo}{\vbox{}}
\renewcommand{\releasename}{Release}
\makeindex

\begin{document}

\maketitle
\sphinxtableofcontents
\phantomsection\label{\detokenize{index::doc}}



\chapter{Tutorials}
\label{\detokenize{tutorials:tutorials}}\label{\detokenize{tutorials:index}}\label{\detokenize{tutorials::doc}}

\section{Theme Tutorial}
\label{\detokenize{howtos/themes:theme-tutorial}}\label{\detokenize{howtos/themes::doc}}
Odoo celebrates freedom. Freedom for the designer to go further and
freedom for the user to customize everything according to their needs.

Ready to create your own theme? Great. Here are some things you should know before you begin. This tutorial is a guide to creating an Odoo theme.

\noindent\sphinxincludegraphics{{Intro}.jpg}


\subsection{An introduction for web designers}
\label{\detokenize{howtos/themes:an-introduction-for-web-designers}}
If you are a web designer using Odoo for the first time, you are in the right place.
This introduction will outline the basics of Odoo theme creation.

\begin{sphinxadmonition}{note}{Note:}
Odoo’s team has created a framework that’s powerful and easy to use. There’s no need to know special syntaxes to use this set of tools.
\end{sphinxadmonition}


\subsubsection{From common CMS to Odoo}
\label{\detokenize{howtos/themes:from-common-cms-to-odoo}}
\begin{sphinxadmonition}{note}{Note:}
If you always think and work in the same way, you’ll probably get the same results. If you want something completely new,  then try something different.
\end{sphinxadmonition}
\begin{quote}

Where is my header.php file?
\end{quote}

This is usually the first question from a web designer used to working with Wordpress or Joomla and coming to Odoo for the first time.

\noindent\sphinxincludegraphics{{cms}.jpg}

Indeed, when using common CMSs, you have to code several files (like header.php, page.php, post.php, etc.) in order to create a basic structure for your website. With those systems, this base structure acts as a design foundation that you have to update over time to ensure compatibility within your CMS. So, even after you have spent hours coding the files, you have not even started on the design yet.

This \sphinxstylestrong{does not} apply to creating Odoo themes.

\begin{sphinxadmonition}{note}{Note:}
We think that theme design should be simple (and powerful). When we created our Website Builder, we decided to start from scratch instead of relying on what already existed. This approach gave us the freedom to focus on the things that are really important for designers: styles, content and the logic behind them. No more struggling with technical stuff.
\end{sphinxadmonition}


\subsubsection{Odoo default theme structure}
\label{\detokenize{howtos/themes:odoo-default-theme-structure}}
Odoo comes with a default theme structure.
It is a very basic “theme” that provides minimal structure and layout. When you create a new theme, you are actually extending this.
Indeed it’s always enabled in your setup and it acts exactly like the CMS’s base structure we mentioned above, except that you don’t have to create or maintain it.
It will upgrade automatically within your Odoo installation and, since it is included in the Website Builder module, everything is smoothly integrated by default.

As a result, you are totally free to focus on design while this structure does the job of providing integrations and functionality.

\noindent\sphinxincludegraphics{{def_structure}.jpg}

\sphinxstylestrong{Main features:}
\begin{itemize}
\item {} 
Basic layouts for pages, blog and eCommerce

\item {} 
Website Builder integration

\item {} 
Basic Snippets

\item {} 
Automatic Less/Sass compiling

\item {} 
Automatic Js and CSS minification and combination

\end{itemize}

\sphinxstylestrong{Main technologies:}
\begin{itemize}
\item {} 
Twitter Bootstrap

\item {} 
jQuery

\item {} 
jQuery UI

\item {} 
underscore.js

\end{itemize}


\subsection{Thinking “modular”}
\label{\detokenize{howtos/themes:thinking-modular}}
An Odoo theme is not a folder containing HTML or PHP files, it’s a modular framework written in XML. Never worked with XML files before? Don’t worry, after following the tutorial, you’ll be able to create your first theme with only basic knowledge of HTML.

Using classical web design workflows, you usually code the layout of the entire page. The result of this is a “static” web page. You can update the content, of course, but your client will need you to work on making even basic changes.

Creating themes for Odoo is a total change of perspective. Instead of defining the complete layout for a page, you can create blocks (snippets) at let the user choose where to “drag\&drop” them, creating the page layout on their own.
We call this modular design.

Imagine an Odoo theme as a “list” of elements and options that you have to create and style.
As a designer, your goal is to style these elements in order to achieve a wonderful result, regardless of where the end user chooses to place them.

Let’s take a tour of our “list” elements:

\begin{figure}[htbp]
\centering
\capstart

\noindent\sphinxincludegraphics{{snippet}.jpg}
\caption{Snippets (or building-blocks)}
\begin{sphinxlegend}
A piece of HTML code.  The user  will  drag\&drop, modify and combine them using our built-in Website Builder interface. You can define sets of options and styles for each snippet. The user will choose from them according to their needs.
\end{sphinxlegend}
\label{\detokenize{howtos/themes:id1}}\end{figure}

\begin{figure}[htbp]
\centering
\capstart

\noindent\sphinxincludegraphics{{page}.jpg}
\caption{Pages}
\begin{sphinxlegend}
These are normal web pages, except that they will be editable by the final user and that you can define an empty area that the user can “fill” by dragging snippets into it.
\end{sphinxlegend}
\label{\detokenize{howtos/themes:id2}}\end{figure}



\begin{figure}[htbp]
\centering
\capstart

\noindent\sphinxincludegraphics{{styles}.jpg}
\caption{Styles}
\begin{sphinxlegend}
Styles are defined using standard CSS files (or Less/Sass). You can define a style as \sphinxstylestrong{default} or \sphinxstylestrong{optional}. The default styles are always active in your theme, the optional styles can be enabled or disabled by the user.
\end{sphinxlegend}
\label{\detokenize{howtos/themes:id3}}\end{figure}

\begin{figure}[htbp]
\centering
\capstart

\noindent\sphinxincludegraphics{{functionalities}.jpg}
\caption{Functionalities}
\begin{sphinxlegend}
Thanks to Odoo’s modularity, everything can be personalized even more. This means there are endless possibilities for your creativity. Adding functionalities is easy and it’s simple to provide the end user with customizable options.
\end{sphinxlegend}
\label{\detokenize{howtos/themes:id4}}\end{figure}


\subsubsection{Odoo’s XML files, an overview}
\label{\detokenize{howtos/themes:odoo-s-xml-files-an-overview}}
Any Odoo XML file starts with encoding specifications.
After that, you have to write your code inside a \sphinxcode{\sphinxupquote{\textless{}odoo\textgreater{}}} tag.

\fvset{hllines={, ,}}%
\begin{sphinxVerbatim}[commandchars=\\\{\}]
[XML]
\PYG{c+cp}{\PYGZlt{}?xml version=\PYGZdq{}1.0\PYGZdq{} encoding=\PYGZdq{}utf\PYGZhy{}8\PYGZdq{} ?\PYGZgt{}}
\PYG{n+nt}{\PYGZlt{}odoo}\PYG{n+nt}{\PYGZgt{}}

    \PYGZsh{}\PYGZsh{} YOUR CODE HERE

\PYG{n+nt}{\PYGZlt{}/odoo\PYGZgt{}}
\end{sphinxVerbatim}

Almost every element and option that you create has to be placed inside a \sphinxcode{\sphinxupquote{\textless{}template\textgreater{}}} tag, like in this example.

\fvset{hllines={, ,}}%
\begin{sphinxVerbatim}[commandchars=\\\{\}]
[XML]
\PYG{n+nt}{\PYGZlt{}template} \PYG{n+na}{id=}\PYG{l+s}{\PYGZdq{}my\PYGZus{}title\PYGZdq{}} \PYG{n+na}{name=}\PYG{l+s}{\PYGZdq{}My title\PYGZdq{}}\PYG{n+nt}{\PYGZgt{}}
  \PYG{n+nt}{\PYGZlt{}h1}\PYG{n+nt}{\PYGZgt{}}This is an HTML block\PYG{n+nt}{\PYGZlt{}/h1\PYGZgt{}}
  \PYG{n+nt}{\PYGZlt{}h2} \PYG{n+na}{class=}\PYG{l+s}{\PYGZdq{}lead\PYGZdq{}}\PYG{n+nt}{\PYGZgt{}}And this is a subtitle\PYG{n+nt}{\PYGZlt{}/h2\PYGZgt{}}
\PYG{n+nt}{\PYGZlt{}/template\PYGZgt{}}
\end{sphinxVerbatim}

\begin{sphinxadmonition}{important}{Important:}
don’t misunderstand what \sphinxcode{\sphinxupquote{template}} means. A template tag only
defines a piece of html code or options - but it does not
necessarily coincide with a visual arrangement of elements.
\end{sphinxadmonition}

The previous code defines a title, but it will not be displayed
anywhere because that \sphinxstyleemphasis{template} is not associated with any part of
the \sphinxstylestrong{Odoo default structure}.  In order to do that you can use
\sphinxstylestrong{xpath}, \sphinxstylestrong{qWeb} or a combination of both.

Keep reading the tutorial to learn to how properly extend it with your own code.


\subsubsection{Update your theme}
\label{\detokenize{howtos/themes:update-your-theme}}
Since XML files are only loaded when you install the theme, you will have to force reloading every time you make changes on an xml file.

To do that, click on the Upgrade button in the module’s page.

\noindent\sphinxincludegraphics{{restart}.png}

\noindent\sphinxincludegraphics{{upgrade_module}.png}


\subsection{Create a theme module}
\label{\detokenize{howtos/themes:create-a-theme-module}}
Odoo’s themes are packaged like modules. Even if you are designing a very simple website for your company or client, you need to package the theme like an Odoo module.
\begin{description}
\item[{\sphinxcode{\sphinxupquote{main folder}}}] \leavevmode
Create a folder and name it like this: \sphinxcode{\sphinxupquote{theme\_}} followed by your
theme’s name.

\item[{\sphinxcode{\sphinxupquote{\_\_manifest\_\_.py}}}] \leavevmode
Create an empty document and save it to your folder as
\sphinxcode{\sphinxupquote{\_\_manifest\_\_.py}}. This will contain the configuration info for
your theme.

\item[{\sphinxcode{\sphinxupquote{\_\_init\_\_.py}}}] \leavevmode
Create another empty file and name it \sphinxcode{\sphinxupquote{\_\_init\_\_.py}}. It’s a
mandatory system file. Create and leave it blank.

\item[{\sphinxcode{\sphinxupquote{views}} and \sphinxcode{\sphinxupquote{static}} folders}] \leavevmode
Create them in the main folder. In \sphinxcode{\sphinxupquote{views}} you’ll place your xml
files that define your snippets, your pages and your
options. \sphinxcode{\sphinxupquote{static}} folder is the right place for your style ,
images and custom js code.

\end{description}

\begin{sphinxadmonition}{important}{Important:}
Use two underscore characters at the beginning
and two at the end of odoo and init file names.
\end{sphinxadmonition}

The final result should be something like this:

\noindent\sphinxincludegraphics{{folder}.jpg}


\subsubsection{Edit \sphinxstyleliteralintitle{\sphinxupquote{\_\_manifest\_\_.py}}}
\label{\detokenize{howtos/themes:edit-manifest-py}}
Open the \sphinxcode{\sphinxupquote{\_\_manifest\_\_.py}} you created and copy/paste the following:

\fvset{hllines={, ,}}%
\begin{sphinxVerbatim}[commandchars=\\\{\}]
\PYG{p}{\PYGZob{}}
  \PYG{l+s+s1}{\PYGZsq{}}\PYG{l+s+s1}{name}\PYG{l+s+s1}{\PYGZsq{}}\PYG{p}{:}\PYG{l+s+s1}{\PYGZsq{}}\PYG{l+s+s1}{Tutorial theme}\PYG{l+s+s1}{\PYGZsq{}}\PYG{p}{,}
  \PYG{l+s+s1}{\PYGZsq{}}\PYG{l+s+s1}{description}\PYG{l+s+s1}{\PYGZsq{}}\PYG{p}{:} \PYG{l+s+s1}{\PYGZsq{}}\PYG{l+s+s1}{A description for your theme.}\PYG{l+s+s1}{\PYGZsq{}}\PYG{p}{,}
  \PYG{l+s+s1}{\PYGZsq{}}\PYG{l+s+s1}{version}\PYG{l+s+s1}{\PYGZsq{}}\PYG{p}{:}\PYG{l+s+s1}{\PYGZsq{}}\PYG{l+s+s1}{1.0}\PYG{l+s+s1}{\PYGZsq{}}\PYG{p}{,}
  \PYG{l+s+s1}{\PYGZsq{}}\PYG{l+s+s1}{author}\PYG{l+s+s1}{\PYGZsq{}}\PYG{p}{:}\PYG{l+s+s1}{\PYGZsq{}}\PYG{l+s+s1}{Your name}\PYG{l+s+s1}{\PYGZsq{}}\PYG{p}{,}

  \PYG{l+s+s1}{\PYGZsq{}}\PYG{l+s+s1}{data}\PYG{l+s+s1}{\PYGZsq{}}\PYG{p}{:} \PYG{p}{[}
  \PYG{p}{]}\PYG{p}{,}
  \PYG{l+s+s1}{\PYGZsq{}}\PYG{l+s+s1}{category}\PYG{l+s+s1}{\PYGZsq{}}\PYG{p}{:} \PYG{l+s+s1}{\PYGZsq{}}\PYG{l+s+s1}{Theme/Creative}\PYG{l+s+s1}{\PYGZsq{}}\PYG{p}{,}
  \PYG{l+s+s1}{\PYGZsq{}}\PYG{l+s+s1}{depends}\PYG{l+s+s1}{\PYGZsq{}}\PYG{p}{:} \PYG{p}{[}\PYG{l+s+s1}{\PYGZsq{}}\PYG{l+s+s1}{website}\PYG{l+s+s1}{\PYGZsq{}}\PYG{p}{]}\PYG{p}{,}
\PYG{p}{\PYGZcb{}}
\end{sphinxVerbatim}

Replace the first four property’s values with anything you like.
These values will be used to identify your new theme in Odoo’s backend.

The \sphinxcode{\sphinxupquote{data}} property will contain the xml files list. Right now it’s empty, but we will add any new files created.

\sphinxcode{\sphinxupquote{category}} defines your module category (always “Theme”) and, after a slash, the subcategory. You can use one subcategory from the Odoo Apps categories list. (\sphinxurl{https://www.odoo.com/apps/themes})

\sphinxcode{\sphinxupquote{depends}} specifies the modules needed by our theme to work properly. For our tutorial theme, we only need website. If you need blogging or eCommerce features as well, you have to add those modules too.

\fvset{hllines={, ,}}%
\begin{sphinxVerbatim}[commandchars=\\\{\}]
\PYG{o}{.}\PYG{o}{.}\PYG{o}{.}
\PYG{l+s+s1}{\PYGZsq{}}\PYG{l+s+s1}{depends}\PYG{l+s+s1}{\PYGZsq{}}\PYG{p}{:} \PYG{p}{[}\PYG{l+s+s1}{\PYGZsq{}}\PYG{l+s+s1}{website}\PYG{l+s+s1}{\PYGZsq{}}\PYG{p}{,} \PYG{l+s+s1}{\PYGZsq{}}\PYG{l+s+s1}{website\PYGZus{}blog}\PYG{l+s+s1}{\PYGZsq{}}\PYG{p}{,} \PYG{l+s+s1}{\PYGZsq{}}\PYG{l+s+s1}{sale}\PYG{l+s+s1}{\PYGZsq{}}\PYG{p}{]}\PYG{p}{,}
\PYG{o}{.}\PYG{o}{.}\PYG{o}{.}
\end{sphinxVerbatim}


\subsubsection{Installing your theme}
\label{\detokenize{howtos/themes:installing-your-theme}}
To install your theme, you just place your theme folder inside addons in your Odoo installation.

After that, navigate to the Settings page, look for your theme and click on the install button.


\subsection{Structure of an Odoo page}
\label{\detokenize{howtos/themes:structure-of-an-odoo-page}}
An Odoo page is the visual result of a combination of 2 kind of elements, \sphinxstylestrong{cross-pages} and \sphinxstylestrong{unique}.
By default, Odoo provides you with a \sphinxstylestrong{Header} and a \sphinxstylestrong{Footer} (cross-pages) and a unique main element that contains the content that makes your page unique.

\begin{sphinxadmonition}{note}{Note:}
Cross-pages elements will be the same on every page. Unique elements are related to a specific page only.
\end{sphinxadmonition}

\noindent\sphinxincludegraphics{{page_structure}.jpg}

To inspect the default layout, simply create a new page using the
Website Builder.  Click on \sphinxmenuselection{Content \(\rightarrow\) New Page} and
add a page name.  Inspect the page using your browser.

\fvset{hllines={, ,}}%
\begin{sphinxVerbatim}[commandchars=\\\{\}]
\PYG{p}{\PYGZlt{}}\PYG{n+nt}{div} \PYG{n+na}{id}\PYG{o}{=}\PYG{l+s}{“wrapwrap”}\PYG{p}{\PYGZgt{}}
  \PYG{p}{\PYGZlt{}}\PYG{n+nt}{header} \PYG{p}{/}\PYG{p}{\PYGZgt{}}
  \PYG{p}{\PYGZlt{}}\PYG{n+nt}{main} \PYG{p}{/}\PYG{p}{\PYGZgt{}}
  \PYG{p}{\PYGZlt{}}\PYG{n+nt}{footer} \PYG{p}{/}\PYG{p}{\PYGZgt{}}
\PYG{p}{\PYGZlt{}}\PYG{p}{/}\PYG{n+nt}{div}\PYG{p}{\PYGZgt{}}
\end{sphinxVerbatim}


\subsubsection{Extend the default Header}
\label{\detokenize{howtos/themes:extend-the-default-header}}
By default, Odoo header contains a responsive navigation menu and the company’s logo. You can easily add new elements or style the existing one.

To do so, create a \sphinxstylestrong{layout.xml} file in your \sphinxstylestrong{views} folder and add the default Odoo xml markup.

\fvset{hllines={, ,}}%
\begin{sphinxVerbatim}[commandchars=\\\{\}]
\PYG{c+cp}{\PYGZlt{}?xml version=\PYGZdq{}1.0\PYGZdq{} encoding=\PYGZdq{}utf\PYGZhy{}8\PYGZdq{} ?\PYGZgt{}}
\PYG{n+nt}{\PYGZlt{}odoo}\PYG{n+nt}{\PYGZgt{}}



\PYG{n+nt}{\PYGZlt{}/odoo\PYGZgt{}}
\end{sphinxVerbatim}

Create a new template into the \sphinxcode{\sphinxupquote{\textless{}odoo\textgreater{}}} tag, copy-pasting the following
code.

\fvset{hllines={, ,}}%
\begin{sphinxVerbatim}[commandchars=\\\{\}]
\PYG{c}{\PYGZlt{}!\PYGZhy{}\PYGZhy{}}\PYG{c}{ Customize header  }\PYG{c}{\PYGZhy{}\PYGZhy{}\PYGZgt{}}
\PYG{n+nt}{\PYGZlt{}template} \PYG{n+na}{id=}\PYG{l+s}{\PYGZdq{}custom\PYGZus{}header\PYGZdq{}} \PYG{n+na}{inherit\PYGZus{}id=}\PYG{l+s}{\PYGZdq{}website.layout\PYGZdq{}} \PYG{n+na}{name=}\PYG{l+s}{\PYGZdq{}Custom Header\PYGZdq{}}\PYG{n+nt}{\PYGZgt{}}

  \PYG{c}{\PYGZlt{}!\PYGZhy{}\PYGZhy{}}\PYG{c}{ Assign an id  }\PYG{c}{\PYGZhy{}\PYGZhy{}\PYGZgt{}}
  \PYG{n+nt}{\PYGZlt{}xpath} \PYG{n+na}{expr=}\PYG{l+s}{\PYGZdq{}//div[@id=\PYGZsq{}wrapwrap\PYGZsq{}]/header\PYGZdq{}} \PYG{n+na}{position=}\PYG{l+s}{\PYGZdq{}attributes\PYGZdq{}}\PYG{n+nt}{\PYGZgt{}}
    \PYG{n+nt}{\PYGZlt{}attribute} \PYG{n+na}{name=}\PYG{l+s}{\PYGZdq{}id\PYGZdq{}}\PYG{n+nt}{\PYGZgt{}}my\PYGZus{}header\PYG{n+nt}{\PYGZlt{}/attribute\PYGZgt{}}
  \PYG{n+nt}{\PYGZlt{}/xpath\PYGZgt{}}

  \PYG{c}{\PYGZlt{}!\PYGZhy{}\PYGZhy{}}\PYG{c}{ Add an element after the top menu  }\PYG{c}{\PYGZhy{}\PYGZhy{}\PYGZgt{}}
  \PYG{n+nt}{\PYGZlt{}xpath} \PYG{n+na}{expr=}\PYG{l+s}{\PYGZdq{}//div[@id=\PYGZsq{}wrapwrap\PYGZsq{}]/header/div\PYGZdq{}} \PYG{n+na}{position=}\PYG{l+s}{\PYGZdq{}after\PYGZdq{}}\PYG{n+nt}{\PYGZgt{}}
    \PYG{n+nt}{\PYGZlt{}div} \PYG{n+na}{class=}\PYG{l+s}{\PYGZdq{}container\PYGZdq{}}\PYG{n+nt}{\PYGZgt{}}
      \PYG{n+nt}{\PYGZlt{}div} \PYG{n+na}{class=}\PYG{l+s}{\PYGZdq{}alert alert\PYGZhy{}info mt16\PYGZdq{}} \PYG{n+na}{role=}\PYG{l+s}{\PYGZdq{}alert\PYGZdq{}}\PYG{n+nt}{\PYGZgt{}}
        \PYG{n+nt}{\PYGZlt{}strong}\PYG{n+nt}{\PYGZgt{}}Welcome\PYG{n+nt}{\PYGZlt{}/strong\PYGZgt{}} in our website!
      \PYG{n+nt}{\PYGZlt{}/div\PYGZgt{}}
    \PYG{n+nt}{\PYGZlt{}/div\PYGZgt{}}
  \PYG{n+nt}{\PYGZlt{}/xpath\PYGZgt{}}
\PYG{n+nt}{\PYGZlt{}/template\PYGZgt{}}
\end{sphinxVerbatim}

The first xpath will add the id \sphinxcode{\sphinxupquote{my\_header}} to the header. It’s the best option if you want to
target css rules to that element and avoid these affecting other content on the page.

\begin{sphinxadmonition}{warning}{Warning:}
Be careful replacing default elements attributes. As your theme will extend the default one,
your changes will take priority in any future Odoo’s update.
\end{sphinxadmonition}

The second xpath will add a welcome message just after the navigation menu.

The last step is to add layout.xml to the list of xml files used by
the theme. To do that, edit your \sphinxcode{\sphinxupquote{\_\_manifest\_\_.py}} file like this

\fvset{hllines={, ,}}%
\begin{sphinxVerbatim}[commandchars=\\\{\}]
\PYG{l+s+s1}{\PYGZsq{}}\PYG{l+s+s1}{data}\PYG{l+s+s1}{\PYGZsq{}}\PYG{p}{:} \PYG{p}{[} \PYG{l+s+s1}{\PYGZsq{}}\PYG{l+s+s1}{views/layout.xml}\PYG{l+s+s1}{\PYGZsq{}} \PYG{p}{]}\PYG{p}{,}
\end{sphinxVerbatim}

Update your theme

\noindent\sphinxincludegraphics{{restart}.png}

Great! We successfully added an id to the
header and an element after the navigation menu. These changes will be
applied to each page of the website.

\noindent\sphinxincludegraphics{{after-menu}.png}


\subsection{Create a specific page layout}
\label{\detokenize{howtos/themes:create-a-specific-page-layout}}
Imagine that we want to create a specific layout for a Services page.
For this page, we need to add a list of services to the top and give the client the possibility of setting the rest of the page’s layout using snippets.

Inside your \sphinxstyleemphasis{views} folder, create a \sphinxstylestrong{pages.xml} file and add the
default Odoo markup.  Inside \sphinxcode{\sphinxupquote{\textless{}odoo\textgreater{}}}, instead of defining a \sphinxcode{\sphinxupquote{\textless{}template\textgreater{}}},
we will create a \sphinxstyleemphasis{page} object.

\fvset{hllines={, ,}}%
\begin{sphinxVerbatim}[commandchars=\\\{\}]
\PYG{c+cp}{\PYGZlt{}?xml version=\PYGZdq{}1.0\PYGZdq{} encoding=\PYGZdq{}utf\PYGZhy{}8\PYGZdq{} ?\PYGZgt{}}
\PYG{n+nt}{\PYGZlt{}odoo}\PYG{n+nt}{\PYGZgt{}}

     \PYG{c}{\PYGZlt{}!\PYGZhy{}\PYGZhy{}}\PYG{c}{ === Services Page === }\PYG{c}{\PYGZhy{}\PYGZhy{}\PYGZgt{}}
     \PYG{n+nt}{\PYGZlt{}record} \PYG{n+na}{id=}\PYG{l+s}{\PYGZdq{}services\PYGZus{}page\PYGZdq{}} \PYG{n+na}{model=}\PYG{l+s}{\PYGZdq{}website.page\PYGZdq{}}\PYG{n+nt}{\PYGZgt{}}
         \PYG{n+nt}{\PYGZlt{}field} \PYG{n+na}{name=}\PYG{l+s}{\PYGZdq{}name\PYGZdq{}}\PYG{n+nt}{\PYGZgt{}}Services page\PYG{n+nt}{\PYGZlt{}/field\PYGZgt{}}
         \PYG{n+nt}{\PYGZlt{}field} \PYG{n+na}{name=}\PYG{l+s}{\PYGZdq{}website\PYGZus{}published\PYGZdq{}}\PYG{n+nt}{\PYGZgt{}}True\PYG{n+nt}{\PYGZlt{}/field\PYGZgt{}}
         \PYG{n+nt}{\PYGZlt{}field} \PYG{n+na}{name=}\PYG{l+s}{\PYGZdq{}url\PYGZdq{}}\PYG{n+nt}{\PYGZgt{}}/services\PYG{n+nt}{\PYGZlt{}/field\PYGZgt{}}
         \PYG{n+nt}{\PYGZlt{}field} \PYG{n+na}{name=}\PYG{l+s}{\PYGZdq{}type\PYGZdq{}}\PYG{n+nt}{\PYGZgt{}}qweb\PYG{n+nt}{\PYGZlt{}/field\PYGZgt{}}
         \PYG{n+nt}{\PYGZlt{}field} \PYG{n+na}{name=}\PYG{l+s}{\PYGZdq{}key\PYGZdq{}}\PYG{n+nt}{\PYGZgt{}}theme\PYGZus{}tutorial.services\PYGZus{}page\PYG{n+nt}{\PYGZlt{}/field\PYGZgt{}}
         \PYG{n+nt}{\PYGZlt{}field} \PYG{n+na}{name=}\PYG{l+s}{\PYGZdq{}arch\PYGZdq{}} \PYG{n+na}{type=}\PYG{l+s}{\PYGZdq{}xml\PYGZdq{}}\PYG{n+nt}{\PYGZgt{}}
             \PYG{n+nt}{\PYGZlt{}t} \PYG{n+na}{t\PYGZhy{}name=}\PYG{l+s}{\PYGZdq{}theme\PYGZus{}tutorial.services\PYGZus{}page\PYGZus{}template\PYGZdq{}}\PYG{n+nt}{\PYGZgt{}}
                 \PYG{n+nt}{\PYGZlt{}h1}\PYG{n+nt}{\PYGZgt{}}Our Services\PYG{n+nt}{\PYGZlt{}/h1\PYGZgt{}}
                 \PYG{n+nt}{\PYGZlt{}ul} \PYG{n+na}{class=}\PYG{l+s}{\PYGZdq{}services\PYGZdq{}}\PYG{n+nt}{\PYGZgt{}}
                     \PYG{n+nt}{\PYGZlt{}li}\PYG{n+nt}{\PYGZgt{}}Cloud Hosting\PYG{n+nt}{\PYGZlt{}/li\PYGZgt{}}
                     \PYG{n+nt}{\PYGZlt{}li}\PYG{n+nt}{\PYGZgt{}}Support\PYG{n+nt}{\PYGZlt{}/li\PYGZgt{}}
                     \PYG{n+nt}{\PYGZlt{}li}\PYG{n+nt}{\PYGZgt{}}Unlimited space\PYG{n+nt}{\PYGZlt{}/li\PYGZgt{}}
                 \PYG{n+nt}{\PYGZlt{}/ul\PYGZgt{}}
             \PYG{n+nt}{\PYGZlt{}/t\PYGZgt{}}
         \PYG{n+nt}{\PYGZlt{}/field\PYGZgt{}}
     \PYG{n+nt}{\PYGZlt{}/record\PYGZgt{}}

 \PYG{n+nt}{\PYGZlt{}/odoo\PYGZgt{}}
\end{sphinxVerbatim}

As you can see, pages come with many additional properties like the \sphinxstyleemphasis{name} or
the \sphinxstyleemphasis{url} where it is reachable.

We successfully created a new page layout, but we haven’t told the
system \sphinxstylestrong{how to use it}. To do that, we can use \sphinxstylestrong{QWeb}. Wrap the
html code into a \sphinxcode{\sphinxupquote{\textless{}t\textgreater{}}} tag, like in this example.

\fvset{hllines={, ,}}%
\begin{sphinxVerbatim}[commandchars=\\\{\}]
\PYG{c}{\PYGZlt{}!\PYGZhy{}\PYGZhy{}}\PYG{c}{ === Services Page === }\PYG{c}{\PYGZhy{}\PYGZhy{}\PYGZgt{}}
\PYG{n+nt}{\PYGZlt{}record} \PYG{n+na}{id=}\PYG{l+s}{\PYGZdq{}services\PYGZus{}page\PYGZdq{}} \PYG{n+na}{model=}\PYG{l+s}{\PYGZdq{}website.page\PYGZdq{}}\PYG{n+nt}{\PYGZgt{}}
    \PYG{n+nt}{\PYGZlt{}field} \PYG{n+na}{name=}\PYG{l+s}{\PYGZdq{}name\PYGZdq{}}\PYG{n+nt}{\PYGZgt{}}Services page\PYG{n+nt}{\PYGZlt{}/field\PYGZgt{}}
    \PYG{n+nt}{\PYGZlt{}field} \PYG{n+na}{name=}\PYG{l+s}{\PYGZdq{}website\PYGZus{}published\PYGZdq{}}\PYG{n+nt}{\PYGZgt{}}True\PYG{n+nt}{\PYGZlt{}/field\PYGZgt{}}
    \PYG{n+nt}{\PYGZlt{}field} \PYG{n+na}{name=}\PYG{l+s}{\PYGZdq{}url\PYGZdq{}}\PYG{n+nt}{\PYGZgt{}}/services\PYG{n+nt}{\PYGZlt{}/field\PYGZgt{}}
    \PYG{n+nt}{\PYGZlt{}field} \PYG{n+na}{name=}\PYG{l+s}{\PYGZdq{}type\PYGZdq{}}\PYG{n+nt}{\PYGZgt{}}qweb\PYG{n+nt}{\PYGZlt{}/field\PYGZgt{}}
    \PYG{n+nt}{\PYGZlt{}field} \PYG{n+na}{name=}\PYG{l+s}{\PYGZdq{}key\PYGZdq{}}\PYG{n+nt}{\PYGZgt{}}theme\PYGZus{}tutorial.services\PYGZus{}page\PYG{n+nt}{\PYGZlt{}/field\PYGZgt{}}
    \PYG{n+nt}{\PYGZlt{}field} \PYG{n+na}{name=}\PYG{l+s}{\PYGZdq{}arch\PYGZdq{}} \PYG{n+na}{type=}\PYG{l+s}{\PYGZdq{}xml\PYGZdq{}}\PYG{n+nt}{\PYGZgt{}}
        \PYG{n+nt}{\PYGZlt{}t} \PYG{n+na}{t\PYGZhy{}name=}\PYG{l+s}{\PYGZdq{}theme\PYGZus{}tutorial.services\PYGZus{}page\PYGZus{}template\PYGZdq{}}\PYG{n+nt}{\PYGZgt{}}
            \PYG{n+nt}{\PYGZlt{}t} \PYG{n+na}{t\PYGZhy{}call=}\PYG{l+s}{\PYGZdq{}website.layout\PYGZdq{}}\PYG{n+nt}{\PYGZgt{}}
                \PYG{n+nt}{\PYGZlt{}div} \PYG{n+na}{id=}\PYG{l+s}{\PYGZdq{}wrap\PYGZdq{}}\PYG{n+nt}{\PYGZgt{}}
                    \PYG{n+nt}{\PYGZlt{}div} \PYG{n+na}{class=}\PYG{l+s}{\PYGZdq{}container\PYGZdq{}}\PYG{n+nt}{\PYGZgt{}}
                        \PYG{n+nt}{\PYGZlt{}h1}\PYG{n+nt}{\PYGZgt{}}Our Services\PYG{n+nt}{\PYGZlt{}/h1\PYGZgt{}}
                        \PYG{n+nt}{\PYGZlt{}ul} \PYG{n+na}{class=}\PYG{l+s}{\PYGZdq{}services\PYGZdq{}}\PYG{n+nt}{\PYGZgt{}}
                            \PYG{n+nt}{\PYGZlt{}li}\PYG{n+nt}{\PYGZgt{}}Cloud Hosting\PYG{n+nt}{\PYGZlt{}/li\PYGZgt{}}
                            \PYG{n+nt}{\PYGZlt{}li}\PYG{n+nt}{\PYGZgt{}}Support\PYG{n+nt}{\PYGZlt{}/li\PYGZgt{}}
                            \PYG{n+nt}{\PYGZlt{}li}\PYG{n+nt}{\PYGZgt{}}Unlimited space\PYG{n+nt}{\PYGZlt{}/li\PYGZgt{}}
                        \PYG{n+nt}{\PYGZlt{}/ul\PYGZgt{}}
                    \PYG{n+nt}{\PYGZlt{}/div\PYGZgt{}}
                \PYG{n+nt}{\PYGZlt{}/div\PYGZgt{}}
            \PYG{n+nt}{\PYGZlt{}/t\PYGZgt{}}
        \PYG{n+nt}{\PYGZlt{}/t\PYGZgt{}}
    \PYG{n+nt}{\PYGZlt{}/field\PYGZgt{}}
\PYG{n+nt}{\PYGZlt{}/record\PYGZgt{}}
\end{sphinxVerbatim}

Using \sphinxcode{\sphinxupquote{\textless{}t t-call="website.layout"\textgreater{}}} we will extend the Odoo
default page layout with our code.

As you can see, we wrapped our code into two \sphinxcode{\sphinxupquote{\textless{}div\textgreater{}}},  one with ID \sphinxcode{\sphinxupquote{wrap}} and the other one with class \sphinxcode{\sphinxupquote{container}}. This is to provide a minimal layout.

The next step is to add an empty area that the user
can fill with snippets. To achieve this, just create a \sphinxcode{\sphinxupquote{div}} with
\sphinxcode{\sphinxupquote{oe\_structure}} class just before closing the \sphinxcode{\sphinxupquote{div\#wrap}} element.

\fvset{hllines={, ,}}%
\begin{sphinxVerbatim}[commandchars=\\\{\}]
\PYG{c+cp}{\PYGZlt{}?xml version=\PYGZdq{}1.0\PYGZdq{} encoding=\PYGZdq{}utf\PYGZhy{}8\PYGZdq{} ?\PYGZgt{}}
\PYG{n+nt}{\PYGZlt{}odoo}\PYG{n+nt}{\PYGZgt{}}

    \PYG{c}{\PYGZlt{}!\PYGZhy{}\PYGZhy{}}\PYG{c}{ === Services Page === }\PYG{c}{\PYGZhy{}\PYGZhy{}\PYGZgt{}}
    \PYG{n+nt}{\PYGZlt{}record} \PYG{n+na}{id=}\PYG{l+s}{\PYGZdq{}services\PYGZus{}page\PYGZdq{}} \PYG{n+na}{model=}\PYG{l+s}{\PYGZdq{}website.page\PYGZdq{}}\PYG{n+nt}{\PYGZgt{}}
        \PYG{n+nt}{\PYGZlt{}field} \PYG{n+na}{name=}\PYG{l+s}{\PYGZdq{}name\PYGZdq{}}\PYG{n+nt}{\PYGZgt{}}Services page\PYG{n+nt}{\PYGZlt{}/field\PYGZgt{}}
        \PYG{n+nt}{\PYGZlt{}field} \PYG{n+na}{name=}\PYG{l+s}{\PYGZdq{}website\PYGZus{}published\PYGZdq{}}\PYG{n+nt}{\PYGZgt{}}True\PYG{n+nt}{\PYGZlt{}/field\PYGZgt{}}
        \PYG{n+nt}{\PYGZlt{}field} \PYG{n+na}{name=}\PYG{l+s}{\PYGZdq{}url\PYGZdq{}}\PYG{n+nt}{\PYGZgt{}}/services\PYG{n+nt}{\PYGZlt{}/field\PYGZgt{}}
        \PYG{n+nt}{\PYGZlt{}field} \PYG{n+na}{name=}\PYG{l+s}{\PYGZdq{}type\PYGZdq{}}\PYG{n+nt}{\PYGZgt{}}qweb\PYG{n+nt}{\PYGZlt{}/field\PYGZgt{}}
        \PYG{n+nt}{\PYGZlt{}field} \PYG{n+na}{name=}\PYG{l+s}{\PYGZdq{}key\PYGZdq{}}\PYG{n+nt}{\PYGZgt{}}theme\PYGZus{}tutorial.services\PYGZus{}page\PYG{n+nt}{\PYGZlt{}/field\PYGZgt{}}
        \PYG{n+nt}{\PYGZlt{}field} \PYG{n+na}{name=}\PYG{l+s}{\PYGZdq{}arch\PYGZdq{}} \PYG{n+na}{type=}\PYG{l+s}{\PYGZdq{}xml\PYGZdq{}}\PYG{n+nt}{\PYGZgt{}}
            \PYG{n+nt}{\PYGZlt{}t} \PYG{n+na}{t\PYGZhy{}name=}\PYG{l+s}{\PYGZdq{}theme\PYGZus{}tutorial.services\PYGZus{}page\PYGZus{}template\PYGZdq{}}\PYG{n+nt}{\PYGZgt{}}
                \PYG{n+nt}{\PYGZlt{}t} \PYG{n+na}{t\PYGZhy{}call=}\PYG{l+s}{\PYGZdq{}website.layout\PYGZdq{}}\PYG{n+nt}{\PYGZgt{}}
                    \PYG{n+nt}{\PYGZlt{}div} \PYG{n+na}{id=}\PYG{l+s}{\PYGZdq{}wrap\PYGZdq{}}\PYG{n+nt}{\PYGZgt{}}
                        \PYG{n+nt}{\PYGZlt{}div} \PYG{n+na}{class=}\PYG{l+s}{\PYGZdq{}container\PYGZdq{}}\PYG{n+nt}{\PYGZgt{}}
                            \PYG{n+nt}{\PYGZlt{}h1}\PYG{n+nt}{\PYGZgt{}}Our Services\PYG{n+nt}{\PYGZlt{}/h1\PYGZgt{}}
                            \PYG{n+nt}{\PYGZlt{}ul} \PYG{n+na}{class=}\PYG{l+s}{\PYGZdq{}services\PYGZdq{}}\PYG{n+nt}{\PYGZgt{}}
                                \PYG{n+nt}{\PYGZlt{}li}\PYG{n+nt}{\PYGZgt{}}Cloud Hosting\PYG{n+nt}{\PYGZlt{}/li\PYGZgt{}}
                                \PYG{n+nt}{\PYGZlt{}li}\PYG{n+nt}{\PYGZgt{}}Support\PYG{n+nt}{\PYGZlt{}/li\PYGZgt{}}
                                \PYG{n+nt}{\PYGZlt{}li}\PYG{n+nt}{\PYGZgt{}}Unlimited space\PYG{n+nt}{\PYGZlt{}/li\PYGZgt{}}
                            \PYG{n+nt}{\PYGZlt{}/ul\PYGZgt{}}

                            \PYG{c}{\PYGZlt{}!\PYGZhy{}\PYGZhy{}}\PYG{c}{ === Snippets\PYGZsq{} area === }\PYG{c}{\PYGZhy{}\PYGZhy{}\PYGZgt{}}
                            \PYG{n+nt}{\PYGZlt{}div} \PYG{n+na}{class=}\PYG{l+s}{\PYGZdq{}oe\PYGZus{}structure\PYGZdq{}} \PYG{n+nt}{/\PYGZgt{}}
                        \PYG{n+nt}{\PYGZlt{}/div\PYGZgt{}}
                    \PYG{n+nt}{\PYGZlt{}/div\PYGZgt{}}
                \PYG{n+nt}{\PYGZlt{}/t\PYGZgt{}}
            \PYG{n+nt}{\PYGZlt{}/t\PYGZgt{}}
        \PYG{n+nt}{\PYGZlt{}/field\PYGZgt{}}
    \PYG{n+nt}{\PYGZlt{}/record\PYGZgt{}}

\PYG{n+nt}{\PYGZlt{}/odoo\PYGZgt{}}
\end{sphinxVerbatim}

\begin{sphinxadmonition}{tip}{Tip:}
You can create as many snippet areas as you like and place them anywhere in your pages.
\end{sphinxadmonition}

It is worth mentioning there is an alternative to create pages using the
\sphinxcode{\sphinxupquote{\textless{}template\textgreater{}}} directive we saw before.

\fvset{hllines={, ,}}%
\begin{sphinxVerbatim}[commandchars=\\\{\}]
\PYG{c+cp}{\PYGZlt{}?xml version=\PYGZdq{}1.0\PYGZdq{} encoding=\PYGZdq{}utf\PYGZhy{}8\PYGZdq{} ?\PYGZgt{}}
\PYG{n+nt}{\PYGZlt{}odoo}\PYG{n+nt}{\PYGZgt{}}

    \PYG{c}{\PYGZlt{}!\PYGZhy{}\PYGZhy{}}\PYG{c}{ === Services Page === }\PYG{c}{\PYGZhy{}\PYGZhy{}\PYGZgt{}}
    \PYG{n+nt}{\PYGZlt{}template} \PYG{n+na}{id=}\PYG{l+s}{\PYGZdq{}services\PYGZus{}page\PYGZus{}template\PYGZdq{}}\PYG{n+nt}{\PYGZgt{}}
        \PYG{n+nt}{\PYGZlt{}t} \PYG{n+na}{t\PYGZhy{}call=}\PYG{l+s}{\PYGZdq{}website.layout\PYGZdq{}}\PYG{n+nt}{\PYGZgt{}}
            \PYG{n+nt}{\PYGZlt{}div} \PYG{n+na}{id=}\PYG{l+s}{\PYGZdq{}wrap\PYGZdq{}}\PYG{n+nt}{\PYGZgt{}}
                \PYG{n+nt}{\PYGZlt{}div} \PYG{n+na}{class=}\PYG{l+s}{\PYGZdq{}container\PYGZdq{}}\PYG{n+nt}{\PYGZgt{}}
                    \PYG{n+nt}{\PYGZlt{}h1}\PYG{n+nt}{\PYGZgt{}}Our Services\PYG{n+nt}{\PYGZlt{}/h1\PYGZgt{}}
                    \PYG{n+nt}{\PYGZlt{}ul} \PYG{n+na}{class=}\PYG{l+s}{\PYGZdq{}services\PYGZdq{}}\PYG{n+nt}{\PYGZgt{}}
                        \PYG{n+nt}{\PYGZlt{}li}\PYG{n+nt}{\PYGZgt{}}Cloud Hosting\PYG{n+nt}{\PYGZlt{}/li\PYGZgt{}}
                        \PYG{n+nt}{\PYGZlt{}li}\PYG{n+nt}{\PYGZgt{}}Support\PYG{n+nt}{\PYGZlt{}/li\PYGZgt{}}
                        \PYG{n+nt}{\PYGZlt{}li}\PYG{n+nt}{\PYGZgt{}}Unlimited space\PYG{n+nt}{\PYGZlt{}/li\PYGZgt{}}
                    \PYG{n+nt}{\PYGZlt{}/ul\PYGZgt{}}

                    \PYG{c}{\PYGZlt{}!\PYGZhy{}\PYGZhy{}}\PYG{c}{ === Snippets\PYGZsq{} area === }\PYG{c}{\PYGZhy{}\PYGZhy{}\PYGZgt{}}
                    \PYG{n+nt}{\PYGZlt{}div} \PYG{n+na}{class=}\PYG{l+s}{\PYGZdq{}oe\PYGZus{}structure\PYGZdq{}} \PYG{n+nt}{/\PYGZgt{}}
                \PYG{n+nt}{\PYGZlt{}/div\PYGZgt{}}
            \PYG{n+nt}{\PYGZlt{}/div\PYGZgt{}}
        \PYG{n+nt}{\PYGZlt{}/t\PYGZgt{}}
    \PYG{n+nt}{\PYGZlt{}/template\PYGZgt{}}
    \PYG{n+nt}{\PYGZlt{}record} \PYG{n+na}{id=}\PYG{l+s}{\PYGZdq{}services\PYGZus{}page\PYGZdq{}} \PYG{n+na}{model=}\PYG{l+s}{\PYGZdq{}website.page\PYGZdq{}}\PYG{n+nt}{\PYGZgt{}}
        \PYG{n+nt}{\PYGZlt{}field} \PYG{n+na}{name=}\PYG{l+s}{\PYGZdq{}name\PYGZdq{}}\PYG{n+nt}{\PYGZgt{}}Services page\PYG{n+nt}{\PYGZlt{}/field\PYGZgt{}}
        \PYG{n+nt}{\PYGZlt{}field} \PYG{n+na}{name=}\PYG{l+s}{\PYGZdq{}website\PYGZus{}published\PYGZdq{}}\PYG{n+nt}{\PYGZgt{}}True\PYG{n+nt}{\PYGZlt{}/field\PYGZgt{}}
        \PYG{n+nt}{\PYGZlt{}field} \PYG{n+na}{name=}\PYG{l+s}{\PYGZdq{}url\PYGZdq{}}\PYG{n+nt}{\PYGZgt{}}/services\PYG{n+nt}{\PYGZlt{}/field\PYGZgt{}}
        \PYG{n+nt}{\PYGZlt{}field} \PYG{n+na}{name=}\PYG{l+s}{\PYGZdq{}view\PYGZus{}id\PYGZdq{}} \PYG{n+na}{ref=}\PYG{l+s}{\PYGZdq{}services\PYGZus{}page\PYGZus{}template\PYGZdq{}}\PYG{n+nt}{/\PYGZgt{}}
    \PYG{n+nt}{\PYGZlt{}/record\PYGZgt{}}

\PYG{n+nt}{\PYGZlt{}/odoo\PYGZgt{}}
\end{sphinxVerbatim}

This would allow your page content to be further customized using \sphinxcode{\sphinxupquote{\textless{}xpath\textgreater{}}}.

Our page is almost ready. Now all we have to do is add \sphinxstylestrong{pages.xml} in our \sphinxstylestrong{\_\_manifest\_\_.py} file

\fvset{hllines={, ,}}%
\begin{sphinxVerbatim}[commandchars=\\\{\}]
\PYG{l+s+s1}{\PYGZsq{}}\PYG{l+s+s1}{data}\PYG{l+s+s1}{\PYGZsq{}}\PYG{p}{:} \PYG{p}{[}
  \PYG{l+s+s1}{\PYGZsq{}}\PYG{l+s+s1}{views/layout.xml}\PYG{l+s+s1}{\PYGZsq{}}\PYG{p}{,}
  \PYG{l+s+s1}{\PYGZsq{}}\PYG{l+s+s1}{views/pages.xml}\PYG{l+s+s1}{\PYGZsq{}}
\PYG{p}{]}\PYG{p}{,}
\end{sphinxVerbatim}

Update your theme

\noindent\sphinxincludegraphics{{restart}.png}

Great, our Services page is ready and you’ll be able to access it by navigating to \sphinxcode{\sphinxupquote{\textless{}yourwebsite\textgreater{}/services}} (the URL we chose above).

You will notice that it’s possible to drag/drop snippets underneath the
\sphinxstyleemphasis{Our Services} list.

\noindent\sphinxincludegraphics{{services_page_nostyle}.png}

Now let’s go back to our \sphinxstyleemphasis{pages.xml} and, after our page template,
copy/paste the following code.

\fvset{hllines={, ,}}%
\begin{sphinxVerbatim}[commandchars=\\\{\}]
\PYG{n+nt}{\PYGZlt{}record} \PYG{n+na}{id=}\PYG{l+s}{\PYGZdq{}services\PYGZus{}page\PYGZus{}link\PYGZdq{}} \PYG{n+na}{model=}\PYG{l+s}{\PYGZdq{}website.menu\PYGZdq{}}\PYG{n+nt}{\PYGZgt{}}
  \PYG{n+nt}{\PYGZlt{}field} \PYG{n+na}{name=}\PYG{l+s}{\PYGZdq{}name\PYGZdq{}}\PYG{n+nt}{\PYGZgt{}}Services\PYG{n+nt}{\PYGZlt{}/field\PYGZgt{}}
  \PYG{n+nt}{\PYGZlt{}field} \PYG{n+na}{name=}\PYG{l+s}{\PYGZdq{}page\PYGZus{}id\PYGZdq{}} \PYG{n+na}{ref=}\PYG{l+s}{\PYGZdq{}services\PYGZus{}page\PYGZdq{}}\PYG{n+nt}{/\PYGZgt{}}
  \PYG{n+nt}{\PYGZlt{}field} \PYG{n+na}{name=}\PYG{l+s}{\PYGZdq{}parent\PYGZus{}id\PYGZdq{}} \PYG{n+na}{ref=}\PYG{l+s}{\PYGZdq{}website.main\PYGZus{}menu\PYGZdq{}} \PYG{n+nt}{/\PYGZgt{}}
  \PYG{n+nt}{\PYGZlt{}field} \PYG{n+na}{name=}\PYG{l+s}{\PYGZdq{}sequence\PYGZdq{}} \PYG{n+na}{type=}\PYG{l+s}{\PYGZdq{}int\PYGZdq{}}\PYG{n+nt}{\PYGZgt{}}99\PYG{n+nt}{\PYGZlt{}/field\PYGZgt{}}
\PYG{n+nt}{\PYGZlt{}/record\PYGZgt{}}
\end{sphinxVerbatim}

This code will add a link to the main menu, referring to the page we created.

\noindent\sphinxincludegraphics{{services_page_menu}.png}

The \sphinxstylestrong{sequence} attribute defines the link’s position in the top menu.
In our example, we set the value to \sphinxcode{\sphinxupquote{99}} in order to place it last. I you want to place it in a particular position, you have to replace the value according to your needs.

As you can see inspecting the \sphinxstyleemphasis{data.xml} file in the \sphinxcode{\sphinxupquote{website}} module, the \sphinxstylestrong{Home} link is set to \sphinxcode{\sphinxupquote{10}} and the \sphinxstylestrong{Contact} us one is set to \sphinxcode{\sphinxupquote{60}} by default.
If, for example, you want to place your link in the \sphinxstylestrong{middle}, you can set your link’s sequence value to \sphinxcode{\sphinxupquote{40}}.


\subsection{Add Styles}
\label{\detokenize{howtos/themes:add-styles}}
Odoo includes Bootstrap by default. This means that you can take advantage of all Bootstrap styles and layout functionalities out of the box.

Of course Bootstrap is not enough if you want to provide a unique design. The following steps will guide you through how to add custom styles to your theme.
The final result won’t be pretty, but will provide you with enough information to build upon on your own.

Let’s start by creating an empty file called \sphinxstylestrong{style.less} and place it in a folder called \sphinxstylestrong{less} in your static folder.
The following rules will style our \sphinxstyleemphasis{Services} page. Copy and paste it, then save the file.

\fvset{hllines={, ,}}%
\begin{sphinxVerbatim}[commandchars=\\\{\}]
\PYG{n+nc}{.}\PYG{n+nc}{services} \PYG{p}{\PYGZob{}}
    \PYG{n+nt}{background}\PYG{n+nd}{:} \PYG{n+nn}{\PYGZsh{}}\PYG{n+nn}{EAEAEA}\PYG{o}{;}
    \PYG{n+nt}{padding}\PYG{n+nd}{:} \PYG{n+nt}{1em}\PYG{o}{;}
    \PYG{n+nt}{margin}\PYG{n+nd}{:} \PYG{n+nt}{2em} \PYG{n+nt}{0} \PYG{n+nt}{3em}\PYG{o}{;}
    \PYG{n+nt}{li} \PYG{p}{\PYGZob{}}
        \PYG{n+nt}{display}\PYG{n+nd}{:} \PYG{n+nt}{block}\PYG{o}{;}
        \PYG{n+nt}{position}\PYG{n+nd}{:} \PYG{n+nt}{relative}\PYG{o}{;}
        \PYG{n+nt}{background\PYGZhy{}color}\PYG{n+nd}{:} \PYG{n+nn}{\PYGZsh{}}\PYG{n+nn}{16a085}\PYG{o}{;}
        \PYG{n+nt}{color}\PYG{n+nd}{:} \PYG{n+nn}{\PYGZsh{}}\PYG{n+nn}{FFF}\PYG{o}{;}
        \PYG{n+nt}{padding}\PYG{n+nd}{:} \PYG{n+nt}{2em}\PYG{o}{;}
        \PYG{n+nt}{text\PYGZhy{}align}\PYG{n+nd}{:} \PYG{n+nt}{center}\PYG{o}{;}
        \PYG{n+nt}{margin\PYGZhy{}bottom}\PYG{n+nd}{:} \PYG{n+nt}{1em}\PYG{o}{;}
        \PYG{n+nt}{font\PYGZhy{}size}\PYG{n+nd}{:} \PYG{n+nt}{1}\PYG{n+nc}{.}\PYG{n+nc}{5em}\PYG{o}{;}
    \PYG{p}{\PYGZcb{}}
\PYG{p}{\PYGZcb{}}
\end{sphinxVerbatim}

Our file is ready but it is not included in our theme yet.

Let’s navigate to the view folder and create an XML file called \sphinxstyleemphasis{assets.xml}. Add the default Odoo xml markup and copy/paste the following code. Remember to replace \sphinxcode{\sphinxupquote{theme folder}} with your theme’s main folder name.

\fvset{hllines={, ,}}%
\begin{sphinxVerbatim}[commandchars=\\\{\}]
\PYG{n+nt}{\PYGZlt{}template} \PYG{n+na}{id=}\PYG{l+s}{\PYGZdq{}mystyle\PYGZdq{}} \PYG{n+na}{name=}\PYG{l+s}{\PYGZdq{}My style\PYGZdq{}} \PYG{n+na}{inherit\PYGZus{}id=}\PYG{l+s}{\PYGZdq{}website.assets\PYGZus{}frontend\PYGZdq{}}\PYG{n+nt}{\PYGZgt{}}
    \PYG{n+nt}{\PYGZlt{}xpath} \PYG{n+na}{expr=}\PYG{l+s}{\PYGZdq{}link[last()]\PYGZdq{}} \PYG{n+na}{position=}\PYG{l+s}{\PYGZdq{}after\PYGZdq{}}\PYG{n+nt}{\PYGZgt{}}
        \PYG{n+nt}{\PYGZlt{}link} \PYG{n+na}{href=}\PYG{l+s}{\PYGZdq{}/theme folder/static/less/style.less\PYGZdq{}} \PYG{n+na}{rel=}\PYG{l+s}{\PYGZdq{}stylesheet\PYGZdq{}} \PYG{n+na}{type=}\PYG{l+s}{\PYGZdq{}text/less\PYGZdq{}}\PYG{n+nt}{/\PYGZgt{}}
    \PYG{n+nt}{\PYGZlt{}/xpath\PYGZgt{}}
\PYG{n+nt}{\PYGZlt{}/template\PYGZgt{}}
\end{sphinxVerbatim}

We just created a template specifying our less file. As you can see,
our template has a special attribute called \sphinxcode{\sphinxupquote{inherit\_id}}.  This
attribute tells Odoo that our template is referring to another one in
order to operate.

In this case, we are referring to \sphinxcode{\sphinxupquote{assets\_frontend}} template,
located in the \sphinxcode{\sphinxupquote{website}} module. \sphinxcode{\sphinxupquote{assets\_frontend}} specifies the
list of assets loaded by the website builder and our goal is to add
our less file to this list.

This can be achieved using xpath with the attributes
\sphinxcode{\sphinxupquote{expr="link{[}last(){]}"}} and \sphinxcode{\sphinxupquote{position="after"}}, which means “\sphinxstyleemphasis{take my
style file and place it after the last link in the list of the
assets}”.

Placing it after the last one, we ensure that our file will
be loaded at the end and take priority.

Finally add \sphinxstylestrong{assets.xml} in your \sphinxstylestrong{\_\_manifest\_\_.py} file.

Update your theme

\noindent\sphinxincludegraphics{{restart}.png}

Our less file is now included in our theme, it will be automatically compiled, minified and combined with all Odoo’s assets.

\noindent\sphinxincludegraphics{{services_page_styled}.png}


\subsection{Create Snippets}
\label{\detokenize{howtos/themes:create-snippets}}
Since snippets are how users design and layout pages, they are the most important element of your design.
Let’s create a snippet for our Service page. The snippet will display three testimonials and it will be editable by the end user using the Website Builder UI.
Navigate to the view folder and create an XML file called \sphinxstylestrong{snippets.xml}.
Add the default Odoo xml markup and copy/paste the following code.
The template contains the HTML markup that will be displayed by the snippet.

\fvset{hllines={, ,}}%
\begin{sphinxVerbatim}[commandchars=\\\{\}]
\PYG{n+nt}{\PYGZlt{}template} \PYG{n+na}{id=}\PYG{l+s}{\PYGZdq{}snippet\PYGZus{}testimonial\PYGZdq{}} \PYG{n+na}{name=}\PYG{l+s}{\PYGZdq{}Testimonial snippet\PYGZdq{}}\PYG{n+nt}{\PYGZgt{}}
  \PYG{n+nt}{\PYGZlt{}section} \PYG{n+na}{class=}\PYG{l+s}{\PYGZdq{}snippet\PYGZus{}testimonial\PYGZdq{}}\PYG{n+nt}{\PYGZgt{}}
    \PYG{n+nt}{\PYGZlt{}div} \PYG{n+na}{class=}\PYG{l+s}{\PYGZdq{}container text\PYGZhy{}center\PYGZdq{}}\PYG{n+nt}{\PYGZgt{}}
      \PYG{n+nt}{\PYGZlt{}div} \PYG{n+na}{class=}\PYG{l+s}{\PYGZdq{}row\PYGZdq{}}\PYG{n+nt}{\PYGZgt{}}
        \PYG{n+nt}{\PYGZlt{}div} \PYG{n+na}{class=}\PYG{l+s}{\PYGZdq{}col\PYGZhy{}md\PYGZhy{}4\PYGZdq{}}\PYG{n+nt}{\PYGZgt{}}
          \PYG{n+nt}{\PYGZlt{}img} \PYG{n+na}{alt=}\PYG{l+s}{\PYGZdq{}client\PYGZdq{}} \PYG{n+na}{class=}\PYG{l+s}{\PYGZdq{}img\PYGZhy{}circle\PYGZdq{}} \PYG{n+na}{src=}\PYG{l+s}{\PYGZdq{}/theme\PYGZus{}tutorial/static/src/img/client\PYGZus{}1.jpg\PYGZdq{}}\PYG{n+nt}{/\PYGZgt{}}
          \PYG{n+nt}{\PYGZlt{}h3}\PYG{n+nt}{\PYGZgt{}}Client Name\PYG{n+nt}{\PYGZlt{}/h3\PYGZgt{}}
          \PYG{n+nt}{\PYGZlt{}p}\PYG{n+nt}{\PYGZgt{}}Lorem ipsum dolor sit amet, consectetur adipiscing elit.\PYG{n+nt}{\PYGZlt{}/p\PYGZgt{}}
        \PYG{n+nt}{\PYGZlt{}/div\PYGZgt{}}
        \PYG{n+nt}{\PYGZlt{}div} \PYG{n+na}{class=}\PYG{l+s}{\PYGZdq{}col\PYGZhy{}md\PYGZhy{}4\PYGZdq{}}\PYG{n+nt}{\PYGZgt{}}
          \PYG{n+nt}{\PYGZlt{}img} \PYG{n+na}{alt=}\PYG{l+s}{\PYGZdq{}client\PYGZdq{}} \PYG{n+na}{class=}\PYG{l+s}{\PYGZdq{}img\PYGZhy{}circle\PYGZdq{}} \PYG{n+na}{src=}\PYG{l+s}{\PYGZdq{}/theme\PYGZus{}tutorial/static/src/img/client\PYGZus{}2.jpg\PYGZdq{}}\PYG{n+nt}{/\PYGZgt{}}
          \PYG{n+nt}{\PYGZlt{}h3}\PYG{n+nt}{\PYGZgt{}}Client Name\PYG{n+nt}{\PYGZlt{}/h3\PYGZgt{}}
          \PYG{n+nt}{\PYGZlt{}p}\PYG{n+nt}{\PYGZgt{}}Lorem ipsum dolor sit amet, consectetur adipiscing elit.\PYG{n+nt}{\PYGZlt{}/p\PYGZgt{}}
        \PYG{n+nt}{\PYGZlt{}/div\PYGZgt{}}
        \PYG{n+nt}{\PYGZlt{}div} \PYG{n+na}{class=}\PYG{l+s}{\PYGZdq{}col\PYGZhy{}md\PYGZhy{}4\PYGZdq{}}\PYG{n+nt}{\PYGZgt{}}
          \PYG{n+nt}{\PYGZlt{}img} \PYG{n+na}{alt=}\PYG{l+s}{\PYGZdq{}client\PYGZdq{}} \PYG{n+na}{class=}\PYG{l+s}{\PYGZdq{}img\PYGZhy{}circle\PYGZdq{}} \PYG{n+na}{src=}\PYG{l+s}{\PYGZdq{}/theme\PYGZus{}tutorial/static/src/img/client\PYGZus{}3.jpg\PYGZdq{}}\PYG{n+nt}{/\PYGZgt{}}
          \PYG{n+nt}{\PYGZlt{}h3}\PYG{n+nt}{\PYGZgt{}}Client Name\PYG{n+nt}{\PYGZlt{}/h3\PYGZgt{}}
          \PYG{n+nt}{\PYGZlt{}p}\PYG{n+nt}{\PYGZgt{}}Lorem ipsum dolor sit amet, consectetur adipiscing elit.\PYG{n+nt}{\PYGZlt{}/p\PYGZgt{}}
        \PYG{n+nt}{\PYGZlt{}/div\PYGZgt{}}
      \PYG{n+nt}{\PYGZlt{}/div\PYGZgt{}}
    \PYG{n+nt}{\PYGZlt{}/div\PYGZgt{}}
  \PYG{n+nt}{\PYGZlt{}/section\PYGZgt{}}
\PYG{n+nt}{\PYGZlt{}/template\PYGZgt{}}
\end{sphinxVerbatim}

As you can see, we used Bootstrap default classes for our three columns. It’s not just about layout, these classes \sphinxstylestrong{will be triggered by the Website Builder to make them resizable by the user}.

The previous code will create the snippet’s content, but we still need to place it into the editor bar, so the user will be able to drag\&drop it into the page. Copy/paste this template in your \sphinxstylestrong{snippets.xml} file.

\fvset{hllines={, ,}}%
\begin{sphinxVerbatim}[commandchars=\\\{\}]
\PYG{n+nt}{\PYGZlt{}template} \PYG{n+na}{id=}\PYG{l+s}{\PYGZdq{}place\PYGZus{}into\PYGZus{}bar\PYGZdq{}} \PYG{n+na}{inherit\PYGZus{}id=}\PYG{l+s}{\PYGZdq{}website.snippets\PYGZdq{}} \PYG{n+na}{name=}\PYG{l+s}{\PYGZdq{}Place into bar\PYGZdq{}}\PYG{n+nt}{\PYGZgt{}}
  \PYG{n+nt}{\PYGZlt{}xpath} \PYG{n+na}{expr=}\PYG{l+s}{\PYGZdq{}//div[@id=\PYGZsq{}snippet\PYGZus{}content\PYGZsq{}]/div[@class=\PYGZsq{}o\PYGZus{}panel\PYGZus{}body\PYGZsq{}]\PYGZdq{}} \PYG{n+na}{position=}\PYG{l+s}{\PYGZdq{}inside\PYGZdq{}}\PYG{n+nt}{\PYGZgt{}}
    \PYG{n+nt}{\PYGZlt{}t} \PYG{n+na}{t\PYGZhy{}snippet=}\PYG{l+s}{\PYGZdq{}theme\PYGZus{}tutorial.snippet\PYGZus{}testimonial\PYGZdq{}}
       \PYG{n+na}{t\PYGZhy{}thumbnail=}\PYG{l+s}{\PYGZdq{}/theme\PYGZus{}tutorial/static/src/img/ui/snippet\PYGZus{}thumb.jpg\PYGZdq{}}\PYG{n+nt}{/\PYGZgt{}}
  \PYG{n+nt}{\PYGZlt{}/xpath\PYGZgt{}}
\PYG{n+nt}{\PYGZlt{}/template\PYGZgt{}}
\end{sphinxVerbatim}

Using xpath, we are targeting a particular element with id
\sphinxcode{\sphinxupquote{snippet\_structure}}. This means that the snippet will appear in the
Structure tab. If you want to change the destination tab, you have just to replace the \sphinxcode{\sphinxupquote{id}} value in the xpath expression.

\noindent\sphinxincludegraphics{{snippet_bar}.png}


\begin{savenotes}\sphinxattablestart
\centering
\begin{tabulary}{\linewidth}[t]{|T|T|}
\hline
\sphinxstyletheadfamily 
Tab Name
&\sphinxstyletheadfamily 
Xpath expression
\\
\hline
Structure
&
\sphinxcode{\sphinxupquote{//div{[}@id='snippet\_structure'{]}}}
\\
\hline
Content
&
\sphinxcode{\sphinxupquote{//div{[}@id='snippet\_content'{]}}}
\\
\hline
Feature
&
\sphinxcode{\sphinxupquote{//div{[}@id='snippet\_feature'{]}}}
\\
\hline
Effect
&
\sphinxcode{\sphinxupquote{//div{[}@id='snippet\_effect'{]}}}
\\
\hline
\end{tabulary}
\par
\sphinxattableend\end{savenotes}

The \sphinxcode{\sphinxupquote{\textless{}t\textgreater{}}} tag will call our snippet’s template and will assign a thumbnail placed in the img folder.
You can now drag your snippet from the snippet bar, drop it in your page and see the result.

\noindent\sphinxincludegraphics{{snippet_default}.png}


\subsection{Snippet options}
\label{\detokenize{howtos/themes:snippet-options}}
Options allow publishers to edit a snippet’s appearance using the Website Builder’s UI.
Using Website Builder functionalities, you can create snippet options easily and automatically add them to the UI.


\subsubsection{Options group properties}
\label{\detokenize{howtos/themes:options-group-properties}}
Options are wrapped in groups. Groups can have properties that define how the included options will interact with the user interface.
\begin{description}
\item[{\sphinxcode{\sphinxupquote{data-selector="{[}css selector(s){]}"}}}] \leavevmode
Bind all the options included into the group to a particular element.

\item[{\sphinxcode{\sphinxupquote{data-js=" custom method name "}}}] \leavevmode
Is used to bind custom Javascript methods.

\item[{\sphinxcode{\sphinxupquote{data-drop-in="{[}css selector(s){]}"}}}] \leavevmode
Defines the list of elements where the snippet can be dropped into.

\item[{\sphinxcode{\sphinxupquote{data-drop-near="{[}css selector(s){]}"}}}] \leavevmode
Defines the list of elements that the snippet can be dropped beside.

\end{description}


\subsubsection{Default option methods}
\label{\detokenize{howtos/themes:default-option-methods}}
Options apply standard CSS classes to the snippet. Depending on the method that you choose, the UI will behave differently.
\begin{description}
\item[{\sphinxcode{\sphinxupquote{data-select-class="{[}class name{]}"}}}] \leavevmode
More data-select-class in the same group defines a list of classes that the user can choose to apply. Only one option can be enabled at a time.

\item[{\sphinxcode{\sphinxupquote{data-toggle-class="{[}class name{]}"}}}] \leavevmode
The data-toggle-class is used to apply one or more CSS classes from the list to a snippet. Multiple selections can be applied at once.

\end{description}

Let’s demonstrate how default options work with a basic example.

We start by adding a new file in our views folder - name it \sphinxstylestrong{options.xml} and add the default Odoo XML markup. Create a new template copy/pasting the following

\fvset{hllines={, ,}}%
\begin{sphinxVerbatim}[commandchars=\\\{\}]
\PYG{n+nt}{\PYGZlt{}template} \PYG{n+na}{id=}\PYG{l+s}{\PYGZdq{}snippet\PYGZus{}testimonial\PYGZus{}opt\PYGZdq{}} \PYG{n+na}{name=}\PYG{l+s}{\PYGZdq{}Snippet Testimonial Options\PYGZdq{}} \PYG{n+na}{inherit\PYGZus{}id=}\PYG{l+s}{\PYGZdq{}website.snippet\PYGZus{}options\PYGZdq{}}\PYG{n+nt}{\PYGZgt{}}
  \PYG{n+nt}{\PYGZlt{}xpath} \PYG{n+na}{expr=}\PYG{l+s}{\PYGZdq{}//div[@data\PYGZhy{}js=\PYGZsq{}background\PYGZsq{}]\PYGZdq{}} \PYG{n+na}{position=}\PYG{l+s}{\PYGZdq{}after\PYGZdq{}}\PYG{n+nt}{\PYGZgt{}}
    \PYG{n+nt}{\PYGZlt{}div} \PYG{n+na}{data\PYGZhy{}selector=}\PYG{l+s}{\PYGZdq{}.snippet\PYGZus{}testimonial\PYGZdq{}}\PYG{n+nt}{\PYGZgt{}} \PYG{c}{\PYGZlt{}!\PYGZhy{}\PYGZhy{}}\PYG{c}{ Options group }\PYG{c}{\PYGZhy{}\PYGZhy{}\PYGZgt{}}
      \PYG{n+nt}{\PYGZlt{}li} \PYG{n+na}{class=}\PYG{l+s}{\PYGZdq{}dropdown\PYGZhy{}submenu\PYGZdq{}}\PYG{n+nt}{\PYGZgt{}}
        \PYG{n+nt}{\PYGZlt{}a} \PYG{n+na}{href=}\PYG{l+s}{\PYGZdq{}\PYGZsh{}\PYGZdq{}}\PYG{n+nt}{\PYGZgt{}}Your Option\PYG{n+nt}{\PYGZlt{}/a\PYGZgt{}}
        \PYG{n+nt}{\PYGZlt{}ul} \PYG{n+na}{class=}\PYG{l+s}{\PYGZdq{}dropdown\PYGZhy{}menu\PYGZdq{}}\PYG{n+nt}{\PYGZgt{}} \PYG{c}{\PYGZlt{}!\PYGZhy{}\PYGZhy{}}\PYG{c}{ Options list }\PYG{c}{\PYGZhy{}\PYGZhy{}\PYGZgt{}}
          \PYG{n+nt}{\PYGZlt{}li} \PYG{n+na}{data\PYGZhy{}select\PYGZhy{}class=}\PYG{l+s}{\PYGZdq{}opt\PYGZus{}shadow\PYGZdq{}}\PYG{n+nt}{\PYGZgt{}}\PYG{n+nt}{\PYGZlt{}a}\PYG{n+nt}{\PYGZgt{}}Shadow Images\PYG{n+nt}{\PYGZlt{}/a\PYGZgt{}}\PYG{n+nt}{\PYGZlt{}/li\PYGZgt{}}
          \PYG{n+nt}{\PYGZlt{}li} \PYG{n+na}{data\PYGZhy{}select\PYGZhy{}class=}\PYG{l+s}{\PYGZdq{}opt\PYGZus{}grey\PYGZus{}bg\PYGZdq{}}\PYG{n+nt}{\PYGZgt{}}\PYG{n+nt}{\PYGZlt{}a}\PYG{n+nt}{\PYGZgt{}}Grey Bg\PYG{n+nt}{\PYGZlt{}/a\PYGZgt{}}\PYG{n+nt}{\PYGZlt{}/li\PYGZgt{}}
          \PYG{n+nt}{\PYGZlt{}li} \PYG{n+na}{data\PYGZhy{}select\PYGZhy{}class=}\PYG{l+s}{\PYGZdq{}\PYGZdq{}}\PYG{n+nt}{\PYGZgt{}}\PYG{n+nt}{\PYGZlt{}a}\PYG{n+nt}{\PYGZgt{}}None\PYG{n+nt}{\PYGZlt{}/a\PYGZgt{}}\PYG{n+nt}{\PYGZlt{}/li\PYGZgt{}}
        \PYG{n+nt}{\PYGZlt{}/ul\PYGZgt{}}
      \PYG{n+nt}{\PYGZlt{}/li\PYGZgt{}}
    \PYG{n+nt}{\PYGZlt{}/div\PYGZgt{}}
  \PYG{n+nt}{\PYGZlt{}/xpath\PYGZgt{}}
 \PYG{n+nt}{\PYGZlt{}/template\PYGZgt{}}
\end{sphinxVerbatim}

\begin{sphinxadmonition}{note}{Note:}
The previous template will inherit the default \sphinxstylestrong{snippet\_options template} adding our options after the \sphinxstylestrong{background} options (xpath expr attribute).
To place your options in a particular order, inspect the \sphinxstylestrong{snippet\_options template} from the \sphinxstylestrong{website module} and add your options before/after the desired position.
\end{sphinxadmonition}

As you can see, we wrapped all our options inside a DIV tag that will
group our options and that will target them to the right selector
(\sphinxcode{\sphinxupquote{data-selector=".snippet\_testimonial"}}).

To define our options we applied \sphinxcode{\sphinxupquote{data-select-class}} attributes to the
\sphinxcode{\sphinxupquote{li}} elements. When the user selects an option, the class contained in
the attribute will automatically be applied to the element.

Since \sphinxcode{\sphinxupquote{selectClass}} method avoids multiple selections, the last “empty”
option will reset the snippet to default.

Add \sphinxstylestrong{options.xml} to \sphinxcode{\sphinxupquote{\_\_manifest\_\_.py}} and update your theme.

\noindent\sphinxincludegraphics{{restart}.png}

Dropping our snippet onto the page, you will notice that our new options are automatically added to the customize menu. Inspecting the page, you will also notice that the class will be applied to the element when selecting an option.

\noindent\sphinxincludegraphics{{snippet_options}.png}

Let’s create some css rules in order to provide a visual feedback for our options. Open our \sphinxstylestrong{style.less} file and add the following

\fvset{hllines={, ,}}%
\begin{sphinxVerbatim}[commandchars=\\\{\}]
\PYG{n+nc}{.}\PYG{n+nc}{snippet\PYGZus{}testimonial} \PYG{p}{\PYGZob{}}
  \PYG{n+nt}{border}\PYG{n+nd}{:} \PYG{n+nt}{1px} \PYG{n+nt}{solid} \PYG{n+nn}{\PYGZsh{}}\PYG{n+nn}{EAEAEA}\PYG{o}{;}
  \PYG{n+nt}{padding}\PYG{n+nd}{:} \PYG{n+nt}{20px}\PYG{o}{;}
\PYG{p}{\PYGZcb{}}

\PYG{c+c1}{// These lines will add a default style for our snippet. Now let\PYGZsq{}s create our custom rules for the options.}

\PYG{n+nc}{.}\PYG{n+nc}{snippet\PYGZus{}testimonial} \PYG{p}{\PYGZob{}}
  \PYG{n+nt}{border}\PYG{n+nd}{:} \PYG{n+nt}{1px} \PYG{n+nt}{solid} \PYG{n+nn}{\PYGZsh{}}\PYG{n+nn}{EAEAEA}\PYG{o}{;}
  \PYG{n+nt}{padding}\PYG{n+nd}{:} \PYG{n+nt}{20px}\PYG{o}{;}

  \PYG{k}{\PYGZam{}}\PYG{n+nc}{.}\PYG{n+nc}{opt\PYGZus{}shadow} \PYG{n+nt}{img} \PYG{p}{\PYGZob{}}
    \PYG{n+nt}{box\PYGZhy{}shadow}\PYG{n+nd}{:} \PYG{n+nt}{0} \PYG{n+nt}{2px} \PYG{n+nt}{5px} \PYG{n+nt}{rgba}\PYG{o}{(}\PYG{n+nt}{51}\PYG{o}{,} \PYG{n+nt}{51}\PYG{o}{,} \PYG{n+nt}{51}\PYG{o}{,} \PYG{n+nt}{0}\PYG{n+nc}{.}\PYG{n+nc}{4}\PYG{o}{)}\PYG{o}{;}
  \PYG{p}{\PYGZcb{}}

  \PYG{k}{\PYGZam{}}\PYG{n+nc}{.}\PYG{n+nc}{opt\PYGZus{}grey\PYGZus{}bg} \PYG{p}{\PYGZob{}}
    \PYG{n+nt}{border}\PYG{n+nd}{:} \PYG{n+nt}{none}\PYG{o}{;}
    \PYG{n+nt}{background\PYGZhy{}color}\PYG{n+nd}{:} \PYG{n+nn}{\PYGZsh{}}\PYG{n+nn}{EAEAEA}\PYG{o}{;}
  \PYG{p}{\PYGZcb{}}
\PYG{p}{\PYGZcb{}}
\end{sphinxVerbatim}

\noindent\sphinxincludegraphics{{snippet_options2}.png}

Great! We successfully created options for our snippet.

Any time the publisher clicks on an option, the system will add the class specified in the data-select-class attribute.

By replacing \sphinxcode{\sphinxupquote{data-select-class}} with \sphinxcode{\sphinxupquote{data-toggle-class}} you will be able to select
more classes at the same time.


\subsubsection{Javascript Options}
\label{\detokenize{howtos/themes:javascript-options}}
\sphinxcode{\sphinxupquote{data-select-class}} and \sphinxcode{\sphinxupquote{data-toggle-class}} are great if you need to perform
simple class change operations. But what if your snippet’s customization needs something more?

As we said before, \sphinxcode{\sphinxupquote{data-js}} propriety can be assigned to an options group in order to define a custom method. Let’s create one for our \sphinxstyleemphasis{testimonials snippet} by adding a \sphinxcode{\sphinxupquote{data-js}} attribute to the option’s group div that we created earlier.

\fvset{hllines={, ,}}%
\begin{sphinxVerbatim}[commandchars=\\\{\}]
\PYG{n+nt}{\PYGZlt{}div} \PYG{n+na}{data\PYGZhy{}js=}\PYG{l+s}{\PYGZdq{}snippet\PYGZus{}testimonial\PYGZus{}options\PYGZdq{}} \PYG{n+na}{data\PYGZhy{}selector=}\PYG{l+s}{\PYGZdq{}.snippet\PYGZus{}testimonial\PYGZdq{}}\PYG{n+nt}{\PYGZgt{}}
  [...]
\PYG{n+nt}{\PYGZlt{}/div\PYGZgt{}}
\end{sphinxVerbatim}

Done. From now on, the Website Builder will look for a
\sphinxcode{\sphinxupquote{snippet\_testimonial\_options}} method each time the publisher enters in edit
mode.

Let’s go one step further by creating a javascript file, name
it \sphinxstylestrong{tutorial\_editor.js} and place it into the \sphinxstylestrong{static} folder.  Copy/paste
the following code

\fvset{hllines={, ,}}%
\begin{sphinxVerbatim}[commandchars=\\\{\}]
\PYG{p}{(}\PYG{k+kd}{function}\PYG{p}{(}\PYG{p}{)} \PYG{p}{\PYGZob{}}
    \PYG{l+s+s1}{\PYGZsq{}use strict\PYGZsq{}}\PYG{p}{;}
    \PYG{k+kd}{var} \PYG{n+nx}{website} \PYG{o}{=} \PYG{n+nx}{odoo}\PYG{p}{.}\PYG{n+nx}{website}\PYG{p}{;}
    \PYG{n+nx}{website}\PYG{p}{.}\PYG{n+nx}{odoo\PYGZus{}website} \PYG{o}{=} \PYG{p}{\PYGZob{}}\PYG{p}{\PYGZcb{}}\PYG{p}{;}
\PYG{p}{\PYGZcb{}}\PYG{p}{)}\PYG{p}{(}\PYG{p}{)}\PYG{p}{;}
\end{sphinxVerbatim}

Great, we successfully created our javascript editor file. This file will contain all the javascript functions used by our snippets in edit mode. Let’s create a new function for our testimonial snippet using the \sphinxcode{\sphinxupquote{snippet\_testimonial\_options}} method that we created before.

\fvset{hllines={, ,}}%
\begin{sphinxVerbatim}[commandchars=\\\{\}]
\PYG{p}{(}\PYG{k+kd}{function}\PYG{p}{(}\PYG{p}{)} \PYG{p}{\PYGZob{}}
    \PYG{l+s+s1}{\PYGZsq{}use strict\PYGZsq{}}\PYG{p}{;}
    \PYG{k+kd}{var} \PYG{n+nx}{website} \PYG{o}{=} \PYG{n+nx}{odoo}\PYG{p}{.}\PYG{n+nx}{website}\PYG{p}{;}
    \PYG{n+nx}{website}\PYG{p}{.}\PYG{n+nx}{odoo\PYGZus{}website} \PYG{o}{=} \PYG{p}{\PYGZob{}}\PYG{p}{\PYGZcb{}}\PYG{p}{;}

    \PYG{n+nx}{website}\PYG{p}{.}\PYG{n+nx}{snippet}\PYG{p}{.}\PYG{n+nx}{options}\PYG{p}{.}\PYG{n+nx}{snippet\PYGZus{}testimonial\PYGZus{}options} \PYG{o}{=} \PYG{n+nx}{website}\PYG{p}{.}\PYG{n+nx}{snippet}\PYG{p}{.}\PYG{n+nx}{Option}\PYG{p}{.}\PYG{n+nx}{extend}\PYG{p}{(}\PYG{p}{\PYGZob{}}
        \PYG{n+nx}{onFocus}\PYG{o}{:} \PYG{k+kd}{function}\PYG{p}{(}\PYG{p}{)} \PYG{p}{\PYGZob{}}
            \PYG{n+nx}{alert}\PYG{p}{(}\PYG{l+s+s2}{\PYGZdq{}On focus!\PYGZdq{}}\PYG{p}{)}\PYG{p}{;}
        \PYG{p}{\PYGZcb{}}
    \PYG{p}{\PYGZcb{}}\PYG{p}{)}
\PYG{p}{\PYGZcb{}}\PYG{p}{)}\PYG{p}{(}\PYG{p}{)}\PYG{p}{;}
\end{sphinxVerbatim}

As you will notice, we used a method called \sphinxcode{\sphinxupquote{onFocus}} to trigger our function. The Website Builder provides several events you can use to trigger your custom functions.


\begin{savenotes}\sphinxattablestart
\centering
\begin{tabulary}{\linewidth}[t]{|T|T|}
\hline
\sphinxstyletheadfamily 
Event
&\sphinxstyletheadfamily 
Description
\\
\hline
\sphinxcode{\sphinxupquote{start}}
&
Fires when the publisher selects the snippet for the first time in an editing session or when the snippet is drag-dropped into the page
\\
\hline
\sphinxcode{\sphinxupquote{onFocus}}
&
Fires each time the snippet is selected by the user or when the snippet is drag-dropped into the page.
\\
\hline
\sphinxcode{\sphinxupquote{onBlur}}
&
This event occurs when a snippet loses focus.
\\
\hline
\sphinxcode{\sphinxupquote{onClone}}
&
Fires just after a snippet is duplicated.
\\
\hline
\sphinxcode{\sphinxupquote{onRemove}}
&
It occurs just before that the snippet is removed.
\\
\hline
\sphinxcode{\sphinxupquote{onBuilt}}
&
Fires just after that the snippet is drag and dropped into a drop zone. When this event is triggered, the content is already inserted in the page.
\\
\hline
\sphinxcode{\sphinxupquote{cleanForSave}}
&
It trigger before the publisher save the page.
\\
\hline
\end{tabulary}
\par
\sphinxattableend\end{savenotes}

Let’s add our new javascript files to the editor assets list.
Go back to \sphinxstylestrong{assets.xml} and create a new template like the previous one.
This time we have to inherit \sphinxcode{\sphinxupquote{assets\_editor}} instead of \sphinxcode{\sphinxupquote{assets\_frontend}}.

\fvset{hllines={, ,}}%
\begin{sphinxVerbatim}[commandchars=\\\{\}]
\PYG{n+nt}{\PYGZlt{}template} \PYG{n+na}{id=}\PYG{l+s}{\PYGZdq{}my\PYGZus{}js\PYGZdq{}} \PYG{n+na}{inherit\PYGZus{}id=}\PYG{l+s}{\PYGZdq{}website.assets\PYGZus{}editor\PYGZdq{}} \PYG{n+na}{name=}\PYG{l+s}{\PYGZdq{}My Js\PYGZdq{}}\PYG{n+nt}{\PYGZgt{}}
  \PYG{n+nt}{\PYGZlt{}xpath} \PYG{n+na}{expr=}\PYG{l+s}{\PYGZdq{}script[last()]\PYGZdq{}} \PYG{n+na}{position=}\PYG{l+s}{\PYGZdq{}after\PYGZdq{}}\PYG{n+nt}{\PYGZgt{}}
    \PYG{n+nt}{\PYGZlt{}script} \PYG{n+na}{type=}\PYG{l+s}{\PYGZdq{}text/javascript\PYGZdq{}} \PYG{n+na}{src=}\PYG{l+s}{\PYGZdq{}/theme\PYGZus{}tutorial/static/src/js/tutorial\PYGZus{}editor.js\PYGZdq{}} \PYG{n+nt}{/\PYGZgt{}}
  \PYG{n+nt}{\PYGZlt{}/xpath\PYGZgt{}}
\PYG{n+nt}{\PYGZlt{}/template\PYGZgt{}}
\end{sphinxVerbatim}

Update your theme

\noindent\sphinxincludegraphics{{restart}.png}

Let’s test our new javascript function. Enter in Edit mode and drop into the page.
You should now see the javascript alert that we bound on the \sphinxcode{\sphinxupquote{onFocus}} event.
If you close it, then click outside of your snippet and then click in it again, the event will trigger again.

\noindent\sphinxincludegraphics{{snippet_custom_method}.png}


\subsection{Editing Reference Guide}
\label{\detokenize{howtos/themes:editing-reference-guide}}
Basically all the elements in a page can be edited by the publisher.
Besides that, some element types and css classes will trigger special Website Builder functionalities when edited.


\subsubsection{Layout}
\label{\detokenize{howtos/themes:layout}}\begin{description}
\item[{\sphinxcode{\sphinxupquote{\textless{}section /\textgreater{}}}}] \leavevmode
Any section element can be edited like a block of content. The publisher can move or duplicate it. It’s also possible to set a background image or color. Section is the standard main container of any snippet.

\item[{\sphinxcode{\sphinxupquote{.row \textgreater{} .col-md-*}}}] \leavevmode
Any medium  bootstrap columns  directly descending from a .row element, will be resizable by the publisher.

\item[{\sphinxcode{\sphinxupquote{contenteditable="False"}}}] \leavevmode
This attribute will prevent editing to the element and all its children.

\item[{\sphinxcode{\sphinxupquote{contenteditable="True"}}}] \leavevmode
Apply it to an element inside a contenteditable=”False” element in order to create an exception and make the element and its children editable.

\item[{\sphinxcode{\sphinxupquote{\textless{}a href=”\#” /\textgreater{}}}}] \leavevmode
In Edit Mode, any link can be edited and styled. Using the “Link Modal” it’s also possible to replace it with a button.

\end{description}


\subsubsection{Media}
\label{\detokenize{howtos/themes:media}}\begin{description}
\item[{\sphinxcode{\sphinxupquote{\textless{}span class=”fa” /\textgreater{}}}}] \leavevmode
Pictogram elements. Editing this element will open the Pictogram library to replace the icon. It’s also possible to transform the elements using CSS.

\item[{\sphinxcode{\sphinxupquote{\textless{}img /\textgreater{}}}}] \leavevmode
Once clicked, the Image Library will open and you can replace images. Transformation is also possible for this kind of element.

\end{description}

\fvset{hllines={, ,}}%
\begin{sphinxVerbatim}[commandchars=\\\{\}]
\PYG{p}{\PYGZlt{}}\PYG{n+nt}{div} \PYG{n+na}{class}\PYG{o}{=}\PYG{l+s}{\PYGZdq{}media\PYGZus{}iframe\PYGZus{}video\PYGZdq{}} \PYG{n+na}{data\PYGZhy{}src}\PYG{o}{=}\PYG{l+s}{\PYGZdq{}[your url]\PYGZdq{}} \PYG{p}{\PYGZgt{}}
  \PYG{p}{\PYGZlt{}}\PYG{n+nt}{div} \PYG{n+na}{class}\PYG{o}{=}\PYG{l+s}{\PYGZdq{}css\PYGZus{}editable\PYGZus{}mode\PYGZus{}display\PYGZdq{}}\PYG{p}{/}\PYG{p}{\PYGZgt{}}
  \PYG{p}{\PYGZlt{}}\PYG{n+nt}{div} \PYG{n+na}{class}\PYG{o}{=}\PYG{l+s}{\PYGZdq{}media\PYGZus{}iframe\PYGZus{}video\PYGZus{}size\PYGZdq{}}\PYG{p}{/}\PYG{p}{\PYGZgt{}}
  \PYG{p}{\PYGZlt{}}\PYG{n+nt}{iframe} \PYG{n+na}{src}\PYG{o}{=}\PYG{l+s}{\PYGZdq{}[your url]\PYGZdq{}}\PYG{p}{/}\PYG{p}{\PYGZgt{}}
\PYG{p}{\PYGZlt{}}\PYG{p}{/}\PYG{n+nt}{div}\PYG{p}{\PYGZgt{}}
\end{sphinxVerbatim}

This html structure will create an \sphinxcode{\sphinxupquote{\textless{}iframe\textgreater{}}} element editable by the publisher.


\subsection{SEO best practice}
\label{\detokenize{howtos/themes:seo-best-practice}}

\subsubsection{Facilitate content insertion}
\label{\detokenize{howtos/themes:facilitate-content-insertion}}
Modern search engine algorithms increasingly focus on content, which means there is less focus on \sphinxstylestrong{keyword saturation} and more focus on whether or not the content is \sphinxstylestrong{actually relevant to the keywords}.

As content is so important for SEO, you should concentrate on giving publishers the tools to easily insert it. It is important that your snippets are “content-responsive”, meaning that they should fit the publisher’s content regardless of size.

Let’s have a look to this example of a classic two column snippet, implemented in two different ways.

\noindent\sphinxincludegraphics{{seo_snippet_wrong}.png}

\sphinxstylestrong{Bad}

Using fixed image, the publisher will be forced to limit the text in order to follow the layout.

\noindent\sphinxincludegraphics{{seo_snippet_good}.png}

\sphinxstylestrong{Good}

Using background images that fit the column height, the publisher will be free to add the content regardless of the image’s height.


\subsubsection{Page segmentation}
\label{\detokenize{howtos/themes:page-segmentation}}
Basically, page segmentation means that a page is divided into several separate parts and these parts are treated as separate entries by search engines.
When you design pages or snippets, you should be sure to use the right tags in order to facilitate search engine indexing.
\begin{description}
\item[{\sphinxcode{\sphinxupquote{\textless{}article\textgreater{}}}}] \leavevmode
Specifies an independent block of content. Within it should be a piece of self-contained content that should make sense on its own. You can nest \sphinxcode{\sphinxupquote{\textless{}article\textgreater{}}} elements within one another. In this case, it’s implied that the nested elements are related to the outer \sphinxcode{\sphinxupquote{\textless{}article\textgreater{}}} element.

\item[{\sphinxcode{\sphinxupquote{\textless{}header\textgreater{}}}}] \leavevmode
Indicates the header section of a self-contained block of content (an \sphinxcode{\sphinxupquote{\textless{}article\textgreater{}}}).

\item[{\sphinxcode{\sphinxupquote{\textless{}section\textgreater{}}}}] \leavevmode
Is the snippet default tag and it specifies a subsection of a block of content. It can be used to split \sphinxcode{\sphinxupquote{\textless{}article\textgreater{}}} content into several parts. It’s advisable to use a heading element (\sphinxcode{\sphinxupquote{\textless{}h1\textgreater{}}} \textendash{} \sphinxcode{\sphinxupquote{\textless{}h6\textgreater{}}}) to define the section’s topic.

\item[{\sphinxcode{\sphinxupquote{\textless{}hgroup\textgreater{}}}}] \leavevmode
Is used to wrap a section of headings (\sphinxcode{\sphinxupquote{\textless{}h1\textgreater{}}} - \sphinxcode{\sphinxupquote{\textless{}h6\textgreater{}}}). A great example would be an article with both a headline and sub-headline at the top:

\fvset{hllines={, ,}}%
\begin{sphinxVerbatim}[commandchars=\\\{\}]
\PYG{p}{\PYGZlt{}}\PYG{n+nt}{hgroup}\PYG{p}{\PYGZgt{}}
  \PYG{p}{\PYGZlt{}}\PYG{n+nt}{h1}\PYG{p}{\PYGZgt{}}Main Title\PYG{p}{\PYGZlt{}}\PYG{p}{/}\PYG{n+nt}{h1}\PYG{p}{\PYGZgt{}}
  \PYG{p}{\PYGZlt{}}\PYG{n+nt}{h2}\PYG{p}{\PYGZgt{}}Subheading\PYG{p}{\PYGZlt{}}\PYG{p}{/}\PYG{n+nt}{h2}\PYG{p}{\PYGZgt{}}
\PYG{p}{\PYGZlt{}}\PYG{p}{/}\PYG{n+nt}{hgroup}\PYG{p}{\PYGZgt{}}
\end{sphinxVerbatim}

\end{description}


\subsubsection{Describe your page}
\label{\detokenize{howtos/themes:describe-your-page}}

\paragraph{Define keywords}
\label{\detokenize{howtos/themes:define-keywords}}
You should use appropriate, relevant keywords and synonyms for those keywords. You can define them for each page using the built-in “Promote” function found in the bar at the top.


\paragraph{Define a title and a description}
\label{\detokenize{howtos/themes:define-a-title-and-a-description}}
Define them using the “Promote” function. Keep your page titles short and include the main keyword phrase for the page.
Good titles evoke an emotional response, ask a question or promise something.

Descriptions, while not important to search engine rankings, are extremely important in gaining user click-through. These are an opportunity to advertise content and to let people searching know exactly whether the given page contains the information they’re looking for. It is important that titles and descriptions on each page are unique.


\section{Building a Website}
\label{\detokenize{howtos/website:building-a-website}}\label{\detokenize{howtos/website::doc}}
\begin{sphinxadmonition}{warning}{Warning:}\begin{itemize}
\item {} 
This guide assumes \sphinxhref{http://docs.python.org/2/tutorial/}{basic knowledge of Python}

\item {} 
This guide assumes {\hyperref[\detokenize{setup/install:setup-install}]{\sphinxcrossref{\DUrole{std,std-ref}{an installed Odoo}}}}

\end{itemize}
\end{sphinxadmonition}


\subsection{Creating a basic module}
\label{\detokenize{howtos/website:creating-a-basic-module}}
In Odoo, tasks are performed by creating modules.

Modules customize the behavior of an Odoo installation, either by adding new
behaviors or by altering existing ones (including behaviors added by other
modules).

{\hyperref[\detokenize{reference/cmdline:reference-cmdline-scaffold}]{\sphinxcrossref{\DUrole{std,std-ref}{Odoo’s scaffolding}}}} can setup a basic
module. To quickly get started simply invoke:

\fvset{hllines={, ,}}%
\begin{sphinxVerbatim}[commandchars=\\\{\}]
\PYG{g+gp}{\PYGZdl{}} ./odoo\PYGZhy{}bin scaffold Academy my\PYGZhy{}modules
\end{sphinxVerbatim}

This will automatically create a \sphinxcode{\sphinxupquote{my-modules}} \sphinxstyleemphasis{module directory} with an
\sphinxcode{\sphinxupquote{academy}} module inside. The directory can be an existing module directory
if you want, but the module name must be unique within the directory.


\subsection{A demonstration module}
\label{\detokenize{howtos/website:a-demonstration-module}}
We have a “complete” module ready for installation.

Although it does absolutely nothing we can install it:
\begin{itemize}
\item {} 
start the Odoo server

\fvset{hllines={, ,}}%
\begin{sphinxVerbatim}[commandchars=\\\{\}]
\PYG{g+gp}{\PYGZdl{}} ./odoo\PYGZhy{}bin \PYGZhy{}\PYGZhy{}addons\PYGZhy{}path addons,my\PYGZhy{}modules
\end{sphinxVerbatim}

\item {} 
go to \sphinxurl{http://localhost:8069}

\item {} 
create a new database including demonstration data

\item {} 
to go \sphinxmenuselection{Settings \(\rightarrow\) Modules \(\rightarrow\) Modules}

\item {} 
in the top-right corner remove the \sphinxstyleemphasis{Installed} filter and search for
\sphinxstyleemphasis{academy}

\item {} 
click the \sphinxmenuselection{Install} button for the \sphinxstyleemphasis{Academy} module

\end{itemize}


\subsection{To the browser}
\label{\detokenize{howtos/website:to-the-browser}}
{\hyperref[\detokenize{reference/http:reference-http-controllers}]{\sphinxcrossref{\DUrole{std,std-ref}{Controllers}}}} interpret browser requests and
send data back.

Add a simple controller and ensure it is imported by \sphinxcode{\sphinxupquote{\_\_init\_\_.py}} (so
Odoo can find it):
\sphinxstyleemphasis{academy/controllers.py}
\fvset{hllines={, 4, 5, 6, 7,}}%
\begin{sphinxVerbatim}[commandchars=\\\{\}]
\PYG{c+c1}{\PYGZsh{} \PYGZhy{}*\PYGZhy{} coding: utf\PYGZhy{}8 \PYGZhy{}*\PYGZhy{}}
\PYG{k+kn}{from} \PYG{n+nn}{odoo} \PYG{k+kn}{import} \PYG{n}{http}

\PYG{k}{class} \PYG{n+nc}{Academy}\PYG{p}{(}\PYG{n}{http}\PYG{o}{.}\PYG{n}{Controller}\PYG{p}{)}\PYG{p}{:}
    \PYG{n+nd}{@http.route}\PYG{p}{(}\PYG{l+s+s1}{\PYGZsq{}}\PYG{l+s+s1}{/academy/academy/}\PYG{l+s+s1}{\PYGZsq{}}\PYG{p}{,} \PYG{n}{auth}\PYG{o}{=}\PYG{l+s+s1}{\PYGZsq{}}\PYG{l+s+s1}{public}\PYG{l+s+s1}{\PYGZsq{}}\PYG{p}{)}
    \PYG{k}{def} \PYG{n+nf}{index}\PYG{p}{(}\PYG{n+nb+bp}{self}\PYG{p}{,} \PYG{o}{*}\PYG{o}{*}\PYG{n}{kw}\PYG{p}{)}\PYG{p}{:}
        \PYG{k}{return} \PYG{l+s+s2}{\PYGZdq{}}\PYG{l+s+s2}{Hello, world}\PYG{l+s+s2}{\PYGZdq{}}

\PYG{c+c1}{\PYGZsh{}     @http.route(\PYGZsq{}/academy/academy/objects/\PYGZsq{}, auth=\PYGZsq{}public\PYGZsq{})}
\PYG{c+c1}{\PYGZsh{}     def list(self, **kw):}
\end{sphinxVerbatim}

Shut down your server (\sphinxcode{\sphinxupquote{\textasciicircum{}C}}) then restart it:

\fvset{hllines={, ,}}%
\begin{sphinxVerbatim}[commandchars=\\\{\}]
\PYG{g+gp}{\PYGZdl{}} ./odoo\PYGZhy{}bin \PYGZhy{}\PYGZhy{}addons\PYGZhy{}path addons,my\PYGZhy{}modules
\end{sphinxVerbatim}

and open a page to \sphinxurl{http://localhost:8069/academy/academy/}, you should see your
“page” appear:

\begin{figure}[htbp]
\centering

\noindent\sphinxincludegraphics{{helloworld}.png}
\end{figure}


\subsection{Templates}
\label{\detokenize{howtos/website:templates}}
Generating HTML in Python isn’t very pleasant.

The usual solution is \sphinxhref{http://en.wikipedia.org/wiki/Web\_template}{templates}, pseudo-documents with placeholders and
display logic. Odoo allows any Python templating system, but provides its
own {\hyperref[\detokenize{reference/qweb:reference-qweb}]{\sphinxcrossref{\DUrole{std,std-ref}{QWeb}}}} templating system which integrates with other
features.

Create a template and ensure the template file is registered in the
\sphinxcode{\sphinxupquote{\_\_manifest\_\_.py}} manifest, and alter the controller to use our template:
\sphinxstyleemphasis{academy/controllers.py}
\fvset{hllines={, 4, 5, 6,}}%
\begin{sphinxVerbatim}[commandchars=\\\{\}]
\PYG{k}{class} \PYG{n+nc}{Academy}\PYG{p}{(}\PYG{n}{http}\PYG{o}{.}\PYG{n}{Controller}\PYG{p}{)}\PYG{p}{:}
    \PYG{n+nd}{@http.route}\PYG{p}{(}\PYG{l+s+s1}{\PYGZsq{}}\PYG{l+s+s1}{/academy/academy/}\PYG{l+s+s1}{\PYGZsq{}}\PYG{p}{,} \PYG{n}{auth}\PYG{o}{=}\PYG{l+s+s1}{\PYGZsq{}}\PYG{l+s+s1}{public}\PYG{l+s+s1}{\PYGZsq{}}\PYG{p}{)}
    \PYG{k}{def} \PYG{n+nf}{index}\PYG{p}{(}\PYG{n+nb+bp}{self}\PYG{p}{,} \PYG{o}{*}\PYG{o}{*}\PYG{n}{kw}\PYG{p}{)}\PYG{p}{:}
        \PYG{k}{return} \PYG{n}{http}\PYG{o}{.}\PYG{n}{request}\PYG{o}{.}\PYG{n}{render}\PYG{p}{(}\PYG{l+s+s1}{\PYGZsq{}}\PYG{l+s+s1}{academy.index}\PYG{l+s+s1}{\PYGZsq{}}\PYG{p}{,} \PYG{p}{\PYGZob{}}
            \PYG{l+s+s1}{\PYGZsq{}}\PYG{l+s+s1}{teachers}\PYG{l+s+s1}{\PYGZsq{}}\PYG{p}{:} \PYG{p}{[}\PYG{l+s+s2}{\PYGZdq{}}\PYG{l+s+s2}{Diana Padilla}\PYG{l+s+s2}{\PYGZdq{}}\PYG{p}{,} \PYG{l+s+s2}{\PYGZdq{}}\PYG{l+s+s2}{Jody Caroll}\PYG{l+s+s2}{\PYGZdq{}}\PYG{p}{,} \PYG{l+s+s2}{\PYGZdq{}}\PYG{l+s+s2}{Lester Vaughn}\PYG{l+s+s2}{\PYGZdq{}}\PYG{p}{]}\PYG{p}{,}
        \PYG{p}{\PYGZcb{}}\PYG{p}{)}

\PYG{c+c1}{\PYGZsh{}     @http.route(\PYGZsq{}/academy/academy/objects/\PYGZsq{}, auth=\PYGZsq{}public\PYGZsq{})}
\PYG{c+c1}{\PYGZsh{}     def list(self, **kw):}
\end{sphinxVerbatim}
\sphinxstyleemphasis{academy/templates.xml}
\fvset{hllines={, 3, 4, 5, 6, 7, 8,}}%
\begin{sphinxVerbatim}[commandchars=\\\{\}]
\PYG{n+nt}{\PYGZlt{}odoo}\PYG{n+nt}{\PYGZgt{}}

        \PYG{n+nt}{\PYGZlt{}template} \PYG{n+na}{id=}\PYG{l+s}{\PYGZdq{}index\PYGZdq{}}\PYG{n+nt}{\PYGZgt{}}
            \PYG{n+nt}{\PYGZlt{}title}\PYG{n+nt}{\PYGZgt{}}Academy\PYG{n+nt}{\PYGZlt{}/title\PYGZgt{}}
            \PYG{n+nt}{\PYGZlt{}t} \PYG{n+na}{t\PYGZhy{}foreach=}\PYG{l+s}{\PYGZdq{}teachers\PYGZdq{}} \PYG{n+na}{t\PYGZhy{}as=}\PYG{l+s}{\PYGZdq{}teacher\PYGZdq{}}\PYG{n+nt}{\PYGZgt{}}
              \PYG{n+nt}{\PYGZlt{}p}\PYG{n+nt}{\PYGZgt{}}\PYG{n+nt}{\PYGZlt{}t} \PYG{n+na}{t\PYGZhy{}esc=}\PYG{l+s}{\PYGZdq{}teacher\PYGZdq{}}\PYG{n+nt}{/\PYGZgt{}}\PYG{n+nt}{\PYGZlt{}/p\PYGZgt{}}
            \PYG{n+nt}{\PYGZlt{}/t\PYGZgt{}}
        \PYG{n+nt}{\PYGZlt{}/template\PYGZgt{}}
        \PYG{c}{\PYGZlt{}!\PYGZhy{}\PYGZhy{}}\PYG{c}{ \PYGZlt{}template id=\PYGZdq{}object\PYGZdq{}\PYGZgt{} }\PYG{c}{\PYGZhy{}\PYGZhy{}\PYGZgt{}}
        \PYG{c}{\PYGZlt{}!\PYGZhy{}\PYGZhy{}}\PYG{c}{   \PYGZlt{}h1\PYGZgt{}\PYGZlt{}t t}\PYG{c}{\PYGZhy{}}\PYG{c}{esc=\PYGZdq{}object.display\PYGZus{}name\PYGZdq{}/\PYGZgt{}\PYGZlt{}/h1\PYGZgt{} }\PYG{c}{\PYGZhy{}\PYGZhy{}\PYGZgt{}}
        \PYG{c}{\PYGZlt{}!\PYGZhy{}\PYGZhy{}}\PYG{c}{   \PYGZlt{}dl\PYGZgt{} }\PYG{c}{\PYGZhy{}\PYGZhy{}\PYGZgt{}}
\end{sphinxVerbatim}

The templates iterates (\sphinxcode{\sphinxupquote{t-foreach}}) on all the teachers (passed through the
\sphinxstyleemphasis{template context}), and prints each teacher in its own paragraph.

Finally restart Odoo and update the module’s data (to install the template)
by going to \sphinxmenuselection{Settings \(\rightarrow\) Modules \(\rightarrow\) Modules \(\rightarrow\)
Academy} and clicking \sphinxmenuselection{Upgrade}.

\begin{sphinxadmonition}{tip}{Tip:}
Alternatively, Odoo can be restarted {\hyperref[\detokenize{reference/cmdline:cmdoption-odoo-bin-u}]{\sphinxcrossref{\sphinxcode{\sphinxupquote{and update modules at
the same time}}}}}:

\fvset{hllines={, ,}}%
\begin{sphinxVerbatim}[commandchars=\\\{\}]
\PYG{g+gp}{\PYGZdl{}} odoo\PYGZhy{}bin \PYGZhy{}\PYGZhy{}addons\PYGZhy{}path addons,my\PYGZhy{}modules \PYGZhy{}d academy \PYGZhy{}u academy
\end{sphinxVerbatim}
\end{sphinxadmonition}

Going to \sphinxurl{http://localhost:8069/academy/academy/} should now result in:

\noindent\sphinxincludegraphics{{basic-list}.png}


\subsection{Storing data in Odoo}
\label{\detokenize{howtos/website:storing-data-in-odoo}}
{\hyperref[\detokenize{reference/orm:reference-orm-model}]{\sphinxcrossref{\DUrole{std,std-ref}{Odoo models}}}} map to database tables.

In the previous section we just displayed a list of string entered statically
in the Python code. This doesn’t allow modifications or persistent storage
so we’ll now move our data to the database.


\subsubsection{Defining the data model}
\label{\detokenize{howtos/website:defining-the-data-model}}
Define a teacher model, and ensure it is imported from \sphinxcode{\sphinxupquote{\_\_init\_\_.py}} so it
is correctly loaded:
\sphinxstyleemphasis{academy/models.py}
\fvset{hllines={, 3, 4, 6,}}%
\begin{sphinxVerbatim}[commandchars=\\\{\}]

\PYG{k+kn}{from} \PYG{n+nn}{odoo} \PYG{k+kn}{import} \PYG{n}{models}\PYG{p}{,} \PYG{n}{fields}\PYG{p}{,} \PYG{n}{api}

\PYG{k}{class} \PYG{n+nc}{Teachers}\PYG{p}{(}\PYG{n}{models}\PYG{o}{.}\PYG{n}{Model}\PYG{p}{)}\PYG{p}{:}
    \PYG{n}{\PYGZus{}name} \PYG{o}{=} \PYG{l+s+s1}{\PYGZsq{}}\PYG{l+s+s1}{academy.teachers}\PYG{l+s+s1}{\PYGZsq{}}

    \PYG{n}{name} \PYG{o}{=} \PYG{n}{fields}\PYG{o}{.}\PYG{n}{Char}\PYG{p}{(}\PYG{p}{)}
\end{sphinxVerbatim}

Then setup {\hyperref[\detokenize{reference/security:reference-security-acl}]{\sphinxcrossref{\DUrole{std,std-ref}{basic access control}}}} for the model
and add them to the manifest:
\sphinxstyleemphasis{academy/\_\_manifest\_\_.py}
\fvset{hllines={, 3,}}%
\begin{sphinxVerbatim}[commandchars=\\\{\}]

    \PYG{c+c1}{\PYGZsh{} always loaded}
    \PYG{l+s+s1}{\PYGZsq{}}\PYG{l+s+s1}{data}\PYG{l+s+s1}{\PYGZsq{}}\PYG{p}{:} \PYG{p}{[}
        \PYG{l+s+s1}{\PYGZsq{}}\PYG{l+s+s1}{security/ir.model.access.csv}\PYG{l+s+s1}{\PYGZsq{}}\PYG{p}{,}
        \PYG{l+s+s1}{\PYGZsq{}}\PYG{l+s+s1}{templates.xml}\PYG{l+s+s1}{\PYGZsq{}}\PYG{p}{,}
    \PYG{p}{]}\PYG{p}{,}
    \PYG{c+c1}{\PYGZsh{} only loaded in demonstration mode}
\end{sphinxVerbatim}
\sphinxstyleemphasis{academy/security/ir.model.access.csv}
\fvset{hllines={, 2,}}%
\begin{sphinxVerbatim}[commandchars=\\\{\}]
id,name,model\PYGZus{}id:id,group\PYGZus{}id:id,perm\PYGZus{}read,perm\PYGZus{}write,perm\PYGZus{}create,perm\PYGZus{}unlink
access\PYGZus{}academy\PYGZus{}teachers,access\PYGZus{}academy\PYGZus{}teachers,model\PYGZus{}academy\PYGZus{}teachers,,1,0,0,0
\end{sphinxVerbatim}

this simply gives read access (\sphinxcode{\sphinxupquote{perm\_read}}) to all users (\sphinxcode{\sphinxupquote{group\_id:id}}
left empty).

\begin{sphinxadmonition}{note}{Note:}
{\hyperref[\detokenize{reference/data:reference-data}]{\sphinxcrossref{\DUrole{std,std-ref}{Data files}}}} (XML or CSV) must be added to the
module manifest, Python files (models or controllers) don’t but have to
be imported from \sphinxcode{\sphinxupquote{\_\_init\_\_.py}} (directly or indirectly)
\end{sphinxadmonition}

\begin{sphinxadmonition}{warning}{Warning:}
the administrator user bypasses access control, they have access to all
models even if not given access
\end{sphinxadmonition}


\subsubsection{Demonstration data}
\label{\detokenize{howtos/website:demonstration-data}}
The second step is to add some demonstration data to the system so it’s
possible to test it easily. This is done by adding a \sphinxcode{\sphinxupquote{demo}}
{\hyperref[\detokenize{reference/data:reference-data}]{\sphinxcrossref{\DUrole{std,std-ref}{data file}}}}, which must be linked from the manifest:
\sphinxstyleemphasis{academy/demo.xml}
\fvset{hllines={, 3, 4, 5, 6, 7, 8, 9, 10, 11,}}%
\begin{sphinxVerbatim}[commandchars=\\\{\}]
\PYG{n+nt}{\PYGZlt{}odoo}\PYG{n+nt}{\PYGZgt{}}

        \PYG{n+nt}{\PYGZlt{}record} \PYG{n+na}{id=}\PYG{l+s}{\PYGZdq{}padilla\PYGZdq{}} \PYG{n+na}{model=}\PYG{l+s}{\PYGZdq{}academy.teachers\PYGZdq{}}\PYG{n+nt}{\PYGZgt{}}
            \PYG{n+nt}{\PYGZlt{}field} \PYG{n+na}{name=}\PYG{l+s}{\PYGZdq{}name\PYGZdq{}}\PYG{n+nt}{\PYGZgt{}}Diana Padilla\PYG{n+nt}{\PYGZlt{}/field\PYGZgt{}}
        \PYG{n+nt}{\PYGZlt{}/record\PYGZgt{}}
        \PYG{n+nt}{\PYGZlt{}record} \PYG{n+na}{id=}\PYG{l+s}{\PYGZdq{}carroll\PYGZdq{}} \PYG{n+na}{model=}\PYG{l+s}{\PYGZdq{}academy.teachers\PYGZdq{}}\PYG{n+nt}{\PYGZgt{}}
            \PYG{n+nt}{\PYGZlt{}field} \PYG{n+na}{name=}\PYG{l+s}{\PYGZdq{}name\PYGZdq{}}\PYG{n+nt}{\PYGZgt{}}Jody Carroll\PYG{n+nt}{\PYGZlt{}/field\PYGZgt{}}
        \PYG{n+nt}{\PYGZlt{}/record\PYGZgt{}}
        \PYG{n+nt}{\PYGZlt{}record} \PYG{n+na}{id=}\PYG{l+s}{\PYGZdq{}vaughn\PYGZdq{}} \PYG{n+na}{model=}\PYG{l+s}{\PYGZdq{}academy.teachers\PYGZdq{}}\PYG{n+nt}{\PYGZgt{}}
            \PYG{n+nt}{\PYGZlt{}field} \PYG{n+na}{name=}\PYG{l+s}{\PYGZdq{}name\PYGZdq{}}\PYG{n+nt}{\PYGZgt{}}Lester Vaughn\PYG{n+nt}{\PYGZlt{}/field\PYGZgt{}}
        \PYG{n+nt}{\PYGZlt{}/record\PYGZgt{}}

\PYG{n+nt}{\PYGZlt{}/odoo\PYGZgt{}}
\end{sphinxVerbatim}

\begin{sphinxadmonition}{tip}{Tip:}
{\hyperref[\detokenize{reference/data:reference-data}]{\sphinxcrossref{\DUrole{std,std-ref}{Data files}}}} can be used for demo and non-demo data.
Demo data are only loaded in “demonstration mode” and can be used for flow
testing and demonstration, non-demo data are always loaded and used as
initial system setup.

In this case we’re using demonstration data because an actual user of the
system would want to input or import their own teachers list, this list
is only useful for testing.
\end{sphinxadmonition}


\subsubsection{Accessing the data}
\label{\detokenize{howtos/website:accessing-the-data}}
The last step is to alter model and template to use our demonstration data:
\begin{enumerate}
\item {} 
fetch the records from the database instead of having a static list

\item {} 
Because {\hyperref[\detokenize{reference/orm:odoo.models.Model.search}]{\sphinxcrossref{\sphinxcode{\sphinxupquote{search()}}}}} returns a set of records
matching the filter (“all records” here), alter the template to print each
teacher’s \sphinxcode{\sphinxupquote{name}}

\end{enumerate}
\sphinxstyleemphasis{academy/controllers.py}
\fvset{hllines={, 4, 6,}}%
\begin{sphinxVerbatim}[commandchars=\\\{\}]
\PYG{k}{class} \PYG{n+nc}{Academy}\PYG{p}{(}\PYG{n}{http}\PYG{o}{.}\PYG{n}{Controller}\PYG{p}{)}\PYG{p}{:}
    \PYG{n+nd}{@http.route}\PYG{p}{(}\PYG{l+s+s1}{\PYGZsq{}}\PYG{l+s+s1}{/academy/academy/}\PYG{l+s+s1}{\PYGZsq{}}\PYG{p}{,} \PYG{n}{auth}\PYG{o}{=}\PYG{l+s+s1}{\PYGZsq{}}\PYG{l+s+s1}{public}\PYG{l+s+s1}{\PYGZsq{}}\PYG{p}{)}
    \PYG{k}{def} \PYG{n+nf}{index}\PYG{p}{(}\PYG{n+nb+bp}{self}\PYG{p}{,} \PYG{o}{*}\PYG{o}{*}\PYG{n}{kw}\PYG{p}{)}\PYG{p}{:}
        \PYG{n}{Teachers} \PYG{o}{=} \PYG{n}{http}\PYG{o}{.}\PYG{n}{request}\PYG{o}{.}\PYG{n}{env}\PYG{p}{[}\PYG{l+s+s1}{\PYGZsq{}}\PYG{l+s+s1}{academy.teachers}\PYG{l+s+s1}{\PYGZsq{}}\PYG{p}{]}
        \PYG{k}{return} \PYG{n}{http}\PYG{o}{.}\PYG{n}{request}\PYG{o}{.}\PYG{n}{render}\PYG{p}{(}\PYG{l+s+s1}{\PYGZsq{}}\PYG{l+s+s1}{academy.index}\PYG{l+s+s1}{\PYGZsq{}}\PYG{p}{,} \PYG{p}{\PYGZob{}}
            \PYG{l+s+s1}{\PYGZsq{}}\PYG{l+s+s1}{teachers}\PYG{l+s+s1}{\PYGZsq{}}\PYG{p}{:} \PYG{n}{Teachers}\PYG{o}{.}\PYG{n}{search}\PYG{p}{(}\PYG{p}{[}\PYG{p}{]}\PYG{p}{)}
        \PYG{p}{\PYGZcb{}}\PYG{p}{)}

\PYG{c+c1}{\PYGZsh{}     @http.route(\PYGZsq{}/academy/academy/objects/\PYGZsq{}, auth=\PYGZsq{}public\PYGZsq{})}
\end{sphinxVerbatim}
\sphinxstyleemphasis{academy/templates.xml}
\fvset{hllines={, 4,}}%
\begin{sphinxVerbatim}[commandchars=\\\{\}]
        \PYG{n+nt}{\PYGZlt{}template} \PYG{n+na}{id=}\PYG{l+s}{\PYGZdq{}index\PYGZdq{}}\PYG{n+nt}{\PYGZgt{}}
            \PYG{n+nt}{\PYGZlt{}title}\PYG{n+nt}{\PYGZgt{}}Academy\PYG{n+nt}{\PYGZlt{}/title\PYGZgt{}}
            \PYG{n+nt}{\PYGZlt{}t} \PYG{n+na}{t\PYGZhy{}foreach=}\PYG{l+s}{\PYGZdq{}teachers\PYGZdq{}} \PYG{n+na}{t\PYGZhy{}as=}\PYG{l+s}{\PYGZdq{}teacher\PYGZdq{}}\PYG{n+nt}{\PYGZgt{}}
                \PYG{n+nt}{\PYGZlt{}p}\PYG{n+nt}{\PYGZgt{}}\PYG{n+nt}{\PYGZlt{}t} \PYG{n+na}{t\PYGZhy{}esc=}\PYG{l+s}{\PYGZdq{}teacher.id\PYGZdq{}}\PYG{n+nt}{/\PYGZgt{}} \PYG{n+nt}{\PYGZlt{}t} \PYG{n+na}{t\PYGZhy{}esc=}\PYG{l+s}{\PYGZdq{}teacher.name\PYGZdq{}}\PYG{n+nt}{/\PYGZgt{}}\PYG{n+nt}{\PYGZlt{}/p\PYGZgt{}}
            \PYG{n+nt}{\PYGZlt{}/t\PYGZgt{}}
        \PYG{n+nt}{\PYGZlt{}/template\PYGZgt{}}
        \PYG{c}{\PYGZlt{}!\PYGZhy{}\PYGZhy{}}\PYG{c}{ \PYGZlt{}template id=\PYGZdq{}object\PYGZdq{}\PYGZgt{} }\PYG{c}{\PYGZhy{}\PYGZhy{}\PYGZgt{}}
\end{sphinxVerbatim}

Restart the server and update the module (in order to update the manifest
and templates and load the demo file) then navigate to
\sphinxurl{http://localhost:8069/academy/academy/}. The page should look slightly
different: names should simply be prefixed by a number (the database
identifier for the teacher).


\subsection{Website support}
\label{\detokenize{howtos/website:website-support}}
Odoo bundles a module dedicated to building websites.

So far we’ve used controllers fairly directly, but Odoo 8 added deeper
integration and a few other services (e.g. default styling, theming) via the
\sphinxcode{\sphinxupquote{website}} module.
\begin{enumerate}
\item {} 
first, add \sphinxcode{\sphinxupquote{website}} as a dependency to \sphinxcode{\sphinxupquote{academy}}

\item {} 
then add the \sphinxcode{\sphinxupquote{website=True}} flag on the controller, this sets up a few
new variables on {\hyperref[\detokenize{reference/http:reference-http-request}]{\sphinxcrossref{\DUrole{std,std-ref}{the request object}}}} and
allows using the website layout in our template

\item {} 
use the website layout in the template

\end{enumerate}
\sphinxstyleemphasis{academy/\_\_manifest\_\_.py}
\fvset{hllines={, 4,}}%
\begin{sphinxVerbatim}[commandchars=\\\{\}]
    \PYG{l+s+s1}{\PYGZsq{}}\PYG{l+s+s1}{version}\PYG{l+s+s1}{\PYGZsq{}}\PYG{p}{:} \PYG{l+s+s1}{\PYGZsq{}}\PYG{l+s+s1}{0.1}\PYG{l+s+s1}{\PYGZsq{}}\PYG{p}{,}

    \PYG{c+c1}{\PYGZsh{} any module necessary for this one to work correctly}
    \PYG{l+s+s1}{\PYGZsq{}}\PYG{l+s+s1}{depends}\PYG{l+s+s1}{\PYGZsq{}}\PYG{p}{:} \PYG{p}{[}\PYG{l+s+s1}{\PYGZsq{}}\PYG{l+s+s1}{website}\PYG{l+s+s1}{\PYGZsq{}}\PYG{p}{]}\PYG{p}{,}

    \PYG{c+c1}{\PYGZsh{} always loaded}
    \PYG{l+s+s1}{\PYGZsq{}}\PYG{l+s+s1}{data}\PYG{l+s+s1}{\PYGZsq{}}\PYG{p}{:} \PYG{p}{[}
\end{sphinxVerbatim}
\sphinxstyleemphasis{academy/controllers.py}
\fvset{hllines={, 4,}}%
\begin{sphinxVerbatim}[commandchars=\\\{\}]
\PYG{k+kn}{from} \PYG{n+nn}{odoo} \PYG{k+kn}{import} \PYG{n}{http}

\PYG{k}{class} \PYG{n+nc}{Academy}\PYG{p}{(}\PYG{n}{http}\PYG{o}{.}\PYG{n}{Controller}\PYG{p}{)}\PYG{p}{:}
    \PYG{n+nd}{@http.route}\PYG{p}{(}\PYG{l+s+s1}{\PYGZsq{}}\PYG{l+s+s1}{/academy/academy/}\PYG{l+s+s1}{\PYGZsq{}}\PYG{p}{,} \PYG{n}{auth}\PYG{o}{=}\PYG{l+s+s1}{\PYGZsq{}}\PYG{l+s+s1}{public}\PYG{l+s+s1}{\PYGZsq{}}\PYG{p}{,} \PYG{n}{website}\PYG{o}{=}\PYG{n+nb+bp}{True}\PYG{p}{)}
    \PYG{k}{def} \PYG{n+nf}{index}\PYG{p}{(}\PYG{n+nb+bp}{self}\PYG{p}{,} \PYG{o}{*}\PYG{o}{*}\PYG{n}{kw}\PYG{p}{)}\PYG{p}{:}
        \PYG{n}{Teachers} \PYG{o}{=} \PYG{n}{http}\PYG{o}{.}\PYG{n}{request}\PYG{o}{.}\PYG{n}{env}\PYG{p}{[}\PYG{l+s+s1}{\PYGZsq{}}\PYG{l+s+s1}{academy.teachers}\PYG{l+s+s1}{\PYGZsq{}}\PYG{p}{]}
        \PYG{k}{return} \PYG{n}{http}\PYG{o}{.}\PYG{n}{request}\PYG{o}{.}\PYG{n}{render}\PYG{p}{(}\PYG{l+s+s1}{\PYGZsq{}}\PYG{l+s+s1}{academy.index}\PYG{l+s+s1}{\PYGZsq{}}\PYG{p}{,} \PYG{p}{\PYGZob{}}
\end{sphinxVerbatim}
\sphinxstyleemphasis{academy/templates.xml}
\fvset{hllines={, 4, 5, 6, 7, 8, 9, 10, 11, 12,}}%
\begin{sphinxVerbatim}[commandchars=\\\{\}]
\PYG{n+nt}{\PYGZlt{}odoo}\PYG{n+nt}{\PYGZgt{}}

        \PYG{n+nt}{\PYGZlt{}template} \PYG{n+na}{id=}\PYG{l+s}{\PYGZdq{}index\PYGZdq{}}\PYG{n+nt}{\PYGZgt{}}
            \PYG{n+nt}{\PYGZlt{}t} \PYG{n+na}{t\PYGZhy{}call=}\PYG{l+s}{\PYGZdq{}website.layout\PYGZdq{}}\PYG{n+nt}{\PYGZgt{}}
                \PYG{n+nt}{\PYGZlt{}t} \PYG{n+na}{t\PYGZhy{}set=}\PYG{l+s}{\PYGZdq{}title\PYGZdq{}}\PYG{n+nt}{\PYGZgt{}}Academy\PYG{n+nt}{\PYGZlt{}/t\PYGZgt{}}
                \PYG{n+nt}{\PYGZlt{}div} \PYG{n+na}{class=}\PYG{l+s}{\PYGZdq{}oe\PYGZus{}structure\PYGZdq{}}\PYG{n+nt}{\PYGZgt{}}
                    \PYG{n+nt}{\PYGZlt{}div} \PYG{n+na}{class=}\PYG{l+s}{\PYGZdq{}container\PYGZdq{}}\PYG{n+nt}{\PYGZgt{}}
                        \PYG{n+nt}{\PYGZlt{}t} \PYG{n+na}{t\PYGZhy{}foreach=}\PYG{l+s}{\PYGZdq{}teachers\PYGZdq{}} \PYG{n+na}{t\PYGZhy{}as=}\PYG{l+s}{\PYGZdq{}teacher\PYGZdq{}}\PYG{n+nt}{\PYGZgt{}}
                            \PYG{n+nt}{\PYGZlt{}p}\PYG{n+nt}{\PYGZgt{}}\PYG{n+nt}{\PYGZlt{}t} \PYG{n+na}{t\PYGZhy{}esc=}\PYG{l+s}{\PYGZdq{}teacher.id\PYGZdq{}}\PYG{n+nt}{/\PYGZgt{}} \PYG{n+nt}{\PYGZlt{}t} \PYG{n+na}{t\PYGZhy{}esc=}\PYG{l+s}{\PYGZdq{}teacher.name\PYGZdq{}}\PYG{n+nt}{/\PYGZgt{}}\PYG{n+nt}{\PYGZlt{}/p\PYGZgt{}}
                        \PYG{n+nt}{\PYGZlt{}/t\PYGZgt{}}
                    \PYG{n+nt}{\PYGZlt{}/div\PYGZgt{}}
                \PYG{n+nt}{\PYGZlt{}/div\PYGZgt{}}
            \PYG{n+nt}{\PYGZlt{}/t\PYGZgt{}}
        \PYG{n+nt}{\PYGZlt{}/template\PYGZgt{}}
        \PYG{c}{\PYGZlt{}!\PYGZhy{}\PYGZhy{}}\PYG{c}{ \PYGZlt{}template id=\PYGZdq{}object\PYGZdq{}\PYGZgt{} }\PYG{c}{\PYGZhy{}\PYGZhy{}\PYGZgt{}}
\end{sphinxVerbatim}

After restarting the server while updating the module (in order to update the
manifest and template) access \sphinxurl{http://localhost:8069/academy/academy/} should
yield a nicer looking page with branding and a number of built-in page
elements (top-level menu, footer, …)

\noindent\sphinxincludegraphics{{layout}.png}

The website layout also provides support for edition tools: click
\sphinxmenuselection{Sign In} (in the top-right), fill the credentials in (\sphinxcode{\sphinxupquote{admin}} /
\sphinxcode{\sphinxupquote{admin}} by default) then click \sphinxmenuselection{Log In}.

You’re now in Odoo “proper”: the administrative interface. For now click on
the \sphinxmenuselection{Website} menu item (top-left corner.

We’re back in the website but as an administrator, with access to advanced
edition features provided by the \sphinxstyleemphasis{website} support:
\begin{itemize}
\item {} 
a template code editor (\sphinxmenuselection{Customize \(\rightarrow\) HTML Editor}) where
you can see and edit all templates used for the current page

\item {} 
the \sphinxmenuselection{Edit} button in the top-left switches to “edition mode” where
blocks (snippets) and rich text edition are available

\item {} 
a number of other features such as mobile preview or \sphinxstyleabbreviation{SEO} (Search
Engine Optimization)

\end{itemize}


\subsection{URLs and routing}
\label{\detokenize{howtos/website:urls-and-routing}}
Controller methods are associated with \sphinxstyleemphasis{routes} via the
{\hyperref[\detokenize{reference/http:odoo.http.route}]{\sphinxcrossref{\sphinxcode{\sphinxupquote{route()}}}}} decorator which takes a routing string and a
number of attributes to customise its behavior or security.

We’ve seen a “literal” routing string, which matches a URL section exactly,
but routing strings can also use \sphinxhref{http://werkzeug.pocoo.org/docs/routing/\#rule-format}{converter patterns} which match bits
of URLs and make those available as local variables. For instance we can
create a new controller method which takes a bit of URL and prints it out:
\sphinxstyleemphasis{academy/controllers.py}
\fvset{hllines={, 4, 5, 6, 7, 8,}}%
\begin{sphinxVerbatim}[commandchars=\\\{\}]
            \PYG{l+s+s1}{\PYGZsq{}}\PYG{l+s+s1}{teachers}\PYG{l+s+s1}{\PYGZsq{}}\PYG{p}{:} \PYG{n}{Teachers}\PYG{o}{.}\PYG{n}{search}\PYG{p}{(}\PYG{p}{[}\PYG{p}{]}\PYG{p}{)}
        \PYG{p}{\PYGZcb{}}\PYG{p}{)}

    \PYG{n+nd}{@http.route}\PYG{p}{(}\PYG{l+s+s1}{\PYGZsq{}}\PYG{l+s+s1}{/academy/\PYGZlt{}name\PYGZgt{}/}\PYG{l+s+s1}{\PYGZsq{}}\PYG{p}{,} \PYG{n}{auth}\PYG{o}{=}\PYG{l+s+s1}{\PYGZsq{}}\PYG{l+s+s1}{public}\PYG{l+s+s1}{\PYGZsq{}}\PYG{p}{,} \PYG{n}{website}\PYG{o}{=}\PYG{n+nb+bp}{True}\PYG{p}{)}
    \PYG{k}{def} \PYG{n+nf}{teacher}\PYG{p}{(}\PYG{n+nb+bp}{self}\PYG{p}{,} \PYG{n}{name}\PYG{p}{)}\PYG{p}{:}
        \PYG{k}{return} \PYG{l+s+s1}{\PYGZsq{}}\PYG{l+s+s1}{\PYGZlt{}h1\PYGZgt{}\PYGZob{}\PYGZcb{}\PYGZlt{}/h1\PYGZgt{}}\PYG{l+s+s1}{\PYGZsq{}}\PYG{o}{.}\PYG{n}{format}\PYG{p}{(}\PYG{n}{name}\PYG{p}{)}


\PYG{c+c1}{\PYGZsh{}     @http.route(\PYGZsq{}/academy/academy/objects/\PYGZsq{}, auth=\PYGZsq{}public\PYGZsq{})}
\PYG{c+c1}{\PYGZsh{}     def list(self, **kw):}
\PYG{c+c1}{\PYGZsh{}         return http.request.render(\PYGZsq{}academy.listing\PYGZsq{}, \PYGZob{}}
\end{sphinxVerbatim}

restart Odoo, access \sphinxurl{http://localhost:8069/academy/Alice/} and
\sphinxurl{http://localhost:8069/academy/Bob/} and see the difference.

As the name indicates, \sphinxhref{http://werkzeug.pocoo.org/docs/routing/\#rule-format}{converter patterns} don’t just do extraction, they
also do \sphinxstyleemphasis{validation} and \sphinxstyleemphasis{conversion}, so we can change the new controller
to only accept integers:
\sphinxstyleemphasis{academy/controllers.py}
\fvset{hllines={, 4, 5, 6,}}%
\begin{sphinxVerbatim}[commandchars=\\\{\}]
            \PYG{l+s+s1}{\PYGZsq{}}\PYG{l+s+s1}{teachers}\PYG{l+s+s1}{\PYGZsq{}}\PYG{p}{:} \PYG{n}{Teachers}\PYG{o}{.}\PYG{n}{search}\PYG{p}{(}\PYG{p}{[}\PYG{p}{]}\PYG{p}{)}
        \PYG{p}{\PYGZcb{}}\PYG{p}{)}

    \PYG{n+nd}{@http.route}\PYG{p}{(}\PYG{l+s+s1}{\PYGZsq{}}\PYG{l+s+s1}{/academy/\PYGZlt{}int:id\PYGZgt{}/}\PYG{l+s+s1}{\PYGZsq{}}\PYG{p}{,} \PYG{n}{auth}\PYG{o}{=}\PYG{l+s+s1}{\PYGZsq{}}\PYG{l+s+s1}{public}\PYG{l+s+s1}{\PYGZsq{}}\PYG{p}{,} \PYG{n}{website}\PYG{o}{=}\PYG{n+nb+bp}{True}\PYG{p}{)}
    \PYG{k}{def} \PYG{n+nf}{teacher}\PYG{p}{(}\PYG{n+nb+bp}{self}\PYG{p}{,} \PYG{n+nb}{id}\PYG{p}{)}\PYG{p}{:}
        \PYG{k}{return} \PYG{l+s+s1}{\PYGZsq{}}\PYG{l+s+s1}{\PYGZlt{}h1\PYGZgt{}\PYGZob{}\PYGZcb{} (\PYGZob{}\PYGZcb{})\PYGZlt{}/h1\PYGZgt{}}\PYG{l+s+s1}{\PYGZsq{}}\PYG{o}{.}\PYG{n}{format}\PYG{p}{(}\PYG{n+nb}{id}\PYG{p}{,} \PYG{n+nb}{type}\PYG{p}{(}\PYG{n+nb}{id}\PYG{p}{)}\PYG{o}{.}\PYG{n}{\PYGZus{}\PYGZus{}name\PYGZus{}\PYGZus{}}\PYG{p}{)}


\PYG{c+c1}{\PYGZsh{}     @http.route(\PYGZsq{}/academy/academy/objects/\PYGZsq{}, auth=\PYGZsq{}public\PYGZsq{})}
\end{sphinxVerbatim}

Restart Odoo, access \sphinxurl{http://localhost:8069/academy/2}, note how the old value
was a string, but the new one was converted to an integers. Try accessing
\sphinxurl{http://localhost:8069/academy/Carol/} and note that the page was not found:
since “Carol” is not an integer, the route was ignored and no route could be
found.

Odoo provides an additional converter called \sphinxcode{\sphinxupquote{model}} which provides records
directly when given their id. Let’s use this to create a generic page for
teacher biographies:
\sphinxstyleemphasis{academy/controllers.py}
\fvset{hllines={, 4, 5, 6, 7, 8,}}%
\begin{sphinxVerbatim}[commandchars=\\\{\}]
            \PYG{l+s+s1}{\PYGZsq{}}\PYG{l+s+s1}{teachers}\PYG{l+s+s1}{\PYGZsq{}}\PYG{p}{:} \PYG{n}{Teachers}\PYG{o}{.}\PYG{n}{search}\PYG{p}{(}\PYG{p}{[}\PYG{p}{]}\PYG{p}{)}
        \PYG{p}{\PYGZcb{}}\PYG{p}{)}

    \PYG{n+nd}{@http.route}\PYG{p}{(}\PYG{l+s+s1}{\PYGZsq{}}\PYG{l+s+s1}{/academy/\PYGZlt{}model(}\PYG{l+s+s1}{\PYGZdq{}}\PYG{l+s+s1}{academy.teachers}\PYG{l+s+s1}{\PYGZdq{}}\PYG{l+s+s1}{):teacher\PYGZgt{}/}\PYG{l+s+s1}{\PYGZsq{}}\PYG{p}{,} \PYG{n}{auth}\PYG{o}{=}\PYG{l+s+s1}{\PYGZsq{}}\PYG{l+s+s1}{public}\PYG{l+s+s1}{\PYGZsq{}}\PYG{p}{,} \PYG{n}{website}\PYG{o}{=}\PYG{n+nb+bp}{True}\PYG{p}{)}
    \PYG{k}{def} \PYG{n+nf}{teacher}\PYG{p}{(}\PYG{n+nb+bp}{self}\PYG{p}{,} \PYG{n}{teacher}\PYG{p}{)}\PYG{p}{:}
        \PYG{k}{return} \PYG{n}{http}\PYG{o}{.}\PYG{n}{request}\PYG{o}{.}\PYG{n}{render}\PYG{p}{(}\PYG{l+s+s1}{\PYGZsq{}}\PYG{l+s+s1}{academy.biography}\PYG{l+s+s1}{\PYGZsq{}}\PYG{p}{,} \PYG{p}{\PYGZob{}}
            \PYG{l+s+s1}{\PYGZsq{}}\PYG{l+s+s1}{person}\PYG{l+s+s1}{\PYGZsq{}}\PYG{p}{:} \PYG{n}{teacher}
        \PYG{p}{\PYGZcb{}}\PYG{p}{)}


\PYG{c+c1}{\PYGZsh{}     @http.route(\PYGZsq{}/academy/academy/objects/\PYGZsq{}, auth=\PYGZsq{}public\PYGZsq{})}
\end{sphinxVerbatim}
\sphinxstyleemphasis{academy/templates.xml}
\fvset{hllines={, 4, 5, 6, 7, 8, 9, 10, 11, 12, 13, 14, 15,}}%
\begin{sphinxVerbatim}[commandchars=\\\{\}]
                \PYG{n+nt}{\PYGZlt{}/div\PYGZgt{}}
            \PYG{n+nt}{\PYGZlt{}/t\PYGZgt{}}
        \PYG{n+nt}{\PYGZlt{}/template\PYGZgt{}}
        \PYG{n+nt}{\PYGZlt{}template} \PYG{n+na}{id=}\PYG{l+s}{\PYGZdq{}biography\PYGZdq{}}\PYG{n+nt}{\PYGZgt{}}
            \PYG{n+nt}{\PYGZlt{}t} \PYG{n+na}{t\PYGZhy{}call=}\PYG{l+s}{\PYGZdq{}website.layout\PYGZdq{}}\PYG{n+nt}{\PYGZgt{}}
                \PYG{n+nt}{\PYGZlt{}t} \PYG{n+na}{t\PYGZhy{}set=}\PYG{l+s}{\PYGZdq{}title\PYGZdq{}}\PYG{n+nt}{\PYGZgt{}}Academy\PYG{n+nt}{\PYGZlt{}/t\PYGZgt{}}
                \PYG{n+nt}{\PYGZlt{}div} \PYG{n+na}{class=}\PYG{l+s}{\PYGZdq{}oe\PYGZus{}structure\PYGZdq{}}\PYG{n+nt}{/\PYGZgt{}}
                \PYG{n+nt}{\PYGZlt{}div} \PYG{n+na}{class=}\PYG{l+s}{\PYGZdq{}oe\PYGZus{}structure\PYGZdq{}}\PYG{n+nt}{\PYGZgt{}}
                    \PYG{n+nt}{\PYGZlt{}div} \PYG{n+na}{class=}\PYG{l+s}{\PYGZdq{}container\PYGZdq{}}\PYG{n+nt}{\PYGZgt{}}
                        \PYG{n+nt}{\PYGZlt{}p}\PYG{n+nt}{\PYGZgt{}}\PYG{n+nt}{\PYGZlt{}t} \PYG{n+na}{t\PYGZhy{}esc=}\PYG{l+s}{\PYGZdq{}person.id\PYGZdq{}}\PYG{n+nt}{/\PYGZgt{}} \PYG{n+nt}{\PYGZlt{}t} \PYG{n+na}{t\PYGZhy{}esc=}\PYG{l+s}{\PYGZdq{}person.name\PYGZdq{}}\PYG{n+nt}{/\PYGZgt{}}\PYG{n+nt}{\PYGZlt{}/p\PYGZgt{}}
                    \PYG{n+nt}{\PYGZlt{}/div\PYGZgt{}}
                \PYG{n+nt}{\PYGZlt{}/div\PYGZgt{}}
                \PYG{n+nt}{\PYGZlt{}div} \PYG{n+na}{class=}\PYG{l+s}{\PYGZdq{}oe\PYGZus{}structure\PYGZdq{}}\PYG{n+nt}{/\PYGZgt{}}
            \PYG{n+nt}{\PYGZlt{}/t\PYGZgt{}}
        \PYG{n+nt}{\PYGZlt{}/template\PYGZgt{}}
        \PYG{c}{\PYGZlt{}!\PYGZhy{}\PYGZhy{}}\PYG{c}{ \PYGZlt{}template id=\PYGZdq{}object\PYGZdq{}\PYGZgt{} }\PYG{c}{\PYGZhy{}\PYGZhy{}\PYGZgt{}}
        \PYG{c}{\PYGZlt{}!\PYGZhy{}\PYGZhy{}}\PYG{c}{   \PYGZlt{}h1\PYGZgt{}\PYGZlt{}t t}\PYG{c}{\PYGZhy{}}\PYG{c}{esc=\PYGZdq{}object.display\PYGZus{}name\PYGZdq{}/\PYGZgt{}\PYGZlt{}/h1\PYGZgt{} }\PYG{c}{\PYGZhy{}\PYGZhy{}\PYGZgt{}}
        \PYG{c}{\PYGZlt{}!\PYGZhy{}\PYGZhy{}}\PYG{c}{   \PYGZlt{}dl\PYGZgt{} }\PYG{c}{\PYGZhy{}\PYGZhy{}\PYGZgt{}}
\end{sphinxVerbatim}

then change the list of model to link to our new controller:
\sphinxstyleemphasis{academy/templates.xml}
\fvset{hllines={, 4, 5, 6,}}%
\begin{sphinxVerbatim}[commandchars=\\\{\}]
                \PYG{n+nt}{\PYGZlt{}div} \PYG{n+na}{class=}\PYG{l+s}{\PYGZdq{}oe\PYGZus{}structure\PYGZdq{}}\PYG{n+nt}{\PYGZgt{}}
                    \PYG{n+nt}{\PYGZlt{}div} \PYG{n+na}{class=}\PYG{l+s}{\PYGZdq{}container\PYGZdq{}}\PYG{n+nt}{\PYGZgt{}}
                        \PYG{n+nt}{\PYGZlt{}t} \PYG{n+na}{t\PYGZhy{}foreach=}\PYG{l+s}{\PYGZdq{}teachers\PYGZdq{}} \PYG{n+na}{t\PYGZhy{}as=}\PYG{l+s}{\PYGZdq{}teacher\PYGZdq{}}\PYG{n+nt}{\PYGZgt{}}
                            \PYG{n+nt}{\PYGZlt{}p}\PYG{n+nt}{\PYGZgt{}}\PYG{n+nt}{\PYGZlt{}a} \PYG{n+na}{t\PYGZhy{}attf\PYGZhy{}href=}\PYG{l+s}{\PYGZdq{}/academy/}\PYG{c+cp}{\PYGZob{}\PYGZob{}} \PYG{n+nv}{slug}\PYG{o}{(}\PYG{n+nv}{teacher}\PYG{o}{)} \PYG{c+cp}{\PYGZcb{}\PYGZcb{}}\PYG{l+s}{\PYGZdq{}}\PYG{n+nt}{\PYGZgt{}}
                              \PYG{n+nt}{\PYGZlt{}t} \PYG{n+na}{t\PYGZhy{}esc=}\PYG{l+s}{\PYGZdq{}teacher.name\PYGZdq{}}\PYG{n+nt}{/\PYGZgt{}}\PYG{n+nt}{\PYGZlt{}/a\PYGZgt{}}
                            \PYG{n+nt}{\PYGZlt{}/p\PYGZgt{}}
                        \PYG{n+nt}{\PYGZlt{}/t\PYGZgt{}}
                    \PYG{n+nt}{\PYGZlt{}/div\PYGZgt{}}
                \PYG{n+nt}{\PYGZlt{}/div\PYGZgt{}}
\end{sphinxVerbatim}

\fvset{hllines={, 4,}}%
\begin{sphinxVerbatim}[commandchars=\\\{\}]
                \PYG{n+nt}{\PYGZlt{}div} \PYG{n+na}{class=}\PYG{l+s}{\PYGZdq{}oe\PYGZus{}structure\PYGZdq{}}\PYG{n+nt}{/\PYGZgt{}}
                \PYG{n+nt}{\PYGZlt{}div} \PYG{n+na}{class=}\PYG{l+s}{\PYGZdq{}oe\PYGZus{}structure\PYGZdq{}}\PYG{n+nt}{\PYGZgt{}}
                    \PYG{n+nt}{\PYGZlt{}div} \PYG{n+na}{class=}\PYG{l+s}{\PYGZdq{}container\PYGZdq{}}\PYG{n+nt}{\PYGZgt{}}
                        \PYG{n+nt}{\PYGZlt{}h3}\PYG{n+nt}{\PYGZgt{}}\PYG{n+nt}{\PYGZlt{}t} \PYG{n+na}{t\PYGZhy{}esc=}\PYG{l+s}{\PYGZdq{}person.name\PYGZdq{}}\PYG{n+nt}{/\PYGZgt{}}\PYG{n+nt}{\PYGZlt{}/h3\PYGZgt{}}
                    \PYG{n+nt}{\PYGZlt{}/div\PYGZgt{}}
                \PYG{n+nt}{\PYGZlt{}/div\PYGZgt{}}
                \PYG{n+nt}{\PYGZlt{}div} \PYG{n+na}{class=}\PYG{l+s}{\PYGZdq{}oe\PYGZus{}structure\PYGZdq{}}\PYG{n+nt}{/\PYGZgt{}}
\end{sphinxVerbatim}

Restart Odoo and upgrade the module, then you can visit each teacher’s page.
As an exercise, try adding blocks to a teacher’s page to write a biography,
then go to another teacher’s page and so forth. You will discover, that your
biography is shared between all teachers, because blocks are added to the
\sphinxstyleemphasis{template}, and the \sphinxstyleemphasis{biography} template is shared between all teachers, when
one page is edited they’re all edited at the same time.


\subsection{Field edition}
\label{\detokenize{howtos/website:field-edition}}
Data which is specific to a record should be saved on that record, so let us
add a new biography field to our teachers:
\sphinxstyleemphasis{academy/models.py}
\fvset{hllines={, 4,}}%
\begin{sphinxVerbatim}[commandchars=\\\{\}]
    \PYG{n}{\PYGZus{}name} \PYG{o}{=} \PYG{l+s+s1}{\PYGZsq{}}\PYG{l+s+s1}{academy.teachers}\PYG{l+s+s1}{\PYGZsq{}}

    \PYG{n}{name} \PYG{o}{=} \PYG{n}{fields}\PYG{o}{.}\PYG{n}{Char}\PYG{p}{(}\PYG{p}{)}
    \PYG{n}{biography} \PYG{o}{=} \PYG{n}{fields}\PYG{o}{.}\PYG{n}{Html}\PYG{p}{(}\PYG{p}{)}
\end{sphinxVerbatim}
\sphinxstyleemphasis{academy/templates.xml}
\fvset{hllines={, 4,}}%
\begin{sphinxVerbatim}[commandchars=\\\{\}]
                \PYG{n+nt}{\PYGZlt{}div} \PYG{n+na}{class=}\PYG{l+s}{\PYGZdq{}oe\PYGZus{}structure\PYGZdq{}}\PYG{n+nt}{\PYGZgt{}}
                    \PYG{n+nt}{\PYGZlt{}div} \PYG{n+na}{class=}\PYG{l+s}{\PYGZdq{}container\PYGZdq{}}\PYG{n+nt}{\PYGZgt{}}
                        \PYG{n+nt}{\PYGZlt{}h3}\PYG{n+nt}{\PYGZgt{}}\PYG{n+nt}{\PYGZlt{}t} \PYG{n+na}{t\PYGZhy{}esc=}\PYG{l+s}{\PYGZdq{}person.name\PYGZdq{}}\PYG{n+nt}{/\PYGZgt{}}\PYG{n+nt}{\PYGZlt{}/h3\PYGZgt{}}
                        \PYG{n+nt}{\PYGZlt{}div}\PYG{n+nt}{\PYGZgt{}}\PYG{n+nt}{\PYGZlt{}t} \PYG{n+na}{t\PYGZhy{}esc=}\PYG{l+s}{\PYGZdq{}person.biography\PYGZdq{}}\PYG{n+nt}{/\PYGZgt{}}\PYG{n+nt}{\PYGZlt{}/div\PYGZgt{}}
                    \PYG{n+nt}{\PYGZlt{}/div\PYGZgt{}}
                \PYG{n+nt}{\PYGZlt{}/div\PYGZgt{}}
                \PYG{n+nt}{\PYGZlt{}div} \PYG{n+na}{class=}\PYG{l+s}{\PYGZdq{}oe\PYGZus{}structure\PYGZdq{}}\PYG{n+nt}{/\PYGZgt{}}
\end{sphinxVerbatim}

Restart Odoo and update the views, reload the teacher’s page and… the field
is invisible since it contains nothing.

For record fields, templates can use a special \sphinxcode{\sphinxupquote{t-field}} directive which
allows editing the field content from the website using field-specific
interfaces. Change the \sphinxstyleemphasis{person} template to use \sphinxcode{\sphinxupquote{t-field}}:
\sphinxstyleemphasis{academy/templates.xml}
\fvset{hllines={, 4, 5,}}%
\begin{sphinxVerbatim}[commandchars=\\\{\}]
                \PYG{n+nt}{\PYGZlt{}div} \PYG{n+na}{class=}\PYG{l+s}{\PYGZdq{}oe\PYGZus{}structure\PYGZdq{}}\PYG{n+nt}{/\PYGZgt{}}
                \PYG{n+nt}{\PYGZlt{}div} \PYG{n+na}{class=}\PYG{l+s}{\PYGZdq{}oe\PYGZus{}structure\PYGZdq{}}\PYG{n+nt}{\PYGZgt{}}
                    \PYG{n+nt}{\PYGZlt{}div} \PYG{n+na}{class=}\PYG{l+s}{\PYGZdq{}container\PYGZdq{}}\PYG{n+nt}{\PYGZgt{}}
                        \PYG{n+nt}{\PYGZlt{}h3} \PYG{n+na}{t\PYGZhy{}field=}\PYG{l+s}{\PYGZdq{}person.name\PYGZdq{}}\PYG{n+nt}{/\PYGZgt{}}
                        \PYG{n+nt}{\PYGZlt{}div} \PYG{n+na}{t\PYGZhy{}field=}\PYG{l+s}{\PYGZdq{}person.biography\PYGZdq{}}\PYG{n+nt}{/\PYGZgt{}}
                    \PYG{n+nt}{\PYGZlt{}/div\PYGZgt{}}
                \PYG{n+nt}{\PYGZlt{}/div\PYGZgt{}}
                \PYG{n+nt}{\PYGZlt{}div} \PYG{n+na}{class=}\PYG{l+s}{\PYGZdq{}oe\PYGZus{}structure\PYGZdq{}}\PYG{n+nt}{/\PYGZgt{}}
\end{sphinxVerbatim}

Restart Odoo and upgrade the module, there is now a placeholder under the
teacher’s name and a new zone for blocks in \sphinxmenuselection{Edit} mode. Content
dropped there is stored in the corresponding teacher’s \sphinxcode{\sphinxupquote{biography}} field, and
thus specific to that teacher.

The teacher’s name is also editable, and when saved the change is visible on
the index page.

\sphinxcode{\sphinxupquote{t-field}} can also take formatting options which depend on the exact field.
For instance if we display the modification date for a teacher’s record:
\sphinxstyleemphasis{academy/templates.xml}
\fvset{hllines={, 4,}}%
\begin{sphinxVerbatim}[commandchars=\\\{\}]
                \PYG{n+nt}{\PYGZlt{}div} \PYG{n+na}{class=}\PYG{l+s}{\PYGZdq{}oe\PYGZus{}structure\PYGZdq{}}\PYG{n+nt}{\PYGZgt{}}
                    \PYG{n+nt}{\PYGZlt{}div} \PYG{n+na}{class=}\PYG{l+s}{\PYGZdq{}container\PYGZdq{}}\PYG{n+nt}{\PYGZgt{}}
                        \PYG{n+nt}{\PYGZlt{}h3} \PYG{n+na}{t\PYGZhy{}field=}\PYG{l+s}{\PYGZdq{}person.name\PYGZdq{}}\PYG{n+nt}{/\PYGZgt{}}
                        \PYG{n+nt}{\PYGZlt{}p}\PYG{n+nt}{\PYGZgt{}}Last modified: \PYG{n+nt}{\PYGZlt{}i} \PYG{n+na}{t\PYGZhy{}field=}\PYG{l+s}{\PYGZdq{}person.write\PYGZus{}date\PYGZdq{}}\PYG{n+nt}{/\PYGZgt{}}\PYG{n+nt}{\PYGZlt{}/p\PYGZgt{}}
                        \PYG{n+nt}{\PYGZlt{}div} \PYG{n+na}{t\PYGZhy{}field=}\PYG{l+s}{\PYGZdq{}person.biography\PYGZdq{}}\PYG{n+nt}{/\PYGZgt{}}
                    \PYG{n+nt}{\PYGZlt{}/div\PYGZgt{}}
                \PYG{n+nt}{\PYGZlt{}/div\PYGZgt{}}
\end{sphinxVerbatim}

it is displayed in a very “computery” manner and hard to read, but we could
ask for a human-readable version:
\sphinxstyleemphasis{academy/templates.xml}
\fvset{hllines={, 4,}}%
\begin{sphinxVerbatim}[commandchars=\\\{\}]
                \PYG{n+nt}{\PYGZlt{}div} \PYG{n+na}{class=}\PYG{l+s}{\PYGZdq{}oe\PYGZus{}structure\PYGZdq{}}\PYG{n+nt}{\PYGZgt{}}
                    \PYG{n+nt}{\PYGZlt{}div} \PYG{n+na}{class=}\PYG{l+s}{\PYGZdq{}container\PYGZdq{}}\PYG{n+nt}{\PYGZgt{}}
                        \PYG{n+nt}{\PYGZlt{}h3} \PYG{n+na}{t\PYGZhy{}field=}\PYG{l+s}{\PYGZdq{}person.name\PYGZdq{}}\PYG{n+nt}{/\PYGZgt{}}
                        \PYG{n+nt}{\PYGZlt{}p}\PYG{n+nt}{\PYGZgt{}}Last modified: \PYG{n+nt}{\PYGZlt{}i} \PYG{n+na}{t\PYGZhy{}field=}\PYG{l+s}{\PYGZdq{}person.write\PYGZus{}date\PYGZdq{}} \PYG{n+na}{t\PYGZhy{}options=}\PYG{l+s}{\PYGZsq{}\PYGZob{}\PYGZdq{}format\PYGZdq{}: \PYGZdq{}long\PYGZdq{}\PYGZcb{}\PYGZsq{}}\PYG{n+nt}{/\PYGZgt{}}\PYG{n+nt}{\PYGZlt{}/p\PYGZgt{}}
                        \PYG{n+nt}{\PYGZlt{}div} \PYG{n+na}{t\PYGZhy{}field=}\PYG{l+s}{\PYGZdq{}person.biography\PYGZdq{}}\PYG{n+nt}{/\PYGZgt{}}
                    \PYG{n+nt}{\PYGZlt{}/div\PYGZgt{}}
                \PYG{n+nt}{\PYGZlt{}/div\PYGZgt{}}
\end{sphinxVerbatim}

or a relative display:
\sphinxstyleemphasis{academy/templates.xml}
\fvset{hllines={, 4,}}%
\begin{sphinxVerbatim}[commandchars=\\\{\}]
                \PYG{n+nt}{\PYGZlt{}div} \PYG{n+na}{class=}\PYG{l+s}{\PYGZdq{}oe\PYGZus{}structure\PYGZdq{}}\PYG{n+nt}{\PYGZgt{}}
                    \PYG{n+nt}{\PYGZlt{}div} \PYG{n+na}{class=}\PYG{l+s}{\PYGZdq{}container\PYGZdq{}}\PYG{n+nt}{\PYGZgt{}}
                        \PYG{n+nt}{\PYGZlt{}h3} \PYG{n+na}{t\PYGZhy{}field=}\PYG{l+s}{\PYGZdq{}person.name\PYGZdq{}}\PYG{n+nt}{/\PYGZgt{}}
                        \PYG{n+nt}{\PYGZlt{}p}\PYG{n+nt}{\PYGZgt{}}Last modified: \PYG{n+nt}{\PYGZlt{}i} \PYG{n+na}{t\PYGZhy{}field=}\PYG{l+s}{\PYGZdq{}person.write\PYGZus{}date\PYGZdq{}} \PYG{n+na}{t\PYGZhy{}options=}\PYG{l+s}{\PYGZsq{}\PYGZob{}\PYGZdq{}widget\PYGZdq{}: \PYGZdq{}relative\PYGZdq{}\PYGZcb{}\PYGZsq{}}\PYG{n+nt}{/\PYGZgt{}}\PYG{n+nt}{\PYGZlt{}/p\PYGZgt{}}
                        \PYG{n+nt}{\PYGZlt{}div} \PYG{n+na}{t\PYGZhy{}field=}\PYG{l+s}{\PYGZdq{}person.biography\PYGZdq{}}\PYG{n+nt}{/\PYGZgt{}}
                    \PYG{n+nt}{\PYGZlt{}/div\PYGZgt{}}
                \PYG{n+nt}{\PYGZlt{}/div\PYGZgt{}}
\end{sphinxVerbatim}


\subsection{Administration and ERP integration}
\label{\detokenize{howtos/website:administration-and-erp-integration}}

\subsubsection{A brief and incomplete introduction to the Odoo administration}
\label{\detokenize{howtos/website:a-brief-and-incomplete-introduction-to-the-odoo-administration}}
The Odoo administration was briefly seen during the {\hyperref[\detokenize{howtos/website:website-support}]{\sphinxcrossref{website support}}} section.
We can go back to it using \sphinxmenuselection{Administrator \(\rightarrow\) Administrator} in
the menu (or \sphinxmenuselection{Sign In} if you’re signed out).

The conceptual structure of the Odoo backend is simple:
\begin{enumerate}
\item {} 
first are menus, a tree (menus can have sub-menus) of records. Menus
without children map to…

\item {} 
actions. Actions have various types: links, reports, code which Odoo should
execute or data display. Data display actions are called \sphinxstyleemphasis{window actions},
and tell Odoo to display a given \sphinxstyleemphasis{model} according to a set of views…

\item {} 
a view has a type, a broad category to which it corresponds (a list,
a graph, a calendar) and an \sphinxstyleemphasis{architecture} which customises the way the
model is displayed inside the view.

\end{enumerate}


\subsubsection{Editing in the Odoo administration}
\label{\detokenize{howtos/website:editing-in-the-odoo-administration}}
By default, an Odoo model is essentially invisible to a user. To make it
visible it must be available through an action, which itself needs to be
reachable, generally through a menu.

Let’s create a menu for our model:
\sphinxstyleemphasis{academy/\_\_manifest\_\_.py}
\fvset{hllines={, 4,}}%
\begin{sphinxVerbatim}[commandchars=\\\{\}]
    \PYG{l+s+s1}{\PYGZsq{}}\PYG{l+s+s1}{data}\PYG{l+s+s1}{\PYGZsq{}}\PYG{p}{:} \PYG{p}{[}
        \PYG{l+s+s1}{\PYGZsq{}}\PYG{l+s+s1}{security/ir.model.access.csv}\PYG{l+s+s1}{\PYGZsq{}}\PYG{p}{,}
        \PYG{l+s+s1}{\PYGZsq{}}\PYG{l+s+s1}{templates.xml}\PYG{l+s+s1}{\PYGZsq{}}\PYG{p}{,}
        \PYG{l+s+s1}{\PYGZsq{}}\PYG{l+s+s1}{views.xml}\PYG{l+s+s1}{\PYGZsq{}}\PYG{p}{,}
    \PYG{p}{]}\PYG{p}{,}
    \PYG{c+c1}{\PYGZsh{} only loaded in demonstration mode}
    \PYG{l+s+s1}{\PYGZsq{}}\PYG{l+s+s1}{demo}\PYG{l+s+s1}{\PYGZsq{}}\PYG{p}{:} \PYG{p}{[}
\end{sphinxVerbatim}
\sphinxstyleemphasis{academy/views.xml}
\fvset{hllines={, 1, 2, 3, 4, 5, 6, 7, 8, 9, 10, 11, 12, 13,}}%
\begin{sphinxVerbatim}[commandchars=\\\{\}]
\PYG{n+nt}{\PYGZlt{}odoo}\PYG{n+nt}{\PYGZgt{}}

  \PYG{n+nt}{\PYGZlt{}record} \PYG{n+na}{id=}\PYG{l+s}{\PYGZdq{}action\PYGZus{}academy\PYGZus{}teachers\PYGZdq{}} \PYG{n+na}{model=}\PYG{l+s}{\PYGZdq{}ir.actions.act\PYGZus{}window\PYGZdq{}}\PYG{n+nt}{\PYGZgt{}}
    \PYG{n+nt}{\PYGZlt{}field} \PYG{n+na}{name=}\PYG{l+s}{\PYGZdq{}name\PYGZdq{}}\PYG{n+nt}{\PYGZgt{}}Academy teachers\PYG{n+nt}{\PYGZlt{}/field\PYGZgt{}}
    \PYG{n+nt}{\PYGZlt{}field} \PYG{n+na}{name=}\PYG{l+s}{\PYGZdq{}res\PYGZus{}model\PYGZdq{}}\PYG{n+nt}{\PYGZgt{}}academy.teachers\PYG{n+nt}{\PYGZlt{}/field\PYGZgt{}}
  \PYG{n+nt}{\PYGZlt{}/record\PYGZgt{}}

  \PYG{n+nt}{\PYGZlt{}menuitem} \PYG{n+na}{sequence=}\PYG{l+s}{\PYGZdq{}0\PYGZdq{}} \PYG{n+na}{id=}\PYG{l+s}{\PYGZdq{}menu\PYGZus{}academy\PYGZdq{}} \PYG{n+na}{name=}\PYG{l+s}{\PYGZdq{}Academy\PYGZdq{}}\PYG{n+nt}{/\PYGZgt{}}
  \PYG{n+nt}{\PYGZlt{}menuitem} \PYG{n+na}{id=}\PYG{l+s}{\PYGZdq{}menu\PYGZus{}academy\PYGZus{}content\PYGZdq{}} \PYG{n+na}{parent=}\PYG{l+s}{\PYGZdq{}menu\PYGZus{}academy\PYGZdq{}}
            \PYG{n+na}{name=}\PYG{l+s}{\PYGZdq{}Academy Content\PYGZdq{}}\PYG{n+nt}{/\PYGZgt{}}
  \PYG{n+nt}{\PYGZlt{}menuitem} \PYG{n+na}{id=}\PYG{l+s}{\PYGZdq{}menu\PYGZus{}academy\PYGZus{}content\PYGZus{}teachers\PYGZdq{}}
            \PYG{n+na}{parent=}\PYG{l+s}{\PYGZdq{}menu\PYGZus{}academy\PYGZus{}content\PYGZdq{}}
            \PYG{n+na}{action=}\PYG{l+s}{\PYGZdq{}action\PYGZus{}academy\PYGZus{}teachers\PYGZdq{}}\PYG{n+nt}{/\PYGZgt{}}
\end{sphinxVerbatim}

then accessing \sphinxurl{http://localhost:8069/web/} in the top left should be a menu
\sphinxmenuselection{Academy}, which is selected by default, as it is the first menu,
and having opened a listing of teachers. From the listing it is possible to
\sphinxmenuselection{Create} new teacher records, and to switch to the “form” by-record
view.

If there is no definition of how to present records (a
{\hyperref[\detokenize{reference/views:reference-views}]{\sphinxcrossref{\DUrole{std,std-ref}{view}}}}) Odoo will automatically create a basic one
on-the-fly. In our case it works for the “list” view for now (only displays
the teacher’s name) but in the “form” view the HTML \sphinxcode{\sphinxupquote{biography}} field is
displayed side-by-side with the \sphinxcode{\sphinxupquote{name}} field and not given enough space.
Let’s define a custom form view to make viewing and editing teacher records
a better experience:
\sphinxstyleemphasis{academy/views.xml}
\fvset{hllines={, 4, 5, 6, 7, 8, 9, 10, 11, 12, 13, 14, 15, 16, 17,}}%
\begin{sphinxVerbatim}[commandchars=\\\{\}]
    \PYG{n+nt}{\PYGZlt{}field} \PYG{n+na}{name=}\PYG{l+s}{\PYGZdq{}res\PYGZus{}model\PYGZdq{}}\PYG{n+nt}{\PYGZgt{}}academy.teachers\PYG{n+nt}{\PYGZlt{}/field\PYGZgt{}}
  \PYG{n+nt}{\PYGZlt{}/record\PYGZgt{}}

  \PYG{n+nt}{\PYGZlt{}record} \PYG{n+na}{id=}\PYG{l+s}{\PYGZdq{}academy\PYGZus{}teacher\PYGZus{}form\PYGZdq{}} \PYG{n+na}{model=}\PYG{l+s}{\PYGZdq{}ir.ui.view\PYGZdq{}}\PYG{n+nt}{\PYGZgt{}}
    \PYG{n+nt}{\PYGZlt{}field} \PYG{n+na}{name=}\PYG{l+s}{\PYGZdq{}name\PYGZdq{}}\PYG{n+nt}{\PYGZgt{}}Academy teachers: form\PYG{n+nt}{\PYGZlt{}/field\PYGZgt{}}
    \PYG{n+nt}{\PYGZlt{}field} \PYG{n+na}{name=}\PYG{l+s}{\PYGZdq{}model\PYGZdq{}}\PYG{n+nt}{\PYGZgt{}}academy.teachers\PYG{n+nt}{\PYGZlt{}/field\PYGZgt{}}
    \PYG{n+nt}{\PYGZlt{}field} \PYG{n+na}{name=}\PYG{l+s}{\PYGZdq{}arch\PYGZdq{}} \PYG{n+na}{type=}\PYG{l+s}{\PYGZdq{}xml\PYGZdq{}}\PYG{n+nt}{\PYGZgt{}}
      \PYG{n+nt}{\PYGZlt{}form}\PYG{n+nt}{\PYGZgt{}}
        \PYG{n+nt}{\PYGZlt{}sheet}\PYG{n+nt}{\PYGZgt{}}
          \PYG{n+nt}{\PYGZlt{}label} \PYG{n+na}{for=}\PYG{l+s}{\PYGZdq{}name\PYGZdq{}}\PYG{n+nt}{/\PYGZgt{}} \PYG{n+nt}{\PYGZlt{}field} \PYG{n+na}{name=}\PYG{l+s}{\PYGZdq{}name\PYGZdq{}}\PYG{n+nt}{/\PYGZgt{}}
          \PYG{n+nt}{\PYGZlt{}label} \PYG{n+na}{for=}\PYG{l+s}{\PYGZdq{}biography\PYGZdq{}}\PYG{n+nt}{/\PYGZgt{}}
          \PYG{n+nt}{\PYGZlt{}field} \PYG{n+na}{name=}\PYG{l+s}{\PYGZdq{}biography\PYGZdq{}}\PYG{n+nt}{/\PYGZgt{}}
        \PYG{n+nt}{\PYGZlt{}/sheet\PYGZgt{}}
      \PYG{n+nt}{\PYGZlt{}/form\PYGZgt{}}
    \PYG{n+nt}{\PYGZlt{}/field\PYGZgt{}}
  \PYG{n+nt}{\PYGZlt{}/record\PYGZgt{}}

  \PYG{n+nt}{\PYGZlt{}menuitem} \PYG{n+na}{sequence=}\PYG{l+s}{\PYGZdq{}0\PYGZdq{}} \PYG{n+na}{id=}\PYG{l+s}{\PYGZdq{}menu\PYGZus{}academy\PYGZdq{}} \PYG{n+na}{name=}\PYG{l+s}{\PYGZdq{}Academy\PYGZdq{}}\PYG{n+nt}{/\PYGZgt{}}
  \PYG{n+nt}{\PYGZlt{}menuitem} \PYG{n+na}{id=}\PYG{l+s}{\PYGZdq{}menu\PYGZus{}academy\PYGZus{}content\PYGZdq{}} \PYG{n+na}{parent=}\PYG{l+s}{\PYGZdq{}menu\PYGZus{}academy\PYGZdq{}}
            \PYG{n+na}{name=}\PYG{l+s}{\PYGZdq{}Academy Content\PYGZdq{}}\PYG{n+nt}{/\PYGZgt{}}
\end{sphinxVerbatim}


\subsubsection{Relations between models}
\label{\detokenize{howtos/website:relations-between-models}}
We have seen a pair of “basic” fields stored directly in the record. There are
{\hyperref[\detokenize{reference/orm:reference-orm-fields-basic}]{\sphinxcrossref{\DUrole{std,std-ref}{a number of basic fields}}}}. The second
broad categories of fields are {\hyperref[\detokenize{reference/orm:reference-orm-fields-relational}]{\sphinxcrossref{\DUrole{std,std-ref}{relational}}}} and used to link records to one another
(within a model or across models).

For demonstration, let’s create a \sphinxstyleemphasis{courses} model. Each course should have a
\sphinxcode{\sphinxupquote{teacher}} field, linking to a single teacher record, but each teacher can
teach many courses:
\sphinxstyleemphasis{academy/models.py}
\fvset{hllines={, 3, 4, 5, 6, 7, 8,}}%
\begin{sphinxVerbatim}[commandchars=\\\{\}]

    \PYG{n}{name} \PYG{o}{=} \PYG{n}{fields}\PYG{o}{.}\PYG{n}{Char}\PYG{p}{(}\PYG{p}{)}
    \PYG{n}{biography} \PYG{o}{=} \PYG{n}{fields}\PYG{o}{.}\PYG{n}{Html}\PYG{p}{(}\PYG{p}{)}

\PYG{k}{class} \PYG{n+nc}{Courses}\PYG{p}{(}\PYG{n}{models}\PYG{o}{.}\PYG{n}{Model}\PYG{p}{)}\PYG{p}{:}
    \PYG{n}{\PYGZus{}name} \PYG{o}{=} \PYG{l+s+s1}{\PYGZsq{}}\PYG{l+s+s1}{academy.courses}\PYG{l+s+s1}{\PYGZsq{}}

    \PYG{n}{name} \PYG{o}{=} \PYG{n}{fields}\PYG{o}{.}\PYG{n}{Char}\PYG{p}{(}\PYG{p}{)}
    \PYG{n}{teacher\PYGZus{}id} \PYG{o}{=} \PYG{n}{fields}\PYG{o}{.}\PYG{n}{Many2one}\PYG{p}{(}\PYG{l+s+s1}{\PYGZsq{}}\PYG{l+s+s1}{academy.teachers}\PYG{l+s+s1}{\PYGZsq{}}\PYG{p}{,} \PYG{n}{string}\PYG{o}{=}\PYG{l+s+s2}{\PYGZdq{}}\PYG{l+s+s2}{Teacher}\PYG{l+s+s2}{\PYGZdq{}}\PYG{p}{)}
\end{sphinxVerbatim}
\sphinxstyleemphasis{academy/security/ir.model.access.csv}
\fvset{hllines={, 3,}}%
\begin{sphinxVerbatim}[commandchars=\\\{\}]
id,name,model\PYGZus{}id:id,group\PYGZus{}id:id,perm\PYGZus{}read,perm\PYGZus{}write,perm\PYGZus{}create,perm\PYGZus{}unlink
access\PYGZus{}academy\PYGZus{}teachers,access\PYGZus{}academy\PYGZus{}teachers,model\PYGZus{}academy\PYGZus{}teachers,,1,0,0,0
access\PYGZus{}academy\PYGZus{}courses,access\PYGZus{}academy\PYGZus{}courses,model\PYGZus{}academy\PYGZus{}courses,,1,0,0,0
\end{sphinxVerbatim}

let’s also add views so we can see and edit a course’s teacher:
\sphinxstyleemphasis{academy/views.xml}
\fvset{hllines={, 4, 5, 6, 7, 8, 9, 10, 11, 12, 13, 14, 15, 16, 17, 18, 19, 20, 21, 22, 23, 24, 25, 26, 27, 28, 29, 30, 31, 32, 33, 34, 35, 36, 37, 38, 39, 40, 41, 42, 46, 47, 48,}}%
\begin{sphinxVerbatim}[commandchars=\\\{\}]
    \PYG{n+nt}{\PYGZlt{}/field\PYGZgt{}}
  \PYG{n+nt}{\PYGZlt{}/record\PYGZgt{}}

  \PYG{n+nt}{\PYGZlt{}record} \PYG{n+na}{id=}\PYG{l+s}{\PYGZdq{}action\PYGZus{}academy\PYGZus{}courses\PYGZdq{}} \PYG{n+na}{model=}\PYG{l+s}{\PYGZdq{}ir.actions.act\PYGZus{}window\PYGZdq{}}\PYG{n+nt}{\PYGZgt{}}
    \PYG{n+nt}{\PYGZlt{}field} \PYG{n+na}{name=}\PYG{l+s}{\PYGZdq{}name\PYGZdq{}}\PYG{n+nt}{\PYGZgt{}}Academy courses\PYG{n+nt}{\PYGZlt{}/field\PYGZgt{}}
    \PYG{n+nt}{\PYGZlt{}field} \PYG{n+na}{name=}\PYG{l+s}{\PYGZdq{}res\PYGZus{}model\PYGZdq{}}\PYG{n+nt}{\PYGZgt{}}academy.courses\PYG{n+nt}{\PYGZlt{}/field\PYGZgt{}}
  \PYG{n+nt}{\PYGZlt{}/record\PYGZgt{}}
  \PYG{n+nt}{\PYGZlt{}record} \PYG{n+na}{id=}\PYG{l+s}{\PYGZdq{}academy\PYGZus{}course\PYGZus{}search\PYGZdq{}} \PYG{n+na}{model=}\PYG{l+s}{\PYGZdq{}ir.ui.view\PYGZdq{}}\PYG{n+nt}{\PYGZgt{}}
    \PYG{n+nt}{\PYGZlt{}field} \PYG{n+na}{name=}\PYG{l+s}{\PYGZdq{}name\PYGZdq{}}\PYG{n+nt}{\PYGZgt{}}Academy courses: search\PYG{n+nt}{\PYGZlt{}/field\PYGZgt{}}
    \PYG{n+nt}{\PYGZlt{}field} \PYG{n+na}{name=}\PYG{l+s}{\PYGZdq{}model\PYGZdq{}}\PYG{n+nt}{\PYGZgt{}}academy.courses\PYG{n+nt}{\PYGZlt{}/field\PYGZgt{}}
    \PYG{n+nt}{\PYGZlt{}field} \PYG{n+na}{name=}\PYG{l+s}{\PYGZdq{}arch\PYGZdq{}} \PYG{n+na}{type=}\PYG{l+s}{\PYGZdq{}xml\PYGZdq{}}\PYG{n+nt}{\PYGZgt{}}
      \PYG{n+nt}{\PYGZlt{}search}\PYG{n+nt}{\PYGZgt{}}
        \PYG{n+nt}{\PYGZlt{}field} \PYG{n+na}{name=}\PYG{l+s}{\PYGZdq{}name\PYGZdq{}}\PYG{n+nt}{/\PYGZgt{}}
        \PYG{n+nt}{\PYGZlt{}field} \PYG{n+na}{name=}\PYG{l+s}{\PYGZdq{}teacher\PYGZus{}id\PYGZdq{}}\PYG{n+nt}{/\PYGZgt{}}
      \PYG{n+nt}{\PYGZlt{}/search\PYGZgt{}}
    \PYG{n+nt}{\PYGZlt{}/field\PYGZgt{}}
  \PYG{n+nt}{\PYGZlt{}/record\PYGZgt{}}
  \PYG{n+nt}{\PYGZlt{}record} \PYG{n+na}{id=}\PYG{l+s}{\PYGZdq{}academy\PYGZus{}course\PYGZus{}list\PYGZdq{}} \PYG{n+na}{model=}\PYG{l+s}{\PYGZdq{}ir.ui.view\PYGZdq{}}\PYG{n+nt}{\PYGZgt{}}
    \PYG{n+nt}{\PYGZlt{}field} \PYG{n+na}{name=}\PYG{l+s}{\PYGZdq{}name\PYGZdq{}}\PYG{n+nt}{\PYGZgt{}}Academy courses: list\PYG{n+nt}{\PYGZlt{}/field\PYGZgt{}}
    \PYG{n+nt}{\PYGZlt{}field} \PYG{n+na}{name=}\PYG{l+s}{\PYGZdq{}model\PYGZdq{}}\PYG{n+nt}{\PYGZgt{}}academy.courses\PYG{n+nt}{\PYGZlt{}/field\PYGZgt{}}
    \PYG{n+nt}{\PYGZlt{}field} \PYG{n+na}{name=}\PYG{l+s}{\PYGZdq{}arch\PYGZdq{}} \PYG{n+na}{type=}\PYG{l+s}{\PYGZdq{}xml\PYGZdq{}}\PYG{n+nt}{\PYGZgt{}}
      \PYG{n+nt}{\PYGZlt{}tree} \PYG{n+na}{string=}\PYG{l+s}{\PYGZdq{}Courses\PYGZdq{}}\PYG{n+nt}{\PYGZgt{}}
        \PYG{n+nt}{\PYGZlt{}field} \PYG{n+na}{name=}\PYG{l+s}{\PYGZdq{}name\PYGZdq{}}\PYG{n+nt}{/\PYGZgt{}}
        \PYG{n+nt}{\PYGZlt{}field} \PYG{n+na}{name=}\PYG{l+s}{\PYGZdq{}teacher\PYGZus{}id\PYGZdq{}}\PYG{n+nt}{/\PYGZgt{}}
      \PYG{n+nt}{\PYGZlt{}/tree\PYGZgt{}}
    \PYG{n+nt}{\PYGZlt{}/field\PYGZgt{}}
  \PYG{n+nt}{\PYGZlt{}/record\PYGZgt{}}
  \PYG{n+nt}{\PYGZlt{}record} \PYG{n+na}{id=}\PYG{l+s}{\PYGZdq{}academy\PYGZus{}course\PYGZus{}form\PYGZdq{}} \PYG{n+na}{model=}\PYG{l+s}{\PYGZdq{}ir.ui.view\PYGZdq{}}\PYG{n+nt}{\PYGZgt{}}
    \PYG{n+nt}{\PYGZlt{}field} \PYG{n+na}{name=}\PYG{l+s}{\PYGZdq{}name\PYGZdq{}}\PYG{n+nt}{\PYGZgt{}}Academy courses: form\PYG{n+nt}{\PYGZlt{}/field\PYGZgt{}}
    \PYG{n+nt}{\PYGZlt{}field} \PYG{n+na}{name=}\PYG{l+s}{\PYGZdq{}model\PYGZdq{}}\PYG{n+nt}{\PYGZgt{}}academy.courses\PYG{n+nt}{\PYGZlt{}/field\PYGZgt{}}
    \PYG{n+nt}{\PYGZlt{}field} \PYG{n+na}{name=}\PYG{l+s}{\PYGZdq{}arch\PYGZdq{}} \PYG{n+na}{type=}\PYG{l+s}{\PYGZdq{}xml\PYGZdq{}}\PYG{n+nt}{\PYGZgt{}}
      \PYG{n+nt}{\PYGZlt{}form}\PYG{n+nt}{\PYGZgt{}}
        \PYG{n+nt}{\PYGZlt{}sheet}\PYG{n+nt}{\PYGZgt{}}
          \PYG{n+nt}{\PYGZlt{}label} \PYG{n+na}{for=}\PYG{l+s}{\PYGZdq{}name\PYGZdq{}}\PYG{n+nt}{/\PYGZgt{}}
          \PYG{n+nt}{\PYGZlt{}field} \PYG{n+na}{name=}\PYG{l+s}{\PYGZdq{}name\PYGZdq{}}\PYG{n+nt}{/\PYGZgt{}}
          \PYG{n+nt}{\PYGZlt{}label} \PYG{n+na}{for=}\PYG{l+s}{\PYGZdq{}teacher\PYGZus{}id\PYGZdq{}}\PYG{n+nt}{/\PYGZgt{}}
          \PYG{n+nt}{\PYGZlt{}field} \PYG{n+na}{name=}\PYG{l+s}{\PYGZdq{}teacher\PYGZus{}id\PYGZdq{}}\PYG{n+nt}{/\PYGZgt{}}
        \PYG{n+nt}{\PYGZlt{}/sheet\PYGZgt{}}
      \PYG{n+nt}{\PYGZlt{}/form\PYGZgt{}}
    \PYG{n+nt}{\PYGZlt{}/field\PYGZgt{}}
  \PYG{n+nt}{\PYGZlt{}/record\PYGZgt{}}

  \PYG{n+nt}{\PYGZlt{}menuitem} \PYG{n+na}{sequence=}\PYG{l+s}{\PYGZdq{}0\PYGZdq{}} \PYG{n+na}{id=}\PYG{l+s}{\PYGZdq{}menu\PYGZus{}academy\PYGZdq{}} \PYG{n+na}{name=}\PYG{l+s}{\PYGZdq{}Academy\PYGZdq{}}\PYG{n+nt}{/\PYGZgt{}}
  \PYG{n+nt}{\PYGZlt{}menuitem} \PYG{n+na}{id=}\PYG{l+s}{\PYGZdq{}menu\PYGZus{}academy\PYGZus{}content\PYGZdq{}} \PYG{n+na}{parent=}\PYG{l+s}{\PYGZdq{}menu\PYGZus{}academy\PYGZdq{}}
            \PYG{n+na}{name=}\PYG{l+s}{\PYGZdq{}Academy Content\PYGZdq{}}\PYG{n+nt}{/\PYGZgt{}}
  \PYG{n+nt}{\PYGZlt{}menuitem} \PYG{n+na}{id=}\PYG{l+s}{\PYGZdq{}menu\PYGZus{}academy\PYGZus{}content\PYGZus{}courses\PYGZdq{}}
            \PYG{n+na}{parent=}\PYG{l+s}{\PYGZdq{}menu\PYGZus{}academy\PYGZus{}content\PYGZdq{}}
            \PYG{n+na}{action=}\PYG{l+s}{\PYGZdq{}action\PYGZus{}academy\PYGZus{}courses\PYGZdq{}}\PYG{n+nt}{/\PYGZgt{}}
  \PYG{n+nt}{\PYGZlt{}menuitem} \PYG{n+na}{id=}\PYG{l+s}{\PYGZdq{}menu\PYGZus{}academy\PYGZus{}content\PYGZus{}teachers\PYGZdq{}}
            \PYG{n+na}{parent=}\PYG{l+s}{\PYGZdq{}menu\PYGZus{}academy\PYGZus{}content\PYGZdq{}}
            \PYG{n+na}{action=}\PYG{l+s}{\PYGZdq{}action\PYGZus{}academy\PYGZus{}teachers\PYGZdq{}}\PYG{n+nt}{/\PYGZgt{}}
\end{sphinxVerbatim}

It should also be possible to create new courses directly from a teacher’s
page, or to see all the courses they teach, so add
{\hyperref[\detokenize{reference/orm:odoo.fields.One2many}]{\sphinxcrossref{\sphinxcode{\sphinxupquote{the inverse relationship}}}}} to the \sphinxstyleemphasis{teachers}
model:
\sphinxstyleemphasis{academy/models.py}
\fvset{hllines={, 4, 5,}}%
\begin{sphinxVerbatim}[commandchars=\\\{\}]
    \PYG{n}{name} \PYG{o}{=} \PYG{n}{fields}\PYG{o}{.}\PYG{n}{Char}\PYG{p}{(}\PYG{p}{)}
    \PYG{n}{biography} \PYG{o}{=} \PYG{n}{fields}\PYG{o}{.}\PYG{n}{Html}\PYG{p}{(}\PYG{p}{)}

    \PYG{n}{course\PYGZus{}ids} \PYG{o}{=} \PYG{n}{fields}\PYG{o}{.}\PYG{n}{One2many}\PYG{p}{(}\PYG{l+s+s1}{\PYGZsq{}}\PYG{l+s+s1}{academy.courses}\PYG{l+s+s1}{\PYGZsq{}}\PYG{p}{,} \PYG{l+s+s1}{\PYGZsq{}}\PYG{l+s+s1}{teacher\PYGZus{}id}\PYG{l+s+s1}{\PYGZsq{}}\PYG{p}{,} \PYG{n}{string}\PYG{o}{=}\PYG{l+s+s2}{\PYGZdq{}}\PYG{l+s+s2}{Courses}\PYG{l+s+s2}{\PYGZdq{}}\PYG{p}{)}

\PYG{k}{class} \PYG{n+nc}{Courses}\PYG{p}{(}\PYG{n}{models}\PYG{o}{.}\PYG{n}{Model}\PYG{p}{)}\PYG{p}{:}
    \PYG{n}{\PYGZus{}name} \PYG{o}{=} \PYG{l+s+s1}{\PYGZsq{}}\PYG{l+s+s1}{academy.courses}\PYG{l+s+s1}{\PYGZsq{}}

\end{sphinxVerbatim}
\sphinxstyleemphasis{academy/views.xml}
\fvset{hllines={, 4, 7, 8, 9, 10, 11, 12,}}%
\begin{sphinxVerbatim}[commandchars=\\\{\}]
      \PYG{n+nt}{\PYGZlt{}form}\PYG{n+nt}{\PYGZgt{}}
        \PYG{n+nt}{\PYGZlt{}sheet}\PYG{n+nt}{\PYGZgt{}}
          \PYG{n+nt}{\PYGZlt{}label} \PYG{n+na}{for=}\PYG{l+s}{\PYGZdq{}name\PYGZdq{}}\PYG{n+nt}{/\PYGZgt{}} \PYG{n+nt}{\PYGZlt{}field} \PYG{n+na}{name=}\PYG{l+s}{\PYGZdq{}name\PYGZdq{}}\PYG{n+nt}{/\PYGZgt{}}

          \PYG{n+nt}{\PYGZlt{}label} \PYG{n+na}{for=}\PYG{l+s}{\PYGZdq{}biography\PYGZdq{}}\PYG{n+nt}{/\PYGZgt{}}
          \PYG{n+nt}{\PYGZlt{}field} \PYG{n+na}{name=}\PYG{l+s}{\PYGZdq{}biography\PYGZdq{}}\PYG{n+nt}{/\PYGZgt{}}

          \PYG{n+nt}{\PYGZlt{}field} \PYG{n+na}{name=}\PYG{l+s}{\PYGZdq{}course\PYGZus{}ids\PYGZdq{}}\PYG{n+nt}{\PYGZgt{}}
            \PYG{n+nt}{\PYGZlt{}tree} \PYG{n+na}{string=}\PYG{l+s}{\PYGZdq{}Courses\PYGZdq{}} \PYG{n+na}{editable=}\PYG{l+s}{\PYGZdq{}bottom\PYGZdq{}}\PYG{n+nt}{\PYGZgt{}}
              \PYG{n+nt}{\PYGZlt{}field} \PYG{n+na}{name=}\PYG{l+s}{\PYGZdq{}name\PYGZdq{}}\PYG{n+nt}{/\PYGZgt{}}
            \PYG{n+nt}{\PYGZlt{}/tree\PYGZgt{}}
          \PYG{n+nt}{\PYGZlt{}/field\PYGZgt{}}
        \PYG{n+nt}{\PYGZlt{}/sheet\PYGZgt{}}
      \PYG{n+nt}{\PYGZlt{}/form\PYGZgt{}}
    \PYG{n+nt}{\PYGZlt{}/field\PYGZgt{}}
\end{sphinxVerbatim}


\subsubsection{Discussions and notifications}
\label{\detokenize{howtos/website:discussions-and-notifications}}
Odoo provides technical models, which don’t directly fulfill business needs
but which add capabilities to business objects without having to build
them by hand.

One of these is the \sphinxstyleemphasis{Chatter} system, part of Odoo’s email and messaging
system, which can add notifications and discussion threads to any model.
The model simply has to {\hyperref[\detokenize{reference/orm:odoo.models.Model._inherit}]{\sphinxcrossref{\sphinxcode{\sphinxupquote{\_inherit}}}}}
\sphinxcode{\sphinxupquote{mail.thread}}, and add the \sphinxcode{\sphinxupquote{message\_ids}} field to its form view to display
the discussion thread. Discussion threads are per-record.

For our academy, it makes sense to allow discussing courses to handle e.g.
scheduling changes or discussions between teachers and assistants:
\sphinxstyleemphasis{academy/models.py}
\fvset{hllines={, 3,}}%
\begin{sphinxVerbatim}[commandchars=\\\{\}]

\PYG{k}{class} \PYG{n+nc}{Courses}\PYG{p}{(}\PYG{n}{models}\PYG{o}{.}\PYG{n}{Model}\PYG{p}{)}\PYG{p}{:}
    \PYG{n}{\PYGZus{}name} \PYG{o}{=} \PYG{l+s+s1}{\PYGZsq{}}\PYG{l+s+s1}{academy.courses}\PYG{l+s+s1}{\PYGZsq{}}
    \PYG{n}{\PYGZus{}inherit} \PYG{o}{=} \PYG{l+s+s1}{\PYGZsq{}}\PYG{l+s+s1}{mail.thread}\PYG{l+s+s1}{\PYGZsq{}}

    \PYG{n}{name} \PYG{o}{=} \PYG{n}{fields}\PYG{o}{.}\PYG{n}{Char}\PYG{p}{(}\PYG{p}{)}
    \PYG{n}{teacher\PYGZus{}id} \PYG{o}{=} \PYG{n}{fields}\PYG{o}{.}\PYG{n}{Many2one}\PYG{p}{(}\PYG{l+s+s1}{\PYGZsq{}}\PYG{l+s+s1}{academy.teachers}\PYG{l+s+s1}{\PYGZsq{}}\PYG{p}{,} \PYG{n}{string}\PYG{o}{=}\PYG{l+s+s2}{\PYGZdq{}}\PYG{l+s+s2}{Teacher}\PYG{l+s+s2}{\PYGZdq{}}\PYG{p}{)}
\end{sphinxVerbatim}
\sphinxstyleemphasis{academy/views.xml}
\fvset{hllines={, 4, 5, 6, 7,}}%
\begin{sphinxVerbatim}[commandchars=\\\{\}]
          \PYG{n+nt}{\PYGZlt{}label} \PYG{n+na}{for=}\PYG{l+s}{\PYGZdq{}teacher\PYGZus{}id\PYGZdq{}}\PYG{n+nt}{/\PYGZgt{}}
          \PYG{n+nt}{\PYGZlt{}field} \PYG{n+na}{name=}\PYG{l+s}{\PYGZdq{}teacher\PYGZus{}id\PYGZdq{}}\PYG{n+nt}{/\PYGZgt{}}
        \PYG{n+nt}{\PYGZlt{}/sheet\PYGZgt{}}
        \PYG{n+nt}{\PYGZlt{}div} \PYG{n+na}{class=}\PYG{l+s}{\PYGZdq{}oe\PYGZus{}chatter\PYGZdq{}}\PYG{n+nt}{\PYGZgt{}}
          \PYG{n+nt}{\PYGZlt{}field} \PYG{n+na}{name=}\PYG{l+s}{\PYGZdq{}message\PYGZus{}follower\PYGZus{}ids\PYGZdq{}} \PYG{n+na}{widget=}\PYG{l+s}{\PYGZdq{}mail\PYGZus{}followers\PYGZdq{}}\PYG{n+nt}{/\PYGZgt{}}
          \PYG{n+nt}{\PYGZlt{}field} \PYG{n+na}{name=}\PYG{l+s}{\PYGZdq{}message\PYGZus{}ids\PYGZdq{}} \PYG{n+na}{widget=}\PYG{l+s}{\PYGZdq{}mail\PYGZus{}thread\PYGZdq{}}\PYG{n+nt}{/\PYGZgt{}}
        \PYG{n+nt}{\PYGZlt{}/div\PYGZgt{}}
      \PYG{n+nt}{\PYGZlt{}/form\PYGZgt{}}
    \PYG{n+nt}{\PYGZlt{}/field\PYGZgt{}}
  \PYG{n+nt}{\PYGZlt{}/record\PYGZgt{}}
\end{sphinxVerbatim}

At the bottom of each course form, there is now a discussion thread and the
possibility for users of the system to leave messages and follow or unfollow
discussions linked to specific courses.


\subsubsection{Selling courses}
\label{\detokenize{howtos/website:selling-courses}}
Odoo also provides business models which allow using or opting in business
needs more directly. For instance the \sphinxcode{\sphinxupquote{website\_sale}} module sets up an
e-commerce site based on the products in the Odoo system. We can easily make
course subscriptions sellable by making our courses specific kinds of
products.

Rather than the previous classical inheritance, this means replacing our
\sphinxstyleemphasis{course} model by the \sphinxstyleemphasis{product} model, and extending products in-place (to
add anything we need to it).

First of all we need to add a dependency on \sphinxcode{\sphinxupquote{website\_sale}} so we get both
products (via \sphinxcode{\sphinxupquote{sale}}) and the ecommerce interface:
\sphinxstyleemphasis{academy/\_\_manifest\_\_.py}
\fvset{hllines={, 4,}}%
\begin{sphinxVerbatim}[commandchars=\\\{\}]
    \PYG{l+s+s1}{\PYGZsq{}}\PYG{l+s+s1}{version}\PYG{l+s+s1}{\PYGZsq{}}\PYG{p}{:} \PYG{l+s+s1}{\PYGZsq{}}\PYG{l+s+s1}{0.1}\PYG{l+s+s1}{\PYGZsq{}}\PYG{p}{,}

    \PYG{c+c1}{\PYGZsh{} any module necessary for this one to work correctly}
    \PYG{l+s+s1}{\PYGZsq{}}\PYG{l+s+s1}{depends}\PYG{l+s+s1}{\PYGZsq{}}\PYG{p}{:} \PYG{p}{[}\PYG{l+s+s1}{\PYGZsq{}}\PYG{l+s+s1}{website\PYGZus{}sale}\PYG{l+s+s1}{\PYGZsq{}}\PYG{p}{]}\PYG{p}{,}

    \PYG{c+c1}{\PYGZsh{} always loaded}
    \PYG{l+s+s1}{\PYGZsq{}}\PYG{l+s+s1}{data}\PYG{l+s+s1}{\PYGZsq{}}\PYG{p}{:} \PYG{p}{[}
\end{sphinxVerbatim}

restart Odoo, update your module, there is now a \sphinxmenuselection{Shop} section in
the website, listing a number of pre-filled (via demonstration data) products.

The second step is to replace the \sphinxstyleemphasis{courses} model by \sphinxcode{\sphinxupquote{product.template}},
and add a new category of product for courses:
\sphinxstyleemphasis{academy/\_\_manifest\_\_.py}
\fvset{hllines={, 4,}}%
\begin{sphinxVerbatim}[commandchars=\\\{\}]
        \PYG{l+s+s1}{\PYGZsq{}}\PYG{l+s+s1}{security/ir.model.access.csv}\PYG{l+s+s1}{\PYGZsq{}}\PYG{p}{,}
        \PYG{l+s+s1}{\PYGZsq{}}\PYG{l+s+s1}{templates.xml}\PYG{l+s+s1}{\PYGZsq{}}\PYG{p}{,}
        \PYG{l+s+s1}{\PYGZsq{}}\PYG{l+s+s1}{views.xml}\PYG{l+s+s1}{\PYGZsq{}}\PYG{p}{,}
        \PYG{l+s+s1}{\PYGZsq{}}\PYG{l+s+s1}{data.xml}\PYG{l+s+s1}{\PYGZsq{}}\PYG{p}{,}
    \PYG{p}{]}\PYG{p}{,}
    \PYG{c+c1}{\PYGZsh{} only loaded in demonstration mode}
    \PYG{l+s+s1}{\PYGZsq{}}\PYG{l+s+s1}{demo}\PYG{l+s+s1}{\PYGZsq{}}\PYG{p}{:} \PYG{p}{[}
\end{sphinxVerbatim}
\sphinxstyleemphasis{academy/data.xml}
\fvset{hllines={, 1, 2, 3, 4, 5, 6,}}%
\begin{sphinxVerbatim}[commandchars=\\\{\}]
\PYG{n+nt}{\PYGZlt{}odoo}\PYG{n+nt}{\PYGZgt{}}
  \PYG{n+nt}{\PYGZlt{}record} \PYG{n+na}{model=}\PYG{l+s}{\PYGZdq{}product.public.category\PYGZdq{}} \PYG{n+na}{id=}\PYG{l+s}{\PYGZdq{}category\PYGZus{}courses\PYGZdq{}}\PYG{n+nt}{\PYGZgt{}}
    \PYG{n+nt}{\PYGZlt{}field} \PYG{n+na}{name=}\PYG{l+s}{\PYGZdq{}name\PYGZdq{}}\PYG{n+nt}{\PYGZgt{}}Courses\PYG{n+nt}{\PYGZlt{}/field\PYGZgt{}}
    \PYG{n+nt}{\PYGZlt{}field} \PYG{n+na}{name=}\PYG{l+s}{\PYGZdq{}parent\PYGZus{}id\PYGZdq{}} \PYG{n+na}{ref=}\PYG{l+s}{\PYGZdq{}website\PYGZus{}sale.categ\PYGZus{}others\PYGZdq{}}\PYG{n+nt}{/\PYGZgt{}}
  \PYG{n+nt}{\PYGZlt{}/record\PYGZgt{}}
\PYG{n+nt}{\PYGZlt{}/odoo\PYGZgt{}}
\end{sphinxVerbatim}
\sphinxstyleemphasis{academy/demo.xml}
\fvset{hllines={, 4, 5, 6, 7, 8, 9, 10, 11, 12, 13, 14, 15, 16, 17, 18, 19, 20, 21, 22, 23, 24, 25, 26, 27, 28,}}%
\begin{sphinxVerbatim}[commandchars=\\\{\}]
            \PYG{n+nt}{\PYGZlt{}field} \PYG{n+na}{name=}\PYG{l+s}{\PYGZdq{}name\PYGZdq{}}\PYG{n+nt}{\PYGZgt{}}Lester Vaughn\PYG{n+nt}{\PYGZlt{}/field\PYGZgt{}}
        \PYG{n+nt}{\PYGZlt{}/record\PYGZgt{}}

        \PYG{n+nt}{\PYGZlt{}record} \PYG{n+na}{id=}\PYG{l+s}{\PYGZdq{}course0\PYGZdq{}} \PYG{n+na}{model=}\PYG{l+s}{\PYGZdq{}product.template\PYGZdq{}}\PYG{n+nt}{\PYGZgt{}}
            \PYG{n+nt}{\PYGZlt{}field} \PYG{n+na}{name=}\PYG{l+s}{\PYGZdq{}name\PYGZdq{}}\PYG{n+nt}{\PYGZgt{}}Course 0\PYG{n+nt}{\PYGZlt{}/field\PYGZgt{}}
            \PYG{n+nt}{\PYGZlt{}field} \PYG{n+na}{name=}\PYG{l+s}{\PYGZdq{}teacher\PYGZus{}id\PYGZdq{}} \PYG{n+na}{ref=}\PYG{l+s}{\PYGZdq{}padilla\PYGZdq{}}\PYG{n+nt}{/\PYGZgt{}}
            \PYG{n+nt}{\PYGZlt{}field} \PYG{n+na}{name=}\PYG{l+s}{\PYGZdq{}public\PYGZus{}categ\PYGZus{}ids\PYGZdq{}} \PYG{n+na}{eval=}\PYG{l+s}{\PYGZdq{}[(4, ref(\PYGZsq{}academy.category\PYGZus{}courses\PYGZsq{}), False)]\PYGZdq{}}\PYG{n+nt}{/\PYGZgt{}}
            \PYG{n+nt}{\PYGZlt{}field} \PYG{n+na}{name=}\PYG{l+s}{\PYGZdq{}website\PYGZus{}published\PYGZdq{}}\PYG{n+nt}{\PYGZgt{}}True\PYG{n+nt}{\PYGZlt{}/field\PYGZgt{}}
            \PYG{n+nt}{\PYGZlt{}field} \PYG{n+na}{name=}\PYG{l+s}{\PYGZdq{}list\PYGZus{}price\PYGZdq{}} \PYG{n+na}{type=}\PYG{l+s}{\PYGZdq{}float\PYGZdq{}}\PYG{n+nt}{\PYGZgt{}}0\PYG{n+nt}{\PYGZlt{}/field\PYGZgt{}}
            \PYG{n+nt}{\PYGZlt{}field} \PYG{n+na}{name=}\PYG{l+s}{\PYGZdq{}type\PYGZdq{}}\PYG{n+nt}{\PYGZgt{}}service\PYG{n+nt}{\PYGZlt{}/field\PYGZgt{}}
        \PYG{n+nt}{\PYGZlt{}/record\PYGZgt{}}
        \PYG{n+nt}{\PYGZlt{}record} \PYG{n+na}{id=}\PYG{l+s}{\PYGZdq{}course1\PYGZdq{}} \PYG{n+na}{model=}\PYG{l+s}{\PYGZdq{}product.template\PYGZdq{}}\PYG{n+nt}{\PYGZgt{}}
            \PYG{n+nt}{\PYGZlt{}field} \PYG{n+na}{name=}\PYG{l+s}{\PYGZdq{}name\PYGZdq{}}\PYG{n+nt}{\PYGZgt{}}Course 1\PYG{n+nt}{\PYGZlt{}/field\PYGZgt{}}
            \PYG{n+nt}{\PYGZlt{}field} \PYG{n+na}{name=}\PYG{l+s}{\PYGZdq{}teacher\PYGZus{}id\PYGZdq{}} \PYG{n+na}{ref=}\PYG{l+s}{\PYGZdq{}padilla\PYGZdq{}}\PYG{n+nt}{/\PYGZgt{}}
            \PYG{n+nt}{\PYGZlt{}field} \PYG{n+na}{name=}\PYG{l+s}{\PYGZdq{}public\PYGZus{}categ\PYGZus{}ids\PYGZdq{}} \PYG{n+na}{eval=}\PYG{l+s}{\PYGZdq{}[(4, ref(\PYGZsq{}academy.category\PYGZus{}courses\PYGZsq{}), False)]\PYGZdq{}}\PYG{n+nt}{/\PYGZgt{}}
            \PYG{n+nt}{\PYGZlt{}field} \PYG{n+na}{name=}\PYG{l+s}{\PYGZdq{}website\PYGZus{}published\PYGZdq{}}\PYG{n+nt}{\PYGZgt{}}True\PYG{n+nt}{\PYGZlt{}/field\PYGZgt{}}
            \PYG{n+nt}{\PYGZlt{}field} \PYG{n+na}{name=}\PYG{l+s}{\PYGZdq{}list\PYGZus{}price\PYGZdq{}} \PYG{n+na}{type=}\PYG{l+s}{\PYGZdq{}float\PYGZdq{}}\PYG{n+nt}{\PYGZgt{}}0\PYG{n+nt}{\PYGZlt{}/field\PYGZgt{}}
            \PYG{n+nt}{\PYGZlt{}field} \PYG{n+na}{name=}\PYG{l+s}{\PYGZdq{}type\PYGZdq{}}\PYG{n+nt}{\PYGZgt{}}service\PYG{n+nt}{\PYGZlt{}/field\PYGZgt{}}
        \PYG{n+nt}{\PYGZlt{}/record\PYGZgt{}}
        \PYG{n+nt}{\PYGZlt{}record} \PYG{n+na}{id=}\PYG{l+s}{\PYGZdq{}course2\PYGZdq{}} \PYG{n+na}{model=}\PYG{l+s}{\PYGZdq{}product.template\PYGZdq{}}\PYG{n+nt}{\PYGZgt{}}
            \PYG{n+nt}{\PYGZlt{}field} \PYG{n+na}{name=}\PYG{l+s}{\PYGZdq{}name\PYGZdq{}}\PYG{n+nt}{\PYGZgt{}}Course 2\PYG{n+nt}{\PYGZlt{}/field\PYGZgt{}}
            \PYG{n+nt}{\PYGZlt{}field} \PYG{n+na}{name=}\PYG{l+s}{\PYGZdq{}teacher\PYGZus{}id\PYGZdq{}} \PYG{n+na}{ref=}\PYG{l+s}{\PYGZdq{}vaughn\PYGZdq{}}\PYG{n+nt}{/\PYGZgt{}}
            \PYG{n+nt}{\PYGZlt{}field} \PYG{n+na}{name=}\PYG{l+s}{\PYGZdq{}public\PYGZus{}categ\PYGZus{}ids\PYGZdq{}} \PYG{n+na}{eval=}\PYG{l+s}{\PYGZdq{}[(4, ref(\PYGZsq{}academy.category\PYGZus{}courses\PYGZsq{}), False)]\PYGZdq{}}\PYG{n+nt}{/\PYGZgt{}}
            \PYG{n+nt}{\PYGZlt{}field} \PYG{n+na}{name=}\PYG{l+s}{\PYGZdq{}website\PYGZus{}published\PYGZdq{}}\PYG{n+nt}{\PYGZgt{}}True\PYG{n+nt}{\PYGZlt{}/field\PYGZgt{}}
            \PYG{n+nt}{\PYGZlt{}field} \PYG{n+na}{name=}\PYG{l+s}{\PYGZdq{}list\PYGZus{}price\PYGZdq{}} \PYG{n+na}{type=}\PYG{l+s}{\PYGZdq{}float\PYGZdq{}}\PYG{n+nt}{\PYGZgt{}}0\PYG{n+nt}{\PYGZlt{}/field\PYGZgt{}}
            \PYG{n+nt}{\PYGZlt{}field} \PYG{n+na}{name=}\PYG{l+s}{\PYGZdq{}type\PYGZdq{}}\PYG{n+nt}{\PYGZgt{}}service\PYG{n+nt}{\PYGZlt{}/field\PYGZgt{}}
        \PYG{n+nt}{\PYGZlt{}/record\PYGZgt{}}

\PYG{n+nt}{\PYGZlt{}/odoo\PYGZgt{}}
\end{sphinxVerbatim}
\sphinxstyleemphasis{academy/models.py}
\fvset{hllines={, 4, 7,}}%
\begin{sphinxVerbatim}[commandchars=\\\{\}]
    \PYG{n}{name} \PYG{o}{=} \PYG{n}{fields}\PYG{o}{.}\PYG{n}{Char}\PYG{p}{(}\PYG{p}{)}
    \PYG{n}{biography} \PYG{o}{=} \PYG{n}{fields}\PYG{o}{.}\PYG{n}{Html}\PYG{p}{(}\PYG{p}{)}

    \PYG{n}{course\PYGZus{}ids} \PYG{o}{=} \PYG{n}{fields}\PYG{o}{.}\PYG{n}{One2many}\PYG{p}{(}\PYG{l+s+s1}{\PYGZsq{}}\PYG{l+s+s1}{product.template}\PYG{l+s+s1}{\PYGZsq{}}\PYG{p}{,} \PYG{l+s+s1}{\PYGZsq{}}\PYG{l+s+s1}{teacher\PYGZus{}id}\PYG{l+s+s1}{\PYGZsq{}}\PYG{p}{,} \PYG{n}{string}\PYG{o}{=}\PYG{l+s+s2}{\PYGZdq{}}\PYG{l+s+s2}{Courses}\PYG{l+s+s2}{\PYGZdq{}}\PYG{p}{)}

\PYG{k}{class} \PYG{n+nc}{Courses}\PYG{p}{(}\PYG{n}{models}\PYG{o}{.}\PYG{n}{Model}\PYG{p}{)}\PYG{p}{:}
    \PYG{n}{\PYGZus{}inherit} \PYG{o}{=} \PYG{l+s+s1}{\PYGZsq{}}\PYG{l+s+s1}{product.template}\PYG{l+s+s1}{\PYGZsq{}}

    \PYG{n}{teacher\PYGZus{}id} \PYG{o}{=} \PYG{n}{fields}\PYG{o}{.}\PYG{n}{Many2one}\PYG{p}{(}\PYG{l+s+s1}{\PYGZsq{}}\PYG{l+s+s1}{academy.teachers}\PYG{l+s+s1}{\PYGZsq{}}\PYG{p}{,} \PYG{n}{string}\PYG{o}{=}\PYG{l+s+s2}{\PYGZdq{}}\PYG{l+s+s2}{Teacher}\PYG{l+s+s2}{\PYGZdq{}}\PYG{p}{)}
\end{sphinxVerbatim}
\sphinxstyleemphasis{academy/security/ir.model.access.csv}
\fvset{hllines={, ,}}%
\begin{sphinxVerbatim}[commandchars=\\\{\}]
id,name,model\PYGZus{}id:id,group\PYGZus{}id:id,perm\PYGZus{}read,perm\PYGZus{}write,perm\PYGZus{}create,perm\PYGZus{}unlink
access\PYGZus{}academy\PYGZus{}teachers,access\PYGZus{}academy\PYGZus{}teachers,model\PYGZus{}academy\PYGZus{}teachers,,1,0,0,0
\end{sphinxVerbatim}
\sphinxstyleemphasis{academy/views.xml}
\fvset{hllines={, ,}}%
\begin{sphinxVerbatim}[commandchars=\\\{\}]
    \PYG{n+nt}{\PYGZlt{}/field\PYGZgt{}}
  \PYG{n+nt}{\PYGZlt{}/record\PYGZgt{}}

  \PYG{n+nt}{\PYGZlt{}menuitem} \PYG{n+na}{sequence=}\PYG{l+s}{\PYGZdq{}0\PYGZdq{}} \PYG{n+na}{id=}\PYG{l+s}{\PYGZdq{}menu\PYGZus{}academy\PYGZdq{}} \PYG{n+na}{name=}\PYG{l+s}{\PYGZdq{}Academy\PYGZdq{}}\PYG{n+nt}{/\PYGZgt{}}
  \PYG{n+nt}{\PYGZlt{}menuitem} \PYG{n+na}{id=}\PYG{l+s}{\PYGZdq{}menu\PYGZus{}academy\PYGZus{}content\PYGZdq{}} \PYG{n+na}{parent=}\PYG{l+s}{\PYGZdq{}menu\PYGZus{}academy\PYGZdq{}}
            \PYG{n+na}{name=}\PYG{l+s}{\PYGZdq{}Academy Content\PYGZdq{}}\PYG{n+nt}{/\PYGZgt{}}
  \PYG{n+nt}{\PYGZlt{}menuitem} \PYG{n+na}{id=}\PYG{l+s}{\PYGZdq{}menu\PYGZus{}academy\PYGZus{}content\PYGZus{}teachers\PYGZdq{}}
            \PYG{n+na}{parent=}\PYG{l+s}{\PYGZdq{}menu\PYGZus{}academy\PYGZus{}content\PYGZdq{}}
            \PYG{n+na}{action=}\PYG{l+s}{\PYGZdq{}action\PYGZus{}academy\PYGZus{}teachers\PYGZdq{}}\PYG{n+nt}{/\PYGZgt{}}
\end{sphinxVerbatim}

With this installed, a few courses are now available in the \sphinxmenuselection{Shop},
though they may have to be looked for.

\begin{sphinxadmonition}{note}{Note:}\begin{itemize}
\item {} 
to extend a model in-place, it’s {\hyperref[\detokenize{reference/orm:odoo.models.Model._inherit}]{\sphinxcrossref{\sphinxcode{\sphinxupquote{inherited}}}}} without giving it a new
{\hyperref[\detokenize{reference/orm:odoo.models.Model._name}]{\sphinxcrossref{\sphinxcode{\sphinxupquote{\_name}}}}}

\item {} 
\sphinxcode{\sphinxupquote{product.template}} already uses the discussions system, so we can
remove it from our extension model

\item {} 
we’re creating our courses as \sphinxstyleemphasis{published} by default so they can be
seen without having to log in

\end{itemize}
\end{sphinxadmonition}


\subsubsection{Altering existing views}
\label{\detokenize{howtos/website:altering-existing-views}}
So far, we have briefly seen:
\begin{itemize}
\item {} 
the creation of new models

\item {} 
the creation of new views

\item {} 
the creation of new records

\item {} 
the alteration of existing models

\end{itemize}

We’re left with the alteration of existing records and the alteration of
existing views. We’ll do both on the \sphinxmenuselection{Shop} pages.

View alteration is done by creating \sphinxstyleemphasis{extension} views, which are applied on
top of the original view and alter it. These alteration views can be added or
removed without modifying the original, making it easier to try things out and
roll changes back.

Since our courses are free, there is no reason to display their price on the
shop page, so we’re going to alter the view and hide the price if it’s 0. The
first task is finding out which view displays the price, this can be done via
\sphinxmenuselection{Customize \(\rightarrow\) HTML Editor} which lets us read the various
templates involved in rendering a page. Going through a few of them, “Product
item” looks a likely culprit.

Altering view architectures is done in 3 steps:
\begin{enumerate}
\item {} 
Create a new view

\item {} 
Extend the view to modify by setting the new view’s \sphinxcode{\sphinxupquote{inherit\_id}} to the
modified view’s external id

\item {} 
In the architecture, use the \sphinxcode{\sphinxupquote{xpath}} tag to select and alter elements
from the modified view

\end{enumerate}
\sphinxstyleemphasis{academy/templates.xml}
\fvset{hllines={, 4, 5, 6, 7, 8, 9, 10,}}%
\begin{sphinxVerbatim}[commandchars=\\\{\}]
                \PYG{n+nt}{\PYGZlt{}div} \PYG{n+na}{class=}\PYG{l+s}{\PYGZdq{}oe\PYGZus{}structure\PYGZdq{}}\PYG{n+nt}{/\PYGZgt{}}
            \PYG{n+nt}{\PYGZlt{}/t\PYGZgt{}}
        \PYG{n+nt}{\PYGZlt{}/template\PYGZgt{}}

        \PYG{n+nt}{\PYGZlt{}template} \PYG{n+na}{id=}\PYG{l+s}{\PYGZdq{}product\PYGZus{}item\PYGZus{}hide\PYGZus{}no\PYGZus{}price\PYGZdq{}} \PYG{n+na}{inherit\PYGZus{}id=}\PYG{l+s}{\PYGZdq{}website\PYGZus{}sale.products\PYGZus{}item\PYGZdq{}}\PYG{n+nt}{\PYGZgt{}}
            \PYG{n+nt}{\PYGZlt{}xpath} \PYG{n+na}{expr=}\PYG{l+s}{\PYGZdq{}//div[hasclass(\PYGZsq{}product\PYGZus{}price\PYGZsq{})]/b\PYGZdq{}} \PYG{n+na}{position=}\PYG{l+s}{\PYGZdq{}attributes\PYGZdq{}}\PYG{n+nt}{\PYGZgt{}}
                \PYG{n+nt}{\PYGZlt{}attribute} \PYG{n+na}{name=}\PYG{l+s}{\PYGZdq{}t\PYGZhy{}if\PYGZdq{}}\PYG{n+nt}{\PYGZgt{}}product.price \PYG{n+ni}{\PYGZam{}gt;} 0\PYG{n+nt}{\PYGZlt{}/attribute\PYGZgt{}}
            \PYG{n+nt}{\PYGZlt{}/xpath\PYGZgt{}}
        \PYG{n+nt}{\PYGZlt{}/template\PYGZgt{}}

        \PYG{c}{\PYGZlt{}!\PYGZhy{}\PYGZhy{}}\PYG{c}{ \PYGZlt{}template id=\PYGZdq{}object\PYGZdq{}\PYGZgt{} }\PYG{c}{\PYGZhy{}\PYGZhy{}\PYGZgt{}}
        \PYG{c}{\PYGZlt{}!\PYGZhy{}\PYGZhy{}}\PYG{c}{   \PYGZlt{}h1\PYGZgt{}\PYGZlt{}t t}\PYG{c}{\PYGZhy{}}\PYG{c}{esc=\PYGZdq{}object.display\PYGZus{}name\PYGZdq{}/\PYGZgt{}\PYGZlt{}/h1\PYGZgt{} }\PYG{c}{\PYGZhy{}\PYGZhy{}\PYGZgt{}}
        \PYG{c}{\PYGZlt{}!\PYGZhy{}\PYGZhy{}}\PYG{c}{   \PYGZlt{}dl\PYGZgt{} }\PYG{c}{\PYGZhy{}\PYGZhy{}\PYGZgt{}}
\end{sphinxVerbatim}

The second thing we will change is making the product categories sidebar
visible by default: \sphinxmenuselection{Customize \(\rightarrow\) Product Categories} lets
you toggle a tree of product categories (used to filter the main display) on
and off.

This is done via the \sphinxcode{\sphinxupquote{customize\_show}} and \sphinxcode{\sphinxupquote{active}} fields of extension
templates: an extension template (such as the one we’ve just created) can be
\sphinxstyleemphasis{customize\_show=True}. This choice will display the view in the \sphinxmenuselection{Customize}
menu with a check box, allowing administrators to activate or disable them
(and easily customize their website pages).

We simply need to modify the \sphinxstyleemphasis{Product Categories} record and set its default
to \sphinxstyleemphasis{active=”True”}:
\sphinxstyleemphasis{academy/templates.xml}
\fvset{hllines={, 4, 5, 6, 7,}}%
\begin{sphinxVerbatim}[commandchars=\\\{\}]
            \PYG{n+nt}{\PYGZlt{}/xpath\PYGZgt{}}
        \PYG{n+nt}{\PYGZlt{}/template\PYGZgt{}}

        \PYG{n+nt}{\PYGZlt{}record} \PYG{n+na}{id=}\PYG{l+s}{\PYGZdq{}website\PYGZus{}sale.products\PYGZus{}categories\PYGZdq{}} \PYG{n+na}{model=}\PYG{l+s}{\PYGZdq{}ir.ui.view\PYGZdq{}}\PYG{n+nt}{\PYGZgt{}}
            \PYG{n+nt}{\PYGZlt{}field} \PYG{n+na}{name=}\PYG{l+s}{\PYGZdq{}active\PYGZdq{}} \PYG{n+na}{eval=}\PYG{l+s}{\PYGZdq{}True\PYGZdq{}}\PYG{n+nt}{/\PYGZgt{}}
        \PYG{n+nt}{\PYGZlt{}/record\PYGZgt{}}

        \PYG{c}{\PYGZlt{}!\PYGZhy{}\PYGZhy{}}\PYG{c}{ \PYGZlt{}template id=\PYGZdq{}object\PYGZdq{}\PYGZgt{} }\PYG{c}{\PYGZhy{}\PYGZhy{}\PYGZgt{}}
        \PYG{c}{\PYGZlt{}!\PYGZhy{}\PYGZhy{}}\PYG{c}{   \PYGZlt{}h1\PYGZgt{}\PYGZlt{}t t}\PYG{c}{\PYGZhy{}}\PYG{c}{esc=\PYGZdq{}object.display\PYGZus{}name\PYGZdq{}/\PYGZgt{}\PYGZlt{}/h1\PYGZgt{} }\PYG{c}{\PYGZhy{}\PYGZhy{}\PYGZgt{}}
        \PYG{c}{\PYGZlt{}!\PYGZhy{}\PYGZhy{}}\PYG{c}{   \PYGZlt{}dl\PYGZgt{} }\PYG{c}{\PYGZhy{}\PYGZhy{}\PYGZgt{}}
\end{sphinxVerbatim}

With this, the \sphinxstyleemphasis{Product Categories} sidebar will automatically be enabled when
the \sphinxstyleemphasis{Academy} module is installed.
\phantomsection\label{\detokenize{howtos/website:postgres}}\phantomsection\label{\detokenize{howtos/website:converter-pattern}}

\section{Building a Module}
\label{\detokenize{howtos/backend:converter-pattern}}\label{\detokenize{howtos/backend:converter-patterns}}\label{\detokenize{howtos/backend::doc}}\label{\detokenize{howtos/backend:building-a-module}}
\begin{sphinxadmonition}{warning}{Warning:}
This tutorial requires {\hyperref[\detokenize{setup/install:setup-install}]{\sphinxcrossref{\DUrole{std,std-ref}{having installed Odoo}}}}
\end{sphinxadmonition}


\subsection{Start/Stop the Odoo server}
\label{\detokenize{howtos/backend:start-stop-the-odoo-server}}
Odoo uses a client/server architecture in which clients are web browsers
accessing the Odoo server via RPC.

Business logic and extension is generally performed on the server side,
although supporting client features (e.g. new data representation such as
interactive maps) can be added to the client.

In order to start the server, simply invoke the command {\hyperref[\detokenize{reference/cmdline:reference-cmdline}]{\sphinxcrossref{\DUrole{std,std-ref}{odoo-bin}}}} in the shell, adding the full path to the file if
necessary:

\fvset{hllines={, ,}}%
\begin{sphinxVerbatim}[commandchars=\\\{\}]
\PYG{n}{odoo}\PYG{o}{\PYGZhy{}}\PYG{n+nb}{bin}
\end{sphinxVerbatim}

The server is stopped by hitting \sphinxcode{\sphinxupquote{Ctrl-C}} twice from the terminal, or by
killing the corresponding OS process.


\subsection{Build an Odoo module}
\label{\detokenize{howtos/backend:build-an-odoo-module}}
Both server and client extensions are packaged as \sphinxstyleemphasis{modules} which are
optionally loaded in a \sphinxstyleemphasis{database}.

Odoo modules can either add brand new business logic to an Odoo system, or
alter and extend existing business logic: a module can be created to add your
country’s accounting rules to Odoo’s generic accounting support, while the
next module adds support for real-time visualisation of a bus fleet.

Everything in Odoo thus starts and ends with modules.


\subsubsection{Composition of a module}
\label{\detokenize{howtos/backend:composition-of-a-module}}
An Odoo module can contain a number of elements:
\begin{description}
\item[{Business objects}] \leavevmode
Declared as Python classes, these resources are automatically persisted
by Odoo based on their configuration

\item[{Data files}] \leavevmode
XML or CSV files declaring metadata (views or reports), configuration
data (modules parameterization), demonstration data and more

\item[{Web controllers}] \leavevmode
Handle requests from web browsers

\item[{Static web data}] \leavevmode
Images, CSS or javascript files used by the web interface or website

\end{description}


\subsubsection{Module structure}
\label{\detokenize{howtos/backend:module-structure}}
Each module is a directory within a \sphinxstyleemphasis{module directory}. Module directories
are specified by using the {\hyperref[\detokenize{reference/cmdline:cmdoption-odoo-bin-addons-path}]{\sphinxcrossref{\sphinxcode{\sphinxupquote{-{-}addons-path}}}}}
option.

\begin{sphinxadmonition}{tip}{Tip:}
most command-line options can also be set using {\hyperref[\detokenize{reference/cmdline:reference-cmdline-config}]{\sphinxcrossref{\DUrole{std,std-ref}{a configuration
file}}}}
\end{sphinxadmonition}

An Odoo module is declared by its {\hyperref[\detokenize{reference/module:reference-module-manifest}]{\sphinxcrossref{\DUrole{std,std-ref}{manifest}}}}.
See the {\hyperref[\detokenize{reference/module:reference-module-manifest}]{\sphinxcrossref{\DUrole{std,std-ref}{manifest documentation}}}} about it.

A module is also a
\sphinxhref{http://docs.python.org/2/tutorial/modules.html\#packages}{Python package}
with a \sphinxcode{\sphinxupquote{\_\_init\_\_.py}} file, containing import instructions for various Python
files in the module.

For instance, if the module has a single \sphinxcode{\sphinxupquote{mymodule.py}} file \sphinxcode{\sphinxupquote{\_\_init\_\_.py}}
might contain:

\fvset{hllines={, ,}}%
\begin{sphinxVerbatim}[commandchars=\\\{\}]
\PYG{k+kn}{from} \PYG{n+nn}{.} \PYG{k}{import} \PYG{n}{mymodule}
\end{sphinxVerbatim}

Odoo provides a mechanism to help set up a new module, {\hyperref[\detokenize{reference/cmdline:reference-cmdline-server}]{\sphinxcrossref{\DUrole{std,std-ref}{odoo-bin}}}} has a subcommand {\hyperref[\detokenize{reference/cmdline:reference-cmdline-scaffold}]{\sphinxcrossref{\DUrole{std,std-ref}{scaffold}}}} to create an empty module:

\fvset{hllines={, ,}}%
\begin{sphinxVerbatim}[commandchars=\\\{\}]
\PYG{g+gp}{\PYGZdl{}} odoo\PYGZhy{}bin scaffold \PYGZlt{}module name\PYGZgt{} \PYGZlt{}where to put it\PYGZgt{}
\end{sphinxVerbatim}

The command creates a subdirectory for your module, and automatically creates a
bunch of standard files for a module. Most of them simply contain commented code
or XML. The usage of most of those files will be explained along this tutorial.

\begin{sphinxadmonition}{note}
Module creation

Use the command line above to  create an empty module Open Academy, and
install it in Odoo.
\begin{enumerate}
\item {} 
Invoke the command \sphinxcode{\sphinxupquote{odoo-bin scaffold openacademy addons}}.

\item {} 
Adapt the manifest file to your module.

\item {} 
Don’t bother about the other files.

\end{enumerate}
\sphinxstyleemphasis{openacademy/\_\_manifest\_\_.py}
\fvset{hllines={, 1, 2, 3, 4, 5, 6, 7, 8, 9, 10, 11, 12, 13, 14, 15, 16, 17, 18, 19, 20, 21, 22, 23, 24, 25, 26, 27, 28, 29, 30, 31, 32, 33, 34, 35,}}%
\begin{sphinxVerbatim}[commandchars=\\\{\}]
\PYG{c+c1}{\PYGZsh{} \PYGZhy{}*\PYGZhy{} coding: utf\PYGZhy{}8 \PYGZhy{}*\PYGZhy{}}
\PYG{p}{\PYGZob{}}
    \PYG{l+s+s1}{\PYGZsq{}}\PYG{l+s+s1}{name}\PYG{l+s+s1}{\PYGZsq{}}\PYG{p}{:} \PYG{l+s+s2}{\PYGZdq{}}\PYG{l+s+s2}{Open Academy}\PYG{l+s+s2}{\PYGZdq{}}\PYG{p}{,}

    \PYG{l+s+s1}{\PYGZsq{}}\PYG{l+s+s1}{summary}\PYG{l+s+s1}{\PYGZsq{}}\PYG{p}{:} \PYG{l+s+s2}{\PYGZdq{}\PYGZdq{}\PYGZdq{}}\PYG{l+s+s2}{Manage trainings}\PYG{l+s+s2}{\PYGZdq{}\PYGZdq{}\PYGZdq{}}\PYG{p}{,}

    \PYG{l+s+s1}{\PYGZsq{}}\PYG{l+s+s1}{description}\PYG{l+s+s1}{\PYGZsq{}}\PYG{p}{:} \PYG{l+s+s2}{\PYGZdq{}\PYGZdq{}\PYGZdq{}}
\PYG{l+s+s2}{        Open Academy module for managing trainings:}
\PYG{l+s+s2}{            \PYGZhy{} training courses}
\PYG{l+s+s2}{            \PYGZhy{} training sessions}
\PYG{l+s+s2}{            \PYGZhy{} attendees registration}
\PYG{l+s+s2}{    }\PYG{l+s+s2}{\PYGZdq{}\PYGZdq{}\PYGZdq{}}\PYG{p}{,}

    \PYG{l+s+s1}{\PYGZsq{}}\PYG{l+s+s1}{author}\PYG{l+s+s1}{\PYGZsq{}}\PYG{p}{:} \PYG{l+s+s2}{\PYGZdq{}}\PYG{l+s+s2}{My Company}\PYG{l+s+s2}{\PYGZdq{}}\PYG{p}{,}
    \PYG{l+s+s1}{\PYGZsq{}}\PYG{l+s+s1}{website}\PYG{l+s+s1}{\PYGZsq{}}\PYG{p}{:} \PYG{l+s+s2}{\PYGZdq{}}\PYG{l+s+s2}{http://www.yourcompany.com}\PYG{l+s+s2}{\PYGZdq{}}\PYG{p}{,}

    \PYG{c+c1}{\PYGZsh{} Categories can be used to filter modules in modules listing}
    \PYG{c+c1}{\PYGZsh{} Check https://github.com/odoo/odoo/blob/master/odoo/addons/base/module/module\PYGZus{}data.xml}
    \PYG{c+c1}{\PYGZsh{} for the full list}
    \PYG{l+s+s1}{\PYGZsq{}}\PYG{l+s+s1}{category}\PYG{l+s+s1}{\PYGZsq{}}\PYG{p}{:} \PYG{l+s+s1}{\PYGZsq{}}\PYG{l+s+s1}{Test}\PYG{l+s+s1}{\PYGZsq{}}\PYG{p}{,}
    \PYG{l+s+s1}{\PYGZsq{}}\PYG{l+s+s1}{version}\PYG{l+s+s1}{\PYGZsq{}}\PYG{p}{:} \PYG{l+s+s1}{\PYGZsq{}}\PYG{l+s+s1}{0.1}\PYG{l+s+s1}{\PYGZsq{}}\PYG{p}{,}

    \PYG{c+c1}{\PYGZsh{} any module necessary for this one to work correctly}
    \PYG{l+s+s1}{\PYGZsq{}}\PYG{l+s+s1}{depends}\PYG{l+s+s1}{\PYGZsq{}}\PYG{p}{:} \PYG{p}{[}\PYG{l+s+s1}{\PYGZsq{}}\PYG{l+s+s1}{base}\PYG{l+s+s1}{\PYGZsq{}}\PYG{p}{]}\PYG{p}{,}

    \PYG{c+c1}{\PYGZsh{} always loaded}
    \PYG{l+s+s1}{\PYGZsq{}}\PYG{l+s+s1}{data}\PYG{l+s+s1}{\PYGZsq{}}\PYG{p}{:} \PYG{p}{[}
        \PYG{c+c1}{\PYGZsh{} \PYGZsq{}security/ir.model.access.csv\PYGZsq{},}
        \PYG{l+s+s1}{\PYGZsq{}}\PYG{l+s+s1}{templates.xml}\PYG{l+s+s1}{\PYGZsq{}}\PYG{p}{,}
    \PYG{p}{]}\PYG{p}{,}
    \PYG{c+c1}{\PYGZsh{} only loaded in demonstration mode}
    \PYG{l+s+s1}{\PYGZsq{}}\PYG{l+s+s1}{demo}\PYG{l+s+s1}{\PYGZsq{}}\PYG{p}{:} \PYG{p}{[}
        \PYG{l+s+s1}{\PYGZsq{}}\PYG{l+s+s1}{demo.xml}\PYG{l+s+s1}{\PYGZsq{}}\PYG{p}{,}
    \PYG{p}{]}\PYG{p}{,}
\PYG{p}{\PYGZcb{}}
\end{sphinxVerbatim}
\sphinxstyleemphasis{openacademy/\_\_init\_\_.py}
\fvset{hllines={, 1, 2, 3,}}%
\begin{sphinxVerbatim}[commandchars=\\\{\}]
\PYG{c+c1}{\PYGZsh{} \PYGZhy{}*\PYGZhy{} coding: utf\PYGZhy{}8 \PYGZhy{}*\PYGZhy{}}
\PYG{k+kn}{from} \PYG{n+nn}{.} \PYG{k+kn}{import} \PYG{n}{controllers}
\PYG{k+kn}{from} \PYG{n+nn}{.} \PYG{k+kn}{import} \PYG{n}{models}
\end{sphinxVerbatim}
\sphinxstyleemphasis{openacademy/controllers.py}
\fvset{hllines={, 1, 2, 3, 4, 5, 6, 7, 8, 9, 10, 11, 12, 13, 14, 15, 16, 17, 18, 19, 20,}}%
\begin{sphinxVerbatim}[commandchars=\\\{\}]
\PYG{c+c1}{\PYGZsh{} \PYGZhy{}*\PYGZhy{} coding: utf\PYGZhy{}8 \PYGZhy{}*\PYGZhy{}}
\PYG{k+kn}{from} \PYG{n+nn}{odoo} \PYG{k+kn}{import} \PYG{n}{http}

\PYG{c+c1}{\PYGZsh{} class Openacademy(http.Controller):}
\PYG{c+c1}{\PYGZsh{}     @http.route(\PYGZsq{}/openacademy/openacademy/\PYGZsq{}, auth=\PYGZsq{}public\PYGZsq{})}
\PYG{c+c1}{\PYGZsh{}     def index(self, **kw):}
\PYG{c+c1}{\PYGZsh{}         return \PYGZdq{}Hello, world\PYGZdq{}}

\PYG{c+c1}{\PYGZsh{}     @http.route(\PYGZsq{}/openacademy/openacademy/objects/\PYGZsq{}, auth=\PYGZsq{}public\PYGZsq{})}
\PYG{c+c1}{\PYGZsh{}     def list(self, **kw):}
\PYG{c+c1}{\PYGZsh{}         return http.request.render(\PYGZsq{}openacademy.listing\PYGZsq{}, \PYGZob{}}
\PYG{c+c1}{\PYGZsh{}             \PYGZsq{}root\PYGZsq{}: \PYGZsq{}/openacademy/openacademy\PYGZsq{},}
\PYG{c+c1}{\PYGZsh{}             \PYGZsq{}objects\PYGZsq{}: http.request.env[\PYGZsq{}openacademy.openacademy\PYGZsq{}].search([]),}
\PYG{c+c1}{\PYGZsh{}         \PYGZcb{})}

\PYG{c+c1}{\PYGZsh{}     @http.route(\PYGZsq{}/openacademy/openacademy/objects/\PYGZlt{}model(\PYGZdq{}openacademy.openacademy\PYGZdq{}):obj\PYGZgt{}/\PYGZsq{}, auth=\PYGZsq{}public\PYGZsq{})}
\PYG{c+c1}{\PYGZsh{}     def object(self, obj, **kw):}
\PYG{c+c1}{\PYGZsh{}         return http.request.render(\PYGZsq{}openacademy.object\PYGZsq{}, \PYGZob{}}
\PYG{c+c1}{\PYGZsh{}             \PYGZsq{}object\PYGZsq{}: obj}
\PYG{c+c1}{\PYGZsh{}         \PYGZcb{})}
\end{sphinxVerbatim}
\sphinxstyleemphasis{openacademy/demo.xml}
\fvset{hllines={, 1, 2, 3, 4, 5, 6, 7, 8, 9, 10, 11, 12, 13, 14, 15, 16, 17, 18, 19, 20, 21, 22, 23, 24, 25,}}%
\begin{sphinxVerbatim}[commandchars=\\\{\}]
\PYG{n+nt}{\PYGZlt{}odoo}\PYG{n+nt}{\PYGZgt{}}

        \PYG{c}{\PYGZlt{}!\PYGZhy{}\PYGZhy{}}\PYG{c}{  }\PYG{c}{\PYGZhy{}\PYGZhy{}\PYGZgt{}}
        \PYG{c}{\PYGZlt{}!\PYGZhy{}\PYGZhy{}}\PYG{c}{   \PYGZlt{}record id=\PYGZdq{}object0\PYGZdq{} model=\PYGZdq{}openacademy.openacademy\PYGZdq{}\PYGZgt{} }\PYG{c}{\PYGZhy{}\PYGZhy{}\PYGZgt{}}
        \PYG{c}{\PYGZlt{}!\PYGZhy{}\PYGZhy{}}\PYG{c}{     \PYGZlt{}field name=\PYGZdq{}name\PYGZdq{}\PYGZgt{}Object 0\PYGZlt{}/field\PYGZgt{} }\PYG{c}{\PYGZhy{}\PYGZhy{}\PYGZgt{}}
        \PYG{c}{\PYGZlt{}!\PYGZhy{}\PYGZhy{}}\PYG{c}{   \PYGZlt{}/record\PYGZgt{} }\PYG{c}{\PYGZhy{}\PYGZhy{}\PYGZgt{}}
        \PYG{c}{\PYGZlt{}!\PYGZhy{}\PYGZhy{}}\PYG{c}{  }\PYG{c}{\PYGZhy{}\PYGZhy{}\PYGZgt{}}
        \PYG{c}{\PYGZlt{}!\PYGZhy{}\PYGZhy{}}\PYG{c}{   \PYGZlt{}record id=\PYGZdq{}object1\PYGZdq{} model=\PYGZdq{}openacademy.openacademy\PYGZdq{}\PYGZgt{} }\PYG{c}{\PYGZhy{}\PYGZhy{}\PYGZgt{}}
        \PYG{c}{\PYGZlt{}!\PYGZhy{}\PYGZhy{}}\PYG{c}{     \PYGZlt{}field name=\PYGZdq{}name\PYGZdq{}\PYGZgt{}Object 1\PYGZlt{}/field\PYGZgt{} }\PYG{c}{\PYGZhy{}\PYGZhy{}\PYGZgt{}}
        \PYG{c}{\PYGZlt{}!\PYGZhy{}\PYGZhy{}}\PYG{c}{   \PYGZlt{}/record\PYGZgt{} }\PYG{c}{\PYGZhy{}\PYGZhy{}\PYGZgt{}}
        \PYG{c}{\PYGZlt{}!\PYGZhy{}\PYGZhy{}}\PYG{c}{  }\PYG{c}{\PYGZhy{}\PYGZhy{}\PYGZgt{}}
        \PYG{c}{\PYGZlt{}!\PYGZhy{}\PYGZhy{}}\PYG{c}{   \PYGZlt{}record id=\PYGZdq{}object2\PYGZdq{} model=\PYGZdq{}openacademy.openacademy\PYGZdq{}\PYGZgt{} }\PYG{c}{\PYGZhy{}\PYGZhy{}\PYGZgt{}}
        \PYG{c}{\PYGZlt{}!\PYGZhy{}\PYGZhy{}}\PYG{c}{     \PYGZlt{}field name=\PYGZdq{}name\PYGZdq{}\PYGZgt{}Object 2\PYGZlt{}/field\PYGZgt{} }\PYG{c}{\PYGZhy{}\PYGZhy{}\PYGZgt{}}
        \PYG{c}{\PYGZlt{}!\PYGZhy{}\PYGZhy{}}\PYG{c}{   \PYGZlt{}/record\PYGZgt{} }\PYG{c}{\PYGZhy{}\PYGZhy{}\PYGZgt{}}
        \PYG{c}{\PYGZlt{}!\PYGZhy{}\PYGZhy{}}\PYG{c}{  }\PYG{c}{\PYGZhy{}\PYGZhy{}\PYGZgt{}}
        \PYG{c}{\PYGZlt{}!\PYGZhy{}\PYGZhy{}}\PYG{c}{   \PYGZlt{}record id=\PYGZdq{}object3\PYGZdq{} model=\PYGZdq{}openacademy.openacademy\PYGZdq{}\PYGZgt{} }\PYG{c}{\PYGZhy{}\PYGZhy{}\PYGZgt{}}
        \PYG{c}{\PYGZlt{}!\PYGZhy{}\PYGZhy{}}\PYG{c}{     \PYGZlt{}field name=\PYGZdq{}name\PYGZdq{}\PYGZgt{}Object 3\PYGZlt{}/field\PYGZgt{} }\PYG{c}{\PYGZhy{}\PYGZhy{}\PYGZgt{}}
        \PYG{c}{\PYGZlt{}!\PYGZhy{}\PYGZhy{}}\PYG{c}{   \PYGZlt{}/record\PYGZgt{} }\PYG{c}{\PYGZhy{}\PYGZhy{}\PYGZgt{}}
        \PYG{c}{\PYGZlt{}!\PYGZhy{}\PYGZhy{}}\PYG{c}{  }\PYG{c}{\PYGZhy{}\PYGZhy{}\PYGZgt{}}
        \PYG{c}{\PYGZlt{}!\PYGZhy{}\PYGZhy{}}\PYG{c}{   \PYGZlt{}record id=\PYGZdq{}object4\PYGZdq{} model=\PYGZdq{}openacademy.openacademy\PYGZdq{}\PYGZgt{} }\PYG{c}{\PYGZhy{}\PYGZhy{}\PYGZgt{}}
        \PYG{c}{\PYGZlt{}!\PYGZhy{}\PYGZhy{}}\PYG{c}{     \PYGZlt{}field name=\PYGZdq{}name\PYGZdq{}\PYGZgt{}Object 4\PYGZlt{}/field\PYGZgt{} }\PYG{c}{\PYGZhy{}\PYGZhy{}\PYGZgt{}}
        \PYG{c}{\PYGZlt{}!\PYGZhy{}\PYGZhy{}}\PYG{c}{   \PYGZlt{}/record\PYGZgt{} }\PYG{c}{\PYGZhy{}\PYGZhy{}\PYGZgt{}}
        \PYG{c}{\PYGZlt{}!\PYGZhy{}\PYGZhy{}}\PYG{c}{  }\PYG{c}{\PYGZhy{}\PYGZhy{}\PYGZgt{}}

\PYG{n+nt}{\PYGZlt{}/odoo\PYGZgt{}}
\end{sphinxVerbatim}
\sphinxstyleemphasis{openacademy/models.py}
\fvset{hllines={, 1, 2, 3, 4, 5, 6, 7, 8,}}%
\begin{sphinxVerbatim}[commandchars=\\\{\}]
\PYG{c+c1}{\PYGZsh{} \PYGZhy{}*\PYGZhy{} coding: utf\PYGZhy{}8 \PYGZhy{}*\PYGZhy{}}

\PYG{k+kn}{from} \PYG{n+nn}{odoo} \PYG{k+kn}{import} \PYG{n}{models}\PYG{p}{,} \PYG{n}{fields}\PYG{p}{,} \PYG{n}{api}

\PYG{c+c1}{\PYGZsh{} class openacademy(models.Model):}
\PYG{c+c1}{\PYGZsh{}     \PYGZus{}name = \PYGZsq{}openacademy.openacademy\PYGZsq{}}

\PYG{c+c1}{\PYGZsh{}     name = fields.Char()}
\end{sphinxVerbatim}
\sphinxstyleemphasis{openacademy/security/ir.model.access.csv}
\fvset{hllines={, 1, 2,}}%
\begin{sphinxVerbatim}[commandchars=\\\{\}]
id,name,model\PYGZus{}id/id,group\PYGZus{}id/id,perm\PYGZus{}read,perm\PYGZus{}write,perm\PYGZus{}create,perm\PYGZus{}unlink
access\PYGZus{}openacademy\PYGZus{}openacademy,openacademy.openacademy,model\PYGZus{}openacademy\PYGZus{}openacademy,,1,0,0,0
\end{sphinxVerbatim}
\sphinxstyleemphasis{openacademy/templates.xml}
\fvset{hllines={, 1, 2, 3, 4, 5, 6, 7, 8, 9, 10, 11, 12, 13, 14, 15, 16, 17, 18, 19, 20, 21, 22,}}%
\begin{sphinxVerbatim}[commandchars=\\\{\}]
\PYG{n+nt}{\PYGZlt{}odoo}\PYG{n+nt}{\PYGZgt{}}

        \PYG{c}{\PYGZlt{}!\PYGZhy{}\PYGZhy{}}\PYG{c}{ \PYGZlt{}template id=\PYGZdq{}listing\PYGZdq{}\PYGZgt{} }\PYG{c}{\PYGZhy{}\PYGZhy{}\PYGZgt{}}
        \PYG{c}{\PYGZlt{}!\PYGZhy{}\PYGZhy{}}\PYG{c}{   \PYGZlt{}ul\PYGZgt{} }\PYG{c}{\PYGZhy{}\PYGZhy{}\PYGZgt{}}
        \PYG{c}{\PYGZlt{}!\PYGZhy{}\PYGZhy{}}\PYG{c}{     \PYGZlt{}li t}\PYG{c}{\PYGZhy{}}\PYG{c}{foreach=\PYGZdq{}objects\PYGZdq{} t}\PYG{c}{\PYGZhy{}}\PYG{c}{as=\PYGZdq{}object\PYGZdq{}\PYGZgt{} }\PYG{c}{\PYGZhy{}\PYGZhy{}\PYGZgt{}}
        \PYG{c}{\PYGZlt{}!\PYGZhy{}\PYGZhy{}}\PYG{c}{       \PYGZlt{}a t}\PYG{c}{\PYGZhy{}}\PYG{c}{attf}\PYG{c}{\PYGZhy{}}\PYG{c}{href=\PYGZdq{}}\PYG{c+cp}{\PYGZob{}\PYGZob{}} \PYG{n+nv}{root} \PYG{c+cp}{\PYGZcb{}\PYGZcb{}}\PYG{c}{/objects/}\PYG{c+cp}{\PYGZob{}\PYGZob{}} \PYG{n+nv}{object}\PYG{n+nv}{.id} \PYG{c+cp}{\PYGZcb{}\PYGZcb{}}\PYG{c}{\PYGZdq{}\PYGZgt{} }\PYG{c}{\PYGZhy{}\PYGZhy{}\PYGZgt{}}
        \PYG{c}{\PYGZlt{}!\PYGZhy{}\PYGZhy{}}\PYG{c}{         \PYGZlt{}t t}\PYG{c}{\PYGZhy{}}\PYG{c}{esc=\PYGZdq{}object.display\PYGZus{}name\PYGZdq{}/\PYGZgt{} }\PYG{c}{\PYGZhy{}\PYGZhy{}\PYGZgt{}}
        \PYG{c}{\PYGZlt{}!\PYGZhy{}\PYGZhy{}}\PYG{c}{       \PYGZlt{}/a\PYGZgt{} }\PYG{c}{\PYGZhy{}\PYGZhy{}\PYGZgt{}}
        \PYG{c}{\PYGZlt{}!\PYGZhy{}\PYGZhy{}}\PYG{c}{     \PYGZlt{}/li\PYGZgt{} }\PYG{c}{\PYGZhy{}\PYGZhy{}\PYGZgt{}}
        \PYG{c}{\PYGZlt{}!\PYGZhy{}\PYGZhy{}}\PYG{c}{   \PYGZlt{}/ul\PYGZgt{} }\PYG{c}{\PYGZhy{}\PYGZhy{}\PYGZgt{}}
        \PYG{c}{\PYGZlt{}!\PYGZhy{}\PYGZhy{}}\PYG{c}{ \PYGZlt{}/template\PYGZgt{} }\PYG{c}{\PYGZhy{}\PYGZhy{}\PYGZgt{}}
        \PYG{c}{\PYGZlt{}!\PYGZhy{}\PYGZhy{}}\PYG{c}{ \PYGZlt{}template id=\PYGZdq{}object\PYGZdq{}\PYGZgt{} }\PYG{c}{\PYGZhy{}\PYGZhy{}\PYGZgt{}}
        \PYG{c}{\PYGZlt{}!\PYGZhy{}\PYGZhy{}}\PYG{c}{   \PYGZlt{}h1\PYGZgt{}\PYGZlt{}t t}\PYG{c}{\PYGZhy{}}\PYG{c}{esc=\PYGZdq{}object.display\PYGZus{}name\PYGZdq{}/\PYGZgt{}\PYGZlt{}/h1\PYGZgt{} }\PYG{c}{\PYGZhy{}\PYGZhy{}\PYGZgt{}}
        \PYG{c}{\PYGZlt{}!\PYGZhy{}\PYGZhy{}}\PYG{c}{   \PYGZlt{}dl\PYGZgt{} }\PYG{c}{\PYGZhy{}\PYGZhy{}\PYGZgt{}}
        \PYG{c}{\PYGZlt{}!\PYGZhy{}\PYGZhy{}}\PYG{c}{     \PYGZlt{}t t}\PYG{c}{\PYGZhy{}}\PYG{c}{foreach=\PYGZdq{}object.\PYGZus{}fields\PYGZdq{} t}\PYG{c}{\PYGZhy{}}\PYG{c}{as=\PYGZdq{}field\PYGZdq{}\PYGZgt{} }\PYG{c}{\PYGZhy{}\PYGZhy{}\PYGZgt{}}
        \PYG{c}{\PYGZlt{}!\PYGZhy{}\PYGZhy{}}\PYG{c}{       \PYGZlt{}dt\PYGZgt{}\PYGZlt{}t t}\PYG{c}{\PYGZhy{}}\PYG{c}{esc=\PYGZdq{}field\PYGZdq{}/\PYGZgt{}\PYGZlt{}/dt\PYGZgt{} }\PYG{c}{\PYGZhy{}\PYGZhy{}\PYGZgt{}}
        \PYG{c}{\PYGZlt{}!\PYGZhy{}\PYGZhy{}}\PYG{c}{       \PYGZlt{}dd\PYGZgt{}\PYGZlt{}t t}\PYG{c}{\PYGZhy{}}\PYG{c}{esc=\PYGZdq{}object[field]\PYGZdq{}/\PYGZgt{}\PYGZlt{}/dd\PYGZgt{} }\PYG{c}{\PYGZhy{}\PYGZhy{}\PYGZgt{}}
        \PYG{c}{\PYGZlt{}!\PYGZhy{}\PYGZhy{}}\PYG{c}{     \PYGZlt{}/t\PYGZgt{} }\PYG{c}{\PYGZhy{}\PYGZhy{}\PYGZgt{}}
        \PYG{c}{\PYGZlt{}!\PYGZhy{}\PYGZhy{}}\PYG{c}{   \PYGZlt{}/dl\PYGZgt{} }\PYG{c}{\PYGZhy{}\PYGZhy{}\PYGZgt{}}
        \PYG{c}{\PYGZlt{}!\PYGZhy{}\PYGZhy{}}\PYG{c}{ \PYGZlt{}/template\PYGZgt{} }\PYG{c}{\PYGZhy{}\PYGZhy{}\PYGZgt{}}

\PYG{n+nt}{\PYGZlt{}/odoo\PYGZgt{}}
\end{sphinxVerbatim}
\end{sphinxadmonition}


\subsubsection{Object-Relational Mapping}
\label{\detokenize{howtos/backend:object-relational-mapping}}
A key component of Odoo is the \sphinxstyleabbreviation{ORM} (Object-Relational Mapping) layer.
This layer avoids having to write most \sphinxstyleabbreviation{SQL} (Structured Query Language)
by hand and provides extensibility and security services%
\begin{footnote}[2]\sphinxAtStartFootnote
writing raw SQL queries is possible, but requires care as it
bypasses all Odoo authentication and security mechanisms.
%
\end{footnote}.

Business objects are declared as Python classes extending
{\hyperref[\detokenize{reference/orm:odoo.models.Model}]{\sphinxcrossref{\sphinxcode{\sphinxupquote{Model}}}}} which integrates them into the automated
persistence system.

Models can be configured by setting a number of attributes at their
definition. The most important attribute is
{\hyperref[\detokenize{reference/orm:odoo.models.Model._name}]{\sphinxcrossref{\sphinxcode{\sphinxupquote{\_name}}}}} which is required and defines the name for
the model in the Odoo system. Here is a minimally complete definition of a
model:

\fvset{hllines={, ,}}%
\begin{sphinxVerbatim}[commandchars=\\\{\}]
\PYG{k+kn}{from} \PYG{n+nn}{odoo} \PYG{k}{import} \PYG{n}{models}
\PYG{k}{class} \PYG{n+nc}{MinimalModel}\PYG{p}{(}\PYG{n}{models}\PYG{o}{.}\PYG{n}{Model}\PYG{p}{)}\PYG{p}{:}
    \PYG{n}{\PYGZus{}name} \PYG{o}{=} \PYG{l+s+s1}{\PYGZsq{}}\PYG{l+s+s1}{test.model}\PYG{l+s+s1}{\PYGZsq{}}
\end{sphinxVerbatim}


\subsubsection{Model fields}
\label{\detokenize{howtos/backend:model-fields}}
Fields are used to define what the model can store and where. Fields are
defined as attributes on the model class:

\fvset{hllines={, ,}}%
\begin{sphinxVerbatim}[commandchars=\\\{\}]
\PYG{k+kn}{from} \PYG{n+nn}{odoo} \PYG{k}{import} \PYG{n}{models}\PYG{p}{,} \PYG{n}{fields}

\PYG{k}{class} \PYG{n+nc}{LessMinimalModel}\PYG{p}{(}\PYG{n}{models}\PYG{o}{.}\PYG{n}{Model}\PYG{p}{)}\PYG{p}{:}
    \PYG{n}{\PYGZus{}name} \PYG{o}{=} \PYG{l+s+s1}{\PYGZsq{}}\PYG{l+s+s1}{test.model2}\PYG{l+s+s1}{\PYGZsq{}}

    \PYG{n}{name} \PYG{o}{=} \PYG{n}{fields}\PYG{o}{.}\PYG{n}{Char}\PYG{p}{(}\PYG{p}{)}
\end{sphinxVerbatim}


\paragraph{Common Attributes}
\label{\detokenize{howtos/backend:common-attributes}}
Much like the model itself, its fields can be configured, by passing
configuration attributes as parameters:

\fvset{hllines={, ,}}%
\begin{sphinxVerbatim}[commandchars=\\\{\}]
\PYG{n}{name} \PYG{o}{=} \PYG{n}{field}\PYG{o}{.}\PYG{n}{Char}\PYG{p}{(}\PYG{n}{required}\PYG{o}{=}\PYG{k+kc}{True}\PYG{p}{)}
\end{sphinxVerbatim}

Some attributes are available on all fields, here are the most common ones:
\begin{description}
\item[{\sphinxcode{\sphinxupquote{string}} (\sphinxcode{\sphinxupquote{unicode}}, default: field’s name)}] \leavevmode
The label of the field in UI (visible by users).

\item[{\sphinxcode{\sphinxupquote{required}} (\sphinxcode{\sphinxupquote{bool}}, default: \sphinxcode{\sphinxupquote{False}})}] \leavevmode
If \sphinxcode{\sphinxupquote{True}}, the field can not be empty, it must either have a default
value or always be given a value when creating a record.

\item[{\sphinxcode{\sphinxupquote{help}} (\sphinxcode{\sphinxupquote{unicode}}, default: \sphinxcode{\sphinxupquote{'{'}}})}] \leavevmode
Long-form, provides a help tooltip to users in the UI.

\item[{\sphinxcode{\sphinxupquote{index}} (\sphinxcode{\sphinxupquote{bool}}, default: \sphinxcode{\sphinxupquote{False}})}] \leavevmode
Requests that Odoo create a \sphinxhref{http://use-the-index-luke.com/sql/preface}{database index} on the column.

\end{description}


\paragraph{Simple fields}
\label{\detokenize{howtos/backend:simple-fields}}
There are two broad categories of fields: “simple” fields which are atomic
values stored directly in the model’s table and “relational” fields linking
records (of the same model or of different models).

Example of simple fields are {\hyperref[\detokenize{reference/orm:odoo.fields.Boolean}]{\sphinxcrossref{\sphinxcode{\sphinxupquote{Boolean}}}}},
{\hyperref[\detokenize{reference/orm:odoo.fields.Date}]{\sphinxcrossref{\sphinxcode{\sphinxupquote{Date}}}}}, {\hyperref[\detokenize{reference/orm:odoo.fields.Char}]{\sphinxcrossref{\sphinxcode{\sphinxupquote{Char}}}}}.


\paragraph{Reserved fields}
\label{\detokenize{howtos/backend:reserved-fields}}
Odoo creates a few fields in all models%
\begin{footnote}[1]\sphinxAtStartFootnote
it is possible to {\hyperref[\detokenize{reference/orm:odoo.models.Model._log_access}]{\sphinxcrossref{\sphinxcode{\sphinxupquote{disable the automatic creation of some
fields}}}}}
%
\end{footnote}. These fields are
managed by the system and shouldn’t be written to. They can be read if
useful or necessary:
\begin{description}
\item[{\sphinxcode{\sphinxupquote{id}} (\sphinxcode{\sphinxupquote{Id}})}] \leavevmode
The unique identifier for a record in its model.

\item[{\sphinxcode{\sphinxupquote{create\_date}} ({\hyperref[\detokenize{reference/orm:odoo.fields.Datetime}]{\sphinxcrossref{\sphinxcode{\sphinxupquote{Datetime}}}}})}] \leavevmode
Creation date of the record.

\item[{\sphinxcode{\sphinxupquote{create\_uid}} ({\hyperref[\detokenize{reference/orm:odoo.fields.Many2one}]{\sphinxcrossref{\sphinxcode{\sphinxupquote{Many2one}}}}})}] \leavevmode
User who created the record.

\item[{\sphinxcode{\sphinxupquote{write\_date}} ({\hyperref[\detokenize{reference/orm:odoo.fields.Datetime}]{\sphinxcrossref{\sphinxcode{\sphinxupquote{Datetime}}}}})}] \leavevmode
Last modification date of the record.

\item[{\sphinxcode{\sphinxupquote{write\_uid}} ({\hyperref[\detokenize{reference/orm:odoo.fields.Many2one}]{\sphinxcrossref{\sphinxcode{\sphinxupquote{Many2one}}}}})}] \leavevmode
user who last modified the record.

\end{description}


\paragraph{Special fields}
\label{\detokenize{howtos/backend:special-fields}}
By default, Odoo also requires a \sphinxcode{\sphinxupquote{name}} field on all models for various
display and search behaviors. The field used for these purposes can be
overridden by setting {\hyperref[\detokenize{reference/orm:odoo.models.Model._rec_name}]{\sphinxcrossref{\sphinxcode{\sphinxupquote{\_rec\_name}}}}}.

\begin{sphinxadmonition}{note}
Define a model

Define a new data model \sphinxstyleemphasis{Course} in the \sphinxstyleemphasis{openacademy} module. A course
has a title and a description. Courses must have a title.

Edit the file \sphinxcode{\sphinxupquote{openacademy/models/models.py}} to include a \sphinxstyleemphasis{Course} class.
\sphinxstyleemphasis{openacademy/models.py}
\fvset{hllines={, 3, 4, 6, 7,}}%
\begin{sphinxVerbatim}[commandchars=\\\{\}]

\PYG{k+kn}{from} \PYG{n+nn}{odoo} \PYG{k+kn}{import} \PYG{n}{models}\PYG{p}{,} \PYG{n}{fields}\PYG{p}{,} \PYG{n}{api}

\PYG{k}{class} \PYG{n+nc}{Course}\PYG{p}{(}\PYG{n}{models}\PYG{o}{.}\PYG{n}{Model}\PYG{p}{)}\PYG{p}{:}
    \PYG{n}{\PYGZus{}name} \PYG{o}{=} \PYG{l+s+s1}{\PYGZsq{}}\PYG{l+s+s1}{openacademy.course}\PYG{l+s+s1}{\PYGZsq{}}

    \PYG{n}{name} \PYG{o}{=} \PYG{n}{fields}\PYG{o}{.}\PYG{n}{Char}\PYG{p}{(}\PYG{n}{string}\PYG{o}{=}\PYG{l+s+s2}{\PYGZdq{}}\PYG{l+s+s2}{Title}\PYG{l+s+s2}{\PYGZdq{}}\PYG{p}{,} \PYG{n}{required}\PYG{o}{=}\PYG{n+nb+bp}{True}\PYG{p}{)}
    \PYG{n}{description} \PYG{o}{=} \PYG{n}{fields}\PYG{o}{.}\PYG{n}{Text}\PYG{p}{(}\PYG{p}{)}
\end{sphinxVerbatim}
\end{sphinxadmonition}


\subsubsection{Data files}
\label{\detokenize{howtos/backend:data-files}}
Odoo is a highly data driven system. Although behavior is customized using
\sphinxhref{http://python.org}{Python} code part of a module’s value is in the data it sets up when loaded.

\begin{sphinxadmonition}{tip}{Tip:}
some modules exist solely to add data into Odoo
\end{sphinxadmonition}

Module data is declared via {\hyperref[\detokenize{reference/data:reference-data}]{\sphinxcrossref{\DUrole{std,std-ref}{data files}}}}, XML files with
\sphinxcode{\sphinxupquote{\textless{}record\textgreater{}}} elements. Each \sphinxcode{\sphinxupquote{\textless{}record\textgreater{}}} element creates or updates a database
record.

\fvset{hllines={, ,}}%
\begin{sphinxVerbatim}[commandchars=\\\{\}]
\PYG{n+nt}{\PYGZlt{}odoo}\PYG{n+nt}{\PYGZgt{}}

        \PYG{n+nt}{\PYGZlt{}record} \PYG{n+na}{model=}\PYG{l+s}{\PYGZdq{}\PYGZob{}model name\PYGZcb{}\PYGZdq{}} \PYG{n+na}{id=}\PYG{l+s}{\PYGZdq{}\PYGZob{}record identifier\PYGZcb{}\PYGZdq{}}\PYG{n+nt}{\PYGZgt{}}
            \PYG{n+nt}{\PYGZlt{}field} \PYG{n+na}{name=}\PYG{l+s}{\PYGZdq{}\PYGZob{}a field name\PYGZcb{}\PYGZdq{}}\PYG{n+nt}{\PYGZgt{}}\PYGZob{}a value\PYGZcb{}\PYG{n+nt}{\PYGZlt{}/field\PYGZgt{}}
        \PYG{n+nt}{\PYGZlt{}/record\PYGZgt{}}

\PYG{n+nt}{\PYGZlt{}/odoo\PYGZgt{}}
\end{sphinxVerbatim}
\begin{itemize}
\item {} 
\sphinxcode{\sphinxupquote{model}} is the name of the Odoo model for the record.

\item {} 
\sphinxcode{\sphinxupquote{id}} is an \DUrole{xref,std,std-term}{external identifier}, it allows referring to the record
(without having to know its in-database identifier).

\item {} 
\sphinxcode{\sphinxupquote{\textless{}field\textgreater{}}} elements have a \sphinxcode{\sphinxupquote{name}} which is the name of the field in the
model (e.g. \sphinxcode{\sphinxupquote{description}}). Their body is the field’s value.

\end{itemize}

Data files have to be declared in the manifest file to be loaded, they can
be declared in the \sphinxcode{\sphinxupquote{'data'}} list (always loaded) or in the \sphinxcode{\sphinxupquote{'demo'}} list
(only loaded in demonstration mode).

\begin{sphinxadmonition}{note}
Define demonstration data

Create demonstration data filling the \sphinxstyleemphasis{Courses} model with a few
demonstration courses.

Edit the file \sphinxcode{\sphinxupquote{openacademy/demo/demo.xml}} to include some data.
\sphinxstyleemphasis{openacademy/demo.xml}
\fvset{hllines={, 3, 4, 5, 6, 7, 8, 9, 10, 11, 12, 13, 14, 15, 16, 17,}}%
\begin{sphinxVerbatim}[commandchars=\\\{\}]
\PYG{n+nt}{\PYGZlt{}odoo}\PYG{n+nt}{\PYGZgt{}}

        \PYG{n+nt}{\PYGZlt{}record} \PYG{n+na}{model=}\PYG{l+s}{\PYGZdq{}openacademy.course\PYGZdq{}} \PYG{n+na}{id=}\PYG{l+s}{\PYGZdq{}course0\PYGZdq{}}\PYG{n+nt}{\PYGZgt{}}
            \PYG{n+nt}{\PYGZlt{}field} \PYG{n+na}{name=}\PYG{l+s}{\PYGZdq{}name\PYGZdq{}}\PYG{n+nt}{\PYGZgt{}}Course 0\PYG{n+nt}{\PYGZlt{}/field\PYGZgt{}}
            \PYG{n+nt}{\PYGZlt{}field} \PYG{n+na}{name=}\PYG{l+s}{\PYGZdq{}description\PYGZdq{}}\PYG{n+nt}{\PYGZgt{}}Course 0\PYGZsq{}s description

Can have multiple lines
            \PYG{n+nt}{\PYGZlt{}/field\PYGZgt{}}
        \PYG{n+nt}{\PYGZlt{}/record\PYGZgt{}}
        \PYG{n+nt}{\PYGZlt{}record} \PYG{n+na}{model=}\PYG{l+s}{\PYGZdq{}openacademy.course\PYGZdq{}} \PYG{n+na}{id=}\PYG{l+s}{\PYGZdq{}course1\PYGZdq{}}\PYG{n+nt}{\PYGZgt{}}
            \PYG{n+nt}{\PYGZlt{}field} \PYG{n+na}{name=}\PYG{l+s}{\PYGZdq{}name\PYGZdq{}}\PYG{n+nt}{\PYGZgt{}}Course 1\PYG{n+nt}{\PYGZlt{}/field\PYGZgt{}}
            \PYG{c}{\PYGZlt{}!\PYGZhy{}\PYGZhy{}}\PYG{c}{ no description for this one }\PYG{c}{\PYGZhy{}\PYGZhy{}\PYGZgt{}}
        \PYG{n+nt}{\PYGZlt{}/record\PYGZgt{}}
        \PYG{n+nt}{\PYGZlt{}record} \PYG{n+na}{model=}\PYG{l+s}{\PYGZdq{}openacademy.course\PYGZdq{}} \PYG{n+na}{id=}\PYG{l+s}{\PYGZdq{}course2\PYGZdq{}}\PYG{n+nt}{\PYGZgt{}}
            \PYG{n+nt}{\PYGZlt{}field} \PYG{n+na}{name=}\PYG{l+s}{\PYGZdq{}name\PYGZdq{}}\PYG{n+nt}{\PYGZgt{}}Course 2\PYG{n+nt}{\PYGZlt{}/field\PYGZgt{}}
            \PYG{n+nt}{\PYGZlt{}field} \PYG{n+na}{name=}\PYG{l+s}{\PYGZdq{}description\PYGZdq{}}\PYG{n+nt}{\PYGZgt{}}Course 2\PYGZsq{}s description\PYG{n+nt}{\PYGZlt{}/field\PYGZgt{}}
        \PYG{n+nt}{\PYGZlt{}/record\PYGZgt{}}

\PYG{n+nt}{\PYGZlt{}/odoo\PYGZgt{}}
\end{sphinxVerbatim}
\end{sphinxadmonition}

\begin{sphinxadmonition}{tip}{Tip:}
The content of the data files is only loaded when a module is
installed or updated.

After making some changes, do not forget to use
{\hyperref[\detokenize{reference/cmdline:reference-cmdline}]{\sphinxcrossref{\DUrole{std,std-ref}{odoo-bin -u openacademy}}}} to save the changes
to your database.
\end{sphinxadmonition}


\subsubsection{Actions and Menus}
\label{\detokenize{howtos/backend:actions-and-menus}}
Actions and menus are regular records in database, usually declared through
data files. Actions can be triggered in three ways:
\begin{enumerate}
\item {} 
by clicking on menu items (linked to specific actions)

\item {} 
by clicking on buttons in views (if these are connected to actions)

\item {} 
as contextual actions on object

\end{enumerate}

Because menus are somewhat complex to declare there is a \sphinxcode{\sphinxupquote{\textless{}menuitem\textgreater{}}}
shortcut to declare an \sphinxcode{\sphinxupquote{ir.ui.menu}} and connect it to the corresponding
action more easily.

\fvset{hllines={, ,}}%
\begin{sphinxVerbatim}[commandchars=\\\{\}]
\PYG{n+nt}{\PYGZlt{}record} \PYG{n+na}{model=}\PYG{l+s}{\PYGZdq{}ir.actions.act\PYGZus{}window\PYGZdq{}} \PYG{n+na}{id=}\PYG{l+s}{\PYGZdq{}action\PYGZus{}list\PYGZus{}ideas\PYGZdq{}}\PYG{n+nt}{\PYGZgt{}}
    \PYG{n+nt}{\PYGZlt{}field} \PYG{n+na}{name=}\PYG{l+s}{\PYGZdq{}name\PYGZdq{}}\PYG{n+nt}{\PYGZgt{}}Ideas\PYG{n+nt}{\PYGZlt{}/field\PYGZgt{}}
    \PYG{n+nt}{\PYGZlt{}field} \PYG{n+na}{name=}\PYG{l+s}{\PYGZdq{}res\PYGZus{}model\PYGZdq{}}\PYG{n+nt}{\PYGZgt{}}idea.idea\PYG{n+nt}{\PYGZlt{}/field\PYGZgt{}}
    \PYG{n+nt}{\PYGZlt{}field} \PYG{n+na}{name=}\PYG{l+s}{\PYGZdq{}view\PYGZus{}mode\PYGZdq{}}\PYG{n+nt}{\PYGZgt{}}tree,form\PYG{n+nt}{\PYGZlt{}/field\PYGZgt{}}
\PYG{n+nt}{\PYGZlt{}/record\PYGZgt{}}
\PYG{n+nt}{\PYGZlt{}menuitem} \PYG{n+na}{id=}\PYG{l+s}{\PYGZdq{}menu\PYGZus{}ideas\PYGZdq{}} \PYG{n+na}{parent=}\PYG{l+s}{\PYGZdq{}menu\PYGZus{}root\PYGZdq{}} \PYG{n+na}{name=}\PYG{l+s}{\PYGZdq{}Ideas\PYGZdq{}} \PYG{n+na}{sequence=}\PYG{l+s}{\PYGZdq{}10\PYGZdq{}}
          \PYG{n+na}{action=}\PYG{l+s}{\PYGZdq{}action\PYGZus{}list\PYGZus{}ideas\PYGZdq{}}\PYG{n+nt}{/\PYGZgt{}}
\end{sphinxVerbatim}

\begin{sphinxadmonition}{danger}{Danger:}
The action must be declared before its corresponding menu in the XML file.

Data files are executed sequentially, the action’s \sphinxcode{\sphinxupquote{id}} must be present
in the database before the menu can be created.
\end{sphinxadmonition}

\begin{sphinxadmonition}{note}
Define new menu entries

Define new menu entries to access courses under the
OpenAcademy menu entry. A user should be able to :
\begin{itemize}
\item {} 
display a list of all the courses

\item {} 
create/modify courses

\end{itemize}
\begin{enumerate}
\item {} 
Create \sphinxcode{\sphinxupquote{openacademy/views/openacademy.xml}} with an action and
the menus triggering the action

\item {} 
Add it to the \sphinxcode{\sphinxupquote{data}} list of \sphinxcode{\sphinxupquote{openacademy/\_\_manifest\_\_.py}}

\end{enumerate}
\sphinxstyleemphasis{openacademy/\_\_manifest\_\_.py}
\fvset{hllines={, 4,}}%
\begin{sphinxVerbatim}[commandchars=\\\{\}]
    \PYG{l+s+s1}{\PYGZsq{}}\PYG{l+s+s1}{data}\PYG{l+s+s1}{\PYGZsq{}}\PYG{p}{:} \PYG{p}{[}
        \PYG{c+c1}{\PYGZsh{} \PYGZsq{}security/ir.model.access.csv\PYGZsq{},}
        \PYG{l+s+s1}{\PYGZsq{}}\PYG{l+s+s1}{templates.xml}\PYG{l+s+s1}{\PYGZsq{}}\PYG{p}{,}
        \PYG{l+s+s1}{\PYGZsq{}}\PYG{l+s+s1}{views/openacademy.xml}\PYG{l+s+s1}{\PYGZsq{}}\PYG{p}{,}
    \PYG{p}{]}\PYG{p}{,}
    \PYG{c+c1}{\PYGZsh{} only loaded in demonstration mode}
    \PYG{l+s+s1}{\PYGZsq{}}\PYG{l+s+s1}{demo}\PYG{l+s+s1}{\PYGZsq{}}\PYG{p}{:} \PYG{p}{[}
\end{sphinxVerbatim}
\sphinxstyleemphasis{openacademy/views/openacademy.xml}
\fvset{hllines={, 1, 2, 3, 4, 5, 6, 7, 8, 9, 10, 11, 12, 13, 14, 15, 16, 17, 18, 19, 20, 21, 22, 23, 24, 25, 26, 27, 28, 29, 30, 31, 32, 33, 34, 35,}}%
\begin{sphinxVerbatim}[commandchars=\\\{\}]
\PYG{c+cp}{\PYGZlt{}?xml version=\PYGZdq{}1.0\PYGZdq{} encoding=\PYGZdq{}UTF\PYGZhy{}8\PYGZdq{}?\PYGZgt{}}
\PYG{n+nt}{\PYGZlt{}odoo}\PYG{n+nt}{\PYGZgt{}}

        \PYG{c}{\PYGZlt{}!\PYGZhy{}\PYGZhy{}}\PYG{c}{ window action }\PYG{c}{\PYGZhy{}\PYGZhy{}\PYGZgt{}}
        \PYG{c}{\PYGZlt{}!\PYGZhy{}\PYGZhy{}}
\PYG{c}{            The following tag is an action definition for a \PYGZdq{}window action\PYGZdq{},}
\PYG{c}{            that is an action opening a view or a set of views}
\PYG{c}{        }\PYG{c}{\PYGZhy{}\PYGZhy{}\PYGZgt{}}
        \PYG{n+nt}{\PYGZlt{}record} \PYG{n+na}{model=}\PYG{l+s}{\PYGZdq{}ir.actions.act\PYGZus{}window\PYGZdq{}} \PYG{n+na}{id=}\PYG{l+s}{\PYGZdq{}course\PYGZus{}list\PYGZus{}action\PYGZdq{}}\PYG{n+nt}{\PYGZgt{}}
            \PYG{n+nt}{\PYGZlt{}field} \PYG{n+na}{name=}\PYG{l+s}{\PYGZdq{}name\PYGZdq{}}\PYG{n+nt}{\PYGZgt{}}Courses\PYG{n+nt}{\PYGZlt{}/field\PYGZgt{}}
            \PYG{n+nt}{\PYGZlt{}field} \PYG{n+na}{name=}\PYG{l+s}{\PYGZdq{}res\PYGZus{}model\PYGZdq{}}\PYG{n+nt}{\PYGZgt{}}openacademy.course\PYG{n+nt}{\PYGZlt{}/field\PYGZgt{}}
            \PYG{n+nt}{\PYGZlt{}field} \PYG{n+na}{name=}\PYG{l+s}{\PYGZdq{}view\PYGZus{}type\PYGZdq{}}\PYG{n+nt}{\PYGZgt{}}form\PYG{n+nt}{\PYGZlt{}/field\PYGZgt{}}
            \PYG{n+nt}{\PYGZlt{}field} \PYG{n+na}{name=}\PYG{l+s}{\PYGZdq{}view\PYGZus{}mode\PYGZdq{}}\PYG{n+nt}{\PYGZgt{}}tree,form\PYG{n+nt}{\PYGZlt{}/field\PYGZgt{}}
            \PYG{n+nt}{\PYGZlt{}field} \PYG{n+na}{name=}\PYG{l+s}{\PYGZdq{}help\PYGZdq{}} \PYG{n+na}{type=}\PYG{l+s}{\PYGZdq{}html\PYGZdq{}}\PYG{n+nt}{\PYGZgt{}}
                \PYG{n+nt}{\PYGZlt{}p} \PYG{n+na}{class=}\PYG{l+s}{\PYGZdq{}oe\PYGZus{}view\PYGZus{}nocontent\PYGZus{}create\PYGZdq{}}\PYG{n+nt}{\PYGZgt{}}Create the first course
                \PYG{n+nt}{\PYGZlt{}/p\PYGZgt{}}
            \PYG{n+nt}{\PYGZlt{}/field\PYGZgt{}}
        \PYG{n+nt}{\PYGZlt{}/record\PYGZgt{}}

        \PYG{c}{\PYGZlt{}!\PYGZhy{}\PYGZhy{}}\PYG{c}{ top level menu: no parent }\PYG{c}{\PYGZhy{}\PYGZhy{}\PYGZgt{}}
        \PYG{n+nt}{\PYGZlt{}menuitem} \PYG{n+na}{id=}\PYG{l+s}{\PYGZdq{}main\PYGZus{}openacademy\PYGZus{}menu\PYGZdq{}} \PYG{n+na}{name=}\PYG{l+s}{\PYGZdq{}Open Academy\PYGZdq{}}\PYG{n+nt}{/\PYGZgt{}}
        \PYG{c}{\PYGZlt{}!\PYGZhy{}\PYGZhy{}}\PYG{c}{ A first level in the left side menu is needed}
\PYG{c}{             before using action= attribute }\PYG{c}{\PYGZhy{}\PYGZhy{}\PYGZgt{}}
        \PYG{n+nt}{\PYGZlt{}menuitem} \PYG{n+na}{id=}\PYG{l+s}{\PYGZdq{}openacademy\PYGZus{}menu\PYGZdq{}} \PYG{n+na}{name=}\PYG{l+s}{\PYGZdq{}Open Academy\PYGZdq{}}
                  \PYG{n+na}{parent=}\PYG{l+s}{\PYGZdq{}main\PYGZus{}openacademy\PYGZus{}menu\PYGZdq{}}\PYG{n+nt}{/\PYGZgt{}}
        \PYG{c}{\PYGZlt{}!\PYGZhy{}\PYGZhy{}}\PYG{c}{ the following menuitem should appear *after*}
\PYG{c}{             its parent openacademy\PYGZus{}menu and *after* its}
\PYG{c}{             action course\PYGZus{}list\PYGZus{}action }\PYG{c}{\PYGZhy{}\PYGZhy{}\PYGZgt{}}
        \PYG{n+nt}{\PYGZlt{}menuitem} \PYG{n+na}{id=}\PYG{l+s}{\PYGZdq{}courses\PYGZus{}menu\PYGZdq{}} \PYG{n+na}{name=}\PYG{l+s}{\PYGZdq{}Courses\PYGZdq{}} \PYG{n+na}{parent=}\PYG{l+s}{\PYGZdq{}openacademy\PYGZus{}menu\PYGZdq{}}
                  \PYG{n+na}{action=}\PYG{l+s}{\PYGZdq{}course\PYGZus{}list\PYGZus{}action\PYGZdq{}}\PYG{n+nt}{/\PYGZgt{}}
        \PYG{c}{\PYGZlt{}!\PYGZhy{}\PYGZhy{}}\PYG{c}{ Full id location:}
\PYG{c}{             action=\PYGZdq{}openacademy.course\PYGZus{}list\PYGZus{}action\PYGZdq{}}
\PYG{c}{             It is not required when it is the same module }\PYG{c}{\PYGZhy{}\PYGZhy{}\PYGZgt{}}

\PYG{n+nt}{\PYGZlt{}/odoo\PYGZgt{}}
\end{sphinxVerbatim}
\end{sphinxadmonition}


\subsection{Basic views}
\label{\detokenize{howtos/backend:basic-views}}
Views define the way the records of a model are displayed. Each type of view
represents a mode of visualization (a list of records, a graph of their
aggregation, …). Views can either be requested generically via their type
(e.g. \sphinxstyleemphasis{a list of partners}) or specifically via their id. For generic
requests, the view with the correct type and the lowest priority will be
used (so the lowest-priority view of each type is the default view for that
type).

{\hyperref[\detokenize{reference/views:reference-views-inheritance}]{\sphinxcrossref{\DUrole{std,std-ref}{View inheritance}}}} allows altering views
declared elsewhere (adding or removing content).


\subsubsection{Generic view declaration}
\label{\detokenize{howtos/backend:generic-view-declaration}}
A view is declared as a record of the model \sphinxcode{\sphinxupquote{ir.ui.view}}. The view type
is implied by the root element of the \sphinxcode{\sphinxupquote{arch}} field:

\fvset{hllines={, ,}}%
\begin{sphinxVerbatim}[commandchars=\\\{\}]
\PYG{n+nt}{\PYGZlt{}record} \PYG{n+na}{model=}\PYG{l+s}{\PYGZdq{}ir.ui.view\PYGZdq{}} \PYG{n+na}{id=}\PYG{l+s}{\PYGZdq{}view\PYGZus{}id\PYGZdq{}}\PYG{n+nt}{\PYGZgt{}}
    \PYG{n+nt}{\PYGZlt{}field} \PYG{n+na}{name=}\PYG{l+s}{\PYGZdq{}name\PYGZdq{}}\PYG{n+nt}{\PYGZgt{}}view.name\PYG{n+nt}{\PYGZlt{}/field\PYGZgt{}}
    \PYG{n+nt}{\PYGZlt{}field} \PYG{n+na}{name=}\PYG{l+s}{\PYGZdq{}model\PYGZdq{}}\PYG{n+nt}{\PYGZgt{}}object\PYGZus{}name\PYG{n+nt}{\PYGZlt{}/field\PYGZgt{}}
    \PYG{n+nt}{\PYGZlt{}field} \PYG{n+na}{name=}\PYG{l+s}{\PYGZdq{}priority\PYGZdq{}} \PYG{n+na}{eval=}\PYG{l+s}{\PYGZdq{}16\PYGZdq{}}\PYG{n+nt}{/\PYGZgt{}}
    \PYG{n+nt}{\PYGZlt{}field} \PYG{n+na}{name=}\PYG{l+s}{\PYGZdq{}arch\PYGZdq{}} \PYG{n+na}{type=}\PYG{l+s}{\PYGZdq{}xml\PYGZdq{}}\PYG{n+nt}{\PYGZgt{}}
        \PYG{c}{\PYGZlt{}!\PYGZhy{}\PYGZhy{}}\PYG{c}{ view content: \PYGZlt{}form\PYGZgt{}, \PYGZlt{}tree\PYGZgt{}, \PYGZlt{}graph\PYGZgt{}, ... }\PYG{c}{\PYGZhy{}\PYGZhy{}\PYGZgt{}}
    \PYG{n+nt}{\PYGZlt{}/field\PYGZgt{}}
\PYG{n+nt}{\PYGZlt{}/record\PYGZgt{}}
\end{sphinxVerbatim}

\begin{sphinxadmonition}{danger}{Danger:}
The view’s content is XML.

The \sphinxcode{\sphinxupquote{arch}} field must thus be declared as \sphinxcode{\sphinxupquote{type="xml"}} to be parsed
correctly.
\end{sphinxadmonition}


\subsubsection{Tree views}
\label{\detokenize{howtos/backend:tree-views}}
Tree views, also called list views, display records in a tabular form.

Their root element is \sphinxcode{\sphinxupquote{\textless{}tree\textgreater{}}}. The simplest form of the tree view simply
lists all the fields to display in the table (each field as a column):

\fvset{hllines={, ,}}%
\begin{sphinxVerbatim}[commandchars=\\\{\}]
\PYG{n+nt}{\PYGZlt{}tree} \PYG{n+na}{string=}\PYG{l+s}{\PYGZdq{}Idea list\PYGZdq{}}\PYG{n+nt}{\PYGZgt{}}
    \PYG{n+nt}{\PYGZlt{}field} \PYG{n+na}{name=}\PYG{l+s}{\PYGZdq{}name\PYGZdq{}}\PYG{n+nt}{/\PYGZgt{}}
    \PYG{n+nt}{\PYGZlt{}field} \PYG{n+na}{name=}\PYG{l+s}{\PYGZdq{}inventor\PYGZus{}id\PYGZdq{}}\PYG{n+nt}{/\PYGZgt{}}
\PYG{n+nt}{\PYGZlt{}/tree\PYGZgt{}}
\end{sphinxVerbatim}


\subsubsection{Form views}
\label{\detokenize{howtos/backend:form-views}}
Forms are used to create and edit single records.

Their root element is \sphinxcode{\sphinxupquote{\textless{}form\textgreater{}}}. They are composed of high-level structure
elements (groups, notebooks) and interactive elements (buttons and fields):

\fvset{hllines={, ,}}%
\begin{sphinxVerbatim}[commandchars=\\\{\}]
\PYG{n+nt}{\PYGZlt{}form} \PYG{n+na}{string=}\PYG{l+s}{\PYGZdq{}Idea form\PYGZdq{}}\PYG{n+nt}{\PYGZgt{}}
    \PYG{n+nt}{\PYGZlt{}group} \PYG{n+na}{colspan=}\PYG{l+s}{\PYGZdq{}4\PYGZdq{}}\PYG{n+nt}{\PYGZgt{}}
        \PYG{n+nt}{\PYGZlt{}group} \PYG{n+na}{colspan=}\PYG{l+s}{\PYGZdq{}2\PYGZdq{}} \PYG{n+na}{col=}\PYG{l+s}{\PYGZdq{}2\PYGZdq{}}\PYG{n+nt}{\PYGZgt{}}
            \PYG{n+nt}{\PYGZlt{}separator} \PYG{n+na}{string=}\PYG{l+s}{\PYGZdq{}General stuff\PYGZdq{}} \PYG{n+na}{colspan=}\PYG{l+s}{\PYGZdq{}2\PYGZdq{}}\PYG{n+nt}{/\PYGZgt{}}
            \PYG{n+nt}{\PYGZlt{}field} \PYG{n+na}{name=}\PYG{l+s}{\PYGZdq{}name\PYGZdq{}}\PYG{n+nt}{/\PYGZgt{}}
            \PYG{n+nt}{\PYGZlt{}field} \PYG{n+na}{name=}\PYG{l+s}{\PYGZdq{}inventor\PYGZus{}id\PYGZdq{}}\PYG{n+nt}{/\PYGZgt{}}
        \PYG{n+nt}{\PYGZlt{}/group\PYGZgt{}}

        \PYG{n+nt}{\PYGZlt{}group} \PYG{n+na}{colspan=}\PYG{l+s}{\PYGZdq{}2\PYGZdq{}} \PYG{n+na}{col=}\PYG{l+s}{\PYGZdq{}2\PYGZdq{}}\PYG{n+nt}{\PYGZgt{}}
            \PYG{n+nt}{\PYGZlt{}separator} \PYG{n+na}{string=}\PYG{l+s}{\PYGZdq{}Dates\PYGZdq{}} \PYG{n+na}{colspan=}\PYG{l+s}{\PYGZdq{}2\PYGZdq{}}\PYG{n+nt}{/\PYGZgt{}}
            \PYG{n+nt}{\PYGZlt{}field} \PYG{n+na}{name=}\PYG{l+s}{\PYGZdq{}active\PYGZdq{}}\PYG{n+nt}{/\PYGZgt{}}
            \PYG{n+nt}{\PYGZlt{}field} \PYG{n+na}{name=}\PYG{l+s}{\PYGZdq{}invent\PYGZus{}date\PYGZdq{}} \PYG{n+na}{readonly=}\PYG{l+s}{\PYGZdq{}1\PYGZdq{}}\PYG{n+nt}{/\PYGZgt{}}
        \PYG{n+nt}{\PYGZlt{}/group\PYGZgt{}}

        \PYG{n+nt}{\PYGZlt{}notebook} \PYG{n+na}{colspan=}\PYG{l+s}{\PYGZdq{}4\PYGZdq{}}\PYG{n+nt}{\PYGZgt{}}
            \PYG{n+nt}{\PYGZlt{}page} \PYG{n+na}{string=}\PYG{l+s}{\PYGZdq{}Description\PYGZdq{}}\PYG{n+nt}{\PYGZgt{}}
                \PYG{n+nt}{\PYGZlt{}field} \PYG{n+na}{name=}\PYG{l+s}{\PYGZdq{}description\PYGZdq{}} \PYG{n+na}{nolabel=}\PYG{l+s}{\PYGZdq{}1\PYGZdq{}}\PYG{n+nt}{/\PYGZgt{}}
            \PYG{n+nt}{\PYGZlt{}/page\PYGZgt{}}
        \PYG{n+nt}{\PYGZlt{}/notebook\PYGZgt{}}

        \PYG{n+nt}{\PYGZlt{}field} \PYG{n+na}{name=}\PYG{l+s}{\PYGZdq{}state\PYGZdq{}}\PYG{n+nt}{/\PYGZgt{}}
    \PYG{n+nt}{\PYGZlt{}/group\PYGZgt{}}
\PYG{n+nt}{\PYGZlt{}/form\PYGZgt{}}
\end{sphinxVerbatim}

\begin{sphinxadmonition}{note}
Customise form view using XML

Create your own form view for the Course object. Data displayed should be:
the name and the description of the course.
\sphinxstyleemphasis{openacademy/views/openacademy.xml}
\fvset{hllines={, 4, 5, 6, 7, 8, 9, 10, 11, 12, 13, 14, 15, 16, 17, 18,}}%
\begin{sphinxVerbatim}[commandchars=\\\{\}]
\PYG{c+cp}{\PYGZlt{}?xml version=\PYGZdq{}1.0\PYGZdq{} encoding=\PYGZdq{}UTF\PYGZhy{}8\PYGZdq{}?\PYGZgt{}}
\PYG{n+nt}{\PYGZlt{}odoo}\PYG{n+nt}{\PYGZgt{}}

        \PYG{n+nt}{\PYGZlt{}record} \PYG{n+na}{model=}\PYG{l+s}{\PYGZdq{}ir.ui.view\PYGZdq{}} \PYG{n+na}{id=}\PYG{l+s}{\PYGZdq{}course\PYGZus{}form\PYGZus{}view\PYGZdq{}}\PYG{n+nt}{\PYGZgt{}}
            \PYG{n+nt}{\PYGZlt{}field} \PYG{n+na}{name=}\PYG{l+s}{\PYGZdq{}name\PYGZdq{}}\PYG{n+nt}{\PYGZgt{}}course.form\PYG{n+nt}{\PYGZlt{}/field\PYGZgt{}}
            \PYG{n+nt}{\PYGZlt{}field} \PYG{n+na}{name=}\PYG{l+s}{\PYGZdq{}model\PYGZdq{}}\PYG{n+nt}{\PYGZgt{}}openacademy.course\PYG{n+nt}{\PYGZlt{}/field\PYGZgt{}}
            \PYG{n+nt}{\PYGZlt{}field} \PYG{n+na}{name=}\PYG{l+s}{\PYGZdq{}arch\PYGZdq{}} \PYG{n+na}{type=}\PYG{l+s}{\PYGZdq{}xml\PYGZdq{}}\PYG{n+nt}{\PYGZgt{}}
                \PYG{n+nt}{\PYGZlt{}form} \PYG{n+na}{string=}\PYG{l+s}{\PYGZdq{}Course Form\PYGZdq{}}\PYG{n+nt}{\PYGZgt{}}
                    \PYG{n+nt}{\PYGZlt{}sheet}\PYG{n+nt}{\PYGZgt{}}
                        \PYG{n+nt}{\PYGZlt{}group}\PYG{n+nt}{\PYGZgt{}}
                            \PYG{n+nt}{\PYGZlt{}field} \PYG{n+na}{name=}\PYG{l+s}{\PYGZdq{}name\PYGZdq{}}\PYG{n+nt}{/\PYGZgt{}}
                            \PYG{n+nt}{\PYGZlt{}field} \PYG{n+na}{name=}\PYG{l+s}{\PYGZdq{}description\PYGZdq{}}\PYG{n+nt}{/\PYGZgt{}}
                        \PYG{n+nt}{\PYGZlt{}/group\PYGZgt{}}
                    \PYG{n+nt}{\PYGZlt{}/sheet\PYGZgt{}}
                \PYG{n+nt}{\PYGZlt{}/form\PYGZgt{}}
            \PYG{n+nt}{\PYGZlt{}/field\PYGZgt{}}
        \PYG{n+nt}{\PYGZlt{}/record\PYGZgt{}}

        \PYG{c}{\PYGZlt{}!\PYGZhy{}\PYGZhy{}}\PYG{c}{ window action }\PYG{c}{\PYGZhy{}\PYGZhy{}\PYGZgt{}}
        \PYG{c}{\PYGZlt{}!\PYGZhy{}\PYGZhy{}}
\PYG{c}{            The following tag is an action definition for a \PYGZdq{}window action\PYGZdq{},}
\end{sphinxVerbatim}
\end{sphinxadmonition}

\begin{sphinxadmonition}{note}
Notebooks

In the Course form view, put the description field under a tab, such that
it will be easier to add other tabs later, containing additional
information.

Modify the Course form view as follows:
\sphinxstyleemphasis{openacademy/views/openacademy.xml}
\fvset{hllines={, 5, 6, 7, 8, 9, 10, 11, 12,}}%
\begin{sphinxVerbatim}[commandchars=\\\{\}]
                    \PYG{n+nt}{\PYGZlt{}sheet}\PYG{n+nt}{\PYGZgt{}}
                        \PYG{n+nt}{\PYGZlt{}group}\PYG{n+nt}{\PYGZgt{}}
                            \PYG{n+nt}{\PYGZlt{}field} \PYG{n+na}{name=}\PYG{l+s}{\PYGZdq{}name\PYGZdq{}}\PYG{n+nt}{/\PYGZgt{}}
                        \PYG{n+nt}{\PYGZlt{}/group\PYGZgt{}}
                        \PYG{n+nt}{\PYGZlt{}notebook}\PYG{n+nt}{\PYGZgt{}}
                            \PYG{n+nt}{\PYGZlt{}page} \PYG{n+na}{string=}\PYG{l+s}{\PYGZdq{}Description\PYGZdq{}}\PYG{n+nt}{\PYGZgt{}}
                                \PYG{n+nt}{\PYGZlt{}field} \PYG{n+na}{name=}\PYG{l+s}{\PYGZdq{}description\PYGZdq{}}\PYG{n+nt}{/\PYGZgt{}}
                            \PYG{n+nt}{\PYGZlt{}/page\PYGZgt{}}
                            \PYG{n+nt}{\PYGZlt{}page} \PYG{n+na}{string=}\PYG{l+s}{\PYGZdq{}About\PYGZdq{}}\PYG{n+nt}{\PYGZgt{}}
                                This is an example of notebooks
                            \PYG{n+nt}{\PYGZlt{}/page\PYGZgt{}}
                        \PYG{n+nt}{\PYGZlt{}/notebook\PYGZgt{}}
                    \PYG{n+nt}{\PYGZlt{}/sheet\PYGZgt{}}
                \PYG{n+nt}{\PYGZlt{}/form\PYGZgt{}}
            \PYG{n+nt}{\PYGZlt{}/field\PYGZgt{}}
\end{sphinxVerbatim}
\end{sphinxadmonition}

Form views can also use plain HTML for more flexible layouts:

\fvset{hllines={, ,}}%
\begin{sphinxVerbatim}[commandchars=\\\{\}]
\PYG{n+nt}{\PYGZlt{}form} \PYG{n+na}{string=}\PYG{l+s}{\PYGZdq{}Idea Form\PYGZdq{}}\PYG{n+nt}{\PYGZgt{}}
    \PYG{n+nt}{\PYGZlt{}header}\PYG{n+nt}{\PYGZgt{}}
        \PYG{n+nt}{\PYGZlt{}button} \PYG{n+na}{string=}\PYG{l+s}{\PYGZdq{}Confirm\PYGZdq{}} \PYG{n+na}{type=}\PYG{l+s}{\PYGZdq{}object\PYGZdq{}} \PYG{n+na}{name=}\PYG{l+s}{\PYGZdq{}action\PYGZus{}confirm\PYGZdq{}}
                \PYG{n+na}{states=}\PYG{l+s}{\PYGZdq{}draft\PYGZdq{}} \PYG{n+na}{class=}\PYG{l+s}{\PYGZdq{}oe\PYGZus{}highlight\PYGZdq{}} \PYG{n+nt}{/\PYGZgt{}}
        \PYG{n+nt}{\PYGZlt{}button} \PYG{n+na}{string=}\PYG{l+s}{\PYGZdq{}Mark as done\PYGZdq{}} \PYG{n+na}{type=}\PYG{l+s}{\PYGZdq{}object\PYGZdq{}} \PYG{n+na}{name=}\PYG{l+s}{\PYGZdq{}action\PYGZus{}done\PYGZdq{}}
                \PYG{n+na}{states=}\PYG{l+s}{\PYGZdq{}confirmed\PYGZdq{}} \PYG{n+na}{class=}\PYG{l+s}{\PYGZdq{}oe\PYGZus{}highlight\PYGZdq{}}\PYG{n+nt}{/\PYGZgt{}}
        \PYG{n+nt}{\PYGZlt{}button} \PYG{n+na}{string=}\PYG{l+s}{\PYGZdq{}Reset to draft\PYGZdq{}} \PYG{n+na}{type=}\PYG{l+s}{\PYGZdq{}object\PYGZdq{}} \PYG{n+na}{name=}\PYG{l+s}{\PYGZdq{}action\PYGZus{}draft\PYGZdq{}}
                \PYG{n+na}{states=}\PYG{l+s}{\PYGZdq{}confirmed,done\PYGZdq{}} \PYG{n+nt}{/\PYGZgt{}}
        \PYG{n+nt}{\PYGZlt{}field} \PYG{n+na}{name=}\PYG{l+s}{\PYGZdq{}state\PYGZdq{}} \PYG{n+na}{widget=}\PYG{l+s}{\PYGZdq{}statusbar\PYGZdq{}}\PYG{n+nt}{/\PYGZgt{}}
    \PYG{n+nt}{\PYGZlt{}/header\PYGZgt{}}
    \PYG{n+nt}{\PYGZlt{}sheet}\PYG{n+nt}{\PYGZgt{}}
        \PYG{n+nt}{\PYGZlt{}div} \PYG{n+na}{class=}\PYG{l+s}{\PYGZdq{}oe\PYGZus{}title\PYGZdq{}}\PYG{n+nt}{\PYGZgt{}}
            \PYG{n+nt}{\PYGZlt{}label} \PYG{n+na}{for=}\PYG{l+s}{\PYGZdq{}name\PYGZdq{}} \PYG{n+na}{class=}\PYG{l+s}{\PYGZdq{}oe\PYGZus{}edit\PYGZus{}only\PYGZdq{}} \PYG{n+na}{string=}\PYG{l+s}{\PYGZdq{}Idea Name\PYGZdq{}} \PYG{n+nt}{/\PYGZgt{}}
            \PYG{n+nt}{\PYGZlt{}h1}\PYG{n+nt}{\PYGZgt{}}\PYG{n+nt}{\PYGZlt{}field} \PYG{n+na}{name=}\PYG{l+s}{\PYGZdq{}name\PYGZdq{}} \PYG{n+nt}{/\PYGZgt{}}\PYG{n+nt}{\PYGZlt{}/h1\PYGZgt{}}
        \PYG{n+nt}{\PYGZlt{}/div\PYGZgt{}}
        \PYG{n+nt}{\PYGZlt{}separator} \PYG{n+na}{string=}\PYG{l+s}{\PYGZdq{}General\PYGZdq{}} \PYG{n+na}{colspan=}\PYG{l+s}{\PYGZdq{}2\PYGZdq{}} \PYG{n+nt}{/\PYGZgt{}}
        \PYG{n+nt}{\PYGZlt{}group} \PYG{n+na}{colspan=}\PYG{l+s}{\PYGZdq{}2\PYGZdq{}} \PYG{n+na}{col=}\PYG{l+s}{\PYGZdq{}2\PYGZdq{}}\PYG{n+nt}{\PYGZgt{}}
            \PYG{n+nt}{\PYGZlt{}field} \PYG{n+na}{name=}\PYG{l+s}{\PYGZdq{}description\PYGZdq{}} \PYG{n+na}{placeholder=}\PYG{l+s}{\PYGZdq{}Idea description...\PYGZdq{}} \PYG{n+nt}{/\PYGZgt{}}
        \PYG{n+nt}{\PYGZlt{}/group\PYGZgt{}}
    \PYG{n+nt}{\PYGZlt{}/sheet\PYGZgt{}}
\PYG{n+nt}{\PYGZlt{}/form\PYGZgt{}}
\end{sphinxVerbatim}


\subsubsection{Search views}
\label{\detokenize{howtos/backend:search-views}}
Search views customize the search field associated with the list view (and
other aggregated views). Their root element is \sphinxcode{\sphinxupquote{\textless{}search\textgreater{}}} and they’re
composed of fields defining which fields can be searched on:

\fvset{hllines={, ,}}%
\begin{sphinxVerbatim}[commandchars=\\\{\}]
\PYG{n+nt}{\PYGZlt{}search}\PYG{n+nt}{\PYGZgt{}}
    \PYG{n+nt}{\PYGZlt{}field} \PYG{n+na}{name=}\PYG{l+s}{\PYGZdq{}name\PYGZdq{}}\PYG{n+nt}{/\PYGZgt{}}
    \PYG{n+nt}{\PYGZlt{}field} \PYG{n+na}{name=}\PYG{l+s}{\PYGZdq{}inventor\PYGZus{}id\PYGZdq{}}\PYG{n+nt}{/\PYGZgt{}}
\PYG{n+nt}{\PYGZlt{}/search\PYGZgt{}}
\end{sphinxVerbatim}

If no search view exists for the model, Odoo generates one which only allows
searching on the \sphinxcode{\sphinxupquote{name}} field.

\begin{sphinxadmonition}{note}
Search courses

Allow searching for courses based on their title or their description.
\sphinxstyleemphasis{openacademy/views/openacademy.xml}
\fvset{hllines={, 4, 5, 6, 7, 8, 9, 10, 11, 12, 13, 14,}}%
\begin{sphinxVerbatim}[commandchars=\\\{\}]
            \PYG{n+nt}{\PYGZlt{}/field\PYGZgt{}}
        \PYG{n+nt}{\PYGZlt{}/record\PYGZgt{}}

        \PYG{n+nt}{\PYGZlt{}record} \PYG{n+na}{model=}\PYG{l+s}{\PYGZdq{}ir.ui.view\PYGZdq{}} \PYG{n+na}{id=}\PYG{l+s}{\PYGZdq{}course\PYGZus{}search\PYGZus{}view\PYGZdq{}}\PYG{n+nt}{\PYGZgt{}}
            \PYG{n+nt}{\PYGZlt{}field} \PYG{n+na}{name=}\PYG{l+s}{\PYGZdq{}name\PYGZdq{}}\PYG{n+nt}{\PYGZgt{}}course.search\PYG{n+nt}{\PYGZlt{}/field\PYGZgt{}}
            \PYG{n+nt}{\PYGZlt{}field} \PYG{n+na}{name=}\PYG{l+s}{\PYGZdq{}model\PYGZdq{}}\PYG{n+nt}{\PYGZgt{}}openacademy.course\PYG{n+nt}{\PYGZlt{}/field\PYGZgt{}}
            \PYG{n+nt}{\PYGZlt{}field} \PYG{n+na}{name=}\PYG{l+s}{\PYGZdq{}arch\PYGZdq{}} \PYG{n+na}{type=}\PYG{l+s}{\PYGZdq{}xml\PYGZdq{}}\PYG{n+nt}{\PYGZgt{}}
                \PYG{n+nt}{\PYGZlt{}search}\PYG{n+nt}{\PYGZgt{}}
                    \PYG{n+nt}{\PYGZlt{}field} \PYG{n+na}{name=}\PYG{l+s}{\PYGZdq{}name\PYGZdq{}}\PYG{n+nt}{/\PYGZgt{}}
                    \PYG{n+nt}{\PYGZlt{}field} \PYG{n+na}{name=}\PYG{l+s}{\PYGZdq{}description\PYGZdq{}}\PYG{n+nt}{/\PYGZgt{}}
                \PYG{n+nt}{\PYGZlt{}/search\PYGZgt{}}
            \PYG{n+nt}{\PYGZlt{}/field\PYGZgt{}}
        \PYG{n+nt}{\PYGZlt{}/record\PYGZgt{}}

        \PYG{c}{\PYGZlt{}!\PYGZhy{}\PYGZhy{}}\PYG{c}{ window action }\PYG{c}{\PYGZhy{}\PYGZhy{}\PYGZgt{}}
        \PYG{c}{\PYGZlt{}!\PYGZhy{}\PYGZhy{}}
\PYG{c}{            The following tag is an action definition for a \PYGZdq{}window action\PYGZdq{},}
\end{sphinxVerbatim}
\end{sphinxadmonition}


\subsection{Relations between models}
\label{\detokenize{howtos/backend:relations-between-models}}
A record from a model may be related to a record from another model. For
instance, a sale order record is related to a client record that contains the
client data; it is also related to its sale order line records.

\begin{sphinxadmonition}{note}
Create a session model

For the module Open Academy, we consider a model for \sphinxstyleemphasis{sessions}: a session
is an occurrence of a course taught at a given time for a given audience.

Create a model for \sphinxstyleemphasis{sessions}. A session has a name, a start date, a
duration and a number of seats. Add an action and a menu item to display
them. Make the new model visible via a menu item.
\begin{enumerate}
\item {} 
Create the class \sphinxstyleemphasis{Session} in \sphinxcode{\sphinxupquote{openacademy/models/models.py}}.

\item {} 
Add access to the session object in \sphinxcode{\sphinxupquote{openacademy/view/openacademy.xml}}.

\end{enumerate}
\sphinxstyleemphasis{openacademy/models.py}
\fvset{hllines={, 3, 4, 5, 6, 7, 8, 9, 10, 11,}}%
\begin{sphinxVerbatim}[commandchars=\\\{\}]

    \PYG{n}{name} \PYG{o}{=} \PYG{n}{fields}\PYG{o}{.}\PYG{n}{Char}\PYG{p}{(}\PYG{n}{string}\PYG{o}{=}\PYG{l+s+s2}{\PYGZdq{}}\PYG{l+s+s2}{Title}\PYG{l+s+s2}{\PYGZdq{}}\PYG{p}{,} \PYG{n}{required}\PYG{o}{=}\PYG{n+nb+bp}{True}\PYG{p}{)}
    \PYG{n}{description} \PYG{o}{=} \PYG{n}{fields}\PYG{o}{.}\PYG{n}{Text}\PYG{p}{(}\PYG{p}{)}


\PYG{k}{class} \PYG{n+nc}{Session}\PYG{p}{(}\PYG{n}{models}\PYG{o}{.}\PYG{n}{Model}\PYG{p}{)}\PYG{p}{:}
    \PYG{n}{\PYGZus{}name} \PYG{o}{=} \PYG{l+s+s1}{\PYGZsq{}}\PYG{l+s+s1}{openacademy.session}\PYG{l+s+s1}{\PYGZsq{}}

    \PYG{n}{name} \PYG{o}{=} \PYG{n}{fields}\PYG{o}{.}\PYG{n}{Char}\PYG{p}{(}\PYG{n}{required}\PYG{o}{=}\PYG{n+nb+bp}{True}\PYG{p}{)}
    \PYG{n}{start\PYGZus{}date} \PYG{o}{=} \PYG{n}{fields}\PYG{o}{.}\PYG{n}{Date}\PYG{p}{(}\PYG{p}{)}
    \PYG{n}{duration} \PYG{o}{=} \PYG{n}{fields}\PYG{o}{.}\PYG{n}{Float}\PYG{p}{(}\PYG{n}{digits}\PYG{o}{=}\PYG{p}{(}\PYG{l+m+mi}{6}\PYG{p}{,} \PYG{l+m+mi}{2}\PYG{p}{)}\PYG{p}{,} \PYG{n}{help}\PYG{o}{=}\PYG{l+s+s2}{\PYGZdq{}}\PYG{l+s+s2}{Duration in days}\PYG{l+s+s2}{\PYGZdq{}}\PYG{p}{)}
    \PYG{n}{seats} \PYG{o}{=} \PYG{n}{fields}\PYG{o}{.}\PYG{n}{Integer}\PYG{p}{(}\PYG{n}{string}\PYG{o}{=}\PYG{l+s+s2}{\PYGZdq{}}\PYG{l+s+s2}{Number of seats}\PYG{l+s+s2}{\PYGZdq{}}\PYG{p}{)}
\end{sphinxVerbatim}
\sphinxstyleemphasis{openacademy/views/openacademy.xml}
\fvset{hllines={, 4, 5, 6, 7, 8, 9, 10, 11, 12, 13, 14, 15, 16, 17, 18, 19, 20, 21, 22, 23, 24, 25, 26, 27, 28, 29, 30, 31, 32,}}%
\begin{sphinxVerbatim}[commandchars=\\\{\}]
        \PYG{c}{\PYGZlt{}!\PYGZhy{}\PYGZhy{}}\PYG{c}{ Full id location:}
\PYG{c}{             action=\PYGZdq{}openacademy.course\PYGZus{}list\PYGZus{}action\PYGZdq{}}
\PYG{c}{             It is not required when it is the same module }\PYG{c}{\PYGZhy{}\PYGZhy{}\PYGZgt{}}

        \PYG{c}{\PYGZlt{}!\PYGZhy{}\PYGZhy{}}\PYG{c}{ session form view }\PYG{c}{\PYGZhy{}\PYGZhy{}\PYGZgt{}}
        \PYG{n+nt}{\PYGZlt{}record} \PYG{n+na}{model=}\PYG{l+s}{\PYGZdq{}ir.ui.view\PYGZdq{}} \PYG{n+na}{id=}\PYG{l+s}{\PYGZdq{}session\PYGZus{}form\PYGZus{}view\PYGZdq{}}\PYG{n+nt}{\PYGZgt{}}
            \PYG{n+nt}{\PYGZlt{}field} \PYG{n+na}{name=}\PYG{l+s}{\PYGZdq{}name\PYGZdq{}}\PYG{n+nt}{\PYGZgt{}}session.form\PYG{n+nt}{\PYGZlt{}/field\PYGZgt{}}
            \PYG{n+nt}{\PYGZlt{}field} \PYG{n+na}{name=}\PYG{l+s}{\PYGZdq{}model\PYGZdq{}}\PYG{n+nt}{\PYGZgt{}}openacademy.session\PYG{n+nt}{\PYGZlt{}/field\PYGZgt{}}
            \PYG{n+nt}{\PYGZlt{}field} \PYG{n+na}{name=}\PYG{l+s}{\PYGZdq{}arch\PYGZdq{}} \PYG{n+na}{type=}\PYG{l+s}{\PYGZdq{}xml\PYGZdq{}}\PYG{n+nt}{\PYGZgt{}}
                \PYG{n+nt}{\PYGZlt{}form} \PYG{n+na}{string=}\PYG{l+s}{\PYGZdq{}Session Form\PYGZdq{}}\PYG{n+nt}{\PYGZgt{}}
                    \PYG{n+nt}{\PYGZlt{}sheet}\PYG{n+nt}{\PYGZgt{}}
                        \PYG{n+nt}{\PYGZlt{}group}\PYG{n+nt}{\PYGZgt{}}
                            \PYG{n+nt}{\PYGZlt{}field} \PYG{n+na}{name=}\PYG{l+s}{\PYGZdq{}name\PYGZdq{}}\PYG{n+nt}{/\PYGZgt{}}
                            \PYG{n+nt}{\PYGZlt{}field} \PYG{n+na}{name=}\PYG{l+s}{\PYGZdq{}start\PYGZus{}date\PYGZdq{}}\PYG{n+nt}{/\PYGZgt{}}
                            \PYG{n+nt}{\PYGZlt{}field} \PYG{n+na}{name=}\PYG{l+s}{\PYGZdq{}duration\PYGZdq{}}\PYG{n+nt}{/\PYGZgt{}}
                            \PYG{n+nt}{\PYGZlt{}field} \PYG{n+na}{name=}\PYG{l+s}{\PYGZdq{}seats\PYGZdq{}}\PYG{n+nt}{/\PYGZgt{}}
                        \PYG{n+nt}{\PYGZlt{}/group\PYGZgt{}}
                    \PYG{n+nt}{\PYGZlt{}/sheet\PYGZgt{}}
                \PYG{n+nt}{\PYGZlt{}/form\PYGZgt{}}
            \PYG{n+nt}{\PYGZlt{}/field\PYGZgt{}}
        \PYG{n+nt}{\PYGZlt{}/record\PYGZgt{}}

        \PYG{n+nt}{\PYGZlt{}record} \PYG{n+na}{model=}\PYG{l+s}{\PYGZdq{}ir.actions.act\PYGZus{}window\PYGZdq{}} \PYG{n+na}{id=}\PYG{l+s}{\PYGZdq{}session\PYGZus{}list\PYGZus{}action\PYGZdq{}}\PYG{n+nt}{\PYGZgt{}}
            \PYG{n+nt}{\PYGZlt{}field} \PYG{n+na}{name=}\PYG{l+s}{\PYGZdq{}name\PYGZdq{}}\PYG{n+nt}{\PYGZgt{}}Sessions\PYG{n+nt}{\PYGZlt{}/field\PYGZgt{}}
            \PYG{n+nt}{\PYGZlt{}field} \PYG{n+na}{name=}\PYG{l+s}{\PYGZdq{}res\PYGZus{}model\PYGZdq{}}\PYG{n+nt}{\PYGZgt{}}openacademy.session\PYG{n+nt}{\PYGZlt{}/field\PYGZgt{}}
            \PYG{n+nt}{\PYGZlt{}field} \PYG{n+na}{name=}\PYG{l+s}{\PYGZdq{}view\PYGZus{}type\PYGZdq{}}\PYG{n+nt}{\PYGZgt{}}form\PYG{n+nt}{\PYGZlt{}/field\PYGZgt{}}
            \PYG{n+nt}{\PYGZlt{}field} \PYG{n+na}{name=}\PYG{l+s}{\PYGZdq{}view\PYGZus{}mode\PYGZdq{}}\PYG{n+nt}{\PYGZgt{}}tree,form\PYG{n+nt}{\PYGZlt{}/field\PYGZgt{}}
        \PYG{n+nt}{\PYGZlt{}/record\PYGZgt{}}

        \PYG{n+nt}{\PYGZlt{}menuitem} \PYG{n+na}{id=}\PYG{l+s}{\PYGZdq{}session\PYGZus{}menu\PYGZdq{}} \PYG{n+na}{name=}\PYG{l+s}{\PYGZdq{}Sessions\PYGZdq{}}
                  \PYG{n+na}{parent=}\PYG{l+s}{\PYGZdq{}openacademy\PYGZus{}menu\PYGZdq{}}
                  \PYG{n+na}{action=}\PYG{l+s}{\PYGZdq{}session\PYGZus{}list\PYGZus{}action\PYGZdq{}}\PYG{n+nt}{/\PYGZgt{}}

\PYG{n+nt}{\PYGZlt{}/odoo\PYGZgt{}}
\end{sphinxVerbatim}

\begin{sphinxadmonition}{note}{Note:}
\sphinxcode{\sphinxupquote{digits=(6, 2)}} specifies the precision of a float number:
6 is the total number of digits, while 2 is the number of
digits after the comma. Note that it results in the number
digits before the comma is a maximum 4
\end{sphinxadmonition}
\end{sphinxadmonition}


\subsubsection{Relational fields}
\label{\detokenize{howtos/backend:relational-fields}}
Relational fields link records, either of the same model (hierarchies) or
between different models.

Relational field types are:
\begin{description}
\item[{{\hyperref[\detokenize{reference/orm:odoo.fields.Many2one}]{\sphinxcrossref{\sphinxcode{\sphinxupquote{Many2one(other\_model, ondelete='set null')}}}}}}] \leavevmode
A simple link to an other object:

\fvset{hllines={, ,}}%
\begin{sphinxVerbatim}[commandchars=\\\{\}]
\PYG{n+nb}{print} \PYG{n}{foo}\PYG{o}{.}\PYG{n}{other\PYGZus{}id}\PYG{o}{.}\PYG{n}{name}
\end{sphinxVerbatim}


\sphinxstrong{See also:}


\sphinxhref{http://www.postgresql.org/docs/9.3/static/tutorial-fk.html}{foreign keys}



\item[{{\hyperref[\detokenize{reference/orm:odoo.fields.One2many}]{\sphinxcrossref{\sphinxcode{\sphinxupquote{One2many(other\_model, related\_field)}}}}}}] \leavevmode
A virtual relationship, inverse of a {\hyperref[\detokenize{reference/orm:odoo.fields.Many2one}]{\sphinxcrossref{\sphinxcode{\sphinxupquote{Many2one}}}}}.
A {\hyperref[\detokenize{reference/orm:odoo.fields.One2many}]{\sphinxcrossref{\sphinxcode{\sphinxupquote{One2many}}}}} behaves as a container of records,
accessing it results in a (possibly empty) set of records:

\fvset{hllines={, ,}}%
\begin{sphinxVerbatim}[commandchars=\\\{\}]
\PYG{k}{for} \PYG{n}{other} \PYG{o+ow}{in} \PYG{n}{foo}\PYG{o}{.}\PYG{n}{other\PYGZus{}ids}\PYG{p}{:}
    \PYG{n+nb}{print} \PYG{n}{other}\PYG{o}{.}\PYG{n}{name}
\end{sphinxVerbatim}

\begin{sphinxadmonition}{danger}{Danger:}
Because a {\hyperref[\detokenize{reference/orm:odoo.fields.One2many}]{\sphinxcrossref{\sphinxcode{\sphinxupquote{One2many}}}}} is a virtual relationship,
there \sphinxstyleemphasis{must} be a {\hyperref[\detokenize{reference/orm:odoo.fields.Many2one}]{\sphinxcrossref{\sphinxcode{\sphinxupquote{Many2one}}}}} field in the
\sphinxcode{\sphinxupquote{\sphinxstyleemphasis{other\_model}}}, and its name \sphinxstyleemphasis{must} be \sphinxcode{\sphinxupquote{\sphinxstyleemphasis{related\_field}}}
\end{sphinxadmonition}

\item[{{\hyperref[\detokenize{reference/orm:odoo.fields.Many2many}]{\sphinxcrossref{\sphinxcode{\sphinxupquote{Many2many(other\_model)}}}}}}] \leavevmode
Bidirectional multiple relationship, any record on one side can be related
to any number of records on the other side. Behaves as a container of
records, accessing it also results in a possibly empty set of records:

\fvset{hllines={, ,}}%
\begin{sphinxVerbatim}[commandchars=\\\{\}]
\PYG{k}{for} \PYG{n}{other} \PYG{o+ow}{in} \PYG{n}{foo}\PYG{o}{.}\PYG{n}{other\PYGZus{}ids}\PYG{p}{:}
    \PYG{n+nb}{print} \PYG{n}{other}\PYG{o}{.}\PYG{n}{name}
\end{sphinxVerbatim}

\end{description}

\begin{sphinxadmonition}{note}
Many2one relations

Using a many2one, modify the \sphinxstyleemphasis{Course} and \sphinxstyleemphasis{Session} models to reflect their
relation with other models:
\begin{itemize}
\item {} 
A course has a \sphinxstyleemphasis{responsible} user; the value of that field is a record of
the built-in model \sphinxcode{\sphinxupquote{res.users}}.

\item {} 
A session has an \sphinxstyleemphasis{instructor}; the value of that field is a record of the
built-in model \sphinxcode{\sphinxupquote{res.partner}}.

\item {} 
A session is related to a \sphinxstyleemphasis{course}; the value of that field is a record
of the model \sphinxcode{\sphinxupquote{openacademy.course}} and is required.

\item {} 
Adapt the views.

\end{itemize}
\begin{enumerate}
\item {} 
Add the relevant \sphinxcode{\sphinxupquote{Many2one}} fields to the models, and

\item {} 
add them in the views.

\end{enumerate}
\sphinxstyleemphasis{openacademy/models.py}
\fvset{hllines={, 4, 5, 6,}}%
\begin{sphinxVerbatim}[commandchars=\\\{\}]
    \PYG{n}{name} \PYG{o}{=} \PYG{n}{fields}\PYG{o}{.}\PYG{n}{Char}\PYG{p}{(}\PYG{n}{string}\PYG{o}{=}\PYG{l+s+s2}{\PYGZdq{}}\PYG{l+s+s2}{Title}\PYG{l+s+s2}{\PYGZdq{}}\PYG{p}{,} \PYG{n}{required}\PYG{o}{=}\PYG{n+nb+bp}{True}\PYG{p}{)}
    \PYG{n}{description} \PYG{o}{=} \PYG{n}{fields}\PYG{o}{.}\PYG{n}{Text}\PYG{p}{(}\PYG{p}{)}

    \PYG{n}{responsible\PYGZus{}id} \PYG{o}{=} \PYG{n}{fields}\PYG{o}{.}\PYG{n}{Many2one}\PYG{p}{(}\PYG{l+s+s1}{\PYGZsq{}}\PYG{l+s+s1}{res.users}\PYG{l+s+s1}{\PYGZsq{}}\PYG{p}{,}
        \PYG{n}{ondelete}\PYG{o}{=}\PYG{l+s+s1}{\PYGZsq{}}\PYG{l+s+s1}{set null}\PYG{l+s+s1}{\PYGZsq{}}\PYG{p}{,} \PYG{n}{string}\PYG{o}{=}\PYG{l+s+s2}{\PYGZdq{}}\PYG{l+s+s2}{Responsible}\PYG{l+s+s2}{\PYGZdq{}}\PYG{p}{,} \PYG{n}{index}\PYG{o}{=}\PYG{n+nb+bp}{True}\PYG{p}{)}


\PYG{k}{class} \PYG{n+nc}{Session}\PYG{p}{(}\PYG{n}{models}\PYG{o}{.}\PYG{n}{Model}\PYG{p}{)}\PYG{p}{:}
    \PYG{n}{\PYGZus{}name} \PYG{o}{=} \PYG{l+s+s1}{\PYGZsq{}}\PYG{l+s+s1}{openacademy.session}\PYG{l+s+s1}{\PYGZsq{}}
\end{sphinxVerbatim}

\fvset{hllines={, 4, 5, 6, 7,}}%
\begin{sphinxVerbatim}[commandchars=\\\{\}]
    \PYG{n}{start\PYGZus{}date} \PYG{o}{=} \PYG{n}{fields}\PYG{o}{.}\PYG{n}{Date}\PYG{p}{(}\PYG{p}{)}
    \PYG{n}{duration} \PYG{o}{=} \PYG{n}{fields}\PYG{o}{.}\PYG{n}{Float}\PYG{p}{(}\PYG{n}{digits}\PYG{o}{=}\PYG{p}{(}\PYG{l+m+mi}{6}\PYG{p}{,} \PYG{l+m+mi}{2}\PYG{p}{)}\PYG{p}{,} \PYG{n}{help}\PYG{o}{=}\PYG{l+s+s2}{\PYGZdq{}}\PYG{l+s+s2}{Duration in days}\PYG{l+s+s2}{\PYGZdq{}}\PYG{p}{)}
    \PYG{n}{seats} \PYG{o}{=} \PYG{n}{fields}\PYG{o}{.}\PYG{n}{Integer}\PYG{p}{(}\PYG{n}{string}\PYG{o}{=}\PYG{l+s+s2}{\PYGZdq{}}\PYG{l+s+s2}{Number of seats}\PYG{l+s+s2}{\PYGZdq{}}\PYG{p}{)}

    \PYG{n}{instructor\PYGZus{}id} \PYG{o}{=} \PYG{n}{fields}\PYG{o}{.}\PYG{n}{Many2one}\PYG{p}{(}\PYG{l+s+s1}{\PYGZsq{}}\PYG{l+s+s1}{res.partner}\PYG{l+s+s1}{\PYGZsq{}}\PYG{p}{,} \PYG{n}{string}\PYG{o}{=}\PYG{l+s+s2}{\PYGZdq{}}\PYG{l+s+s2}{Instructor}\PYG{l+s+s2}{\PYGZdq{}}\PYG{p}{)}
    \PYG{n}{course\PYGZus{}id} \PYG{o}{=} \PYG{n}{fields}\PYG{o}{.}\PYG{n}{Many2one}\PYG{p}{(}\PYG{l+s+s1}{\PYGZsq{}}\PYG{l+s+s1}{openacademy.course}\PYG{l+s+s1}{\PYGZsq{}}\PYG{p}{,}
        \PYG{n}{ondelete}\PYG{o}{=}\PYG{l+s+s1}{\PYGZsq{}}\PYG{l+s+s1}{cascade}\PYG{l+s+s1}{\PYGZsq{}}\PYG{p}{,} \PYG{n}{string}\PYG{o}{=}\PYG{l+s+s2}{\PYGZdq{}}\PYG{l+s+s2}{Course}\PYG{l+s+s2}{\PYGZdq{}}\PYG{p}{,} \PYG{n}{required}\PYG{o}{=}\PYG{n+nb+bp}{True}\PYG{p}{)}
\end{sphinxVerbatim}
\sphinxstyleemphasis{openacademy/views/openacademy.xml}
\fvset{hllines={, 4,}}%
\begin{sphinxVerbatim}[commandchars=\\\{\}]
                    \PYG{n+nt}{\PYGZlt{}sheet}\PYG{n+nt}{\PYGZgt{}}
                        \PYG{n+nt}{\PYGZlt{}group}\PYG{n+nt}{\PYGZgt{}}
                            \PYG{n+nt}{\PYGZlt{}field} \PYG{n+na}{name=}\PYG{l+s}{\PYGZdq{}name\PYGZdq{}}\PYG{n+nt}{/\PYGZgt{}}
                            \PYG{n+nt}{\PYGZlt{}field} \PYG{n+na}{name=}\PYG{l+s}{\PYGZdq{}responsible\PYGZus{}id\PYGZdq{}}\PYG{n+nt}{/\PYGZgt{}}
                        \PYG{n+nt}{\PYGZlt{}/group\PYGZgt{}}
                        \PYG{n+nt}{\PYGZlt{}notebook}\PYG{n+nt}{\PYGZgt{}}
                            \PYG{n+nt}{\PYGZlt{}page} \PYG{n+na}{string=}\PYG{l+s}{\PYGZdq{}Description\PYGZdq{}}\PYG{n+nt}{\PYGZgt{}}
\end{sphinxVerbatim}

\fvset{hllines={, 4, 5, 6, 7, 8, 9, 10, 11, 12, 13, 14, 15,}}%
\begin{sphinxVerbatim}[commandchars=\\\{\}]
            \PYG{n+nt}{\PYGZlt{}/field\PYGZgt{}}
        \PYG{n+nt}{\PYGZlt{}/record\PYGZgt{}}

        \PYG{c}{\PYGZlt{}!\PYGZhy{}\PYGZhy{}}\PYG{c}{ override the automatically generated list view for courses }\PYG{c}{\PYGZhy{}\PYGZhy{}\PYGZgt{}}
        \PYG{n+nt}{\PYGZlt{}record} \PYG{n+na}{model=}\PYG{l+s}{\PYGZdq{}ir.ui.view\PYGZdq{}} \PYG{n+na}{id=}\PYG{l+s}{\PYGZdq{}course\PYGZus{}tree\PYGZus{}view\PYGZdq{}}\PYG{n+nt}{\PYGZgt{}}
            \PYG{n+nt}{\PYGZlt{}field} \PYG{n+na}{name=}\PYG{l+s}{\PYGZdq{}name\PYGZdq{}}\PYG{n+nt}{\PYGZgt{}}course.tree\PYG{n+nt}{\PYGZlt{}/field\PYGZgt{}}
            \PYG{n+nt}{\PYGZlt{}field} \PYG{n+na}{name=}\PYG{l+s}{\PYGZdq{}model\PYGZdq{}}\PYG{n+nt}{\PYGZgt{}}openacademy.course\PYG{n+nt}{\PYGZlt{}/field\PYGZgt{}}
            \PYG{n+nt}{\PYGZlt{}field} \PYG{n+na}{name=}\PYG{l+s}{\PYGZdq{}arch\PYGZdq{}} \PYG{n+na}{type=}\PYG{l+s}{\PYGZdq{}xml\PYGZdq{}}\PYG{n+nt}{\PYGZgt{}}
                \PYG{n+nt}{\PYGZlt{}tree} \PYG{n+na}{string=}\PYG{l+s}{\PYGZdq{}Course Tree\PYGZdq{}}\PYG{n+nt}{\PYGZgt{}}
                    \PYG{n+nt}{\PYGZlt{}field} \PYG{n+na}{name=}\PYG{l+s}{\PYGZdq{}name\PYGZdq{}}\PYG{n+nt}{/\PYGZgt{}}
                    \PYG{n+nt}{\PYGZlt{}field} \PYG{n+na}{name=}\PYG{l+s}{\PYGZdq{}responsible\PYGZus{}id\PYGZdq{}}\PYG{n+nt}{/\PYGZgt{}}
                \PYG{n+nt}{\PYGZlt{}/tree\PYGZgt{}}
            \PYG{n+nt}{\PYGZlt{}/field\PYGZgt{}}
        \PYG{n+nt}{\PYGZlt{}/record\PYGZgt{}}

        \PYG{c}{\PYGZlt{}!\PYGZhy{}\PYGZhy{}}\PYG{c}{ window action }\PYG{c}{\PYGZhy{}\PYGZhy{}\PYGZgt{}}
        \PYG{c}{\PYGZlt{}!\PYGZhy{}\PYGZhy{}}
\PYG{c}{            The following tag is an action definition for a \PYGZdq{}window action\PYGZdq{},}
\end{sphinxVerbatim}

\fvset{hllines={, 4, 5, 6, 7, 8, 9, 10, 11, 12, 13, 20, 21, 22, 23, 24, 25, 26, 27, 28, 29, 30, 31,}}%
\begin{sphinxVerbatim}[commandchars=\\\{\}]
                \PYG{n+nt}{\PYGZlt{}form} \PYG{n+na}{string=}\PYG{l+s}{\PYGZdq{}Session Form\PYGZdq{}}\PYG{n+nt}{\PYGZgt{}}
                    \PYG{n+nt}{\PYGZlt{}sheet}\PYG{n+nt}{\PYGZgt{}}
                        \PYG{n+nt}{\PYGZlt{}group}\PYG{n+nt}{\PYGZgt{}}
                            \PYG{n+nt}{\PYGZlt{}group} \PYG{n+na}{string=}\PYG{l+s}{\PYGZdq{}General\PYGZdq{}}\PYG{n+nt}{\PYGZgt{}}
                                \PYG{n+nt}{\PYGZlt{}field} \PYG{n+na}{name=}\PYG{l+s}{\PYGZdq{}course\PYGZus{}id\PYGZdq{}}\PYG{n+nt}{/\PYGZgt{}}
                                \PYG{n+nt}{\PYGZlt{}field} \PYG{n+na}{name=}\PYG{l+s}{\PYGZdq{}name\PYGZdq{}}\PYG{n+nt}{/\PYGZgt{}}
                                \PYG{n+nt}{\PYGZlt{}field} \PYG{n+na}{name=}\PYG{l+s}{\PYGZdq{}instructor\PYGZus{}id\PYGZdq{}}\PYG{n+nt}{/\PYGZgt{}}
                            \PYG{n+nt}{\PYGZlt{}/group\PYGZgt{}}
                            \PYG{n+nt}{\PYGZlt{}group} \PYG{n+na}{string=}\PYG{l+s}{\PYGZdq{}Schedule\PYGZdq{}}\PYG{n+nt}{\PYGZgt{}}
                                \PYG{n+nt}{\PYGZlt{}field} \PYG{n+na}{name=}\PYG{l+s}{\PYGZdq{}start\PYGZus{}date\PYGZdq{}}\PYG{n+nt}{/\PYGZgt{}}
                                \PYG{n+nt}{\PYGZlt{}field} \PYG{n+na}{name=}\PYG{l+s}{\PYGZdq{}duration\PYGZdq{}}\PYG{n+nt}{/\PYGZgt{}}
                                \PYG{n+nt}{\PYGZlt{}field} \PYG{n+na}{name=}\PYG{l+s}{\PYGZdq{}seats\PYGZdq{}}\PYG{n+nt}{/\PYGZgt{}}
                            \PYG{n+nt}{\PYGZlt{}/group\PYGZgt{}}
                        \PYG{n+nt}{\PYGZlt{}/group\PYGZgt{}}
                    \PYG{n+nt}{\PYGZlt{}/sheet\PYGZgt{}}
                \PYG{n+nt}{\PYGZlt{}/form\PYGZgt{}}
            \PYG{n+nt}{\PYGZlt{}/field\PYGZgt{}}
        \PYG{n+nt}{\PYGZlt{}/record\PYGZgt{}}

        \PYG{c}{\PYGZlt{}!\PYGZhy{}\PYGZhy{}}\PYG{c}{ session tree/list view }\PYG{c}{\PYGZhy{}\PYGZhy{}\PYGZgt{}}
        \PYG{n+nt}{\PYGZlt{}record} \PYG{n+na}{model=}\PYG{l+s}{\PYGZdq{}ir.ui.view\PYGZdq{}} \PYG{n+na}{id=}\PYG{l+s}{\PYGZdq{}session\PYGZus{}tree\PYGZus{}view\PYGZdq{}}\PYG{n+nt}{\PYGZgt{}}
            \PYG{n+nt}{\PYGZlt{}field} \PYG{n+na}{name=}\PYG{l+s}{\PYGZdq{}name\PYGZdq{}}\PYG{n+nt}{\PYGZgt{}}session.tree\PYG{n+nt}{\PYGZlt{}/field\PYGZgt{}}
            \PYG{n+nt}{\PYGZlt{}field} \PYG{n+na}{name=}\PYG{l+s}{\PYGZdq{}model\PYGZdq{}}\PYG{n+nt}{\PYGZgt{}}openacademy.session\PYG{n+nt}{\PYGZlt{}/field\PYGZgt{}}
            \PYG{n+nt}{\PYGZlt{}field} \PYG{n+na}{name=}\PYG{l+s}{\PYGZdq{}arch\PYGZdq{}} \PYG{n+na}{type=}\PYG{l+s}{\PYGZdq{}xml\PYGZdq{}}\PYG{n+nt}{\PYGZgt{}}
                \PYG{n+nt}{\PYGZlt{}tree} \PYG{n+na}{string=}\PYG{l+s}{\PYGZdq{}Session Tree\PYGZdq{}}\PYG{n+nt}{\PYGZgt{}}
                    \PYG{n+nt}{\PYGZlt{}field} \PYG{n+na}{name=}\PYG{l+s}{\PYGZdq{}name\PYGZdq{}}\PYG{n+nt}{/\PYGZgt{}}
                    \PYG{n+nt}{\PYGZlt{}field} \PYG{n+na}{name=}\PYG{l+s}{\PYGZdq{}course\PYGZus{}id\PYGZdq{}}\PYG{n+nt}{/\PYGZgt{}}
                \PYG{n+nt}{\PYGZlt{}/tree\PYGZgt{}}
            \PYG{n+nt}{\PYGZlt{}/field\PYGZgt{}}
        \PYG{n+nt}{\PYGZlt{}/record\PYGZgt{}}

        \PYG{n+nt}{\PYGZlt{}record} \PYG{n+na}{model=}\PYG{l+s}{\PYGZdq{}ir.actions.act\PYGZus{}window\PYGZdq{}} \PYG{n+na}{id=}\PYG{l+s}{\PYGZdq{}session\PYGZus{}list\PYGZus{}action\PYGZdq{}}\PYG{n+nt}{\PYGZgt{}}
            \PYG{n+nt}{\PYGZlt{}field} \PYG{n+na}{name=}\PYG{l+s}{\PYGZdq{}name\PYGZdq{}}\PYG{n+nt}{\PYGZgt{}}Sessions\PYG{n+nt}{\PYGZlt{}/field\PYGZgt{}}
            \PYG{n+nt}{\PYGZlt{}field} \PYG{n+na}{name=}\PYG{l+s}{\PYGZdq{}res\PYGZus{}model\PYGZdq{}}\PYG{n+nt}{\PYGZgt{}}openacademy.session\PYG{n+nt}{\PYGZlt{}/field\PYGZgt{}}
\end{sphinxVerbatim}
\end{sphinxadmonition}

\begin{sphinxadmonition}{note}
Inverse one2many relations

Using the inverse relational field one2many, modify the models to reflect
the relation between courses and sessions.
\begin{enumerate}
\item {} 
Modify the \sphinxcode{\sphinxupquote{Course}} class, and

\item {} 
add the field in the course form view.

\end{enumerate}
\sphinxstyleemphasis{openacademy/models.py}
\fvset{hllines={, 3, 4,}}%
\begin{sphinxVerbatim}[commandchars=\\\{\}]

    \PYG{n}{responsible\PYGZus{}id} \PYG{o}{=} \PYG{n}{fields}\PYG{o}{.}\PYG{n}{Many2one}\PYG{p}{(}\PYG{l+s+s1}{\PYGZsq{}}\PYG{l+s+s1}{res.users}\PYG{l+s+s1}{\PYGZsq{}}\PYG{p}{,}
        \PYG{n}{ondelete}\PYG{o}{=}\PYG{l+s+s1}{\PYGZsq{}}\PYG{l+s+s1}{set null}\PYG{l+s+s1}{\PYGZsq{}}\PYG{p}{,} \PYG{n}{string}\PYG{o}{=}\PYG{l+s+s2}{\PYGZdq{}}\PYG{l+s+s2}{Responsible}\PYG{l+s+s2}{\PYGZdq{}}\PYG{p}{,} \PYG{n}{index}\PYG{o}{=}\PYG{n+nb+bp}{True}\PYG{p}{)}
    \PYG{n}{session\PYGZus{}ids} \PYG{o}{=} \PYG{n}{fields}\PYG{o}{.}\PYG{n}{One2many}\PYG{p}{(}
        \PYG{l+s+s1}{\PYGZsq{}}\PYG{l+s+s1}{openacademy.session}\PYG{l+s+s1}{\PYGZsq{}}\PYG{p}{,} \PYG{l+s+s1}{\PYGZsq{}}\PYG{l+s+s1}{course\PYGZus{}id}\PYG{l+s+s1}{\PYGZsq{}}\PYG{p}{,} \PYG{n}{string}\PYG{o}{=}\PYG{l+s+s2}{\PYGZdq{}}\PYG{l+s+s2}{Sessions}\PYG{l+s+s2}{\PYGZdq{}}\PYG{p}{)}


\PYG{k}{class} \PYG{n+nc}{Session}\PYG{p}{(}\PYG{n}{models}\PYG{o}{.}\PYG{n}{Model}\PYG{p}{)}\PYG{p}{:}
\end{sphinxVerbatim}
\sphinxstyleemphasis{openacademy/views/openacademy.xml}
\fvset{hllines={, 4, 5, 6, 7, 8, 9, 10,}}%
\begin{sphinxVerbatim}[commandchars=\\\{\}]
                            \PYG{n+nt}{\PYGZlt{}page} \PYG{n+na}{string=}\PYG{l+s}{\PYGZdq{}Description\PYGZdq{}}\PYG{n+nt}{\PYGZgt{}}
                                \PYG{n+nt}{\PYGZlt{}field} \PYG{n+na}{name=}\PYG{l+s}{\PYGZdq{}description\PYGZdq{}}\PYG{n+nt}{/\PYGZgt{}}
                            \PYG{n+nt}{\PYGZlt{}/page\PYGZgt{}}
                            \PYG{n+nt}{\PYGZlt{}page} \PYG{n+na}{string=}\PYG{l+s}{\PYGZdq{}Sessions\PYGZdq{}}\PYG{n+nt}{\PYGZgt{}}
                                \PYG{n+nt}{\PYGZlt{}field} \PYG{n+na}{name=}\PYG{l+s}{\PYGZdq{}session\PYGZus{}ids\PYGZdq{}}\PYG{n+nt}{\PYGZgt{}}
                                    \PYG{n+nt}{\PYGZlt{}tree} \PYG{n+na}{string=}\PYG{l+s}{\PYGZdq{}Registered sessions\PYGZdq{}}\PYG{n+nt}{\PYGZgt{}}
                                        \PYG{n+nt}{\PYGZlt{}field} \PYG{n+na}{name=}\PYG{l+s}{\PYGZdq{}name\PYGZdq{}}\PYG{n+nt}{/\PYGZgt{}}
                                        \PYG{n+nt}{\PYGZlt{}field} \PYG{n+na}{name=}\PYG{l+s}{\PYGZdq{}instructor\PYGZus{}id\PYGZdq{}}\PYG{n+nt}{/\PYGZgt{}}
                                    \PYG{n+nt}{\PYGZlt{}/tree\PYGZgt{}}
                                \PYG{n+nt}{\PYGZlt{}/field\PYGZgt{}}
                            \PYG{n+nt}{\PYGZlt{}/page\PYGZgt{}}
                        \PYG{n+nt}{\PYGZlt{}/notebook\PYGZgt{}}
                    \PYG{n+nt}{\PYGZlt{}/sheet\PYGZgt{}}
\end{sphinxVerbatim}
\end{sphinxadmonition}

\begin{sphinxadmonition}{note}
Multiple many2many relations

Using the relational field many2many, modify the \sphinxstyleemphasis{Session} model to relate
every session to a set of \sphinxstyleemphasis{attendees}. Attendees will be represented by
partner records, so we will relate to the built-in model \sphinxcode{\sphinxupquote{res.partner}}.
Adapt the views accordingly.
\begin{enumerate}
\item {} 
Modify the \sphinxcode{\sphinxupquote{Session}} class, and

\item {} 
add the field in the form view.

\end{enumerate}
\sphinxstyleemphasis{openacademy/models.py}
\fvset{hllines={, 4,}}%
\begin{sphinxVerbatim}[commandchars=\\\{\}]
    \PYG{n}{instructor\PYGZus{}id} \PYG{o}{=} \PYG{n}{fields}\PYG{o}{.}\PYG{n}{Many2one}\PYG{p}{(}\PYG{l+s+s1}{\PYGZsq{}}\PYG{l+s+s1}{res.partner}\PYG{l+s+s1}{\PYGZsq{}}\PYG{p}{,} \PYG{n}{string}\PYG{o}{=}\PYG{l+s+s2}{\PYGZdq{}}\PYG{l+s+s2}{Instructor}\PYG{l+s+s2}{\PYGZdq{}}\PYG{p}{)}
    \PYG{n}{course\PYGZus{}id} \PYG{o}{=} \PYG{n}{fields}\PYG{o}{.}\PYG{n}{Many2one}\PYG{p}{(}\PYG{l+s+s1}{\PYGZsq{}}\PYG{l+s+s1}{openacademy.course}\PYG{l+s+s1}{\PYGZsq{}}\PYG{p}{,}
        \PYG{n}{ondelete}\PYG{o}{=}\PYG{l+s+s1}{\PYGZsq{}}\PYG{l+s+s1}{cascade}\PYG{l+s+s1}{\PYGZsq{}}\PYG{p}{,} \PYG{n}{string}\PYG{o}{=}\PYG{l+s+s2}{\PYGZdq{}}\PYG{l+s+s2}{Course}\PYG{l+s+s2}{\PYGZdq{}}\PYG{p}{,} \PYG{n}{required}\PYG{o}{=}\PYG{n+nb+bp}{True}\PYG{p}{)}
    \PYG{n}{attendee\PYGZus{}ids} \PYG{o}{=} \PYG{n}{fields}\PYG{o}{.}\PYG{n}{Many2many}\PYG{p}{(}\PYG{l+s+s1}{\PYGZsq{}}\PYG{l+s+s1}{res.partner}\PYG{l+s+s1}{\PYGZsq{}}\PYG{p}{,} \PYG{n}{string}\PYG{o}{=}\PYG{l+s+s2}{\PYGZdq{}}\PYG{l+s+s2}{Attendees}\PYG{l+s+s2}{\PYGZdq{}}\PYG{p}{)}
\end{sphinxVerbatim}
\sphinxstyleemphasis{openacademy/views/openacademy.xml}
\fvset{hllines={, 4, 5,}}%
\begin{sphinxVerbatim}[commandchars=\\\{\}]
                                \PYG{n+nt}{\PYGZlt{}field} \PYG{n+na}{name=}\PYG{l+s}{\PYGZdq{}seats\PYGZdq{}}\PYG{n+nt}{/\PYGZgt{}}
                            \PYG{n+nt}{\PYGZlt{}/group\PYGZgt{}}
                        \PYG{n+nt}{\PYGZlt{}/group\PYGZgt{}}
                        \PYG{n+nt}{\PYGZlt{}label} \PYG{n+na}{for=}\PYG{l+s}{\PYGZdq{}attendee\PYGZus{}ids\PYGZdq{}}\PYG{n+nt}{/\PYGZgt{}}
                        \PYG{n+nt}{\PYGZlt{}field} \PYG{n+na}{name=}\PYG{l+s}{\PYGZdq{}attendee\PYGZus{}ids\PYGZdq{}}\PYG{n+nt}{/\PYGZgt{}}
                    \PYG{n+nt}{\PYGZlt{}/sheet\PYGZgt{}}
                \PYG{n+nt}{\PYGZlt{}/form\PYGZgt{}}
            \PYG{n+nt}{\PYGZlt{}/field\PYGZgt{}}
\end{sphinxVerbatim}
\end{sphinxadmonition}


\subsection{Inheritance}
\label{\detokenize{howtos/backend:inheritance}}

\subsubsection{Model inheritance}
\label{\detokenize{howtos/backend:model-inheritance}}
Odoo provides two \sphinxstyleemphasis{inheritance} mechanisms to extend an existing model in a
modular way.

The first inheritance mechanism allows a module to modify the behavior of a
model defined in another module:
\begin{itemize}
\item {} 
add fields to a model,

\item {} 
override the definition of fields on a model,

\item {} 
add constraints to a model,

\item {} 
add methods to a model,

\item {} 
override existing methods on a model.

\end{itemize}

The second inheritance mechanism (delegation) allows to link every record of a
model to a record in a parent model, and provides transparent access to the
fields of the parent record.

\noindent{\hspace*{\fill}\sphinxincludegraphics{{inheritance_methods}.png}\hspace*{\fill}}


\sphinxstrong{See also:}

\begin{itemize}
\item {} 
{\hyperref[\detokenize{reference/orm:odoo.models.Model._inherit}]{\sphinxcrossref{\sphinxcode{\sphinxupquote{\_inherit}}}}}

\item {} 
{\hyperref[\detokenize{reference/orm:odoo.models.Model._inherits}]{\sphinxcrossref{\sphinxcode{\sphinxupquote{\_inherits}}}}}

\end{itemize}




\subsubsection{View inheritance}
\label{\detokenize{howtos/backend:view-inheritance}}
Instead of modifying existing views in place (by overwriting them), Odoo
provides view inheritance where children “extension” views are applied on top of
root views, and can add or remove content from their parent.

An extension view references its parent using the \sphinxcode{\sphinxupquote{inherit\_id}} field, and
instead of a single view its \sphinxcode{\sphinxupquote{arch}} field is composed of any number of
\sphinxcode{\sphinxupquote{xpath}} elements selecting and altering the content of their parent view:

\fvset{hllines={, ,}}%
\begin{sphinxVerbatim}[commandchars=\\\{\}]
\PYG{c}{\PYGZlt{}!\PYGZhy{}\PYGZhy{}}\PYG{c}{ improved idea categories list }\PYG{c}{\PYGZhy{}\PYGZhy{}\PYGZgt{}}
\PYG{n+nt}{\PYGZlt{}record} \PYG{n+na}{id=}\PYG{l+s}{\PYGZdq{}idea\PYGZus{}category\PYGZus{}list2\PYGZdq{}} \PYG{n+na}{model=}\PYG{l+s}{\PYGZdq{}ir.ui.view\PYGZdq{}}\PYG{n+nt}{\PYGZgt{}}
    \PYG{n+nt}{\PYGZlt{}field} \PYG{n+na}{name=}\PYG{l+s}{\PYGZdq{}name\PYGZdq{}}\PYG{n+nt}{\PYGZgt{}}id.category.list2\PYG{n+nt}{\PYGZlt{}/field\PYGZgt{}}
    \PYG{n+nt}{\PYGZlt{}field} \PYG{n+na}{name=}\PYG{l+s}{\PYGZdq{}model\PYGZdq{}}\PYG{n+nt}{\PYGZgt{}}idea.category\PYG{n+nt}{\PYGZlt{}/field\PYGZgt{}}
    \PYG{n+nt}{\PYGZlt{}field} \PYG{n+na}{name=}\PYG{l+s}{\PYGZdq{}inherit\PYGZus{}id\PYGZdq{}} \PYG{n+na}{ref=}\PYG{l+s}{\PYGZdq{}id\PYGZus{}category\PYGZus{}list\PYGZdq{}}\PYG{n+nt}{/\PYGZgt{}}
    \PYG{n+nt}{\PYGZlt{}field} \PYG{n+na}{name=}\PYG{l+s}{\PYGZdq{}arch\PYGZdq{}} \PYG{n+na}{type=}\PYG{l+s}{\PYGZdq{}xml\PYGZdq{}}\PYG{n+nt}{\PYGZgt{}}
        \PYG{c}{\PYGZlt{}!\PYGZhy{}\PYGZhy{}}\PYG{c}{ find field description and add the field}
\PYG{c}{             idea\PYGZus{}ids after it }\PYG{c}{\PYGZhy{}\PYGZhy{}\PYGZgt{}}
        \PYG{n+nt}{\PYGZlt{}xpath} \PYG{n+na}{expr=}\PYG{l+s}{\PYGZdq{}//field[@name=\PYGZsq{}description\PYGZsq{}]\PYGZdq{}} \PYG{n+na}{position=}\PYG{l+s}{\PYGZdq{}after\PYGZdq{}}\PYG{n+nt}{\PYGZgt{}}
          \PYG{n+nt}{\PYGZlt{}field} \PYG{n+na}{name=}\PYG{l+s}{\PYGZdq{}idea\PYGZus{}ids\PYGZdq{}} \PYG{n+na}{string=}\PYG{l+s}{\PYGZdq{}Number of ideas\PYGZdq{}}\PYG{n+nt}{/\PYGZgt{}}
        \PYG{n+nt}{\PYGZlt{}/xpath\PYGZgt{}}
    \PYG{n+nt}{\PYGZlt{}/field\PYGZgt{}}
\PYG{n+nt}{\PYGZlt{}/record\PYGZgt{}}
\end{sphinxVerbatim}
\begin{description}
\item[{\sphinxcode{\sphinxupquote{expr}}}] \leavevmode
An \sphinxhref{http://w3.org/TR/xpath}{XPath} expression selecting a single element in the parent view.
Raises an error if it matches no element or more than one

\item[{\sphinxcode{\sphinxupquote{position}}}] \leavevmode
Operation to apply to the matched element:
\begin{description}
\item[{\sphinxcode{\sphinxupquote{inside}}}] \leavevmode
appends \sphinxcode{\sphinxupquote{xpath}}’s body at the end of the matched element

\item[{\sphinxcode{\sphinxupquote{replace}}}] \leavevmode
replaces the matched element with the \sphinxcode{\sphinxupquote{xpath}}’s body, replacing any \sphinxcode{\sphinxupquote{\$0}} node occurrence
in the new body with the original element

\item[{\sphinxcode{\sphinxupquote{before}}}] \leavevmode
inserts the \sphinxcode{\sphinxupquote{xpath}}’s body as a sibling before the matched element

\item[{\sphinxcode{\sphinxupquote{after}}}] \leavevmode
inserts the \sphinxcode{\sphinxupquote{xpaths}}’s body as a sibling after the matched element

\item[{\sphinxcode{\sphinxupquote{attributes}}}] \leavevmode
alters the attributes of the matched element using special
\sphinxcode{\sphinxupquote{attribute}} elements in the \sphinxcode{\sphinxupquote{xpath}}’s body

\end{description}

\end{description}

\begin{sphinxadmonition}{tip}{Tip:}
When matching a single element, the \sphinxcode{\sphinxupquote{position}} attribute can be set directly
on the element to be found. Both inheritances below will give the same result.

\fvset{hllines={, ,}}%
\begin{sphinxVerbatim}[commandchars=\\\{\}]
\PYG{n+nt}{\PYGZlt{}xpath} \PYG{n+na}{expr=}\PYG{l+s}{\PYGZdq{}//field[@name=\PYGZsq{}description\PYGZsq{}]\PYGZdq{}} \PYG{n+na}{position=}\PYG{l+s}{\PYGZdq{}after\PYGZdq{}}\PYG{n+nt}{\PYGZgt{}}
    \PYG{n+nt}{\PYGZlt{}field} \PYG{n+na}{name=}\PYG{l+s}{\PYGZdq{}idea\PYGZus{}ids\PYGZdq{}} \PYG{n+nt}{/\PYGZgt{}}
\PYG{n+nt}{\PYGZlt{}/xpath\PYGZgt{}}

\PYG{n+nt}{\PYGZlt{}field} \PYG{n+na}{name=}\PYG{l+s}{\PYGZdq{}description\PYGZdq{}} \PYG{n+na}{position=}\PYG{l+s}{\PYGZdq{}after\PYGZdq{}}\PYG{n+nt}{\PYGZgt{}}
    \PYG{n+nt}{\PYGZlt{}field} \PYG{n+na}{name=}\PYG{l+s}{\PYGZdq{}idea\PYGZus{}ids\PYGZdq{}} \PYG{n+nt}{/\PYGZgt{}}
\PYG{n+nt}{\PYGZlt{}/field\PYGZgt{}}
\end{sphinxVerbatim}
\end{sphinxadmonition}

\begin{sphinxadmonition}{note}
Alter existing content
\begin{itemize}
\item {} 
Using model inheritance, modify the existing \sphinxstyleemphasis{Partner} model to add an
\sphinxcode{\sphinxupquote{instructor}} boolean field, and a many2many field that corresponds to
the session-partner relation

\item {} 
Using view inheritance, display this fields in the partner form view

\end{itemize}

\begin{sphinxadmonition}{note}{Note:}
This is the opportunity to introduce the developer mode to
inspect the view, find its external ID and the place to put the
new field.
\end{sphinxadmonition}
\begin{enumerate}
\item {} 
Create a file \sphinxcode{\sphinxupquote{openacademy/models/partner.py}} and import it in
\sphinxcode{\sphinxupquote{\_\_init\_\_.py}}

\item {} 
Create a file \sphinxcode{\sphinxupquote{openacademy/views/partner.xml}} and add it to
\sphinxcode{\sphinxupquote{\_\_manifest\_\_.py}}

\end{enumerate}
\sphinxstyleemphasis{openacademy/\_\_init\_\_.py}
\fvset{hllines={, 4,}}%
\begin{sphinxVerbatim}[commandchars=\\\{\}]
\PYG{c+c1}{\PYGZsh{} \PYGZhy{}*\PYGZhy{} coding: utf\PYGZhy{}8 \PYGZhy{}*\PYGZhy{}}
\PYG{k+kn}{from} \PYG{n+nn}{.} \PYG{k+kn}{import} \PYG{n}{controllers}
\PYG{k+kn}{from} \PYG{n+nn}{.} \PYG{k+kn}{import} \PYG{n}{models}
\PYG{k+kn}{from} \PYG{n+nn}{.} \PYG{k+kn}{import} \PYG{n}{partner}
\end{sphinxVerbatim}
\sphinxstyleemphasis{openacademy/\_\_manifest\_\_.py}
\fvset{hllines={, 4,}}%
\begin{sphinxVerbatim}[commandchars=\\\{\}]
        \PYG{c+c1}{\PYGZsh{} \PYGZsq{}security/ir.model.access.csv\PYGZsq{},}
        \PYG{l+s+s1}{\PYGZsq{}}\PYG{l+s+s1}{templates.xml}\PYG{l+s+s1}{\PYGZsq{}}\PYG{p}{,}
        \PYG{l+s+s1}{\PYGZsq{}}\PYG{l+s+s1}{views/openacademy.xml}\PYG{l+s+s1}{\PYGZsq{}}\PYG{p}{,}
        \PYG{l+s+s1}{\PYGZsq{}}\PYG{l+s+s1}{views/partner.xml}\PYG{l+s+s1}{\PYGZsq{}}\PYG{p}{,}
    \PYG{p}{]}\PYG{p}{,}
    \PYG{c+c1}{\PYGZsh{} only loaded in demonstration mode}
    \PYG{l+s+s1}{\PYGZsq{}}\PYG{l+s+s1}{demo}\PYG{l+s+s1}{\PYGZsq{}}\PYG{p}{:} \PYG{p}{[}
\end{sphinxVerbatim}
\sphinxstyleemphasis{openacademy/partner.py}
\fvset{hllines={, 1, 2, 3, 4, 5, 6, 7, 8, 9, 10, 11, 12,}}%
\begin{sphinxVerbatim}[commandchars=\\\{\}]
\PYG{c+c1}{\PYGZsh{} \PYGZhy{}*\PYGZhy{} coding: utf\PYGZhy{}8 \PYGZhy{}*\PYGZhy{}}
\PYG{k+kn}{from} \PYG{n+nn}{odoo} \PYG{k+kn}{import} \PYG{n}{fields}\PYG{p}{,} \PYG{n}{models}

\PYG{k}{class} \PYG{n+nc}{Partner}\PYG{p}{(}\PYG{n}{models}\PYG{o}{.}\PYG{n}{Model}\PYG{p}{)}\PYG{p}{:}
    \PYG{n}{\PYGZus{}inherit} \PYG{o}{=} \PYG{l+s+s1}{\PYGZsq{}}\PYG{l+s+s1}{res.partner}\PYG{l+s+s1}{\PYGZsq{}}

    \PYG{c+c1}{\PYGZsh{} Add a new column to the res.partner model, by default partners are not}
    \PYG{c+c1}{\PYGZsh{} instructors}
    \PYG{n}{instructor} \PYG{o}{=} \PYG{n}{fields}\PYG{o}{.}\PYG{n}{Boolean}\PYG{p}{(}\PYG{l+s+s2}{\PYGZdq{}}\PYG{l+s+s2}{Instructor}\PYG{l+s+s2}{\PYGZdq{}}\PYG{p}{,} \PYG{n}{default}\PYG{o}{=}\PYG{n+nb+bp}{False}\PYG{p}{)}

    \PYG{n}{session\PYGZus{}ids} \PYG{o}{=} \PYG{n}{fields}\PYG{o}{.}\PYG{n}{Many2many}\PYG{p}{(}\PYG{l+s+s1}{\PYGZsq{}}\PYG{l+s+s1}{openacademy.session}\PYG{l+s+s1}{\PYGZsq{}}\PYG{p}{,}
        \PYG{n}{string}\PYG{o}{=}\PYG{l+s+s2}{\PYGZdq{}}\PYG{l+s+s2}{Attended Sessions}\PYG{l+s+s2}{\PYGZdq{}}\PYG{p}{,} \PYG{n}{readonly}\PYG{o}{=}\PYG{n+nb+bp}{True}\PYG{p}{)}
\end{sphinxVerbatim}
\sphinxstyleemphasis{openacademy/views/partner.xml}
\fvset{hllines={, 1, 2, 3, 4, 5, 6, 7, 8, 9, 10, 11, 12, 13, 14, 15, 16, 17, 18, 19, 20, 21, 22, 23, 24, 25, 26, 27, 28, 29, 30, 31, 32,}}%
\begin{sphinxVerbatim}[commandchars=\\\{\}]
\PYG{c+cp}{\PYGZlt{}?xml version=\PYGZdq{}1.0\PYGZdq{} encoding=\PYGZdq{}UTF\PYGZhy{}8\PYGZdq{}?\PYGZgt{}}
 \PYG{n+nt}{\PYGZlt{}odoo}\PYG{n+nt}{\PYGZgt{}}

        \PYG{c}{\PYGZlt{}!\PYGZhy{}\PYGZhy{}}\PYG{c}{ Add instructor field to existing view }\PYG{c}{\PYGZhy{}\PYGZhy{}\PYGZgt{}}
        \PYG{n+nt}{\PYGZlt{}record} \PYG{n+na}{model=}\PYG{l+s}{\PYGZdq{}ir.ui.view\PYGZdq{}} \PYG{n+na}{id=}\PYG{l+s}{\PYGZdq{}partner\PYGZus{}instructor\PYGZus{}form\PYGZus{}view\PYGZdq{}}\PYG{n+nt}{\PYGZgt{}}
            \PYG{n+nt}{\PYGZlt{}field} \PYG{n+na}{name=}\PYG{l+s}{\PYGZdq{}name\PYGZdq{}}\PYG{n+nt}{\PYGZgt{}}partner.instructor\PYG{n+nt}{\PYGZlt{}/field\PYGZgt{}}
            \PYG{n+nt}{\PYGZlt{}field} \PYG{n+na}{name=}\PYG{l+s}{\PYGZdq{}model\PYGZdq{}}\PYG{n+nt}{\PYGZgt{}}res.partner\PYG{n+nt}{\PYGZlt{}/field\PYGZgt{}}
            \PYG{n+nt}{\PYGZlt{}field} \PYG{n+na}{name=}\PYG{l+s}{\PYGZdq{}inherit\PYGZus{}id\PYGZdq{}} \PYG{n+na}{ref=}\PYG{l+s}{\PYGZdq{}base.view\PYGZus{}partner\PYGZus{}form\PYGZdq{}}\PYG{n+nt}{/\PYGZgt{}}
            \PYG{n+nt}{\PYGZlt{}field} \PYG{n+na}{name=}\PYG{l+s}{\PYGZdq{}arch\PYGZdq{}} \PYG{n+na}{type=}\PYG{l+s}{\PYGZdq{}xml\PYGZdq{}}\PYG{n+nt}{\PYGZgt{}}
                \PYG{n+nt}{\PYGZlt{}notebook} \PYG{n+na}{position=}\PYG{l+s}{\PYGZdq{}inside\PYGZdq{}}\PYG{n+nt}{\PYGZgt{}}
                    \PYG{n+nt}{\PYGZlt{}page} \PYG{n+na}{string=}\PYG{l+s}{\PYGZdq{}Sessions\PYGZdq{}}\PYG{n+nt}{\PYGZgt{}}
                        \PYG{n+nt}{\PYGZlt{}group}\PYG{n+nt}{\PYGZgt{}}
                            \PYG{n+nt}{\PYGZlt{}field} \PYG{n+na}{name=}\PYG{l+s}{\PYGZdq{}instructor\PYGZdq{}}\PYG{n+nt}{/\PYGZgt{}}
                            \PYG{n+nt}{\PYGZlt{}field} \PYG{n+na}{name=}\PYG{l+s}{\PYGZdq{}session\PYGZus{}ids\PYGZdq{}}\PYG{n+nt}{/\PYGZgt{}}
                        \PYG{n+nt}{\PYGZlt{}/group\PYGZgt{}}
                    \PYG{n+nt}{\PYGZlt{}/page\PYGZgt{}}
                \PYG{n+nt}{\PYGZlt{}/notebook\PYGZgt{}}
            \PYG{n+nt}{\PYGZlt{}/field\PYGZgt{}}
        \PYG{n+nt}{\PYGZlt{}/record\PYGZgt{}}

        \PYG{n+nt}{\PYGZlt{}record} \PYG{n+na}{model=}\PYG{l+s}{\PYGZdq{}ir.actions.act\PYGZus{}window\PYGZdq{}} \PYG{n+na}{id=}\PYG{l+s}{\PYGZdq{}contact\PYGZus{}list\PYGZus{}action\PYGZdq{}}\PYG{n+nt}{\PYGZgt{}}
            \PYG{n+nt}{\PYGZlt{}field} \PYG{n+na}{name=}\PYG{l+s}{\PYGZdq{}name\PYGZdq{}}\PYG{n+nt}{\PYGZgt{}}Contacts\PYG{n+nt}{\PYGZlt{}/field\PYGZgt{}}
            \PYG{n+nt}{\PYGZlt{}field} \PYG{n+na}{name=}\PYG{l+s}{\PYGZdq{}res\PYGZus{}model\PYGZdq{}}\PYG{n+nt}{\PYGZgt{}}res.partner\PYG{n+nt}{\PYGZlt{}/field\PYGZgt{}}
            \PYG{n+nt}{\PYGZlt{}field} \PYG{n+na}{name=}\PYG{l+s}{\PYGZdq{}view\PYGZus{}mode\PYGZdq{}}\PYG{n+nt}{\PYGZgt{}}tree,form\PYG{n+nt}{\PYGZlt{}/field\PYGZgt{}}
        \PYG{n+nt}{\PYGZlt{}/record\PYGZgt{}}
        \PYG{n+nt}{\PYGZlt{}menuitem} \PYG{n+na}{id=}\PYG{l+s}{\PYGZdq{}configuration\PYGZus{}menu\PYGZdq{}} \PYG{n+na}{name=}\PYG{l+s}{\PYGZdq{}Configuration\PYGZdq{}}
                  \PYG{n+na}{parent=}\PYG{l+s}{\PYGZdq{}main\PYGZus{}openacademy\PYGZus{}menu\PYGZdq{}}\PYG{n+nt}{/\PYGZgt{}}
        \PYG{n+nt}{\PYGZlt{}menuitem} \PYG{n+na}{id=}\PYG{l+s}{\PYGZdq{}contact\PYGZus{}menu\PYGZdq{}} \PYG{n+na}{name=}\PYG{l+s}{\PYGZdq{}Contacts\PYGZdq{}}
                  \PYG{n+na}{parent=}\PYG{l+s}{\PYGZdq{}configuration\PYGZus{}menu\PYGZdq{}}
                  \PYG{n+na}{action=}\PYG{l+s}{\PYGZdq{}contact\PYGZus{}list\PYGZus{}action\PYGZdq{}}\PYG{n+nt}{/\PYGZgt{}}

\PYG{n+nt}{\PYGZlt{}/odoo\PYGZgt{}}
\end{sphinxVerbatim}
\end{sphinxadmonition}


\paragraph{Domains}
\label{\detokenize{howtos/backend:domains}}
In Odoo, {\hyperref[\detokenize{reference/orm:reference-orm-domains}]{\sphinxcrossref{\DUrole{std,std-ref}{Domains}}}} are values that encode conditions on
records. A domain is a  list of criteria used to select a subset of a model’s
records. Each criteria is a triple with a field name, an operator and a value.

For instance, when used on the \sphinxstyleemphasis{Product} model the following domain selects
all \sphinxstyleemphasis{services} with a unit price over \sphinxstyleemphasis{1000}:

\fvset{hllines={, ,}}%
\begin{sphinxVerbatim}[commandchars=\\\{\}]
\PYG{p}{[}\PYG{p}{(}\PYG{l+s+s1}{\PYGZsq{}}\PYG{l+s+s1}{product\PYGZus{}type}\PYG{l+s+s1}{\PYGZsq{}}\PYG{p}{,} \PYG{l+s+s1}{\PYGZsq{}}\PYG{l+s+s1}{=}\PYG{l+s+s1}{\PYGZsq{}}\PYG{p}{,} \PYG{l+s+s1}{\PYGZsq{}}\PYG{l+s+s1}{service}\PYG{l+s+s1}{\PYGZsq{}}\PYG{p}{)}\PYG{p}{,} \PYG{p}{(}\PYG{l+s+s1}{\PYGZsq{}}\PYG{l+s+s1}{unit\PYGZus{}price}\PYG{l+s+s1}{\PYGZsq{}}\PYG{p}{,} \PYG{l+s+s1}{\PYGZsq{}}\PYG{l+s+s1}{\PYGZgt{}}\PYG{l+s+s1}{\PYGZsq{}}\PYG{p}{,} \PYG{l+m+mi}{1000}\PYG{p}{)}\PYG{p}{]}
\end{sphinxVerbatim}

By default criteria are combined with an implicit AND. The logical operators
\sphinxcode{\sphinxupquote{\&}} (AND), \sphinxcode{\sphinxupquote{\textbar{}}} (OR) and \sphinxcode{\sphinxupquote{!}} (NOT) can be used to explicitly combine
criteria. They are used in prefix position (the operator is inserted before
its arguments rather than between). For instance to select products “which are
services \sphinxstyleemphasis{OR} have a unit price which is \sphinxstyleemphasis{NOT} between 1000 and 2000”:

\fvset{hllines={, ,}}%
\begin{sphinxVerbatim}[commandchars=\\\{\}]
\PYG{p}{[}\PYG{l+s+s1}{\PYGZsq{}}\PYG{l+s+s1}{\textbar{}}\PYG{l+s+s1}{\PYGZsq{}}\PYG{p}{,}
    \PYG{p}{(}\PYG{l+s+s1}{\PYGZsq{}}\PYG{l+s+s1}{product\PYGZus{}type}\PYG{l+s+s1}{\PYGZsq{}}\PYG{p}{,} \PYG{l+s+s1}{\PYGZsq{}}\PYG{l+s+s1}{=}\PYG{l+s+s1}{\PYGZsq{}}\PYG{p}{,} \PYG{l+s+s1}{\PYGZsq{}}\PYG{l+s+s1}{service}\PYG{l+s+s1}{\PYGZsq{}}\PYG{p}{)}\PYG{p}{,}
    \PYG{l+s+s1}{\PYGZsq{}}\PYG{l+s+s1}{!}\PYG{l+s+s1}{\PYGZsq{}}\PYG{p}{,} \PYG{l+s+s1}{\PYGZsq{}}\PYG{l+s+s1}{\PYGZam{}}\PYG{l+s+s1}{\PYGZsq{}}\PYG{p}{,}
        \PYG{p}{(}\PYG{l+s+s1}{\PYGZsq{}}\PYG{l+s+s1}{unit\PYGZus{}price}\PYG{l+s+s1}{\PYGZsq{}}\PYG{p}{,} \PYG{l+s+s1}{\PYGZsq{}}\PYG{l+s+s1}{\PYGZgt{}=}\PYG{l+s+s1}{\PYGZsq{}}\PYG{p}{,} \PYG{l+m+mi}{1000}\PYG{p}{)}\PYG{p}{,}
        \PYG{p}{(}\PYG{l+s+s1}{\PYGZsq{}}\PYG{l+s+s1}{unit\PYGZus{}price}\PYG{l+s+s1}{\PYGZsq{}}\PYG{p}{,} \PYG{l+s+s1}{\PYGZsq{}}\PYG{l+s+s1}{\PYGZlt{}}\PYG{l+s+s1}{\PYGZsq{}}\PYG{p}{,} \PYG{l+m+mi}{2000}\PYG{p}{)}\PYG{p}{]}
\end{sphinxVerbatim}

A \sphinxcode{\sphinxupquote{domain}} parameter can be added to relational fields to limit valid
records for the relation when trying to select records in the client interface.

\begin{sphinxadmonition}{note}
Domains on relational fields

When selecting the instructor for a \sphinxstyleemphasis{Session}, only instructors (partners
with \sphinxcode{\sphinxupquote{instructor}} set to \sphinxcode{\sphinxupquote{True}}) should be visible.
\sphinxstyleemphasis{openacademy/models.py}
\fvset{hllines={, 4, 5,}}%
\begin{sphinxVerbatim}[commandchars=\\\{\}]
    \PYG{n}{duration} \PYG{o}{=} \PYG{n}{fields}\PYG{o}{.}\PYG{n}{Float}\PYG{p}{(}\PYG{n}{digits}\PYG{o}{=}\PYG{p}{(}\PYG{l+m+mi}{6}\PYG{p}{,} \PYG{l+m+mi}{2}\PYG{p}{)}\PYG{p}{,} \PYG{n}{help}\PYG{o}{=}\PYG{l+s+s2}{\PYGZdq{}}\PYG{l+s+s2}{Duration in days}\PYG{l+s+s2}{\PYGZdq{}}\PYG{p}{)}
    \PYG{n}{seats} \PYG{o}{=} \PYG{n}{fields}\PYG{o}{.}\PYG{n}{Integer}\PYG{p}{(}\PYG{n}{string}\PYG{o}{=}\PYG{l+s+s2}{\PYGZdq{}}\PYG{l+s+s2}{Number of seats}\PYG{l+s+s2}{\PYGZdq{}}\PYG{p}{)}

    \PYG{n}{instructor\PYGZus{}id} \PYG{o}{=} \PYG{n}{fields}\PYG{o}{.}\PYG{n}{Many2one}\PYG{p}{(}\PYG{l+s+s1}{\PYGZsq{}}\PYG{l+s+s1}{res.partner}\PYG{l+s+s1}{\PYGZsq{}}\PYG{p}{,} \PYG{n}{string}\PYG{o}{=}\PYG{l+s+s2}{\PYGZdq{}}\PYG{l+s+s2}{Instructor}\PYG{l+s+s2}{\PYGZdq{}}\PYG{p}{,}
        \PYG{n}{domain}\PYG{o}{=}\PYG{p}{[}\PYG{p}{(}\PYG{l+s+s1}{\PYGZsq{}}\PYG{l+s+s1}{instructor}\PYG{l+s+s1}{\PYGZsq{}}\PYG{p}{,} \PYG{l+s+s1}{\PYGZsq{}}\PYG{l+s+s1}{=}\PYG{l+s+s1}{\PYGZsq{}}\PYG{p}{,} \PYG{n+nb+bp}{True}\PYG{p}{)}\PYG{p}{]}\PYG{p}{)}
    \PYG{n}{course\PYGZus{}id} \PYG{o}{=} \PYG{n}{fields}\PYG{o}{.}\PYG{n}{Many2one}\PYG{p}{(}\PYG{l+s+s1}{\PYGZsq{}}\PYG{l+s+s1}{openacademy.course}\PYG{l+s+s1}{\PYGZsq{}}\PYG{p}{,}
        \PYG{n}{ondelete}\PYG{o}{=}\PYG{l+s+s1}{\PYGZsq{}}\PYG{l+s+s1}{cascade}\PYG{l+s+s1}{\PYGZsq{}}\PYG{p}{,} \PYG{n}{string}\PYG{o}{=}\PYG{l+s+s2}{\PYGZdq{}}\PYG{l+s+s2}{Course}\PYG{l+s+s2}{\PYGZdq{}}\PYG{p}{,} \PYG{n}{required}\PYG{o}{=}\PYG{n+nb+bp}{True}\PYG{p}{)}
    \PYG{n}{attendee\PYGZus{}ids} \PYG{o}{=} \PYG{n}{fields}\PYG{o}{.}\PYG{n}{Many2many}\PYG{p}{(}\PYG{l+s+s1}{\PYGZsq{}}\PYG{l+s+s1}{res.partner}\PYG{l+s+s1}{\PYGZsq{}}\PYG{p}{,} \PYG{n}{string}\PYG{o}{=}\PYG{l+s+s2}{\PYGZdq{}}\PYG{l+s+s2}{Attendees}\PYG{l+s+s2}{\PYGZdq{}}\PYG{p}{)}
\end{sphinxVerbatim}

\begin{sphinxadmonition}{note}{Note:}
A domain declared as a literal list is evaluated server-side and
can’t refer to dynamic values on the right-hand side, a domain
declared as a string is evaluated client-side and allows
field names on the right-hand side
\end{sphinxadmonition}
\end{sphinxadmonition}

\begin{sphinxadmonition}{note}
More complex domains

Create new partner categories \sphinxstyleemphasis{Teacher / Level 1} and \sphinxstyleemphasis{Teacher / Level 2}.
The instructor for a session can be either an instructor or a teacher
(of any level).
\begin{enumerate}
\item {} 
Modify the \sphinxstyleemphasis{Session} model’s domain

\item {} 
Modify \sphinxcode{\sphinxupquote{openacademy/view/partner.xml}} to get access to
\sphinxstyleemphasis{Partner categories}:

\end{enumerate}
\sphinxstyleemphasis{openacademy/models.py}
\fvset{hllines={, 4, 5,}}%
\begin{sphinxVerbatim}[commandchars=\\\{\}]
    \PYG{n}{seats} \PYG{o}{=} \PYG{n}{fields}\PYG{o}{.}\PYG{n}{Integer}\PYG{p}{(}\PYG{n}{string}\PYG{o}{=}\PYG{l+s+s2}{\PYGZdq{}}\PYG{l+s+s2}{Number of seats}\PYG{l+s+s2}{\PYGZdq{}}\PYG{p}{)}

    \PYG{n}{instructor\PYGZus{}id} \PYG{o}{=} \PYG{n}{fields}\PYG{o}{.}\PYG{n}{Many2one}\PYG{p}{(}\PYG{l+s+s1}{\PYGZsq{}}\PYG{l+s+s1}{res.partner}\PYG{l+s+s1}{\PYGZsq{}}\PYG{p}{,} \PYG{n}{string}\PYG{o}{=}\PYG{l+s+s2}{\PYGZdq{}}\PYG{l+s+s2}{Instructor}\PYG{l+s+s2}{\PYGZdq{}}\PYG{p}{,}
        \PYG{n}{domain}\PYG{o}{=}\PYG{p}{[}\PYG{l+s+s1}{\PYGZsq{}}\PYG{l+s+s1}{\textbar{}}\PYG{l+s+s1}{\PYGZsq{}}\PYG{p}{,} \PYG{p}{(}\PYG{l+s+s1}{\PYGZsq{}}\PYG{l+s+s1}{instructor}\PYG{l+s+s1}{\PYGZsq{}}\PYG{p}{,} \PYG{l+s+s1}{\PYGZsq{}}\PYG{l+s+s1}{=}\PYG{l+s+s1}{\PYGZsq{}}\PYG{p}{,} \PYG{n+nb+bp}{True}\PYG{p}{)}\PYG{p}{,}
                     \PYG{p}{(}\PYG{l+s+s1}{\PYGZsq{}}\PYG{l+s+s1}{category\PYGZus{}id.name}\PYG{l+s+s1}{\PYGZsq{}}\PYG{p}{,} \PYG{l+s+s1}{\PYGZsq{}}\PYG{l+s+s1}{ilike}\PYG{l+s+s1}{\PYGZsq{}}\PYG{p}{,} \PYG{l+s+s2}{\PYGZdq{}}\PYG{l+s+s2}{Teacher}\PYG{l+s+s2}{\PYGZdq{}}\PYG{p}{)}\PYG{p}{]}\PYG{p}{)}
    \PYG{n}{course\PYGZus{}id} \PYG{o}{=} \PYG{n}{fields}\PYG{o}{.}\PYG{n}{Many2one}\PYG{p}{(}\PYG{l+s+s1}{\PYGZsq{}}\PYG{l+s+s1}{openacademy.course}\PYG{l+s+s1}{\PYGZsq{}}\PYG{p}{,}
        \PYG{n}{ondelete}\PYG{o}{=}\PYG{l+s+s1}{\PYGZsq{}}\PYG{l+s+s1}{cascade}\PYG{l+s+s1}{\PYGZsq{}}\PYG{p}{,} \PYG{n}{string}\PYG{o}{=}\PYG{l+s+s2}{\PYGZdq{}}\PYG{l+s+s2}{Course}\PYG{l+s+s2}{\PYGZdq{}}\PYG{p}{,} \PYG{n}{required}\PYG{o}{=}\PYG{n+nb+bp}{True}\PYG{p}{)}
    \PYG{n}{attendee\PYGZus{}ids} \PYG{o}{=} \PYG{n}{fields}\PYG{o}{.}\PYG{n}{Many2many}\PYG{p}{(}\PYG{l+s+s1}{\PYGZsq{}}\PYG{l+s+s1}{res.partner}\PYG{l+s+s1}{\PYGZsq{}}\PYG{p}{,} \PYG{n}{string}\PYG{o}{=}\PYG{l+s+s2}{\PYGZdq{}}\PYG{l+s+s2}{Attendees}\PYG{l+s+s2}{\PYGZdq{}}\PYG{p}{)}
\end{sphinxVerbatim}
\sphinxstyleemphasis{openacademy/views/partner.xml}
\fvset{hllines={, 4, 5, 6, 7, 8, 9, 10, 11, 12, 13, 14, 15, 16, 17, 18, 19,}}%
\begin{sphinxVerbatim}[commandchars=\\\{\}]
        \PYG{n+nt}{\PYGZlt{}menuitem} \PYG{n+na}{id=}\PYG{l+s}{\PYGZdq{}contact\PYGZus{}menu\PYGZdq{}} \PYG{n+na}{name=}\PYG{l+s}{\PYGZdq{}Contacts\PYGZdq{}}
                  \PYG{n+na}{parent=}\PYG{l+s}{\PYGZdq{}configuration\PYGZus{}menu\PYGZdq{}}
                  \PYG{n+na}{action=}\PYG{l+s}{\PYGZdq{}contact\PYGZus{}list\PYGZus{}action\PYGZdq{}}\PYG{n+nt}{/\PYGZgt{}}

        \PYG{n+nt}{\PYGZlt{}record} \PYG{n+na}{model=}\PYG{l+s}{\PYGZdq{}ir.actions.act\PYGZus{}window\PYGZdq{}} \PYG{n+na}{id=}\PYG{l+s}{\PYGZdq{}contact\PYGZus{}cat\PYGZus{}list\PYGZus{}action\PYGZdq{}}\PYG{n+nt}{\PYGZgt{}}
            \PYG{n+nt}{\PYGZlt{}field} \PYG{n+na}{name=}\PYG{l+s}{\PYGZdq{}name\PYGZdq{}}\PYG{n+nt}{\PYGZgt{}}Contact Tags\PYG{n+nt}{\PYGZlt{}/field\PYGZgt{}}
            \PYG{n+nt}{\PYGZlt{}field} \PYG{n+na}{name=}\PYG{l+s}{\PYGZdq{}res\PYGZus{}model\PYGZdq{}}\PYG{n+nt}{\PYGZgt{}}res.partner.category\PYG{n+nt}{\PYGZlt{}/field\PYGZgt{}}
            \PYG{n+nt}{\PYGZlt{}field} \PYG{n+na}{name=}\PYG{l+s}{\PYGZdq{}view\PYGZus{}mode\PYGZdq{}}\PYG{n+nt}{\PYGZgt{}}tree,form\PYG{n+nt}{\PYGZlt{}/field\PYGZgt{}}
        \PYG{n+nt}{\PYGZlt{}/record\PYGZgt{}}
        \PYG{n+nt}{\PYGZlt{}menuitem} \PYG{n+na}{id=}\PYG{l+s}{\PYGZdq{}contact\PYGZus{}cat\PYGZus{}menu\PYGZdq{}} \PYG{n+na}{name=}\PYG{l+s}{\PYGZdq{}Contact Tags\PYGZdq{}}
                  \PYG{n+na}{parent=}\PYG{l+s}{\PYGZdq{}configuration\PYGZus{}menu\PYGZdq{}}
                  \PYG{n+na}{action=}\PYG{l+s}{\PYGZdq{}contact\PYGZus{}cat\PYGZus{}list\PYGZus{}action\PYGZdq{}}\PYG{n+nt}{/\PYGZgt{}}

        \PYG{n+nt}{\PYGZlt{}record} \PYG{n+na}{model=}\PYG{l+s}{\PYGZdq{}res.partner.category\PYGZdq{}} \PYG{n+na}{id=}\PYG{l+s}{\PYGZdq{}teacher1\PYGZdq{}}\PYG{n+nt}{\PYGZgt{}}
            \PYG{n+nt}{\PYGZlt{}field} \PYG{n+na}{name=}\PYG{l+s}{\PYGZdq{}name\PYGZdq{}}\PYG{n+nt}{\PYGZgt{}}Teacher / Level 1\PYG{n+nt}{\PYGZlt{}/field\PYGZgt{}}
        \PYG{n+nt}{\PYGZlt{}/record\PYGZgt{}}
        \PYG{n+nt}{\PYGZlt{}record} \PYG{n+na}{model=}\PYG{l+s}{\PYGZdq{}res.partner.category\PYGZdq{}} \PYG{n+na}{id=}\PYG{l+s}{\PYGZdq{}teacher2\PYGZdq{}}\PYG{n+nt}{\PYGZgt{}}
            \PYG{n+nt}{\PYGZlt{}field} \PYG{n+na}{name=}\PYG{l+s}{\PYGZdq{}name\PYGZdq{}}\PYG{n+nt}{\PYGZgt{}}Teacher / Level 2\PYG{n+nt}{\PYGZlt{}/field\PYGZgt{}}
        \PYG{n+nt}{\PYGZlt{}/record\PYGZgt{}}

\PYG{n+nt}{\PYGZlt{}/odoo\PYGZgt{}}
\end{sphinxVerbatim}
\end{sphinxadmonition}


\subsection{Computed fields and default values}
\label{\detokenize{howtos/backend:computed-fields-and-default-values}}
So far fields have been stored directly in and retrieved directly from the
database. Fields can also be \sphinxstyleemphasis{computed}. In that case, the field’s value is not
retrieved from the database but computed on-the-fly by calling a method of the
model.

To create a computed field, create a field and set its attribute
\sphinxcode{\sphinxupquote{compute}} to the name of a method. The computation
method should simply set the value of the field to compute on every record in
\sphinxcode{\sphinxupquote{self}}.

\begin{sphinxadmonition}{danger}{Danger:}
\sphinxcode{\sphinxupquote{self}} is a collection

The object \sphinxcode{\sphinxupquote{self}} is a \sphinxstyleemphasis{recordset}, i.e., an ordered collection of
records. It supports the standard Python operations on collections, like
\sphinxcode{\sphinxupquote{len(self)}} and \sphinxcode{\sphinxupquote{iter(self)}}, plus extra set operations like \sphinxcode{\sphinxupquote{recs1 +
recs2}}.

Iterating over \sphinxcode{\sphinxupquote{self}} gives the records one by one, where each record is
itself a collection of size 1. You can access/assign fields on single
records by using the dot notation, like \sphinxcode{\sphinxupquote{record.name}}.
\end{sphinxadmonition}

\fvset{hllines={, ,}}%
\begin{sphinxVerbatim}[commandchars=\\\{\}]
\PYG{k+kn}{import} \PYG{n+nn}{random}
\PYG{k+kn}{from} \PYG{n+nn}{odoo} \PYG{k+kn}{import} \PYG{n}{models}\PYG{p}{,} \PYG{n}{fields}\PYG{p}{,} \PYG{n}{api}

\PYG{k}{class} \PYG{n+nc}{ComputedModel}\PYG{p}{(}\PYG{n}{models}\PYG{o}{.}\PYG{n}{Model}\PYG{p}{)}\PYG{p}{:}
    \PYG{n}{\PYGZus{}name} \PYG{o}{=} \PYG{l+s+s1}{\PYGZsq{}}\PYG{l+s+s1}{test.computed}\PYG{l+s+s1}{\PYGZsq{}}

    \PYG{n}{name} \PYG{o}{=} \PYG{n}{fields}\PYG{o}{.}\PYG{n}{Char}\PYG{p}{(}\PYG{n}{compute}\PYG{o}{=}\PYG{l+s+s1}{\PYGZsq{}}\PYG{l+s+s1}{\PYGZus{}compute\PYGZus{}name}\PYG{l+s+s1}{\PYGZsq{}}\PYG{p}{)}

    \PYG{n+nd}{@api.multi}
    \PYG{k}{def} \PYG{n+nf}{\PYGZus{}compute\PYGZus{}name}\PYG{p}{(}\PYG{n+nb+bp}{self}\PYG{p}{)}\PYG{p}{:}
        \PYG{k}{for} \PYG{n}{record} \PYG{o+ow}{in} \PYG{n+nb+bp}{self}\PYG{p}{:}
            \PYG{n}{record}\PYG{o}{.}\PYG{n}{name} \PYG{o}{=} \PYG{n+nb}{str}\PYG{p}{(}\PYG{n}{random}\PYG{o}{.}\PYG{n}{randint}\PYG{p}{(}\PYG{l+m+mi}{1}\PYG{p}{,} \PYG{l+m+mf}{1e6}\PYG{p}{)}\PYG{p}{)}
\end{sphinxVerbatim}


\subsubsection{Dependencies}
\label{\detokenize{howtos/backend:dependencies}}
The value of a computed field usually depends on the values of other fields on
the computed record. The ORM expects the developer to specify those dependencies
on the compute method with the decorator {\hyperref[\detokenize{reference/orm:odoo.api.depends}]{\sphinxcrossref{\sphinxcode{\sphinxupquote{depends()}}}}}.
The given dependencies are used by the ORM to trigger the recomputation of the
field whenever some of its dependencies have been modified:

\fvset{hllines={, ,}}%
\begin{sphinxVerbatim}[commandchars=\\\{\}]
\PYG{k+kn}{from} \PYG{n+nn}{odoo} \PYG{k}{import} \PYG{n}{models}\PYG{p}{,} \PYG{n}{fields}\PYG{p}{,} \PYG{n}{api}

\PYG{k}{class} \PYG{n+nc}{ComputedModel}\PYG{p}{(}\PYG{n}{models}\PYG{o}{.}\PYG{n}{Model}\PYG{p}{)}\PYG{p}{:}
    \PYG{n}{\PYGZus{}name} \PYG{o}{=} \PYG{l+s+s1}{\PYGZsq{}}\PYG{l+s+s1}{test.computed}\PYG{l+s+s1}{\PYGZsq{}}

    \PYG{n}{name} \PYG{o}{=} \PYG{n}{fields}\PYG{o}{.}\PYG{n}{Char}\PYG{p}{(}\PYG{n}{compute}\PYG{o}{=}\PYG{l+s+s1}{\PYGZsq{}}\PYG{l+s+s1}{\PYGZus{}compute\PYGZus{}name}\PYG{l+s+s1}{\PYGZsq{}}\PYG{p}{)}
    \PYG{n}{value} \PYG{o}{=} \PYG{n}{fields}\PYG{o}{.}\PYG{n}{Integer}\PYG{p}{(}\PYG{p}{)}

    \PYG{n+nd}{@api}\PYG{o}{.}\PYG{n}{depends}\PYG{p}{(}\PYG{l+s+s1}{\PYGZsq{}}\PYG{l+s+s1}{value}\PYG{l+s+s1}{\PYGZsq{}}\PYG{p}{)}
    \PYG{k}{def} \PYG{n+nf}{\PYGZus{}compute\PYGZus{}name}\PYG{p}{(}\PYG{n+nb+bp}{self}\PYG{p}{)}\PYG{p}{:}
        \PYG{k}{for} \PYG{n}{record} \PYG{o+ow}{in} \PYG{n+nb+bp}{self}\PYG{p}{:}
            \PYG{n}{record}\PYG{o}{.}\PYG{n}{name} \PYG{o}{=} \PYG{l+s+s2}{\PYGZdq{}}\PYG{l+s+s2}{Record with value }\PYG{l+s+si}{\PYGZpc{}s}\PYG{l+s+s2}{\PYGZdq{}} \PYG{o}{\PYGZpc{}} \PYG{n}{record}\PYG{o}{.}\PYG{n}{value}
\end{sphinxVerbatim}

\begin{sphinxadmonition}{note}
Computed fields
\begin{itemize}
\item {} 
Add the percentage of taken seats to the \sphinxstyleemphasis{Session} model

\item {} 
Display that field in the tree and form views

\item {} 
Display the field as a progress bar

\end{itemize}
\begin{enumerate}
\item {} 
Add a computed field to \sphinxstyleemphasis{Session}

\item {} 
Show the field in the \sphinxstyleemphasis{Session} view:

\end{enumerate}
\sphinxstyleemphasis{openacademy/models.py}
\fvset{hllines={, 4, 5, 6, 7, 8, 9, 10, 11, 12, 13,}}%
\begin{sphinxVerbatim}[commandchars=\\\{\}]
    \PYG{n}{course\PYGZus{}id} \PYG{o}{=} \PYG{n}{fields}\PYG{o}{.}\PYG{n}{Many2one}\PYG{p}{(}\PYG{l+s+s1}{\PYGZsq{}}\PYG{l+s+s1}{openacademy.course}\PYG{l+s+s1}{\PYGZsq{}}\PYG{p}{,}
        \PYG{n}{ondelete}\PYG{o}{=}\PYG{l+s+s1}{\PYGZsq{}}\PYG{l+s+s1}{cascade}\PYG{l+s+s1}{\PYGZsq{}}\PYG{p}{,} \PYG{n}{string}\PYG{o}{=}\PYG{l+s+s2}{\PYGZdq{}}\PYG{l+s+s2}{Course}\PYG{l+s+s2}{\PYGZdq{}}\PYG{p}{,} \PYG{n}{required}\PYG{o}{=}\PYG{n+nb+bp}{True}\PYG{p}{)}
    \PYG{n}{attendee\PYGZus{}ids} \PYG{o}{=} \PYG{n}{fields}\PYG{o}{.}\PYG{n}{Many2many}\PYG{p}{(}\PYG{l+s+s1}{\PYGZsq{}}\PYG{l+s+s1}{res.partner}\PYG{l+s+s1}{\PYGZsq{}}\PYG{p}{,} \PYG{n}{string}\PYG{o}{=}\PYG{l+s+s2}{\PYGZdq{}}\PYG{l+s+s2}{Attendees}\PYG{l+s+s2}{\PYGZdq{}}\PYG{p}{)}

    \PYG{n}{taken\PYGZus{}seats} \PYG{o}{=} \PYG{n}{fields}\PYG{o}{.}\PYG{n}{Float}\PYG{p}{(}\PYG{n}{string}\PYG{o}{=}\PYG{l+s+s2}{\PYGZdq{}}\PYG{l+s+s2}{Taken seats}\PYG{l+s+s2}{\PYGZdq{}}\PYG{p}{,} \PYG{n}{compute}\PYG{o}{=}\PYG{l+s+s1}{\PYGZsq{}}\PYG{l+s+s1}{\PYGZus{}taken\PYGZus{}seats}\PYG{l+s+s1}{\PYGZsq{}}\PYG{p}{)}

    \PYG{n+nd}{@api.depends}\PYG{p}{(}\PYG{l+s+s1}{\PYGZsq{}}\PYG{l+s+s1}{seats}\PYG{l+s+s1}{\PYGZsq{}}\PYG{p}{,} \PYG{l+s+s1}{\PYGZsq{}}\PYG{l+s+s1}{attendee\PYGZus{}ids}\PYG{l+s+s1}{\PYGZsq{}}\PYG{p}{)}
    \PYG{k}{def} \PYG{n+nf}{\PYGZus{}taken\PYGZus{}seats}\PYG{p}{(}\PYG{n+nb+bp}{self}\PYG{p}{)}\PYG{p}{:}
        \PYG{k}{for} \PYG{n}{r} \PYG{o+ow}{in} \PYG{n+nb+bp}{self}\PYG{p}{:}
            \PYG{k}{if} \PYG{o+ow}{not} \PYG{n}{r}\PYG{o}{.}\PYG{n}{seats}\PYG{p}{:}
                \PYG{n}{r}\PYG{o}{.}\PYG{n}{taken\PYGZus{}seats} \PYG{o}{=} \PYG{l+m+mf}{0.0}
            \PYG{k}{else}\PYG{p}{:}
                \PYG{n}{r}\PYG{o}{.}\PYG{n}{taken\PYGZus{}seats} \PYG{o}{=} \PYG{l+m+mf}{100.0} \PYG{o}{*} \PYG{n+nb}{len}\PYG{p}{(}\PYG{n}{r}\PYG{o}{.}\PYG{n}{attendee\PYGZus{}ids}\PYG{p}{)} \PYG{o}{/} \PYG{n}{r}\PYG{o}{.}\PYG{n}{seats}
\end{sphinxVerbatim}
\sphinxstyleemphasis{openacademy/views/openacademy.xml}
\fvset{hllines={, 4,}}%
\begin{sphinxVerbatim}[commandchars=\\\{\}]
                                \PYG{n+nt}{\PYGZlt{}field} \PYG{n+na}{name=}\PYG{l+s}{\PYGZdq{}start\PYGZus{}date\PYGZdq{}}\PYG{n+nt}{/\PYGZgt{}}
                                \PYG{n+nt}{\PYGZlt{}field} \PYG{n+na}{name=}\PYG{l+s}{\PYGZdq{}duration\PYGZdq{}}\PYG{n+nt}{/\PYGZgt{}}
                                \PYG{n+nt}{\PYGZlt{}field} \PYG{n+na}{name=}\PYG{l+s}{\PYGZdq{}seats\PYGZdq{}}\PYG{n+nt}{/\PYGZgt{}}
                                \PYG{n+nt}{\PYGZlt{}field} \PYG{n+na}{name=}\PYG{l+s}{\PYGZdq{}taken\PYGZus{}seats\PYGZdq{}} \PYG{n+na}{widget=}\PYG{l+s}{\PYGZdq{}progressbar\PYGZdq{}}\PYG{n+nt}{/\PYGZgt{}}
                            \PYG{n+nt}{\PYGZlt{}/group\PYGZgt{}}
                        \PYG{n+nt}{\PYGZlt{}/group\PYGZgt{}}
                        \PYG{n+nt}{\PYGZlt{}label} \PYG{n+na}{for=}\PYG{l+s}{\PYGZdq{}attendee\PYGZus{}ids\PYGZdq{}}\PYG{n+nt}{/\PYGZgt{}}
\end{sphinxVerbatim}

\fvset{hllines={, 4,}}%
\begin{sphinxVerbatim}[commandchars=\\\{\}]
                \PYG{n+nt}{\PYGZlt{}tree} \PYG{n+na}{string=}\PYG{l+s}{\PYGZdq{}Session Tree\PYGZdq{}}\PYG{n+nt}{\PYGZgt{}}
                    \PYG{n+nt}{\PYGZlt{}field} \PYG{n+na}{name=}\PYG{l+s}{\PYGZdq{}name\PYGZdq{}}\PYG{n+nt}{/\PYGZgt{}}
                    \PYG{n+nt}{\PYGZlt{}field} \PYG{n+na}{name=}\PYG{l+s}{\PYGZdq{}course\PYGZus{}id\PYGZdq{}}\PYG{n+nt}{/\PYGZgt{}}
                    \PYG{n+nt}{\PYGZlt{}field} \PYG{n+na}{name=}\PYG{l+s}{\PYGZdq{}taken\PYGZus{}seats\PYGZdq{}} \PYG{n+na}{widget=}\PYG{l+s}{\PYGZdq{}progressbar\PYGZdq{}}\PYG{n+nt}{/\PYGZgt{}}
                \PYG{n+nt}{\PYGZlt{}/tree\PYGZgt{}}
            \PYG{n+nt}{\PYGZlt{}/field\PYGZgt{}}
        \PYG{n+nt}{\PYGZlt{}/record\PYGZgt{}}
\end{sphinxVerbatim}
\end{sphinxadmonition}


\subsubsection{Default values}
\label{\detokenize{howtos/backend:default-values}}
Any field can be given a default value. In the field definition, add the option
\sphinxcode{\sphinxupquote{default=X}} where \sphinxcode{\sphinxupquote{X}} is either a Python literal value (boolean, integer,
float, string), or a function taking a recordset and returning a value:

\fvset{hllines={, ,}}%
\begin{sphinxVerbatim}[commandchars=\\\{\}]
\PYG{n}{name} \PYG{o}{=} \PYG{n}{fields}\PYG{o}{.}\PYG{n}{Char}\PYG{p}{(}\PYG{n}{default}\PYG{o}{=}\PYG{l+s+s2}{\PYGZdq{}}\PYG{l+s+s2}{Unknown}\PYG{l+s+s2}{\PYGZdq{}}\PYG{p}{)}
\PYG{n}{user\PYGZus{}id} \PYG{o}{=} \PYG{n}{fields}\PYG{o}{.}\PYG{n}{Many2one}\PYG{p}{(}\PYG{l+s+s1}{\PYGZsq{}}\PYG{l+s+s1}{res.users}\PYG{l+s+s1}{\PYGZsq{}}\PYG{p}{,} \PYG{n}{default}\PYG{o}{=}\PYG{k}{lambda} \PYG{n+nb+bp}{self}\PYG{p}{:} \PYG{n+nb+bp}{self}\PYG{o}{.}\PYG{n}{env}\PYG{o}{.}\PYG{n}{user}\PYG{p}{)}
\end{sphinxVerbatim}

\begin{sphinxadmonition}{note}{Note:}
The object \sphinxcode{\sphinxupquote{self.env}} gives access to request parameters and other useful
things:
\begin{itemize}
\item {} 
\sphinxcode{\sphinxupquote{self.env.cr}} or \sphinxcode{\sphinxupquote{self.\_cr}} is the database \sphinxstyleemphasis{cursor} object; it is
used for querying the database

\item {} 
\sphinxcode{\sphinxupquote{self.env.uid}} or \sphinxcode{\sphinxupquote{self.\_uid}} is the current user’s database id

\item {} 
\sphinxcode{\sphinxupquote{self.env.user}} is the current user’s record

\item {} 
\sphinxcode{\sphinxupquote{self.env.context}} or \sphinxcode{\sphinxupquote{self.\_context}} is the context dictionary

\item {} 
\sphinxcode{\sphinxupquote{self.env.ref(xml\_id)}} returns the record corresponding to an XML id

\item {} 
\sphinxcode{\sphinxupquote{self.env{[}model\_name{]}}} returns an instance of the given model

\end{itemize}
\end{sphinxadmonition}

\begin{sphinxadmonition}{note}
Active objects \textendash{} Default values
\begin{itemize}
\item {} 
Define the start\_date default value as today (see
{\hyperref[\detokenize{reference/orm:odoo.fields.Date}]{\sphinxcrossref{\sphinxcode{\sphinxupquote{Date}}}}}).

\item {} 
Add a field \sphinxcode{\sphinxupquote{active}} in the class Session, and set sessions as active by
default.

\end{itemize}
\sphinxstyleemphasis{openacademy/models.py}
\fvset{hllines={, 4, 7,}}%
\begin{sphinxVerbatim}[commandchars=\\\{\}]
    \PYG{n}{\PYGZus{}name} \PYG{o}{=} \PYG{l+s+s1}{\PYGZsq{}}\PYG{l+s+s1}{openacademy.session}\PYG{l+s+s1}{\PYGZsq{}}

    \PYG{n}{name} \PYG{o}{=} \PYG{n}{fields}\PYG{o}{.}\PYG{n}{Char}\PYG{p}{(}\PYG{n}{required}\PYG{o}{=}\PYG{n+nb+bp}{True}\PYG{p}{)}
    \PYG{n}{start\PYGZus{}date} \PYG{o}{=} \PYG{n}{fields}\PYG{o}{.}\PYG{n}{Date}\PYG{p}{(}\PYG{n}{default}\PYG{o}{=}\PYG{n}{fields}\PYG{o}{.}\PYG{n}{Date}\PYG{o}{.}\PYG{n}{today}\PYG{p}{)}
    \PYG{n}{duration} \PYG{o}{=} \PYG{n}{fields}\PYG{o}{.}\PYG{n}{Float}\PYG{p}{(}\PYG{n}{digits}\PYG{o}{=}\PYG{p}{(}\PYG{l+m+mi}{6}\PYG{p}{,} \PYG{l+m+mi}{2}\PYG{p}{)}\PYG{p}{,} \PYG{n}{help}\PYG{o}{=}\PYG{l+s+s2}{\PYGZdq{}}\PYG{l+s+s2}{Duration in days}\PYG{l+s+s2}{\PYGZdq{}}\PYG{p}{)}
    \PYG{n}{seats} \PYG{o}{=} \PYG{n}{fields}\PYG{o}{.}\PYG{n}{Integer}\PYG{p}{(}\PYG{n}{string}\PYG{o}{=}\PYG{l+s+s2}{\PYGZdq{}}\PYG{l+s+s2}{Number of seats}\PYG{l+s+s2}{\PYGZdq{}}\PYG{p}{)}
    \PYG{n}{active} \PYG{o}{=} \PYG{n}{fields}\PYG{o}{.}\PYG{n}{Boolean}\PYG{p}{(}\PYG{n}{default}\PYG{o}{=}\PYG{n+nb+bp}{True}\PYG{p}{)}

    \PYG{n}{instructor\PYGZus{}id} \PYG{o}{=} \PYG{n}{fields}\PYG{o}{.}\PYG{n}{Many2one}\PYG{p}{(}\PYG{l+s+s1}{\PYGZsq{}}\PYG{l+s+s1}{res.partner}\PYG{l+s+s1}{\PYGZsq{}}\PYG{p}{,} \PYG{n}{string}\PYG{o}{=}\PYG{l+s+s2}{\PYGZdq{}}\PYG{l+s+s2}{Instructor}\PYG{l+s+s2}{\PYGZdq{}}\PYG{p}{,}
        \PYG{n}{domain}\PYG{o}{=}\PYG{p}{[}\PYG{l+s+s1}{\PYGZsq{}}\PYG{l+s+s1}{\textbar{}}\PYG{l+s+s1}{\PYGZsq{}}\PYG{p}{,} \PYG{p}{(}\PYG{l+s+s1}{\PYGZsq{}}\PYG{l+s+s1}{instructor}\PYG{l+s+s1}{\PYGZsq{}}\PYG{p}{,} \PYG{l+s+s1}{\PYGZsq{}}\PYG{l+s+s1}{=}\PYG{l+s+s1}{\PYGZsq{}}\PYG{p}{,} \PYG{n+nb+bp}{True}\PYG{p}{)}\PYG{p}{,}
\end{sphinxVerbatim}
\sphinxstyleemphasis{openacademy/views/openacademy.xml}
\fvset{hllines={, 4,}}%
\begin{sphinxVerbatim}[commandchars=\\\{\}]
                                \PYG{n+nt}{\PYGZlt{}field} \PYG{n+na}{name=}\PYG{l+s}{\PYGZdq{}course\PYGZus{}id\PYGZdq{}}\PYG{n+nt}{/\PYGZgt{}}
                                \PYG{n+nt}{\PYGZlt{}field} \PYG{n+na}{name=}\PYG{l+s}{\PYGZdq{}name\PYGZdq{}}\PYG{n+nt}{/\PYGZgt{}}
                                \PYG{n+nt}{\PYGZlt{}field} \PYG{n+na}{name=}\PYG{l+s}{\PYGZdq{}instructor\PYGZus{}id\PYGZdq{}}\PYG{n+nt}{/\PYGZgt{}}
                                \PYG{n+nt}{\PYGZlt{}field} \PYG{n+na}{name=}\PYG{l+s}{\PYGZdq{}active\PYGZdq{}}\PYG{n+nt}{/\PYGZgt{}}
                            \PYG{n+nt}{\PYGZlt{}/group\PYGZgt{}}
                            \PYG{n+nt}{\PYGZlt{}group} \PYG{n+na}{string=}\PYG{l+s}{\PYGZdq{}Schedule\PYGZdq{}}\PYG{n+nt}{\PYGZgt{}}
                                \PYG{n+nt}{\PYGZlt{}field} \PYG{n+na}{name=}\PYG{l+s}{\PYGZdq{}start\PYGZus{}date\PYGZdq{}}\PYG{n+nt}{/\PYGZgt{}}
\end{sphinxVerbatim}

\begin{sphinxadmonition}{note}{Note:}
Odoo has built-in rules making fields with an \sphinxcode{\sphinxupquote{active}} field set
to \sphinxcode{\sphinxupquote{False}} invisible.
\end{sphinxadmonition}
\end{sphinxadmonition}


\subsection{Onchange}
\label{\detokenize{howtos/backend:onchange}}
The “onchange” mechanism provides a way for the client interface to update a
form whenever the user has filled in a value in a field, without saving anything
to the database.

For instance, suppose a model has three fields \sphinxcode{\sphinxupquote{amount}}, \sphinxcode{\sphinxupquote{unit\_price}} and
\sphinxcode{\sphinxupquote{price}}, and you want to update the price on the form when any of the other
fields is modified. To achieve this, define a method where \sphinxcode{\sphinxupquote{self}} represents
the record in the form view, and decorate it with {\hyperref[\detokenize{reference/orm:odoo.api.onchange}]{\sphinxcrossref{\sphinxcode{\sphinxupquote{onchange()}}}}}
to specify on which field it has to be triggered. Any change you make on
\sphinxcode{\sphinxupquote{self}} will be reflected on the form.

\fvset{hllines={, ,}}%
\begin{sphinxVerbatim}[commandchars=\\\{\}]
\PYG{c}{\PYGZlt{}!\PYGZhy{}\PYGZhy{}}\PYG{c}{ content of form view }\PYG{c}{\PYGZhy{}\PYGZhy{}\PYGZgt{}}
\PYG{n+nt}{\PYGZlt{}field} \PYG{n+na}{name=}\PYG{l+s}{\PYGZdq{}amount\PYGZdq{}}\PYG{n+nt}{/\PYGZgt{}}
\PYG{n+nt}{\PYGZlt{}field} \PYG{n+na}{name=}\PYG{l+s}{\PYGZdq{}unit\PYGZus{}price\PYGZdq{}}\PYG{n+nt}{/\PYGZgt{}}
\PYG{n+nt}{\PYGZlt{}field} \PYG{n+na}{name=}\PYG{l+s}{\PYGZdq{}price\PYGZdq{}} \PYG{n+na}{readonly=}\PYG{l+s}{\PYGZdq{}1\PYGZdq{}}\PYG{n+nt}{/\PYGZgt{}}
\end{sphinxVerbatim}

\fvset{hllines={, ,}}%
\begin{sphinxVerbatim}[commandchars=\\\{\}]
\PYG{c+c1}{\PYGZsh{} onchange handler}
\PYG{n+nd}{@api.onchange}\PYG{p}{(}\PYG{l+s+s1}{\PYGZsq{}}\PYG{l+s+s1}{amount}\PYG{l+s+s1}{\PYGZsq{}}\PYG{p}{,} \PYG{l+s+s1}{\PYGZsq{}}\PYG{l+s+s1}{unit\PYGZus{}price}\PYG{l+s+s1}{\PYGZsq{}}\PYG{p}{)}
\PYG{k}{def} \PYG{n+nf}{\PYGZus{}onchange\PYGZus{}price}\PYG{p}{(}\PYG{n+nb+bp}{self}\PYG{p}{)}\PYG{p}{:}
    \PYG{c+c1}{\PYGZsh{} set auto\PYGZhy{}changing field}
    \PYG{n+nb+bp}{self}\PYG{o}{.}\PYG{n}{price} \PYG{o}{=} \PYG{n+nb+bp}{self}\PYG{o}{.}\PYG{n}{amount} \PYG{o}{*} \PYG{n+nb+bp}{self}\PYG{o}{.}\PYG{n}{unit\PYGZus{}price}
    \PYG{c+c1}{\PYGZsh{} Can optionally return a warning and domains}
    \PYG{k}{return} \PYG{p}{\PYGZob{}}
        \PYG{l+s+s1}{\PYGZsq{}}\PYG{l+s+s1}{warning}\PYG{l+s+s1}{\PYGZsq{}}\PYG{p}{:} \PYG{p}{\PYGZob{}}
            \PYG{l+s+s1}{\PYGZsq{}}\PYG{l+s+s1}{title}\PYG{l+s+s1}{\PYGZsq{}}\PYG{p}{:} \PYG{l+s+s2}{\PYGZdq{}}\PYG{l+s+s2}{Something bad happened}\PYG{l+s+s2}{\PYGZdq{}}\PYG{p}{,}
            \PYG{l+s+s1}{\PYGZsq{}}\PYG{l+s+s1}{message}\PYG{l+s+s1}{\PYGZsq{}}\PYG{p}{:} \PYG{l+s+s2}{\PYGZdq{}}\PYG{l+s+s2}{It was very bad indeed}\PYG{l+s+s2}{\PYGZdq{}}\PYG{p}{,}
        \PYG{p}{\PYGZcb{}}
    \PYG{p}{\PYGZcb{}}
\end{sphinxVerbatim}

For computed fields, valued \sphinxcode{\sphinxupquote{onchange}} behavior is built-in as can be seen by
playing with the \sphinxstyleemphasis{Session} form: change the number of seats or participants, and
the \sphinxcode{\sphinxupquote{taken\_seats}} progressbar is automatically updated.

\begin{sphinxadmonition}{note}
Warning

Add an explicit onchange to warn about invalid values, like a negative
number of seats, or more participants than seats.
\sphinxstyleemphasis{openacademy/models.py}
\fvset{hllines={, 4, 5, 6, 7, 8, 9, 10, 11, 12, 13, 14, 15, 16, 17, 18, 19, 20,}}%
\begin{sphinxVerbatim}[commandchars=\\\{\}]
                \PYG{n}{r}\PYG{o}{.}\PYG{n}{taken\PYGZus{}seats} \PYG{o}{=} \PYG{l+m+mf}{0.0}
            \PYG{k}{else}\PYG{p}{:}
                \PYG{n}{r}\PYG{o}{.}\PYG{n}{taken\PYGZus{}seats} \PYG{o}{=} \PYG{l+m+mf}{100.0} \PYG{o}{*} \PYG{n+nb}{len}\PYG{p}{(}\PYG{n}{r}\PYG{o}{.}\PYG{n}{attendee\PYGZus{}ids}\PYG{p}{)} \PYG{o}{/} \PYG{n}{r}\PYG{o}{.}\PYG{n}{seats}

    \PYG{n+nd}{@api.onchange}\PYG{p}{(}\PYG{l+s+s1}{\PYGZsq{}}\PYG{l+s+s1}{seats}\PYG{l+s+s1}{\PYGZsq{}}\PYG{p}{,} \PYG{l+s+s1}{\PYGZsq{}}\PYG{l+s+s1}{attendee\PYGZus{}ids}\PYG{l+s+s1}{\PYGZsq{}}\PYG{p}{)}
    \PYG{k}{def} \PYG{n+nf}{\PYGZus{}verify\PYGZus{}valid\PYGZus{}seats}\PYG{p}{(}\PYG{n+nb+bp}{self}\PYG{p}{)}\PYG{p}{:}
        \PYG{k}{if} \PYG{n+nb+bp}{self}\PYG{o}{.}\PYG{n}{seats} \PYG{o}{\PYGZlt{}} \PYG{l+m+mi}{0}\PYG{p}{:}
            \PYG{k}{return} \PYG{p}{\PYGZob{}}
                \PYG{l+s+s1}{\PYGZsq{}}\PYG{l+s+s1}{warning}\PYG{l+s+s1}{\PYGZsq{}}\PYG{p}{:} \PYG{p}{\PYGZob{}}
                    \PYG{l+s+s1}{\PYGZsq{}}\PYG{l+s+s1}{title}\PYG{l+s+s1}{\PYGZsq{}}\PYG{p}{:} \PYG{l+s+s2}{\PYGZdq{}}\PYG{l+s+s2}{Incorrect }\PYG{l+s+s2}{\PYGZsq{}}\PYG{l+s+s2}{seats}\PYG{l+s+s2}{\PYGZsq{}}\PYG{l+s+s2}{ value}\PYG{l+s+s2}{\PYGZdq{}}\PYG{p}{,}
                    \PYG{l+s+s1}{\PYGZsq{}}\PYG{l+s+s1}{message}\PYG{l+s+s1}{\PYGZsq{}}\PYG{p}{:} \PYG{l+s+s2}{\PYGZdq{}}\PYG{l+s+s2}{The number of available seats may not be negative}\PYG{l+s+s2}{\PYGZdq{}}\PYG{p}{,}
                \PYG{p}{\PYGZcb{}}\PYG{p}{,}
            \PYG{p}{\PYGZcb{}}
        \PYG{k}{if} \PYG{n+nb+bp}{self}\PYG{o}{.}\PYG{n}{seats} \PYG{o}{\PYGZlt{}} \PYG{n+nb}{len}\PYG{p}{(}\PYG{n+nb+bp}{self}\PYG{o}{.}\PYG{n}{attendee\PYGZus{}ids}\PYG{p}{)}\PYG{p}{:}
            \PYG{k}{return} \PYG{p}{\PYGZob{}}
                \PYG{l+s+s1}{\PYGZsq{}}\PYG{l+s+s1}{warning}\PYG{l+s+s1}{\PYGZsq{}}\PYG{p}{:} \PYG{p}{\PYGZob{}}
                    \PYG{l+s+s1}{\PYGZsq{}}\PYG{l+s+s1}{title}\PYG{l+s+s1}{\PYGZsq{}}\PYG{p}{:} \PYG{l+s+s2}{\PYGZdq{}}\PYG{l+s+s2}{Too many attendees}\PYG{l+s+s2}{\PYGZdq{}}\PYG{p}{,}
                    \PYG{l+s+s1}{\PYGZsq{}}\PYG{l+s+s1}{message}\PYG{l+s+s1}{\PYGZsq{}}\PYG{p}{:} \PYG{l+s+s2}{\PYGZdq{}}\PYG{l+s+s2}{Increase seats or remove excess attendees}\PYG{l+s+s2}{\PYGZdq{}}\PYG{p}{,}
                \PYG{p}{\PYGZcb{}}\PYG{p}{,}
            \PYG{p}{\PYGZcb{}}
\end{sphinxVerbatim}
\end{sphinxadmonition}


\subsection{Model constraints}
\label{\detokenize{howtos/backend:model-constraints}}
Odoo provides two ways to set up automatically verified invariants:
{\hyperref[\detokenize{reference/orm:odoo.api.constrains}]{\sphinxcrossref{\sphinxcode{\sphinxupquote{Python constraints}}}}} and
{\hyperref[\detokenize{reference/orm:odoo.models.Model._sql_constraints}]{\sphinxcrossref{\sphinxcode{\sphinxupquote{SQL constraints}}}}}.

A Python constraint is defined as a method decorated with
{\hyperref[\detokenize{reference/orm:odoo.api.constrains}]{\sphinxcrossref{\sphinxcode{\sphinxupquote{constrains()}}}}}, and invoked on a recordset. The decorator
specifies which fields are involved in the constraint, so that the constraint is
automatically evaluated when one of them is modified. The method is expected to
raise an exception if its invariant is not satisfied:

\fvset{hllines={, ,}}%
\begin{sphinxVerbatim}[commandchars=\\\{\}]
\PYG{k+kn}{from} \PYG{n+nn}{odoo}\PYG{n+nn}{.}\PYG{n+nn}{exceptions} \PYG{k}{import} \PYG{n}{ValidationError}

\PYG{n+nd}{@api}\PYG{o}{.}\PYG{n}{constrains}\PYG{p}{(}\PYG{l+s+s1}{\PYGZsq{}}\PYG{l+s+s1}{age}\PYG{l+s+s1}{\PYGZsq{}}\PYG{p}{)}
\PYG{k}{def} \PYG{n+nf}{\PYGZus{}check\PYGZus{}something}\PYG{p}{(}\PYG{n+nb+bp}{self}\PYG{p}{)}\PYG{p}{:}
    \PYG{k}{for} \PYG{n}{record} \PYG{o+ow}{in} \PYG{n+nb+bp}{self}\PYG{p}{:}
        \PYG{k}{if} \PYG{n}{record}\PYG{o}{.}\PYG{n}{age} \PYG{o}{\PYGZgt{}} \PYG{l+m+mi}{20}\PYG{p}{:}
            \PYG{k}{raise} \PYG{n}{ValidationError}\PYG{p}{(}\PYG{l+s+s2}{\PYGZdq{}}\PYG{l+s+s2}{Your record is too old: }\PYG{l+s+si}{\PYGZpc{}s}\PYG{l+s+s2}{\PYGZdq{}} \PYG{o}{\PYGZpc{}} \PYG{n}{record}\PYG{o}{.}\PYG{n}{age}\PYG{p}{)}
    \PYG{c+c1}{\PYGZsh{} all records passed the test, don\PYGZsq{}t return anything}
\end{sphinxVerbatim}

\begin{sphinxadmonition}{note}
Add Python constraints

Add a constraint that checks that the instructor is not present in the
attendees of his/her own session.
\sphinxstyleemphasis{openacademy/models.py}
\fvset{hllines={, 3,}}%
\begin{sphinxVerbatim}[commandchars=\\\{\}]
\PYG{c+c1}{\PYGZsh{} \PYGZhy{}*\PYGZhy{} coding: utf\PYGZhy{}8 \PYGZhy{}*\PYGZhy{}}

\PYG{k+kn}{from} \PYG{n+nn}{odoo} \PYG{k+kn}{import} \PYG{n}{models}\PYG{p}{,} \PYG{n}{fields}\PYG{p}{,} \PYG{n}{api}\PYG{p}{,} \PYG{n}{exceptions}

\PYG{k}{class} \PYG{n+nc}{Course}\PYG{p}{(}\PYG{n}{models}\PYG{o}{.}\PYG{n}{Model}\PYG{p}{)}\PYG{p}{:}
    \PYG{n}{\PYGZus{}name} \PYG{o}{=} \PYG{l+s+s1}{\PYGZsq{}}\PYG{l+s+s1}{openacademy.course}\PYG{l+s+s1}{\PYGZsq{}}
\end{sphinxVerbatim}

\fvset{hllines={, 4, 5, 6, 7, 8, 9,}}%
\begin{sphinxVerbatim}[commandchars=\\\{\}]
                    \PYG{l+s+s1}{\PYGZsq{}}\PYG{l+s+s1}{message}\PYG{l+s+s1}{\PYGZsq{}}\PYG{p}{:} \PYG{l+s+s2}{\PYGZdq{}}\PYG{l+s+s2}{Increase seats or remove excess attendees}\PYG{l+s+s2}{\PYGZdq{}}\PYG{p}{,}
                \PYG{p}{\PYGZcb{}}\PYG{p}{,}
            \PYG{p}{\PYGZcb{}}

    \PYG{n+nd}{@api.constrains}\PYG{p}{(}\PYG{l+s+s1}{\PYGZsq{}}\PYG{l+s+s1}{instructor\PYGZus{}id}\PYG{l+s+s1}{\PYGZsq{}}\PYG{p}{,} \PYG{l+s+s1}{\PYGZsq{}}\PYG{l+s+s1}{attendee\PYGZus{}ids}\PYG{l+s+s1}{\PYGZsq{}}\PYG{p}{)}
    \PYG{k}{def} \PYG{n+nf}{\PYGZus{}check\PYGZus{}instructor\PYGZus{}not\PYGZus{}in\PYGZus{}attendees}\PYG{p}{(}\PYG{n+nb+bp}{self}\PYG{p}{)}\PYG{p}{:}
        \PYG{k}{for} \PYG{n}{r} \PYG{o+ow}{in} \PYG{n+nb+bp}{self}\PYG{p}{:}
            \PYG{k}{if} \PYG{n}{r}\PYG{o}{.}\PYG{n}{instructor\PYGZus{}id} \PYG{o+ow}{and} \PYG{n}{r}\PYG{o}{.}\PYG{n}{instructor\PYGZus{}id} \PYG{o+ow}{in} \PYG{n}{r}\PYG{o}{.}\PYG{n}{attendee\PYGZus{}ids}\PYG{p}{:}
                \PYG{k}{raise} \PYG{n}{exceptions}\PYG{o}{.}\PYG{n}{ValidationError}\PYG{p}{(}\PYG{l+s+s2}{\PYGZdq{}}\PYG{l+s+s2}{A session}\PYG{l+s+s2}{\PYGZsq{}}\PYG{l+s+s2}{s instructor can}\PYG{l+s+s2}{\PYGZsq{}}\PYG{l+s+s2}{t be an attendee}\PYG{l+s+s2}{\PYGZdq{}}\PYG{p}{)}
\end{sphinxVerbatim}
\end{sphinxadmonition}

SQL constraints are defined through the model attribute
{\hyperref[\detokenize{reference/orm:odoo.models.Model._sql_constraints}]{\sphinxcrossref{\sphinxcode{\sphinxupquote{\_sql\_constraints}}}}}. The latter is assigned to a list
of triples of strings \sphinxcode{\sphinxupquote{(name, sql\_definition, message)}}, where \sphinxcode{\sphinxupquote{name}} is a
valid SQL constraint name, \sphinxcode{\sphinxupquote{sql\_definition}} is a \sphinxhref{http://www.postgresql.org/docs/9.3/static/ddl-constraints.html}{table\_constraint} expression,
and \sphinxcode{\sphinxupquote{message}} is the error message.

\begin{sphinxadmonition}{note}
Add SQL constraints

With the help of \sphinxhref{http://www.postgresql.org/docs/9.3/static/ddl-constraints.html}{PostgreSQL’s documentation} , add the following
constraints:
\begin{enumerate}
\item {} 
CHECK that the course description and the course title are different

\item {} 
Make the Course’s name UNIQUE

\end{enumerate}
\sphinxstyleemphasis{openacademy/models.py}
\fvset{hllines={, 4, 5, 6, 7, 8, 9, 10, 11, 12, 13,}}%
\begin{sphinxVerbatim}[commandchars=\\\{\}]
    \PYG{n}{session\PYGZus{}ids} \PYG{o}{=} \PYG{n}{fields}\PYG{o}{.}\PYG{n}{One2many}\PYG{p}{(}
        \PYG{l+s+s1}{\PYGZsq{}}\PYG{l+s+s1}{openacademy.session}\PYG{l+s+s1}{\PYGZsq{}}\PYG{p}{,} \PYG{l+s+s1}{\PYGZsq{}}\PYG{l+s+s1}{course\PYGZus{}id}\PYG{l+s+s1}{\PYGZsq{}}\PYG{p}{,} \PYG{n}{string}\PYG{o}{=}\PYG{l+s+s2}{\PYGZdq{}}\PYG{l+s+s2}{Sessions}\PYG{l+s+s2}{\PYGZdq{}}\PYG{p}{)}

    \PYG{n}{\PYGZus{}sql\PYGZus{}constraints} \PYG{o}{=} \PYG{p}{[}
        \PYG{p}{(}\PYG{l+s+s1}{\PYGZsq{}}\PYG{l+s+s1}{name\PYGZus{}description\PYGZus{}check}\PYG{l+s+s1}{\PYGZsq{}}\PYG{p}{,}
         \PYG{l+s+s1}{\PYGZsq{}}\PYG{l+s+s1}{CHECK(name != description)}\PYG{l+s+s1}{\PYGZsq{}}\PYG{p}{,}
         \PYG{l+s+s2}{\PYGZdq{}}\PYG{l+s+s2}{The title of the course should not be the description}\PYG{l+s+s2}{\PYGZdq{}}\PYG{p}{)}\PYG{p}{,}

        \PYG{p}{(}\PYG{l+s+s1}{\PYGZsq{}}\PYG{l+s+s1}{name\PYGZus{}unique}\PYG{l+s+s1}{\PYGZsq{}}\PYG{p}{,}
         \PYG{l+s+s1}{\PYGZsq{}}\PYG{l+s+s1}{UNIQUE(name)}\PYG{l+s+s1}{\PYGZsq{}}\PYG{p}{,}
         \PYG{l+s+s2}{\PYGZdq{}}\PYG{l+s+s2}{The course title must be unique}\PYG{l+s+s2}{\PYGZdq{}}\PYG{p}{)}\PYG{p}{,}
    \PYG{p}{]}


\PYG{k}{class} \PYG{n+nc}{Session}\PYG{p}{(}\PYG{n}{models}\PYG{o}{.}\PYG{n}{Model}\PYG{p}{)}\PYG{p}{:}
    \PYG{n}{\PYGZus{}name} \PYG{o}{=} \PYG{l+s+s1}{\PYGZsq{}}\PYG{l+s+s1}{openacademy.session}\PYG{l+s+s1}{\PYGZsq{}}
\end{sphinxVerbatim}
\end{sphinxadmonition}

\begin{sphinxadmonition}{note}
Exercise 6 - Add a duplicate option

Since we added a constraint for the Course name uniqueness, it is not
possible to use the “duplicate” function anymore (\sphinxmenuselection{Form \(\rightarrow\)
Duplicate}).

Re-implement your own “copy” method which allows to duplicate the Course
object, changing the original name into “Copy of {[}original name{]}”.
\sphinxstyleemphasis{openacademy/models.py}
\fvset{hllines={, 4, 5, 6, 7, 8, 9, 10, 11, 12, 13, 14, 15, 16, 17,}}%
\begin{sphinxVerbatim}[commandchars=\\\{\}]
    \PYG{n}{session\PYGZus{}ids} \PYG{o}{=} \PYG{n}{fields}\PYG{o}{.}\PYG{n}{One2many}\PYG{p}{(}
        \PYG{l+s+s1}{\PYGZsq{}}\PYG{l+s+s1}{openacademy.session}\PYG{l+s+s1}{\PYGZsq{}}\PYG{p}{,} \PYG{l+s+s1}{\PYGZsq{}}\PYG{l+s+s1}{course\PYGZus{}id}\PYG{l+s+s1}{\PYGZsq{}}\PYG{p}{,} \PYG{n}{string}\PYG{o}{=}\PYG{l+s+s2}{\PYGZdq{}}\PYG{l+s+s2}{Sessions}\PYG{l+s+s2}{\PYGZdq{}}\PYG{p}{)}

    \PYG{n+nd}{@api.multi}
    \PYG{k}{def} \PYG{n+nf}{copy}\PYG{p}{(}\PYG{n+nb+bp}{self}\PYG{p}{,} \PYG{n}{default}\PYG{o}{=}\PYG{n+nb+bp}{None}\PYG{p}{)}\PYG{p}{:}
        \PYG{n}{default} \PYG{o}{=} \PYG{n+nb}{dict}\PYG{p}{(}\PYG{n}{default} \PYG{o+ow}{or} \PYG{p}{\PYGZob{}}\PYG{p}{\PYGZcb{}}\PYG{p}{)}

        \PYG{n}{copied\PYGZus{}count} \PYG{o}{=} \PYG{n+nb+bp}{self}\PYG{o}{.}\PYG{n}{search\PYGZus{}count}\PYG{p}{(}
            \PYG{p}{[}\PYG{p}{(}\PYG{l+s+s1}{\PYGZsq{}}\PYG{l+s+s1}{name}\PYG{l+s+s1}{\PYGZsq{}}\PYG{p}{,} \PYG{l+s+s1}{\PYGZsq{}}\PYG{l+s+s1}{=like}\PYG{l+s+s1}{\PYGZsq{}}\PYG{p}{,} \PYG{l+s+s2}{u\PYGZdq{}}\PYG{l+s+s2}{Copy of \PYGZob{}\PYGZcb{}}\PYG{l+s+s2}{\PYGZpc{}}\PYG{l+s+s2}{\PYGZdq{}}\PYG{o}{.}\PYG{n}{format}\PYG{p}{(}\PYG{n+nb+bp}{self}\PYG{o}{.}\PYG{n}{name}\PYG{p}{)}\PYG{p}{)}\PYG{p}{]}\PYG{p}{)}
        \PYG{k}{if} \PYG{o+ow}{not} \PYG{n}{copied\PYGZus{}count}\PYG{p}{:}
            \PYG{n}{new\PYGZus{}name} \PYG{o}{=} \PYG{l+s+s2}{u\PYGZdq{}}\PYG{l+s+s2}{Copy of \PYGZob{}\PYGZcb{}}\PYG{l+s+s2}{\PYGZdq{}}\PYG{o}{.}\PYG{n}{format}\PYG{p}{(}\PYG{n+nb+bp}{self}\PYG{o}{.}\PYG{n}{name}\PYG{p}{)}
        \PYG{k}{else}\PYG{p}{:}
            \PYG{n}{new\PYGZus{}name} \PYG{o}{=} \PYG{l+s+s2}{u\PYGZdq{}}\PYG{l+s+s2}{Copy of \PYGZob{}\PYGZcb{} (\PYGZob{}\PYGZcb{})}\PYG{l+s+s2}{\PYGZdq{}}\PYG{o}{.}\PYG{n}{format}\PYG{p}{(}\PYG{n+nb+bp}{self}\PYG{o}{.}\PYG{n}{name}\PYG{p}{,} \PYG{n}{copied\PYGZus{}count}\PYG{p}{)}

        \PYG{n}{default}\PYG{p}{[}\PYG{l+s+s1}{\PYGZsq{}}\PYG{l+s+s1}{name}\PYG{l+s+s1}{\PYGZsq{}}\PYG{p}{]} \PYG{o}{=} \PYG{n}{new\PYGZus{}name}
        \PYG{k}{return} \PYG{n+nb}{super}\PYG{p}{(}\PYG{n}{Course}\PYG{p}{,} \PYG{n+nb+bp}{self}\PYG{p}{)}\PYG{o}{.}\PYG{n}{copy}\PYG{p}{(}\PYG{n}{default}\PYG{p}{)}

    \PYG{n}{\PYGZus{}sql\PYGZus{}constraints} \PYG{o}{=} \PYG{p}{[}
        \PYG{p}{(}\PYG{l+s+s1}{\PYGZsq{}}\PYG{l+s+s1}{name\PYGZus{}description\PYGZus{}check}\PYG{l+s+s1}{\PYGZsq{}}\PYG{p}{,}
         \PYG{l+s+s1}{\PYGZsq{}}\PYG{l+s+s1}{CHECK(name != description)}\PYG{l+s+s1}{\PYGZsq{}}\PYG{p}{,}
\end{sphinxVerbatim}
\end{sphinxadmonition}


\subsection{Advanced Views}
\label{\detokenize{howtos/backend:advanced-views}}

\subsubsection{Tree views}
\label{\detokenize{howtos/backend:id3}}
Tree views can take supplementary attributes to further customize their
behavior:
\begin{description}
\item[{\sphinxcode{\sphinxupquote{decoration-\{\$name\}}}}] \leavevmode
allow changing the style of a row’s text based on the corresponding
record’s attributes.

Values are Python expressions. For each record, the expression is evaluated
with the record’s attributes as context values and if \sphinxcode{\sphinxupquote{true}}, the
corresponding style is applied to the row. Other context values are
\sphinxcode{\sphinxupquote{uid}} (the id of the current user) and \sphinxcode{\sphinxupquote{current\_date}} (the current date
as a string of the form \sphinxcode{\sphinxupquote{yyyy-MM-dd}}).

\sphinxcode{\sphinxupquote{\{\$name\}}} can be \sphinxcode{\sphinxupquote{bf}} (\sphinxcode{\sphinxupquote{font-weight: bold}}), \sphinxcode{\sphinxupquote{it}}
(\sphinxcode{\sphinxupquote{font-style: italic}}), or any \sphinxhref{https://getbootstrap.com/docs/3.3/components/\#available-variations}{bootstrap contextual color} (\sphinxcode{\sphinxupquote{danger}},
\sphinxcode{\sphinxupquote{info}}, \sphinxcode{\sphinxupquote{muted}}, \sphinxcode{\sphinxupquote{primary}}, \sphinxcode{\sphinxupquote{success}} or \sphinxcode{\sphinxupquote{warning}}).

\fvset{hllines={, ,}}%
\begin{sphinxVerbatim}[commandchars=\\\{\}]
\PYG{n+nt}{\PYGZlt{}tree} \PYG{n+na}{string=}\PYG{l+s}{\PYGZdq{}Idea Categories\PYGZdq{}} \PYG{n+na}{decoration\PYGZhy{}info=}\PYG{l+s}{\PYGZdq{}state==\PYGZsq{}draft\PYGZsq{}\PYGZdq{}}
    \PYG{n+na}{decoration\PYGZhy{}danger=}\PYG{l+s}{\PYGZdq{}state==\PYGZsq{}trashed\PYGZsq{}\PYGZdq{}}\PYG{n+nt}{\PYGZgt{}}
    \PYG{n+nt}{\PYGZlt{}field} \PYG{n+na}{name=}\PYG{l+s}{\PYGZdq{}name\PYGZdq{}}\PYG{n+nt}{/\PYGZgt{}}
    \PYG{n+nt}{\PYGZlt{}field} \PYG{n+na}{name=}\PYG{l+s}{\PYGZdq{}state\PYGZdq{}}\PYG{n+nt}{/\PYGZgt{}}
\PYG{n+nt}{\PYGZlt{}/tree\PYGZgt{}}
\end{sphinxVerbatim}

\item[{\sphinxcode{\sphinxupquote{editable}}}] \leavevmode
Either \sphinxcode{\sphinxupquote{"top"}} or \sphinxcode{\sphinxupquote{"bottom"}}. Makes the tree view editable in-place
(rather than having to go through the form view), the value is the
position where new rows appear.

\end{description}

\begin{sphinxadmonition}{note}
List coloring

Modify the Session tree view in such a way that sessions lasting less than
5 days are colored blue, and the ones lasting more than 15 days are
colored red.

Modify the session tree view:
\sphinxstyleemphasis{openacademy/views/openacademy.xml}
\fvset{hllines={, 4, 7,}}%
\begin{sphinxVerbatim}[commandchars=\\\{\}]
            \PYG{n+nt}{\PYGZlt{}field} \PYG{n+na}{name=}\PYG{l+s}{\PYGZdq{}name\PYGZdq{}}\PYG{n+nt}{\PYGZgt{}}session.tree\PYG{n+nt}{\PYGZlt{}/field\PYGZgt{}}
            \PYG{n+nt}{\PYGZlt{}field} \PYG{n+na}{name=}\PYG{l+s}{\PYGZdq{}model\PYGZdq{}}\PYG{n+nt}{\PYGZgt{}}openacademy.session\PYG{n+nt}{\PYGZlt{}/field\PYGZgt{}}
            \PYG{n+nt}{\PYGZlt{}field} \PYG{n+na}{name=}\PYG{l+s}{\PYGZdq{}arch\PYGZdq{}} \PYG{n+na}{type=}\PYG{l+s}{\PYGZdq{}xml\PYGZdq{}}\PYG{n+nt}{\PYGZgt{}}
                \PYG{n+nt}{\PYGZlt{}tree} \PYG{n+na}{string=}\PYG{l+s}{\PYGZdq{}Session Tree\PYGZdq{}} \PYG{n+na}{decoration\PYGZhy{}info=}\PYG{l+s}{\PYGZdq{}duration\PYGZam{}lt;5\PYGZdq{}} \PYG{n+na}{decoration\PYGZhy{}danger=}\PYG{l+s}{\PYGZdq{}duration\PYGZam{}gt;15\PYGZdq{}}\PYG{n+nt}{\PYGZgt{}}
                    \PYG{n+nt}{\PYGZlt{}field} \PYG{n+na}{name=}\PYG{l+s}{\PYGZdq{}name\PYGZdq{}}\PYG{n+nt}{/\PYGZgt{}}
                    \PYG{n+nt}{\PYGZlt{}field} \PYG{n+na}{name=}\PYG{l+s}{\PYGZdq{}course\PYGZus{}id\PYGZdq{}}\PYG{n+nt}{/\PYGZgt{}}
                    \PYG{n+nt}{\PYGZlt{}field} \PYG{n+na}{name=}\PYG{l+s}{\PYGZdq{}duration\PYGZdq{}} \PYG{n+na}{invisible=}\PYG{l+s}{\PYGZdq{}1\PYGZdq{}}\PYG{n+nt}{/\PYGZgt{}}
                    \PYG{n+nt}{\PYGZlt{}field} \PYG{n+na}{name=}\PYG{l+s}{\PYGZdq{}taken\PYGZus{}seats\PYGZdq{}} \PYG{n+na}{widget=}\PYG{l+s}{\PYGZdq{}progressbar\PYGZdq{}}\PYG{n+nt}{/\PYGZgt{}}
                \PYG{n+nt}{\PYGZlt{}/tree\PYGZgt{}}
            \PYG{n+nt}{\PYGZlt{}/field\PYGZgt{}}
\end{sphinxVerbatim}
\end{sphinxadmonition}


\subsubsection{Calendars}
\label{\detokenize{howtos/backend:calendars}}
Displays records as calendar events. Their root element is \sphinxcode{\sphinxupquote{\textless{}calendar\textgreater{}}} and
their most common attributes are:
\begin{description}
\item[{\sphinxcode{\sphinxupquote{color}}}] \leavevmode
The name of the field used for \sphinxstyleemphasis{color segmentation}. Colors are
automatically distributed to events, but events in the same color segment
(records which have the same value for their \sphinxcode{\sphinxupquote{@color}} field) will be
given the same color.

\item[{\sphinxcode{\sphinxupquote{date\_start}}}] \leavevmode
record’s field holding the start date/time for the event

\item[{\sphinxcode{\sphinxupquote{date\_stop}} (optional)}] \leavevmode
record’s field holding the end date/time for the event

\item[{\sphinxcode{\sphinxupquote{string}}}] \leavevmode
record’s field to define the label for each calendar event

\end{description}

\fvset{hllines={, ,}}%
\begin{sphinxVerbatim}[commandchars=\\\{\}]
\PYG{n+nt}{\PYGZlt{}calendar} \PYG{n+na}{string=}\PYG{l+s}{\PYGZdq{}Ideas\PYGZdq{}} \PYG{n+na}{date\PYGZus{}start=}\PYG{l+s}{\PYGZdq{}invent\PYGZus{}date\PYGZdq{}} \PYG{n+na}{color=}\PYG{l+s}{\PYGZdq{}inventor\PYGZus{}id\PYGZdq{}}\PYG{n+nt}{\PYGZgt{}}
    \PYG{n+nt}{\PYGZlt{}field} \PYG{n+na}{name=}\PYG{l+s}{\PYGZdq{}name\PYGZdq{}}\PYG{n+nt}{/\PYGZgt{}}
\PYG{n+nt}{\PYGZlt{}/calendar\PYGZgt{}}
\end{sphinxVerbatim}

\begin{sphinxadmonition}{note}
Calendar view

Add a Calendar view to the \sphinxstyleemphasis{Session} model enabling the user to view the
events associated to the Open Academy.
\begin{enumerate}
\item {} 
Add an \sphinxcode{\sphinxupquote{end\_date}} field computed from \sphinxcode{\sphinxupquote{start\_date}} and
\sphinxcode{\sphinxupquote{duration}}

\begin{sphinxadmonition}{tip}{Tip:}
the inverse function makes the field writable, and allows
moving the sessions (via drag and drop) in the calendar view
\end{sphinxadmonition}

\item {} 
Add a calendar view to the \sphinxstyleemphasis{Session} model

\item {} 
And add the calendar view to the \sphinxstyleemphasis{Session} model’s actions

\end{enumerate}
\sphinxstyleemphasis{openacademy/models.py}
\fvset{hllines={, 3,}}%
\begin{sphinxVerbatim}[commandchars=\\\{\}]
\PYG{c+c1}{\PYGZsh{} \PYGZhy{}*\PYGZhy{} coding: utf\PYGZhy{}8 \PYGZhy{}*\PYGZhy{}}

\PYG{k+kn}{from} \PYG{n+nn}{datetime} \PYG{k+kn}{import} \PYG{n}{timedelta}
\PYG{k+kn}{from} \PYG{n+nn}{odoo} \PYG{k+kn}{import} \PYG{n}{models}\PYG{p}{,} \PYG{n}{fields}\PYG{p}{,} \PYG{n}{api}\PYG{p}{,} \PYG{n}{exceptions}

\PYG{k}{class} \PYG{n+nc}{Course}\PYG{p}{(}\PYG{n}{models}\PYG{o}{.}\PYG{n}{Model}\PYG{p}{)}\PYG{p}{:}
\end{sphinxVerbatim}

\fvset{hllines={, 4, 5,}}%
\begin{sphinxVerbatim}[commandchars=\\\{\}]
    \PYG{n}{attendee\PYGZus{}ids} \PYG{o}{=} \PYG{n}{fields}\PYG{o}{.}\PYG{n}{Many2many}\PYG{p}{(}\PYG{l+s+s1}{\PYGZsq{}}\PYG{l+s+s1}{res.partner}\PYG{l+s+s1}{\PYGZsq{}}\PYG{p}{,} \PYG{n}{string}\PYG{o}{=}\PYG{l+s+s2}{\PYGZdq{}}\PYG{l+s+s2}{Attendees}\PYG{l+s+s2}{\PYGZdq{}}\PYG{p}{)}

    \PYG{n}{taken\PYGZus{}seats} \PYG{o}{=} \PYG{n}{fields}\PYG{o}{.}\PYG{n}{Float}\PYG{p}{(}\PYG{n}{string}\PYG{o}{=}\PYG{l+s+s2}{\PYGZdq{}}\PYG{l+s+s2}{Taken seats}\PYG{l+s+s2}{\PYGZdq{}}\PYG{p}{,} \PYG{n}{compute}\PYG{o}{=}\PYG{l+s+s1}{\PYGZsq{}}\PYG{l+s+s1}{\PYGZus{}taken\PYGZus{}seats}\PYG{l+s+s1}{\PYGZsq{}}\PYG{p}{)}
    \PYG{n}{end\PYGZus{}date} \PYG{o}{=} \PYG{n}{fields}\PYG{o}{.}\PYG{n}{Date}\PYG{p}{(}\PYG{n}{string}\PYG{o}{=}\PYG{l+s+s2}{\PYGZdq{}}\PYG{l+s+s2}{End Date}\PYG{l+s+s2}{\PYGZdq{}}\PYG{p}{,} \PYG{n}{store}\PYG{o}{=}\PYG{n+nb+bp}{True}\PYG{p}{,}
        \PYG{n}{compute}\PYG{o}{=}\PYG{l+s+s1}{\PYGZsq{}}\PYG{l+s+s1}{\PYGZus{}get\PYGZus{}end\PYGZus{}date}\PYG{l+s+s1}{\PYGZsq{}}\PYG{p}{,} \PYG{n}{inverse}\PYG{o}{=}\PYG{l+s+s1}{\PYGZsq{}}\PYG{l+s+s1}{\PYGZus{}set\PYGZus{}end\PYGZus{}date}\PYG{l+s+s1}{\PYGZsq{}}\PYG{p}{)}

    \PYG{n+nd}{@api.depends}\PYG{p}{(}\PYG{l+s+s1}{\PYGZsq{}}\PYG{l+s+s1}{seats}\PYG{l+s+s1}{\PYGZsq{}}\PYG{p}{,} \PYG{l+s+s1}{\PYGZsq{}}\PYG{l+s+s1}{attendee\PYGZus{}ids}\PYG{l+s+s1}{\PYGZsq{}}\PYG{p}{)}
    \PYG{k}{def} \PYG{n+nf}{\PYGZus{}taken\PYGZus{}seats}\PYG{p}{(}\PYG{n+nb+bp}{self}\PYG{p}{)}\PYG{p}{:}
\end{sphinxVerbatim}

\fvset{hllines={, 4, 5, 6, 7, 8, 9, 10, 11, 12, 13, 14, 15, 16, 17, 18, 19, 20, 21, 22, 23, 24, 25, 26, 27,}}%
\begin{sphinxVerbatim}[commandchars=\\\{\}]
                \PYG{p}{\PYGZcb{}}\PYG{p}{,}
            \PYG{p}{\PYGZcb{}}

    \PYG{n+nd}{@api.depends}\PYG{p}{(}\PYG{l+s+s1}{\PYGZsq{}}\PYG{l+s+s1}{start\PYGZus{}date}\PYG{l+s+s1}{\PYGZsq{}}\PYG{p}{,} \PYG{l+s+s1}{\PYGZsq{}}\PYG{l+s+s1}{duration}\PYG{l+s+s1}{\PYGZsq{}}\PYG{p}{)}
    \PYG{k}{def} \PYG{n+nf}{\PYGZus{}get\PYGZus{}end\PYGZus{}date}\PYG{p}{(}\PYG{n+nb+bp}{self}\PYG{p}{)}\PYG{p}{:}
        \PYG{k}{for} \PYG{n}{r} \PYG{o+ow}{in} \PYG{n+nb+bp}{self}\PYG{p}{:}
            \PYG{k}{if} \PYG{o+ow}{not} \PYG{p}{(}\PYG{n}{r}\PYG{o}{.}\PYG{n}{start\PYGZus{}date} \PYG{o+ow}{and} \PYG{n}{r}\PYG{o}{.}\PYG{n}{duration}\PYG{p}{)}\PYG{p}{:}
                \PYG{n}{r}\PYG{o}{.}\PYG{n}{end\PYGZus{}date} \PYG{o}{=} \PYG{n}{r}\PYG{o}{.}\PYG{n}{start\PYGZus{}date}
                \PYG{k}{continue}

            \PYG{c+c1}{\PYGZsh{} Add duration to start\PYGZus{}date, but: Monday + 5 days = Saturday, so}
            \PYG{c+c1}{\PYGZsh{} subtract one second to get on Friday instead}
            \PYG{n}{start} \PYG{o}{=} \PYG{n}{fields}\PYG{o}{.}\PYG{n}{Datetime}\PYG{o}{.}\PYG{n}{from\PYGZus{}string}\PYG{p}{(}\PYG{n}{r}\PYG{o}{.}\PYG{n}{start\PYGZus{}date}\PYG{p}{)}
            \PYG{n}{duration} \PYG{o}{=} \PYG{n}{timedelta}\PYG{p}{(}\PYG{n}{days}\PYG{o}{=}\PYG{n}{r}\PYG{o}{.}\PYG{n}{duration}\PYG{p}{,} \PYG{n}{seconds}\PYG{o}{=}\PYG{o}{\PYGZhy{}}\PYG{l+m+mi}{1}\PYG{p}{)}
            \PYG{n}{r}\PYG{o}{.}\PYG{n}{end\PYGZus{}date} \PYG{o}{=} \PYG{n}{start} \PYG{o}{+} \PYG{n}{duration}

    \PYG{k}{def} \PYG{n+nf}{\PYGZus{}set\PYGZus{}end\PYGZus{}date}\PYG{p}{(}\PYG{n+nb+bp}{self}\PYG{p}{)}\PYG{p}{:}
        \PYG{k}{for} \PYG{n}{r} \PYG{o+ow}{in} \PYG{n+nb+bp}{self}\PYG{p}{:}
            \PYG{k}{if} \PYG{o+ow}{not} \PYG{p}{(}\PYG{n}{r}\PYG{o}{.}\PYG{n}{start\PYGZus{}date} \PYG{o+ow}{and} \PYG{n}{r}\PYG{o}{.}\PYG{n}{end\PYGZus{}date}\PYG{p}{)}\PYG{p}{:}
                \PYG{k}{continue}

            \PYG{c+c1}{\PYGZsh{} Compute the difference between dates, but: Friday \PYGZhy{} Monday = 4 days,}
            \PYG{c+c1}{\PYGZsh{} so add one day to get 5 days instead}
            \PYG{n}{start\PYGZus{}date} \PYG{o}{=} \PYG{n}{fields}\PYG{o}{.}\PYG{n}{Datetime}\PYG{o}{.}\PYG{n}{from\PYGZus{}string}\PYG{p}{(}\PYG{n}{r}\PYG{o}{.}\PYG{n}{start\PYGZus{}date}\PYG{p}{)}
            \PYG{n}{end\PYGZus{}date} \PYG{o}{=} \PYG{n}{fields}\PYG{o}{.}\PYG{n}{Datetime}\PYG{o}{.}\PYG{n}{from\PYGZus{}string}\PYG{p}{(}\PYG{n}{r}\PYG{o}{.}\PYG{n}{end\PYGZus{}date}\PYG{p}{)}
            \PYG{n}{r}\PYG{o}{.}\PYG{n}{duration} \PYG{o}{=} \PYG{p}{(}\PYG{n}{end\PYGZus{}date} \PYG{o}{\PYGZhy{}} \PYG{n}{start\PYGZus{}date}\PYG{p}{)}\PYG{o}{.}\PYG{n}{days} \PYG{o}{+} \PYG{l+m+mi}{1}

    \PYG{n+nd}{@api.constrains}\PYG{p}{(}\PYG{l+s+s1}{\PYGZsq{}}\PYG{l+s+s1}{instructor\PYGZus{}id}\PYG{l+s+s1}{\PYGZsq{}}\PYG{p}{,} \PYG{l+s+s1}{\PYGZsq{}}\PYG{l+s+s1}{attendee\PYGZus{}ids}\PYG{l+s+s1}{\PYGZsq{}}\PYG{p}{)}
    \PYG{k}{def} \PYG{n+nf}{\PYGZus{}check\PYGZus{}instructor\PYGZus{}not\PYGZus{}in\PYGZus{}attendees}\PYG{p}{(}\PYG{n+nb+bp}{self}\PYG{p}{)}\PYG{p}{:}
        \PYG{k}{for} \PYG{n}{r} \PYG{o+ow}{in} \PYG{n+nb+bp}{self}\PYG{p}{:}
\end{sphinxVerbatim}
\sphinxstyleemphasis{openacademy/views/openacademy.xml}
\fvset{hllines={, 4, 5, 6, 7, 8, 9, 10, 11, 12, 13, 14, 19,}}%
\begin{sphinxVerbatim}[commandchars=\\\{\}]
            \PYG{n+nt}{\PYGZlt{}/field\PYGZgt{}}
        \PYG{n+nt}{\PYGZlt{}/record\PYGZgt{}}

        \PYG{c}{\PYGZlt{}!\PYGZhy{}\PYGZhy{}}\PYG{c}{ calendar view }\PYG{c}{\PYGZhy{}\PYGZhy{}\PYGZgt{}}
        \PYG{n+nt}{\PYGZlt{}record} \PYG{n+na}{model=}\PYG{l+s}{\PYGZdq{}ir.ui.view\PYGZdq{}} \PYG{n+na}{id=}\PYG{l+s}{\PYGZdq{}session\PYGZus{}calendar\PYGZus{}view\PYGZdq{}}\PYG{n+nt}{\PYGZgt{}}
            \PYG{n+nt}{\PYGZlt{}field} \PYG{n+na}{name=}\PYG{l+s}{\PYGZdq{}name\PYGZdq{}}\PYG{n+nt}{\PYGZgt{}}session.calendar\PYG{n+nt}{\PYGZlt{}/field\PYGZgt{}}
            \PYG{n+nt}{\PYGZlt{}field} \PYG{n+na}{name=}\PYG{l+s}{\PYGZdq{}model\PYGZdq{}}\PYG{n+nt}{\PYGZgt{}}openacademy.session\PYG{n+nt}{\PYGZlt{}/field\PYGZgt{}}
            \PYG{n+nt}{\PYGZlt{}field} \PYG{n+na}{name=}\PYG{l+s}{\PYGZdq{}arch\PYGZdq{}} \PYG{n+na}{type=}\PYG{l+s}{\PYGZdq{}xml\PYGZdq{}}\PYG{n+nt}{\PYGZgt{}}
                \PYG{n+nt}{\PYGZlt{}calendar} \PYG{n+na}{string=}\PYG{l+s}{\PYGZdq{}Session Calendar\PYGZdq{}} \PYG{n+na}{date\PYGZus{}start=}\PYG{l+s}{\PYGZdq{}start\PYGZus{}date\PYGZdq{}} \PYG{n+na}{date\PYGZus{}stop=}\PYG{l+s}{\PYGZdq{}end\PYGZus{}date\PYGZdq{}} \PYG{n+na}{color=}\PYG{l+s}{\PYGZdq{}instructor\PYGZus{}id\PYGZdq{}}\PYG{n+nt}{\PYGZgt{}}
                    \PYG{n+nt}{\PYGZlt{}field} \PYG{n+na}{name=}\PYG{l+s}{\PYGZdq{}name\PYGZdq{}}\PYG{n+nt}{/\PYGZgt{}}
                \PYG{n+nt}{\PYGZlt{}/calendar\PYGZgt{}}
            \PYG{n+nt}{\PYGZlt{}/field\PYGZgt{}}
        \PYG{n+nt}{\PYGZlt{}/record\PYGZgt{}}

        \PYG{n+nt}{\PYGZlt{}record} \PYG{n+na}{model=}\PYG{l+s}{\PYGZdq{}ir.actions.act\PYGZus{}window\PYGZdq{}} \PYG{n+na}{id=}\PYG{l+s}{\PYGZdq{}session\PYGZus{}list\PYGZus{}action\PYGZdq{}}\PYG{n+nt}{\PYGZgt{}}
            \PYG{n+nt}{\PYGZlt{}field} \PYG{n+na}{name=}\PYG{l+s}{\PYGZdq{}name\PYGZdq{}}\PYG{n+nt}{\PYGZgt{}}Sessions\PYG{n+nt}{\PYGZlt{}/field\PYGZgt{}}
            \PYG{n+nt}{\PYGZlt{}field} \PYG{n+na}{name=}\PYG{l+s}{\PYGZdq{}res\PYGZus{}model\PYGZdq{}}\PYG{n+nt}{\PYGZgt{}}openacademy.session\PYG{n+nt}{\PYGZlt{}/field\PYGZgt{}}
            \PYG{n+nt}{\PYGZlt{}field} \PYG{n+na}{name=}\PYG{l+s}{\PYGZdq{}view\PYGZus{}type\PYGZdq{}}\PYG{n+nt}{\PYGZgt{}}form\PYG{n+nt}{\PYGZlt{}/field\PYGZgt{}}
            \PYG{n+nt}{\PYGZlt{}field} \PYG{n+na}{name=}\PYG{l+s}{\PYGZdq{}view\PYGZus{}mode\PYGZdq{}}\PYG{n+nt}{\PYGZgt{}}tree,form,calendar\PYG{n+nt}{\PYGZlt{}/field\PYGZgt{}}
        \PYG{n+nt}{\PYGZlt{}/record\PYGZgt{}}

        \PYG{n+nt}{\PYGZlt{}menuitem} \PYG{n+na}{id=}\PYG{l+s}{\PYGZdq{}session\PYGZus{}menu\PYGZdq{}} \PYG{n+na}{name=}\PYG{l+s}{\PYGZdq{}Sessions\PYGZdq{}}
\end{sphinxVerbatim}
\end{sphinxadmonition}


\subsubsection{Search views}
\label{\detokenize{howtos/backend:id4}}
Search view \sphinxcode{\sphinxupquote{\textless{}field\textgreater{}}} elements can have a \sphinxcode{\sphinxupquote{@filter\_domain}} that overrides
the domain generated for searching on the given field. In the given domain,
\sphinxcode{\sphinxupquote{self}} represents the value entered by the user. In the example below, it is
used to search on both fields \sphinxcode{\sphinxupquote{name}} and \sphinxcode{\sphinxupquote{description}}.

Search views can also contain \sphinxcode{\sphinxupquote{\textless{}filter\textgreater{}}} elements, which act as toggles for
predefined searches. Filters must have one of the following attributes:
\begin{description}
\item[{\sphinxcode{\sphinxupquote{domain}}}] \leavevmode
add the given domain to the current search

\item[{\sphinxcode{\sphinxupquote{context}}}] \leavevmode
add some context to the current search; use the key \sphinxcode{\sphinxupquote{group\_by}} to group
results on the given field name

\end{description}

\fvset{hllines={, ,}}%
\begin{sphinxVerbatim}[commandchars=\\\{\}]
\PYG{n+nt}{\PYGZlt{}search} \PYG{n+na}{string=}\PYG{l+s}{\PYGZdq{}Ideas\PYGZdq{}}\PYG{n+nt}{\PYGZgt{}}
    \PYG{n+nt}{\PYGZlt{}field} \PYG{n+na}{name=}\PYG{l+s}{\PYGZdq{}name\PYGZdq{}}\PYG{n+nt}{/\PYGZgt{}}
    \PYG{n+nt}{\PYGZlt{}field} \PYG{n+na}{name=}\PYG{l+s}{\PYGZdq{}description\PYGZdq{}} \PYG{n+na}{string=}\PYG{l+s}{\PYGZdq{}Name and description\PYGZdq{}}
           \PYG{n+na}{filter\PYGZus{}domain=}\PYG{l+s}{\PYGZdq{}[\PYGZsq{}\textbar{}\PYGZsq{}, (\PYGZsq{}name\PYGZsq{}, \PYGZsq{}ilike\PYGZsq{}, self), (\PYGZsq{}description\PYGZsq{}, \PYGZsq{}ilike\PYGZsq{}, self)]\PYGZdq{}}\PYG{n+nt}{/\PYGZgt{}}
    \PYG{n+nt}{\PYGZlt{}field} \PYG{n+na}{name=}\PYG{l+s}{\PYGZdq{}inventor\PYGZus{}id\PYGZdq{}}\PYG{n+nt}{/\PYGZgt{}}
    \PYG{n+nt}{\PYGZlt{}field} \PYG{n+na}{name=}\PYG{l+s}{\PYGZdq{}country\PYGZus{}id\PYGZdq{}} \PYG{n+na}{widget=}\PYG{l+s}{\PYGZdq{}selection\PYGZdq{}}\PYG{n+nt}{/\PYGZgt{}}

    \PYG{n+nt}{\PYGZlt{}filter} \PYG{n+na}{name=}\PYG{l+s}{\PYGZdq{}my\PYGZus{}ideas\PYGZdq{}} \PYG{n+na}{string=}\PYG{l+s}{\PYGZdq{}My Ideas\PYGZdq{}}
            \PYG{n+na}{domain=}\PYG{l+s}{\PYGZdq{}[(\PYGZsq{}inventor\PYGZus{}id\PYGZsq{}, \PYGZsq{}=\PYGZsq{}, uid)]\PYGZdq{}}\PYG{n+nt}{/\PYGZgt{}}
    \PYG{n+nt}{\PYGZlt{}group} \PYG{n+na}{string=}\PYG{l+s}{\PYGZdq{}Group By\PYGZdq{}}\PYG{n+nt}{\PYGZgt{}}
        \PYG{n+nt}{\PYGZlt{}filter} \PYG{n+na}{name=}\PYG{l+s}{\PYGZdq{}group\PYGZus{}by\PYGZus{}inventor\PYGZdq{}} \PYG{n+na}{string=}\PYG{l+s}{\PYGZdq{}Inventor\PYGZdq{}}
                \PYG{n+na}{context=}\PYG{l+s}{\PYGZdq{}\PYGZob{}\PYGZsq{}group\PYGZus{}by\PYGZsq{}: \PYGZsq{}inventor\PYGZus{}id\PYGZsq{}\PYGZcb{}\PYGZdq{}}\PYG{n+nt}{/\PYGZgt{}}
    \PYG{n+nt}{\PYGZlt{}/group\PYGZgt{}}
\PYG{n+nt}{\PYGZlt{}/search\PYGZgt{}}
\end{sphinxVerbatim}

To use a non-default search view in an action, it should be linked using the
\sphinxcode{\sphinxupquote{search\_view\_id}} field of the action record.

The action can also set default values for search fields through its
\sphinxcode{\sphinxupquote{context}} field: context keys of the form
\sphinxcode{\sphinxupquote{search\_default\_\sphinxstyleemphasis{field\_name}}} will initialize \sphinxstyleemphasis{field\_name} with the
provided value. Search filters must have an optional \sphinxcode{\sphinxupquote{@name}} to have a
default and behave as booleans (they can only be enabled by default).

\begin{sphinxadmonition}{note}
Search views
\begin{enumerate}
\item {} 
Add a button to filter the courses for which the current user is the
responsible in the course search view. Make it selected by default.

\item {} 
Add a button to group courses by responsible user.

\end{enumerate}
\sphinxstyleemphasis{openacademy/views/openacademy.xml}
\fvset{hllines={, 4, 5, 6, 7, 8, 9,}}%
\begin{sphinxVerbatim}[commandchars=\\\{\}]
                \PYG{n+nt}{\PYGZlt{}search}\PYG{n+nt}{\PYGZgt{}}
                    \PYG{n+nt}{\PYGZlt{}field} \PYG{n+na}{name=}\PYG{l+s}{\PYGZdq{}name\PYGZdq{}}\PYG{n+nt}{/\PYGZgt{}}
                    \PYG{n+nt}{\PYGZlt{}field} \PYG{n+na}{name=}\PYG{l+s}{\PYGZdq{}description\PYGZdq{}}\PYG{n+nt}{/\PYGZgt{}}
                    \PYG{n+nt}{\PYGZlt{}filter} \PYG{n+na}{name=}\PYG{l+s}{\PYGZdq{}my\PYGZus{}courses\PYGZdq{}} \PYG{n+na}{string=}\PYG{l+s}{\PYGZdq{}My Courses\PYGZdq{}}
                            \PYG{n+na}{domain=}\PYG{l+s}{\PYGZdq{}[(\PYGZsq{}responsible\PYGZus{}id\PYGZsq{}, \PYGZsq{}=\PYGZsq{}, uid)]\PYGZdq{}}\PYG{n+nt}{/\PYGZgt{}}
                    \PYG{n+nt}{\PYGZlt{}group} \PYG{n+na}{string=}\PYG{l+s}{\PYGZdq{}Group By\PYGZdq{}}\PYG{n+nt}{\PYGZgt{}}
                        \PYG{n+nt}{\PYGZlt{}filter} \PYG{n+na}{name=}\PYG{l+s}{\PYGZdq{}by\PYGZus{}responsible\PYGZdq{}} \PYG{n+na}{string=}\PYG{l+s}{\PYGZdq{}Responsible\PYGZdq{}}
                                \PYG{n+na}{context=}\PYG{l+s}{\PYGZdq{}\PYGZob{}\PYGZsq{}group\PYGZus{}by\PYGZsq{}: \PYGZsq{}responsible\PYGZus{}id\PYGZsq{}\PYGZcb{}\PYGZdq{}}\PYG{n+nt}{/\PYGZgt{}}
                    \PYG{n+nt}{\PYGZlt{}/group\PYGZgt{}}
                \PYG{n+nt}{\PYGZlt{}/search\PYGZgt{}}
            \PYG{n+nt}{\PYGZlt{}/field\PYGZgt{}}
        \PYG{n+nt}{\PYGZlt{}/record\PYGZgt{}}
\end{sphinxVerbatim}

\fvset{hllines={, 4,}}%
\begin{sphinxVerbatim}[commandchars=\\\{\}]
            \PYG{n+nt}{\PYGZlt{}field} \PYG{n+na}{name=}\PYG{l+s}{\PYGZdq{}res\PYGZus{}model\PYGZdq{}}\PYG{n+nt}{\PYGZgt{}}openacademy.course\PYG{n+nt}{\PYGZlt{}/field\PYGZgt{}}
            \PYG{n+nt}{\PYGZlt{}field} \PYG{n+na}{name=}\PYG{l+s}{\PYGZdq{}view\PYGZus{}type\PYGZdq{}}\PYG{n+nt}{\PYGZgt{}}form\PYG{n+nt}{\PYGZlt{}/field\PYGZgt{}}
            \PYG{n+nt}{\PYGZlt{}field} \PYG{n+na}{name=}\PYG{l+s}{\PYGZdq{}view\PYGZus{}mode\PYGZdq{}}\PYG{n+nt}{\PYGZgt{}}tree,form\PYG{n+nt}{\PYGZlt{}/field\PYGZgt{}}
            \PYG{n+nt}{\PYGZlt{}field} \PYG{n+na}{name=}\PYG{l+s}{\PYGZdq{}context\PYGZdq{}} \PYG{n+na}{eval=}\PYG{l+s}{\PYGZdq{}\PYGZob{}\PYGZsq{}search\PYGZus{}default\PYGZus{}my\PYGZus{}courses\PYGZsq{}: 1\PYGZcb{}\PYGZdq{}}\PYG{n+nt}{/\PYGZgt{}}
            \PYG{n+nt}{\PYGZlt{}field} \PYG{n+na}{name=}\PYG{l+s}{\PYGZdq{}help\PYGZdq{}} \PYG{n+na}{type=}\PYG{l+s}{\PYGZdq{}html\PYGZdq{}}\PYG{n+nt}{\PYGZgt{}}
                \PYG{n+nt}{\PYGZlt{}p} \PYG{n+na}{class=}\PYG{l+s}{\PYGZdq{}oe\PYGZus{}view\PYGZus{}nocontent\PYGZus{}create\PYGZdq{}}\PYG{n+nt}{\PYGZgt{}}Create the first course
                \PYG{n+nt}{\PYGZlt{}/p\PYGZgt{}}
\end{sphinxVerbatim}
\end{sphinxadmonition}


\subsubsection{Gantt}
\label{\detokenize{howtos/backend:gantt}}
\begin{sphinxadmonition}{warning}{Warning:}
The gantt view requires the web\_gantt module which is present in
{\hyperref[\detokenize{setup/install:setup-install-editions}]{\sphinxcrossref{\DUrole{std,std-ref}{the enterprise edition}}}} version.
\end{sphinxadmonition}

Horizontal bar charts typically used to show project planning and advancement,
their root element is \sphinxcode{\sphinxupquote{\textless{}gantt\textgreater{}}}.

\fvset{hllines={, ,}}%
\begin{sphinxVerbatim}[commandchars=\\\{\}]
\PYG{n+nt}{\PYGZlt{}gantt} \PYG{n+na}{string=}\PYG{l+s}{\PYGZdq{}Ideas\PYGZdq{}}
       \PYG{n+na}{date\PYGZus{}start=}\PYG{l+s}{\PYGZdq{}invent\PYGZus{}date\PYGZdq{}}
       \PYG{n+na}{date\PYGZus{}stop=}\PYG{l+s}{\PYGZdq{}date\PYGZus{}finished\PYGZdq{}}
       \PYG{n+na}{progress=}\PYG{l+s}{\PYGZdq{}progress\PYGZdq{}}
       \PYG{n+na}{default\PYGZus{}group\PYGZus{}by=}\PYG{l+s}{\PYGZdq{}inventor\PYGZus{}id\PYGZdq{}} \PYG{n+nt}{/\PYGZgt{}}
\end{sphinxVerbatim}

\begin{sphinxadmonition}{note}
Gantt charts

Add a Gantt Chart enabling the user to view the sessions scheduling linked
to the Open Academy module. The sessions should be grouped by instructor.
\begin{enumerate}
\item {} 
Create a computed field expressing the session’s duration in hours

\item {} 
Add the gantt view’s definition, and add the gantt view to the
\sphinxstyleemphasis{Session} model’s action

\end{enumerate}
\sphinxstyleemphasis{openacademy/models.py}
\fvset{hllines={, 4, 5, 6,}}%
\begin{sphinxVerbatim}[commandchars=\\\{\}]
    \PYG{n}{end\PYGZus{}date} \PYG{o}{=} \PYG{n}{fields}\PYG{o}{.}\PYG{n}{Date}\PYG{p}{(}\PYG{n}{string}\PYG{o}{=}\PYG{l+s+s2}{\PYGZdq{}}\PYG{l+s+s2}{End Date}\PYG{l+s+s2}{\PYGZdq{}}\PYG{p}{,} \PYG{n}{store}\PYG{o}{=}\PYG{n+nb+bp}{True}\PYG{p}{,}
        \PYG{n}{compute}\PYG{o}{=}\PYG{l+s+s1}{\PYGZsq{}}\PYG{l+s+s1}{\PYGZus{}get\PYGZus{}end\PYGZus{}date}\PYG{l+s+s1}{\PYGZsq{}}\PYG{p}{,} \PYG{n}{inverse}\PYG{o}{=}\PYG{l+s+s1}{\PYGZsq{}}\PYG{l+s+s1}{\PYGZus{}set\PYGZus{}end\PYGZus{}date}\PYG{l+s+s1}{\PYGZsq{}}\PYG{p}{)}

    \PYG{n}{hours} \PYG{o}{=} \PYG{n}{fields}\PYG{o}{.}\PYG{n}{Float}\PYG{p}{(}\PYG{n}{string}\PYG{o}{=}\PYG{l+s+s2}{\PYGZdq{}}\PYG{l+s+s2}{Duration in hours}\PYG{l+s+s2}{\PYGZdq{}}\PYG{p}{,}
                         \PYG{n}{compute}\PYG{o}{=}\PYG{l+s+s1}{\PYGZsq{}}\PYG{l+s+s1}{\PYGZus{}get\PYGZus{}hours}\PYG{l+s+s1}{\PYGZsq{}}\PYG{p}{,} \PYG{n}{inverse}\PYG{o}{=}\PYG{l+s+s1}{\PYGZsq{}}\PYG{l+s+s1}{\PYGZus{}set\PYGZus{}hours}\PYG{l+s+s1}{\PYGZsq{}}\PYG{p}{)}

    \PYG{n+nd}{@api.depends}\PYG{p}{(}\PYG{l+s+s1}{\PYGZsq{}}\PYG{l+s+s1}{seats}\PYG{l+s+s1}{\PYGZsq{}}\PYG{p}{,} \PYG{l+s+s1}{\PYGZsq{}}\PYG{l+s+s1}{attendee\PYGZus{}ids}\PYG{l+s+s1}{\PYGZsq{}}\PYG{p}{)}
    \PYG{k}{def} \PYG{n+nf}{\PYGZus{}taken\PYGZus{}seats}\PYG{p}{(}\PYG{n+nb+bp}{self}\PYG{p}{)}\PYG{p}{:}
        \PYG{k}{for} \PYG{n}{r} \PYG{o+ow}{in} \PYG{n+nb+bp}{self}\PYG{p}{:}
\end{sphinxVerbatim}

\fvset{hllines={, 4, 5, 6, 7, 8, 9, 10, 11, 12,}}%
\begin{sphinxVerbatim}[commandchars=\\\{\}]
            \PYG{n}{end\PYGZus{}date} \PYG{o}{=} \PYG{n}{fields}\PYG{o}{.}\PYG{n}{Datetime}\PYG{o}{.}\PYG{n}{from\PYGZus{}string}\PYG{p}{(}\PYG{n}{r}\PYG{o}{.}\PYG{n}{end\PYGZus{}date}\PYG{p}{)}
            \PYG{n}{r}\PYG{o}{.}\PYG{n}{duration} \PYG{o}{=} \PYG{p}{(}\PYG{n}{end\PYGZus{}date} \PYG{o}{\PYGZhy{}} \PYG{n}{start\PYGZus{}date}\PYG{p}{)}\PYG{o}{.}\PYG{n}{days} \PYG{o}{+} \PYG{l+m+mi}{1}

    \PYG{n+nd}{@api.depends}\PYG{p}{(}\PYG{l+s+s1}{\PYGZsq{}}\PYG{l+s+s1}{duration}\PYG{l+s+s1}{\PYGZsq{}}\PYG{p}{)}
    \PYG{k}{def} \PYG{n+nf}{\PYGZus{}get\PYGZus{}hours}\PYG{p}{(}\PYG{n+nb+bp}{self}\PYG{p}{)}\PYG{p}{:}
        \PYG{k}{for} \PYG{n}{r} \PYG{o+ow}{in} \PYG{n+nb+bp}{self}\PYG{p}{:}
            \PYG{n}{r}\PYG{o}{.}\PYG{n}{hours} \PYG{o}{=} \PYG{n}{r}\PYG{o}{.}\PYG{n}{duration} \PYG{o}{*} \PYG{l+m+mi}{24}

    \PYG{k}{def} \PYG{n+nf}{\PYGZus{}set\PYGZus{}hours}\PYG{p}{(}\PYG{n+nb+bp}{self}\PYG{p}{)}\PYG{p}{:}
        \PYG{k}{for} \PYG{n}{r} \PYG{o+ow}{in} \PYG{n+nb+bp}{self}\PYG{p}{:}
            \PYG{n}{r}\PYG{o}{.}\PYG{n}{duration} \PYG{o}{=} \PYG{n}{r}\PYG{o}{.}\PYG{n}{hours} \PYG{o}{/} \PYG{l+m+mi}{24}

    \PYG{n+nd}{@api.constrains}\PYG{p}{(}\PYG{l+s+s1}{\PYGZsq{}}\PYG{l+s+s1}{instructor\PYGZus{}id}\PYG{l+s+s1}{\PYGZsq{}}\PYG{p}{,} \PYG{l+s+s1}{\PYGZsq{}}\PYG{l+s+s1}{attendee\PYGZus{}ids}\PYG{l+s+s1}{\PYGZsq{}}\PYG{p}{)}
    \PYG{k}{def} \PYG{n+nf}{\PYGZus{}check\PYGZus{}instructor\PYGZus{}not\PYGZus{}in\PYGZus{}attendees}\PYG{p}{(}\PYG{n+nb+bp}{self}\PYG{p}{)}\PYG{p}{:}
        \PYG{k}{for} \PYG{n}{r} \PYG{o+ow}{in} \PYG{n+nb+bp}{self}\PYG{p}{:}
\end{sphinxVerbatim}
\sphinxstyleemphasis{openacademy/views/openacademy.xml}
\fvset{hllines={, 4, 5, 6, 7, 8, 9, 10, 11, 12, 13, 14, 15, 20,}}%
\begin{sphinxVerbatim}[commandchars=\\\{\}]
            \PYG{n+nt}{\PYGZlt{}/field\PYGZgt{}}
        \PYG{n+nt}{\PYGZlt{}/record\PYGZgt{}}

        \PYG{n+nt}{\PYGZlt{}record} \PYG{n+na}{model=}\PYG{l+s}{\PYGZdq{}ir.ui.view\PYGZdq{}} \PYG{n+na}{id=}\PYG{l+s}{\PYGZdq{}session\PYGZus{}gantt\PYGZus{}view\PYGZdq{}}\PYG{n+nt}{\PYGZgt{}}
            \PYG{n+nt}{\PYGZlt{}field} \PYG{n+na}{name=}\PYG{l+s}{\PYGZdq{}name\PYGZdq{}}\PYG{n+nt}{\PYGZgt{}}session.gantt\PYG{n+nt}{\PYGZlt{}/field\PYGZgt{}}
            \PYG{n+nt}{\PYGZlt{}field} \PYG{n+na}{name=}\PYG{l+s}{\PYGZdq{}model\PYGZdq{}}\PYG{n+nt}{\PYGZgt{}}openacademy.session\PYG{n+nt}{\PYGZlt{}/field\PYGZgt{}}
            \PYG{n+nt}{\PYGZlt{}field} \PYG{n+na}{name=}\PYG{l+s}{\PYGZdq{}arch\PYGZdq{}} \PYG{n+na}{type=}\PYG{l+s}{\PYGZdq{}xml\PYGZdq{}}\PYG{n+nt}{\PYGZgt{}}
                \PYG{n+nt}{\PYGZlt{}gantt} \PYG{n+na}{string=}\PYG{l+s}{\PYGZdq{}Session Gantt\PYGZdq{}}
                       \PYG{n+na}{date\PYGZus{}start=}\PYG{l+s}{\PYGZdq{}start\PYGZus{}date\PYGZdq{}} \PYG{n+na}{date\PYGZus{}delay=}\PYG{l+s}{\PYGZdq{}hours\PYGZdq{}}
                       \PYG{n+na}{default\PYGZus{}group\PYGZus{}by=}\PYG{l+s}{\PYGZsq{}instructor\PYGZus{}id\PYGZsq{}}\PYG{n+nt}{\PYGZgt{}}
                    \PYG{c}{\PYGZlt{}!\PYGZhy{}\PYGZhy{}}\PYG{c}{ \PYGZlt{}field name=\PYGZdq{}name\PYGZdq{}/\PYGZgt{} this is not required after Odoo 10.0 }\PYG{c}{\PYGZhy{}\PYGZhy{}\PYGZgt{}}
                \PYG{n+nt}{\PYGZlt{}/gantt\PYGZgt{}}
            \PYG{n+nt}{\PYGZlt{}/field\PYGZgt{}}
        \PYG{n+nt}{\PYGZlt{}/record\PYGZgt{}}

        \PYG{n+nt}{\PYGZlt{}record} \PYG{n+na}{model=}\PYG{l+s}{\PYGZdq{}ir.actions.act\PYGZus{}window\PYGZdq{}} \PYG{n+na}{id=}\PYG{l+s}{\PYGZdq{}session\PYGZus{}list\PYGZus{}action\PYGZdq{}}\PYG{n+nt}{\PYGZgt{}}
            \PYG{n+nt}{\PYGZlt{}field} \PYG{n+na}{name=}\PYG{l+s}{\PYGZdq{}name\PYGZdq{}}\PYG{n+nt}{\PYGZgt{}}Sessions\PYG{n+nt}{\PYGZlt{}/field\PYGZgt{}}
            \PYG{n+nt}{\PYGZlt{}field} \PYG{n+na}{name=}\PYG{l+s}{\PYGZdq{}res\PYGZus{}model\PYGZdq{}}\PYG{n+nt}{\PYGZgt{}}openacademy.session\PYG{n+nt}{\PYGZlt{}/field\PYGZgt{}}
            \PYG{n+nt}{\PYGZlt{}field} \PYG{n+na}{name=}\PYG{l+s}{\PYGZdq{}view\PYGZus{}type\PYGZdq{}}\PYG{n+nt}{\PYGZgt{}}form\PYG{n+nt}{\PYGZlt{}/field\PYGZgt{}}
            \PYG{n+nt}{\PYGZlt{}field} \PYG{n+na}{name=}\PYG{l+s}{\PYGZdq{}view\PYGZus{}mode\PYGZdq{}}\PYG{n+nt}{\PYGZgt{}}tree,form,calendar,gantt\PYG{n+nt}{\PYGZlt{}/field\PYGZgt{}}
        \PYG{n+nt}{\PYGZlt{}/record\PYGZgt{}}

        \PYG{n+nt}{\PYGZlt{}menuitem} \PYG{n+na}{id=}\PYG{l+s}{\PYGZdq{}session\PYGZus{}menu\PYGZdq{}} \PYG{n+na}{name=}\PYG{l+s}{\PYGZdq{}Sessions\PYGZdq{}}
\end{sphinxVerbatim}
\end{sphinxadmonition}


\subsubsection{Graph views}
\label{\detokenize{howtos/backend:graph-views}}
Graph views allow aggregated overview and analysis of models, their root
element is \sphinxcode{\sphinxupquote{\textless{}graph\textgreater{}}}.

\begin{sphinxadmonition}{note}{Note:}
Pivot views (element \sphinxcode{\sphinxupquote{\textless{}pivot\textgreater{}}}) a multidimensional table, allows the
selection of filers and dimensions to get the right aggregated dataset
before moving to a more graphical overview. The pivot view shares the same
content definition as graph views.
\end{sphinxadmonition}

Graph views have 4 display modes, the default mode is selected using the
\sphinxcode{\sphinxupquote{@type}} attribute.
\begin{description}
\item[{Bar (default)}] \leavevmode
a bar chart, the first dimension is used to define groups on the
horizontal axis, other dimensions define aggregated bars within each group.

By default bars are side-by-side, they can be stacked by using
\sphinxcode{\sphinxupquote{@stacked="True"}} on the \sphinxcode{\sphinxupquote{\textless{}graph\textgreater{}}}

\item[{Line}] \leavevmode
2-dimensional line chart

\item[{Pie}] \leavevmode
2-dimensional pie

\end{description}

Graph views contain \sphinxcode{\sphinxupquote{\textless{}field\textgreater{}}} with a mandatory \sphinxcode{\sphinxupquote{@type}} attribute taking
the values:
\begin{description}
\item[{\sphinxcode{\sphinxupquote{row}} (default)}] \leavevmode
the field should be aggregated by default

\item[{\sphinxcode{\sphinxupquote{measure}}}] \leavevmode
the field should be aggregated rather than grouped on

\end{description}

\fvset{hllines={, ,}}%
\begin{sphinxVerbatim}[commandchars=\\\{\}]
\PYG{n+nt}{\PYGZlt{}graph} \PYG{n+na}{string=}\PYG{l+s}{\PYGZdq{}Total idea score by Inventor\PYGZdq{}}\PYG{n+nt}{\PYGZgt{}}
    \PYG{n+nt}{\PYGZlt{}field} \PYG{n+na}{name=}\PYG{l+s}{\PYGZdq{}inventor\PYGZus{}id\PYGZdq{}}\PYG{n+nt}{/\PYGZgt{}}
    \PYG{n+nt}{\PYGZlt{}field} \PYG{n+na}{name=}\PYG{l+s}{\PYGZdq{}score\PYGZdq{}} \PYG{n+na}{type=}\PYG{l+s}{\PYGZdq{}measure\PYGZdq{}}\PYG{n+nt}{/\PYGZgt{}}
\PYG{n+nt}{\PYGZlt{}/graph\PYGZgt{}}
\end{sphinxVerbatim}

\begin{sphinxadmonition}{warning}{Warning:}
Graph views perform aggregations on database values, they do not work
with non-stored computed fields.
\end{sphinxadmonition}

\begin{sphinxadmonition}{note}
Graph view

Add a Graph view in the Session object that displays, for each course, the
number of attendees under the form of a bar chart.
\begin{enumerate}
\item {} 
Add the number of attendees as a stored computed field

\item {} 
Then add the relevant view

\end{enumerate}
\sphinxstyleemphasis{openacademy/models.py}
\fvset{hllines={, 4, 5, 6,}}%
\begin{sphinxVerbatim}[commandchars=\\\{\}]
    \PYG{n}{hours} \PYG{o}{=} \PYG{n}{fields}\PYG{o}{.}\PYG{n}{Float}\PYG{p}{(}\PYG{n}{string}\PYG{o}{=}\PYG{l+s+s2}{\PYGZdq{}}\PYG{l+s+s2}{Duration in hours}\PYG{l+s+s2}{\PYGZdq{}}\PYG{p}{,}
                         \PYG{n}{compute}\PYG{o}{=}\PYG{l+s+s1}{\PYGZsq{}}\PYG{l+s+s1}{\PYGZus{}get\PYGZus{}hours}\PYG{l+s+s1}{\PYGZsq{}}\PYG{p}{,} \PYG{n}{inverse}\PYG{o}{=}\PYG{l+s+s1}{\PYGZsq{}}\PYG{l+s+s1}{\PYGZus{}set\PYGZus{}hours}\PYG{l+s+s1}{\PYGZsq{}}\PYG{p}{)}

    \PYG{n}{attendees\PYGZus{}count} \PYG{o}{=} \PYG{n}{fields}\PYG{o}{.}\PYG{n}{Integer}\PYG{p}{(}
        \PYG{n}{string}\PYG{o}{=}\PYG{l+s+s2}{\PYGZdq{}}\PYG{l+s+s2}{Attendees count}\PYG{l+s+s2}{\PYGZdq{}}\PYG{p}{,} \PYG{n}{compute}\PYG{o}{=}\PYG{l+s+s1}{\PYGZsq{}}\PYG{l+s+s1}{\PYGZus{}get\PYGZus{}attendees\PYGZus{}count}\PYG{l+s+s1}{\PYGZsq{}}\PYG{p}{,} \PYG{n}{store}\PYG{o}{=}\PYG{n+nb+bp}{True}\PYG{p}{)}

    \PYG{n+nd}{@api.depends}\PYG{p}{(}\PYG{l+s+s1}{\PYGZsq{}}\PYG{l+s+s1}{seats}\PYG{l+s+s1}{\PYGZsq{}}\PYG{p}{,} \PYG{l+s+s1}{\PYGZsq{}}\PYG{l+s+s1}{attendee\PYGZus{}ids}\PYG{l+s+s1}{\PYGZsq{}}\PYG{p}{)}
    \PYG{k}{def} \PYG{n+nf}{\PYGZus{}taken\PYGZus{}seats}\PYG{p}{(}\PYG{n+nb+bp}{self}\PYG{p}{)}\PYG{p}{:}
        \PYG{k}{for} \PYG{n}{r} \PYG{o+ow}{in} \PYG{n+nb+bp}{self}\PYG{p}{:}
\end{sphinxVerbatim}

\fvset{hllines={, 4, 5, 6, 7, 8,}}%
\begin{sphinxVerbatim}[commandchars=\\\{\}]
        \PYG{k}{for} \PYG{n}{r} \PYG{o+ow}{in} \PYG{n+nb+bp}{self}\PYG{p}{:}
            \PYG{n}{r}\PYG{o}{.}\PYG{n}{duration} \PYG{o}{=} \PYG{n}{r}\PYG{o}{.}\PYG{n}{hours} \PYG{o}{/} \PYG{l+m+mi}{24}

    \PYG{n+nd}{@api.depends}\PYG{p}{(}\PYG{l+s+s1}{\PYGZsq{}}\PYG{l+s+s1}{attendee\PYGZus{}ids}\PYG{l+s+s1}{\PYGZsq{}}\PYG{p}{)}
    \PYG{k}{def} \PYG{n+nf}{\PYGZus{}get\PYGZus{}attendees\PYGZus{}count}\PYG{p}{(}\PYG{n+nb+bp}{self}\PYG{p}{)}\PYG{p}{:}
        \PYG{k}{for} \PYG{n}{r} \PYG{o+ow}{in} \PYG{n+nb+bp}{self}\PYG{p}{:}
            \PYG{n}{r}\PYG{o}{.}\PYG{n}{attendees\PYGZus{}count} \PYG{o}{=} \PYG{n+nb}{len}\PYG{p}{(}\PYG{n}{r}\PYG{o}{.}\PYG{n}{attendee\PYGZus{}ids}\PYG{p}{)}

    \PYG{n+nd}{@api.constrains}\PYG{p}{(}\PYG{l+s+s1}{\PYGZsq{}}\PYG{l+s+s1}{instructor\PYGZus{}id}\PYG{l+s+s1}{\PYGZsq{}}\PYG{p}{,} \PYG{l+s+s1}{\PYGZsq{}}\PYG{l+s+s1}{attendee\PYGZus{}ids}\PYG{l+s+s1}{\PYGZsq{}}\PYG{p}{)}
    \PYG{k}{def} \PYG{n+nf}{\PYGZus{}check\PYGZus{}instructor\PYGZus{}not\PYGZus{}in\PYGZus{}attendees}\PYG{p}{(}\PYG{n+nb+bp}{self}\PYG{p}{)}\PYG{p}{:}
        \PYG{k}{for} \PYG{n}{r} \PYG{o+ow}{in} \PYG{n+nb+bp}{self}\PYG{p}{:}
\end{sphinxVerbatim}
\sphinxstyleemphasis{openacademy/views/openacademy.xml}
\fvset{hllines={, 4, 5, 6, 7, 8, 9, 10, 11, 12, 13, 14, 19,}}%
\begin{sphinxVerbatim}[commandchars=\\\{\}]
            \PYG{n+nt}{\PYGZlt{}/field\PYGZgt{}}
        \PYG{n+nt}{\PYGZlt{}/record\PYGZgt{}}

        \PYG{n+nt}{\PYGZlt{}record} \PYG{n+na}{model=}\PYG{l+s}{\PYGZdq{}ir.ui.view\PYGZdq{}} \PYG{n+na}{id=}\PYG{l+s}{\PYGZdq{}openacademy\PYGZus{}session\PYGZus{}graph\PYGZus{}view\PYGZdq{}}\PYG{n+nt}{\PYGZgt{}}
            \PYG{n+nt}{\PYGZlt{}field} \PYG{n+na}{name=}\PYG{l+s}{\PYGZdq{}name\PYGZdq{}}\PYG{n+nt}{\PYGZgt{}}openacademy.session.graph\PYG{n+nt}{\PYGZlt{}/field\PYGZgt{}}
            \PYG{n+nt}{\PYGZlt{}field} \PYG{n+na}{name=}\PYG{l+s}{\PYGZdq{}model\PYGZdq{}}\PYG{n+nt}{\PYGZgt{}}openacademy.session\PYG{n+nt}{\PYGZlt{}/field\PYGZgt{}}
            \PYG{n+nt}{\PYGZlt{}field} \PYG{n+na}{name=}\PYG{l+s}{\PYGZdq{}arch\PYGZdq{}} \PYG{n+na}{type=}\PYG{l+s}{\PYGZdq{}xml\PYGZdq{}}\PYG{n+nt}{\PYGZgt{}}
                \PYG{n+nt}{\PYGZlt{}graph} \PYG{n+na}{string=}\PYG{l+s}{\PYGZdq{}Participations by Courses\PYGZdq{}}\PYG{n+nt}{\PYGZgt{}}
                    \PYG{n+nt}{\PYGZlt{}field} \PYG{n+na}{name=}\PYG{l+s}{\PYGZdq{}course\PYGZus{}id\PYGZdq{}}\PYG{n+nt}{/\PYGZgt{}}
                    \PYG{n+nt}{\PYGZlt{}field} \PYG{n+na}{name=}\PYG{l+s}{\PYGZdq{}attendees\PYGZus{}count\PYGZdq{}} \PYG{n+na}{type=}\PYG{l+s}{\PYGZdq{}measure\PYGZdq{}}\PYG{n+nt}{/\PYGZgt{}}
                \PYG{n+nt}{\PYGZlt{}/graph\PYGZgt{}}
            \PYG{n+nt}{\PYGZlt{}/field\PYGZgt{}}
        \PYG{n+nt}{\PYGZlt{}/record\PYGZgt{}}

        \PYG{n+nt}{\PYGZlt{}record} \PYG{n+na}{model=}\PYG{l+s}{\PYGZdq{}ir.actions.act\PYGZus{}window\PYGZdq{}} \PYG{n+na}{id=}\PYG{l+s}{\PYGZdq{}session\PYGZus{}list\PYGZus{}action\PYGZdq{}}\PYG{n+nt}{\PYGZgt{}}
            \PYG{n+nt}{\PYGZlt{}field} \PYG{n+na}{name=}\PYG{l+s}{\PYGZdq{}name\PYGZdq{}}\PYG{n+nt}{\PYGZgt{}}Sessions\PYG{n+nt}{\PYGZlt{}/field\PYGZgt{}}
            \PYG{n+nt}{\PYGZlt{}field} \PYG{n+na}{name=}\PYG{l+s}{\PYGZdq{}res\PYGZus{}model\PYGZdq{}}\PYG{n+nt}{\PYGZgt{}}openacademy.session\PYG{n+nt}{\PYGZlt{}/field\PYGZgt{}}
            \PYG{n+nt}{\PYGZlt{}field} \PYG{n+na}{name=}\PYG{l+s}{\PYGZdq{}view\PYGZus{}type\PYGZdq{}}\PYG{n+nt}{\PYGZgt{}}form\PYG{n+nt}{\PYGZlt{}/field\PYGZgt{}}
            \PYG{n+nt}{\PYGZlt{}field} \PYG{n+na}{name=}\PYG{l+s}{\PYGZdq{}view\PYGZus{}mode\PYGZdq{}}\PYG{n+nt}{\PYGZgt{}}tree,form,calendar,gantt,graph\PYG{n+nt}{\PYGZlt{}/field\PYGZgt{}}
        \PYG{n+nt}{\PYGZlt{}/record\PYGZgt{}}

        \PYG{n+nt}{\PYGZlt{}menuitem} \PYG{n+na}{id=}\PYG{l+s}{\PYGZdq{}session\PYGZus{}menu\PYGZdq{}} \PYG{n+na}{name=}\PYG{l+s}{\PYGZdq{}Sessions\PYGZdq{}}
\end{sphinxVerbatim}
\end{sphinxadmonition}


\subsubsection{Kanban}
\label{\detokenize{howtos/backend:kanban}}
Used to organize tasks, production processes, etc… their root element is
\sphinxcode{\sphinxupquote{\textless{}kanban\textgreater{}}}.

A kanban view shows a set of cards possibly grouped in columns. Each card
represents a record, and each column the values of an aggregation field.

For instance, project tasks may be organized by stage (each column is a
stage), or by responsible (each column is a user), and so on.

Kanban views define the structure of each card as a mix of form elements
(including basic HTML) and {\hyperref[\detokenize{reference/qweb:reference-qweb}]{\sphinxcrossref{\DUrole{std,std-ref}{QWeb}}}}.

\begin{sphinxadmonition}{note}
Kanban view

Add a Kanban view that displays sessions grouped by course (columns are
thus courses).
\begin{enumerate}
\item {} 
Add an integer \sphinxcode{\sphinxupquote{color}} field to the \sphinxstyleemphasis{Session} model

\item {} 
Add the kanban view and update the action

\end{enumerate}
\sphinxstyleemphasis{openacademy/models.py}
\fvset{hllines={, 4,}}%
\begin{sphinxVerbatim}[commandchars=\\\{\}]
    \PYG{n}{duration} \PYG{o}{=} \PYG{n}{fields}\PYG{o}{.}\PYG{n}{Float}\PYG{p}{(}\PYG{n}{digits}\PYG{o}{=}\PYG{p}{(}\PYG{l+m+mi}{6}\PYG{p}{,} \PYG{l+m+mi}{2}\PYG{p}{)}\PYG{p}{,} \PYG{n}{help}\PYG{o}{=}\PYG{l+s+s2}{\PYGZdq{}}\PYG{l+s+s2}{Duration in days}\PYG{l+s+s2}{\PYGZdq{}}\PYG{p}{)}
    \PYG{n}{seats} \PYG{o}{=} \PYG{n}{fields}\PYG{o}{.}\PYG{n}{Integer}\PYG{p}{(}\PYG{n}{string}\PYG{o}{=}\PYG{l+s+s2}{\PYGZdq{}}\PYG{l+s+s2}{Number of seats}\PYG{l+s+s2}{\PYGZdq{}}\PYG{p}{)}
    \PYG{n}{active} \PYG{o}{=} \PYG{n}{fields}\PYG{o}{.}\PYG{n}{Boolean}\PYG{p}{(}\PYG{n}{default}\PYG{o}{=}\PYG{n+nb+bp}{True}\PYG{p}{)}
    \PYG{n}{color} \PYG{o}{=} \PYG{n}{fields}\PYG{o}{.}\PYG{n}{Integer}\PYG{p}{(}\PYG{p}{)}

    \PYG{n}{instructor\PYGZus{}id} \PYG{o}{=} \PYG{n}{fields}\PYG{o}{.}\PYG{n}{Many2one}\PYG{p}{(}\PYG{l+s+s1}{\PYGZsq{}}\PYG{l+s+s1}{res.partner}\PYG{l+s+s1}{\PYGZsq{}}\PYG{p}{,} \PYG{n}{string}\PYG{o}{=}\PYG{l+s+s2}{\PYGZdq{}}\PYG{l+s+s2}{Instructor}\PYG{l+s+s2}{\PYGZdq{}}\PYG{p}{,}
        \PYG{n}{domain}\PYG{o}{=}\PYG{p}{[}\PYG{l+s+s1}{\PYGZsq{}}\PYG{l+s+s1}{\textbar{}}\PYG{l+s+s1}{\PYGZsq{}}\PYG{p}{,} \PYG{p}{(}\PYG{l+s+s1}{\PYGZsq{}}\PYG{l+s+s1}{instructor}\PYG{l+s+s1}{\PYGZsq{}}\PYG{p}{,} \PYG{l+s+s1}{\PYGZsq{}}\PYG{l+s+s1}{=}\PYG{l+s+s1}{\PYGZsq{}}\PYG{p}{,} \PYG{n+nb+bp}{True}\PYG{p}{)}\PYG{p}{,}
\end{sphinxVerbatim}
\sphinxstyleemphasis{openacademy/views/openacademy.xml}
\fvset{hllines={, 4, 5, 6, 7, 8, 9, 10, 11, 12, 13, 14, 15, 16, 17, 18, 19, 20, 21, 22, 23, 24, 25, 26, 27, 28, 29, 30, 31, 32, 33, 34, 35, 36, 37, 38, 39, 40, 41, 42, 43, 44, 45, 46, 47, 48, 49, 54,}}%
\begin{sphinxVerbatim}[commandchars=\\\{\}]
            \PYGZlt{}/field\PYGZgt{}
        \PYGZlt{}/record\PYGZgt{}

        \PYGZlt{}record model=\PYGZdq{}ir.ui.view\PYGZdq{} id=\PYGZdq{}view\PYGZus{}openacad\PYGZus{}session\PYGZus{}kanban\PYGZdq{}\PYGZgt{}
            \PYGZlt{}field name=\PYGZdq{}name\PYGZdq{}\PYGZgt{}openacad.session.kanban\PYGZlt{}/field\PYGZgt{}
            \PYGZlt{}field name=\PYGZdq{}model\PYGZdq{}\PYGZgt{}openacademy.session\PYGZlt{}/field\PYGZgt{}
            \PYGZlt{}field name=\PYGZdq{}arch\PYGZdq{} type=\PYGZdq{}xml\PYGZdq{}\PYGZgt{}
                \PYGZlt{}kanban default\PYGZus{}group\PYGZus{}by=\PYGZdq{}course\PYGZus{}id\PYGZdq{}\PYGZgt{}
                    \PYGZlt{}field name=\PYGZdq{}color\PYGZdq{}/\PYGZgt{}
                    \PYGZlt{}templates\PYGZgt{}
                        \PYGZlt{}t t\PYGZhy{}name=\PYGZdq{}kanban\PYGZhy{}box\PYGZdq{}\PYGZgt{}
                            \PYGZlt{}div
                                    t\PYGZhy{}attf\PYGZhy{}class=\PYGZdq{}oe\PYGZus{}kanban\PYGZus{}color\PYGZus{}\PYGZob{}\PYGZob{}kanban\PYGZus{}getcolor(record.color.raw\PYGZus{}value)\PYGZcb{}\PYGZcb{}
                                                  oe\PYGZus{}kanban\PYGZus{}global\PYGZus{}click\PYGZus{}edit oe\PYGZus{}semantic\PYGZus{}html\PYGZus{}override
                                                  oe\PYGZus{}kanban\PYGZus{}card \PYGZob{}\PYGZob{}record.group\PYGZus{}fancy==1 ? \PYGZsq{}oe\PYGZus{}kanban\PYGZus{}card\PYGZus{}fancy\PYGZsq{} : \PYGZsq{}\PYGZsq{}\PYGZcb{}\PYGZcb{}\PYGZdq{}\PYGZgt{}
                                \PYGZlt{}div class=\PYGZdq{}oe\PYGZus{}dropdown\PYGZus{}kanban\PYGZdq{}\PYGZgt{}
                                    \PYGZlt{}!\PYGZhy{}\PYGZhy{} dropdown menu \PYGZhy{}\PYGZhy{}\PYGZgt{}
                                    \PYGZlt{}div class=\PYGZdq{}oe\PYGZus{}dropdown\PYGZus{}toggle\PYGZdq{}\PYGZgt{}
                                        \PYGZlt{}i class=\PYGZdq{}fa fa\PYGZhy{}bars fa\PYGZhy{}lg\PYGZdq{}/\PYGZgt{}
                                        \PYGZlt{}ul class=\PYGZdq{}oe\PYGZus{}dropdown\PYGZus{}menu\PYGZdq{}\PYGZgt{}
                                            \PYGZlt{}li\PYGZgt{}
                                                \PYGZlt{}a type=\PYGZdq{}delete\PYGZdq{}\PYGZgt{}Delete\PYGZlt{}/a\PYGZgt{}
                                            \PYGZlt{}/li\PYGZgt{}
                                            \PYGZlt{}li\PYGZgt{}
                                                \PYGZlt{}ul class=\PYGZdq{}oe\PYGZus{}kanban\PYGZus{}colorpicker\PYGZdq{}
                                                    data\PYGZhy{}field=\PYGZdq{}color\PYGZdq{}/\PYGZgt{}
                                            \PYGZlt{}/li\PYGZgt{}
                                        \PYGZlt{}/ul\PYGZgt{}
                                    \PYGZlt{}/div\PYGZgt{}
                                    \PYGZlt{}div class=\PYGZdq{}oe\PYGZus{}clear\PYGZdq{}\PYGZgt{}\PYGZlt{}/div\PYGZgt{}
                                \PYGZlt{}/div\PYGZgt{}
                                \PYGZlt{}div t\PYGZhy{}attf\PYGZhy{}class=\PYGZdq{}oe\PYGZus{}kanban\PYGZus{}content\PYGZdq{}\PYGZgt{}
                                    \PYGZlt{}!\PYGZhy{}\PYGZhy{} title \PYGZhy{}\PYGZhy{}\PYGZgt{}
                                    Session name:
                                    \PYGZlt{}field name=\PYGZdq{}name\PYGZdq{}/\PYGZgt{}
                                    \PYGZlt{}br/\PYGZgt{}
                                    Start date:
                                    \PYGZlt{}field name=\PYGZdq{}start\PYGZus{}date\PYGZdq{}/\PYGZgt{}
                                    \PYGZlt{}br/\PYGZgt{}
                                    duration:
                                    \PYGZlt{}field name=\PYGZdq{}duration\PYGZdq{}/\PYGZgt{}
                                \PYGZlt{}/div\PYGZgt{}
                            \PYGZlt{}/div\PYGZgt{}
                        \PYGZlt{}/t\PYGZgt{}
                    \PYGZlt{}/templates\PYGZgt{}
                \PYGZlt{}/kanban\PYGZgt{}
            \PYGZlt{}/field\PYGZgt{}
        \PYGZlt{}/record\PYGZgt{}

        \PYGZlt{}record model=\PYGZdq{}ir.actions.act\PYGZus{}window\PYGZdq{} id=\PYGZdq{}session\PYGZus{}list\PYGZus{}action\PYGZdq{}\PYGZgt{}
            \PYGZlt{}field name=\PYGZdq{}name\PYGZdq{}\PYGZgt{}Sessions\PYGZlt{}/field\PYGZgt{}
            \PYGZlt{}field name=\PYGZdq{}res\PYGZus{}model\PYGZdq{}\PYGZgt{}openacademy.session\PYGZlt{}/field\PYGZgt{}
            \PYGZlt{}field name=\PYGZdq{}view\PYGZus{}type\PYGZdq{}\PYGZgt{}form\PYGZlt{}/field\PYGZgt{}
            \PYGZlt{}field name=\PYGZdq{}view\PYGZus{}mode\PYGZdq{}\PYGZgt{}tree,form,calendar,gantt,graph,kanban\PYGZlt{}/field\PYGZgt{}
        \PYGZlt{}/record\PYGZgt{}

        \PYGZlt{}menuitem id=\PYGZdq{}session\PYGZus{}menu\PYGZdq{} name=\PYGZdq{}Sessions\PYGZdq{}
\end{sphinxVerbatim}
\end{sphinxadmonition}


\subsection{Security}
\label{\detokenize{howtos/backend:security}}
Access control mechanisms must be configured to achieve a coherent security
policy.


\subsubsection{Group-based access control mechanisms}
\label{\detokenize{howtos/backend:group-based-access-control-mechanisms}}
Groups are created as normal records on the model \sphinxcode{\sphinxupquote{res.groups}}, and granted
menu access via menu definitions. However even without a menu, objects may
still be accessible indirectly, so actual object-level permissions (read,
write, create, unlink) must be defined for groups. They are usually inserted
via CSV files inside modules. It is also possible to restrict access to
specific fields on a view or object using the field’s groups attribute.


\subsubsection{Access rights}
\label{\detokenize{howtos/backend:access-rights}}
Access rights are defined as records of the model \sphinxcode{\sphinxupquote{ir.model.access}}. Each
access right is associated to a model, a group (or no group for global
access), and a set of permissions: read, write, create, unlink. Such access
rights are usually created by a CSV file named after its model:
\sphinxcode{\sphinxupquote{ir.model.access.csv}}.

\fvset{hllines={, ,}}%
\begin{sphinxVerbatim}[commandchars=\\\{\}]
id,name,model\PYGZus{}id/id,group\PYGZus{}id/id,perm\PYGZus{}read,perm\PYGZus{}write,perm\PYGZus{}create,perm\PYGZus{}unlink
access\PYGZus{}idea\PYGZus{}idea,idea.idea,model\PYGZus{}idea\PYGZus{}idea,base.group\PYGZus{}user,1,1,1,0
access\PYGZus{}idea\PYGZus{}vote,idea.vote,model\PYGZus{}idea\PYGZus{}vote,base.group\PYGZus{}user,1,1,1,0
\end{sphinxVerbatim}

\begin{sphinxadmonition}{note}
Add access control through the Odoo interface

Create a new user “John Smith”. Then create a group
“OpenAcademy / Session Read” with read access to the \sphinxstyleemphasis{Session} model.
\begin{enumerate}
\item {} 
Create a new user \sphinxstyleemphasis{John Smith} through
\sphinxmenuselection{Settings \(\rightarrow\) Users \(\rightarrow\) Users}

\item {} 
Create a new group \sphinxcode{\sphinxupquote{session\_read}} through
\sphinxmenuselection{Settings \(\rightarrow\) Users \(\rightarrow\) Groups}, it should have
read access on the \sphinxstyleemphasis{Session} model

\item {} 
Edit \sphinxstyleemphasis{John Smith} to make them a member of \sphinxcode{\sphinxupquote{session\_read}}

\item {} 
Log in as \sphinxstyleemphasis{John Smith} to check the access rights are correct

\end{enumerate}
\end{sphinxadmonition}

\begin{sphinxadmonition}{note}
Add access control through data files in your module

Using data files,
\begin{itemize}
\item {} 
Create a group \sphinxstyleemphasis{OpenAcademy / Manager} with full access to all
OpenAcademy models

\item {} 
Make \sphinxstyleemphasis{Session} and \sphinxstyleemphasis{Course} readable by all users

\end{itemize}
\begin{enumerate}
\item {} 
Create a new file \sphinxcode{\sphinxupquote{openacademy/security/security.xml}} to
hold the OpenAcademy Manager group

\item {} 
Edit the file \sphinxcode{\sphinxupquote{openacademy/security/ir.model.access.csv}} with
the access rights to the models

\item {} 
Finally update \sphinxcode{\sphinxupquote{openacademy/\_\_manifest\_\_.py}} to add the new data
files to it

\end{enumerate}
\sphinxstyleemphasis{openacademy/\_\_manifest\_\_.py}
\fvset{hllines={, 3, 4,}}%
\begin{sphinxVerbatim}[commandchars=\\\{\}]

    \PYG{c+c1}{\PYGZsh{} always loaded}
    \PYG{l+s+s1}{\PYGZsq{}}\PYG{l+s+s1}{data}\PYG{l+s+s1}{\PYGZsq{}}\PYG{p}{:} \PYG{p}{[}
        \PYG{l+s+s1}{\PYGZsq{}}\PYG{l+s+s1}{security/security.xml}\PYG{l+s+s1}{\PYGZsq{}}\PYG{p}{,}
        \PYG{l+s+s1}{\PYGZsq{}}\PYG{l+s+s1}{security/ir.model.access.csv}\PYG{l+s+s1}{\PYGZsq{}}\PYG{p}{,}
        \PYG{l+s+s1}{\PYGZsq{}}\PYG{l+s+s1}{templates.xml}\PYG{l+s+s1}{\PYGZsq{}}\PYG{p}{,}
        \PYG{l+s+s1}{\PYGZsq{}}\PYG{l+s+s1}{views/openacademy.xml}\PYG{l+s+s1}{\PYGZsq{}}\PYG{p}{,}
        \PYG{l+s+s1}{\PYGZsq{}}\PYG{l+s+s1}{views/partner.xml}\PYG{l+s+s1}{\PYGZsq{}}\PYG{p}{,}
\end{sphinxVerbatim}
\sphinxstyleemphasis{openacademy/security/ir.model.access.csv}
\fvset{hllines={, 2, 3, 4, 5,}}%
\begin{sphinxVerbatim}[commandchars=\\\{\}]
id,name,model\PYGZus{}id/id,group\PYGZus{}id/id,perm\PYGZus{}read,perm\PYGZus{}write,perm\PYGZus{}create,perm\PYGZus{}unlink
course\PYGZus{}manager,course manager,model\PYGZus{}openacademy\PYGZus{}course,group\PYGZus{}manager,1,1,1,1
session\PYGZus{}manager,session manager,model\PYGZus{}openacademy\PYGZus{}session,group\PYGZus{}manager,1,1,1,1
course\PYGZus{}read\PYGZus{}all,course all,model\PYGZus{}openacademy\PYGZus{}course,,1,0,0,0
session\PYGZus{}read\PYGZus{}all,session all,model\PYGZus{}openacademy\PYGZus{}session,,1,0,0,0
\end{sphinxVerbatim}
\sphinxstyleemphasis{openacademy/security/security.xml}
\fvset{hllines={, 1, 2, 3, 4, 5, 6, 7,}}%
\begin{sphinxVerbatim}[commandchars=\\\{\}]
\PYG{n+nt}{\PYGZlt{}odoo}\PYG{n+nt}{\PYGZgt{}}

        \PYG{n+nt}{\PYGZlt{}record} \PYG{n+na}{id=}\PYG{l+s}{\PYGZdq{}group\PYGZus{}manager\PYGZdq{}} \PYG{n+na}{model=}\PYG{l+s}{\PYGZdq{}res.groups\PYGZdq{}}\PYG{n+nt}{\PYGZgt{}}
            \PYG{n+nt}{\PYGZlt{}field} \PYG{n+na}{name=}\PYG{l+s}{\PYGZdq{}name\PYGZdq{}}\PYG{n+nt}{\PYGZgt{}}OpenAcademy / Manager\PYG{n+nt}{\PYGZlt{}/field\PYGZgt{}}
        \PYG{n+nt}{\PYGZlt{}/record\PYGZgt{}}

\PYG{n+nt}{\PYGZlt{}/odoo\PYGZgt{}}
\end{sphinxVerbatim}
\end{sphinxadmonition}


\subsubsection{Record rules}
\label{\detokenize{howtos/backend:record-rules}}
A record rule restricts the access rights to a subset of records of the given
model. A rule is a record of the model \sphinxcode{\sphinxupquote{ir.rule}}, and is associated to a
model, a number of groups (many2many field), permissions to which the
restriction applies, and a domain. The domain specifies to which records the
access rights are limited.

Here is an example of a rule that prevents the deletion of leads that are not
in state \sphinxcode{\sphinxupquote{cancel}}. Notice that the value of the field \sphinxcode{\sphinxupquote{groups}} must follow
the same convention as the method {\hyperref[\detokenize{reference/orm:odoo.models.Model.write}]{\sphinxcrossref{\sphinxcode{\sphinxupquote{write()}}}}} of the ORM.

\fvset{hllines={, ,}}%
\begin{sphinxVerbatim}[commandchars=\\\{\}]
\PYG{n+nt}{\PYGZlt{}record} \PYG{n+na}{id=}\PYG{l+s}{\PYGZdq{}delete\PYGZus{}cancelled\PYGZus{}only\PYGZdq{}} \PYG{n+na}{model=}\PYG{l+s}{\PYGZdq{}ir.rule\PYGZdq{}}\PYG{n+nt}{\PYGZgt{}}
    \PYG{n+nt}{\PYGZlt{}field} \PYG{n+na}{name=}\PYG{l+s}{\PYGZdq{}name\PYGZdq{}}\PYG{n+nt}{\PYGZgt{}}Only cancelled leads may be deleted\PYG{n+nt}{\PYGZlt{}/field\PYGZgt{}}
    \PYG{n+nt}{\PYGZlt{}field} \PYG{n+na}{name=}\PYG{l+s}{\PYGZdq{}model\PYGZus{}id\PYGZdq{}} \PYG{n+na}{ref=}\PYG{l+s}{\PYGZdq{}crm.model\PYGZus{}crm\PYGZus{}lead\PYGZdq{}}\PYG{n+nt}{/\PYGZgt{}}
    \PYG{n+nt}{\PYGZlt{}field} \PYG{n+na}{name=}\PYG{l+s}{\PYGZdq{}groups\PYGZdq{}} \PYG{n+na}{eval=}\PYG{l+s}{\PYGZdq{}[(4, ref(\PYGZsq{}sales\PYGZus{}team.group\PYGZus{}sale\PYGZus{}manager\PYGZsq{}))]\PYGZdq{}}\PYG{n+nt}{/\PYGZgt{}}
    \PYG{n+nt}{\PYGZlt{}field} \PYG{n+na}{name=}\PYG{l+s}{\PYGZdq{}perm\PYGZus{}read\PYGZdq{}} \PYG{n+na}{eval=}\PYG{l+s}{\PYGZdq{}0\PYGZdq{}}\PYG{n+nt}{/\PYGZgt{}}
    \PYG{n+nt}{\PYGZlt{}field} \PYG{n+na}{name=}\PYG{l+s}{\PYGZdq{}perm\PYGZus{}write\PYGZdq{}} \PYG{n+na}{eval=}\PYG{l+s}{\PYGZdq{}0\PYGZdq{}}\PYG{n+nt}{/\PYGZgt{}}
    \PYG{n+nt}{\PYGZlt{}field} \PYG{n+na}{name=}\PYG{l+s}{\PYGZdq{}perm\PYGZus{}create\PYGZdq{}} \PYG{n+na}{eval=}\PYG{l+s}{\PYGZdq{}0\PYGZdq{}}\PYG{n+nt}{/\PYGZgt{}}
    \PYG{n+nt}{\PYGZlt{}field} \PYG{n+na}{name=}\PYG{l+s}{\PYGZdq{}perm\PYGZus{}unlink\PYGZdq{}} \PYG{n+na}{eval=}\PYG{l+s}{\PYGZdq{}1\PYGZdq{}} \PYG{n+nt}{/\PYGZgt{}}
    \PYG{n+nt}{\PYGZlt{}field} \PYG{n+na}{name=}\PYG{l+s}{\PYGZdq{}domain\PYGZus{}force\PYGZdq{}}\PYG{n+nt}{\PYGZgt{}}[(\PYGZsq{}state\PYGZsq{},\PYGZsq{}=\PYGZsq{},\PYGZsq{}cancel\PYGZsq{})]\PYG{n+nt}{\PYGZlt{}/field\PYGZgt{}}
\PYG{n+nt}{\PYGZlt{}/record\PYGZgt{}}
\end{sphinxVerbatim}

\begin{sphinxadmonition}{note}
Record rule

Add a record rule for the model Course and the group
“OpenAcademy / Manager”, that restricts \sphinxcode{\sphinxupquote{write}} and \sphinxcode{\sphinxupquote{unlink}} accesses
to the responsible of a course. If a course has no responsible, all users
of the group must be able to modify it.

Create a new rule in \sphinxcode{\sphinxupquote{openacademy/security/security.xml}}:
\sphinxstyleemphasis{openacademy/security/security.xml}
\fvset{hllines={, 4, 5, 6, 7, 8, 9, 10, 11, 12, 13, 14, 15, 16, 17,}}%
\begin{sphinxVerbatim}[commandchars=\\\{\}]
        \PYG{n+nt}{\PYGZlt{}record} \PYG{n+na}{id=}\PYG{l+s}{\PYGZdq{}group\PYGZus{}manager\PYGZdq{}} \PYG{n+na}{model=}\PYG{l+s}{\PYGZdq{}res.groups\PYGZdq{}}\PYG{n+nt}{\PYGZgt{}}
            \PYG{n+nt}{\PYGZlt{}field} \PYG{n+na}{name=}\PYG{l+s}{\PYGZdq{}name\PYGZdq{}}\PYG{n+nt}{\PYGZgt{}}OpenAcademy / Manager\PYG{n+nt}{\PYGZlt{}/field\PYGZgt{}}
        \PYG{n+nt}{\PYGZlt{}/record\PYGZgt{}}
    
        \PYG{n+nt}{\PYGZlt{}record} \PYG{n+na}{id=}\PYG{l+s}{\PYGZdq{}only\PYGZus{}responsible\PYGZus{}can\PYGZus{}modify\PYGZdq{}} \PYG{n+na}{model=}\PYG{l+s}{\PYGZdq{}ir.rule\PYGZdq{}}\PYG{n+nt}{\PYGZgt{}}
            \PYG{n+nt}{\PYGZlt{}field} \PYG{n+na}{name=}\PYG{l+s}{\PYGZdq{}name\PYGZdq{}}\PYG{n+nt}{\PYGZgt{}}Only Responsible can modify Course\PYG{n+nt}{\PYGZlt{}/field\PYGZgt{}}
            \PYG{n+nt}{\PYGZlt{}field} \PYG{n+na}{name=}\PYG{l+s}{\PYGZdq{}model\PYGZus{}id\PYGZdq{}} \PYG{n+na}{ref=}\PYG{l+s}{\PYGZdq{}model\PYGZus{}openacademy\PYGZus{}course\PYGZdq{}}\PYG{n+nt}{/\PYGZgt{}}
            \PYG{n+nt}{\PYGZlt{}field} \PYG{n+na}{name=}\PYG{l+s}{\PYGZdq{}groups\PYGZdq{}} \PYG{n+na}{eval=}\PYG{l+s}{\PYGZdq{}[(4, ref(\PYGZsq{}openacademy.group\PYGZus{}manager\PYGZsq{}))]\PYGZdq{}}\PYG{n+nt}{/\PYGZgt{}}
            \PYG{n+nt}{\PYGZlt{}field} \PYG{n+na}{name=}\PYG{l+s}{\PYGZdq{}perm\PYGZus{}read\PYGZdq{}} \PYG{n+na}{eval=}\PYG{l+s}{\PYGZdq{}0\PYGZdq{}}\PYG{n+nt}{/\PYGZgt{}}
            \PYG{n+nt}{\PYGZlt{}field} \PYG{n+na}{name=}\PYG{l+s}{\PYGZdq{}perm\PYGZus{}write\PYGZdq{}} \PYG{n+na}{eval=}\PYG{l+s}{\PYGZdq{}1\PYGZdq{}}\PYG{n+nt}{/\PYGZgt{}}
            \PYG{n+nt}{\PYGZlt{}field} \PYG{n+na}{name=}\PYG{l+s}{\PYGZdq{}perm\PYGZus{}create\PYGZdq{}} \PYG{n+na}{eval=}\PYG{l+s}{\PYGZdq{}0\PYGZdq{}}\PYG{n+nt}{/\PYGZgt{}}
            \PYG{n+nt}{\PYGZlt{}field} \PYG{n+na}{name=}\PYG{l+s}{\PYGZdq{}perm\PYGZus{}unlink\PYGZdq{}} \PYG{n+na}{eval=}\PYG{l+s}{\PYGZdq{}1\PYGZdq{}}\PYG{n+nt}{/\PYGZgt{}}
            \PYG{n+nt}{\PYGZlt{}field} \PYG{n+na}{name=}\PYG{l+s}{\PYGZdq{}domain\PYGZus{}force\PYGZdq{}}\PYG{n+nt}{\PYGZgt{}}
                [\PYGZsq{}\textbar{}\PYGZsq{}, (\PYGZsq{}responsible\PYGZus{}id\PYGZsq{},\PYGZsq{}=\PYGZsq{},False),
                      (\PYGZsq{}responsible\PYGZus{}id\PYGZsq{},\PYGZsq{}=\PYGZsq{},user.id)]
            \PYG{n+nt}{\PYGZlt{}/field\PYGZgt{}}
        \PYG{n+nt}{\PYGZlt{}/record\PYGZgt{}}

\PYG{n+nt}{\PYGZlt{}/odoo\PYGZgt{}}
\end{sphinxVerbatim}
\end{sphinxadmonition}


\subsection{Wizards}
\label{\detokenize{howtos/backend:wizards}}
Wizards describe interactive sessions with the user (or dialog boxes) through
dynamic forms. A wizard is simply a model that extends the class
\sphinxcode{\sphinxupquote{TransientModel}} instead of
{\hyperref[\detokenize{reference/orm:odoo.models.Model}]{\sphinxcrossref{\sphinxcode{\sphinxupquote{Model}}}}}. The class
\sphinxcode{\sphinxupquote{TransientModel}} extends {\hyperref[\detokenize{reference/orm:odoo.models.Model}]{\sphinxcrossref{\sphinxcode{\sphinxupquote{Model}}}}}
and reuse all its existing mechanisms, with the following particularities:
\begin{itemize}
\item {} 
Wizard records are not meant to be persistent; they are automatically deleted
from the database after a certain time. This is why they are called
\sphinxstyleemphasis{transient}.

\item {} 
Wizard models do not require explicit access rights: users have all
permissions on wizard records.

\item {} 
Wizard records may refer to regular records or wizard records through many2one
fields, but regular records \sphinxstyleemphasis{cannot} refer to wizard records through a
many2one field.

\end{itemize}

We want to create a wizard that allow users to create attendees for a particular
session, or for a list of sessions at once.

\begin{sphinxadmonition}{note}
Define the wizard

Create a wizard model with a many2one relationship with the \sphinxstyleemphasis{Session}
model and a many2many relationship with the \sphinxstyleemphasis{Partner} model.

Add a new file \sphinxcode{\sphinxupquote{openacademy/wizard.py}}:
\sphinxstyleemphasis{openacademy/\_\_init\_\_.py}
\fvset{hllines={, 4,}}%
\begin{sphinxVerbatim}[commandchars=\\\{\}]
\PYG{k+kn}{from} \PYG{n+nn}{.} \PYG{k+kn}{import} \PYG{n}{controllers}
\PYG{k+kn}{from} \PYG{n+nn}{.} \PYG{k+kn}{import} \PYG{n}{models}
\PYG{k+kn}{from} \PYG{n+nn}{.} \PYG{k+kn}{import} \PYG{n}{partner}
\PYG{k+kn}{from} \PYG{n+nn}{.} \PYG{k+kn}{import} \PYG{n}{wizard}
\end{sphinxVerbatim}
\sphinxstyleemphasis{openacademy/wizard.py}
\fvset{hllines={, 1, 2, 3, 4, 5, 6, 7, 8, 9, 10,}}%
\begin{sphinxVerbatim}[commandchars=\\\{\}]
\PYG{c+c1}{\PYGZsh{} \PYGZhy{}*\PYGZhy{} coding: utf\PYGZhy{}8 \PYGZhy{}*\PYGZhy{}}

\PYG{k+kn}{from} \PYG{n+nn}{odoo} \PYG{k+kn}{import} \PYG{n}{models}\PYG{p}{,} \PYG{n}{fields}\PYG{p}{,} \PYG{n}{api}

\PYG{k}{class} \PYG{n+nc}{Wizard}\PYG{p}{(}\PYG{n}{models}\PYG{o}{.}\PYG{n}{TransientModel}\PYG{p}{)}\PYG{p}{:}
    \PYG{n}{\PYGZus{}name} \PYG{o}{=} \PYG{l+s+s1}{\PYGZsq{}}\PYG{l+s+s1}{openacademy.wizard}\PYG{l+s+s1}{\PYGZsq{}}

    \PYG{n}{session\PYGZus{}id} \PYG{o}{=} \PYG{n}{fields}\PYG{o}{.}\PYG{n}{Many2one}\PYG{p}{(}\PYG{l+s+s1}{\PYGZsq{}}\PYG{l+s+s1}{openacademy.session}\PYG{l+s+s1}{\PYGZsq{}}\PYG{p}{,}
        \PYG{n}{string}\PYG{o}{=}\PYG{l+s+s2}{\PYGZdq{}}\PYG{l+s+s2}{Session}\PYG{l+s+s2}{\PYGZdq{}}\PYG{p}{,} \PYG{n}{required}\PYG{o}{=}\PYG{n+nb+bp}{True}\PYG{p}{)}
    \PYG{n}{attendee\PYGZus{}ids} \PYG{o}{=} \PYG{n}{fields}\PYG{o}{.}\PYG{n}{Many2many}\PYG{p}{(}\PYG{l+s+s1}{\PYGZsq{}}\PYG{l+s+s1}{res.partner}\PYG{l+s+s1}{\PYGZsq{}}\PYG{p}{,} \PYG{n}{string}\PYG{o}{=}\PYG{l+s+s2}{\PYGZdq{}}\PYG{l+s+s2}{Attendees}\PYG{l+s+s2}{\PYGZdq{}}\PYG{p}{)}
\end{sphinxVerbatim}
\end{sphinxadmonition}


\subsubsection{Launching wizards}
\label{\detokenize{howtos/backend:launching-wizards}}
Wizards are launched by \sphinxcode{\sphinxupquote{ir.actions.act\_window}} records, with the field
\sphinxcode{\sphinxupquote{target}} set to the value \sphinxcode{\sphinxupquote{new}}. The latter opens the wizard view into a
popup window. The action may be triggered by a menu item.

There is another way to launch the wizard: using an \sphinxcode{\sphinxupquote{ir.actions.act\_window}}
record like above, but with an extra field \sphinxcode{\sphinxupquote{src\_model}} that specifies in the
context of which model the action is available. The wizard will appear in the
contextual actions of the model, above the main view. Because of some internal
hooks in the ORM, such an action is declared in XML with the tag \sphinxcode{\sphinxupquote{act\_window}}.

\fvset{hllines={, ,}}%
\begin{sphinxVerbatim}[commandchars=\\\{\}]
\PYG{o}{\PYGZlt{}}\PYG{n}{act\PYGZus{}window} \PYG{n+nb}{id}\PYG{o}{=}\PYG{l+s+s2}{\PYGZdq{}}\PYG{l+s+s2}{launch\PYGZus{}the\PYGZus{}wizard}\PYG{l+s+s2}{\PYGZdq{}}
            \PYG{n}{name}\PYG{o}{=}\PYG{l+s+s2}{\PYGZdq{}}\PYG{l+s+s2}{Launch the Wizard}\PYG{l+s+s2}{\PYGZdq{}}
            \PYG{n}{src\PYGZus{}model}\PYG{o}{=}\PYG{l+s+s2}{\PYGZdq{}}\PYG{l+s+s2}{context.model.name}\PYG{l+s+s2}{\PYGZdq{}}
            \PYG{n}{res\PYGZus{}model}\PYG{o}{=}\PYG{l+s+s2}{\PYGZdq{}}\PYG{l+s+s2}{wizard.model.name}\PYG{l+s+s2}{\PYGZdq{}}
            \PYG{n}{view\PYGZus{}mode}\PYG{o}{=}\PYG{l+s+s2}{\PYGZdq{}}\PYG{l+s+s2}{form}\PYG{l+s+s2}{\PYGZdq{}}
            \PYG{n}{target}\PYG{o}{=}\PYG{l+s+s2}{\PYGZdq{}}\PYG{l+s+s2}{new}\PYG{l+s+s2}{\PYGZdq{}}
            \PYG{n}{key2}\PYG{o}{=}\PYG{l+s+s2}{\PYGZdq{}}\PYG{l+s+s2}{client\PYGZus{}action\PYGZus{}multi}\PYG{l+s+s2}{\PYGZdq{}}\PYG{o}{/}\PYG{o}{\PYGZgt{}}
\end{sphinxVerbatim}

Wizards use regular views and their buttons may use the attribute
\sphinxcode{\sphinxupquote{special="cancel"}} to close the wizard window without saving.

\begin{sphinxadmonition}{note}
Launch the wizard
\begin{enumerate}
\item {} 
Define a form view for the wizard.

\item {} 
Add the action to launch it in the context of the \sphinxstyleemphasis{Session} model.

\item {} 
Define a default value for the session field in the wizard; use the
context parameter \sphinxcode{\sphinxupquote{self.\_context}} to retrieve the current session.

\end{enumerate}
\sphinxstyleemphasis{openacademy/wizard.py}
\fvset{hllines={, 4, 5, 6, 8,}}%
\begin{sphinxVerbatim}[commandchars=\\\{\}]
\PYG{k}{class} \PYG{n+nc}{Wizard}\PYG{p}{(}\PYG{n}{models}\PYG{o}{.}\PYG{n}{TransientModel}\PYG{p}{)}\PYG{p}{:}
    \PYG{n}{\PYGZus{}name} \PYG{o}{=} \PYG{l+s+s1}{\PYGZsq{}}\PYG{l+s+s1}{openacademy.wizard}\PYG{l+s+s1}{\PYGZsq{}}

    \PYG{k}{def} \PYG{n+nf}{\PYGZus{}default\PYGZus{}session}\PYG{p}{(}\PYG{n+nb+bp}{self}\PYG{p}{)}\PYG{p}{:}
        \PYG{k}{return} \PYG{n+nb+bp}{self}\PYG{o}{.}\PYG{n}{env}\PYG{p}{[}\PYG{l+s+s1}{\PYGZsq{}}\PYG{l+s+s1}{openacademy.session}\PYG{l+s+s1}{\PYGZsq{}}\PYG{p}{]}\PYG{o}{.}\PYG{n}{browse}\PYG{p}{(}\PYG{n+nb+bp}{self}\PYG{o}{.}\PYG{n}{\PYGZus{}context}\PYG{o}{.}\PYG{n}{get}\PYG{p}{(}\PYG{l+s+s1}{\PYGZsq{}}\PYG{l+s+s1}{active\PYGZus{}id}\PYG{l+s+s1}{\PYGZsq{}}\PYG{p}{)}\PYG{p}{)}

    \PYG{n}{session\PYGZus{}id} \PYG{o}{=} \PYG{n}{fields}\PYG{o}{.}\PYG{n}{Many2one}\PYG{p}{(}\PYG{l+s+s1}{\PYGZsq{}}\PYG{l+s+s1}{openacademy.session}\PYG{l+s+s1}{\PYGZsq{}}\PYG{p}{,}
        \PYG{n}{string}\PYG{o}{=}\PYG{l+s+s2}{\PYGZdq{}}\PYG{l+s+s2}{Session}\PYG{l+s+s2}{\PYGZdq{}}\PYG{p}{,} \PYG{n}{required}\PYG{o}{=}\PYG{n+nb+bp}{True}\PYG{p}{,} \PYG{n}{default}\PYG{o}{=}\PYG{n}{\PYGZus{}default\PYGZus{}session}\PYG{p}{)}
    \PYG{n}{attendee\PYGZus{}ids} \PYG{o}{=} \PYG{n}{fields}\PYG{o}{.}\PYG{n}{Many2many}\PYG{p}{(}\PYG{l+s+s1}{\PYGZsq{}}\PYG{l+s+s1}{res.partner}\PYG{l+s+s1}{\PYGZsq{}}\PYG{p}{,} \PYG{n}{string}\PYG{o}{=}\PYG{l+s+s2}{\PYGZdq{}}\PYG{l+s+s2}{Attendees}\PYG{l+s+s2}{\PYGZdq{}}\PYG{p}{)}
\end{sphinxVerbatim}
\sphinxstyleemphasis{openacademy/views/openacademy.xml}
\fvset{hllines={, 4, 5, 6, 7, 8, 9, 10, 11, 12, 13, 14, 15, 16, 17, 18, 19, 20, 21, 22, 23, 24,}}%
\begin{sphinxVerbatim}[commandchars=\\\{\}]
                  parent=\PYGZdq{}openacademy\PYGZus{}menu\PYGZdq{}
                  action=\PYGZdq{}session\PYGZus{}list\PYGZus{}action\PYGZdq{}/\PYGZgt{}

        \PYG{n+nt}{\PYGZlt{}record} \PYG{n+na}{model=}\PYG{l+s}{\PYGZdq{}ir.ui.view\PYGZdq{}} \PYG{n+na}{id=}\PYG{l+s}{\PYGZdq{}wizard\PYGZus{}form\PYGZus{}view\PYGZdq{}}\PYG{n+nt}{\PYGZgt{}}
            \PYG{n+nt}{\PYGZlt{}field} \PYG{n+na}{name=}\PYG{l+s}{\PYGZdq{}name\PYGZdq{}}\PYG{n+nt}{\PYGZgt{}}wizard.form\PYG{n+nt}{\PYGZlt{}/field\PYGZgt{}}
            \PYG{n+nt}{\PYGZlt{}field} \PYG{n+na}{name=}\PYG{l+s}{\PYGZdq{}model\PYGZdq{}}\PYG{n+nt}{\PYGZgt{}}openacademy.wizard\PYG{n+nt}{\PYGZlt{}/field\PYGZgt{}}
            \PYG{n+nt}{\PYGZlt{}field} \PYG{n+na}{name=}\PYG{l+s}{\PYGZdq{}arch\PYGZdq{}} \PYG{n+na}{type=}\PYG{l+s}{\PYGZdq{}xml\PYGZdq{}}\PYG{n+nt}{\PYGZgt{}}
                \PYG{n+nt}{\PYGZlt{}form} \PYG{n+na}{string=}\PYG{l+s}{\PYGZdq{}Add Attendees\PYGZdq{}}\PYG{n+nt}{\PYGZgt{}}
                    \PYG{n+nt}{\PYGZlt{}group}\PYG{n+nt}{\PYGZgt{}}
                        \PYG{n+nt}{\PYGZlt{}field} \PYG{n+na}{name=}\PYG{l+s}{\PYGZdq{}session\PYGZus{}id\PYGZdq{}}\PYG{n+nt}{/\PYGZgt{}}
                        \PYG{n+nt}{\PYGZlt{}field} \PYG{n+na}{name=}\PYG{l+s}{\PYGZdq{}attendee\PYGZus{}ids\PYGZdq{}}\PYG{n+nt}{/\PYGZgt{}}
                    \PYG{n+nt}{\PYGZlt{}/group\PYGZgt{}}
                \PYG{n+nt}{\PYGZlt{}/form\PYGZgt{}}
            \PYG{n+nt}{\PYGZlt{}/field\PYGZgt{}}
        \PYG{n+nt}{\PYGZlt{}/record\PYGZgt{}}

        \PYG{n+nt}{\PYGZlt{}act\PYGZus{}window} \PYG{n+na}{id=}\PYG{l+s}{\PYGZdq{}launch\PYGZus{}session\PYGZus{}wizard\PYGZdq{}}
                    \PYG{n+na}{name=}\PYG{l+s}{\PYGZdq{}Add Attendees\PYGZdq{}}
                    \PYG{n+na}{src\PYGZus{}model=}\PYG{l+s}{\PYGZdq{}openacademy.session\PYGZdq{}}
                    \PYG{n+na}{res\PYGZus{}model=}\PYG{l+s}{\PYGZdq{}openacademy.wizard\PYGZdq{}}
                    \PYG{n+na}{view\PYGZus{}mode=}\PYG{l+s}{\PYGZdq{}form\PYGZdq{}}
                    \PYG{n+na}{target=}\PYG{l+s}{\PYGZdq{}new\PYGZdq{}}
                    \PYG{n+na}{key2=}\PYG{l+s}{\PYGZdq{}client\PYGZus{}action\PYGZus{}multi\PYGZdq{}}\PYG{n+nt}{/\PYGZgt{}}

\PYG{n+nt}{\PYGZlt{}/odoo\PYGZgt{}}
\end{sphinxVerbatim}
\end{sphinxadmonition}

\begin{sphinxadmonition}{note}
Register attendees

Add buttons to the wizard, and implement the corresponding method for adding
the attendees to the given session.
\sphinxstyleemphasis{openacademy/views/openacademy.xml}
\fvset{hllines={, 4, 5, 6, 7, 8, 9,}}%
\begin{sphinxVerbatim}[commandchars=\\\{\}]
                        \PYG{n+nt}{\PYGZlt{}field} \PYG{n+na}{name=}\PYG{l+s}{\PYGZdq{}session\PYGZus{}id\PYGZdq{}}\PYG{n+nt}{/\PYGZgt{}}
                        \PYG{n+nt}{\PYGZlt{}field} \PYG{n+na}{name=}\PYG{l+s}{\PYGZdq{}attendee\PYGZus{}ids\PYGZdq{}}\PYG{n+nt}{/\PYGZgt{}}
                    \PYG{n+nt}{\PYGZlt{}/group\PYGZgt{}}
                    \PYG{n+nt}{\PYGZlt{}footer}\PYG{n+nt}{\PYGZgt{}}
                        \PYG{n+nt}{\PYGZlt{}button} \PYG{n+na}{name=}\PYG{l+s}{\PYGZdq{}subscribe\PYGZdq{}} \PYG{n+na}{type=}\PYG{l+s}{\PYGZdq{}object\PYGZdq{}}
                                \PYG{n+na}{string=}\PYG{l+s}{\PYGZdq{}Subscribe\PYGZdq{}} \PYG{n+na}{class=}\PYG{l+s}{\PYGZdq{}oe\PYGZus{}highlight\PYGZdq{}}\PYG{n+nt}{/\PYGZgt{}}
                        or
                        \PYG{n+nt}{\PYGZlt{}button} \PYG{n+na}{special=}\PYG{l+s}{\PYGZdq{}cancel\PYGZdq{}} \PYG{n+na}{string=}\PYG{l+s}{\PYGZdq{}Cancel\PYGZdq{}}\PYG{n+nt}{/\PYGZgt{}}
                    \PYG{n+nt}{\PYGZlt{}/footer\PYGZgt{}}
                \PYG{n+nt}{\PYGZlt{}/form\PYGZgt{}}
            \PYG{n+nt}{\PYGZlt{}/field\PYGZgt{}}
        \PYG{n+nt}{\PYGZlt{}/record\PYGZgt{}}
\end{sphinxVerbatim}
\sphinxstyleemphasis{openacademy/wizard.py}
\fvset{hllines={, 4, 5, 6, 7, 8,}}%
\begin{sphinxVerbatim}[commandchars=\\\{\}]
    \PYG{n}{session\PYGZus{}id} \PYG{o}{=} \PYG{n}{fields}\PYG{o}{.}\PYG{n}{Many2one}\PYG{p}{(}\PYG{l+s+s1}{\PYGZsq{}}\PYG{l+s+s1}{openacademy.session}\PYG{l+s+s1}{\PYGZsq{}}\PYG{p}{,}
        \PYG{n}{string}\PYG{o}{=}\PYG{l+s+s2}{\PYGZdq{}}\PYG{l+s+s2}{Session}\PYG{l+s+s2}{\PYGZdq{}}\PYG{p}{,} \PYG{n}{required}\PYG{o}{=}\PYG{n+nb+bp}{True}\PYG{p}{,} \PYG{n}{default}\PYG{o}{=}\PYG{n}{\PYGZus{}default\PYGZus{}session}\PYG{p}{)}
    \PYG{n}{attendee\PYGZus{}ids} \PYG{o}{=} \PYG{n}{fields}\PYG{o}{.}\PYG{n}{Many2many}\PYG{p}{(}\PYG{l+s+s1}{\PYGZsq{}}\PYG{l+s+s1}{res.partner}\PYG{l+s+s1}{\PYGZsq{}}\PYG{p}{,} \PYG{n}{string}\PYG{o}{=}\PYG{l+s+s2}{\PYGZdq{}}\PYG{l+s+s2}{Attendees}\PYG{l+s+s2}{\PYGZdq{}}\PYG{p}{)}

    \PYG{n+nd}{@api.multi}
    \PYG{k}{def} \PYG{n+nf}{subscribe}\PYG{p}{(}\PYG{n+nb+bp}{self}\PYG{p}{)}\PYG{p}{:}
        \PYG{n+nb+bp}{self}\PYG{o}{.}\PYG{n}{session\PYGZus{}id}\PYG{o}{.}\PYG{n}{attendee\PYGZus{}ids} \PYG{o}{\textbar{}}\PYG{o}{=} \PYG{n+nb+bp}{self}\PYG{o}{.}\PYG{n}{attendee\PYGZus{}ids}
        \PYG{k}{return} \PYG{p}{\PYGZob{}}\PYG{p}{\PYGZcb{}}
\end{sphinxVerbatim}
\end{sphinxadmonition}

\begin{sphinxadmonition}{note}
Register attendees to multiple sessions

Modify the wizard model so that attendees can be registered to multiple
sessions.
\sphinxstyleemphasis{openacademy/views/openacademy.xml}
\fvset{hllines={, 4,}}%
\begin{sphinxVerbatim}[commandchars=\\\{\}]
            \PYG{n+nt}{\PYGZlt{}field} \PYG{n+na}{name=}\PYG{l+s}{\PYGZdq{}arch\PYGZdq{}} \PYG{n+na}{type=}\PYG{l+s}{\PYGZdq{}xml\PYGZdq{}}\PYG{n+nt}{\PYGZgt{}}
                \PYG{n+nt}{\PYGZlt{}form} \PYG{n+na}{string=}\PYG{l+s}{\PYGZdq{}Add Attendees\PYGZdq{}}\PYG{n+nt}{\PYGZgt{}}
                    \PYG{n+nt}{\PYGZlt{}group}\PYG{n+nt}{\PYGZgt{}}
                        \PYG{n+nt}{\PYGZlt{}field} \PYG{n+na}{name=}\PYG{l+s}{\PYGZdq{}session\PYGZus{}ids\PYGZdq{}}\PYG{n+nt}{/\PYGZgt{}}
                        \PYG{n+nt}{\PYGZlt{}field} \PYG{n+na}{name=}\PYG{l+s}{\PYGZdq{}attendee\PYGZus{}ids\PYGZdq{}}\PYG{n+nt}{/\PYGZgt{}}
                    \PYG{n+nt}{\PYGZlt{}/group\PYGZgt{}}
                    \PYG{n+nt}{\PYGZlt{}footer}\PYG{n+nt}{\PYGZgt{}}
\end{sphinxVerbatim}
\sphinxstyleemphasis{openacademy/wizard.py}
\fvset{hllines={, 4, 5, 7, 8, 13, 14,}}%
\begin{sphinxVerbatim}[commandchars=\\\{\}]
\PYG{k}{class} \PYG{n+nc}{Wizard}\PYG{p}{(}\PYG{n}{models}\PYG{o}{.}\PYG{n}{TransientModel}\PYG{p}{)}\PYG{p}{:}
    \PYG{n}{\PYGZus{}name} \PYG{o}{=} \PYG{l+s+s1}{\PYGZsq{}}\PYG{l+s+s1}{openacademy.wizard}\PYG{l+s+s1}{\PYGZsq{}}

    \PYG{k}{def} \PYG{n+nf}{\PYGZus{}default\PYGZus{}sessions}\PYG{p}{(}\PYG{n+nb+bp}{self}\PYG{p}{)}\PYG{p}{:}
        \PYG{k}{return} \PYG{n+nb+bp}{self}\PYG{o}{.}\PYG{n}{env}\PYG{p}{[}\PYG{l+s+s1}{\PYGZsq{}}\PYG{l+s+s1}{openacademy.session}\PYG{l+s+s1}{\PYGZsq{}}\PYG{p}{]}\PYG{o}{.}\PYG{n}{browse}\PYG{p}{(}\PYG{n+nb+bp}{self}\PYG{o}{.}\PYG{n}{\PYGZus{}context}\PYG{o}{.}\PYG{n}{get}\PYG{p}{(}\PYG{l+s+s1}{\PYGZsq{}}\PYG{l+s+s1}{active\PYGZus{}ids}\PYG{l+s+s1}{\PYGZsq{}}\PYG{p}{)}\PYG{p}{)}

    \PYG{n}{session\PYGZus{}ids} \PYG{o}{=} \PYG{n}{fields}\PYG{o}{.}\PYG{n}{Many2many}\PYG{p}{(}\PYG{l+s+s1}{\PYGZsq{}}\PYG{l+s+s1}{openacademy.session}\PYG{l+s+s1}{\PYGZsq{}}\PYG{p}{,}
        \PYG{n}{string}\PYG{o}{=}\PYG{l+s+s2}{\PYGZdq{}}\PYG{l+s+s2}{Sessions}\PYG{l+s+s2}{\PYGZdq{}}\PYG{p}{,} \PYG{n}{required}\PYG{o}{=}\PYG{n+nb+bp}{True}\PYG{p}{,} \PYG{n}{default}\PYG{o}{=}\PYG{n}{\PYGZus{}default\PYGZus{}sessions}\PYG{p}{)}
    \PYG{n}{attendee\PYGZus{}ids} \PYG{o}{=} \PYG{n}{fields}\PYG{o}{.}\PYG{n}{Many2many}\PYG{p}{(}\PYG{l+s+s1}{\PYGZsq{}}\PYG{l+s+s1}{res.partner}\PYG{l+s+s1}{\PYGZsq{}}\PYG{p}{,} \PYG{n}{string}\PYG{o}{=}\PYG{l+s+s2}{\PYGZdq{}}\PYG{l+s+s2}{Attendees}\PYG{l+s+s2}{\PYGZdq{}}\PYG{p}{)}

    \PYG{n+nd}{@api.multi}
    \PYG{k}{def} \PYG{n+nf}{subscribe}\PYG{p}{(}\PYG{n+nb+bp}{self}\PYG{p}{)}\PYG{p}{:}
        \PYG{k}{for} \PYG{n}{session} \PYG{o+ow}{in} \PYG{n+nb+bp}{self}\PYG{o}{.}\PYG{n}{session\PYGZus{}ids}\PYG{p}{:}
            \PYG{n}{session}\PYG{o}{.}\PYG{n}{attendee\PYGZus{}ids} \PYG{o}{\textbar{}}\PYG{o}{=} \PYG{n+nb+bp}{self}\PYG{o}{.}\PYG{n}{attendee\PYGZus{}ids}
        \PYG{k}{return} \PYG{p}{\PYGZob{}}\PYG{p}{\PYGZcb{}}
\end{sphinxVerbatim}
\end{sphinxadmonition}


\subsection{Internationalization}
\label{\detokenize{howtos/backend:internationalization}}
Each module can provide its own translations within the i18n directory, by
having files named LANG.po where LANG is the locale code for the language, or
the language and country combination when they differ (e.g. pt.po or
pt\_BR.po). Translations will be loaded automatically by Odoo for all
enabled languages. Developers always use English when creating a module, then
export the module terms using Odoo’s gettext POT export feature
(\sphinxmenuselection{Settings \(\rightarrow\) Translations \(\rightarrow\) Import/Export \(\rightarrow\) Export
Translation} without specifying a language), to create the module template POT
file, and then derive the translated PO files. Many IDE’s have plugins or modes
for editing and merging PO/POT files.

\begin{sphinxadmonition}{tip}{Tip:}
The Portable Object files generated by Odoo are published on
\sphinxhref{https://www.transifex.com/odoo/public/}{Transifex}, making it
easy to translate the software.
\end{sphinxadmonition}

\fvset{hllines={, ,}}%
\begin{sphinxVerbatim}[commandchars=\\\{\}]
\textbar{}\PYGZhy{} idea/ \PYGZsh{} The module directory
   \textbar{}\PYGZhy{} i18n/ \PYGZsh{} Translation files
      \textbar{} \PYGZhy{} idea.pot \PYGZsh{} Translation Template (exported from Odoo)
      \textbar{} \PYGZhy{} fr.po \PYGZsh{} French translation
      \textbar{} \PYGZhy{} pt\PYGZus{}BR.po \PYGZsh{} Brazilian Portuguese translation
      \textbar{} (...)
\end{sphinxVerbatim}

\begin{sphinxadmonition}{tip}{Tip:}
By default Odoo’s POT export only extracts labels inside XML files or
inside field definitions in Python code, but any Python string can be
translated this way by surrounding it with the function \sphinxcode{\sphinxupquote{odoo.\_()}}
(e.g. \sphinxcode{\sphinxupquote{\_("Label")}})
\end{sphinxadmonition}

\begin{sphinxadmonition}{note}
Translate a module

Choose a second language for your Odoo installation. Translate your
module using the facilities provided by Odoo.
\begin{enumerate}
\item {} 
Create a directory \sphinxcode{\sphinxupquote{openacademy/i18n/}}

\item {} 
Install whichever language you want (
\sphinxmenuselection{Administration \(\rightarrow\) Translations \(\rightarrow\) Load an
Official Translation})

\item {} 
Synchronize translatable terms (\sphinxmenuselection{Administration \(\rightarrow\)
Translations \(\rightarrow\) Application Terms \(\rightarrow\) Synchronize Translations})

\item {} 
Create a template translation file by exporting (
\sphinxmenuselection{Administration \(\rightarrow\) Translations -\textgreater{} Import/Export
\(\rightarrow\) Export Translation}) without specifying a language, save in
\sphinxcode{\sphinxupquote{openacademy/i18n/}}

\item {} 
Create a translation file by exporting (
\sphinxmenuselection{Administration \(\rightarrow\) Translations \(\rightarrow\) Import/Export
\(\rightarrow\) Export Translation}) and specifying a language. Save it in
\sphinxcode{\sphinxupquote{openacademy/i18n/}}

\item {} 
Open the exported translation file (with a basic text editor or a
dedicated PO-file editor e.g. \sphinxhref{http://poedit.net}{POEdit} and translate the missing
terms

\item {} 
In \sphinxcode{\sphinxupquote{models.py}}, add an import statement for the function
\sphinxcode{\sphinxupquote{odoo.\_}} and mark missing strings as translatable

\item {} 
Repeat steps 3-6

\end{enumerate}
\sphinxstyleemphasis{openacademy/models.py}
\fvset{hllines={, 4,}}%
\begin{sphinxVerbatim}[commandchars=\\\{\}]
\PYG{c+c1}{\PYGZsh{} \PYGZhy{}*\PYGZhy{} coding: utf\PYGZhy{}8 \PYGZhy{}*\PYGZhy{}}

\PYG{k+kn}{from} \PYG{n+nn}{datetime} \PYG{k+kn}{import} \PYG{n}{timedelta}
\PYG{k+kn}{from} \PYG{n+nn}{odoo} \PYG{k+kn}{import} \PYG{n}{models}\PYG{p}{,} \PYG{n}{fields}\PYG{p}{,} \PYG{n}{api}\PYG{p}{,} \PYG{n}{exceptions}\PYG{p}{,} \PYG{n}{\PYGZus{}}

\PYG{k}{class} \PYG{n+nc}{Course}\PYG{p}{(}\PYG{n}{models}\PYG{o}{.}\PYG{n}{Model}\PYG{p}{)}\PYG{p}{:}
    \PYG{n}{\PYGZus{}name} \PYG{o}{=} \PYG{l+s+s1}{\PYGZsq{}}\PYG{l+s+s1}{openacademy.course}\PYG{l+s+s1}{\PYGZsq{}}
\end{sphinxVerbatim}

\fvset{hllines={, 4, 6, 8,}}%
\begin{sphinxVerbatim}[commandchars=\\\{\}]
        \PYG{n}{default} \PYG{o}{=} \PYG{n+nb}{dict}\PYG{p}{(}\PYG{n}{default} \PYG{o+ow}{or} \PYG{p}{\PYGZob{}}\PYG{p}{\PYGZcb{}}\PYG{p}{)}

        \PYG{n}{copied\PYGZus{}count} \PYG{o}{=} \PYG{n+nb+bp}{self}\PYG{o}{.}\PYG{n}{search\PYGZus{}count}\PYG{p}{(}
            \PYG{p}{[}\PYG{p}{(}\PYG{l+s+s1}{\PYGZsq{}}\PYG{l+s+s1}{name}\PYG{l+s+s1}{\PYGZsq{}}\PYG{p}{,} \PYG{l+s+s1}{\PYGZsq{}}\PYG{l+s+s1}{=like}\PYG{l+s+s1}{\PYGZsq{}}\PYG{p}{,} \PYG{n}{\PYGZus{}}\PYG{p}{(}\PYG{l+s+s2}{u\PYGZdq{}}\PYG{l+s+s2}{Copy of \PYGZob{}\PYGZcb{}}\PYG{l+s+s2}{\PYGZpc{}}\PYG{l+s+s2}{\PYGZdq{}}\PYG{p}{)}\PYG{o}{.}\PYG{n}{format}\PYG{p}{(}\PYG{n+nb+bp}{self}\PYG{o}{.}\PYG{n}{name}\PYG{p}{)}\PYG{p}{)}\PYG{p}{]}\PYG{p}{)}
        \PYG{k}{if} \PYG{o+ow}{not} \PYG{n}{copied\PYGZus{}count}\PYG{p}{:}
            \PYG{n}{new\PYGZus{}name} \PYG{o}{=} \PYG{n}{\PYGZus{}}\PYG{p}{(}\PYG{l+s+s2}{u\PYGZdq{}}\PYG{l+s+s2}{Copy of \PYGZob{}\PYGZcb{}}\PYG{l+s+s2}{\PYGZdq{}}\PYG{p}{)}\PYG{o}{.}\PYG{n}{format}\PYG{p}{(}\PYG{n+nb+bp}{self}\PYG{o}{.}\PYG{n}{name}\PYG{p}{)}
        \PYG{k}{else}\PYG{p}{:}
            \PYG{n}{new\PYGZus{}name} \PYG{o}{=} \PYG{n}{\PYGZus{}}\PYG{p}{(}\PYG{l+s+s2}{u\PYGZdq{}}\PYG{l+s+s2}{Copy of \PYGZob{}\PYGZcb{} (\PYGZob{}\PYGZcb{})}\PYG{l+s+s2}{\PYGZdq{}}\PYG{p}{)}\PYG{o}{.}\PYG{n}{format}\PYG{p}{(}\PYG{n+nb+bp}{self}\PYG{o}{.}\PYG{n}{name}\PYG{p}{,} \PYG{n}{copied\PYGZus{}count}\PYG{p}{)}

        \PYG{n}{default}\PYG{p}{[}\PYG{l+s+s1}{\PYGZsq{}}\PYG{l+s+s1}{name}\PYG{l+s+s1}{\PYGZsq{}}\PYG{p}{]} \PYG{o}{=} \PYG{n}{new\PYGZus{}name}
        \PYG{k}{return} \PYG{n+nb}{super}\PYG{p}{(}\PYG{n}{Course}\PYG{p}{,} \PYG{n+nb+bp}{self}\PYG{p}{)}\PYG{o}{.}\PYG{n}{copy}\PYG{p}{(}\PYG{n}{default}\PYG{p}{)}
\end{sphinxVerbatim}

\fvset{hllines={, 4, 5, 11, 12,}}%
\begin{sphinxVerbatim}[commandchars=\\\{\}]
        \PYG{k}{if} \PYG{n+nb+bp}{self}\PYG{o}{.}\PYG{n}{seats} \PYG{o}{\PYGZlt{}} \PYG{l+m+mi}{0}\PYG{p}{:}
            \PYG{k}{return} \PYG{p}{\PYGZob{}}
                \PYG{l+s+s1}{\PYGZsq{}}\PYG{l+s+s1}{warning}\PYG{l+s+s1}{\PYGZsq{}}\PYG{p}{:} \PYG{p}{\PYGZob{}}
                    \PYG{l+s+s1}{\PYGZsq{}}\PYG{l+s+s1}{title}\PYG{l+s+s1}{\PYGZsq{}}\PYG{p}{:} \PYG{n}{\PYGZus{}}\PYG{p}{(}\PYG{l+s+s2}{\PYGZdq{}}\PYG{l+s+s2}{Incorrect }\PYG{l+s+s2}{\PYGZsq{}}\PYG{l+s+s2}{seats}\PYG{l+s+s2}{\PYGZsq{}}\PYG{l+s+s2}{ value}\PYG{l+s+s2}{\PYGZdq{}}\PYG{p}{)}\PYG{p}{,}
                    \PYG{l+s+s1}{\PYGZsq{}}\PYG{l+s+s1}{message}\PYG{l+s+s1}{\PYGZsq{}}\PYG{p}{:} \PYG{n}{\PYGZus{}}\PYG{p}{(}\PYG{l+s+s2}{\PYGZdq{}}\PYG{l+s+s2}{The number of available seats may not be negative}\PYG{l+s+s2}{\PYGZdq{}}\PYG{p}{)}\PYG{p}{,}
                \PYG{p}{\PYGZcb{}}\PYG{p}{,}
            \PYG{p}{\PYGZcb{}}
        \PYG{k}{if} \PYG{n+nb+bp}{self}\PYG{o}{.}\PYG{n}{seats} \PYG{o}{\PYGZlt{}} \PYG{n+nb}{len}\PYG{p}{(}\PYG{n+nb+bp}{self}\PYG{o}{.}\PYG{n}{attendee\PYGZus{}ids}\PYG{p}{)}\PYG{p}{:}
            \PYG{k}{return} \PYG{p}{\PYGZob{}}
                \PYG{l+s+s1}{\PYGZsq{}}\PYG{l+s+s1}{warning}\PYG{l+s+s1}{\PYGZsq{}}\PYG{p}{:} \PYG{p}{\PYGZob{}}
                    \PYG{l+s+s1}{\PYGZsq{}}\PYG{l+s+s1}{title}\PYG{l+s+s1}{\PYGZsq{}}\PYG{p}{:} \PYG{n}{\PYGZus{}}\PYG{p}{(}\PYG{l+s+s2}{\PYGZdq{}}\PYG{l+s+s2}{Too many attendees}\PYG{l+s+s2}{\PYGZdq{}}\PYG{p}{)}\PYG{p}{,}
                    \PYG{l+s+s1}{\PYGZsq{}}\PYG{l+s+s1}{message}\PYG{l+s+s1}{\PYGZsq{}}\PYG{p}{:} \PYG{n}{\PYGZus{}}\PYG{p}{(}\PYG{l+s+s2}{\PYGZdq{}}\PYG{l+s+s2}{Increase seats or remove excess attendees}\PYG{l+s+s2}{\PYGZdq{}}\PYG{p}{)}\PYG{p}{,}
                \PYG{p}{\PYGZcb{}}\PYG{p}{,}
            \PYG{p}{\PYGZcb{}}

\end{sphinxVerbatim}

\fvset{hllines={, 4,}}%
\begin{sphinxVerbatim}[commandchars=\\\{\}]
    \PYG{k}{def} \PYG{n+nf}{\PYGZus{}check\PYGZus{}instructor\PYGZus{}not\PYGZus{}in\PYGZus{}attendees}\PYG{p}{(}\PYG{n+nb+bp}{self}\PYG{p}{)}\PYG{p}{:}
        \PYG{k}{for} \PYG{n}{r} \PYG{o+ow}{in} \PYG{n+nb+bp}{self}\PYG{p}{:}
            \PYG{k}{if} \PYG{n}{r}\PYG{o}{.}\PYG{n}{instructor\PYGZus{}id} \PYG{o+ow}{and} \PYG{n}{r}\PYG{o}{.}\PYG{n}{instructor\PYGZus{}id} \PYG{o+ow}{in} \PYG{n}{r}\PYG{o}{.}\PYG{n}{attendee\PYGZus{}ids}\PYG{p}{:}
                \PYG{k}{raise} \PYG{n}{exceptions}\PYG{o}{.}\PYG{n}{ValidationError}\PYG{p}{(}\PYG{n}{\PYGZus{}}\PYG{p}{(}\PYG{l+s+s2}{\PYGZdq{}}\PYG{l+s+s2}{A session}\PYG{l+s+s2}{\PYGZsq{}}\PYG{l+s+s2}{s instructor can}\PYG{l+s+s2}{\PYGZsq{}}\PYG{l+s+s2}{t be an attendee}\PYG{l+s+s2}{\PYGZdq{}}\PYG{p}{)}\PYG{p}{)}
\end{sphinxVerbatim}
\end{sphinxadmonition}


\subsection{Reporting}
\label{\detokenize{howtos/backend:reporting}}

\subsubsection{Printed reports}
\label{\detokenize{howtos/backend:printed-reports}}
Odoo 11.0 uses a report engine based on {\hyperref[\detokenize{reference/qweb:reference-qweb}]{\sphinxcrossref{\DUrole{std,std-ref}{QWeb}}}},
\sphinxhref{http://getbootstrap.com}{Twitter Bootstrap} and \sphinxhref{http://wkhtmltopdf.org}{Wkhtmltopdf}.

A report is a combination two elements:
\begin{itemize}
\item {} 
an \sphinxcode{\sphinxupquote{ir.actions.report}}, for which a \sphinxcode{\sphinxupquote{\textless{}report\textgreater{}}} shortcut element is
provided, it sets up various basic parameters for the report (default
type, whether the report should be saved to the database after generation,…)

\fvset{hllines={, ,}}%
\begin{sphinxVerbatim}[commandchars=\\\{\}]
\PYG{n+nt}{\PYGZlt{}report}
    \PYG{n+na}{id=}\PYG{l+s}{\PYGZdq{}account\PYGZus{}invoices\PYGZdq{}}
    \PYG{n+na}{model=}\PYG{l+s}{\PYGZdq{}account.invoice\PYGZdq{}}
    \PYG{n+na}{string=}\PYG{l+s}{\PYGZdq{}Invoices\PYGZdq{}}
    \PYG{n+na}{report\PYGZus{}type=}\PYG{l+s}{\PYGZdq{}qweb\PYGZhy{}pdf\PYGZdq{}}
    \PYG{n+na}{name=}\PYG{l+s}{\PYGZdq{}account.report\PYGZus{}invoice\PYGZdq{}}
    \PYG{n+na}{file=}\PYG{l+s}{\PYGZdq{}account.report\PYGZus{}invoice\PYGZdq{}}
    \PYG{n+na}{attachment\PYGZus{}use=}\PYG{l+s}{\PYGZdq{}True\PYGZdq{}}
    \PYG{n+na}{attachment=}\PYG{l+s}{\PYGZdq{}(object.state in (\PYGZsq{}open\PYGZsq{},\PYGZsq{}paid\PYGZsq{})) and}
\PYG{l+s}{        (\PYGZsq{}INV\PYGZsq{}+(object.number or \PYGZsq{}\PYGZsq{}).replace(\PYGZsq{}/\PYGZsq{},\PYGZsq{}\PYGZsq{})+\PYGZsq{}.pdf\PYGZsq{})\PYGZdq{}}
\PYG{n+nt}{/\PYGZgt{}}
\end{sphinxVerbatim}

\item {} 
A standard {\hyperref[\detokenize{reference/views:reference-views-qweb}]{\sphinxcrossref{\DUrole{std,std-ref}{QWeb view}}}} for the actual report:

\fvset{hllines={, ,}}%
\begin{sphinxVerbatim}[commandchars=\\\{\}]
\PYG{n+nt}{\PYGZlt{}t} \PYG{n+na}{t\PYGZhy{}call=}\PYG{l+s}{\PYGZdq{}web.html\PYGZus{}container\PYGZdq{}}\PYG{n+nt}{\PYGZgt{}}
    \PYG{n+nt}{\PYGZlt{}t} \PYG{n+na}{t\PYGZhy{}foreach=}\PYG{l+s}{\PYGZdq{}docs\PYGZdq{}} \PYG{n+na}{t\PYGZhy{}as=}\PYG{l+s}{\PYGZdq{}o\PYGZdq{}}\PYG{n+nt}{\PYGZgt{}}
        \PYG{n+nt}{\PYGZlt{}t} \PYG{n+na}{t\PYGZhy{}call=}\PYG{l+s}{\PYGZdq{}web.external\PYGZus{}layout\PYGZdq{}}\PYG{n+nt}{\PYGZgt{}}
            \PYG{n+nt}{\PYGZlt{}div} \PYG{n+na}{class=}\PYG{l+s}{\PYGZdq{}page\PYGZdq{}}\PYG{n+nt}{\PYGZgt{}}
                \PYG{n+nt}{\PYGZlt{}h2}\PYG{n+nt}{\PYGZgt{}}Report title\PYG{n+nt}{\PYGZlt{}/h2\PYGZgt{}}
            \PYG{n+nt}{\PYGZlt{}/div\PYGZgt{}}
        \PYG{n+nt}{\PYGZlt{}/t\PYGZgt{}}
    \PYG{n+nt}{\PYGZlt{}/t\PYGZgt{}}
\PYG{n+nt}{\PYGZlt{}/t\PYGZgt{}}

the standard rendering context provides a number of elements, the most
important being:

{}`{}`docs{}`{}`
    the records for which the report is printed
{}`{}`user{}`{}`
    the user printing the report
\end{sphinxVerbatim}

\end{itemize}

Because reports are standard web pages, they are available through a URL and
output parameters can be manipulated through this URL, for instance the HTML
version of the \sphinxstyleemphasis{Invoice} report is available through
\sphinxurl{http://localhost:8069/report/html/account.report\_invoice/1} (if \sphinxcode{\sphinxupquote{account}} is
installed) and the PDF version through
\sphinxurl{http://localhost:8069/report/pdf/account.report\_invoice/1}.

\phantomsection\label{\detokenize{howtos/backend:reference-backend-reporting-printed-reports-pdf-without-styles}}
\begin{sphinxadmonition}{danger}{Danger:}
If it appears that your PDF report is missing the styles (i.e. the text
appears but the style/layout is different from the html version), probably
your \sphinxhref{http://wkhtmltopdf.org}{wkhtmltopdf} process cannot reach your web server to download them.

If you check your server logs and see that the CSS styles are not being
downloaded when generating a PDF report, most surely this is the problem.

The \sphinxhref{http://wkhtmltopdf.org}{wkhtmltopdf} process will use the \sphinxcode{\sphinxupquote{web.base.url}} system parameter as
the \sphinxstyleemphasis{root path} to all linked files, but this parameter is automatically
updated each time the Administrator is logged in. If your server resides
behind some kind of proxy, that could not be reachable. You can fix this by
adding one of these system parameters:
\begin{itemize}
\item {} 
\sphinxcode{\sphinxupquote{report.url}}, pointing to an URL reachable from your server
(probably \sphinxcode{\sphinxupquote{http://localhost:8069}} or something similar). It will be
used for this particular purpose only.

\item {} 
\sphinxcode{\sphinxupquote{web.base.url.freeze}}, when set to \sphinxcode{\sphinxupquote{True}}, will stop the
automatic updates to \sphinxcode{\sphinxupquote{web.base.url}}.

\end{itemize}
\end{sphinxadmonition}

\begin{sphinxadmonition}{note}
Create a report for the Session model

For each session, it should display session’s name, its start and end,
and list the session’s attendees.
\sphinxstyleemphasis{openacademy/\_\_manifest\_\_.py}
\fvset{hllines={, 4,}}%
\begin{sphinxVerbatim}[commandchars=\\\{\}]
        \PYG{l+s+s1}{\PYGZsq{}}\PYG{l+s+s1}{templates.xml}\PYG{l+s+s1}{\PYGZsq{}}\PYG{p}{,}
        \PYG{l+s+s1}{\PYGZsq{}}\PYG{l+s+s1}{views/openacademy.xml}\PYG{l+s+s1}{\PYGZsq{}}\PYG{p}{,}
        \PYG{l+s+s1}{\PYGZsq{}}\PYG{l+s+s1}{views/partner.xml}\PYG{l+s+s1}{\PYGZsq{}}\PYG{p}{,}
        \PYG{l+s+s1}{\PYGZsq{}}\PYG{l+s+s1}{reports.xml}\PYG{l+s+s1}{\PYGZsq{}}\PYG{p}{,}
    \PYG{p}{]}\PYG{p}{,}
    \PYG{c+c1}{\PYGZsh{} only loaded in demonstration mode}
    \PYG{l+s+s1}{\PYGZsq{}}\PYG{l+s+s1}{demo}\PYG{l+s+s1}{\PYGZsq{}}\PYG{p}{:} \PYG{p}{[}
\end{sphinxVerbatim}
\sphinxstyleemphasis{openacademy/reports.xml}
\fvset{hllines={, 1, 2, 3, 4, 5, 6, 7, 8, 9, 10, 11, 12, 13, 14, 15, 16, 17, 18, 19, 20, 21, 22, 23, 24, 25, 26, 27, 28, 29, 30,}}%
\begin{sphinxVerbatim}[commandchars=\\\{\}]
\PYG{n+nt}{\PYGZlt{}odoo}\PYG{n+nt}{\PYGZgt{}}

    \PYG{n+nt}{\PYGZlt{}report}
        \PYG{n+na}{id=}\PYG{l+s}{\PYGZdq{}report\PYGZus{}session\PYGZdq{}}
        \PYG{n+na}{model=}\PYG{l+s}{\PYGZdq{}openacademy.session\PYGZdq{}}
        \PYG{n+na}{string=}\PYG{l+s}{\PYGZdq{}Session Report\PYGZdq{}}
        \PYG{n+na}{name=}\PYG{l+s}{\PYGZdq{}openacademy.report\PYGZus{}session\PYGZus{}view\PYGZdq{}}
        \PYG{n+na}{file=}\PYG{l+s}{\PYGZdq{}openacademy.report\PYGZus{}session\PYGZdq{}}
        \PYG{n+na}{report\PYGZus{}type=}\PYG{l+s}{\PYGZdq{}qweb\PYGZhy{}pdf\PYGZdq{}} \PYG{n+nt}{/\PYGZgt{}}

    \PYG{n+nt}{\PYGZlt{}template} \PYG{n+na}{id=}\PYG{l+s}{\PYGZdq{}report\PYGZus{}session\PYGZus{}view\PYGZdq{}}\PYG{n+nt}{\PYGZgt{}}
        \PYG{n+nt}{\PYGZlt{}t} \PYG{n+na}{t\PYGZhy{}call=}\PYG{l+s}{\PYGZdq{}web.html\PYGZus{}container\PYGZdq{}}\PYG{n+nt}{\PYGZgt{}}
            \PYG{n+nt}{\PYGZlt{}t} \PYG{n+na}{t\PYGZhy{}foreach=}\PYG{l+s}{\PYGZdq{}docs\PYGZdq{}} \PYG{n+na}{t\PYGZhy{}as=}\PYG{l+s}{\PYGZdq{}doc\PYGZdq{}}\PYG{n+nt}{\PYGZgt{}}
                \PYG{n+nt}{\PYGZlt{}t} \PYG{n+na}{t\PYGZhy{}call=}\PYG{l+s}{\PYGZdq{}web.external\PYGZus{}layout\PYGZdq{}}\PYG{n+nt}{\PYGZgt{}}
                    \PYG{n+nt}{\PYGZlt{}div} \PYG{n+na}{class=}\PYG{l+s}{\PYGZdq{}page\PYGZdq{}}\PYG{n+nt}{\PYGZgt{}}
                        \PYG{n+nt}{\PYGZlt{}h2} \PYG{n+na}{t\PYGZhy{}field=}\PYG{l+s}{\PYGZdq{}doc.name\PYGZdq{}}\PYG{n+nt}{/\PYGZgt{}}
                        \PYG{n+nt}{\PYGZlt{}p}\PYG{n+nt}{\PYGZgt{}}From \PYG{n+nt}{\PYGZlt{}span} \PYG{n+na}{t\PYGZhy{}field=}\PYG{l+s}{\PYGZdq{}doc.start\PYGZus{}date\PYGZdq{}}\PYG{n+nt}{/\PYGZgt{}} to \PYG{n+nt}{\PYGZlt{}span} \PYG{n+na}{t\PYGZhy{}field=}\PYG{l+s}{\PYGZdq{}doc.end\PYGZus{}date\PYGZdq{}}\PYG{n+nt}{/\PYGZgt{}}\PYG{n+nt}{\PYGZlt{}/p\PYGZgt{}}
                        \PYG{n+nt}{\PYGZlt{}h3}\PYG{n+nt}{\PYGZgt{}}Attendees:\PYG{n+nt}{\PYGZlt{}/h3\PYGZgt{}}
                        \PYG{n+nt}{\PYGZlt{}ul}\PYG{n+nt}{\PYGZgt{}}
                            \PYG{n+nt}{\PYGZlt{}t} \PYG{n+na}{t\PYGZhy{}foreach=}\PYG{l+s}{\PYGZdq{}doc.attendee\PYGZus{}ids\PYGZdq{}} \PYG{n+na}{t\PYGZhy{}as=}\PYG{l+s}{\PYGZdq{}attendee\PYGZdq{}}\PYG{n+nt}{\PYGZgt{}}
                                \PYG{n+nt}{\PYGZlt{}li}\PYG{n+nt}{\PYGZgt{}}\PYG{n+nt}{\PYGZlt{}span} \PYG{n+na}{t\PYGZhy{}field=}\PYG{l+s}{\PYGZdq{}attendee.name\PYGZdq{}}\PYG{n+nt}{/\PYGZgt{}}\PYG{n+nt}{\PYGZlt{}/li\PYGZgt{}}
                            \PYG{n+nt}{\PYGZlt{}/t\PYGZgt{}}
                        \PYG{n+nt}{\PYGZlt{}/ul\PYGZgt{}}
                    \PYG{n+nt}{\PYGZlt{}/div\PYGZgt{}}
                \PYG{n+nt}{\PYGZlt{}/t\PYGZgt{}}
            \PYG{n+nt}{\PYGZlt{}/t\PYGZgt{}}
        \PYG{n+nt}{\PYGZlt{}/t\PYGZgt{}}
    \PYG{n+nt}{\PYGZlt{}/template\PYGZgt{}}

\PYG{n+nt}{\PYGZlt{}/odoo\PYGZgt{}}
\end{sphinxVerbatim}
\end{sphinxadmonition}


\subsubsection{Dashboards}
\label{\detokenize{howtos/backend:dashboards}}
\begin{sphinxadmonition}{note}
Define a Dashboard

Define a dashboard containing the graph view you created, the sessions
calendar view and a list view of the courses (switchable to a form
view). This dashboard should be available through a menuitem in the menu,
and automatically displayed in the web client when the OpenAcademy main
menu is selected.
\begin{enumerate}
\item {} 
Create a file \sphinxcode{\sphinxupquote{openacademy/views/session\_board.xml}}. It should contain
the board view, the actions referenced in that view, an action to
open the dashboard and a re-definition of the main menu item to add
the dashboard action

\begin{sphinxadmonition}{note}{Note:}
Available dashboard styles are \sphinxcode{\sphinxupquote{1}}, \sphinxcode{\sphinxupquote{1-1}}, \sphinxcode{\sphinxupquote{1-2}},
\sphinxcode{\sphinxupquote{2-1}} and \sphinxcode{\sphinxupquote{1-1-1}}
\end{sphinxadmonition}

\item {} 
Update \sphinxcode{\sphinxupquote{openacademy/\_\_manifest\_\_.py}} to reference the new data
file

\end{enumerate}
\sphinxstyleemphasis{openacademy/\_\_manifest\_\_.py}
\fvset{hllines={, 4,}}%
\begin{sphinxVerbatim}[commandchars=\\\{\}]
    \PYG{l+s+s1}{\PYGZsq{}}\PYG{l+s+s1}{version}\PYG{l+s+s1}{\PYGZsq{}}\PYG{p}{:} \PYG{l+s+s1}{\PYGZsq{}}\PYG{l+s+s1}{0.1}\PYG{l+s+s1}{\PYGZsq{}}\PYG{p}{,}

    \PYG{c+c1}{\PYGZsh{} any module necessary for this one to work correctly}
    \PYG{l+s+s1}{\PYGZsq{}}\PYG{l+s+s1}{depends}\PYG{l+s+s1}{\PYGZsq{}}\PYG{p}{:} \PYG{p}{[}\PYG{l+s+s1}{\PYGZsq{}}\PYG{l+s+s1}{base}\PYG{l+s+s1}{\PYGZsq{}}\PYG{p}{,} \PYG{l+s+s1}{\PYGZsq{}}\PYG{l+s+s1}{board}\PYG{l+s+s1}{\PYGZsq{}}\PYG{p}{]}\PYG{p}{,}

    \PYG{c+c1}{\PYGZsh{} always loaded}
    \PYG{l+s+s1}{\PYGZsq{}}\PYG{l+s+s1}{data}\PYG{l+s+s1}{\PYGZsq{}}\PYG{p}{:} \PYG{p}{[}
\end{sphinxVerbatim}

\fvset{hllines={, 4,}}%
\begin{sphinxVerbatim}[commandchars=\\\{\}]
        \PYG{l+s+s1}{\PYGZsq{}}\PYG{l+s+s1}{templates.xml}\PYG{l+s+s1}{\PYGZsq{}}\PYG{p}{,}
        \PYG{l+s+s1}{\PYGZsq{}}\PYG{l+s+s1}{views/openacademy.xml}\PYG{l+s+s1}{\PYGZsq{}}\PYG{p}{,}
        \PYG{l+s+s1}{\PYGZsq{}}\PYG{l+s+s1}{views/partner.xml}\PYG{l+s+s1}{\PYGZsq{}}\PYG{p}{,}
        \PYG{l+s+s1}{\PYGZsq{}}\PYG{l+s+s1}{views/session\PYGZus{}board.xml}\PYG{l+s+s1}{\PYGZsq{}}\PYG{p}{,}
        \PYG{l+s+s1}{\PYGZsq{}}\PYG{l+s+s1}{reports.xml}\PYG{l+s+s1}{\PYGZsq{}}\PYG{p}{,}
    \PYG{p}{]}\PYG{p}{,}
    \PYG{c+c1}{\PYGZsh{} only loaded in demonstration mode}
\end{sphinxVerbatim}
\sphinxstyleemphasis{openacademy/views/session\_board.xml}
\fvset{hllines={, 1, 2, 3, 4, 5, 6, 7, 8, 9, 10, 11, 12, 13, 14, 15, 16, 17, 18, 19, 20, 21, 22, 23, 24, 25, 26, 27, 28, 29, 30, 31, 32, 33, 34, 35, 36, 37, 38, 39, 40, 41, 42, 43, 44, 45, 46, 47, 48, 49, 50, 51, 52, 53, 54, 55, 56, 57, 58, 59, 60, 61, 62, 63, 64, 65, 66,}}%
\begin{sphinxVerbatim}[commandchars=\\\{\}]
\PYG{c+cp}{\PYGZlt{}?xml version=\PYGZdq{}1.0\PYGZdq{}?\PYGZgt{}}
\PYG{n+nt}{\PYGZlt{}odoo}\PYG{n+nt}{\PYGZgt{}}

        \PYG{n+nt}{\PYGZlt{}record} \PYG{n+na}{model=}\PYG{l+s}{\PYGZdq{}ir.actions.act\PYGZus{}window\PYGZdq{}} \PYG{n+na}{id=}\PYG{l+s}{\PYGZdq{}act\PYGZus{}session\PYGZus{}graph\PYGZdq{}}\PYG{n+nt}{\PYGZgt{}}
            \PYG{n+nt}{\PYGZlt{}field} \PYG{n+na}{name=}\PYG{l+s}{\PYGZdq{}name\PYGZdq{}}\PYG{n+nt}{\PYGZgt{}}Attendees by course\PYG{n+nt}{\PYGZlt{}/field\PYGZgt{}}
            \PYG{n+nt}{\PYGZlt{}field} \PYG{n+na}{name=}\PYG{l+s}{\PYGZdq{}res\PYGZus{}model\PYGZdq{}}\PYG{n+nt}{\PYGZgt{}}openacademy.session\PYG{n+nt}{\PYGZlt{}/field\PYGZgt{}}
            \PYG{n+nt}{\PYGZlt{}field} \PYG{n+na}{name=}\PYG{l+s}{\PYGZdq{}view\PYGZus{}type\PYGZdq{}}\PYG{n+nt}{\PYGZgt{}}form\PYG{n+nt}{\PYGZlt{}/field\PYGZgt{}}
            \PYG{n+nt}{\PYGZlt{}field} \PYG{n+na}{name=}\PYG{l+s}{\PYGZdq{}view\PYGZus{}mode\PYGZdq{}}\PYG{n+nt}{\PYGZgt{}}graph\PYG{n+nt}{\PYGZlt{}/field\PYGZgt{}}
            \PYG{n+nt}{\PYGZlt{}field} \PYG{n+na}{name=}\PYG{l+s}{\PYGZdq{}view\PYGZus{}id\PYGZdq{}}
                   \PYG{n+na}{ref=}\PYG{l+s}{\PYGZdq{}openacademy.openacademy\PYGZus{}session\PYGZus{}graph\PYGZus{}view\PYGZdq{}}\PYG{n+nt}{/\PYGZgt{}}
        \PYG{n+nt}{\PYGZlt{}/record\PYGZgt{}}
        \PYG{n+nt}{\PYGZlt{}record} \PYG{n+na}{model=}\PYG{l+s}{\PYGZdq{}ir.actions.act\PYGZus{}window\PYGZdq{}} \PYG{n+na}{id=}\PYG{l+s}{\PYGZdq{}act\PYGZus{}session\PYGZus{}calendar\PYGZdq{}}\PYG{n+nt}{\PYGZgt{}}
            \PYG{n+nt}{\PYGZlt{}field} \PYG{n+na}{name=}\PYG{l+s}{\PYGZdq{}name\PYGZdq{}}\PYG{n+nt}{\PYGZgt{}}Sessions\PYG{n+nt}{\PYGZlt{}/field\PYGZgt{}}
            \PYG{n+nt}{\PYGZlt{}field} \PYG{n+na}{name=}\PYG{l+s}{\PYGZdq{}res\PYGZus{}model\PYGZdq{}}\PYG{n+nt}{\PYGZgt{}}openacademy.session\PYG{n+nt}{\PYGZlt{}/field\PYGZgt{}}
            \PYG{n+nt}{\PYGZlt{}field} \PYG{n+na}{name=}\PYG{l+s}{\PYGZdq{}view\PYGZus{}type\PYGZdq{}}\PYG{n+nt}{\PYGZgt{}}form\PYG{n+nt}{\PYGZlt{}/field\PYGZgt{}}
            \PYG{n+nt}{\PYGZlt{}field} \PYG{n+na}{name=}\PYG{l+s}{\PYGZdq{}view\PYGZus{}mode\PYGZdq{}}\PYG{n+nt}{\PYGZgt{}}calendar\PYG{n+nt}{\PYGZlt{}/field\PYGZgt{}}
            \PYG{n+nt}{\PYGZlt{}field} \PYG{n+na}{name=}\PYG{l+s}{\PYGZdq{}view\PYGZus{}id\PYGZdq{}} \PYG{n+na}{ref=}\PYG{l+s}{\PYGZdq{}openacademy.session\PYGZus{}calendar\PYGZus{}view\PYGZdq{}}\PYG{n+nt}{/\PYGZgt{}}
        \PYG{n+nt}{\PYGZlt{}/record\PYGZgt{}}
        \PYG{n+nt}{\PYGZlt{}record} \PYG{n+na}{model=}\PYG{l+s}{\PYGZdq{}ir.actions.act\PYGZus{}window\PYGZdq{}} \PYG{n+na}{id=}\PYG{l+s}{\PYGZdq{}act\PYGZus{}course\PYGZus{}list\PYGZdq{}}\PYG{n+nt}{\PYGZgt{}}
            \PYG{n+nt}{\PYGZlt{}field} \PYG{n+na}{name=}\PYG{l+s}{\PYGZdq{}name\PYGZdq{}}\PYG{n+nt}{\PYGZgt{}}Courses\PYG{n+nt}{\PYGZlt{}/field\PYGZgt{}}
            \PYG{n+nt}{\PYGZlt{}field} \PYG{n+na}{name=}\PYG{l+s}{\PYGZdq{}res\PYGZus{}model\PYGZdq{}}\PYG{n+nt}{\PYGZgt{}}openacademy.course\PYG{n+nt}{\PYGZlt{}/field\PYGZgt{}}
            \PYG{n+nt}{\PYGZlt{}field} \PYG{n+na}{name=}\PYG{l+s}{\PYGZdq{}view\PYGZus{}type\PYGZdq{}}\PYG{n+nt}{\PYGZgt{}}form\PYG{n+nt}{\PYGZlt{}/field\PYGZgt{}}
            \PYG{n+nt}{\PYGZlt{}field} \PYG{n+na}{name=}\PYG{l+s}{\PYGZdq{}view\PYGZus{}mode\PYGZdq{}}\PYG{n+nt}{\PYGZgt{}}tree,form\PYG{n+nt}{\PYGZlt{}/field\PYGZgt{}}
        \PYG{n+nt}{\PYGZlt{}/record\PYGZgt{}}
        \PYG{n+nt}{\PYGZlt{}record} \PYG{n+na}{model=}\PYG{l+s}{\PYGZdq{}ir.ui.view\PYGZdq{}} \PYG{n+na}{id=}\PYG{l+s}{\PYGZdq{}board\PYGZus{}session\PYGZus{}form\PYGZdq{}}\PYG{n+nt}{\PYGZgt{}}
            \PYG{n+nt}{\PYGZlt{}field} \PYG{n+na}{name=}\PYG{l+s}{\PYGZdq{}name\PYGZdq{}}\PYG{n+nt}{\PYGZgt{}}Session Dashboard Form\PYG{n+nt}{\PYGZlt{}/field\PYGZgt{}}
            \PYG{n+nt}{\PYGZlt{}field} \PYG{n+na}{name=}\PYG{l+s}{\PYGZdq{}model\PYGZdq{}}\PYG{n+nt}{\PYGZgt{}}board.board\PYG{n+nt}{\PYGZlt{}/field\PYGZgt{}}
            \PYG{n+nt}{\PYGZlt{}field} \PYG{n+na}{name=}\PYG{l+s}{\PYGZdq{}type\PYGZdq{}}\PYG{n+nt}{\PYGZgt{}}form\PYG{n+nt}{\PYGZlt{}/field\PYGZgt{}}
            \PYG{n+nt}{\PYGZlt{}field} \PYG{n+na}{name=}\PYG{l+s}{\PYGZdq{}arch\PYGZdq{}} \PYG{n+na}{type=}\PYG{l+s}{\PYGZdq{}xml\PYGZdq{}}\PYG{n+nt}{\PYGZgt{}}
                \PYG{n+nt}{\PYGZlt{}form} \PYG{n+na}{string=}\PYG{l+s}{\PYGZdq{}Session Dashboard\PYGZdq{}}\PYG{n+nt}{\PYGZgt{}}
                    \PYG{n+nt}{\PYGZlt{}board} \PYG{n+na}{style=}\PYG{l+s}{\PYGZdq{}2\PYGZhy{}1\PYGZdq{}}\PYG{n+nt}{\PYGZgt{}}
                        \PYG{n+nt}{\PYGZlt{}column}\PYG{n+nt}{\PYGZgt{}}
                            \PYG{n+nt}{\PYGZlt{}action}
                                \PYG{n+na}{string=}\PYG{l+s}{\PYGZdq{}Attendees by course\PYGZdq{}}
                                \PYG{n+na}{name=}\PYG{l+s}{\PYGZdq{}\PYGZpc{}(act\PYGZus{}session\PYGZus{}graph)d\PYGZdq{}}
                                \PYG{n+na}{height=}\PYG{l+s}{\PYGZdq{}150\PYGZdq{}}
                                \PYG{n+na}{width=}\PYG{l+s}{\PYGZdq{}510\PYGZdq{}}\PYG{n+nt}{/\PYGZgt{}}
                            \PYG{n+nt}{\PYGZlt{}action}
                                \PYG{n+na}{string=}\PYG{l+s}{\PYGZdq{}Sessions\PYGZdq{}}
                                \PYG{n+na}{name=}\PYG{l+s}{\PYGZdq{}\PYGZpc{}(act\PYGZus{}session\PYGZus{}calendar)d\PYGZdq{}}\PYG{n+nt}{/\PYGZgt{}}
                        \PYG{n+nt}{\PYGZlt{}/column\PYGZgt{}}
                        \PYG{n+nt}{\PYGZlt{}column}\PYG{n+nt}{\PYGZgt{}}
                            \PYG{n+nt}{\PYGZlt{}action}
                                \PYG{n+na}{string=}\PYG{l+s}{\PYGZdq{}Courses\PYGZdq{}}
                                \PYG{n+na}{name=}\PYG{l+s}{\PYGZdq{}\PYGZpc{}(act\PYGZus{}course\PYGZus{}list)d\PYGZdq{}}\PYG{n+nt}{/\PYGZgt{}}
                        \PYG{n+nt}{\PYGZlt{}/column\PYGZgt{}}
                    \PYG{n+nt}{\PYGZlt{}/board\PYGZgt{}}
                \PYG{n+nt}{\PYGZlt{}/form\PYGZgt{}}
            \PYG{n+nt}{\PYGZlt{}/field\PYGZgt{}}
        \PYG{n+nt}{\PYGZlt{}/record\PYGZgt{}}
        \PYG{n+nt}{\PYGZlt{}record} \PYG{n+na}{model=}\PYG{l+s}{\PYGZdq{}ir.actions.act\PYGZus{}window\PYGZdq{}} \PYG{n+na}{id=}\PYG{l+s}{\PYGZdq{}open\PYGZus{}board\PYGZus{}session\PYGZdq{}}\PYG{n+nt}{\PYGZgt{}}
          \PYG{n+nt}{\PYGZlt{}field} \PYG{n+na}{name=}\PYG{l+s}{\PYGZdq{}name\PYGZdq{}}\PYG{n+nt}{\PYGZgt{}}Session Dashboard\PYG{n+nt}{\PYGZlt{}/field\PYGZgt{}}
          \PYG{n+nt}{\PYGZlt{}field} \PYG{n+na}{name=}\PYG{l+s}{\PYGZdq{}res\PYGZus{}model\PYGZdq{}}\PYG{n+nt}{\PYGZgt{}}board.board\PYG{n+nt}{\PYGZlt{}/field\PYGZgt{}}
          \PYG{n+nt}{\PYGZlt{}field} \PYG{n+na}{name=}\PYG{l+s}{\PYGZdq{}view\PYGZus{}type\PYGZdq{}}\PYG{n+nt}{\PYGZgt{}}form\PYG{n+nt}{\PYGZlt{}/field\PYGZgt{}}
          \PYG{n+nt}{\PYGZlt{}field} \PYG{n+na}{name=}\PYG{l+s}{\PYGZdq{}view\PYGZus{}mode\PYGZdq{}}\PYG{n+nt}{\PYGZgt{}}form\PYG{n+nt}{\PYGZlt{}/field\PYGZgt{}}
          \PYG{n+nt}{\PYGZlt{}field} \PYG{n+na}{name=}\PYG{l+s}{\PYGZdq{}usage\PYGZdq{}}\PYG{n+nt}{\PYGZgt{}}menu\PYG{n+nt}{\PYGZlt{}/field\PYGZgt{}}
          \PYG{n+nt}{\PYGZlt{}field} \PYG{n+na}{name=}\PYG{l+s}{\PYGZdq{}view\PYGZus{}id\PYGZdq{}} \PYG{n+na}{ref=}\PYG{l+s}{\PYGZdq{}board\PYGZus{}session\PYGZus{}form\PYGZdq{}}\PYG{n+nt}{/\PYGZgt{}}
        \PYG{n+nt}{\PYGZlt{}/record\PYGZgt{}}

        \PYG{n+nt}{\PYGZlt{}menuitem}
            \PYG{n+na}{name=}\PYG{l+s}{\PYGZdq{}Session Dashboard\PYGZdq{}} \PYG{n+na}{parent=}\PYG{l+s}{\PYGZdq{}base.menu\PYGZus{}reporting\PYGZus{}dashboard\PYGZdq{}}
            \PYG{n+na}{action=}\PYG{l+s}{\PYGZdq{}open\PYGZus{}board\PYGZus{}session\PYGZdq{}}
            \PYG{n+na}{sequence=}\PYG{l+s}{\PYGZdq{}1\PYGZdq{}}
            \PYG{n+na}{id=}\PYG{l+s}{\PYGZdq{}menu\PYGZus{}board\PYGZus{}session\PYGZdq{}}\PYG{n+nt}{/\PYGZgt{}}

\PYG{n+nt}{\PYGZlt{}/odoo\PYGZgt{}}
\end{sphinxVerbatim}
\end{sphinxadmonition}


\subsection{WebServices}
\label{\detokenize{howtos/backend:webservices}}
The web-service module offer a common interface for all web-services :
\begin{itemize}
\item {} 
XML-RPC

\item {} 
JSON-RPC

\end{itemize}

Business objects can also be accessed via the distributed object
mechanism. They can all be modified via the client interface with contextual
views.

Odoo is accessible through XML-RPC/JSON-RPC interfaces, for which libraries
exist in many languages.


\subsubsection{XML-RPC Library}
\label{\detokenize{howtos/backend:xml-rpc-library}}
The following example is a Python 3 program that interacts with an Odoo
server with the library \sphinxcode{\sphinxupquote{xmlrpc.client}}:

\fvset{hllines={, ,}}%
\begin{sphinxVerbatim}[commandchars=\\\{\}]
\PYG{k+kn}{import} \PYG{n+nn}{xmlrpc}\PYG{n+nn}{.}\PYG{n+nn}{client}

\PYG{n}{root} \PYG{o}{=} \PYG{l+s+s1}{\PYGZsq{}}\PYG{l+s+s1}{http://}\PYG{l+s+si}{\PYGZpc{}s}\PYG{l+s+s1}{:}\PYG{l+s+si}{\PYGZpc{}d}\PYG{l+s+s1}{/xmlrpc/}\PYG{l+s+s1}{\PYGZsq{}} \PYG{o}{\PYGZpc{}} \PYG{p}{(}\PYG{n}{HOST}\PYG{p}{,} \PYG{n}{PORT}\PYG{p}{)}

\PYG{n}{uid} \PYG{o}{=} \PYG{n}{xmlrpc}\PYG{o}{.}\PYG{n}{client}\PYG{o}{.}\PYG{n}{ServerProxy}\PYG{p}{(}\PYG{n}{root} \PYG{o}{+} \PYG{l+s+s1}{\PYGZsq{}}\PYG{l+s+s1}{common}\PYG{l+s+s1}{\PYGZsq{}}\PYG{p}{)}\PYG{o}{.}\PYG{n}{login}\PYG{p}{(}\PYG{n}{DB}\PYG{p}{,} \PYG{n}{USER}\PYG{p}{,} \PYG{n}{PASS}\PYG{p}{)}
\PYG{n+nb}{print}\PYG{p}{(}\PYG{l+s+s2}{\PYGZdq{}}\PYG{l+s+s2}{Logged in as }\PYG{l+s+si}{\PYGZpc{}s}\PYG{l+s+s2}{ (uid: }\PYG{l+s+si}{\PYGZpc{}d}\PYG{l+s+s2}{)}\PYG{l+s+s2}{\PYGZdq{}} \PYG{o}{\PYGZpc{}} \PYG{p}{(}\PYG{n}{USER}\PYG{p}{,} \PYG{n}{uid}\PYG{p}{)}\PYG{p}{)}

\PYG{c+c1}{\PYGZsh{} Create a new note}
\PYG{n}{sock} \PYG{o}{=} \PYG{n}{xmlrpc}\PYG{o}{.}\PYG{n}{client}\PYG{o}{.}\PYG{n}{ServerProxy}\PYG{p}{(}\PYG{n}{root} \PYG{o}{+} \PYG{l+s+s1}{\PYGZsq{}}\PYG{l+s+s1}{object}\PYG{l+s+s1}{\PYGZsq{}}\PYG{p}{)}
\PYG{n}{args} \PYG{o}{=} \PYG{p}{\PYGZob{}}
    \PYG{l+s+s1}{\PYGZsq{}}\PYG{l+s+s1}{color}\PYG{l+s+s1}{\PYGZsq{}} \PYG{p}{:} \PYG{l+m+mi}{8}\PYG{p}{,}
    \PYG{l+s+s1}{\PYGZsq{}}\PYG{l+s+s1}{memo}\PYG{l+s+s1}{\PYGZsq{}} \PYG{p}{:} \PYG{l+s+s1}{\PYGZsq{}}\PYG{l+s+s1}{This is a note}\PYG{l+s+s1}{\PYGZsq{}}\PYG{p}{,}
    \PYG{l+s+s1}{\PYGZsq{}}\PYG{l+s+s1}{create\PYGZus{}uid}\PYG{l+s+s1}{\PYGZsq{}}\PYG{p}{:} \PYG{n}{uid}\PYG{p}{,}
\PYG{p}{\PYGZcb{}}
\PYG{n}{note\PYGZus{}id} \PYG{o}{=} \PYG{n}{sock}\PYG{o}{.}\PYG{n}{execute}\PYG{p}{(}\PYG{n}{DB}\PYG{p}{,} \PYG{n}{uid}\PYG{p}{,} \PYG{n}{PASS}\PYG{p}{,} \PYG{l+s+s1}{\PYGZsq{}}\PYG{l+s+s1}{note.note}\PYG{l+s+s1}{\PYGZsq{}}\PYG{p}{,} \PYG{l+s+s1}{\PYGZsq{}}\PYG{l+s+s1}{create}\PYG{l+s+s1}{\PYGZsq{}}\PYG{p}{,} \PYG{n}{args}\PYG{p}{)}
\end{sphinxVerbatim}

\begin{sphinxadmonition}{note}
Add a new service to the client

Write a Python program able to send XML-RPC requests to a PC running
Odoo (yours, or your instructor’s). This program should display all
the sessions, and their corresponding number of seats. It should also
create a new session for one of the courses.

\fvset{hllines={, ,}}%
\begin{sphinxVerbatim}[commandchars=\\\{\}]
\PYG{k+kn}{import} \PYG{n+nn}{functools}
\PYG{k+kn}{import} \PYG{n+nn}{xmlrpc.client}
\PYG{n}{HOST} \PYG{o}{=} \PYG{l+s+s1}{\PYGZsq{}}\PYG{l+s+s1}{localhost}\PYG{l+s+s1}{\PYGZsq{}}
\PYG{n}{PORT} \PYG{o}{=} \PYG{l+m+mi}{8069}
\PYG{n}{DB} \PYG{o}{=} \PYG{l+s+s1}{\PYGZsq{}}\PYG{l+s+s1}{openacademy}\PYG{l+s+s1}{\PYGZsq{}}
\PYG{n}{USER} \PYG{o}{=} \PYG{l+s+s1}{\PYGZsq{}}\PYG{l+s+s1}{admin}\PYG{l+s+s1}{\PYGZsq{}}
\PYG{n}{PASS} \PYG{o}{=} \PYG{l+s+s1}{\PYGZsq{}}\PYG{l+s+s1}{admin}\PYG{l+s+s1}{\PYGZsq{}}
\PYG{n}{ROOT} \PYG{o}{=} \PYG{l+s+s1}{\PYGZsq{}}\PYG{l+s+s1}{http://}\PYG{l+s+si}{\PYGZpc{}s}\PYG{l+s+s1}{:}\PYG{l+s+si}{\PYGZpc{}d}\PYG{l+s+s1}{/xmlrpc/}\PYG{l+s+s1}{\PYGZsq{}} \PYG{o}{\PYGZpc{}} \PYG{p}{(}\PYG{n}{HOST}\PYG{p}{,}\PYG{n}{PORT}\PYG{p}{)}

\PYG{c+c1}{\PYGZsh{} 1. Login}
\PYG{n}{uid} \PYG{o}{=} \PYG{n}{xmlrpc}\PYG{o}{.}\PYG{n}{client}\PYG{o}{.}\PYG{n}{ServerProxy}\PYG{p}{(}\PYG{n}{ROOT} \PYG{o}{+} \PYG{l+s+s1}{\PYGZsq{}}\PYG{l+s+s1}{common}\PYG{l+s+s1}{\PYGZsq{}}\PYG{p}{)}\PYG{o}{.}\PYG{n}{login}\PYG{p}{(}\PYG{n}{DB}\PYG{p}{,}\PYG{n}{USER}\PYG{p}{,}\PYG{n}{PASS}\PYG{p}{)}
\PYG{k}{print}\PYG{p}{(}\PYG{l+s+s2}{\PYGZdq{}}\PYG{l+s+s2}{Logged in as }\PYG{l+s+si}{\PYGZpc{}s}\PYG{l+s+s2}{ (uid:}\PYG{l+s+si}{\PYGZpc{}d}\PYG{l+s+s2}{)}\PYG{l+s+s2}{\PYGZdq{}} \PYG{o}{\PYGZpc{}} \PYG{p}{(}\PYG{n}{USER}\PYG{p}{,}\PYG{n}{uid}\PYG{p}{)}\PYG{p}{)}

\PYG{n}{call} \PYG{o}{=} \PYG{n}{functools}\PYG{o}{.}\PYG{n}{partial}\PYG{p}{(}
    \PYG{n}{xmlrpc}\PYG{o}{.}\PYG{n}{client}\PYG{o}{.}\PYG{n}{ServerProxy}\PYG{p}{(}\PYG{n}{ROOT} \PYG{o}{+} \PYG{l+s+s1}{\PYGZsq{}}\PYG{l+s+s1}{object}\PYG{l+s+s1}{\PYGZsq{}}\PYG{p}{)}\PYG{o}{.}\PYG{n}{execute}\PYG{p}{,}
    \PYG{n}{DB}\PYG{p}{,} \PYG{n}{uid}\PYG{p}{,} \PYG{n}{PASS}\PYG{p}{)}

\PYG{c+c1}{\PYGZsh{} 2. Read the sessions}
\PYG{n}{sessions} \PYG{o}{=} \PYG{n}{call}\PYG{p}{(}\PYG{l+s+s1}{\PYGZsq{}}\PYG{l+s+s1}{openacademy.session}\PYG{l+s+s1}{\PYGZsq{}}\PYG{p}{,}\PYG{l+s+s1}{\PYGZsq{}}\PYG{l+s+s1}{search\PYGZus{}read}\PYG{l+s+s1}{\PYGZsq{}}\PYG{p}{,} \PYG{p}{[}\PYG{p}{]}\PYG{p}{,} \PYG{p}{[}\PYG{l+s+s1}{\PYGZsq{}}\PYG{l+s+s1}{name}\PYG{l+s+s1}{\PYGZsq{}}\PYG{p}{,}\PYG{l+s+s1}{\PYGZsq{}}\PYG{l+s+s1}{seats}\PYG{l+s+s1}{\PYGZsq{}}\PYG{p}{]}\PYG{p}{)}
\PYG{k}{for} \PYG{n}{session} \PYG{o+ow}{in} \PYG{n}{sessions}\PYG{p}{:}
    \PYG{k}{print}\PYG{p}{(}\PYG{l+s+s2}{\PYGZdq{}}\PYG{l+s+s2}{Session }\PYG{l+s+si}{\PYGZpc{}s}\PYG{l+s+s2}{ (}\PYG{l+s+si}{\PYGZpc{}s}\PYG{l+s+s2}{ seats)}\PYG{l+s+s2}{\PYGZdq{}} \PYG{o}{\PYGZpc{}} \PYG{p}{(}\PYG{n}{session}\PYG{p}{[}\PYG{l+s+s1}{\PYGZsq{}}\PYG{l+s+s1}{name}\PYG{l+s+s1}{\PYGZsq{}}\PYG{p}{]}\PYG{p}{,} \PYG{n}{session}\PYG{p}{[}\PYG{l+s+s1}{\PYGZsq{}}\PYG{l+s+s1}{seats}\PYG{l+s+s1}{\PYGZsq{}}\PYG{p}{]}\PYG{p}{)}\PYG{p}{)}
\PYG{c+c1}{\PYGZsh{} 3.create a new session}
\PYG{n}{session\PYGZus{}id} \PYG{o}{=} \PYG{n}{call}\PYG{p}{(}\PYG{l+s+s1}{\PYGZsq{}}\PYG{l+s+s1}{openacademy.session}\PYG{l+s+s1}{\PYGZsq{}}\PYG{p}{,} \PYG{l+s+s1}{\PYGZsq{}}\PYG{l+s+s1}{create}\PYG{l+s+s1}{\PYGZsq{}}\PYG{p}{,} \PYG{p}{\PYGZob{}}
    \PYG{l+s+s1}{\PYGZsq{}}\PYG{l+s+s1}{name}\PYG{l+s+s1}{\PYGZsq{}} \PYG{p}{:} \PYG{l+s+s1}{\PYGZsq{}}\PYG{l+s+s1}{My session}\PYG{l+s+s1}{\PYGZsq{}}\PYG{p}{,}
    \PYG{l+s+s1}{\PYGZsq{}}\PYG{l+s+s1}{course\PYGZus{}id}\PYG{l+s+s1}{\PYGZsq{}} \PYG{p}{:} \PYG{l+m+mi}{2}\PYG{p}{,}
\PYG{p}{\PYGZcb{}}\PYG{p}{)}
\end{sphinxVerbatim}

Instead of using a hard-coded course id, the code can look up a course
by name:

\fvset{hllines={, ,}}%
\begin{sphinxVerbatim}[commandchars=\\\{\}]
\PYG{c+c1}{\PYGZsh{} 3.create a new session for the \PYGZdq{}Functional\PYGZdq{} course}
\PYG{n}{course\PYGZus{}id} \PYG{o}{=} \PYG{n}{call}\PYG{p}{(}\PYG{l+s+s1}{\PYGZsq{}}\PYG{l+s+s1}{openacademy.course}\PYG{l+s+s1}{\PYGZsq{}}\PYG{p}{,} \PYG{l+s+s1}{\PYGZsq{}}\PYG{l+s+s1}{search}\PYG{l+s+s1}{\PYGZsq{}}\PYG{p}{,} \PYG{p}{[}\PYG{p}{(}\PYG{l+s+s1}{\PYGZsq{}}\PYG{l+s+s1}{name}\PYG{l+s+s1}{\PYGZsq{}}\PYG{p}{,}\PYG{l+s+s1}{\PYGZsq{}}\PYG{l+s+s1}{ilike}\PYG{l+s+s1}{\PYGZsq{}}\PYG{p}{,}\PYG{l+s+s1}{\PYGZsq{}}\PYG{l+s+s1}{Functional}\PYG{l+s+s1}{\PYGZsq{}}\PYG{p}{)}\PYG{p}{]}\PYG{p}{)}\PYG{p}{[}\PYG{l+m+mi}{0}\PYG{p}{]}
\PYG{n}{session\PYGZus{}id} \PYG{o}{=} \PYG{n}{call}\PYG{p}{(}\PYG{l+s+s1}{\PYGZsq{}}\PYG{l+s+s1}{openacademy.session}\PYG{l+s+s1}{\PYGZsq{}}\PYG{p}{,} \PYG{l+s+s1}{\PYGZsq{}}\PYG{l+s+s1}{create}\PYG{l+s+s1}{\PYGZsq{}}\PYG{p}{,} \PYG{p}{\PYGZob{}}
    \PYG{l+s+s1}{\PYGZsq{}}\PYG{l+s+s1}{name}\PYG{l+s+s1}{\PYGZsq{}} \PYG{p}{:} \PYG{l+s+s1}{\PYGZsq{}}\PYG{l+s+s1}{My session}\PYG{l+s+s1}{\PYGZsq{}}\PYG{p}{,}
    \PYG{l+s+s1}{\PYGZsq{}}\PYG{l+s+s1}{course\PYGZus{}id}\PYG{l+s+s1}{\PYGZsq{}} \PYG{p}{:} \PYG{n}{course\PYGZus{}id}\PYG{p}{,}
\PYG{p}{\PYGZcb{}}\PYG{p}{)}
\end{sphinxVerbatim}
\end{sphinxadmonition}


\subsubsection{JSON-RPC Library}
\label{\detokenize{howtos/backend:json-rpc-library}}
The following example is a Python 3 program that interacts with an Odoo server
with the standard Python libraries \sphinxcode{\sphinxupquote{urllib.request}} and \sphinxcode{\sphinxupquote{json}}. This
example assumes the \sphinxstylestrong{Productivity} app (\sphinxcode{\sphinxupquote{note}}) is installed:

\fvset{hllines={, ,}}%
\begin{sphinxVerbatim}[commandchars=\\\{\}]
\PYG{k+kn}{import} \PYG{n+nn}{json}
\PYG{k+kn}{import} \PYG{n+nn}{random}
\PYG{k+kn}{import} \PYG{n+nn}{urllib}\PYG{n+nn}{.}\PYG{n+nn}{request}

\PYG{n}{HOST} \PYG{o}{=} \PYG{l+s+s1}{\PYGZsq{}}\PYG{l+s+s1}{localhost}\PYG{l+s+s1}{\PYGZsq{}}
\PYG{n}{PORT} \PYG{o}{=} \PYG{l+m+mi}{8069}
\PYG{n}{DB} \PYG{o}{=} \PYG{l+s+s1}{\PYGZsq{}}\PYG{l+s+s1}{openacademy}\PYG{l+s+s1}{\PYGZsq{}}
\PYG{n}{USER} \PYG{o}{=} \PYG{l+s+s1}{\PYGZsq{}}\PYG{l+s+s1}{admin}\PYG{l+s+s1}{\PYGZsq{}}
\PYG{n}{PASS} \PYG{o}{=} \PYG{l+s+s1}{\PYGZsq{}}\PYG{l+s+s1}{admin}\PYG{l+s+s1}{\PYGZsq{}}

\PYG{k}{def} \PYG{n+nf}{json\PYGZus{}rpc}\PYG{p}{(}\PYG{n}{url}\PYG{p}{,} \PYG{n}{method}\PYG{p}{,} \PYG{n}{params}\PYG{p}{)}\PYG{p}{:}
    \PYG{n}{data} \PYG{o}{=} \PYG{p}{\PYGZob{}}
        \PYG{l+s+s2}{\PYGZdq{}}\PYG{l+s+s2}{jsonrpc}\PYG{l+s+s2}{\PYGZdq{}}\PYG{p}{:} \PYG{l+s+s2}{\PYGZdq{}}\PYG{l+s+s2}{2.0}\PYG{l+s+s2}{\PYGZdq{}}\PYG{p}{,}
        \PYG{l+s+s2}{\PYGZdq{}}\PYG{l+s+s2}{method}\PYG{l+s+s2}{\PYGZdq{}}\PYG{p}{:} \PYG{n}{method}\PYG{p}{,}
        \PYG{l+s+s2}{\PYGZdq{}}\PYG{l+s+s2}{params}\PYG{l+s+s2}{\PYGZdq{}}\PYG{p}{:} \PYG{n}{params}\PYG{p}{,}
        \PYG{l+s+s2}{\PYGZdq{}}\PYG{l+s+s2}{id}\PYG{l+s+s2}{\PYGZdq{}}\PYG{p}{:} \PYG{n}{random}\PYG{o}{.}\PYG{n}{randint}\PYG{p}{(}\PYG{l+m+mi}{0}\PYG{p}{,} \PYG{l+m+mi}{1000000000}\PYG{p}{)}\PYG{p}{,}
    \PYG{p}{\PYGZcb{}}
    \PYG{n}{req} \PYG{o}{=} \PYG{n}{urllib}\PYG{o}{.}\PYG{n}{request}\PYG{o}{.}\PYG{n}{Request}\PYG{p}{(}\PYG{n}{url}\PYG{o}{=}\PYG{n}{url}\PYG{p}{,} \PYG{n}{data}\PYG{o}{=}\PYG{n}{json}\PYG{o}{.}\PYG{n}{dumps}\PYG{p}{(}\PYG{n}{data}\PYG{p}{)}\PYG{o}{.}\PYG{n}{encode}\PYG{p}{(}\PYG{p}{)}\PYG{p}{,} \PYG{n}{headers}\PYG{o}{=}\PYG{p}{\PYGZob{}}
        \PYG{l+s+s2}{\PYGZdq{}}\PYG{l+s+s2}{Content\PYGZhy{}Type}\PYG{l+s+s2}{\PYGZdq{}}\PYG{p}{:}\PYG{l+s+s2}{\PYGZdq{}}\PYG{l+s+s2}{application/json}\PYG{l+s+s2}{\PYGZdq{}}\PYG{p}{,}
    \PYG{p}{\PYGZcb{}}\PYG{p}{)}
    \PYG{n}{reply} \PYG{o}{=} \PYG{n}{json}\PYG{o}{.}\PYG{n}{loads}\PYG{p}{(}\PYG{n}{urllib}\PYG{o}{.}\PYG{n}{request}\PYG{o}{.}\PYG{n}{urlopen}\PYG{p}{(}\PYG{n}{req}\PYG{p}{)}\PYG{o}{.}\PYG{n}{read}\PYG{p}{(}\PYG{p}{)}\PYG{o}{.}\PYG{n}{decode}\PYG{p}{(}\PYG{l+s+s1}{\PYGZsq{}}\PYG{l+s+s1}{UTF\PYGZhy{}8}\PYG{l+s+s1}{\PYGZsq{}}\PYG{p}{)}\PYG{p}{)}
    \PYG{k}{if} \PYG{n}{reply}\PYG{o}{.}\PYG{n}{get}\PYG{p}{(}\PYG{l+s+s2}{\PYGZdq{}}\PYG{l+s+s2}{error}\PYG{l+s+s2}{\PYGZdq{}}\PYG{p}{)}\PYG{p}{:}
        \PYG{k}{raise} \PYG{n+ne}{Exception}\PYG{p}{(}\PYG{n}{reply}\PYG{p}{[}\PYG{l+s+s2}{\PYGZdq{}}\PYG{l+s+s2}{error}\PYG{l+s+s2}{\PYGZdq{}}\PYG{p}{]}\PYG{p}{)}
    \PYG{k}{return} \PYG{n}{reply}\PYG{p}{[}\PYG{l+s+s2}{\PYGZdq{}}\PYG{l+s+s2}{result}\PYG{l+s+s2}{\PYGZdq{}}\PYG{p}{]}

\PYG{k}{def} \PYG{n+nf}{call}\PYG{p}{(}\PYG{n}{url}\PYG{p}{,} \PYG{n}{service}\PYG{p}{,} \PYG{n}{method}\PYG{p}{,} \PYG{o}{*}\PYG{n}{args}\PYG{p}{)}\PYG{p}{:}
    \PYG{k}{return} \PYG{n}{json\PYGZus{}rpc}\PYG{p}{(}\PYG{n}{url}\PYG{p}{,} \PYG{l+s+s2}{\PYGZdq{}}\PYG{l+s+s2}{call}\PYG{l+s+s2}{\PYGZdq{}}\PYG{p}{,} \PYG{p}{\PYGZob{}}\PYG{l+s+s2}{\PYGZdq{}}\PYG{l+s+s2}{service}\PYG{l+s+s2}{\PYGZdq{}}\PYG{p}{:} \PYG{n}{service}\PYG{p}{,} \PYG{l+s+s2}{\PYGZdq{}}\PYG{l+s+s2}{method}\PYG{l+s+s2}{\PYGZdq{}}\PYG{p}{:} \PYG{n}{method}\PYG{p}{,} \PYG{l+s+s2}{\PYGZdq{}}\PYG{l+s+s2}{args}\PYG{l+s+s2}{\PYGZdq{}}\PYG{p}{:} \PYG{n}{args}\PYG{p}{\PYGZcb{}}\PYG{p}{)}

\PYG{c+c1}{\PYGZsh{} log in the given database}
\PYG{n}{url} \PYG{o}{=} \PYG{l+s+s2}{\PYGZdq{}}\PYG{l+s+s2}{http://}\PYG{l+s+si}{\PYGZpc{}s}\PYG{l+s+s2}{:}\PYG{l+s+si}{\PYGZpc{}s}\PYG{l+s+s2}{/jsonrpc}\PYG{l+s+s2}{\PYGZdq{}} \PYG{o}{\PYGZpc{}} \PYG{p}{(}\PYG{n}{HOST}\PYG{p}{,} \PYG{n}{PORT}\PYG{p}{)}
\PYG{n}{uid} \PYG{o}{=} \PYG{n}{call}\PYG{p}{(}\PYG{n}{url}\PYG{p}{,} \PYG{l+s+s2}{\PYGZdq{}}\PYG{l+s+s2}{common}\PYG{l+s+s2}{\PYGZdq{}}\PYG{p}{,} \PYG{l+s+s2}{\PYGZdq{}}\PYG{l+s+s2}{login}\PYG{l+s+s2}{\PYGZdq{}}\PYG{p}{,} \PYG{n}{DB}\PYG{p}{,} \PYG{n}{USER}\PYG{p}{,} \PYG{n}{PASS}\PYG{p}{)}

\PYG{c+c1}{\PYGZsh{} create a new note}
\PYG{n}{args} \PYG{o}{=} \PYG{p}{\PYGZob{}}
    \PYG{l+s+s1}{\PYGZsq{}}\PYG{l+s+s1}{color}\PYG{l+s+s1}{\PYGZsq{}}\PYG{p}{:} \PYG{l+m+mi}{8}\PYG{p}{,}
    \PYG{l+s+s1}{\PYGZsq{}}\PYG{l+s+s1}{memo}\PYG{l+s+s1}{\PYGZsq{}}\PYG{p}{:} \PYG{l+s+s1}{\PYGZsq{}}\PYG{l+s+s1}{This is another note}\PYG{l+s+s1}{\PYGZsq{}}\PYG{p}{,}
    \PYG{l+s+s1}{\PYGZsq{}}\PYG{l+s+s1}{create\PYGZus{}uid}\PYG{l+s+s1}{\PYGZsq{}}\PYG{p}{:} \PYG{n}{uid}\PYG{p}{,}
\PYG{p}{\PYGZcb{}}
\PYG{n}{note\PYGZus{}id} \PYG{o}{=} \PYG{n}{call}\PYG{p}{(}\PYG{n}{url}\PYG{p}{,} \PYG{l+s+s2}{\PYGZdq{}}\PYG{l+s+s2}{object}\PYG{l+s+s2}{\PYGZdq{}}\PYG{p}{,} \PYG{l+s+s2}{\PYGZdq{}}\PYG{l+s+s2}{execute}\PYG{l+s+s2}{\PYGZdq{}}\PYG{p}{,} \PYG{n}{DB}\PYG{p}{,} \PYG{n}{uid}\PYG{p}{,} \PYG{n}{PASS}\PYG{p}{,} \PYG{l+s+s1}{\PYGZsq{}}\PYG{l+s+s1}{note.note}\PYG{l+s+s1}{\PYGZsq{}}\PYG{p}{,} \PYG{l+s+s1}{\PYGZsq{}}\PYG{l+s+s1}{create}\PYG{l+s+s1}{\PYGZsq{}}\PYG{p}{,} \PYG{n}{args}\PYG{p}{)}
\end{sphinxVerbatim}

Examples can be easily adapted from XML-RPC to JSON-RPC.

\begin{sphinxadmonition}{note}{Note:}
There are a number of high-level APIs in various languages to access Odoo
systems without \sphinxstyleemphasis{explicitly} going through XML-RPC or JSON-RPC, such as:
\begin{itemize}
\item {} 
\sphinxurl{https://github.com/akretion/ooor}

\item {} 
\sphinxurl{https://github.com/syleam/openobject-library}

\item {} 
\sphinxurl{https://github.com/nicolas-van/openerp-client-lib}

\item {} 
\sphinxurl{http://pythonhosted.org/OdooRPC}

\item {} 
\sphinxurl{https://github.com/abhishek-jaiswal/php-openerp-lib}

\end{itemize}
\end{sphinxadmonition}
\phantomsection\label{\detokenize{howtos/backend:postgresql-s-documentation}}

\section{Customizing the web client}
\label{\detokenize{howtos/web:wkhtmltopdf}}\label{\detokenize{howtos/web::doc}}\label{\detokenize{howtos/web:customizing-the-web-client}}
Note: this section is really really out of date. It will be updated someday,
but meanwhile, this tutorial will probably be frustrating to follow, since it
was written a long time ago.

This guide is about creating modules for Odoo’s web client.

To create websites with Odoo, see {\hyperref[\detokenize{howtos/website::doc}]{\sphinxcrossref{\DUrole{doc}{Building a Website}}}}; to add business capabilities
or extend existing business systems of Odoo, see {\hyperref[\detokenize{howtos/backend::doc}]{\sphinxcrossref{\DUrole{doc}{Building a Module}}}}.

\begin{sphinxadmonition}{warning}{Warning:}
This guide assumes knowledge of:
\begin{itemize}
\item {} 
Javascript basics and good practices

\item {} 
\sphinxhref{http://jquery.org}{jQuery}

\item {} 
\sphinxhref{http://underscorejs.org}{Underscore.js}

\end{itemize}

It also requires {\hyperref[\detokenize{setup/install:setup-install}]{\sphinxcrossref{\DUrole{std,std-ref}{an installed Odoo}}}}, and \sphinxhref{http://git-scm.com}{Git}.
\end{sphinxadmonition}


\subsection{A Simple Module}
\label{\detokenize{howtos/web:a-simple-module}}
Let’s start with a simple Odoo module holding basic web component
configuration and letting us test the web framework.

The example module is available online and can be downloaded using the
following command:

\fvset{hllines={, ,}}%
\begin{sphinxVerbatim}[commandchars=\\\{\}]
\PYG{g+gp}{\PYGZdl{}} git clone http://github.com/odoo/petstore
\end{sphinxVerbatim}

This will create a \sphinxcode{\sphinxupquote{petstore}} folder wherever you executed the command.
You then need to add that folder to Odoo’s
{\hyperref[\detokenize{reference/cmdline:cmdoption-odoo-bin-addons-path}]{\sphinxcrossref{\sphinxcode{\sphinxupquote{addons path}}}}}, create a new database and
install the \sphinxcode{\sphinxupquote{oepetstore}} module.

If you browse the \sphinxcode{\sphinxupquote{petstore}} folder, you should see the following content:

\fvset{hllines={, ,}}%
\begin{sphinxVerbatim}[commandchars=\\\{\}]
oepetstore
\textbar{}\PYGZhy{}\PYGZhy{} images
\textbar{}   \textbar{}\PYGZhy{}\PYGZhy{} alligator.jpg
\textbar{}   \textbar{}\PYGZhy{}\PYGZhy{} ball.jpg
\textbar{}   \textbar{}\PYGZhy{}\PYGZhy{} crazy\PYGZus{}circle.jpg
\textbar{}   \textbar{}\PYGZhy{}\PYGZhy{} fish.jpg
\textbar{}   {}`\PYGZhy{}\PYGZhy{} mice.jpg
\textbar{}\PYGZhy{}\PYGZhy{} \PYGZus{}\PYGZus{}init\PYGZus{}\PYGZus{}.py
\textbar{}\PYGZhy{}\PYGZhy{} oepetstore.message\PYGZus{}of\PYGZus{}the\PYGZus{}day.csv
\textbar{}\PYGZhy{}\PYGZhy{} \PYGZus{}\PYGZus{}manifest\PYGZus{}\PYGZus{}.py
\textbar{}\PYGZhy{}\PYGZhy{} petstore\PYGZus{}data.xml
\textbar{}\PYGZhy{}\PYGZhy{} petstore.py
\textbar{}\PYGZhy{}\PYGZhy{} petstore.xml
{}`\PYGZhy{}\PYGZhy{} static
    {}`\PYGZhy{}\PYGZhy{} src
        \textbar{}\PYGZhy{}\PYGZhy{} css
        \textbar{}   {}`\PYGZhy{}\PYGZhy{} petstore.css
        \textbar{}\PYGZhy{}\PYGZhy{} js
        \textbar{}   {}`\PYGZhy{}\PYGZhy{} petstore.js
        {}`\PYGZhy{}\PYGZhy{} xml
            {}`\PYGZhy{}\PYGZhy{} petstore.xml
\end{sphinxVerbatim}

The module already holds various server customizations. We’ll come back to
these later, for now let’s focus on the web-related content, in the \sphinxcode{\sphinxupquote{static}}
folder.

Files used in the “web” side of an Odoo module must be placed in a \sphinxcode{\sphinxupquote{static}}
folder so they are available to a web browser, files outside that folder can
not be fetched by browsers. The \sphinxcode{\sphinxupquote{src/css}}, \sphinxcode{\sphinxupquote{src/js}} and \sphinxcode{\sphinxupquote{src/xml}}
sub-folders are conventional and not strictly necessary.
\begin{description}
\item[{\sphinxcode{\sphinxupquote{oepetstore/static/css/petstore.css}}}] \leavevmode
Currently empty, will hold the \sphinxhref{http://www.w3.org/Style/CSS/Overview.en.html}{CSS} for pet store content

\item[{\sphinxcode{\sphinxupquote{oepetstore/static/xml/petstore.xml}}}] \leavevmode
Mostly empty, will hold {\hyperref[\detokenize{reference/qweb:reference-qweb}]{\sphinxcrossref{\DUrole{std,std-ref}{QWeb}}}} templates

\item[{\sphinxcode{\sphinxupquote{oepetstore/static/js/petstore.js}}}] \leavevmode
The most important (and interesting) part, contains the logic of the
application (or at least its web-browser side) as javascript. It should
currently look like:

\fvset{hllines={, ,}}%
\begin{sphinxVerbatim}[commandchars=\\\{\}]
\PYG{n+nx}{odoo}\PYG{p}{.}\PYG{n+nx}{oepetstore} \PYG{o}{=} \PYG{k+kd}{function}\PYG{p}{(}\PYG{n+nx}{instance}\PYG{p}{,} \PYG{n+nx}{local}\PYG{p}{)} \PYG{p}{\PYGZob{}}
    \PYG{k+kd}{var} \PYG{n+nx}{\PYGZus{}t} \PYG{o}{=} \PYG{n+nx}{instance}\PYG{p}{.}\PYG{n+nx}{web}\PYG{p}{.}\PYG{n+nx}{\PYGZus{}t}\PYG{p}{,}
        \PYG{n+nx}{\PYGZus{}lt} \PYG{o}{=} \PYG{n+nx}{instance}\PYG{p}{.}\PYG{n+nx}{web}\PYG{p}{.}\PYG{n+nx}{\PYGZus{}lt}\PYG{p}{;}
    \PYG{k+kd}{var} \PYG{n+nx}{QWeb} \PYG{o}{=} \PYG{n+nx}{instance}\PYG{p}{.}\PYG{n+nx}{web}\PYG{p}{.}\PYG{n+nx}{qweb}\PYG{p}{;}

    \PYG{n+nx}{local}\PYG{p}{.}\PYG{n+nx}{HomePage} \PYG{o}{=} \PYG{n+nx}{instance}\PYG{p}{.}\PYG{n+nx}{Widget}\PYG{p}{.}\PYG{n+nx}{extend}\PYG{p}{(}\PYG{p}{\PYGZob{}}
        \PYG{n+nx}{start}\PYG{o}{:} \PYG{k+kd}{function}\PYG{p}{(}\PYG{p}{)} \PYG{p}{\PYGZob{}}
            \PYG{n+nx}{console}\PYG{p}{.}\PYG{n+nx}{log}\PYG{p}{(}\PYG{l+s+s2}{\PYGZdq{}pet store home page loaded\PYGZdq{}}\PYG{p}{)}\PYG{p}{;}
        \PYG{p}{\PYGZcb{}}\PYG{p}{,}
    \PYG{p}{\PYGZcb{}}\PYG{p}{)}\PYG{p}{;}

    \PYG{n+nx}{instance}\PYG{p}{.}\PYG{n+nx}{web}\PYG{p}{.}\PYG{n+nx}{client\PYGZus{}actions}\PYG{p}{.}\PYG{n+nx}{add}\PYG{p}{(}
        \PYG{l+s+s1}{\PYGZsq{}petstore.homepage\PYGZsq{}}\PYG{p}{,} \PYG{l+s+s1}{\PYGZsq{}instance.oepetstore.HomePage\PYGZsq{}}\PYG{p}{)}\PYG{p}{;}
\PYG{p}{\PYGZcb{}}
\end{sphinxVerbatim}

\end{description}

Which only prints a small message in the browser’s console.

The files in the \sphinxcode{\sphinxupquote{static}} folder, need to be defined within the module in order for them to be loaded correctly. Everything in \sphinxcode{\sphinxupquote{src/xml}} is defined in \sphinxcode{\sphinxupquote{\_\_manifest\_\_.py}} while the contents of \sphinxcode{\sphinxupquote{src/css}} and \sphinxcode{\sphinxupquote{src/js}} are defined in \sphinxcode{\sphinxupquote{petstore.xml}}, or a similar file.

\begin{sphinxadmonition}{warning}{Warning:}
All JavaScript files are concatenated and \DUrole{xref,std,std-term}{minified} to improve
application load time.

One of the drawback is debugging becomes more difficult as
individual files disappear and the code is made significantly less
readable. It is possible to disable this process by enabling the
“developer mode”: log into your Odoo instance (user \sphinxstyleemphasis{admin} password
\sphinxstyleemphasis{admin} by default) open the user menu (in the top-right corner of the
Odoo screen) and select \sphinxmenuselection{About Odoo} then \sphinxmenuselection{Activate
the developer mode}:

\noindent{\hspace*{\fill}\sphinxincludegraphics{{about_odoo}.png}\hspace*{\fill}}

\noindent{\hspace*{\fill}\sphinxincludegraphics{{devmode}.png}\hspace*{\fill}}

This will reload the web client with optimizations disabled, making
development and debugging significantly more comfortable.
\end{sphinxadmonition}


\subsection{Odoo JavaScript Module}
\label{\detokenize{howtos/web:odoo-javascript-module}}\label{\detokenize{howtos/web:index-0}}
Javascript doesn’t have built-in modules. As a result variables defined in
different files are all mashed together and may conflict. This has given rise
to various module patterns used to build clean namespaces and limit risks of
naming conflicts.

The Odoo framework uses one such pattern to define modules within web addons,
in order to both namespace code and correctly order its loading.

\sphinxcode{\sphinxupquote{oepetstore/static/js/petstore.js}} contains a module declaration:

\fvset{hllines={, ,}}%
\begin{sphinxVerbatim}[commandchars=\\\{\}]
\PYG{n+nx}{odoo}\PYG{p}{.}\PYG{n+nx}{oepetstore} \PYG{o}{=} \PYG{k+kd}{function}\PYG{p}{(}\PYG{n+nx}{instance}\PYG{p}{,} \PYG{n+nx}{local}\PYG{p}{)} \PYG{p}{\PYGZob{}}
    \PYG{n+nx}{local}\PYG{p}{.}\PYG{n+nx}{xxx} \PYG{o}{=} \PYG{p}{...}\PYG{p}{;}
\PYG{p}{\PYGZcb{}}
\end{sphinxVerbatim}

In Odoo web, modules are declared as functions set on the global \sphinxcode{\sphinxupquote{odoo}}
variable. The function’s name must be the same as the addon (in this case
\sphinxcode{\sphinxupquote{oepetstore}}) so the framework can find it, and automatically initialize it.

When the web client loads your module it will call the root function
and provide two parameters:
\begin{itemize}
\item {} 
the first parameter is the current instance of the Odoo web client, it gives
access to various capabilities defined by the Odoo (translations,
network services) as well as objects defined by the core or by other
modules.

\item {} 
the second parameter is your own local namespace automatically created by
the web client. Objects and variables which should be accessible from
outside your module (either because the Odoo web client needs to call them
or because others may want to customize them) should be set inside that
namespace.

\end{itemize}


\subsection{Classes}
\label{\detokenize{howtos/web:classes}}
Much as modules, and contrary to most object-oriented languages, javascript
does not build in \sphinxstyleemphasis{classes}%
\begin{footnote}[1]\sphinxAtStartFootnote
as a separate concept from instances. In many languages classes
are full-fledged objects and themselves instance (of
metaclasses) but there remains two fairly separate hierarchies
between classes and instances
%
\end{footnote} although it provides roughly
equivalent (if lower-level and more verbose) mechanisms.

For simplicity and developer-friendliness Odoo web provides a class
system based on John Resig’s \sphinxhref{http://ejohn.org/blog/simple-javascript-inheritance/}{Simple JavaScript Inheritance}.

New classes are defined by calling the \sphinxcode{\sphinxupquote{extend()}}
method of \sphinxcode{\sphinxupquote{odoo.web.Class()}}:

\fvset{hllines={, ,}}%
\begin{sphinxVerbatim}[commandchars=\\\{\}]
\PYG{k+kd}{var} \PYG{n+nx}{MyClass} \PYG{o}{=} \PYG{n+nx}{instance}\PYG{p}{.}\PYG{n+nx}{web}\PYG{p}{.}\PYG{n+nx}{Class}\PYG{p}{.}\PYG{n+nx}{extend}\PYG{p}{(}\PYG{p}{\PYGZob{}}
    \PYG{n+nx}{say\PYGZus{}hello}\PYG{o}{:} \PYG{k+kd}{function}\PYG{p}{(}\PYG{p}{)} \PYG{p}{\PYGZob{}}
        \PYG{n+nx}{console}\PYG{p}{.}\PYG{n+nx}{log}\PYG{p}{(}\PYG{l+s+s2}{\PYGZdq{}hello\PYGZdq{}}\PYG{p}{)}\PYG{p}{;}
    \PYG{p}{\PYGZcb{}}\PYG{p}{,}
\PYG{p}{\PYGZcb{}}\PYG{p}{)}\PYG{p}{;}
\end{sphinxVerbatim}

The \sphinxcode{\sphinxupquote{extend()}} method takes a dictionary describing
the new class’s content (methods and static attributes). In this case, it will
only have a \sphinxcode{\sphinxupquote{say\_hello}} method which takes no parameters.

Classes are instantiated using the \sphinxcode{\sphinxupquote{new}} operator:

\fvset{hllines={, ,}}%
\begin{sphinxVerbatim}[commandchars=\\\{\}]
\PYG{k+kd}{var} \PYG{n+nx}{my\PYGZus{}object} \PYG{o}{=} \PYG{k}{new} \PYG{n+nx}{MyClass}\PYG{p}{(}\PYG{p}{)}\PYG{p}{;}
\PYG{n+nx}{my\PYGZus{}object}\PYG{p}{.}\PYG{n+nx}{say\PYGZus{}hello}\PYG{p}{(}\PYG{p}{)}\PYG{p}{;}
\PYG{c+c1}{// print \PYGZdq{}hello\PYGZdq{} in the console}
\end{sphinxVerbatim}

And attributes of the instance can be accessed via \sphinxcode{\sphinxupquote{this}}:

\fvset{hllines={, ,}}%
\begin{sphinxVerbatim}[commandchars=\\\{\}]
\PYG{k+kd}{var} \PYG{n+nx}{MyClass} \PYG{o}{=} \PYG{n+nx}{instance}\PYG{p}{.}\PYG{n+nx}{web}\PYG{p}{.}\PYG{n+nx}{Class}\PYG{p}{.}\PYG{n+nx}{extend}\PYG{p}{(}\PYG{p}{\PYGZob{}}
    \PYG{n+nx}{say\PYGZus{}hello}\PYG{o}{:} \PYG{k+kd}{function}\PYG{p}{(}\PYG{p}{)} \PYG{p}{\PYGZob{}}
        \PYG{n+nx}{console}\PYG{p}{.}\PYG{n+nx}{log}\PYG{p}{(}\PYG{l+s+s2}{\PYGZdq{}hello\PYGZdq{}}\PYG{p}{,} \PYG{k}{this}\PYG{p}{.}\PYG{n+nx}{name}\PYG{p}{)}\PYG{p}{;}
    \PYG{p}{\PYGZcb{}}\PYG{p}{,}
\PYG{p}{\PYGZcb{}}\PYG{p}{)}\PYG{p}{;}

\PYG{k+kd}{var} \PYG{n+nx}{my\PYGZus{}object} \PYG{o}{=} \PYG{k}{new} \PYG{n+nx}{MyClass}\PYG{p}{(}\PYG{p}{)}\PYG{p}{;}
\PYG{n+nx}{my\PYGZus{}object}\PYG{p}{.}\PYG{n+nx}{name} \PYG{o}{=} \PYG{l+s+s2}{\PYGZdq{}Bob\PYGZdq{}}\PYG{p}{;}
\PYG{n+nx}{my\PYGZus{}object}\PYG{p}{.}\PYG{n+nx}{say\PYGZus{}hello}\PYG{p}{(}\PYG{p}{)}\PYG{p}{;}
\PYG{c+c1}{// print \PYGZdq{}hello Bob\PYGZdq{} in the console}
\end{sphinxVerbatim}

Classes can provide an initializer to perform the initial setup of the
instance, by defining an \sphinxcode{\sphinxupquote{init()}} method. The initializer receives the
parameters passed when using the \sphinxcode{\sphinxupquote{new}} operator:

\fvset{hllines={, ,}}%
\begin{sphinxVerbatim}[commandchars=\\\{\}]
\PYG{k+kd}{var} \PYG{n+nx}{MyClass} \PYG{o}{=} \PYG{n+nx}{instance}\PYG{p}{.}\PYG{n+nx}{web}\PYG{p}{.}\PYG{n+nx}{Class}\PYG{p}{.}\PYG{n+nx}{extend}\PYG{p}{(}\PYG{p}{\PYGZob{}}
    \PYG{n+nx}{init}\PYG{o}{:} \PYG{k+kd}{function}\PYG{p}{(}\PYG{n+nx}{name}\PYG{p}{)} \PYG{p}{\PYGZob{}}
        \PYG{k}{this}\PYG{p}{.}\PYG{n+nx}{name} \PYG{o}{=} \PYG{n+nx}{name}\PYG{p}{;}
    \PYG{p}{\PYGZcb{}}\PYG{p}{,}
    \PYG{n+nx}{say\PYGZus{}hello}\PYG{o}{:} \PYG{k+kd}{function}\PYG{p}{(}\PYG{p}{)} \PYG{p}{\PYGZob{}}
        \PYG{n+nx}{console}\PYG{p}{.}\PYG{n+nx}{log}\PYG{p}{(}\PYG{l+s+s2}{\PYGZdq{}hello\PYGZdq{}}\PYG{p}{,} \PYG{k}{this}\PYG{p}{.}\PYG{n+nx}{name}\PYG{p}{)}\PYG{p}{;}
    \PYG{p}{\PYGZcb{}}\PYG{p}{,}
\PYG{p}{\PYGZcb{}}\PYG{p}{)}\PYG{p}{;}

\PYG{k+kd}{var} \PYG{n+nx}{my\PYGZus{}object} \PYG{o}{=} \PYG{k}{new} \PYG{n+nx}{MyClass}\PYG{p}{(}\PYG{l+s+s2}{\PYGZdq{}Bob\PYGZdq{}}\PYG{p}{)}\PYG{p}{;}
\PYG{n+nx}{my\PYGZus{}object}\PYG{p}{.}\PYG{n+nx}{say\PYGZus{}hello}\PYG{p}{(}\PYG{p}{)}\PYG{p}{;}
\PYG{c+c1}{// print \PYGZdq{}hello Bob\PYGZdq{} in the console}
\end{sphinxVerbatim}

It is also possible to create subclasses from existing (used-defined) classes
by calling \sphinxcode{\sphinxupquote{extend()}} on the parent class, as is done
to subclass \sphinxcode{\sphinxupquote{Class()}}:

\fvset{hllines={, ,}}%
\begin{sphinxVerbatim}[commandchars=\\\{\}]
\PYG{k+kd}{var} \PYG{n+nx}{MySpanishClass} \PYG{o}{=} \PYG{n+nx}{MyClass}\PYG{p}{.}\PYG{n+nx}{extend}\PYG{p}{(}\PYG{p}{\PYGZob{}}
    \PYG{n+nx}{say\PYGZus{}hello}\PYG{o}{:} \PYG{k+kd}{function}\PYG{p}{(}\PYG{p}{)} \PYG{p}{\PYGZob{}}
        \PYG{n+nx}{console}\PYG{p}{.}\PYG{n+nx}{log}\PYG{p}{(}\PYG{l+s+s2}{\PYGZdq{}hola\PYGZdq{}}\PYG{p}{,} \PYG{k}{this}\PYG{p}{.}\PYG{n+nx}{name}\PYG{p}{)}\PYG{p}{;}
    \PYG{p}{\PYGZcb{}}\PYG{p}{,}
\PYG{p}{\PYGZcb{}}\PYG{p}{)}\PYG{p}{;}

\PYG{k+kd}{var} \PYG{n+nx}{my\PYGZus{}object} \PYG{o}{=} \PYG{k}{new} \PYG{n+nx}{MySpanishClass}\PYG{p}{(}\PYG{l+s+s2}{\PYGZdq{}Bob\PYGZdq{}}\PYG{p}{)}\PYG{p}{;}
\PYG{n+nx}{my\PYGZus{}object}\PYG{p}{.}\PYG{n+nx}{say\PYGZus{}hello}\PYG{p}{(}\PYG{p}{)}\PYG{p}{;}
\PYG{c+c1}{// print \PYGZdq{}hola Bob\PYGZdq{} in the console}
\end{sphinxVerbatim}

When overriding a method using inheritance, you can use \sphinxcode{\sphinxupquote{this.\_super()}} to
call the original method:

\fvset{hllines={, ,}}%
\begin{sphinxVerbatim}[commandchars=\\\{\}]
\PYG{k+kd}{var} \PYG{n+nx}{MySpanishClass} \PYG{o}{=} \PYG{n+nx}{MyClass}\PYG{p}{.}\PYG{n+nx}{extend}\PYG{p}{(}\PYG{p}{\PYGZob{}}
    \PYG{n+nx}{say\PYGZus{}hello}\PYG{o}{:} \PYG{k+kd}{function}\PYG{p}{(}\PYG{p}{)} \PYG{p}{\PYGZob{}}
        \PYG{k}{this}\PYG{p}{.}\PYG{n+nx}{\PYGZus{}super}\PYG{p}{(}\PYG{p}{)}\PYG{p}{;}
        \PYG{n+nx}{console}\PYG{p}{.}\PYG{n+nx}{log}\PYG{p}{(}\PYG{l+s+s2}{\PYGZdq{}translation in Spanish: hola\PYGZdq{}}\PYG{p}{,} \PYG{k}{this}\PYG{p}{.}\PYG{n+nx}{name}\PYG{p}{)}\PYG{p}{;}
    \PYG{p}{\PYGZcb{}}\PYG{p}{,}
\PYG{p}{\PYGZcb{}}\PYG{p}{)}\PYG{p}{;}

\PYG{k+kd}{var} \PYG{n+nx}{my\PYGZus{}object} \PYG{o}{=} \PYG{k}{new} \PYG{n+nx}{MySpanishClass}\PYG{p}{(}\PYG{l+s+s2}{\PYGZdq{}Bob\PYGZdq{}}\PYG{p}{)}\PYG{p}{;}
\PYG{n+nx}{my\PYGZus{}object}\PYG{p}{.}\PYG{n+nx}{say\PYGZus{}hello}\PYG{p}{(}\PYG{p}{)}\PYG{p}{;}
\PYG{c+c1}{// print \PYGZdq{}hello Bob \PYGZbs{}n translation in Spanish: hola Bob\PYGZdq{} in the console}
\end{sphinxVerbatim}

\begin{sphinxadmonition}{warning}{Warning:}
\sphinxcode{\sphinxupquote{\_super}} is not a standard method, it is set on-the-fly to the next
method in the current inheritance chain, if any. It is only defined
during the \sphinxstyleemphasis{synchronous} part of a method call, for use in asynchronous
handlers (after network calls or in \sphinxcode{\sphinxupquote{setTimeout}} callbacks) a reference
to its value should be retained, it should not be accessed via \sphinxcode{\sphinxupquote{this}}:

\fvset{hllines={, ,}}%
\begin{sphinxVerbatim}[commandchars=\\\{\}]
\PYG{c+c1}{// broken, will generate an error}
\PYG{n+nx}{say\PYGZus{}hello}\PYG{o}{:} \PYG{k+kd}{function} \PYG{p}{(}\PYG{p}{)} \PYG{p}{\PYGZob{}}
    \PYG{n+nx}{setTimeout}\PYG{p}{(}\PYG{k+kd}{function} \PYG{p}{(}\PYG{p}{)} \PYG{p}{\PYGZob{}}
        \PYG{k}{this}\PYG{p}{.}\PYG{n+nx}{\PYGZus{}super}\PYG{p}{(}\PYG{p}{)}\PYG{p}{;}
    \PYG{p}{\PYGZcb{}}\PYG{p}{.}\PYG{n+nx}{bind}\PYG{p}{(}\PYG{k}{this}\PYG{p}{)}\PYG{p}{,} \PYG{l+m+mi}{0}\PYG{p}{)}\PYG{p}{;}
\PYG{p}{\PYGZcb{}}

\PYG{c+c1}{// correct}
\PYG{n+nx}{say\PYGZus{}hello}\PYG{o}{:} \PYG{k+kd}{function} \PYG{p}{(}\PYG{p}{)} \PYG{p}{\PYGZob{}}
    \PYG{c+c1}{// don\PYGZsq{}t forget .bind()}
    \PYG{k+kd}{var} \PYG{n+nx}{\PYGZus{}super} \PYG{o}{=} \PYG{k}{this}\PYG{p}{.}\PYG{n+nx}{\PYGZus{}super}\PYG{p}{.}\PYG{n+nx}{bind}\PYG{p}{(}\PYG{k}{this}\PYG{p}{)}\PYG{p}{;}
    \PYG{n+nx}{setTimeout}\PYG{p}{(}\PYG{k+kd}{function} \PYG{p}{(}\PYG{p}{)} \PYG{p}{\PYGZob{}}
        \PYG{n+nx}{\PYGZus{}super}\PYG{p}{(}\PYG{p}{)}\PYG{p}{;}
    \PYG{p}{\PYGZcb{}}\PYG{p}{.}\PYG{n+nx}{bind}\PYG{p}{(}\PYG{k}{this}\PYG{p}{)}\PYG{p}{,} \PYG{l+m+mi}{0}\PYG{p}{)}\PYG{p}{;}
\PYG{p}{\PYGZcb{}}
\end{sphinxVerbatim}
\end{sphinxadmonition}


\subsection{Widgets Basics}
\label{\detokenize{howtos/web:widgets-basics}}
The Odoo web client bundles \sphinxhref{http://jquery.org}{jQuery} for easy DOM manipulation. It is useful
and provides a better API than standard \sphinxhref{http://www.w3.org/TR/DOM-Level-3-Core/}{W3C DOM}%
\begin{footnote}[2]\sphinxAtStartFootnote
as well as papering over cross-browser differences, although
this has become less necessary over time
%
\end{footnote}, but
insufficient to structure complex applications leading to difficult
maintenance.

Much like object-oriented desktop UI toolkits (e.g. \sphinxhref{http://qt-project.org}{Qt}, \sphinxhref{https://developer.apple.com/technologies/mac/cocoa.html}{Cocoa} or \sphinxhref{http://www.gtk.org}{GTK}),
Odoo Web makes specific components responsible for sections of a page. In
Odoo web, the base for such components is the \sphinxcode{\sphinxupquote{Widget()}}
class, a component specialized in handling a page section and displaying
information for the user.


\subsubsection{Your First Widget}
\label{\detokenize{howtos/web:your-first-widget}}
The initial demonstration module already provides a basic widget:

\fvset{hllines={, ,}}%
\begin{sphinxVerbatim}[commandchars=\\\{\}]
\PYG{n+nx}{local}\PYG{p}{.}\PYG{n+nx}{HomePage} \PYG{o}{=} \PYG{n+nx}{instance}\PYG{p}{.}\PYG{n+nx}{Widget}\PYG{p}{.}\PYG{n+nx}{extend}\PYG{p}{(}\PYG{p}{\PYGZob{}}
    \PYG{n+nx}{start}\PYG{o}{:} \PYG{k+kd}{function}\PYG{p}{(}\PYG{p}{)} \PYG{p}{\PYGZob{}}
        \PYG{n+nx}{console}\PYG{p}{.}\PYG{n+nx}{log}\PYG{p}{(}\PYG{l+s+s2}{\PYGZdq{}pet store home page loaded\PYGZdq{}}\PYG{p}{)}\PYG{p}{;}
    \PYG{p}{\PYGZcb{}}\PYG{p}{,}
\PYG{p}{\PYGZcb{}}\PYG{p}{)}\PYG{p}{;}
\end{sphinxVerbatim}

It extends \sphinxcode{\sphinxupquote{Widget()}} and overrides the standard method
\sphinxcode{\sphinxupquote{start()}}, which — much like the previous \sphinxcode{\sphinxupquote{MyClass}}
— does little for now.

This line at the end of the file:

\fvset{hllines={, ,}}%
\begin{sphinxVerbatim}[commandchars=\\\{\}]
\PYG{n+nx}{instance}\PYG{p}{.}\PYG{n+nx}{web}\PYG{p}{.}\PYG{n+nx}{client\PYGZus{}actions}\PYG{p}{.}\PYG{n+nx}{add}\PYG{p}{(}
    \PYG{l+s+s1}{\PYGZsq{}petstore.homepage\PYGZsq{}}\PYG{p}{,} \PYG{l+s+s1}{\PYGZsq{}instance.oepetstore.HomePage\PYGZsq{}}\PYG{p}{)}\PYG{p}{;}
\end{sphinxVerbatim}

registers our basic widget as a client action. Client actions will be
explained later, for now this is just what allows our widget to
be called and displayed when we select the
\sphinxmenuselection{Pet Store \(\rightarrow\) Pet Store \(\rightarrow\) Home Page} menu.

\begin{sphinxadmonition}{warning}{Warning:}
because the widget will be called from outside our module, the web client
needs its “fully qualified” name, not the local version.
\end{sphinxadmonition}


\subsubsection{Display Content}
\label{\detokenize{howtos/web:display-content}}
Widgets have a number of methods and features, but the basics are simple:
\begin{itemize}
\item {} 
set up a widget

\item {} 
format the widget’s data

\item {} 
display the widget

\end{itemize}

The \sphinxcode{\sphinxupquote{HomePage}} widget already has a \sphinxcode{\sphinxupquote{start()}}
method. That method is part of the normal widget lifecycle and automatically
called once the widget is inserted in the page. We can use it to display some
content.

All widgets have a \sphinxcode{\sphinxupquote{\$el}} which represents the
section of page they’re in charge of (as a \sphinxhref{http://jquery.org}{jQuery} object). Widget content
should be inserted there. By default, \sphinxcode{\sphinxupquote{\$el}} is an
empty \sphinxcode{\sphinxupquote{\textless{}div\textgreater{}}} element.

A \sphinxcode{\sphinxupquote{\textless{}div\textgreater{}}} element is usually invisible to the user if it has no content (or
without specific styles giving it a size) which is why nothing is displayed
on the page when \sphinxcode{\sphinxupquote{HomePage}} is launched.

Let’s add some content to the widget’s root element, using jQuery:

\fvset{hllines={, ,}}%
\begin{sphinxVerbatim}[commandchars=\\\{\}]
\PYG{n+nx}{local}\PYG{p}{.}\PYG{n+nx}{HomePage} \PYG{o}{=} \PYG{n+nx}{instance}\PYG{p}{.}\PYG{n+nx}{Widget}\PYG{p}{.}\PYG{n+nx}{extend}\PYG{p}{(}\PYG{p}{\PYGZob{}}
    \PYG{n+nx}{start}\PYG{o}{:} \PYG{k+kd}{function}\PYG{p}{(}\PYG{p}{)} \PYG{p}{\PYGZob{}}
        \PYG{k}{this}\PYG{p}{.}\PYG{n+nx}{\PYGZdl{}el}\PYG{p}{.}\PYG{n+nx}{append}\PYG{p}{(}\PYG{l+s+s2}{\PYGZdq{}\PYGZlt{}div\PYGZgt{}Hello dear Odoo user!\PYGZlt{}/div\PYGZgt{}\PYGZdq{}}\PYG{p}{)}\PYG{p}{;}
    \PYG{p}{\PYGZcb{}}\PYG{p}{,}
\PYG{p}{\PYGZcb{}}\PYG{p}{)}\PYG{p}{;}
\end{sphinxVerbatim}

That message will now appear when you open \sphinxmenuselection{Pet Store
\(\rightarrow\) Pet Store \(\rightarrow\) Home Page}

\begin{sphinxadmonition}{note}{Note:}
to refresh the javascript code loaded in Odoo Web, you will need to reload
the page. There is no need to restart the Odoo server.
\end{sphinxadmonition}

The \sphinxcode{\sphinxupquote{HomePage}} widget is used by Odoo Web and managed automatically.
To learn how to use a widget “from scratch” let’s create a new one:

\fvset{hllines={, ,}}%
\begin{sphinxVerbatim}[commandchars=\\\{\}]
\PYG{n+nx}{local}\PYG{p}{.}\PYG{n+nx}{GreetingsWidget} \PYG{o}{=} \PYG{n+nx}{instance}\PYG{p}{.}\PYG{n+nx}{Widget}\PYG{p}{.}\PYG{n+nx}{extend}\PYG{p}{(}\PYG{p}{\PYGZob{}}
    \PYG{n+nx}{start}\PYG{o}{:} \PYG{k+kd}{function}\PYG{p}{(}\PYG{p}{)} \PYG{p}{\PYGZob{}}
        \PYG{k}{this}\PYG{p}{.}\PYG{n+nx}{\PYGZdl{}el}\PYG{p}{.}\PYG{n+nx}{append}\PYG{p}{(}\PYG{l+s+s2}{\PYGZdq{}\PYGZlt{}div\PYGZgt{}We are so happy to see you again in this menu!\PYGZlt{}/div\PYGZgt{}\PYGZdq{}}\PYG{p}{)}\PYG{p}{;}
    \PYG{p}{\PYGZcb{}}\PYG{p}{,}
\PYG{p}{\PYGZcb{}}\PYG{p}{)}\PYG{p}{;}
\end{sphinxVerbatim}

We can now add our \sphinxcode{\sphinxupquote{GreetingsWidget}} to the \sphinxcode{\sphinxupquote{HomePage}} by using the
\sphinxcode{\sphinxupquote{GreetingsWidget}}’s \sphinxcode{\sphinxupquote{appendTo()}} method:

\fvset{hllines={, ,}}%
\begin{sphinxVerbatim}[commandchars=\\\{\}]
\PYG{n+nx}{local}\PYG{p}{.}\PYG{n+nx}{HomePage} \PYG{o}{=} \PYG{n+nx}{instance}\PYG{p}{.}\PYG{n+nx}{Widget}\PYG{p}{.}\PYG{n+nx}{extend}\PYG{p}{(}\PYG{p}{\PYGZob{}}
    \PYG{n+nx}{start}\PYG{o}{:} \PYG{k+kd}{function}\PYG{p}{(}\PYG{p}{)} \PYG{p}{\PYGZob{}}
        \PYG{k}{this}\PYG{p}{.}\PYG{n+nx}{\PYGZdl{}el}\PYG{p}{.}\PYG{n+nx}{append}\PYG{p}{(}\PYG{l+s+s2}{\PYGZdq{}\PYGZlt{}div\PYGZgt{}Hello dear Odoo user!\PYGZlt{}/div\PYGZgt{}\PYGZdq{}}\PYG{p}{)}\PYG{p}{;}
        \PYG{k+kd}{var} \PYG{n+nx}{greeting} \PYG{o}{=} \PYG{k}{new} \PYG{n+nx}{local}\PYG{p}{.}\PYG{n+nx}{GreetingsWidget}\PYG{p}{(}\PYG{k}{this}\PYG{p}{)}\PYG{p}{;}
        \PYG{k}{return} \PYG{n+nx}{greeting}\PYG{p}{.}\PYG{n+nx}{appendTo}\PYG{p}{(}\PYG{k}{this}\PYG{p}{.}\PYG{n+nx}{\PYGZdl{}el}\PYG{p}{)}\PYG{p}{;}
    \PYG{p}{\PYGZcb{}}\PYG{p}{,}
\PYG{p}{\PYGZcb{}}\PYG{p}{)}\PYG{p}{;}
\end{sphinxVerbatim}
\begin{itemize}
\item {} 
\sphinxcode{\sphinxupquote{HomePage}} first adds its own content to its DOM root

\item {} 
\sphinxcode{\sphinxupquote{HomePage}} then instantiates \sphinxcode{\sphinxupquote{GreetingsWidget}}

\item {} 
Finally it tells \sphinxcode{\sphinxupquote{GreetingsWidget}} where to insert itself, delegating part
of its \sphinxcode{\sphinxupquote{\$el}} to the \sphinxcode{\sphinxupquote{GreetingsWidget}}.

\end{itemize}

When the \sphinxcode{\sphinxupquote{appendTo()}} method is called, it asks the
widget to insert itself at the specified position and to display its content.
The \sphinxcode{\sphinxupquote{start()}} method will be called during the call
to \sphinxcode{\sphinxupquote{appendTo()}}.

To see what happens under the displayed interface, we will use the browser’s
DOM Explorer. But first let’s alter our widgets slightly so we can more easily
find where they are, by \sphinxcode{\sphinxupquote{adding a class to their root elements}}:

\fvset{hllines={, ,}}%
\begin{sphinxVerbatim}[commandchars=\\\{\}]
\PYG{n+nx}{local}\PYG{p}{.}\PYG{n+nx}{HomePage} \PYG{o}{=} \PYG{n+nx}{instance}\PYG{p}{.}\PYG{n+nx}{Widget}\PYG{p}{.}\PYG{n+nx}{extend}\PYG{p}{(}\PYG{p}{\PYGZob{}}
    \PYG{n+nx}{className}\PYG{o}{:} \PYG{l+s+s1}{\PYGZsq{}oe\PYGZus{}petstore\PYGZus{}homepage\PYGZsq{}}\PYG{p}{,}
    \PYG{p}{...}
\PYG{p}{\PYGZcb{}}\PYG{p}{)}\PYG{p}{;}
\PYG{n+nx}{local}\PYG{p}{.}\PYG{n+nx}{GreetingsWidget} \PYG{o}{=} \PYG{n+nx}{instance}\PYG{p}{.}\PYG{n+nx}{Widget}\PYG{p}{.}\PYG{n+nx}{extend}\PYG{p}{(}\PYG{p}{\PYGZob{}}
    \PYG{n+nx}{className}\PYG{o}{:} \PYG{l+s+s1}{\PYGZsq{}oe\PYGZus{}petstore\PYGZus{}greetings\PYGZsq{}}\PYG{p}{,}
    \PYG{p}{...}
\PYG{p}{\PYGZcb{}}\PYG{p}{)}\PYG{p}{;}
\end{sphinxVerbatim}

If you can find the relevant section of the DOM (right-click on the text
then \sphinxmenuselection{Inspect Element}), it should look like this:

\fvset{hllines={, ,}}%
\begin{sphinxVerbatim}[commandchars=\\\{\}]
\PYG{p}{\PYGZlt{}}\PYG{n+nt}{div} \PYG{n+na}{class}\PYG{o}{=}\PYG{l+s}{\PYGZdq{}oe\PYGZus{}petstore\PYGZus{}homepage\PYGZdq{}}\PYG{p}{\PYGZgt{}}
    \PYG{p}{\PYGZlt{}}\PYG{n+nt}{div}\PYG{p}{\PYGZgt{}}Hello dear Odoo user!\PYG{p}{\PYGZlt{}}\PYG{p}{/}\PYG{n+nt}{div}\PYG{p}{\PYGZgt{}}
    \PYG{p}{\PYGZlt{}}\PYG{n+nt}{div} \PYG{n+na}{class}\PYG{o}{=}\PYG{l+s}{\PYGZdq{}oe\PYGZus{}petstore\PYGZus{}greetings\PYGZdq{}}\PYG{p}{\PYGZgt{}}
        \PYG{p}{\PYGZlt{}}\PYG{n+nt}{div}\PYG{p}{\PYGZgt{}}We are so happy to see you again in this menu!\PYG{p}{\PYGZlt{}}\PYG{p}{/}\PYG{n+nt}{div}\PYG{p}{\PYGZgt{}}
    \PYG{p}{\PYGZlt{}}\PYG{p}{/}\PYG{n+nt}{div}\PYG{p}{\PYGZgt{}}
\PYG{p}{\PYGZlt{}}\PYG{p}{/}\PYG{n+nt}{div}\PYG{p}{\PYGZgt{}}
\end{sphinxVerbatim}

Which clearly shows the two \sphinxcode{\sphinxupquote{\textless{}div\textgreater{}}} elements automatically created by
\sphinxcode{\sphinxupquote{Widget()}}, because we added some classes on them.

We can also see the two message-holding divs we added ourselves

Finally, note the \sphinxcode{\sphinxupquote{\textless{}div class="oe\_petstore\_greetings"\textgreater{}}} element which
represents the \sphinxcode{\sphinxupquote{GreetingsWidget}} instance is \sphinxstyleemphasis{inside} the
\sphinxcode{\sphinxupquote{\textless{}div class="oe\_petstore\_homepage"\textgreater{}}} which represents the \sphinxcode{\sphinxupquote{HomePage}}
instance, since we appended


\subsubsection{Widget Parents and Children}
\label{\detokenize{howtos/web:widget-parents-and-children}}
In the previous part, we instantiated a widget using this syntax:

\fvset{hllines={, ,}}%
\begin{sphinxVerbatim}[commandchars=\\\{\}]
\PYG{k}{new} \PYG{n+nx}{local}\PYG{p}{.}\PYG{n+nx}{GreetingsWidget}\PYG{p}{(}\PYG{k}{this}\PYG{p}{)}\PYG{p}{;}
\end{sphinxVerbatim}

The first argument is \sphinxcode{\sphinxupquote{this}}, which in that case was a \sphinxcode{\sphinxupquote{HomePage}}
instance. This tells the widget being created which other widget is its
\sphinxstyleemphasis{parent}.

As we’ve seen, widgets are usually inserted in the DOM by another widget and
\sphinxstyleemphasis{inside} that other widget’s root element. This means most widgets are “part”
of another widget, and exist on behalf of it. We call the container the
\sphinxstyleemphasis{parent}, and the contained widget the \sphinxstyleemphasis{child}.

Due to multiple technical and conceptual reasons, it is necessary for a widget
to know who is its parent and who are its children.
\begin{description}
\item[{\sphinxcode{\sphinxupquote{getParent()}}}] \leavevmode
can be used to get the parent of a widget:

\fvset{hllines={, ,}}%
\begin{sphinxVerbatim}[commandchars=\\\{\}]
\PYG{n+nx}{local}\PYG{p}{.}\PYG{n+nx}{GreetingsWidget} \PYG{o}{=} \PYG{n+nx}{instance}\PYG{p}{.}\PYG{n+nx}{Widget}\PYG{p}{.}\PYG{n+nx}{extend}\PYG{p}{(}\PYG{p}{\PYGZob{}}
    \PYG{n+nx}{start}\PYG{o}{:} \PYG{k+kd}{function}\PYG{p}{(}\PYG{p}{)} \PYG{p}{\PYGZob{}}
        \PYG{n+nx}{console}\PYG{p}{.}\PYG{n+nx}{log}\PYG{p}{(}\PYG{k}{this}\PYG{p}{.}\PYG{n+nx}{getParent}\PYG{p}{(}\PYG{p}{)}\PYG{p}{.}\PYG{n+nx}{\PYGZdl{}el} \PYG{p}{)}\PYG{p}{;}
        \PYG{c+c1}{// will print \PYGZdq{}div.oe\PYGZus{}petstore\PYGZus{}homepage\PYGZdq{} in the console}
    \PYG{p}{\PYGZcb{}}\PYG{p}{,}
\PYG{p}{\PYGZcb{}}\PYG{p}{)}\PYG{p}{;}
\end{sphinxVerbatim}

\item[{\sphinxcode{\sphinxupquote{getChildren()}}}] \leavevmode
can be used to get a list of its children:

\fvset{hllines={, ,}}%
\begin{sphinxVerbatim}[commandchars=\\\{\}]
\PYG{n+nx}{local}\PYG{p}{.}\PYG{n+nx}{HomePage} \PYG{o}{=} \PYG{n+nx}{instance}\PYG{p}{.}\PYG{n+nx}{Widget}\PYG{p}{.}\PYG{n+nx}{extend}\PYG{p}{(}\PYG{p}{\PYGZob{}}
    \PYG{n+nx}{start}\PYG{o}{:} \PYG{k+kd}{function}\PYG{p}{(}\PYG{p}{)} \PYG{p}{\PYGZob{}}
        \PYG{k+kd}{var} \PYG{n+nx}{greeting} \PYG{o}{=} \PYG{k}{new} \PYG{n+nx}{local}\PYG{p}{.}\PYG{n+nx}{GreetingsWidget}\PYG{p}{(}\PYG{k}{this}\PYG{p}{)}\PYG{p}{;}
        \PYG{n+nx}{greeting}\PYG{p}{.}\PYG{n+nx}{appendTo}\PYG{p}{(}\PYG{k}{this}\PYG{p}{.}\PYG{n+nx}{\PYGZdl{}el}\PYG{p}{)}\PYG{p}{;}
        \PYG{n+nx}{console}\PYG{p}{.}\PYG{n+nx}{log}\PYG{p}{(}\PYG{k}{this}\PYG{p}{.}\PYG{n+nx}{getChildren}\PYG{p}{(}\PYG{p}{)}\PYG{p}{[}\PYG{l+m+mi}{0}\PYG{p}{]}\PYG{p}{.}\PYG{n+nx}{\PYGZdl{}el}\PYG{p}{)}\PYG{p}{;}
        \PYG{c+c1}{// will print \PYGZdq{}div.oe\PYGZus{}petstore\PYGZus{}greetings\PYGZdq{} in the console}
    \PYG{p}{\PYGZcb{}}\PYG{p}{,}
\PYG{p}{\PYGZcb{}}\PYG{p}{)}\PYG{p}{;}
\end{sphinxVerbatim}

\end{description}

When overriding the \sphinxcode{\sphinxupquote{init()}} method of a widget it is
\sphinxstyleemphasis{of the utmost importance} to pass the parent to the \sphinxcode{\sphinxupquote{this.\_super()}} call,
otherwise the relation will not be set up correctly:

\fvset{hllines={, ,}}%
\begin{sphinxVerbatim}[commandchars=\\\{\}]
\PYG{n+nx}{local}\PYG{p}{.}\PYG{n+nx}{GreetingsWidget} \PYG{o}{=} \PYG{n+nx}{instance}\PYG{p}{.}\PYG{n+nx}{Widget}\PYG{p}{.}\PYG{n+nx}{extend}\PYG{p}{(}\PYG{p}{\PYGZob{}}
    \PYG{n+nx}{init}\PYG{o}{:} \PYG{k+kd}{function}\PYG{p}{(}\PYG{n+nx}{parent}\PYG{p}{,} \PYG{n+nx}{name}\PYG{p}{)} \PYG{p}{\PYGZob{}}
        \PYG{k}{this}\PYG{p}{.}\PYG{n+nx}{\PYGZus{}super}\PYG{p}{(}\PYG{n+nx}{parent}\PYG{p}{)}\PYG{p}{;}
        \PYG{k}{this}\PYG{p}{.}\PYG{n+nx}{name} \PYG{o}{=} \PYG{n+nx}{name}\PYG{p}{;}
    \PYG{p}{\PYGZcb{}}\PYG{p}{,}
\PYG{p}{\PYGZcb{}}\PYG{p}{)}\PYG{p}{;}
\end{sphinxVerbatim}

Finally, if a widget does not have a parent (e.g. because it’s the root
widget of the application), \sphinxcode{\sphinxupquote{null}} can be provided as parent:

\fvset{hllines={, ,}}%
\begin{sphinxVerbatim}[commandchars=\\\{\}]
\PYG{k}{new} \PYG{n+nx}{local}\PYG{p}{.}\PYG{n+nx}{GreetingsWidget}\PYG{p}{(}\PYG{k+kc}{null}\PYG{p}{)}\PYG{p}{;}
\end{sphinxVerbatim}


\subsubsection{Destroying Widgets}
\label{\detokenize{howtos/web:destroying-widgets}}
If you can display content to your users, you should also be able to erase
it. This is done via the \sphinxcode{\sphinxupquote{destroy()}} method:

\fvset{hllines={, ,}}%
\begin{sphinxVerbatim}[commandchars=\\\{\}]
\PYG{n+nx}{greeting}\PYG{p}{.}\PYG{n+nx}{destroy}\PYG{p}{(}\PYG{p}{)}\PYG{p}{;}
\end{sphinxVerbatim}

When a widget is destroyed it will first call
\sphinxcode{\sphinxupquote{destroy()}} on all its children. Then it erases itself
from the DOM. If you have set up permanent structures in
\sphinxcode{\sphinxupquote{init()}} or \sphinxcode{\sphinxupquote{start()}} which
must be explicitly cleaned up (because the garbage collector will not handle
them), you can override \sphinxcode{\sphinxupquote{destroy()}}.

\begin{sphinxadmonition}{danger}{Danger:}
when overriding \sphinxcode{\sphinxupquote{destroy()}}, \sphinxcode{\sphinxupquote{\_super()}}
\sphinxstyleemphasis{must always} be called otherwise the widget and its children are not
correctly cleaned up leaving possible memory leaks and “phantom events”,
even if no error is displayed
\end{sphinxadmonition}


\subsection{The QWeb Template Engine}
\label{\detokenize{howtos/web:the-qweb-template-engine}}
In the previous section we added content to our widgets by directly
manipulating (and adding to) their DOM:

\fvset{hllines={, ,}}%
\begin{sphinxVerbatim}[commandchars=\\\{\}]
\PYG{k}{this}\PYG{p}{.}\PYG{n+nx}{\PYGZdl{}el}\PYG{p}{.}\PYG{n+nx}{append}\PYG{p}{(}\PYG{l+s+s2}{\PYGZdq{}\PYGZlt{}div\PYGZgt{}Hello dear Odoo user!\PYGZlt{}/div\PYGZgt{}\PYGZdq{}}\PYG{p}{)}\PYG{p}{;}
\end{sphinxVerbatim}

This allows generating and displaying any type of content, but gets unwieldy
when generating significant amounts of DOM (lots of duplication, quoting
issues, …)

As many other environments, Odoo’s solution is to use a \sphinxhref{http://en.wikipedia.org/wiki/Web\_template\_system}{template engine}.
Odoo’s template engine is called {\hyperref[\detokenize{reference/qweb:reference-qweb}]{\sphinxcrossref{\DUrole{std,std-ref}{QWeb}}}}.

QWeb is an XML-based templating language, similar to \sphinxhref{http://en.wikipedia.org/wiki/Genshi\_(templating\_language)}{Genshi}, \sphinxhref{http://en.wikipedia.org/wiki/Thymeleaf}{Thymeleaf} or \sphinxhref{http://en.wikipedia.org/wiki/Facelets}{Facelets}. It has the following
characteristics:
\begin{itemize}
\item {} 
It’s implemented fully in JavaScript and rendered in the browser

\item {} 
Each template file (XML files) contains multiple templates

\item {} 
It has special support in Odoo Web’s \sphinxcode{\sphinxupquote{Widget()}}, though it
can be used outside of Odoo’s web client (and it’s possible to use
\sphinxcode{\sphinxupquote{Widget()}} without relying on QWeb)

\end{itemize}

\begin{sphinxadmonition}{note}{Note:}
The rationale behind using QWeb instead of existing javascript template
engines is the extensibility of pre-existing (third-party) templates, much
like Odoo {\hyperref[\detokenize{reference/views:reference-views}]{\sphinxcrossref{\DUrole{std,std-ref}{views}}}}.

Most javascript template engines are text-based which precludes easy
structural extensibility where an XML-based templating engine can be
generically altered using e.g. XPath or CSS and a tree-alteration DSL (or
even just XSLT). This flexibility and extensibility is a core
characteristic of Odoo, and losing it was considered unacceptable.
\end{sphinxadmonition}


\subsubsection{Using QWeb}
\label{\detokenize{howtos/web:using-qweb}}
First let’s define a simple QWeb template in the almost-empty
\sphinxcode{\sphinxupquote{oepetstore/static/src/xml/petstore.xml}} file:

\fvset{hllines={, ,}}%
\begin{sphinxVerbatim}[commandchars=\\\{\}]
\PYG{c+cp}{\PYGZlt{}?xml version=\PYGZdq{}1.0\PYGZdq{} encoding=\PYGZdq{}UTF\PYGZhy{}8\PYGZdq{}?\PYGZgt{}}
\PYG{n+nt}{\PYGZlt{}templates} \PYG{n+na}{xml:space=}\PYG{l+s}{\PYGZdq{}preserve\PYGZdq{}}\PYG{n+nt}{\PYGZgt{}}
    \PYG{n+nt}{\PYGZlt{}t} \PYG{n+na}{t\PYGZhy{}name=}\PYG{l+s}{\PYGZdq{}HomePageTemplate\PYGZdq{}}\PYG{n+nt}{\PYGZgt{}}
        \PYG{n+nt}{\PYGZlt{}div} \PYG{n+na}{style=}\PYG{l+s}{\PYGZdq{}background\PYGZhy{}color: red;\PYGZdq{}}\PYG{n+nt}{\PYGZgt{}}This is some simple HTML\PYG{n+nt}{\PYGZlt{}/div\PYGZgt{}}
    \PYG{n+nt}{\PYGZlt{}/t\PYGZgt{}}
\PYG{n+nt}{\PYGZlt{}/templates\PYGZgt{}}
\end{sphinxVerbatim}

Now we can use this template inside of the \sphinxcode{\sphinxupquote{HomePage}} widget. Using the
\sphinxcode{\sphinxupquote{QWeb}} loader variable defined at the top of the page, we can call to the
template defined in the XML file:

\fvset{hllines={, ,}}%
\begin{sphinxVerbatim}[commandchars=\\\{\}]
\PYG{n+nx}{local}\PYG{p}{.}\PYG{n+nx}{HomePage} \PYG{o}{=} \PYG{n+nx}{instance}\PYG{p}{.}\PYG{n+nx}{Widget}\PYG{p}{.}\PYG{n+nx}{extend}\PYG{p}{(}\PYG{p}{\PYGZob{}}
    \PYG{n+nx}{start}\PYG{o}{:} \PYG{k+kd}{function}\PYG{p}{(}\PYG{p}{)} \PYG{p}{\PYGZob{}}
        \PYG{k}{this}\PYG{p}{.}\PYG{n+nx}{\PYGZdl{}el}\PYG{p}{.}\PYG{n+nx}{append}\PYG{p}{(}\PYG{n+nx}{QWeb}\PYG{p}{.}\PYG{n+nx}{render}\PYG{p}{(}\PYG{l+s+s2}{\PYGZdq{}HomePageTemplate\PYGZdq{}}\PYG{p}{)}\PYG{p}{)}\PYG{p}{;}
    \PYG{p}{\PYGZcb{}}\PYG{p}{,}
\PYG{p}{\PYGZcb{}}\PYG{p}{)}\PYG{p}{;}
\end{sphinxVerbatim}

\sphinxcode{\sphinxupquote{QWeb.render()}} looks for the specified template, renders it to a string
and returns the result.

However, because \sphinxcode{\sphinxupquote{Widget()}} has special integration for QWeb
the template can be set directly on the widget via its
\sphinxcode{\sphinxupquote{template}} attribute:

\fvset{hllines={, ,}}%
\begin{sphinxVerbatim}[commandchars=\\\{\}]
\PYG{n+nx}{local}\PYG{p}{.}\PYG{n+nx}{HomePage} \PYG{o}{=} \PYG{n+nx}{instance}\PYG{p}{.}\PYG{n+nx}{Widget}\PYG{p}{.}\PYG{n+nx}{extend}\PYG{p}{(}\PYG{p}{\PYGZob{}}
    \PYG{n+nx}{template}\PYG{o}{:} \PYG{l+s+s2}{\PYGZdq{}HomePageTemplate\PYGZdq{}}\PYG{p}{,}
    \PYG{n+nx}{start}\PYG{o}{:} \PYG{k+kd}{function}\PYG{p}{(}\PYG{p}{)} \PYG{p}{\PYGZob{}}
        \PYG{p}{...}
    \PYG{p}{\PYGZcb{}}\PYG{p}{,}
\PYG{p}{\PYGZcb{}}\PYG{p}{)}\PYG{p}{;}
\end{sphinxVerbatim}

Although the result looks similar, there are two differences between these
usages:
\begin{itemize}
\item {} 
with the second version, the template is rendered right before
\sphinxcode{\sphinxupquote{start()}} is called

\item {} 
in the first version the template’s content is added to the widget’s root
element, whereas in the second version the template’s root element is
directly \sphinxstyleemphasis{set as} the widget’s root element. Which is why the “greetings”
sub-widget also gets a red background

\end{itemize}

\begin{sphinxadmonition}{warning}{Warning:}
templates should have a single non-\sphinxcode{\sphinxupquote{t}} root element, especially if
they’re set as a widget’s \sphinxcode{\sphinxupquote{template}}. If there are
multiple “root elements”, results are undefined (usually only the first
root element will be used and the others will be ignored)
\end{sphinxadmonition}


\paragraph{QWeb Context}
\label{\detokenize{howtos/web:qweb-context}}
QWeb templates can be given data and can contain basic display logic.

For explicit calls to \sphinxcode{\sphinxupquote{QWeb.render()}}, the template data is passed as
second parameter:

\fvset{hllines={, ,}}%
\begin{sphinxVerbatim}[commandchars=\\\{\}]
\PYG{n+nx}{QWeb}\PYG{p}{.}\PYG{n+nx}{render}\PYG{p}{(}\PYG{l+s+s2}{\PYGZdq{}HomePageTemplate\PYGZdq{}}\PYG{p}{,} \PYG{p}{\PYGZob{}}\PYG{n+nx}{name}\PYG{o}{:} \PYG{l+s+s2}{\PYGZdq{}Klaus\PYGZdq{}}\PYG{p}{\PYGZcb{}}\PYG{p}{)}\PYG{p}{;}
\end{sphinxVerbatim}

with the template modified to:

\fvset{hllines={, ,}}%
\begin{sphinxVerbatim}[commandchars=\\\{\}]
\PYG{n+nt}{\PYGZlt{}t} \PYG{n+na}{t\PYGZhy{}name=}\PYG{l+s}{\PYGZdq{}HomePageTemplate\PYGZdq{}}\PYG{n+nt}{\PYGZgt{}}
    \PYG{n+nt}{\PYGZlt{}div}\PYG{n+nt}{\PYGZgt{}}Hello \PYG{n+nt}{\PYGZlt{}t} \PYG{n+na}{t\PYGZhy{}esc=}\PYG{l+s}{\PYGZdq{}name\PYGZdq{}}\PYG{n+nt}{/\PYGZgt{}}\PYG{n+nt}{\PYGZlt{}/div\PYGZgt{}}
\PYG{n+nt}{\PYGZlt{}/t\PYGZgt{}}
\end{sphinxVerbatim}

will result in:

\fvset{hllines={, ,}}%
\begin{sphinxVerbatim}[commandchars=\\\{\}]
\PYG{p}{\PYGZlt{}}\PYG{n+nt}{div}\PYG{p}{\PYGZgt{}}Hello Klaus\PYG{p}{\PYGZlt{}}\PYG{p}{/}\PYG{n+nt}{div}\PYG{p}{\PYGZgt{}}
\end{sphinxVerbatim}

When using \sphinxcode{\sphinxupquote{Widget()}}’s integration it is not possible to
provide additional data to the template. The template will be given a single
\sphinxcode{\sphinxupquote{widget}} context variable, referencing the widget being rendered right
before \sphinxcode{\sphinxupquote{start()}} is called (the widget’s state will
essentially be that set up by \sphinxcode{\sphinxupquote{init()}}):

\fvset{hllines={, ,}}%
\begin{sphinxVerbatim}[commandchars=\\\{\}]
\PYG{n+nt}{\PYGZlt{}t} \PYG{n+na}{t\PYGZhy{}name=}\PYG{l+s}{\PYGZdq{}HomePageTemplate\PYGZdq{}}\PYG{n+nt}{\PYGZgt{}}
    \PYG{n+nt}{\PYGZlt{}div}\PYG{n+nt}{\PYGZgt{}}Hello \PYG{n+nt}{\PYGZlt{}t} \PYG{n+na}{t\PYGZhy{}esc=}\PYG{l+s}{\PYGZdq{}widget.name\PYGZdq{}}\PYG{n+nt}{/\PYGZgt{}}\PYG{n+nt}{\PYGZlt{}/div\PYGZgt{}}
\PYG{n+nt}{\PYGZlt{}/t\PYGZgt{}}
\end{sphinxVerbatim}

\fvset{hllines={, ,}}%
\begin{sphinxVerbatim}[commandchars=\\\{\}]
\PYG{n+nx}{local}\PYG{p}{.}\PYG{n+nx}{HomePage} \PYG{o}{=} \PYG{n+nx}{instance}\PYG{p}{.}\PYG{n+nx}{Widget}\PYG{p}{.}\PYG{n+nx}{extend}\PYG{p}{(}\PYG{p}{\PYGZob{}}
    \PYG{n+nx}{template}\PYG{o}{:} \PYG{l+s+s2}{\PYGZdq{}HomePageTemplate\PYGZdq{}}\PYG{p}{,}
    \PYG{n+nx}{init}\PYG{o}{:} \PYG{k+kd}{function}\PYG{p}{(}\PYG{n+nx}{parent}\PYG{p}{)} \PYG{p}{\PYGZob{}}
        \PYG{k}{this}\PYG{p}{.}\PYG{n+nx}{\PYGZus{}super}\PYG{p}{(}\PYG{n+nx}{parent}\PYG{p}{)}\PYG{p}{;}
        \PYG{k}{this}\PYG{p}{.}\PYG{n+nx}{name} \PYG{o}{=} \PYG{l+s+s2}{\PYGZdq{}Mordecai\PYGZdq{}}\PYG{p}{;}
    \PYG{p}{\PYGZcb{}}\PYG{p}{,}
    \PYG{n+nx}{start}\PYG{o}{:} \PYG{k+kd}{function}\PYG{p}{(}\PYG{p}{)} \PYG{p}{\PYGZob{}}
    \PYG{p}{\PYGZcb{}}\PYG{p}{,}
\PYG{p}{\PYGZcb{}}\PYG{p}{)}\PYG{p}{;}
\end{sphinxVerbatim}

Result:

\fvset{hllines={, ,}}%
\begin{sphinxVerbatim}[commandchars=\\\{\}]
\PYG{p}{\PYGZlt{}}\PYG{n+nt}{div}\PYG{p}{\PYGZgt{}}Hello Mordecai\PYG{p}{\PYGZlt{}}\PYG{p}{/}\PYG{n+nt}{div}\PYG{p}{\PYGZgt{}}
\end{sphinxVerbatim}


\paragraph{Template Declaration}
\label{\detokenize{howtos/web:template-declaration}}
We’ve seen how to \sphinxstyleemphasis{render} QWeb templates, let’s now see the syntax of
the templates themselves.

A QWeb template is composed of regular XML mixed with QWeb \sphinxstyleemphasis{directives}. A
QWeb directive is declared with XML attributes starting with \sphinxcode{\sphinxupquote{t-}}.

The most basic directive is \sphinxcode{\sphinxupquote{t-name}}, used to declare new templates in
a template file:

\fvset{hllines={, ,}}%
\begin{sphinxVerbatim}[commandchars=\\\{\}]
\PYG{n+nt}{\PYGZlt{}templates}\PYG{n+nt}{\PYGZgt{}}
    \PYG{n+nt}{\PYGZlt{}t} \PYG{n+na}{t\PYGZhy{}name=}\PYG{l+s}{\PYGZdq{}HomePageTemplate\PYGZdq{}}\PYG{n+nt}{\PYGZgt{}}
        \PYG{n+nt}{\PYGZlt{}div}\PYG{n+nt}{\PYGZgt{}}This is some simple HTML\PYG{n+nt}{\PYGZlt{}/div\PYGZgt{}}
    \PYG{n+nt}{\PYGZlt{}/t\PYGZgt{}}
\PYG{n+nt}{\PYGZlt{}/templates\PYGZgt{}}
\end{sphinxVerbatim}

\sphinxcode{\sphinxupquote{t-name}} takes the name of the template being defined, and declares that
it can be called using \sphinxcode{\sphinxupquote{QWeb.render()}}. It can only be used at the
top-level of a template file.


\paragraph{Escaping}
\label{\detokenize{howtos/web:escaping}}
The \sphinxcode{\sphinxupquote{t-esc}} directive can be used to output text:

\fvset{hllines={, ,}}%
\begin{sphinxVerbatim}[commandchars=\\\{\}]
\PYG{n+nt}{\PYGZlt{}div}\PYG{n+nt}{\PYGZgt{}}Hello \PYG{n+nt}{\PYGZlt{}t} \PYG{n+na}{t\PYGZhy{}esc=}\PYG{l+s}{\PYGZdq{}name\PYGZdq{}}\PYG{n+nt}{/\PYGZgt{}}\PYG{n+nt}{\PYGZlt{}/div\PYGZgt{}}
\end{sphinxVerbatim}

It takes a Javascript expression which is evaluated, the result of the
expression is then HTML-escaped and inserted in the document. Since it’s an
expression it’s possible to provide just a variable name as above, or a more
complex expression like a computation:

\fvset{hllines={, ,}}%
\begin{sphinxVerbatim}[commandchars=\\\{\}]
\PYG{n+nt}{\PYGZlt{}div}\PYG{n+nt}{\PYGZgt{}}\PYG{n+nt}{\PYGZlt{}t} \PYG{n+na}{t\PYGZhy{}esc=}\PYG{l+s}{\PYGZdq{}3+5\PYGZdq{}}\PYG{n+nt}{/\PYGZgt{}}\PYG{n+nt}{\PYGZlt{}/div\PYGZgt{}}
\end{sphinxVerbatim}

or method calls:

\fvset{hllines={, ,}}%
\begin{sphinxVerbatim}[commandchars=\\\{\}]
\PYG{n+nt}{\PYGZlt{}div}\PYG{n+nt}{\PYGZgt{}}\PYG{n+nt}{\PYGZlt{}t} \PYG{n+na}{t\PYGZhy{}esc=}\PYG{l+s}{\PYGZdq{}name.toUpperCase()\PYGZdq{}}\PYG{n+nt}{/\PYGZgt{}}\PYG{n+nt}{\PYGZlt{}/div\PYGZgt{}}
\end{sphinxVerbatim}


\paragraph{Outputting HTML}
\label{\detokenize{howtos/web:outputting-html}}
To inject HTML in the page being rendered, use \sphinxcode{\sphinxupquote{t-raw}}. Like \sphinxcode{\sphinxupquote{t-esc}} it
takes an arbitrary Javascript expression as parameter, but it does not
perform an HTML-escape step.

\fvset{hllines={, ,}}%
\begin{sphinxVerbatim}[commandchars=\\\{\}]
\PYG{n+nt}{\PYGZlt{}div}\PYG{n+nt}{\PYGZgt{}}\PYG{n+nt}{\PYGZlt{}t} \PYG{n+na}{t\PYGZhy{}raw=}\PYG{l+s}{\PYGZdq{}name.link(user\PYGZus{}account)\PYGZdq{}}\PYG{n+nt}{/\PYGZgt{}}\PYG{n+nt}{\PYGZlt{}/div\PYGZgt{}}
\end{sphinxVerbatim}

\begin{sphinxadmonition}{danger}{Danger:}
\sphinxcode{\sphinxupquote{t-raw}} \sphinxstyleemphasis{must not} be used on any data which may contain non-escaped
user-provided content as this leads to \sphinxhref{http://en.wikipedia.org/wiki/Cross-site\_scripting}{cross-site scripting}
vulnerabilities
\end{sphinxadmonition}


\paragraph{Conditionals}
\label{\detokenize{howtos/web:conditionals}}
QWeb can have conditional blocks using \sphinxcode{\sphinxupquote{t-if}}. The directive takes an
arbitrary expression, if the expression is falsy (\sphinxcode{\sphinxupquote{false}}, \sphinxcode{\sphinxupquote{null}}, \sphinxcode{\sphinxupquote{0}}
or an empty string) the whole block is suppressed, otherwise it is displayed.

\fvset{hllines={, ,}}%
\begin{sphinxVerbatim}[commandchars=\\\{\}]
\PYG{n+nt}{\PYGZlt{}div}\PYG{n+nt}{\PYGZgt{}}
    \PYG{n+nt}{\PYGZlt{}t} \PYG{n+na}{t\PYGZhy{}if=}\PYG{l+s}{\PYGZdq{}true == true\PYGZdq{}}\PYG{n+nt}{\PYGZgt{}}
        true is true
    \PYG{n+nt}{\PYGZlt{}/t\PYGZgt{}}
    \PYG{n+nt}{\PYGZlt{}t} \PYG{n+na}{t\PYGZhy{}if=}\PYG{l+s}{\PYGZdq{}true == false\PYGZdq{}}\PYG{n+nt}{\PYGZgt{}}
        true is not true
    \PYG{n+nt}{\PYGZlt{}/t\PYGZgt{}}
\PYG{n+nt}{\PYGZlt{}/div\PYGZgt{}}
\end{sphinxVerbatim}

\begin{sphinxadmonition}{note}{Note:}
QWeb doesn’t have an “else” structure, use a second \sphinxcode{\sphinxupquote{t-if}} with the
original condition inverted. You may want to store the condition in a
local variable if it’s a complex or expensive expression.
\end{sphinxadmonition}


\paragraph{Iteration}
\label{\detokenize{howtos/web:iteration}}
To iterate on a list, use \sphinxcode{\sphinxupquote{t-foreach}} and \sphinxcode{\sphinxupquote{t-as}}. \sphinxcode{\sphinxupquote{t-foreach}} takes an
expression returning a list to iterate on \sphinxcode{\sphinxupquote{t-as}} takes a variable name to
bind to each item during iteration.

\fvset{hllines={, ,}}%
\begin{sphinxVerbatim}[commandchars=\\\{\}]
\PYG{n+nt}{\PYGZlt{}div}\PYG{n+nt}{\PYGZgt{}}
    \PYG{n+nt}{\PYGZlt{}t} \PYG{n+na}{t\PYGZhy{}foreach=}\PYG{l+s}{\PYGZdq{}names\PYGZdq{}} \PYG{n+na}{t\PYGZhy{}as=}\PYG{l+s}{\PYGZdq{}name\PYGZdq{}}\PYG{n+nt}{\PYGZgt{}}
        \PYG{n+nt}{\PYGZlt{}div}\PYG{n+nt}{\PYGZgt{}}
            Hello \PYG{n+nt}{\PYGZlt{}t} \PYG{n+na}{t\PYGZhy{}esc=}\PYG{l+s}{\PYGZdq{}name\PYGZdq{}}\PYG{n+nt}{/\PYGZgt{}}
        \PYG{n+nt}{\PYGZlt{}/div\PYGZgt{}}
    \PYG{n+nt}{\PYGZlt{}/t\PYGZgt{}}
\PYG{n+nt}{\PYGZlt{}/div\PYGZgt{}}
\end{sphinxVerbatim}

\begin{sphinxadmonition}{note}{Note:}
\sphinxcode{\sphinxupquote{t-foreach}} can also be used with numbers and objects
(dictionaries)
\end{sphinxadmonition}


\paragraph{Defining attributes}
\label{\detokenize{howtos/web:defining-attributes}}
QWeb provides two related directives to define computed attributes:
\sphinxcode{\sphinxupquote{t-att-\sphinxstyleemphasis{name}}} and \sphinxcode{\sphinxupquote{t-attf-\sphinxstyleemphasis{name}}}. In either case, \sphinxstyleemphasis{name} is the
name of the attribute to create (e.g. \sphinxcode{\sphinxupquote{t-att-id}} defines the attribute
\sphinxcode{\sphinxupquote{id}} after rendering).

\sphinxcode{\sphinxupquote{t-att-}} takes a javascript expression whose result is set as the
attribute’s value, it is most useful if all of the attribute’s value is
computed:

\fvset{hllines={, ,}}%
\begin{sphinxVerbatim}[commandchars=\\\{\}]
\PYG{n+nt}{\PYGZlt{}div}\PYG{n+nt}{\PYGZgt{}}
    Input your name:
    \PYG{n+nt}{\PYGZlt{}input} \PYG{n+na}{type=}\PYG{l+s}{\PYGZdq{}text\PYGZdq{}} \PYG{n+na}{t\PYGZhy{}att\PYGZhy{}value=}\PYG{l+s}{\PYGZdq{}defaultName\PYGZdq{}}\PYG{n+nt}{/\PYGZgt{}}
\PYG{n+nt}{\PYGZlt{}/div\PYGZgt{}}
\end{sphinxVerbatim}

\sphinxcode{\sphinxupquote{t-attf-}} takes a \sphinxstyleemphasis{format string}. A format string is literal text with
interpolation blocks inside, an interpolation block is a javascript
expression between \sphinxcode{\sphinxupquote{\{\{}} and \sphinxcode{\sphinxupquote{\}\}}}, which will be replaced by the result
of the expression. It is most useful for attributes which are partially
literal and partially computed such as a class:

\fvset{hllines={, ,}}%
\begin{sphinxVerbatim}[commandchars=\\\{\}]
\PYG{n+nt}{\PYGZlt{}div} \PYG{n+na}{t\PYGZhy{}attf\PYGZhy{}class=}\PYG{l+s}{\PYGZdq{}container \PYGZob{}\PYGZob{} left ? \PYGZsq{}text\PYGZhy{}left\PYGZsq{} : \PYGZsq{}\PYGZsq{} \PYGZcb{}\PYGZcb{} \PYGZob{}\PYGZob{} extra\PYGZus{}class \PYGZcb{}\PYGZcb{}\PYGZdq{}}\PYG{n+nt}{\PYGZgt{}}
    insert content here
\PYG{n+nt}{\PYGZlt{}/div\PYGZgt{}}
\end{sphinxVerbatim}


\paragraph{Calling other templates}
\label{\detokenize{howtos/web:calling-other-templates}}
Templates can be split into sub-templates (for simplicity, maintainability,
reusability or to avoid excessive markup nesting).

This is done using the \sphinxcode{\sphinxupquote{t-call}} directive, which takes the name of the
template to render:

\fvset{hllines={, ,}}%
\begin{sphinxVerbatim}[commandchars=\\\{\}]
\PYG{n+nt}{\PYGZlt{}t} \PYG{n+na}{t\PYGZhy{}name=}\PYG{l+s}{\PYGZdq{}A\PYGZdq{}}\PYG{n+nt}{\PYGZgt{}}
    \PYG{n+nt}{\PYGZlt{}div} \PYG{n+na}{class=}\PYG{l+s}{\PYGZdq{}i\PYGZhy{}am\PYGZhy{}a\PYGZdq{}}\PYG{n+nt}{\PYGZgt{}}
        \PYG{n+nt}{\PYGZlt{}t} \PYG{n+na}{t\PYGZhy{}call=}\PYG{l+s}{\PYGZdq{}B\PYGZdq{}}\PYG{n+nt}{/\PYGZgt{}}
    \PYG{n+nt}{\PYGZlt{}/div\PYGZgt{}}
\PYG{n+nt}{\PYGZlt{}/t\PYGZgt{}}
\PYG{n+nt}{\PYGZlt{}t} \PYG{n+na}{t\PYGZhy{}name=}\PYG{l+s}{\PYGZdq{}B\PYGZdq{}}\PYG{n+nt}{\PYGZgt{}}
    \PYG{n+nt}{\PYGZlt{}div} \PYG{n+na}{class=}\PYG{l+s}{\PYGZdq{}i\PYGZhy{}am\PYGZhy{}b\PYGZdq{}}\PYG{n+nt}{/\PYGZgt{}}
\PYG{n+nt}{\PYGZlt{}/t\PYGZgt{}}
\end{sphinxVerbatim}

rendering the \sphinxcode{\sphinxupquote{A}} template will result in:

\fvset{hllines={, ,}}%
\begin{sphinxVerbatim}[commandchars=\\\{\}]
\PYG{n+nt}{\PYGZlt{}div} \PYG{n+na}{class=}\PYG{l+s}{\PYGZdq{}i\PYGZhy{}am\PYGZhy{}a\PYGZdq{}}\PYG{n+nt}{\PYGZgt{}}
    \PYG{n+nt}{\PYGZlt{}div} \PYG{n+na}{class=}\PYG{l+s}{\PYGZdq{}i\PYGZhy{}am\PYGZhy{}b\PYGZdq{}}\PYG{n+nt}{/\PYGZgt{}}
\PYG{n+nt}{\PYGZlt{}/div\PYGZgt{}}
\end{sphinxVerbatim}

Sub-templates inherit the rendering context of their caller.


\paragraph{To Learn More About QWeb}
\label{\detokenize{howtos/web:to-learn-more-about-qweb}}
For a QWeb reference, see {\hyperref[\detokenize{reference/qweb:reference-qweb}]{\sphinxcrossref{\DUrole{std,std-ref}{QWeb}}}}.


\paragraph{Exercise}
\label{\detokenize{howtos/web:exercise}}
\begin{sphinxadmonition}{note}
Usage of QWeb in Widgets

Create a widget whose constructor takes two parameters aside from
\sphinxcode{\sphinxupquote{parent}}: \sphinxcode{\sphinxupquote{product\_names}} and \sphinxcode{\sphinxupquote{color}}.
\begin{itemize}
\item {} 
\sphinxcode{\sphinxupquote{product\_names}} should an array of strings, each one the name of a
product

\item {} 
\sphinxcode{\sphinxupquote{color}} is a string containing a color in CSS color format (ie:
\sphinxcode{\sphinxupquote{\#000000}} for black).

\end{itemize}

The widget should display the given product names one under the other,
each one in a separate box with a background color with the value of
\sphinxcode{\sphinxupquote{color}} and a border. You should use QWeb to render the HTML. Any
necessary CSS should be in \sphinxcode{\sphinxupquote{oepetstore/static/src/css/petstore.css}}.

Use the widget in \sphinxcode{\sphinxupquote{HomePage}} with half a dozen products.

\fvset{hllines={, ,}}%
\begin{sphinxVerbatim}[commandchars=\\\{\}]
\PYG{n+nx}{odoo}\PYG{p}{.}\PYG{n+nx}{oepetstore} \PYG{o}{=} \PYG{k+kd}{function}\PYG{p}{(}\PYG{n+nx}{instance}\PYG{p}{,} \PYG{n+nx}{local}\PYG{p}{)} \PYG{p}{\PYGZob{}}
    \PYG{k+kd}{var} \PYG{n+nx}{\PYGZus{}t} \PYG{o}{=} \PYG{n+nx}{instance}\PYG{p}{.}\PYG{n+nx}{web}\PYG{p}{.}\PYG{n+nx}{\PYGZus{}t}\PYG{p}{,}
        \PYG{n+nx}{\PYGZus{}lt} \PYG{o}{=} \PYG{n+nx}{instance}\PYG{p}{.}\PYG{n+nx}{web}\PYG{p}{.}\PYG{n+nx}{\PYGZus{}lt}\PYG{p}{;}
    \PYG{k+kd}{var} \PYG{n+nx}{QWeb} \PYG{o}{=} \PYG{n+nx}{instance}\PYG{p}{.}\PYG{n+nx}{web}\PYG{p}{.}\PYG{n+nx}{qweb}\PYG{p}{;}

    \PYG{n+nx}{local}\PYG{p}{.}\PYG{n+nx}{HomePage} \PYG{o}{=} \PYG{n+nx}{instance}\PYG{p}{.}\PYG{n+nx}{Widget}\PYG{p}{.}\PYG{n+nx}{extend}\PYG{p}{(}\PYG{p}{\PYGZob{}}
        \PYG{n+nx}{start}\PYG{o}{:} \PYG{k+kd}{function}\PYG{p}{(}\PYG{p}{)} \PYG{p}{\PYGZob{}}
            \PYG{k+kd}{var} \PYG{n+nx}{products} \PYG{o}{=} \PYG{k}{new} \PYG{n+nx}{local}\PYG{p}{.}\PYG{n+nx}{ProductsWidget}\PYG{p}{(}
                \PYG{k}{this}\PYG{p}{,} \PYG{p}{[}\PYG{l+s+s2}{\PYGZdq{}cpu\PYGZdq{}}\PYG{p}{,} \PYG{l+s+s2}{\PYGZdq{}mouse\PYGZdq{}}\PYG{p}{,} \PYG{l+s+s2}{\PYGZdq{}keyboard\PYGZdq{}}\PYG{p}{,} \PYG{l+s+s2}{\PYGZdq{}graphic card\PYGZdq{}}\PYG{p}{,} \PYG{l+s+s2}{\PYGZdq{}screen\PYGZdq{}}\PYG{p}{]}\PYG{p}{,} \PYG{l+s+s2}{\PYGZdq{}\PYGZsh{}00FF00\PYGZdq{}}\PYG{p}{)}\PYG{p}{;}
            \PYG{n+nx}{products}\PYG{p}{.}\PYG{n+nx}{appendTo}\PYG{p}{(}\PYG{k}{this}\PYG{p}{.}\PYG{n+nx}{\PYGZdl{}el}\PYG{p}{)}\PYG{p}{;}
        \PYG{p}{\PYGZcb{}}\PYG{p}{,}
    \PYG{p}{\PYGZcb{}}\PYG{p}{)}\PYG{p}{;}

    \PYG{n+nx}{local}\PYG{p}{.}\PYG{n+nx}{ProductsWidget} \PYG{o}{=} \PYG{n+nx}{instance}\PYG{p}{.}\PYG{n+nx}{Widget}\PYG{p}{.}\PYG{n+nx}{extend}\PYG{p}{(}\PYG{p}{\PYGZob{}}
        \PYG{n+nx}{template}\PYG{o}{:} \PYG{l+s+s2}{\PYGZdq{}ProductsWidget\PYGZdq{}}\PYG{p}{,}
        \PYG{n+nx}{init}\PYG{o}{:} \PYG{k+kd}{function}\PYG{p}{(}\PYG{n+nx}{parent}\PYG{p}{,} \PYG{n+nx}{products}\PYG{p}{,} \PYG{n+nx}{color}\PYG{p}{)} \PYG{p}{\PYGZob{}}
            \PYG{k}{this}\PYG{p}{.}\PYG{n+nx}{\PYGZus{}super}\PYG{p}{(}\PYG{n+nx}{parent}\PYG{p}{)}\PYG{p}{;}
            \PYG{k}{this}\PYG{p}{.}\PYG{n+nx}{products} \PYG{o}{=} \PYG{n+nx}{products}\PYG{p}{;}
            \PYG{k}{this}\PYG{p}{.}\PYG{n+nx}{color} \PYG{o}{=} \PYG{n+nx}{color}\PYG{p}{;}
        \PYG{p}{\PYGZcb{}}\PYG{p}{,}
    \PYG{p}{\PYGZcb{}}\PYG{p}{)}\PYG{p}{;}

    \PYG{n+nx}{instance}\PYG{p}{.}\PYG{n+nx}{web}\PYG{p}{.}\PYG{n+nx}{client\PYGZus{}actions}\PYG{p}{.}\PYG{n+nx}{add}\PYG{p}{(}
        \PYG{l+s+s1}{\PYGZsq{}petstore.homepage\PYGZsq{}}\PYG{p}{,} \PYG{l+s+s1}{\PYGZsq{}instance.oepetstore.HomePage\PYGZsq{}}\PYG{p}{)}\PYG{p}{;}
\PYG{p}{\PYGZcb{}}
\end{sphinxVerbatim}

\fvset{hllines={, ,}}%
\begin{sphinxVerbatim}[commandchars=\\\{\}]
\PYG{c+cp}{\PYGZlt{}?xml version=\PYGZdq{}1.0\PYGZdq{} encoding=\PYGZdq{}UTF\PYGZhy{}8\PYGZdq{}?\PYGZgt{}}
\PYG{n+nt}{\PYGZlt{}templates} \PYG{n+na}{xml:space=}\PYG{l+s}{\PYGZdq{}preserve\PYGZdq{}}\PYG{n+nt}{\PYGZgt{}}
    \PYG{n+nt}{\PYGZlt{}t} \PYG{n+na}{t\PYGZhy{}name=}\PYG{l+s}{\PYGZdq{}ProductsWidget\PYGZdq{}}\PYG{n+nt}{\PYGZgt{}}
        \PYG{n+nt}{\PYGZlt{}div}\PYG{n+nt}{\PYGZgt{}}
            \PYG{n+nt}{\PYGZlt{}t} \PYG{n+na}{t\PYGZhy{}foreach=}\PYG{l+s}{\PYGZdq{}widget.products\PYGZdq{}} \PYG{n+na}{t\PYGZhy{}as=}\PYG{l+s}{\PYGZdq{}product\PYGZdq{}}\PYG{n+nt}{\PYGZgt{}}
                \PYG{n+nt}{\PYGZlt{}span} \PYG{n+na}{class=}\PYG{l+s}{\PYGZdq{}oe\PYGZus{}products\PYGZus{}item\PYGZdq{}}
                      \PYG{n+na}{t\PYGZhy{}attf\PYGZhy{}style=}\PYG{l+s}{\PYGZdq{}background\PYGZhy{}color: \PYGZob{}\PYGZob{} widget.color \PYGZcb{}\PYGZcb{};\PYGZdq{}}\PYG{n+nt}{\PYGZgt{}}
                    \PYG{n+nt}{\PYGZlt{}t} \PYG{n+na}{t\PYGZhy{}esc=}\PYG{l+s}{\PYGZdq{}product\PYGZdq{}}\PYG{n+nt}{/\PYGZgt{}}
                \PYG{n+nt}{\PYGZlt{}/span\PYGZgt{}}
                \PYG{n+nt}{\PYGZlt{}br}\PYG{n+nt}{/\PYGZgt{}}
            \PYG{n+nt}{\PYGZlt{}/t\PYGZgt{}}
        \PYG{n+nt}{\PYGZlt{}/div\PYGZgt{}}
    \PYG{n+nt}{\PYGZlt{}/t\PYGZgt{}}
\PYG{n+nt}{\PYGZlt{}/templates\PYGZgt{}}
\end{sphinxVerbatim}

\fvset{hllines={, ,}}%
\begin{sphinxVerbatim}[commandchars=\\\{\}]
\PYG{n+nc}{.oe\PYGZus{}products\PYGZus{}item} \PYG{p}{\PYGZob{}}
    \PYG{n+nb}{display}\PYG{o}{:} \PYG{n+nb}{inline}\PYG{o}{\PYGZhy{}}\PYG{n+nb}{block}\PYG{p}{;}
    \PYG{n+nb}{padding}\PYG{o}{:} \PYG{l+m}{3px}\PYG{p}{;}
    \PYG{n+nb}{margin}\PYG{o}{:} \PYG{l+m}{5px}\PYG{p}{;}
    \PYG{n+nb}{border}\PYG{o}{:} \PYG{l+m}{1px} \PYG{n+nb}{solid} \PYG{n+nb}{black}\PYG{p}{;}
    \PYG{n+nb}{border}\PYG{o}{\PYGZhy{}}\PYG{n}{radius}\PYG{o}{:} \PYG{l+m}{3px}\PYG{p}{;}
\PYG{p}{\PYGZcb{}}
\end{sphinxVerbatim}

\noindent{\hspace*{\fill}\sphinxincludegraphics[width=0.700\linewidth]{{qweb}.png}\hspace*{\fill}}
\end{sphinxadmonition}


\subsection{Widget Helpers}
\label{\detokenize{howtos/web:widget-helpers}}

\subsubsection{\sphinxstyleliteralintitle{\sphinxupquote{Widget}}’s jQuery Selector}
\label{\detokenize{howtos/web:widget-s-jquery-selector}}
Selecting DOM elements within a widget can be performed by calling the
\sphinxcode{\sphinxupquote{find()}} method on the widget’s DOM root:

\fvset{hllines={, ,}}%
\begin{sphinxVerbatim}[commandchars=\\\{\}]
\PYG{k}{this}\PYG{p}{.}\PYG{n+nx}{\PYGZdl{}el}\PYG{p}{.}\PYG{n+nx}{find}\PYG{p}{(}\PYG{l+s+s2}{\PYGZdq{}input.my\PYGZus{}input\PYGZdq{}}\PYG{p}{)}\PYG{p}{...}
\end{sphinxVerbatim}

But because it’s a common operation, \sphinxcode{\sphinxupquote{Widget()}} provides an
equivalent shortcut through the \sphinxcode{\sphinxupquote{\$()}} method:

\fvset{hllines={, ,}}%
\begin{sphinxVerbatim}[commandchars=\\\{\}]
\PYG{n+nx}{local}\PYG{p}{.}\PYG{n+nx}{MyWidget} \PYG{o}{=} \PYG{n+nx}{instance}\PYG{p}{.}\PYG{n+nx}{Widget}\PYG{p}{.}\PYG{n+nx}{extend}\PYG{p}{(}\PYG{p}{\PYGZob{}}
    \PYG{n+nx}{start}\PYG{o}{:} \PYG{k+kd}{function}\PYG{p}{(}\PYG{p}{)} \PYG{p}{\PYGZob{}}
        \PYG{k}{this}\PYG{p}{.}\PYG{n+nx}{\PYGZdl{}}\PYG{p}{(}\PYG{l+s+s2}{\PYGZdq{}input.my\PYGZus{}input\PYGZdq{}}\PYG{p}{)}\PYG{p}{...}
    \PYG{p}{\PYGZcb{}}\PYG{p}{,}
\PYG{p}{\PYGZcb{}}\PYG{p}{)}\PYG{p}{;}
\end{sphinxVerbatim}

\begin{sphinxadmonition}{warning}{Warning:}
The global jQuery function \sphinxcode{\sphinxupquote{\$()}} should \sphinxstyleemphasis{never} be used unless it is
absolutely necessary: selection on a widget’s root are scoped to the
widget and local to it, but selections with \sphinxcode{\sphinxupquote{\$()}} are global to the
page/application and may match parts of other widgets and views, leading
to odd or dangerous side-effects. Since a widget should generally act
only on the DOM section it owns, there is no cause for global selection.
\end{sphinxadmonition}


\subsubsection{Easier DOM Events Binding}
\label{\detokenize{howtos/web:easier-dom-events-binding}}
We have previously bound DOM events using normal jQuery event handlers (e.g.
\sphinxcode{\sphinxupquote{.click()}} or \sphinxcode{\sphinxupquote{.change()}}) on widget elements:

\fvset{hllines={, ,}}%
\begin{sphinxVerbatim}[commandchars=\\\{\}]
\PYG{n+nx}{local}\PYG{p}{.}\PYG{n+nx}{MyWidget} \PYG{o}{=} \PYG{n+nx}{instance}\PYG{p}{.}\PYG{n+nx}{Widget}\PYG{p}{.}\PYG{n+nx}{extend}\PYG{p}{(}\PYG{p}{\PYGZob{}}
    \PYG{n+nx}{start}\PYG{o}{:} \PYG{k+kd}{function}\PYG{p}{(}\PYG{p}{)} \PYG{p}{\PYGZob{}}
        \PYG{k+kd}{var} \PYG{n+nx}{self} \PYG{o}{=} \PYG{k}{this}\PYG{p}{;}
        \PYG{k}{this}\PYG{p}{.}\PYG{n+nx}{\PYGZdl{}}\PYG{p}{(}\PYG{l+s+s2}{\PYGZdq{}.my\PYGZus{}button\PYGZdq{}}\PYG{p}{)}\PYG{p}{.}\PYG{n+nx}{click}\PYG{p}{(}\PYG{k+kd}{function}\PYG{p}{(}\PYG{p}{)} \PYG{p}{\PYGZob{}}
            \PYG{n+nx}{self}\PYG{p}{.}\PYG{n+nx}{button\PYGZus{}clicked}\PYG{p}{(}\PYG{p}{)}\PYG{p}{;}
        \PYG{p}{\PYGZcb{}}\PYG{p}{)}\PYG{p}{;}
    \PYG{p}{\PYGZcb{}}\PYG{p}{,}
    \PYG{n+nx}{button\PYGZus{}clicked}\PYG{o}{:} \PYG{k+kd}{function}\PYG{p}{(}\PYG{p}{)} \PYG{p}{\PYGZob{}}
        \PYG{p}{.}\PYG{p}{.}
    \PYG{p}{\PYGZcb{}}\PYG{p}{,}
\PYG{p}{\PYGZcb{}}\PYG{p}{)}\PYG{p}{;}
\end{sphinxVerbatim}

While this works it has a few issues:
\begin{enumerate}
\item {} 
it is rather verbose

\item {} 
it does not support replacing the widget’s root element at runtime as
the binding is only performed when \sphinxcode{\sphinxupquote{start()}} is run (during widget
initialization)

\item {} 
it requires dealing with \sphinxcode{\sphinxupquote{this}}-binding issues

\end{enumerate}

Widgets thus provide a shortcut to DOM event binding via
\sphinxcode{\sphinxupquote{events}}:

\fvset{hllines={, ,}}%
\begin{sphinxVerbatim}[commandchars=\\\{\}]
\PYG{n+nx}{local}\PYG{p}{.}\PYG{n+nx}{MyWidget} \PYG{o}{=} \PYG{n+nx}{instance}\PYG{p}{.}\PYG{n+nx}{Widget}\PYG{p}{.}\PYG{n+nx}{extend}\PYG{p}{(}\PYG{p}{\PYGZob{}}
    \PYG{n+nx}{events}\PYG{o}{:} \PYG{p}{\PYGZob{}}
        \PYG{l+s+s2}{\PYGZdq{}click .my\PYGZus{}button\PYGZdq{}}\PYG{o}{:} \PYG{l+s+s2}{\PYGZdq{}button\PYGZus{}clicked\PYGZdq{}}\PYG{p}{,}
    \PYG{p}{\PYGZcb{}}\PYG{p}{,}
    \PYG{n+nx}{button\PYGZus{}clicked}\PYG{o}{:} \PYG{k+kd}{function}\PYG{p}{(}\PYG{p}{)} \PYG{p}{\PYGZob{}}
        \PYG{p}{.}\PYG{p}{.}
    \PYG{p}{\PYGZcb{}}
\PYG{p}{\PYGZcb{}}\PYG{p}{)}\PYG{p}{;}
\end{sphinxVerbatim}

\sphinxcode{\sphinxupquote{events}} is an object (mapping) of an event to the
function or method to call when the event is triggered:
\begin{itemize}
\item {} 
the key is an event name, possibly refined with a CSS selector in which
case only if the event happens on a selected sub-element will the function
or method run: \sphinxcode{\sphinxupquote{click}} will handle all clicks within the widget, but
\sphinxcode{\sphinxupquote{click .my\_button}} will only handle clicks in elements bearing the
\sphinxcode{\sphinxupquote{my\_button}} class

\item {} 
the value is the action to perform when the event is triggered

It can be either a function:

\fvset{hllines={, ,}}%
\begin{sphinxVerbatim}[commandchars=\\\{\}]
\PYG{n+nx}{events}\PYG{o}{:} \PYG{p}{\PYGZob{}}
    \PYG{l+s+s1}{\PYGZsq{}click\PYGZsq{}}\PYG{o}{:} \PYG{k+kd}{function} \PYG{p}{(}\PYG{n+nx}{e}\PYG{p}{)} \PYG{p}{\PYGZob{}} \PYG{c+cm}{/* code here */} \PYG{p}{\PYGZcb{}}
\PYG{p}{\PYGZcb{}}
\end{sphinxVerbatim}

or the name of a method on the object (see example above).

In either case, the \sphinxcode{\sphinxupquote{this}} is the widget instance and the handler is given
a single parameter, the \sphinxhref{http://api.jquery.com/category/events/event-object/}{jQuery event object} for the event.

\end{itemize}


\subsection{Widget Events and Properties}
\label{\detokenize{howtos/web:widget-events-and-properties}}

\subsubsection{Events}
\label{\detokenize{howtos/web:events}}
Widgets provide an event system (separate from the DOM/jQuery event system
described above): a widget can fire events on itself, and other widgets (or
itself) can bind themselves and listen for these events:

\fvset{hllines={, ,}}%
\begin{sphinxVerbatim}[commandchars=\\\{\}]
\PYG{n+nx}{local}\PYG{p}{.}\PYG{n+nx}{ConfirmWidget} \PYG{o}{=} \PYG{n+nx}{instance}\PYG{p}{.}\PYG{n+nx}{Widget}\PYG{p}{.}\PYG{n+nx}{extend}\PYG{p}{(}\PYG{p}{\PYGZob{}}
    \PYG{n+nx}{events}\PYG{o}{:} \PYG{p}{\PYGZob{}}
        \PYG{l+s+s1}{\PYGZsq{}click button.ok\PYGZus{}button\PYGZsq{}}\PYG{o}{:} \PYG{k+kd}{function} \PYG{p}{(}\PYG{p}{)} \PYG{p}{\PYGZob{}}
            \PYG{k}{this}\PYG{p}{.}\PYG{n+nx}{trigger}\PYG{p}{(}\PYG{l+s+s1}{\PYGZsq{}user\PYGZus{}chose\PYGZsq{}}\PYG{p}{,} \PYG{k+kc}{true}\PYG{p}{)}\PYG{p}{;}
        \PYG{p}{\PYGZcb{}}\PYG{p}{,}
        \PYG{l+s+s1}{\PYGZsq{}click button.cancel\PYGZus{}button\PYGZsq{}}\PYG{o}{:} \PYG{k+kd}{function} \PYG{p}{(}\PYG{p}{)} \PYG{p}{\PYGZob{}}
            \PYG{k}{this}\PYG{p}{.}\PYG{n+nx}{trigger}\PYG{p}{(}\PYG{l+s+s1}{\PYGZsq{}user\PYGZus{}chose\PYGZsq{}}\PYG{p}{,} \PYG{k+kc}{false}\PYG{p}{)}\PYG{p}{;}
        \PYG{p}{\PYGZcb{}}
    \PYG{p}{\PYGZcb{}}\PYG{p}{,}
    \PYG{n+nx}{start}\PYG{o}{:} \PYG{k+kd}{function}\PYG{p}{(}\PYG{p}{)} \PYG{p}{\PYGZob{}}
        \PYG{k}{this}\PYG{p}{.}\PYG{n+nx}{\PYGZdl{}el}\PYG{p}{.}\PYG{n+nx}{append}\PYG{p}{(}\PYG{l+s+s2}{\PYGZdq{}\PYGZlt{}div\PYGZgt{}Are you sure you want to perform this action?\PYGZlt{}/div\PYGZgt{}\PYGZdq{}} \PYG{o}{+}
            \PYG{l+s+s2}{\PYGZdq{}\PYGZlt{}button class=\PYGZsq{}ok\PYGZus{}button\PYGZsq{}\PYGZgt{}Ok\PYGZlt{}/button\PYGZgt{}\PYGZdq{}} \PYG{o}{+}
            \PYG{l+s+s2}{\PYGZdq{}\PYGZlt{}button class=\PYGZsq{}cancel\PYGZus{}button\PYGZsq{}\PYGZgt{}Cancel\PYGZlt{}/button\PYGZgt{}\PYGZdq{}}\PYG{p}{)}\PYG{p}{;}
    \PYG{p}{\PYGZcb{}}\PYG{p}{,}
\PYG{p}{\PYGZcb{}}\PYG{p}{)}\PYG{p}{;}
\end{sphinxVerbatim}

This widget acts as a facade, transforming user input (through DOM events)
into a documentable internal event to which parent widgets can bind
themselves.

\sphinxcode{\sphinxupquote{trigger()}} takes the name of the event to trigger as
its first (mandatory) argument, any further arguments are treated as event
data and passed directly to listeners.

We can then set up a parent event instantiating our generic widget and
listening to the \sphinxcode{\sphinxupquote{user\_chose}} event using \sphinxcode{\sphinxupquote{on()}}:

\fvset{hllines={, ,}}%
\begin{sphinxVerbatim}[commandchars=\\\{\}]
\PYG{n+nx}{local}\PYG{p}{.}\PYG{n+nx}{HomePage} \PYG{o}{=} \PYG{n+nx}{instance}\PYG{p}{.}\PYG{n+nx}{Widget}\PYG{p}{.}\PYG{n+nx}{extend}\PYG{p}{(}\PYG{p}{\PYGZob{}}
    \PYG{n+nx}{start}\PYG{o}{:} \PYG{k+kd}{function}\PYG{p}{(}\PYG{p}{)} \PYG{p}{\PYGZob{}}
        \PYG{k+kd}{var} \PYG{n+nx}{widget} \PYG{o}{=} \PYG{k}{new} \PYG{n+nx}{local}\PYG{p}{.}\PYG{n+nx}{ConfirmWidget}\PYG{p}{(}\PYG{k}{this}\PYG{p}{)}\PYG{p}{;}
        \PYG{n+nx}{widget}\PYG{p}{.}\PYG{n+nx}{on}\PYG{p}{(}\PYG{l+s+s2}{\PYGZdq{}user\PYGZus{}chose\PYGZdq{}}\PYG{p}{,} \PYG{k}{this}\PYG{p}{,} \PYG{k}{this}\PYG{p}{.}\PYG{n+nx}{user\PYGZus{}chose}\PYG{p}{)}\PYG{p}{;}
        \PYG{n+nx}{widget}\PYG{p}{.}\PYG{n+nx}{appendTo}\PYG{p}{(}\PYG{k}{this}\PYG{p}{.}\PYG{n+nx}{\PYGZdl{}el}\PYG{p}{)}\PYG{p}{;}
    \PYG{p}{\PYGZcb{}}\PYG{p}{,}
    \PYG{n+nx}{user\PYGZus{}chose}\PYG{o}{:} \PYG{k+kd}{function}\PYG{p}{(}\PYG{n+nx}{confirm}\PYG{p}{)} \PYG{p}{\PYGZob{}}
        \PYG{k}{if} \PYG{p}{(}\PYG{n+nx}{confirm}\PYG{p}{)} \PYG{p}{\PYGZob{}}
            \PYG{n+nx}{console}\PYG{p}{.}\PYG{n+nx}{log}\PYG{p}{(}\PYG{l+s+s2}{\PYGZdq{}The user agreed to continue\PYGZdq{}}\PYG{p}{)}\PYG{p}{;}
        \PYG{p}{\PYGZcb{}} \PYG{k}{else} \PYG{p}{\PYGZob{}}
            \PYG{n+nx}{console}\PYG{p}{.}\PYG{n+nx}{log}\PYG{p}{(}\PYG{l+s+s2}{\PYGZdq{}The user refused to continue\PYGZdq{}}\PYG{p}{)}\PYG{p}{;}
        \PYG{p}{\PYGZcb{}}
    \PYG{p}{\PYGZcb{}}\PYG{p}{,}
\PYG{p}{\PYGZcb{}}\PYG{p}{)}\PYG{p}{;}
\end{sphinxVerbatim}

\sphinxcode{\sphinxupquote{on()}} binds a function to be called when the
event identified by \sphinxcode{\sphinxupquote{event\_name}} is. The \sphinxcode{\sphinxupquote{func}} argument is the
function to call and \sphinxcode{\sphinxupquote{object}} is the object to which that function is
related if it is a method. The bound function will be called with the
additional arguments of \sphinxcode{\sphinxupquote{trigger()}} if it has
any. Example:

\fvset{hllines={, ,}}%
\begin{sphinxVerbatim}[commandchars=\\\{\}]
\PYG{n+nx}{start}\PYG{o}{:} \PYG{k+kd}{function}\PYG{p}{(}\PYG{p}{)} \PYG{p}{\PYGZob{}}
    \PYG{k+kd}{var} \PYG{n+nx}{widget} \PYG{o}{=} \PYG{p}{...}
    \PYG{n+nx}{widget}\PYG{p}{.}\PYG{n+nx}{on}\PYG{p}{(}\PYG{l+s+s2}{\PYGZdq{}my\PYGZus{}event\PYGZdq{}}\PYG{p}{,} \PYG{k}{this}\PYG{p}{,} \PYG{k}{this}\PYG{p}{.}\PYG{n+nx}{my\PYGZus{}event\PYGZus{}triggered}\PYG{p}{)}\PYG{p}{;}
    \PYG{n+nx}{widget}\PYG{p}{.}\PYG{n+nx}{trigger}\PYG{p}{(}\PYG{l+s+s2}{\PYGZdq{}my\PYGZus{}event\PYGZdq{}}\PYG{p}{,} \PYG{l+m+mi}{1}\PYG{p}{,} \PYG{l+m+mi}{2}\PYG{p}{,} \PYG{l+m+mi}{3}\PYG{p}{)}\PYG{p}{;}
\PYG{p}{\PYGZcb{}}\PYG{p}{,}
\PYG{n+nx}{my\PYGZus{}event\PYGZus{}triggered}\PYG{o}{:} \PYG{k+kd}{function}\PYG{p}{(}\PYG{n+nx}{a}\PYG{p}{,} \PYG{n+nx}{b}\PYG{p}{,} \PYG{n+nx}{c}\PYG{p}{)} \PYG{p}{\PYGZob{}}
    \PYG{n+nx}{console}\PYG{p}{.}\PYG{n+nx}{log}\PYG{p}{(}\PYG{n+nx}{a}\PYG{p}{,} \PYG{n+nx}{b}\PYG{p}{,} \PYG{n+nx}{c}\PYG{p}{)}\PYG{p}{;}
    \PYG{c+c1}{// will print \PYGZdq{}1 2 3\PYGZdq{}}
\PYG{p}{\PYGZcb{}}
\end{sphinxVerbatim}

\begin{sphinxadmonition}{note}{Note:}
Triggering events on an other widget is generally a bad idea. The main
exception to that rule is \sphinxcode{\sphinxupquote{odoo.web.bus}} which exists specifically
to broadcasts evens in which any widget could be interested throughout
the Odoo web application.
\end{sphinxadmonition}


\subsubsection{Properties}
\label{\detokenize{howtos/web:properties}}
Properties are very similar to normal object attributes in that they allow
storing data on a widget instance, however they have the additional feature
that they trigger events when set:

\fvset{hllines={, ,}}%
\begin{sphinxVerbatim}[commandchars=\\\{\}]
\PYG{n+nx}{start}\PYG{o}{:} \PYG{k+kd}{function}\PYG{p}{(}\PYG{p}{)} \PYG{p}{\PYGZob{}}
    \PYG{k}{this}\PYG{p}{.}\PYG{n+nx}{widget} \PYG{o}{=} \PYG{p}{...}
    \PYG{k}{this}\PYG{p}{.}\PYG{n+nx}{widget}\PYG{p}{.}\PYG{n+nx}{on}\PYG{p}{(}\PYG{l+s+s2}{\PYGZdq{}change:name\PYGZdq{}}\PYG{p}{,} \PYG{k}{this}\PYG{p}{,} \PYG{k}{this}\PYG{p}{.}\PYG{n+nx}{name\PYGZus{}changed}\PYG{p}{)}\PYG{p}{;}
    \PYG{k}{this}\PYG{p}{.}\PYG{n+nx}{widget}\PYG{p}{.}\PYG{n+nx}{set}\PYG{p}{(}\PYG{l+s+s2}{\PYGZdq{}name\PYGZdq{}}\PYG{p}{,} \PYG{l+s+s2}{\PYGZdq{}Nicolas\PYGZdq{}}\PYG{p}{)}\PYG{p}{;}
\PYG{p}{\PYGZcb{}}\PYG{p}{,}
\PYG{n+nx}{name\PYGZus{}changed}\PYG{o}{:} \PYG{k+kd}{function}\PYG{p}{(}\PYG{p}{)} \PYG{p}{\PYGZob{}}
    \PYG{n+nx}{console}\PYG{p}{.}\PYG{n+nx}{log}\PYG{p}{(}\PYG{l+s+s2}{\PYGZdq{}The new value of the property \PYGZsq{}name\PYGZsq{} is\PYGZdq{}}\PYG{p}{,} \PYG{k}{this}\PYG{p}{.}\PYG{n+nx}{widget}\PYG{p}{.}\PYG{n+nx}{get}\PYG{p}{(}\PYG{l+s+s2}{\PYGZdq{}name\PYGZdq{}}\PYG{p}{)}\PYG{p}{)}\PYG{p}{;}
\PYG{p}{\PYGZcb{}}
\end{sphinxVerbatim}
\begin{itemize}
\item {} 
\sphinxcode{\sphinxupquote{set()}} sets the value of a property and triggers
\sphinxcode{\sphinxupquote{change:\sphinxstyleemphasis{propname}}} (where \sphinxstyleemphasis{propname} is the property name passed as
first parameter to \sphinxcode{\sphinxupquote{set()}}) and \sphinxcode{\sphinxupquote{change}}

\item {} 
\sphinxcode{\sphinxupquote{get()}} retrieves the value of a property.

\end{itemize}


\subsubsection{Exercise}
\label{\detokenize{howtos/web:id3}}
\begin{sphinxadmonition}{note}
Widget Properties and Events

Create a widget \sphinxcode{\sphinxupquote{ColorInputWidget}} that will display 3 \sphinxcode{\sphinxupquote{\textless{}input
type="text"\textgreater{}}}. Each of these \sphinxcode{\sphinxupquote{\textless{}input\textgreater{}}} is dedicated to type a
hexadecimal number from 00 to FF. When any of these \sphinxcode{\sphinxupquote{\textless{}input\textgreater{}}} is
modified by the user the widget must query the content of the three
\sphinxcode{\sphinxupquote{\textless{}input\textgreater{}}}, concatenate their values to have a complete CSS color code
(ie: \sphinxcode{\sphinxupquote{\#00FF00}}) and put the result in a property named \sphinxcode{\sphinxupquote{color}}. Please
note the jQuery \sphinxcode{\sphinxupquote{change()}} event that you can bind on any HTML
\sphinxcode{\sphinxupquote{\textless{}input\textgreater{}}} element and the \sphinxcode{\sphinxupquote{val()}} method that can query the current
value of that \sphinxcode{\sphinxupquote{\textless{}input\textgreater{}}} could be useful to you for this exercise.

Then, modify the \sphinxcode{\sphinxupquote{HomePage}} widget to instantiate \sphinxcode{\sphinxupquote{ColorInputWidget}}
and display it. The \sphinxcode{\sphinxupquote{HomePage}} widget should also display an empty
rectangle. That rectangle must always, at any moment, have the same
background color as the color in the \sphinxcode{\sphinxupquote{color}} property of the
\sphinxcode{\sphinxupquote{ColorInputWidget}} instance.

Use QWeb to generate all HTML.

\fvset{hllines={, ,}}%
\begin{sphinxVerbatim}[commandchars=\\\{\}]
\PYG{n+nx}{odoo}\PYG{p}{.}\PYG{n+nx}{oepetstore} \PYG{o}{=} \PYG{k+kd}{function}\PYG{p}{(}\PYG{n+nx}{instance}\PYG{p}{,} \PYG{n+nx}{local}\PYG{p}{)} \PYG{p}{\PYGZob{}}
    \PYG{k+kd}{var} \PYG{n+nx}{\PYGZus{}t} \PYG{o}{=} \PYG{n+nx}{instance}\PYG{p}{.}\PYG{n+nx}{web}\PYG{p}{.}\PYG{n+nx}{\PYGZus{}t}\PYG{p}{,}
        \PYG{n+nx}{\PYGZus{}lt} \PYG{o}{=} \PYG{n+nx}{instance}\PYG{p}{.}\PYG{n+nx}{web}\PYG{p}{.}\PYG{n+nx}{\PYGZus{}lt}\PYG{p}{;}
    \PYG{k+kd}{var} \PYG{n+nx}{QWeb} \PYG{o}{=} \PYG{n+nx}{instance}\PYG{p}{.}\PYG{n+nx}{web}\PYG{p}{.}\PYG{n+nx}{qweb}\PYG{p}{;}

    \PYG{n+nx}{local}\PYG{p}{.}\PYG{n+nx}{ColorInputWidget} \PYG{o}{=} \PYG{n+nx}{instance}\PYG{p}{.}\PYG{n+nx}{Widget}\PYG{p}{.}\PYG{n+nx}{extend}\PYG{p}{(}\PYG{p}{\PYGZob{}}
        \PYG{n+nx}{template}\PYG{o}{:} \PYG{l+s+s2}{\PYGZdq{}ColorInputWidget\PYGZdq{}}\PYG{p}{,}
        \PYG{n+nx}{events}\PYG{o}{:} \PYG{p}{\PYGZob{}}
            \PYG{l+s+s1}{\PYGZsq{}change input\PYGZsq{}}\PYG{o}{:} \PYG{l+s+s1}{\PYGZsq{}input\PYGZus{}changed\PYGZsq{}}
        \PYG{p}{\PYGZcb{}}\PYG{p}{,}
        \PYG{n+nx}{start}\PYG{o}{:} \PYG{k+kd}{function}\PYG{p}{(}\PYG{p}{)} \PYG{p}{\PYGZob{}}
            \PYG{k}{this}\PYG{p}{.}\PYG{n+nx}{input\PYGZus{}changed}\PYG{p}{(}\PYG{p}{)}\PYG{p}{;}
            \PYG{k}{return} \PYG{k}{this}\PYG{p}{.}\PYG{n+nx}{\PYGZus{}super}\PYG{p}{(}\PYG{p}{)}\PYG{p}{;}
        \PYG{p}{\PYGZcb{}}\PYG{p}{,}
        \PYG{n+nx}{input\PYGZus{}changed}\PYG{o}{:} \PYG{k+kd}{function}\PYG{p}{(}\PYG{p}{)} \PYG{p}{\PYGZob{}}
            \PYG{k+kd}{var} \PYG{n+nx}{color} \PYG{o}{=} \PYG{p}{[}
                \PYG{l+s+s2}{\PYGZdq{}\PYGZsh{}\PYGZdq{}}\PYG{p}{,}
                \PYG{k}{this}\PYG{p}{.}\PYG{n+nx}{\PYGZdl{}}\PYG{p}{(}\PYG{l+s+s2}{\PYGZdq{}.oe\PYGZus{}color\PYGZus{}red\PYGZdq{}}\PYG{p}{)}\PYG{p}{.}\PYG{n+nx}{val}\PYG{p}{(}\PYG{p}{)}\PYG{p}{,}
                \PYG{k}{this}\PYG{p}{.}\PYG{n+nx}{\PYGZdl{}}\PYG{p}{(}\PYG{l+s+s2}{\PYGZdq{}.oe\PYGZus{}color\PYGZus{}green\PYGZdq{}}\PYG{p}{)}\PYG{p}{.}\PYG{n+nx}{val}\PYG{p}{(}\PYG{p}{)}\PYG{p}{,}
                \PYG{k}{this}\PYG{p}{.}\PYG{n+nx}{\PYGZdl{}}\PYG{p}{(}\PYG{l+s+s2}{\PYGZdq{}.oe\PYGZus{}color\PYGZus{}blue\PYGZdq{}}\PYG{p}{)}\PYG{p}{.}\PYG{n+nx}{val}\PYG{p}{(}\PYG{p}{)}
            \PYG{p}{]}\PYG{p}{.}\PYG{n+nx}{join}\PYG{p}{(}\PYG{l+s+s1}{\PYGZsq{}\PYGZsq{}}\PYG{p}{)}\PYG{p}{;}
            \PYG{k}{this}\PYG{p}{.}\PYG{n+nx}{set}\PYG{p}{(}\PYG{l+s+s2}{\PYGZdq{}color\PYGZdq{}}\PYG{p}{,} \PYG{n+nx}{color}\PYG{p}{)}\PYG{p}{;}
        \PYG{p}{\PYGZcb{}}\PYG{p}{,}
    \PYG{p}{\PYGZcb{}}\PYG{p}{)}\PYG{p}{;}

    \PYG{n+nx}{local}\PYG{p}{.}\PYG{n+nx}{HomePage} \PYG{o}{=} \PYG{n+nx}{instance}\PYG{p}{.}\PYG{n+nx}{Widget}\PYG{p}{.}\PYG{n+nx}{extend}\PYG{p}{(}\PYG{p}{\PYGZob{}}
        \PYG{n+nx}{template}\PYG{o}{:} \PYG{l+s+s2}{\PYGZdq{}HomePage\PYGZdq{}}\PYG{p}{,}
        \PYG{n+nx}{start}\PYG{o}{:} \PYG{k+kd}{function}\PYG{p}{(}\PYG{p}{)} \PYG{p}{\PYGZob{}}
            \PYG{k}{this}\PYG{p}{.}\PYG{n+nx}{colorInput} \PYG{o}{=} \PYG{k}{new} \PYG{n+nx}{local}\PYG{p}{.}\PYG{n+nx}{ColorInputWidget}\PYG{p}{(}\PYG{k}{this}\PYG{p}{)}\PYG{p}{;}
            \PYG{k}{this}\PYG{p}{.}\PYG{n+nx}{colorInput}\PYG{p}{.}\PYG{n+nx}{on}\PYG{p}{(}\PYG{l+s+s2}{\PYGZdq{}change:color\PYGZdq{}}\PYG{p}{,} \PYG{k}{this}\PYG{p}{,} \PYG{k}{this}\PYG{p}{.}\PYG{n+nx}{color\PYGZus{}changed}\PYG{p}{)}\PYG{p}{;}
            \PYG{k}{return} \PYG{k}{this}\PYG{p}{.}\PYG{n+nx}{colorInput}\PYG{p}{.}\PYG{n+nx}{appendTo}\PYG{p}{(}\PYG{k}{this}\PYG{p}{.}\PYG{n+nx}{\PYGZdl{}el}\PYG{p}{)}\PYG{p}{;}
        \PYG{p}{\PYGZcb{}}\PYG{p}{,}
        \PYG{n+nx}{color\PYGZus{}changed}\PYG{o}{:} \PYG{k+kd}{function}\PYG{p}{(}\PYG{p}{)} \PYG{p}{\PYGZob{}}
            \PYG{k}{this}\PYG{p}{.}\PYG{n+nx}{\PYGZdl{}}\PYG{p}{(}\PYG{l+s+s2}{\PYGZdq{}.oe\PYGZus{}color\PYGZus{}div\PYGZdq{}}\PYG{p}{)}\PYG{p}{.}\PYG{n+nx}{css}\PYG{p}{(}\PYG{l+s+s2}{\PYGZdq{}background\PYGZhy{}color\PYGZdq{}}\PYG{p}{,} \PYG{k}{this}\PYG{p}{.}\PYG{n+nx}{colorInput}\PYG{p}{.}\PYG{n+nx}{get}\PYG{p}{(}\PYG{l+s+s2}{\PYGZdq{}color\PYGZdq{}}\PYG{p}{)}\PYG{p}{)}\PYG{p}{;}
        \PYG{p}{\PYGZcb{}}\PYG{p}{,}
    \PYG{p}{\PYGZcb{}}\PYG{p}{)}\PYG{p}{;}

    \PYG{n+nx}{instance}\PYG{p}{.}\PYG{n+nx}{web}\PYG{p}{.}\PYG{n+nx}{client\PYGZus{}actions}\PYG{p}{.}\PYG{n+nx}{add}\PYG{p}{(}\PYG{l+s+s1}{\PYGZsq{}petstore.homepage\PYGZsq{}}\PYG{p}{,} \PYG{l+s+s1}{\PYGZsq{}instance.oepetstore.HomePage\PYGZsq{}}\PYG{p}{)}\PYG{p}{;}
\PYG{p}{\PYGZcb{}}
\end{sphinxVerbatim}

\fvset{hllines={, ,}}%
\begin{sphinxVerbatim}[commandchars=\\\{\}]
\PYG{c+cp}{\PYGZlt{}?xml version=\PYGZdq{}1.0\PYGZdq{} encoding=\PYGZdq{}UTF\PYGZhy{}8\PYGZdq{}?\PYGZgt{}}
\PYG{n+nt}{\PYGZlt{}templates} \PYG{n+na}{xml:space=}\PYG{l+s}{\PYGZdq{}preserve\PYGZdq{}}\PYG{n+nt}{\PYGZgt{}}
    \PYG{n+nt}{\PYGZlt{}t} \PYG{n+na}{t\PYGZhy{}name=}\PYG{l+s}{\PYGZdq{}ColorInputWidget\PYGZdq{}}\PYG{n+nt}{\PYGZgt{}}
        \PYG{n+nt}{\PYGZlt{}div}\PYG{n+nt}{\PYGZgt{}}
            Red: \PYG{n+nt}{\PYGZlt{}input} \PYG{n+na}{type=}\PYG{l+s}{\PYGZdq{}text\PYGZdq{}} \PYG{n+na}{class=}\PYG{l+s}{\PYGZdq{}oe\PYGZus{}color\PYGZus{}red\PYGZdq{}} \PYG{n+na}{value=}\PYG{l+s}{\PYGZdq{}00\PYGZdq{}}\PYG{n+nt}{\PYGZgt{}}\PYG{n+nt}{\PYGZlt{}/input\PYGZgt{}}\PYG{n+nt}{\PYGZlt{}br} \PYG{n+nt}{/\PYGZgt{}}
            Green: \PYG{n+nt}{\PYGZlt{}input} \PYG{n+na}{type=}\PYG{l+s}{\PYGZdq{}text\PYGZdq{}} \PYG{n+na}{class=}\PYG{l+s}{\PYGZdq{}oe\PYGZus{}color\PYGZus{}green\PYGZdq{}} \PYG{n+na}{value=}\PYG{l+s}{\PYGZdq{}00\PYGZdq{}}\PYG{n+nt}{\PYGZgt{}}\PYG{n+nt}{\PYGZlt{}/input\PYGZgt{}}\PYG{n+nt}{\PYGZlt{}br} \PYG{n+nt}{/\PYGZgt{}}
            Blue: \PYG{n+nt}{\PYGZlt{}input} \PYG{n+na}{type=}\PYG{l+s}{\PYGZdq{}text\PYGZdq{}} \PYG{n+na}{class=}\PYG{l+s}{\PYGZdq{}oe\PYGZus{}color\PYGZus{}blue\PYGZdq{}} \PYG{n+na}{value=}\PYG{l+s}{\PYGZdq{}00\PYGZdq{}}\PYG{n+nt}{\PYGZgt{}}\PYG{n+nt}{\PYGZlt{}/input\PYGZgt{}}\PYG{n+nt}{\PYGZlt{}br} \PYG{n+nt}{/\PYGZgt{}}
        \PYG{n+nt}{\PYGZlt{}/div\PYGZgt{}}
    \PYG{n+nt}{\PYGZlt{}/t\PYGZgt{}}
    \PYG{n+nt}{\PYGZlt{}t} \PYG{n+na}{t\PYGZhy{}name=}\PYG{l+s}{\PYGZdq{}HomePage\PYGZdq{}}\PYG{n+nt}{\PYGZgt{}}
        \PYG{n+nt}{\PYGZlt{}div}\PYG{n+nt}{\PYGZgt{}}
            \PYG{n+nt}{\PYGZlt{}div} \PYG{n+na}{class=}\PYG{l+s}{\PYGZdq{}oe\PYGZus{}color\PYGZus{}div\PYGZdq{}}\PYG{n+nt}{\PYGZgt{}}\PYG{n+nt}{\PYGZlt{}/div\PYGZgt{}}
        \PYG{n+nt}{\PYGZlt{}/div\PYGZgt{}}
    \PYG{n+nt}{\PYGZlt{}/t\PYGZgt{}}
\PYG{n+nt}{\PYGZlt{}/templates\PYGZgt{}}
\end{sphinxVerbatim}

\fvset{hllines={, ,}}%
\begin{sphinxVerbatim}[commandchars=\\\{\}]
\PYG{n+nc}{.oe\PYGZus{}color\PYGZus{}div} \PYG{p}{\PYGZob{}}
    \PYG{n+nb}{width}\PYG{o}{:} \PYG{l+m}{100px}\PYG{p}{;}
    \PYG{n+nb}{height}\PYG{o}{:} \PYG{l+m}{100px}\PYG{p}{;}
    \PYG{n+nb}{margin}\PYG{o}{:} \PYG{l+m}{10px}\PYG{p}{;}
\PYG{p}{\PYGZcb{}}
\end{sphinxVerbatim}
\end{sphinxadmonition}


\subsection{Modify existing widgets and classes}
\label{\detokenize{howtos/web:modify-existing-widgets-and-classes}}
The class system of the Odoo web framework allows direct modification of
existing classes using the \sphinxcode{\sphinxupquote{include()}} method:

\fvset{hllines={, ,}}%
\begin{sphinxVerbatim}[commandchars=\\\{\}]
\PYG{k+kd}{var} \PYG{n+nx}{TestClass} \PYG{o}{=} \PYG{n+nx}{instance}\PYG{p}{.}\PYG{n+nx}{web}\PYG{p}{.}\PYG{n+nx}{Class}\PYG{p}{.}\PYG{n+nx}{extend}\PYG{p}{(}\PYG{p}{\PYGZob{}}
    \PYG{n+nx}{testMethod}\PYG{o}{:} \PYG{k+kd}{function}\PYG{p}{(}\PYG{p}{)} \PYG{p}{\PYGZob{}}
        \PYG{k}{return} \PYG{l+s+s2}{\PYGZdq{}hello\PYGZdq{}}\PYG{p}{;}
    \PYG{p}{\PYGZcb{}}\PYG{p}{,}
\PYG{p}{\PYGZcb{}}\PYG{p}{)}\PYG{p}{;}

\PYG{n+nx}{TestClass}\PYG{p}{.}\PYG{n+nx}{include}\PYG{p}{(}\PYG{p}{\PYGZob{}}
    \PYG{n+nx}{testMethod}\PYG{o}{:} \PYG{k+kd}{function}\PYG{p}{(}\PYG{p}{)} \PYG{p}{\PYGZob{}}
        \PYG{k}{return} \PYG{k}{this}\PYG{p}{.}\PYG{n+nx}{\PYGZus{}super}\PYG{p}{(}\PYG{p}{)} \PYG{o}{+} \PYG{l+s+s2}{\PYGZdq{} world\PYGZdq{}}\PYG{p}{;}
    \PYG{p}{\PYGZcb{}}\PYG{p}{,}
\PYG{p}{\PYGZcb{}}\PYG{p}{)}\PYG{p}{;}

\PYG{n+nx}{console}\PYG{p}{.}\PYG{n+nx}{log}\PYG{p}{(}\PYG{k}{new} \PYG{n+nx}{TestClass}\PYG{p}{(}\PYG{p}{)}\PYG{p}{.}\PYG{n+nx}{testMethod}\PYG{p}{(}\PYG{p}{)}\PYG{p}{)}\PYG{p}{;}
\PYG{c+c1}{// will print \PYGZdq{}hello world\PYGZdq{}}
\end{sphinxVerbatim}

This system is similar to the inheritance mechanism, except it will alter the
target class in-place instead of creating a new class.

In that case, \sphinxcode{\sphinxupquote{this.\_super()}} will call the original implementation of a
method being replaced/redefined. If the class already had sub-classes, all
calls to \sphinxcode{\sphinxupquote{this.\_super()}} in sub-classes will call the new implementations
defined in the call to \sphinxcode{\sphinxupquote{include()}}. This will also work
if some instances of the class (or of any of its sub-classes) were created
prior to the call to \sphinxcode{\sphinxupquote{include()}}.


\subsection{Translations}
\label{\detokenize{howtos/web:translations}}
The process to translate text in Python and JavaScript code is very
similar. You could have noticed these lines at the beginning of the
\sphinxcode{\sphinxupquote{petstore.js}} file:

\fvset{hllines={, ,}}%
\begin{sphinxVerbatim}[commandchars=\\\{\}]
\PYG{k+kd}{var} \PYG{n+nx}{\PYGZus{}t} \PYG{o}{=} \PYG{n+nx}{instance}\PYG{p}{.}\PYG{n+nx}{web}\PYG{p}{.}\PYG{n+nx}{\PYGZus{}t}\PYG{p}{,}
    \PYG{n+nx}{\PYGZus{}lt} \PYG{o}{=} \PYG{n+nx}{instance}\PYG{p}{.}\PYG{n+nx}{web}\PYG{p}{.}\PYG{n+nx}{\PYGZus{}lt}\PYG{p}{;}
\end{sphinxVerbatim}

These lines are simply used to import the translation functions in the current
JavaScript module. They are used thus:

\fvset{hllines={, ,}}%
\begin{sphinxVerbatim}[commandchars=\\\{\}]
\PYG{k}{this}\PYG{p}{.}\PYG{n+nx}{\PYGZdl{}el}\PYG{p}{.}\PYG{n+nx}{text}\PYG{p}{(}\PYG{n+nx}{\PYGZus{}t}\PYG{p}{(}\PYG{l+s+s2}{\PYGZdq{}Hello user!\PYGZdq{}}\PYG{p}{)}\PYG{p}{)}\PYG{p}{;}
\end{sphinxVerbatim}

In Odoo, translations files are automatically generated by scanning the source
code. All piece of code that calls a certain function are detected and their
content is added to a translation file that will then be sent to the
translators. In Python, the function is \sphinxcode{\sphinxupquote{\_()}}. In JavaScript the function is
\sphinxcode{\sphinxupquote{\_t()}} (and also \sphinxcode{\sphinxupquote{\_lt()}}).

\sphinxcode{\sphinxupquote{\_t()}} will return the translation defined for the text it is given. If no
translation is defined for that text, it will return the original text as-is.

\begin{sphinxadmonition}{note}{Note:}
To inject user-provided values in translatable strings, it is recommended
to use \sphinxhref{http://gabceb.github.io/underscore.string.site/\#sprintf}{\_.str.sprintf} with named
arguments \sphinxstyleemphasis{after} the translation:

\fvset{hllines={, ,}}%
\begin{sphinxVerbatim}[commandchars=\\\{\}]
\PYG{k}{this}\PYG{p}{.}\PYG{n+nx}{\PYGZdl{}el}\PYG{p}{.}\PYG{n+nx}{text}\PYG{p}{(}\PYG{n+nx}{\PYGZus{}}\PYG{p}{.}\PYG{n+nx}{str}\PYG{p}{.}\PYG{n+nx}{sprintf}\PYG{p}{(}
    \PYG{n+nx}{\PYGZus{}t}\PYG{p}{(}\PYG{l+s+s2}{\PYGZdq{}Hello, \PYGZpc{}(user)s!\PYGZdq{}}\PYG{p}{)}\PYG{p}{,} \PYG{p}{\PYGZob{}}
    \PYG{n+nx}{user}\PYG{o}{:} \PYG{l+s+s2}{\PYGZdq{}Ed\PYGZdq{}}
\PYG{p}{\PYGZcb{}}\PYG{p}{)}\PYG{p}{)}\PYG{p}{;}
\end{sphinxVerbatim}

This makes translatable strings more readable to translators, and gives
them more flexibility to reorder or ignore parameters.
\end{sphinxadmonition}

\sphinxcode{\sphinxupquote{\_lt()}} (“lazy translate”) is similar but somewhat more
complex: instead of translating its parameter immediately, it returns
an object which, when converted to a string, will perform the translation.

It is used to define translatable terms before the translations system is
initialized, for class attributes for instance (as modules are loaded before
the user’s language is configured and translations are downloaded).


\subsection{Communication with the Odoo Server}
\label{\detokenize{howtos/web:communication-with-the-odoo-server}}

\subsubsection{Contacting Models}
\label{\detokenize{howtos/web:contacting-models}}
Most operations with Odoo involve communicating with \sphinxstyleemphasis{models} implementing
business concern, these models will then (potentially) interact with some
storage engine (usually \sphinxhref{http://en.wikipedia.org/wiki/PostgreSQL}{PostgreSQL}).

Although \sphinxhref{http://jquery.org}{jQuery} provides a \sphinxhref{http://api.jquery.com/jquery.ajax/}{\$.ajax} function for network interactions,
communicating with Odoo requires additional metadata whose setup before every
call would be verbose and error-prone. As a result, Odoo web provides
higher-level communication primitives.

To demonstrate this, the file \sphinxcode{\sphinxupquote{petstore.py}} already contains a small model
with a sample method:

\fvset{hllines={, ,}}%
\begin{sphinxVerbatim}[commandchars=\\\{\}]
\PYG{k}{class} \PYG{n+nc}{message\PYGZus{}of\PYGZus{}the\PYGZus{}day}\PYG{p}{(}\PYG{n}{models}\PYG{o}{.}\PYG{n}{Model}\PYG{p}{)}\PYG{p}{:}
    \PYG{n}{\PYGZus{}name} \PYG{o}{=} \PYG{l+s+s2}{\PYGZdq{}}\PYG{l+s+s2}{oepetstore.message\PYGZus{}of\PYGZus{}the\PYGZus{}day}\PYG{l+s+s2}{\PYGZdq{}}

    \PYG{n+nd}{@api.model}
    \PYG{k}{def} \PYG{n+nf}{my\PYGZus{}method}\PYG{p}{(}\PYG{n+nb+bp}{self}\PYG{p}{)}\PYG{p}{:}
        \PYG{k}{return} \PYG{p}{\PYGZob{}}\PYG{l+s+s2}{\PYGZdq{}}\PYG{l+s+s2}{hello}\PYG{l+s+s2}{\PYGZdq{}}\PYG{p}{:} \PYG{l+s+s2}{\PYGZdq{}}\PYG{l+s+s2}{world}\PYG{l+s+s2}{\PYGZdq{}}\PYG{p}{\PYGZcb{}}

    \PYG{n}{message} \PYG{o}{=} \PYG{n}{fields}\PYG{o}{.}\PYG{n}{Text}\PYG{p}{(}\PYG{p}{)}\PYG{p}{,}
    \PYG{n}{color} \PYG{o}{=} \PYG{n}{fields}\PYG{o}{.}\PYG{n}{Char}\PYG{p}{(}\PYG{n}{size}\PYG{o}{=}\PYG{l+m+mi}{20}\PYG{p}{)}\PYG{p}{,}
\end{sphinxVerbatim}

This declares a model with two fields, and a method \sphinxcode{\sphinxupquote{my\_method()}} which
returns a literal dictionary.

Here is a sample widget that calls \sphinxcode{\sphinxupquote{my\_method()}} and displays the result:

\fvset{hllines={, ,}}%
\begin{sphinxVerbatim}[commandchars=\\\{\}]
\PYG{n+nx}{local}\PYG{p}{.}\PYG{n+nx}{HomePage} \PYG{o}{=} \PYG{n+nx}{instance}\PYG{p}{.}\PYG{n+nx}{Widget}\PYG{p}{.}\PYG{n+nx}{extend}\PYG{p}{(}\PYG{p}{\PYGZob{}}
    \PYG{n+nx}{start}\PYG{o}{:} \PYG{k+kd}{function}\PYG{p}{(}\PYG{p}{)} \PYG{p}{\PYGZob{}}
        \PYG{k+kd}{var} \PYG{n+nx}{self} \PYG{o}{=} \PYG{k}{this}\PYG{p}{;}
        \PYG{k+kd}{var} \PYG{n+nx}{model} \PYG{o}{=} \PYG{k}{new} \PYG{n+nx}{instance}\PYG{p}{.}\PYG{n+nx}{web}\PYG{p}{.}\PYG{n+nx}{Model}\PYG{p}{(}\PYG{l+s+s2}{\PYGZdq{}oepetstore.message\PYGZus{}of\PYGZus{}the\PYGZus{}day\PYGZdq{}}\PYG{p}{)}\PYG{p}{;}
        \PYG{n+nx}{model}\PYG{p}{.}\PYG{n+nx}{call}\PYG{p}{(}\PYG{l+s+s2}{\PYGZdq{}my\PYGZus{}method\PYGZdq{}}\PYG{p}{,} \PYG{p}{\PYGZob{}}\PYG{n+nx}{context}\PYG{o}{:} \PYG{k}{new} \PYG{n+nx}{instance}\PYG{p}{.}\PYG{n+nx}{web}\PYG{p}{.}\PYG{n+nx}{CompoundContext}\PYG{p}{(}\PYG{p}{)}\PYG{p}{\PYGZcb{}}\PYG{p}{)}\PYG{p}{.}\PYG{n+nx}{then}\PYG{p}{(}\PYG{k+kd}{function}\PYG{p}{(}\PYG{n+nx}{result}\PYG{p}{)} \PYG{p}{\PYGZob{}}
            \PYG{n+nx}{self}\PYG{p}{.}\PYG{n+nx}{\PYGZdl{}el}\PYG{p}{.}\PYG{n+nx}{append}\PYG{p}{(}\PYG{l+s+s2}{\PYGZdq{}\PYGZlt{}div\PYGZgt{}Hello \PYGZdq{}} \PYG{o}{+} \PYG{n+nx}{result}\PYG{p}{[}\PYG{l+s+s2}{\PYGZdq{}hello\PYGZdq{}}\PYG{p}{]} \PYG{o}{+} \PYG{l+s+s2}{\PYGZdq{}\PYGZlt{}/div\PYGZgt{}\PYGZdq{}}\PYG{p}{)}\PYG{p}{;}
            \PYG{c+c1}{// will show \PYGZdq{}Hello world\PYGZdq{} to the user}
        \PYG{p}{\PYGZcb{}}\PYG{p}{)}\PYG{p}{;}
    \PYG{p}{\PYGZcb{}}\PYG{p}{,}
\PYG{p}{\PYGZcb{}}\PYG{p}{)}\PYG{p}{;}
\end{sphinxVerbatim}

The class used to call Odoo models is \sphinxcode{\sphinxupquote{odoo.Model()}}. It is
instantiated with the Odoo model’s name as first parameter
(\sphinxcode{\sphinxupquote{oepetstore.message\_of\_the\_day}} here).

\sphinxcode{\sphinxupquote{call()}} can be used to call any (public) method of an
Odoo model. It takes the following positional arguments:
\begin{description}
\item[{\sphinxcode{\sphinxupquote{name}}}] \leavevmode
The name of the method to call, \sphinxcode{\sphinxupquote{my\_method}} here

\item[{\sphinxcode{\sphinxupquote{args}}}] \leavevmode
an array of \sphinxhref{https://docs.python.org/2/glossary.html\#term-argument}{positional arguments} to provide to the method. Because the
example has no positional argument to provide, the \sphinxcode{\sphinxupquote{args}} parameter is not
provided.

Here is an other example with positional arguments:

\fvset{hllines={, ,}}%
\begin{sphinxVerbatim}[commandchars=\\\{\}]
\PYG{n+nd}{@api.model}
\PYG{k}{def} \PYG{n+nf}{my\PYGZus{}method2}\PYG{p}{(}\PYG{n+nb+bp}{self}\PYG{p}{,} \PYG{n}{a}\PYG{p}{,} \PYG{n}{b}\PYG{p}{,} \PYG{n}{c}\PYG{p}{)}\PYG{p}{:} \PYG{o}{.}\PYG{o}{.}\PYG{o}{.}
\end{sphinxVerbatim}

\fvset{hllines={, ,}}%
\begin{sphinxVerbatim}[commandchars=\\\{\}]
\PYG{n+nx}{model}\PYG{p}{.}\PYG{n+nx}{call}\PYG{p}{(}\PYG{l+s+s2}{\PYGZdq{}my\PYGZus{}method\PYGZdq{}}\PYG{p}{,} \PYG{p}{[}\PYG{l+m+mi}{1}\PYG{p}{,} \PYG{l+m+mi}{2}\PYG{p}{,} \PYG{l+m+mi}{3}\PYG{p}{]}\PYG{p}{,} \PYG{p}{...}
\PYG{c+c1}{// with this a=1, b=2 and c=3}
\end{sphinxVerbatim}

\item[{\sphinxcode{\sphinxupquote{kwargs}}}] \leavevmode
a mapping of \sphinxhref{https://docs.python.org/2/glossary.html\#term-argument}{keyword arguments} to pass. The example provides a single
named argument \sphinxcode{\sphinxupquote{context}}.

\fvset{hllines={, ,}}%
\begin{sphinxVerbatim}[commandchars=\\\{\}]
\PYG{n+nd}{@api.model}
\PYG{k}{def} \PYG{n+nf}{my\PYGZus{}method2}\PYG{p}{(}\PYG{n+nb+bp}{self}\PYG{p}{,} \PYG{n}{a}\PYG{p}{,} \PYG{n}{b}\PYG{p}{,} \PYG{n}{c}\PYG{p}{)}\PYG{p}{:} \PYG{o}{.}\PYG{o}{.}\PYG{o}{.}
\end{sphinxVerbatim}

\fvset{hllines={, ,}}%
\begin{sphinxVerbatim}[commandchars=\\\{\}]
\PYG{n+nx}{model}\PYG{p}{.}\PYG{n+nx}{call}\PYG{p}{(}\PYG{l+s+s2}{\PYGZdq{}my\PYGZus{}method\PYGZdq{}}\PYG{p}{,} \PYG{p}{[}\PYG{p}{]}\PYG{p}{,} \PYG{p}{\PYGZob{}}\PYG{n+nx}{a}\PYG{o}{:} \PYG{l+m+mi}{1}\PYG{p}{,} \PYG{n+nx}{b}\PYG{o}{:} \PYG{l+m+mi}{2}\PYG{p}{,} \PYG{n+nx}{c}\PYG{o}{:} \PYG{l+m+mi}{3}\PYG{p}{,} \PYG{p}{...}
\PYG{c+c1}{// with this a=1, b=2 and c=3}
\end{sphinxVerbatim}

\end{description}

\sphinxcode{\sphinxupquote{call()}} returns a deferred resolved with the value
returned by the model’s method as first argument.


\subsubsection{CompoundContext}
\label{\detokenize{howtos/web:compoundcontext}}
The previous section used a \sphinxcode{\sphinxupquote{context}} argument which was not explained in
the method call:

\fvset{hllines={, ,}}%
\begin{sphinxVerbatim}[commandchars=\\\{\}]
\PYG{n+nx}{model}\PYG{p}{.}\PYG{n+nx}{call}\PYG{p}{(}\PYG{l+s+s2}{\PYGZdq{}my\PYGZus{}method\PYGZdq{}}\PYG{p}{,} \PYG{p}{\PYGZob{}}\PYG{n+nx}{context}\PYG{o}{:} \PYG{k}{new} \PYG{n+nx}{instance}\PYG{p}{.}\PYG{n+nx}{web}\PYG{p}{.}\PYG{n+nx}{CompoundContext}\PYG{p}{(}\PYG{p}{)}\PYG{p}{\PYGZcb{}}\PYG{p}{)}
\end{sphinxVerbatim}

The context is like a “magic” argument that the web client will always give to
the server when calling a method. The context is a dictionary containing
multiple keys. One of the most important key is the language of the user, used
by the server to translate all the messages of the application. Another one is
the time zone of the user, used to compute correctly dates and times if Odoo
is used by people in different countries.

The \sphinxcode{\sphinxupquote{argument}} is necessary in all methods, otherwise bad things could
happen (such as the application not being translated correctly). That’s why,
when you call a model’s method, you should always provide that argument. The
solution to achieve that is to use \sphinxcode{\sphinxupquote{odoo.web.CompoundContext()}}.

\sphinxcode{\sphinxupquote{CompoundContext()}} is a class used to pass the user’s
context (with language, time zone, etc…) to the server as well as adding new
keys to the context (some models’ methods use arbitrary keys added to the
context). It is created by giving to its constructor any number of
dictionaries or other \sphinxcode{\sphinxupquote{CompoundContext()}} instances. It will
merge all those contexts before sending them to the server.

\fvset{hllines={, ,}}%
\begin{sphinxVerbatim}[commandchars=\\\{\}]
\PYG{n+nx}{model}\PYG{p}{.}\PYG{n+nx}{call}\PYG{p}{(}\PYG{l+s+s2}{\PYGZdq{}my\PYGZus{}method\PYGZdq{}}\PYG{p}{,} \PYG{p}{\PYGZob{}}\PYG{n+nx}{context}\PYG{o}{:} \PYG{k}{new} \PYG{n+nx}{instance}\PYG{p}{.}\PYG{n+nx}{web}\PYG{p}{.}\PYG{n+nx}{CompoundContext}\PYG{p}{(}\PYG{p}{\PYGZob{}}\PYG{l+s+s1}{\PYGZsq{}new\PYGZus{}key\PYGZsq{}}\PYG{o}{:} \PYG{l+s+s1}{\PYGZsq{}key\PYGZus{}value\PYGZsq{}}\PYG{p}{\PYGZcb{}}\PYG{p}{)}\PYG{p}{\PYGZcb{}}\PYG{p}{)}
\end{sphinxVerbatim}

\fvset{hllines={, ,}}%
\begin{sphinxVerbatim}[commandchars=\\\{\}]
\PYG{n+nd}{@api.model}
\PYG{k}{def} \PYG{n+nf}{my\PYGZus{}method}\PYG{p}{(}\PYG{n+nb+bp}{self}\PYG{p}{)}\PYG{p}{:}
    \PYG{k}{print} \PYG{n+nb+bp}{self}\PYG{o}{.}\PYG{n}{env}\PYG{o}{.}\PYG{n}{context}
    \PYG{o}{/}\PYG{o}{/} \PYG{n}{will} \PYG{k}{print}\PYG{p}{:} \PYG{p}{\PYGZob{}}\PYG{l+s+s1}{\PYGZsq{}}\PYG{l+s+s1}{lang}\PYG{l+s+s1}{\PYGZsq{}}\PYG{p}{:} \PYG{l+s+s1}{\PYGZsq{}}\PYG{l+s+s1}{en\PYGZus{}US}\PYG{l+s+s1}{\PYGZsq{}}\PYG{p}{,} \PYG{l+s+s1}{\PYGZsq{}}\PYG{l+s+s1}{new\PYGZus{}key}\PYG{l+s+s1}{\PYGZsq{}}\PYG{p}{:} \PYG{l+s+s1}{\PYGZsq{}}\PYG{l+s+s1}{key\PYGZus{}value}\PYG{l+s+s1}{\PYGZsq{}}\PYG{p}{,} \PYG{l+s+s1}{\PYGZsq{}}\PYG{l+s+s1}{tz}\PYG{l+s+s1}{\PYGZsq{}}\PYG{p}{:} \PYG{l+s+s1}{\PYGZsq{}}\PYG{l+s+s1}{Europe/Brussels}\PYG{l+s+s1}{\PYGZsq{}}\PYG{p}{,} \PYG{l+s+s1}{\PYGZsq{}}\PYG{l+s+s1}{uid}\PYG{l+s+s1}{\PYGZsq{}}\PYG{p}{:} \PYG{l+m+mi}{1}\PYG{p}{\PYGZcb{}}
\end{sphinxVerbatim}

You can see the dictionary in the argument \sphinxcode{\sphinxupquote{context}} contains some keys that
are related to the configuration of the current user in Odoo plus the
\sphinxcode{\sphinxupquote{new\_key}} key that was added when instantiating
\sphinxcode{\sphinxupquote{CompoundContext()}}.


\subsubsection{Queries}
\label{\detokenize{howtos/web:queries}}
While \sphinxcode{\sphinxupquote{call()}} is sufficient for any interaction with Odoo
models, Odoo Web provides a helper for simpler and clearer querying of models
(fetching of records based on various conditions):
\sphinxcode{\sphinxupquote{query()}} which acts as a shortcut for the common
combination of {\hyperref[\detokenize{reference/orm:odoo.models.Model.search}]{\sphinxcrossref{\sphinxcode{\sphinxupquote{search()}}}}} and
:{\hyperref[\detokenize{reference/orm:odoo.models.Model.read}]{\sphinxcrossref{\sphinxcode{\sphinxupquote{read()}}}}}. It provides a clearer syntax to search
and read models:

\fvset{hllines={, ,}}%
\begin{sphinxVerbatim}[commandchars=\\\{\}]
\PYG{n+nx}{model}\PYG{p}{.}\PYG{n+nx}{query}\PYG{p}{(}\PYG{p}{[}\PYG{l+s+s1}{\PYGZsq{}name\PYGZsq{}}\PYG{p}{,} \PYG{l+s+s1}{\PYGZsq{}login\PYGZsq{}}\PYG{p}{,} \PYG{l+s+s1}{\PYGZsq{}user\PYGZus{}email\PYGZsq{}}\PYG{p}{,} \PYG{l+s+s1}{\PYGZsq{}signature\PYGZsq{}}\PYG{p}{]}\PYG{p}{)}
     \PYG{p}{.}\PYG{n+nx}{filter}\PYG{p}{(}\PYG{p}{[}\PYG{p}{[}\PYG{l+s+s1}{\PYGZsq{}active\PYGZsq{}}\PYG{p}{,} \PYG{l+s+s1}{\PYGZsq{}=\PYGZsq{}}\PYG{p}{,} \PYG{k+kc}{true}\PYG{p}{]}\PYG{p}{,} \PYG{p}{[}\PYG{l+s+s1}{\PYGZsq{}company\PYGZus{}id\PYGZsq{}}\PYG{p}{,} \PYG{l+s+s1}{\PYGZsq{}=\PYGZsq{}}\PYG{p}{,} \PYG{n+nx}{main\PYGZus{}company}\PYG{p}{]}\PYG{p}{]}\PYG{p}{)}
     \PYG{p}{.}\PYG{n+nx}{limit}\PYG{p}{(}\PYG{l+m+mi}{15}\PYG{p}{)}
     \PYG{p}{.}\PYG{n+nx}{all}\PYG{p}{(}\PYG{p}{)}\PYG{p}{.}\PYG{n+nx}{then}\PYG{p}{(}\PYG{k+kd}{function} \PYG{p}{(}\PYG{n+nx}{users}\PYG{p}{)} \PYG{p}{\PYGZob{}}
    \PYG{c+c1}{// do work with users records}
\PYG{p}{\PYGZcb{}}\PYG{p}{)}\PYG{p}{;}
\end{sphinxVerbatim}

versus:

\fvset{hllines={, ,}}%
\begin{sphinxVerbatim}[commandchars=\\\{\}]
\PYG{n+nx}{model}\PYG{p}{.}\PYG{n+nx}{call}\PYG{p}{(}\PYG{l+s+s1}{\PYGZsq{}search\PYGZsq{}}\PYG{p}{,} \PYG{p}{[}\PYG{p}{[}\PYG{l+s+s1}{\PYGZsq{}active\PYGZsq{}}\PYG{p}{,} \PYG{l+s+s1}{\PYGZsq{}=\PYGZsq{}}\PYG{p}{,} \PYG{k+kc}{true}\PYG{p}{]}\PYG{p}{,} \PYG{p}{[}\PYG{l+s+s1}{\PYGZsq{}company\PYGZus{}id\PYGZsq{}}\PYG{p}{,} \PYG{l+s+s1}{\PYGZsq{}=\PYGZsq{}}\PYG{p}{,} \PYG{n+nx}{main\PYGZus{}company}\PYG{p}{]}\PYG{p}{]}\PYG{p}{,} \PYG{p}{\PYGZob{}}\PYG{n+nx}{limit}\PYG{o}{:} \PYG{l+m+mi}{15}\PYG{p}{\PYGZcb{}}\PYG{p}{)}
    \PYG{p}{.}\PYG{n+nx}{then}\PYG{p}{(}\PYG{k+kd}{function} \PYG{p}{(}\PYG{n+nx}{ids}\PYG{p}{)} \PYG{p}{\PYGZob{}}
        \PYG{k}{return} \PYG{n+nx}{model}\PYG{p}{.}\PYG{n+nx}{call}\PYG{p}{(}\PYG{l+s+s1}{\PYGZsq{}read\PYGZsq{}}\PYG{p}{,} \PYG{p}{[}\PYG{n+nx}{ids}\PYG{p}{,} \PYG{p}{[}\PYG{l+s+s1}{\PYGZsq{}name\PYGZsq{}}\PYG{p}{,} \PYG{l+s+s1}{\PYGZsq{}login\PYGZsq{}}\PYG{p}{,} \PYG{l+s+s1}{\PYGZsq{}user\PYGZus{}email\PYGZsq{}}\PYG{p}{,} \PYG{l+s+s1}{\PYGZsq{}signature\PYGZsq{}}\PYG{p}{]}\PYG{p}{]}\PYG{p}{)}\PYG{p}{;}
    \PYG{p}{\PYGZcb{}}\PYG{p}{)}
    \PYG{p}{.}\PYG{n+nx}{then}\PYG{p}{(}\PYG{k+kd}{function} \PYG{p}{(}\PYG{n+nx}{users}\PYG{p}{)} \PYG{p}{\PYGZob{}}
        \PYG{c+c1}{// do work with users records}
    \PYG{p}{\PYGZcb{}}\PYG{p}{)}\PYG{p}{;}
\end{sphinxVerbatim}
\begin{itemize}
\item {} 
\sphinxcode{\sphinxupquote{query()}} takes an optional list of fields as
parameter (if no field is provided, all fields of the model are fetched). It
returns a \sphinxcode{\sphinxupquote{odoo.web.Query()}} which can be further customized before
being executed

\item {} 
\sphinxcode{\sphinxupquote{Query()}} represents the query being built. It is
immutable, methods to customize the query actually return a modified copy,
so it’s possible to use the original and the new version side-by-side. See
\sphinxcode{\sphinxupquote{Query()}} for its customization options.

\end{itemize}

When the query is set up as desired, simply call
\sphinxcode{\sphinxupquote{all()}} to execute it and return a
deferred to its result. The result is the same as
{\hyperref[\detokenize{reference/orm:odoo.models.Model.read}]{\sphinxcrossref{\sphinxcode{\sphinxupquote{read()}}}}}’s, an array of dictionaries where each
dictionary is a requested record, with each requested field a dictionary key.


\subsection{Exercises}
\label{\detokenize{howtos/web:exercises}}
\begin{sphinxadmonition}{note}
Message of the Day

Create a \sphinxcode{\sphinxupquote{MessageOfTheDay}}  widget displaying the last record of the
\sphinxcode{\sphinxupquote{oepetstore.message\_of\_the\_day}} model. The widget should fetch its
record as soon as it is displayed.

Display the widget in the Pet Store home page.

\fvset{hllines={, ,}}%
\begin{sphinxVerbatim}[commandchars=\\\{\}]
\PYG{n+nx}{odoo}\PYG{p}{.}\PYG{n+nx}{oepetstore} \PYG{o}{=} \PYG{k+kd}{function}\PYG{p}{(}\PYG{n+nx}{instance}\PYG{p}{,} \PYG{n+nx}{local}\PYG{p}{)} \PYG{p}{\PYGZob{}}
    \PYG{k+kd}{var} \PYG{n+nx}{\PYGZus{}t} \PYG{o}{=} \PYG{n+nx}{instance}\PYG{p}{.}\PYG{n+nx}{web}\PYG{p}{.}\PYG{n+nx}{\PYGZus{}t}\PYG{p}{,}
        \PYG{n+nx}{\PYGZus{}lt} \PYG{o}{=} \PYG{n+nx}{instance}\PYG{p}{.}\PYG{n+nx}{web}\PYG{p}{.}\PYG{n+nx}{\PYGZus{}lt}\PYG{p}{;}
    \PYG{k+kd}{var} \PYG{n+nx}{QWeb} \PYG{o}{=} \PYG{n+nx}{instance}\PYG{p}{.}\PYG{n+nx}{web}\PYG{p}{.}\PYG{n+nx}{qweb}\PYG{p}{;}

    \PYG{n+nx}{local}\PYG{p}{.}\PYG{n+nx}{HomePage} \PYG{o}{=} \PYG{n+nx}{instance}\PYG{p}{.}\PYG{n+nx}{Widget}\PYG{p}{.}\PYG{n+nx}{extend}\PYG{p}{(}\PYG{p}{\PYGZob{}}
        \PYG{n+nx}{template}\PYG{o}{:} \PYG{l+s+s2}{\PYGZdq{}HomePage\PYGZdq{}}\PYG{p}{,}
        \PYG{n+nx}{start}\PYG{o}{:} \PYG{k+kd}{function}\PYG{p}{(}\PYG{p}{)} \PYG{p}{\PYGZob{}}
            \PYG{k}{return} \PYG{k}{new} \PYG{n+nx}{local}\PYG{p}{.}\PYG{n+nx}{MessageOfTheDay}\PYG{p}{(}\PYG{k}{this}\PYG{p}{)}\PYG{p}{.}\PYG{n+nx}{appendTo}\PYG{p}{(}\PYG{k}{this}\PYG{p}{.}\PYG{n+nx}{\PYGZdl{}el}\PYG{p}{)}\PYG{p}{;}
        \PYG{p}{\PYGZcb{}}\PYG{p}{,}
    \PYG{p}{\PYGZcb{}}\PYG{p}{)}\PYG{p}{;}

    \PYG{n+nx}{instance}\PYG{p}{.}\PYG{n+nx}{web}\PYG{p}{.}\PYG{n+nx}{client\PYGZus{}actions}\PYG{p}{.}\PYG{n+nx}{add}\PYG{p}{(}\PYG{l+s+s1}{\PYGZsq{}petstore.homepage\PYGZsq{}}\PYG{p}{,} \PYG{l+s+s1}{\PYGZsq{}instance.oepetstore.HomePage\PYGZsq{}}\PYG{p}{)}\PYG{p}{;}

    \PYG{n+nx}{local}\PYG{p}{.}\PYG{n+nx}{MessageOfTheDay} \PYG{o}{=} \PYG{n+nx}{instance}\PYG{p}{.}\PYG{n+nx}{Widget}\PYG{p}{.}\PYG{n+nx}{extend}\PYG{p}{(}\PYG{p}{\PYGZob{}}
        \PYG{n+nx}{template}\PYG{o}{:} \PYG{l+s+s2}{\PYGZdq{}MessageOfTheDay\PYGZdq{}}\PYG{p}{,}
        \PYG{n+nx}{start}\PYG{o}{:} \PYG{k+kd}{function}\PYG{p}{(}\PYG{p}{)} \PYG{p}{\PYGZob{}}
            \PYG{k+kd}{var} \PYG{n+nx}{self} \PYG{o}{=} \PYG{k}{this}\PYG{p}{;}
            \PYG{k}{return} \PYG{k}{new} \PYG{n+nx}{instance}\PYG{p}{.}\PYG{n+nx}{web}\PYG{p}{.}\PYG{n+nx}{Model}\PYG{p}{(}\PYG{l+s+s2}{\PYGZdq{}oepetstore.message\PYGZus{}of\PYGZus{}the\PYGZus{}day\PYGZdq{}}\PYG{p}{)}
                \PYG{p}{.}\PYG{n+nx}{query}\PYG{p}{(}\PYG{p}{[}\PYG{l+s+s2}{\PYGZdq{}message\PYGZdq{}}\PYG{p}{]}\PYG{p}{)}
                \PYG{p}{.}\PYG{n+nx}{order\PYGZus{}by}\PYG{p}{(}\PYG{l+s+s1}{\PYGZsq{}\PYGZhy{}create\PYGZus{}date\PYGZsq{}}\PYG{p}{,} \PYG{l+s+s1}{\PYGZsq{}\PYGZhy{}id\PYGZsq{}}\PYG{p}{)}
                \PYG{p}{.}\PYG{n+nx}{first}\PYG{p}{(}\PYG{p}{)}
                \PYG{p}{.}\PYG{n+nx}{then}\PYG{p}{(}\PYG{k+kd}{function}\PYG{p}{(}\PYG{n+nx}{result}\PYG{p}{)} \PYG{p}{\PYGZob{}}
                    \PYG{n+nx}{self}\PYG{p}{.}\PYG{n+nx}{\PYGZdl{}}\PYG{p}{(}\PYG{l+s+s2}{\PYGZdq{}.oe\PYGZus{}mywidget\PYGZus{}message\PYGZus{}of\PYGZus{}the\PYGZus{}day\PYGZdq{}}\PYG{p}{)}\PYG{p}{.}\PYG{n+nx}{text}\PYG{p}{(}\PYG{n+nx}{result}\PYG{p}{.}\PYG{n+nx}{message}\PYG{p}{)}\PYG{p}{;}
                \PYG{p}{\PYGZcb{}}\PYG{p}{)}\PYG{p}{;}
        \PYG{p}{\PYGZcb{}}\PYG{p}{,}
    \PYG{p}{\PYGZcb{}}\PYG{p}{)}\PYG{p}{;}

\PYG{p}{\PYGZcb{}}
\end{sphinxVerbatim}

\fvset{hllines={, ,}}%
\begin{sphinxVerbatim}[commandchars=\\\{\}]
\PYG{c+cp}{\PYGZlt{}?xml version=\PYGZdq{}1.0\PYGZdq{} encoding=\PYGZdq{}UTF\PYGZhy{}8\PYGZdq{}?\PYGZgt{}}
\PYG{n+nt}{\PYGZlt{}templates} \PYG{n+na}{xml:space=}\PYG{l+s}{\PYGZdq{}preserve\PYGZdq{}}\PYG{n+nt}{\PYGZgt{}}
    \PYG{n+nt}{\PYGZlt{}t} \PYG{n+na}{t\PYGZhy{}name=}\PYG{l+s}{\PYGZdq{}HomePage\PYGZdq{}}\PYG{n+nt}{\PYGZgt{}}
        \PYG{n+nt}{\PYGZlt{}div} \PYG{n+na}{class=}\PYG{l+s}{\PYGZdq{}oe\PYGZus{}petstore\PYGZus{}homepage\PYGZdq{}}\PYG{n+nt}{\PYGZgt{}}
        \PYG{n+nt}{\PYGZlt{}/div\PYGZgt{}}
    \PYG{n+nt}{\PYGZlt{}/t\PYGZgt{}}
    \PYG{n+nt}{\PYGZlt{}t} \PYG{n+na}{t\PYGZhy{}name=}\PYG{l+s}{\PYGZdq{}MessageOfTheDay\PYGZdq{}}\PYG{n+nt}{\PYGZgt{}}
        \PYG{n+nt}{\PYGZlt{}div} \PYG{n+na}{class=}\PYG{l+s}{\PYGZdq{}oe\PYGZus{}petstore\PYGZus{}motd\PYGZdq{}}\PYG{n+nt}{\PYGZgt{}}
            \PYG{n+nt}{\PYGZlt{}p} \PYG{n+na}{class=}\PYG{l+s}{\PYGZdq{}oe\PYGZus{}mywidget\PYGZus{}message\PYGZus{}of\PYGZus{}the\PYGZus{}day\PYGZdq{}}\PYG{n+nt}{\PYGZgt{}}\PYG{n+nt}{\PYGZlt{}/p\PYGZgt{}}
        \PYG{n+nt}{\PYGZlt{}/div\PYGZgt{}}
    \PYG{n+nt}{\PYGZlt{}/t\PYGZgt{}}
\PYG{n+nt}{\PYGZlt{}/templates\PYGZgt{}}
\end{sphinxVerbatim}

\fvset{hllines={, ,}}%
\begin{sphinxVerbatim}[commandchars=\\\{\}]
\PYG{n+nc}{.oe\PYGZus{}petstore\PYGZus{}motd} \PYG{p}{\PYGZob{}}
    \PYG{n+nb}{margin}\PYG{o}{:} \PYG{l+m}{5px}\PYG{p}{;}
    \PYG{n+nb}{padding}\PYG{o}{:} \PYG{l+m}{5px}\PYG{p}{;}
    \PYG{n+nb}{border}\PYG{o}{\PYGZhy{}}\PYG{n}{radius}\PYG{o}{:} \PYG{l+m}{3px}\PYG{p}{;}
    \PYG{n+nb}{background\PYGZhy{}color}\PYG{o}{:} \PYG{l+m}{\PYGZsh{}F0EEEE}\PYG{p}{;}
\PYG{p}{\PYGZcb{}}
\end{sphinxVerbatim}
\end{sphinxadmonition}

\begin{sphinxadmonition}{note}
Pet Toys List

Create a \sphinxcode{\sphinxupquote{PetToysList}} widget displaying 5 toys (using their name and
their images).

The pet toys are not stored in a new model, instead they’re stored in
\sphinxcode{\sphinxupquote{product.product}} using a special category \sphinxstyleemphasis{Pet Toys}. You can see the
pre-generated toys and add new ones by going to
\sphinxmenuselection{Pet Store \(\rightarrow\) Pet Store \(\rightarrow\) Pet Toys}. You will probably
need to explore \sphinxcode{\sphinxupquote{product.product}} to create the right domain to
select just pet toys.

In Odoo, images are generally stored in regular fields encoded as
\sphinxhref{http://en.wikipedia.org/wiki/Base64}{base64}, HTML supports displaying images straight from base64 with
\sphinxcode{\sphinxupquote{\textless{}img src="data:\sphinxstyleemphasis{mime\_type};base64,\sphinxstyleemphasis{base64\_image\_data}"/\textgreater{}}}

The \sphinxcode{\sphinxupquote{PetToysList}} widget should be displayed on the home page on the
right of the \sphinxcode{\sphinxupquote{MessageOfTheDay}} widget. You will need to make some layout
with CSS to achieve this.

\fvset{hllines={, ,}}%
\begin{sphinxVerbatim}[commandchars=\\\{\}]
\PYG{n+nx}{odoo}\PYG{p}{.}\PYG{n+nx}{oepetstore} \PYG{o}{=} \PYG{k+kd}{function}\PYG{p}{(}\PYG{n+nx}{instance}\PYG{p}{,} \PYG{n+nx}{local}\PYG{p}{)} \PYG{p}{\PYGZob{}}
    \PYG{k+kd}{var} \PYG{n+nx}{\PYGZus{}t} \PYG{o}{=} \PYG{n+nx}{instance}\PYG{p}{.}\PYG{n+nx}{web}\PYG{p}{.}\PYG{n+nx}{\PYGZus{}t}\PYG{p}{,}
        \PYG{n+nx}{\PYGZus{}lt} \PYG{o}{=} \PYG{n+nx}{instance}\PYG{p}{.}\PYG{n+nx}{web}\PYG{p}{.}\PYG{n+nx}{\PYGZus{}lt}\PYG{p}{;}
    \PYG{k+kd}{var} \PYG{n+nx}{QWeb} \PYG{o}{=} \PYG{n+nx}{instance}\PYG{p}{.}\PYG{n+nx}{web}\PYG{p}{.}\PYG{n+nx}{qweb}\PYG{p}{;}

    \PYG{n+nx}{local}\PYG{p}{.}\PYG{n+nx}{HomePage} \PYG{o}{=} \PYG{n+nx}{instance}\PYG{p}{.}\PYG{n+nx}{Widget}\PYG{p}{.}\PYG{n+nx}{extend}\PYG{p}{(}\PYG{p}{\PYGZob{}}
        \PYG{n+nx}{template}\PYG{o}{:} \PYG{l+s+s2}{\PYGZdq{}HomePage\PYGZdq{}}\PYG{p}{,}
        \PYG{n+nx}{start}\PYG{o}{:} \PYG{k+kd}{function} \PYG{p}{(}\PYG{p}{)} \PYG{p}{\PYGZob{}}
            \PYG{k}{return} \PYG{n+nx}{\PYGZdl{}}\PYG{p}{.}\PYG{n+nx}{when}\PYG{p}{(}
                \PYG{k}{new} \PYG{n+nx}{local}\PYG{p}{.}\PYG{n+nx}{PetToysList}\PYG{p}{(}\PYG{k}{this}\PYG{p}{)}\PYG{p}{.}\PYG{n+nx}{appendTo}\PYG{p}{(}\PYG{k}{this}\PYG{p}{.}\PYG{n+nx}{\PYGZdl{}}\PYG{p}{(}\PYG{l+s+s1}{\PYGZsq{}.oe\PYGZus{}petstore\PYGZus{}homepage\PYGZus{}left\PYGZsq{}}\PYG{p}{)}\PYG{p}{)}\PYG{p}{,}
                \PYG{k}{new} \PYG{n+nx}{local}\PYG{p}{.}\PYG{n+nx}{MessageOfTheDay}\PYG{p}{(}\PYG{k}{this}\PYG{p}{)}\PYG{p}{.}\PYG{n+nx}{appendTo}\PYG{p}{(}\PYG{k}{this}\PYG{p}{.}\PYG{n+nx}{\PYGZdl{}}\PYG{p}{(}\PYG{l+s+s1}{\PYGZsq{}.oe\PYGZus{}petstore\PYGZus{}homepage\PYGZus{}right\PYGZsq{}}\PYG{p}{)}\PYG{p}{)}
            \PYG{p}{)}\PYG{p}{;}
        \PYG{p}{\PYGZcb{}}
    \PYG{p}{\PYGZcb{}}\PYG{p}{)}\PYG{p}{;}
    \PYG{n+nx}{instance}\PYG{p}{.}\PYG{n+nx}{web}\PYG{p}{.}\PYG{n+nx}{client\PYGZus{}actions}\PYG{p}{.}\PYG{n+nx}{add}\PYG{p}{(}\PYG{l+s+s1}{\PYGZsq{}petstore.homepage\PYGZsq{}}\PYG{p}{,} \PYG{l+s+s1}{\PYGZsq{}instance.oepetstore.HomePage\PYGZsq{}}\PYG{p}{)}\PYG{p}{;}

    \PYG{n+nx}{local}\PYG{p}{.}\PYG{n+nx}{MessageOfTheDay} \PYG{o}{=} \PYG{n+nx}{instance}\PYG{p}{.}\PYG{n+nx}{Widget}\PYG{p}{.}\PYG{n+nx}{extend}\PYG{p}{(}\PYG{p}{\PYGZob{}}
        \PYG{n+nx}{template}\PYG{o}{:} \PYG{l+s+s1}{\PYGZsq{}MessageOfTheDay\PYGZsq{}}\PYG{p}{,}
        \PYG{n+nx}{start}\PYG{o}{:} \PYG{k+kd}{function} \PYG{p}{(}\PYG{p}{)} \PYG{p}{\PYGZob{}}
            \PYG{k+kd}{var} \PYG{n+nx}{self} \PYG{o}{=} \PYG{k}{this}\PYG{p}{;}
            \PYG{k}{return} \PYG{k}{new} \PYG{n+nx}{instance}\PYG{p}{.}\PYG{n+nx}{web}\PYG{p}{.}\PYG{n+nx}{Model}\PYG{p}{(}\PYG{l+s+s1}{\PYGZsq{}oepetstore.message\PYGZus{}of\PYGZus{}the\PYGZus{}day\PYGZsq{}}\PYG{p}{)}
                \PYG{p}{.}\PYG{n+nx}{query}\PYG{p}{(}\PYG{p}{[}\PYG{l+s+s2}{\PYGZdq{}message\PYGZdq{}}\PYG{p}{]}\PYG{p}{)}
                \PYG{p}{.}\PYG{n+nx}{order\PYGZus{}by}\PYG{p}{(}\PYG{l+s+s1}{\PYGZsq{}\PYGZhy{}create\PYGZus{}date\PYGZsq{}}\PYG{p}{,} \PYG{l+s+s1}{\PYGZsq{}\PYGZhy{}id\PYGZsq{}}\PYG{p}{)}
                \PYG{p}{.}\PYG{n+nx}{first}\PYG{p}{(}\PYG{p}{)}
                \PYG{p}{.}\PYG{n+nx}{then}\PYG{p}{(}\PYG{k+kd}{function} \PYG{p}{(}\PYG{n+nx}{result}\PYG{p}{)} \PYG{p}{\PYGZob{}}
                    \PYG{n+nx}{self}\PYG{p}{.}\PYG{n+nx}{\PYGZdl{}}\PYG{p}{(}\PYG{l+s+s2}{\PYGZdq{}.oe\PYGZus{}mywidget\PYGZus{}message\PYGZus{}of\PYGZus{}the\PYGZus{}day\PYGZdq{}}\PYG{p}{)}\PYG{p}{.}\PYG{n+nx}{text}\PYG{p}{(}\PYG{n+nx}{result}\PYG{p}{.}\PYG{n+nx}{message}\PYG{p}{)}\PYG{p}{;}
                \PYG{p}{\PYGZcb{}}\PYG{p}{)}\PYG{p}{;}
        \PYG{p}{\PYGZcb{}}
    \PYG{p}{\PYGZcb{}}\PYG{p}{)}\PYG{p}{;}

    \PYG{n+nx}{local}\PYG{p}{.}\PYG{n+nx}{PetToysList} \PYG{o}{=} \PYG{n+nx}{instance}\PYG{p}{.}\PYG{n+nx}{Widget}\PYG{p}{.}\PYG{n+nx}{extend}\PYG{p}{(}\PYG{p}{\PYGZob{}}
        \PYG{n+nx}{template}\PYG{o}{:} \PYG{l+s+s1}{\PYGZsq{}PetToysList\PYGZsq{}}\PYG{p}{,}
        \PYG{n+nx}{start}\PYG{o}{:} \PYG{k+kd}{function} \PYG{p}{(}\PYG{p}{)} \PYG{p}{\PYGZob{}}
            \PYG{k+kd}{var} \PYG{n+nx}{self} \PYG{o}{=} \PYG{k}{this}\PYG{p}{;}
            \PYG{k}{return} \PYG{k}{new} \PYG{n+nx}{instance}\PYG{p}{.}\PYG{n+nx}{web}\PYG{p}{.}\PYG{n+nx}{Model}\PYG{p}{(}\PYG{l+s+s1}{\PYGZsq{}product.product\PYGZsq{}}\PYG{p}{)}
                \PYG{p}{.}\PYG{n+nx}{query}\PYG{p}{(}\PYG{p}{[}\PYG{l+s+s1}{\PYGZsq{}name\PYGZsq{}}\PYG{p}{,} \PYG{l+s+s1}{\PYGZsq{}image\PYGZsq{}}\PYG{p}{]}\PYG{p}{)}
                \PYG{p}{.}\PYG{n+nx}{filter}\PYG{p}{(}\PYG{p}{[}\PYG{p}{[}\PYG{l+s+s1}{\PYGZsq{}categ\PYGZus{}id.name\PYGZsq{}}\PYG{p}{,} \PYG{l+s+s1}{\PYGZsq{}=\PYGZsq{}}\PYG{p}{,} \PYG{l+s+s2}{\PYGZdq{}Pet Toys\PYGZdq{}}\PYG{p}{]}\PYG{p}{]}\PYG{p}{)}
                \PYG{p}{.}\PYG{n+nx}{limit}\PYG{p}{(}\PYG{l+m+mi}{5}\PYG{p}{)}
                \PYG{p}{.}\PYG{n+nx}{all}\PYG{p}{(}\PYG{p}{)}
                \PYG{p}{.}\PYG{n+nx}{then}\PYG{p}{(}\PYG{k+kd}{function} \PYG{p}{(}\PYG{n+nx}{results}\PYG{p}{)} \PYG{p}{\PYGZob{}}
                    \PYG{n+nx}{\PYGZus{}}\PYG{p}{(}\PYG{n+nx}{results}\PYG{p}{)}\PYG{p}{.}\PYG{n+nx}{each}\PYG{p}{(}\PYG{k+kd}{function} \PYG{p}{(}\PYG{n+nx}{item}\PYG{p}{)} \PYG{p}{\PYGZob{}}
                        \PYG{n+nx}{self}\PYG{p}{.}\PYG{n+nx}{\PYGZdl{}el}\PYG{p}{.}\PYG{n+nx}{append}\PYG{p}{(}\PYG{n+nx}{QWeb}\PYG{p}{.}\PYG{n+nx}{render}\PYG{p}{(}\PYG{l+s+s1}{\PYGZsq{}PetToy\PYGZsq{}}\PYG{p}{,} \PYG{p}{\PYGZob{}}\PYG{n+nx}{item}\PYG{o}{:} \PYG{n+nx}{item}\PYG{p}{\PYGZcb{}}\PYG{p}{)}\PYG{p}{)}\PYG{p}{;}
                    \PYG{p}{\PYGZcb{}}\PYG{p}{)}\PYG{p}{;}
                \PYG{p}{\PYGZcb{}}\PYG{p}{)}\PYG{p}{;}
        \PYG{p}{\PYGZcb{}}
    \PYG{p}{\PYGZcb{}}\PYG{p}{)}\PYG{p}{;}
\PYG{p}{\PYGZcb{}}
\end{sphinxVerbatim}

\fvset{hllines={, ,}}%
\begin{sphinxVerbatim}[commandchars=\\\{\}]
\PYG{c+cp}{\PYGZlt{}?xml version=\PYGZdq{}1.0\PYGZdq{} encoding=\PYGZdq{}UTF\PYGZhy{}8\PYGZdq{}?\PYGZgt{}}

\PYG{n+nt}{\PYGZlt{}templates} \PYG{n+na}{xml:space=}\PYG{l+s}{\PYGZdq{}preserve\PYGZdq{}}\PYG{n+nt}{\PYGZgt{}}
    \PYG{n+nt}{\PYGZlt{}t} \PYG{n+na}{t\PYGZhy{}name=}\PYG{l+s}{\PYGZdq{}HomePage\PYGZdq{}}\PYG{n+nt}{\PYGZgt{}}
        \PYG{n+nt}{\PYGZlt{}div} \PYG{n+na}{class=}\PYG{l+s}{\PYGZdq{}oe\PYGZus{}petstore\PYGZus{}homepage\PYGZdq{}}\PYG{n+nt}{\PYGZgt{}}
            \PYG{n+nt}{\PYGZlt{}div} \PYG{n+na}{class=}\PYG{l+s}{\PYGZdq{}oe\PYGZus{}petstore\PYGZus{}homepage\PYGZus{}left\PYGZdq{}}\PYG{n+nt}{\PYGZgt{}}\PYG{n+nt}{\PYGZlt{}/div\PYGZgt{}}
            \PYG{n+nt}{\PYGZlt{}div} \PYG{n+na}{class=}\PYG{l+s}{\PYGZdq{}oe\PYGZus{}petstore\PYGZus{}homepage\PYGZus{}right\PYGZdq{}}\PYG{n+nt}{\PYGZgt{}}\PYG{n+nt}{\PYGZlt{}/div\PYGZgt{}}
        \PYG{n+nt}{\PYGZlt{}/div\PYGZgt{}}
    \PYG{n+nt}{\PYGZlt{}/t\PYGZgt{}}
    \PYG{n+nt}{\PYGZlt{}t} \PYG{n+na}{t\PYGZhy{}name=}\PYG{l+s}{\PYGZdq{}MessageOfTheDay\PYGZdq{}}\PYG{n+nt}{\PYGZgt{}}
        \PYG{n+nt}{\PYGZlt{}div} \PYG{n+na}{class=}\PYG{l+s}{\PYGZdq{}oe\PYGZus{}petstore\PYGZus{}motd\PYGZdq{}}\PYG{n+nt}{\PYGZgt{}}
            \PYG{n+nt}{\PYGZlt{}p} \PYG{n+na}{class=}\PYG{l+s}{\PYGZdq{}oe\PYGZus{}mywidget\PYGZus{}message\PYGZus{}of\PYGZus{}the\PYGZus{}day\PYGZdq{}}\PYG{n+nt}{\PYGZgt{}}\PYG{n+nt}{\PYGZlt{}/p\PYGZgt{}}
        \PYG{n+nt}{\PYGZlt{}/div\PYGZgt{}}
    \PYG{n+nt}{\PYGZlt{}/t\PYGZgt{}}
    \PYG{n+nt}{\PYGZlt{}t} \PYG{n+na}{t\PYGZhy{}name=}\PYG{l+s}{\PYGZdq{}PetToysList\PYGZdq{}}\PYG{n+nt}{\PYGZgt{}}
        \PYG{n+nt}{\PYGZlt{}div} \PYG{n+na}{class=}\PYG{l+s}{\PYGZdq{}oe\PYGZus{}petstore\PYGZus{}pettoyslist\PYGZdq{}}\PYG{n+nt}{\PYGZgt{}}
        \PYG{n+nt}{\PYGZlt{}/div\PYGZgt{}}
    \PYG{n+nt}{\PYGZlt{}/t\PYGZgt{}}
    \PYG{n+nt}{\PYGZlt{}t} \PYG{n+na}{t\PYGZhy{}name=}\PYG{l+s}{\PYGZdq{}PetToy\PYGZdq{}}\PYG{n+nt}{\PYGZgt{}}
        \PYG{n+nt}{\PYGZlt{}div} \PYG{n+na}{class=}\PYG{l+s}{\PYGZdq{}oe\PYGZus{}petstore\PYGZus{}pettoy\PYGZdq{}}\PYG{n+nt}{\PYGZgt{}}
            \PYG{n+nt}{\PYGZlt{}p}\PYG{n+nt}{\PYGZgt{}}\PYG{n+nt}{\PYGZlt{}t} \PYG{n+na}{t\PYGZhy{}esc=}\PYG{l+s}{\PYGZdq{}item.name\PYGZdq{}}\PYG{n+nt}{/\PYGZgt{}}\PYG{n+nt}{\PYGZlt{}/p\PYGZgt{}}
            \PYG{n+nt}{\PYGZlt{}p}\PYG{n+nt}{\PYGZgt{}}\PYG{n+nt}{\PYGZlt{}img} \PYG{n+na}{t\PYGZhy{}att\PYGZhy{}src=}\PYG{l+s}{\PYGZdq{}\PYGZsq{}data:image/jpg;base64,\PYGZsq{}+item.image\PYGZdq{}}\PYG{n+nt}{/\PYGZgt{}}\PYG{n+nt}{\PYGZlt{}/p\PYGZgt{}}
        \PYG{n+nt}{\PYGZlt{}/div\PYGZgt{}}
    \PYG{n+nt}{\PYGZlt{}/t\PYGZgt{}}
\PYG{n+nt}{\PYGZlt{}/templates\PYGZgt{}}
\end{sphinxVerbatim}

\fvset{hllines={, ,}}%
\begin{sphinxVerbatim}[commandchars=\\\{\}]
\PYG{n+nc}{.oe\PYGZus{}petstore\PYGZus{}homepage} \PYG{p}{\PYGZob{}}
    \PYG{n+nb}{display}\PYG{o}{:} \PYG{n}{table}\PYG{p}{;}
\PYG{p}{\PYGZcb{}}

\PYG{n+nc}{.oe\PYGZus{}petstore\PYGZus{}homepage\PYGZus{}left} \PYG{p}{\PYGZob{}}
    \PYG{n+nb}{display}\PYG{o}{:} \PYG{n+nb}{table\PYGZhy{}cell}\PYG{p}{;}
    \PYG{n+nb}{width} \PYG{o}{:} \PYG{l+m}{300px}\PYG{p}{;}
\PYG{p}{\PYGZcb{}}

\PYG{n+nc}{.oe\PYGZus{}petstore\PYGZus{}homepage\PYGZus{}right} \PYG{p}{\PYGZob{}}
    \PYG{n+nb}{display}\PYG{o}{:} \PYG{n+nb}{table\PYGZhy{}cell}\PYG{p}{;}
    \PYG{n+nb}{width} \PYG{o}{:} \PYG{l+m}{300px}\PYG{p}{;}
\PYG{p}{\PYGZcb{}}

\PYG{n+nc}{.oe\PYGZus{}petstore\PYGZus{}motd} \PYG{p}{\PYGZob{}}
    \PYG{n+nb}{margin}\PYG{o}{:} \PYG{l+m}{5px}\PYG{p}{;}
    \PYG{n+nb}{padding}\PYG{o}{:} \PYG{l+m}{5px}\PYG{p}{;}
    \PYG{n+nb}{border}\PYG{o}{\PYGZhy{}}\PYG{n}{radius}\PYG{o}{:} \PYG{l+m}{3px}\PYG{p}{;}
    \PYG{n+nb}{background\PYGZhy{}color}\PYG{o}{:} \PYG{l+m}{\PYGZsh{}F0EEEE}\PYG{p}{;}
\PYG{p}{\PYGZcb{}}

\PYG{n+nc}{.oe\PYGZus{}petstore\PYGZus{}pettoyslist} \PYG{p}{\PYGZob{}}
    \PYG{n+nb}{padding}\PYG{o}{:} \PYG{l+m}{5px}\PYG{p}{;}
\PYG{p}{\PYGZcb{}}

\PYG{n+nc}{.oe\PYGZus{}petstore\PYGZus{}pettoy} \PYG{p}{\PYGZob{}}
    \PYG{n+nb}{margin}\PYG{o}{:} \PYG{l+m}{5px}\PYG{p}{;}
    \PYG{n+nb}{padding}\PYG{o}{:} \PYG{l+m}{5px}\PYG{p}{;}
    \PYG{n+nb}{border}\PYG{o}{\PYGZhy{}}\PYG{n}{radius}\PYG{o}{:} \PYG{l+m}{3px}\PYG{p}{;}
    \PYG{n+nb}{background\PYGZhy{}color}\PYG{o}{:} \PYG{l+m}{\PYGZsh{}F0EEEE}\PYG{p}{;}
\PYG{p}{\PYGZcb{}}
\end{sphinxVerbatim}
\end{sphinxadmonition}


\subsection{Existing web components}
\label{\detokenize{howtos/web:existing-web-components}}

\subsubsection{The Action Manager}
\label{\detokenize{howtos/web:the-action-manager}}
In Odoo, many operations start from an {\hyperref[\detokenize{reference/actions:reference-actions}]{\sphinxcrossref{\DUrole{std,std-ref}{action}}}}:
opening a menu item (to a view), printing a report, …

Actions are pieces of data describing how a client should react to the
activation of a piece of content. Actions can be stored (and read through a
model) or they can be generated on-the fly (locally to the client by
javascript code, or remotely by a method of a model).

In Odoo Web, the component responsible for handling and reacting to these
actions is the \sphinxstyleemphasis{Action Manager}.


\paragraph{Using the Action Manager}
\label{\detokenize{howtos/web:using-the-action-manager}}
The action manager can be invoked explicitly from javascript code by creating
a dictionary describing {\hyperref[\detokenize{reference/actions:reference-actions}]{\sphinxcrossref{\DUrole{std,std-ref}{an action}}}} of the right
type, and calling an action manager instance with it.

\sphinxcode{\sphinxupquote{do\_action()}} is a shortcut of \sphinxcode{\sphinxupquote{Widget()}}
looking up the “current” action manager and executing the action:

\fvset{hllines={, ,}}%
\begin{sphinxVerbatim}[commandchars=\\\{\}]
\PYG{n+nx}{instance}\PYG{p}{.}\PYG{n+nx}{web}\PYG{p}{.}\PYG{n+nx}{TestWidget} \PYG{o}{=} \PYG{n+nx}{instance}\PYG{p}{.}\PYG{n+nx}{Widget}\PYG{p}{.}\PYG{n+nx}{extend}\PYG{p}{(}\PYG{p}{\PYGZob{}}
    \PYG{n+nx}{dispatch\PYGZus{}to\PYGZus{}new\PYGZus{}action}\PYG{o}{:} \PYG{k+kd}{function}\PYG{p}{(}\PYG{p}{)} \PYG{p}{\PYGZob{}}
        \PYG{k}{this}\PYG{p}{.}\PYG{n+nx}{do\PYGZus{}action}\PYG{p}{(}\PYG{p}{\PYGZob{}}
            \PYG{n+nx}{type}\PYG{o}{:} \PYG{l+s+s1}{\PYGZsq{}ir.actions.act\PYGZus{}window\PYGZsq{}}\PYG{p}{,}
            \PYG{n+nx}{res\PYGZus{}model}\PYG{o}{:} \PYG{l+s+s2}{\PYGZdq{}product.product\PYGZdq{}}\PYG{p}{,}
            \PYG{n+nx}{res\PYGZus{}id}\PYG{o}{:} \PYG{l+m+mi}{1}\PYG{p}{,}
            \PYG{n+nx}{views}\PYG{o}{:} \PYG{p}{[}\PYG{p}{[}\PYG{k+kc}{false}\PYG{p}{,} \PYG{l+s+s1}{\PYGZsq{}form\PYGZsq{}}\PYG{p}{]}\PYG{p}{]}\PYG{p}{,}
            \PYG{n+nx}{target}\PYG{o}{:} \PYG{l+s+s1}{\PYGZsq{}current\PYGZsq{}}\PYG{p}{,}
            \PYG{n+nx}{context}\PYG{o}{:} \PYG{p}{\PYGZob{}}\PYG{p}{\PYGZcb{}}\PYG{p}{,}
        \PYG{p}{\PYGZcb{}}\PYG{p}{)}\PYG{p}{;}
    \PYG{p}{\PYGZcb{}}\PYG{p}{,}
\PYG{p}{\PYGZcb{}}\PYG{p}{)}\PYG{p}{;}
\end{sphinxVerbatim}

The most common action \sphinxcode{\sphinxupquote{type}} is \sphinxcode{\sphinxupquote{ir.actions.act\_window}} which provides
views to a model (displays a model in various manners), its most common
attributes are:
\begin{description}
\item[{\sphinxcode{\sphinxupquote{res\_model}}}] \leavevmode
The model to display in views

\item[{\sphinxcode{\sphinxupquote{res\_id}} (optional)}] \leavevmode
For form views, a preselected record in \sphinxcode{\sphinxupquote{res\_model}}

\item[{\sphinxcode{\sphinxupquote{views}}}] \leavevmode
Lists the views available through the action. A list of
\sphinxcode{\sphinxupquote{{[}view\_id, view\_type{]}}}, \sphinxcode{\sphinxupquote{view\_id}} can either be the database identifier
of a view of the right type, or \sphinxcode{\sphinxupquote{false}} to use the view by default for
the specified type. View types can not be present multiple times. The action
will open the first view of the list by default.

\item[{\sphinxcode{\sphinxupquote{target}}}] \leavevmode
Either \sphinxcode{\sphinxupquote{current}} (the default) which replaces the “content” section of the
web client by the action, or \sphinxcode{\sphinxupquote{new}} to open the action in a dialog box.

\item[{\sphinxcode{\sphinxupquote{context}}}] \leavevmode
Additional context data to use within the action.

\end{description}

\begin{sphinxadmonition}{note}
Jump to Product

Modify the \sphinxcode{\sphinxupquote{PetToysList}} component so clicking on a toy replaces the
homepage by the toy’s form view.

\fvset{hllines={, ,}}%
\begin{sphinxVerbatim}[commandchars=\\\{\}]
\PYG{n+nx}{local}\PYG{p}{.}\PYG{n+nx}{PetToysList} \PYG{o}{=} \PYG{n+nx}{instance}\PYG{p}{.}\PYG{n+nx}{Widget}\PYG{p}{.}\PYG{n+nx}{extend}\PYG{p}{(}\PYG{p}{\PYGZob{}}
    \PYG{n+nx}{template}\PYG{o}{:} \PYG{l+s+s1}{\PYGZsq{}PetToysList\PYGZsq{}}\PYG{p}{,}
    \PYG{n+nx}{events}\PYG{o}{:} \PYG{p}{\PYGZob{}}
        \PYG{l+s+s1}{\PYGZsq{}click .oe\PYGZus{}petstore\PYGZus{}pettoy\PYGZsq{}}\PYG{o}{:} \PYG{l+s+s1}{\PYGZsq{}selected\PYGZus{}item\PYGZsq{}}\PYG{p}{,}
    \PYG{p}{\PYGZcb{}}\PYG{p}{,}
    \PYG{n+nx}{start}\PYG{o}{:} \PYG{k+kd}{function} \PYG{p}{(}\PYG{p}{)} \PYG{p}{\PYGZob{}}
        \PYG{k+kd}{var} \PYG{n+nx}{self} \PYG{o}{=} \PYG{k}{this}\PYG{p}{;}
        \PYG{k}{return} \PYG{k}{new} \PYG{n+nx}{instance}\PYG{p}{.}\PYG{n+nx}{web}\PYG{p}{.}\PYG{n+nx}{Model}\PYG{p}{(}\PYG{l+s+s1}{\PYGZsq{}product.product\PYGZsq{}}\PYG{p}{)}
            \PYG{p}{.}\PYG{n+nx}{query}\PYG{p}{(}\PYG{p}{[}\PYG{l+s+s1}{\PYGZsq{}name\PYGZsq{}}\PYG{p}{,} \PYG{l+s+s1}{\PYGZsq{}image\PYGZsq{}}\PYG{p}{]}\PYG{p}{)}
            \PYG{p}{.}\PYG{n+nx}{filter}\PYG{p}{(}\PYG{p}{[}\PYG{p}{[}\PYG{l+s+s1}{\PYGZsq{}categ\PYGZus{}id.name\PYGZsq{}}\PYG{p}{,} \PYG{l+s+s1}{\PYGZsq{}=\PYGZsq{}}\PYG{p}{,} \PYG{l+s+s2}{\PYGZdq{}Pet Toys\PYGZdq{}}\PYG{p}{]}\PYG{p}{]}\PYG{p}{)}
            \PYG{p}{.}\PYG{n+nx}{limit}\PYG{p}{(}\PYG{l+m+mi}{5}\PYG{p}{)}
            \PYG{p}{.}\PYG{n+nx}{all}\PYG{p}{(}\PYG{p}{)}
            \PYG{p}{.}\PYG{n+nx}{then}\PYG{p}{(}\PYG{k+kd}{function} \PYG{p}{(}\PYG{n+nx}{results}\PYG{p}{)} \PYG{p}{\PYGZob{}}
                \PYG{n+nx}{\PYGZus{}}\PYG{p}{(}\PYG{n+nx}{results}\PYG{p}{)}\PYG{p}{.}\PYG{n+nx}{each}\PYG{p}{(}\PYG{k+kd}{function} \PYG{p}{(}\PYG{n+nx}{item}\PYG{p}{)} \PYG{p}{\PYGZob{}}
                    \PYG{n+nx}{self}\PYG{p}{.}\PYG{n+nx}{\PYGZdl{}el}\PYG{p}{.}\PYG{n+nx}{append}\PYG{p}{(}\PYG{n+nx}{QWeb}\PYG{p}{.}\PYG{n+nx}{render}\PYG{p}{(}\PYG{l+s+s1}{\PYGZsq{}PetToy\PYGZsq{}}\PYG{p}{,} \PYG{p}{\PYGZob{}}\PYG{n+nx}{item}\PYG{o}{:} \PYG{n+nx}{item}\PYG{p}{\PYGZcb{}}\PYG{p}{)}\PYG{p}{)}\PYG{p}{;}
                \PYG{p}{\PYGZcb{}}\PYG{p}{)}\PYG{p}{;}
            \PYG{p}{\PYGZcb{}}\PYG{p}{)}\PYG{p}{;}
    \PYG{p}{\PYGZcb{}}\PYG{p}{,}
    \PYG{n+nx}{selected\PYGZus{}item}\PYG{o}{:} \PYG{k+kd}{function} \PYG{p}{(}\PYG{n+nx}{event}\PYG{p}{)} \PYG{p}{\PYGZob{}}
        \PYG{k}{this}\PYG{p}{.}\PYG{n+nx}{do\PYGZus{}action}\PYG{p}{(}\PYG{p}{\PYGZob{}}
            \PYG{n+nx}{type}\PYG{o}{:} \PYG{l+s+s1}{\PYGZsq{}ir.actions.act\PYGZus{}window\PYGZsq{}}\PYG{p}{,}
            \PYG{n+nx}{res\PYGZus{}model}\PYG{o}{:} \PYG{l+s+s1}{\PYGZsq{}product.product\PYGZsq{}}\PYG{p}{,}
            \PYG{n+nx}{res\PYGZus{}id}\PYG{o}{:} \PYG{n+nx}{\PYGZdl{}}\PYG{p}{(}\PYG{n+nx}{event}\PYG{p}{.}\PYG{n+nx}{currentTarget}\PYG{p}{)}\PYG{p}{.}\PYG{n+nx}{data}\PYG{p}{(}\PYG{l+s+s1}{\PYGZsq{}id\PYGZsq{}}\PYG{p}{)}\PYG{p}{,}
            \PYG{n+nx}{views}\PYG{o}{:} \PYG{p}{[}\PYG{p}{[}\PYG{k+kc}{false}\PYG{p}{,} \PYG{l+s+s1}{\PYGZsq{}form\PYGZsq{}}\PYG{p}{]}\PYG{p}{]}\PYG{p}{,}
        \PYG{p}{\PYGZcb{}}\PYG{p}{)}\PYG{p}{;}
    \PYG{p}{\PYGZcb{}}\PYG{p}{,}
\PYG{p}{\PYGZcb{}}\PYG{p}{)}\PYG{p}{;}
\end{sphinxVerbatim}

\fvset{hllines={, ,}}%
\begin{sphinxVerbatim}[commandchars=\\\{\}]
\PYG{n+nt}{\PYGZlt{}t} \PYG{n+na}{t\PYGZhy{}name=}\PYG{l+s}{\PYGZdq{}PetToy\PYGZdq{}}\PYG{n+nt}{\PYGZgt{}}
    \PYG{n+nt}{\PYGZlt{}div} \PYG{n+na}{class=}\PYG{l+s}{\PYGZdq{}oe\PYGZus{}petstore\PYGZus{}pettoy\PYGZdq{}} \PYG{n+na}{t\PYGZhy{}att\PYGZhy{}data\PYGZhy{}id=}\PYG{l+s}{\PYGZdq{}item.id\PYGZdq{}}\PYG{n+nt}{\PYGZgt{}}
        \PYG{n+nt}{\PYGZlt{}p}\PYG{n+nt}{\PYGZgt{}}\PYG{n+nt}{\PYGZlt{}t} \PYG{n+na}{t\PYGZhy{}esc=}\PYG{l+s}{\PYGZdq{}item.name\PYGZdq{}}\PYG{n+nt}{/\PYGZgt{}}\PYG{n+nt}{\PYGZlt{}/p\PYGZgt{}}
        \PYG{n+nt}{\PYGZlt{}p}\PYG{n+nt}{\PYGZgt{}}\PYG{n+nt}{\PYGZlt{}img} \PYG{n+na}{t\PYGZhy{}attf\PYGZhy{}src=}\PYG{l+s}{\PYGZdq{}data:image/jpg;base64,\PYGZob{}\PYGZob{}item.image\PYGZcb{}\PYGZcb{}\PYGZdq{}}\PYG{n+nt}{/\PYGZgt{}}\PYG{n+nt}{\PYGZlt{}/p\PYGZgt{}}
    \PYG{n+nt}{\PYGZlt{}/div\PYGZgt{}}
\PYG{n+nt}{\PYGZlt{}/t\PYGZgt{}}
\end{sphinxVerbatim}
\end{sphinxadmonition}


\subsubsection{Client Actions}
\label{\detokenize{howtos/web:client-actions}}
Throughout this guide, we used a simple \sphinxcode{\sphinxupquote{HomePage}} widget which the web
client automatically starts when we select the right menu item. But how did
the Odoo web know to start this widget? Because the widget is registered as
a \sphinxstyleemphasis{client action}.

A client action is (as its name implies) an action type defined almost
entirely in the client, in javascript for Odoo web. The server simply sends
an action tag (an arbitrary name), and optionally adds a few parameters, but
beyond that \sphinxstyleemphasis{everything} is handled by custom client code.

Our widget is registered as the handler for the client action through this:

\fvset{hllines={, ,}}%
\begin{sphinxVerbatim}[commandchars=\\\{\}]
\PYG{n+nx}{instance}\PYG{p}{.}\PYG{n+nx}{web}\PYG{p}{.}\PYG{n+nx}{client\PYGZus{}actions}\PYG{p}{.}\PYG{n+nx}{add}\PYG{p}{(}\PYG{l+s+s1}{\PYGZsq{}petstore.homepage\PYGZsq{}}\PYG{p}{,} \PYG{l+s+s1}{\PYGZsq{}instance.oepetstore.HomePage\PYGZsq{}}\PYG{p}{)}\PYG{p}{;}
\end{sphinxVerbatim}

\sphinxcode{\sphinxupquote{instance.web.client\_actions}} is a \sphinxcode{\sphinxupquote{Registry()}} in which
the action manager looks up client action handlers when it needs to execute
one. The first parameter of \sphinxcode{\sphinxupquote{add()}} is the name
(tag) of the client action, and the second parameter is the path to the widget
from the Odoo web client root.

When a client action must be executed, the action manager looks up its tag
in the registry, walks the specified path and displays the widget it finds at
the end.

\begin{sphinxadmonition}{note}{Note:}
a client action handler can also be a regular function, in whch case
it’ll be called and its result (if any) will be interpreted as the
next action to execute.
\end{sphinxadmonition}

On the server side, we had simply defined an \sphinxcode{\sphinxupquote{ir.actions.client}} action:

\fvset{hllines={, ,}}%
\begin{sphinxVerbatim}[commandchars=\\\{\}]
\PYG{n+nt}{\PYGZlt{}record} \PYG{n+na}{id=}\PYG{l+s}{\PYGZdq{}action\PYGZus{}home\PYGZus{}page\PYGZdq{}} \PYG{n+na}{model=}\PYG{l+s}{\PYGZdq{}ir.actions.client\PYGZdq{}}\PYG{n+nt}{\PYGZgt{}}
    \PYG{n+nt}{\PYGZlt{}field} \PYG{n+na}{name=}\PYG{l+s}{\PYGZdq{}tag\PYGZdq{}}\PYG{n+nt}{\PYGZgt{}}petstore.homepage\PYG{n+nt}{\PYGZlt{}/field\PYGZgt{}}
\PYG{n+nt}{\PYGZlt{}/record\PYGZgt{}}
\end{sphinxVerbatim}

and a menu opening the action:

\fvset{hllines={, ,}}%
\begin{sphinxVerbatim}[commandchars=\\\{\}]
\PYG{n+nt}{\PYGZlt{}menuitem} \PYG{n+na}{id=}\PYG{l+s}{\PYGZdq{}home\PYGZus{}page\PYGZus{}petstore\PYGZus{}menu\PYGZdq{}} \PYG{n+na}{parent=}\PYG{l+s}{\PYGZdq{}petstore\PYGZus{}menu\PYGZdq{}}
          \PYG{n+na}{name=}\PYG{l+s}{\PYGZdq{}Home Page\PYGZdq{}} \PYG{n+na}{action=}\PYG{l+s}{\PYGZdq{}action\PYGZus{}home\PYGZus{}page\PYGZdq{}}\PYG{n+nt}{/\PYGZgt{}}
\end{sphinxVerbatim}


\subsubsection{Architecture of the Views}
\label{\detokenize{howtos/web:architecture-of-the-views}}
Much of Odoo web’s usefulness (and complexity) resides in views. Each view
type is a way of displaying a model in the client.


\paragraph{The View Manager}
\label{\detokenize{howtos/web:the-view-manager}}
When an \sphinxcode{\sphinxupquote{ActionManager}} instance receive an action of type
\sphinxcode{\sphinxupquote{ir.actions.act\_window}}, it delegates the synchronization and handling of
the views themselves to a \sphinxstyleemphasis{view manager}, which will then set up one or
multiple views depending on the original action’s requirements:

\noindent{\hspace*{\fill}\sphinxincludegraphics[width=0.400\linewidth]{{viewarchitecture}.png}\hspace*{\fill}}


\paragraph{The Views}
\label{\detokenize{howtos/web:the-views}}
Most {\hyperref[\detokenize{reference/views:reference-views}]{\sphinxcrossref{\DUrole{std,std-ref}{Odoo views}}}} are implemented through a subclass
of \sphinxcode{\sphinxupquote{odoo.web.View()}} which provides a bit of generic basic structure
for handling events and displaying model information.

The \sphinxstyleemphasis{search view} is considered a view type by the main Odoo framework, but
handled separately by the web client (as it’s a more permanent fixture and
can interact with other views, which regular views don’t do).

A view is responsible for loading its own description XML (using
{\hyperref[\detokenize{reference/orm:odoo.models.Model.fields_view_get}]{\sphinxcrossref{\sphinxcode{\sphinxupquote{fields\_view\_get}}}}}) and any other data source
it needs. To that purpose, views are provided with an optional view
identifier set as the \sphinxcode{\sphinxupquote{view\_id}} attribute.

Views are also provided with a \sphinxcode{\sphinxupquote{DataSet()}} instance which
holds most necessary model information (the model name and possibly various
record ids).

Views may also want to handle search queries by overriding
\sphinxcode{\sphinxupquote{do\_search()}}, and updating their
\sphinxcode{\sphinxupquote{DataSet()}} as necessary.


\subsubsection{The Form View Fields}
\label{\detokenize{howtos/web:the-form-view-fields}}
A common need is the extension of the web form view to add new ways of
displaying fields.

All built-in fields have a default display implementation, a new
form widget may be necessary to correctly interact with a new field type
(e.g. a \DUrole{xref,std,std-term}{GIS} field) or to provide new representations and ways to
interact with existing field types (e.g. validate
{\hyperref[\detokenize{reference/orm:odoo.fields.Char}]{\sphinxcrossref{\sphinxcode{\sphinxupquote{Char}}}}} fields which should contain email addresses
and display them as email links).

To explicitly specify which form widget should be used to display a field,
simply use the \sphinxcode{\sphinxupquote{widget}} attribute in the view’s XML description:

\fvset{hllines={, ,}}%
\begin{sphinxVerbatim}[commandchars=\\\{\}]
\PYG{n+nt}{\PYGZlt{}field} \PYG{n+na}{name=}\PYG{l+s}{\PYGZdq{}contact\PYGZus{}mail\PYGZdq{}} \PYG{n+na}{widget=}\PYG{l+s}{\PYGZdq{}email\PYGZdq{}}\PYG{n+nt}{/\PYGZgt{}}
\end{sphinxVerbatim}

\begin{sphinxadmonition}{note}{Note:}\begin{itemize}
\item {} 
the same widget is used in both “view” (read-only) and “edition” modes
of a form view, it’s not possible to use a widget in one and an other
widget in the other

\item {} 
and a given field (name) can not be used multiple times in the same form

\item {} 
a widget may ignore the current mode of the form view and remain the
same in both view and edition

\end{itemize}
\end{sphinxadmonition}

Fields are instantiated by the form view after it has read its XML description
and constructed the corresponding HTML representing that description. After
that, the form view will communicate with the field objects using some
methods. These methods are defined by the \sphinxcode{\sphinxupquote{FieldInterface}}
interface. Almost all fields inherit the \sphinxcode{\sphinxupquote{AbstractField}} abstract
class. That class defines some default mechanisms that need to be implemented
by most fields.

Here are some of the responsibilities of a field class:
\begin{itemize}
\item {} 
The field class must display and allow the user to edit the value of the field.

\item {} 
It must correctly implement the 3 field attributes available in all fields
of Odoo. The \sphinxcode{\sphinxupquote{AbstractField}} class already implements an algorithm that
dynamically calculates the value of these attributes (they can change at any
moment because their value change according to the value of other
fields). Their values are stored in \sphinxstyleemphasis{Widget Properties} (the widget
properties were explained earlier in this guide). It is the responsibility
of each field class to check these widget properties and dynamically adapt
depending of their values. Here is a description of each of these
attributes:
\begin{itemize}
\item {} 
\sphinxcode{\sphinxupquote{required}}: The field must have a value before saving. If \sphinxcode{\sphinxupquote{required}}
is \sphinxcode{\sphinxupquote{true}} and the field doesn’t have a value, the method
\sphinxcode{\sphinxupquote{is\_valid()}} of the field must return \sphinxcode{\sphinxupquote{false}}.

\item {} 
\sphinxcode{\sphinxupquote{invisible}}: When this is \sphinxcode{\sphinxupquote{true}}, the field must be invisible. The
\sphinxcode{\sphinxupquote{AbstractField}} class already has a basic implementation of this
behavior that fits most fields.

\item {} 
\sphinxcode{\sphinxupquote{readonly}}: When \sphinxcode{\sphinxupquote{true}}, the field must not be editable by the
user. Most fields in Odoo have a completely different behavior depending
on the value of \sphinxcode{\sphinxupquote{readonly}}. As example, the \sphinxcode{\sphinxupquote{FieldChar}} displays an
HTML \sphinxcode{\sphinxupquote{\textless{}input\textgreater{}}} when it is editable and simply displays the text when
it is read-only. This also means it has much more code it would need to
implement only one behavior, but this is necessary to ensure a good user
experience.

\end{itemize}

\item {} 
Fields have two methods, \sphinxcode{\sphinxupquote{set\_value()}} and \sphinxcode{\sphinxupquote{get\_value()}}, which are
called by the form view to give it the value to display and get back the new
value entered by the user. These methods must be able to handle the value as
given by the Odoo server when a \sphinxcode{\sphinxupquote{read()}} is performed on a model and give
back a valid value for a \sphinxcode{\sphinxupquote{write()}}.  Remember that the JavaScript/Python
data types used to represent the values given by \sphinxcode{\sphinxupquote{read()}} and given to
\sphinxcode{\sphinxupquote{write()}} is not necessarily the same in Odoo. As example, when you read a
many2one, it is always a tuple whose first value is the id of the pointed
record and the second one is the name get (ie: \sphinxcode{\sphinxupquote{(15, "Agrolait")}}). But
when you write a many2one it must be a single integer, not a tuple
anymore. \sphinxcode{\sphinxupquote{AbstractField}} has a default implementation of these methods
that works well for simple data type and set a widget property named
\sphinxcode{\sphinxupquote{value}}.

\end{itemize}

Please note that, to better understand how to implement fields, you are
strongly encouraged to look at the definition of the \sphinxcode{\sphinxupquote{FieldInterface}}
interface and the \sphinxcode{\sphinxupquote{AbstractField}} class directly in the code of the Odoo web
client.


\paragraph{Creating a New Type of Field}
\label{\detokenize{howtos/web:creating-a-new-type-of-field}}
In this part we will explain how to create a new type of field. The example
here will be to re-implement the \sphinxcode{\sphinxupquote{FieldChar}} class and progressively explain
each part.


\subparagraph{Simple Read-Only Field}
\label{\detokenize{howtos/web:simple-read-only-field}}
Here is a first implementation that will only display text. The
user will not be able to modify the content of the field.

\fvset{hllines={, ,}}%
\begin{sphinxVerbatim}[commandchars=\\\{\}]
\PYG{n+nx}{local}\PYG{p}{.}\PYG{n+nx}{FieldChar2} \PYG{o}{=} \PYG{n+nx}{instance}\PYG{p}{.}\PYG{n+nx}{web}\PYG{p}{.}\PYG{n+nx}{form}\PYG{p}{.}\PYG{n+nx}{AbstractField}\PYG{p}{.}\PYG{n+nx}{extend}\PYG{p}{(}\PYG{p}{\PYGZob{}}
    \PYG{n+nx}{init}\PYG{o}{:} \PYG{k+kd}{function}\PYG{p}{(}\PYG{p}{)} \PYG{p}{\PYGZob{}}
        \PYG{k}{this}\PYG{p}{.}\PYG{n+nx}{\PYGZus{}super}\PYG{p}{.}\PYG{n+nx}{apply}\PYG{p}{(}\PYG{k}{this}\PYG{p}{,} \PYG{n+nx}{arguments}\PYG{p}{)}\PYG{p}{;}
        \PYG{k}{this}\PYG{p}{.}\PYG{n+nx}{set}\PYG{p}{(}\PYG{l+s+s2}{\PYGZdq{}value\PYGZdq{}}\PYG{p}{,} \PYG{l+s+s2}{\PYGZdq{}\PYGZdq{}}\PYG{p}{)}\PYG{p}{;}
    \PYG{p}{\PYGZcb{}}\PYG{p}{,}
    \PYG{n+nx}{render\PYGZus{}value}\PYG{o}{:} \PYG{k+kd}{function}\PYG{p}{(}\PYG{p}{)} \PYG{p}{\PYGZob{}}
        \PYG{k}{this}\PYG{p}{.}\PYG{n+nx}{\PYGZdl{}el}\PYG{p}{.}\PYG{n+nx}{text}\PYG{p}{(}\PYG{k}{this}\PYG{p}{.}\PYG{n+nx}{get}\PYG{p}{(}\PYG{l+s+s2}{\PYGZdq{}value\PYGZdq{}}\PYG{p}{)}\PYG{p}{)}\PYG{p}{;}
    \PYG{p}{\PYGZcb{}}\PYG{p}{,}
\PYG{p}{\PYGZcb{}}\PYG{p}{)}\PYG{p}{;}

\PYG{n+nx}{instance}\PYG{p}{.}\PYG{n+nx}{web}\PYG{p}{.}\PYG{n+nx}{form}\PYG{p}{.}\PYG{n+nx}{widgets}\PYG{p}{.}\PYG{n+nx}{add}\PYG{p}{(}\PYG{l+s+s1}{\PYGZsq{}char2\PYGZsq{}}\PYG{p}{,} \PYG{l+s+s1}{\PYGZsq{}instance.oepetstore.FieldChar2\PYGZsq{}}\PYG{p}{)}\PYG{p}{;}
\end{sphinxVerbatim}

In this example, we declare a class named \sphinxcode{\sphinxupquote{FieldChar2}} inheriting from
\sphinxcode{\sphinxupquote{AbstractField}}. We also register this class in the registry
\sphinxcode{\sphinxupquote{instance.web.form.widgets}} under the key \sphinxcode{\sphinxupquote{char2}}. That will allow us to
use this new field in any form view by specifying \sphinxcode{\sphinxupquote{widget="char2"}} in the
\sphinxcode{\sphinxupquote{\textless{}field/\textgreater{}}} tag in the XML declaration of the view.

In this example, we define a single method: \sphinxcode{\sphinxupquote{render\_value()}}. All it does is
display the widget property \sphinxcode{\sphinxupquote{value}}.  Those are two tools defined by the
\sphinxcode{\sphinxupquote{AbstractField}} class. As explained before, the form view will call the
method \sphinxcode{\sphinxupquote{set\_value()}} of the field to set the value to display. This method
already has a default implementation in \sphinxcode{\sphinxupquote{AbstractField}} which simply sets
the widget property \sphinxcode{\sphinxupquote{value}}. \sphinxcode{\sphinxupquote{AbstractField}} also watch the
\sphinxcode{\sphinxupquote{change:value}} event on itself and calls the \sphinxcode{\sphinxupquote{render\_value()}} when it
occurs. So, \sphinxcode{\sphinxupquote{render\_value()}} is a convenience method to implement in child
classes to perform some operation each time the value of the field changes.

In the \sphinxcode{\sphinxupquote{init()}} method, we also define the default value of the field if
none is specified by the form view (here we assume the default value of a
\sphinxcode{\sphinxupquote{char}} field should be an empty string).


\subparagraph{Read-Write Field}
\label{\detokenize{howtos/web:read-write-field}}
Read-only fields, which only display content and don’t allow the
user to modify it can be useful, but most fields in Odoo also allow editing.
This makes the field classes more complicated, mostly because fields are
supposed to handle both editable and non-editable mode, those modes are
often completely different (for design and usability purpose) and the fields
must be able to switch between modes at any moment.

To know in which mode the current field should be, the \sphinxcode{\sphinxupquote{AbstractField}} class
sets a widget property named \sphinxcode{\sphinxupquote{effective\_readonly}}. The field should watch
for changes in that widget property and display the correct mode
accordingly. Example:

\fvset{hllines={, ,}}%
\begin{sphinxVerbatim}[commandchars=\\\{\}]
\PYG{n+nx}{local}\PYG{p}{.}\PYG{n+nx}{FieldChar2} \PYG{o}{=} \PYG{n+nx}{instance}\PYG{p}{.}\PYG{n+nx}{web}\PYG{p}{.}\PYG{n+nx}{form}\PYG{p}{.}\PYG{n+nx}{AbstractField}\PYG{p}{.}\PYG{n+nx}{extend}\PYG{p}{(}\PYG{p}{\PYGZob{}}
    \PYG{n+nx}{init}\PYG{o}{:} \PYG{k+kd}{function}\PYG{p}{(}\PYG{p}{)} \PYG{p}{\PYGZob{}}
        \PYG{k}{this}\PYG{p}{.}\PYG{n+nx}{\PYGZus{}super}\PYG{p}{.}\PYG{n+nx}{apply}\PYG{p}{(}\PYG{k}{this}\PYG{p}{,} \PYG{n+nx}{arguments}\PYG{p}{)}\PYG{p}{;}
        \PYG{k}{this}\PYG{p}{.}\PYG{n+nx}{set}\PYG{p}{(}\PYG{l+s+s2}{\PYGZdq{}value\PYGZdq{}}\PYG{p}{,} \PYG{l+s+s2}{\PYGZdq{}\PYGZdq{}}\PYG{p}{)}\PYG{p}{;}
    \PYG{p}{\PYGZcb{}}\PYG{p}{,}
    \PYG{n+nx}{start}\PYG{o}{:} \PYG{k+kd}{function}\PYG{p}{(}\PYG{p}{)} \PYG{p}{\PYGZob{}}
        \PYG{k}{this}\PYG{p}{.}\PYG{n+nx}{on}\PYG{p}{(}\PYG{l+s+s2}{\PYGZdq{}change:effective\PYGZus{}readonly\PYGZdq{}}\PYG{p}{,} \PYG{k}{this}\PYG{p}{,} \PYG{k+kd}{function}\PYG{p}{(}\PYG{p}{)} \PYG{p}{\PYGZob{}}
            \PYG{k}{this}\PYG{p}{.}\PYG{n+nx}{display\PYGZus{}field}\PYG{p}{(}\PYG{p}{)}\PYG{p}{;}
            \PYG{k}{this}\PYG{p}{.}\PYG{n+nx}{render\PYGZus{}value}\PYG{p}{(}\PYG{p}{)}\PYG{p}{;}
        \PYG{p}{\PYGZcb{}}\PYG{p}{)}\PYG{p}{;}
        \PYG{k}{this}\PYG{p}{.}\PYG{n+nx}{display\PYGZus{}field}\PYG{p}{(}\PYG{p}{)}\PYG{p}{;}
        \PYG{k}{return} \PYG{k}{this}\PYG{p}{.}\PYG{n+nx}{\PYGZus{}super}\PYG{p}{(}\PYG{p}{)}\PYG{p}{;}
    \PYG{p}{\PYGZcb{}}\PYG{p}{,}
    \PYG{n+nx}{display\PYGZus{}field}\PYG{o}{:} \PYG{k+kd}{function}\PYG{p}{(}\PYG{p}{)} \PYG{p}{\PYGZob{}}
        \PYG{k+kd}{var} \PYG{n+nx}{self} \PYG{o}{=} \PYG{k}{this}\PYG{p}{;}
        \PYG{k}{this}\PYG{p}{.}\PYG{n+nx}{\PYGZdl{}el}\PYG{p}{.}\PYG{n+nx}{html}\PYG{p}{(}\PYG{n+nx}{QWeb}\PYG{p}{.}\PYG{n+nx}{render}\PYG{p}{(}\PYG{l+s+s2}{\PYGZdq{}FieldChar2\PYGZdq{}}\PYG{p}{,} \PYG{p}{\PYGZob{}}\PYG{n+nx}{widget}\PYG{o}{:} \PYG{k}{this}\PYG{p}{\PYGZcb{}}\PYG{p}{)}\PYG{p}{)}\PYG{p}{;}
        \PYG{k}{if} \PYG{p}{(}\PYG{o}{!} \PYG{k}{this}\PYG{p}{.}\PYG{n+nx}{get}\PYG{p}{(}\PYG{l+s+s2}{\PYGZdq{}effective\PYGZus{}readonly\PYGZdq{}}\PYG{p}{)}\PYG{p}{)} \PYG{p}{\PYGZob{}}
            \PYG{k}{this}\PYG{p}{.}\PYG{n+nx}{\PYGZdl{}}\PYG{p}{(}\PYG{l+s+s2}{\PYGZdq{}input\PYGZdq{}}\PYG{p}{)}\PYG{p}{.}\PYG{n+nx}{change}\PYG{p}{(}\PYG{k+kd}{function}\PYG{p}{(}\PYG{p}{)} \PYG{p}{\PYGZob{}}
                \PYG{n+nx}{self}\PYG{p}{.}\PYG{n+nx}{internal\PYGZus{}set\PYGZus{}value}\PYG{p}{(}\PYG{n+nx}{self}\PYG{p}{.}\PYG{n+nx}{\PYGZdl{}}\PYG{p}{(}\PYG{l+s+s2}{\PYGZdq{}input\PYGZdq{}}\PYG{p}{)}\PYG{p}{.}\PYG{n+nx}{val}\PYG{p}{(}\PYG{p}{)}\PYG{p}{)}\PYG{p}{;}
            \PYG{p}{\PYGZcb{}}\PYG{p}{)}\PYG{p}{;}
        \PYG{p}{\PYGZcb{}}
    \PYG{p}{\PYGZcb{}}\PYG{p}{,}
    \PYG{n+nx}{render\PYGZus{}value}\PYG{o}{:} \PYG{k+kd}{function}\PYG{p}{(}\PYG{p}{)} \PYG{p}{\PYGZob{}}
        \PYG{k}{if} \PYG{p}{(}\PYG{k}{this}\PYG{p}{.}\PYG{n+nx}{get}\PYG{p}{(}\PYG{l+s+s2}{\PYGZdq{}effective\PYGZus{}readonly\PYGZdq{}}\PYG{p}{)}\PYG{p}{)} \PYG{p}{\PYGZob{}}
            \PYG{k}{this}\PYG{p}{.}\PYG{n+nx}{\PYGZdl{}el}\PYG{p}{.}\PYG{n+nx}{text}\PYG{p}{(}\PYG{k}{this}\PYG{p}{.}\PYG{n+nx}{get}\PYG{p}{(}\PYG{l+s+s2}{\PYGZdq{}value\PYGZdq{}}\PYG{p}{)}\PYG{p}{)}\PYG{p}{;}
        \PYG{p}{\PYGZcb{}} \PYG{k}{else} \PYG{p}{\PYGZob{}}
            \PYG{k}{this}\PYG{p}{.}\PYG{n+nx}{\PYGZdl{}}\PYG{p}{(}\PYG{l+s+s2}{\PYGZdq{}input\PYGZdq{}}\PYG{p}{)}\PYG{p}{.}\PYG{n+nx}{val}\PYG{p}{(}\PYG{k}{this}\PYG{p}{.}\PYG{n+nx}{get}\PYG{p}{(}\PYG{l+s+s2}{\PYGZdq{}value\PYGZdq{}}\PYG{p}{)}\PYG{p}{)}\PYG{p}{;}
        \PYG{p}{\PYGZcb{}}
    \PYG{p}{\PYGZcb{}}\PYG{p}{,}
\PYG{p}{\PYGZcb{}}\PYG{p}{)}\PYG{p}{;}

\PYG{n+nx}{instance}\PYG{p}{.}\PYG{n+nx}{web}\PYG{p}{.}\PYG{n+nx}{form}\PYG{p}{.}\PYG{n+nx}{widgets}\PYG{p}{.}\PYG{n+nx}{add}\PYG{p}{(}\PYG{l+s+s1}{\PYGZsq{}char2\PYGZsq{}}\PYG{p}{,} \PYG{l+s+s1}{\PYGZsq{}instance.oepetstore.FieldChar2\PYGZsq{}}\PYG{p}{)}\PYG{p}{;}
\end{sphinxVerbatim}

\fvset{hllines={, ,}}%
\begin{sphinxVerbatim}[commandchars=\\\{\}]
\PYG{n+nt}{\PYGZlt{}t} \PYG{n+na}{t\PYGZhy{}name=}\PYG{l+s}{\PYGZdq{}FieldChar2\PYGZdq{}}\PYG{n+nt}{\PYGZgt{}}
    \PYG{n+nt}{\PYGZlt{}div} \PYG{n+na}{class=}\PYG{l+s}{\PYGZdq{}oe\PYGZus{}field\PYGZus{}char2\PYGZdq{}}\PYG{n+nt}{\PYGZgt{}}
        \PYG{n+nt}{\PYGZlt{}t} \PYG{n+na}{t\PYGZhy{}if=}\PYG{l+s}{\PYGZdq{}! widget.get(\PYGZsq{}effective\PYGZus{}readonly\PYGZsq{})\PYGZdq{}}\PYG{n+nt}{\PYGZgt{}}
            \PYG{n+nt}{\PYGZlt{}input} \PYG{n+na}{type=}\PYG{l+s}{\PYGZdq{}text\PYGZdq{}}\PYG{n+nt}{\PYGZgt{}}\PYG{n+nt}{\PYGZlt{}/input\PYGZgt{}}
        \PYG{n+nt}{\PYGZlt{}/t\PYGZgt{}}
    \PYG{n+nt}{\PYGZlt{}/div\PYGZgt{}}
\PYG{n+nt}{\PYGZlt{}/t\PYGZgt{}}
\end{sphinxVerbatim}

In the \sphinxcode{\sphinxupquote{start()}} method (which is called immediately after a widget has been
appended to the DOM), we bind on the event \sphinxcode{\sphinxupquote{change:effective\_readonly}}. That
allows us to redisplay the field each time the widget property
\sphinxcode{\sphinxupquote{effective\_readonly}} changes. This event handler will call
\sphinxcode{\sphinxupquote{display\_field()}}, which is also called directly in \sphinxcode{\sphinxupquote{start()}}. This
\sphinxcode{\sphinxupquote{display\_field()}} was created specifically for this field, it’s not a method
defined in \sphinxcode{\sphinxupquote{AbstractField}} or any other class. We can use this method
to display the content of the field depending on the current mode.

From now on the conception of this field is typical, except there is a
lot of verifications to know the state of the \sphinxcode{\sphinxupquote{effective\_readonly}} property:
\begin{itemize}
\item {} 
In the QWeb template used to display the content of the widget, it displays
an \sphinxcode{\sphinxupquote{\textless{}input type="text" /\textgreater{}}} if we are in read-write mode and nothing in
particular in read-only mode.

\item {} 
In the \sphinxcode{\sphinxupquote{display\_field()}} method, we have to bind on the \sphinxcode{\sphinxupquote{change}} event
of the \sphinxcode{\sphinxupquote{\textless{}input type="text" /\textgreater{}}} to know when the user has changed the
value. When it happens, we call the \sphinxcode{\sphinxupquote{internal\_set\_value()}} method with the
new value of the field. This is a convenience method provided by the
\sphinxcode{\sphinxupquote{AbstractField}} class. That method will set a new value in the \sphinxcode{\sphinxupquote{value}}
property but will not trigger a call to \sphinxcode{\sphinxupquote{render\_value()}} (which is not
necessary since the \sphinxcode{\sphinxupquote{\textless{}input type="text" /\textgreater{}}} already contains the correct
value).

\item {} 
In \sphinxcode{\sphinxupquote{render\_value()}}, we use a completely different code to display the
value of the field depending if we are in read-only or in read-write mode.

\end{itemize}

\begin{sphinxadmonition}{note}
Create a Color Field

Create a \sphinxcode{\sphinxupquote{FieldColor}} class. The value of this field should be a string
containing a color code like those used in CSS (example: \sphinxcode{\sphinxupquote{\#FF0000}} for
red). In read-only mode, this color field should display a little block
whose color corresponds to the value of the field. In read-write mode, you
should display an \sphinxcode{\sphinxupquote{\textless{}input type="color" /\textgreater{}}}. That type of \sphinxcode{\sphinxupquote{\textless{}input /\textgreater{}}}
is an HTML5 component that doesn’t work in all browsers but works well in
Google Chrome. So it’s OK to use as an exercise.

You can use that widget in the form view of the \sphinxcode{\sphinxupquote{message\_of\_the\_day}}
model for its field named \sphinxcode{\sphinxupquote{color}}. As a bonus, you can change the
\sphinxcode{\sphinxupquote{MessageOfTheDay}} widget created in the previous part of this guide to
display the message of the day with the background color indicated in the
\sphinxcode{\sphinxupquote{color}} field.

\fvset{hllines={, ,}}%
\begin{sphinxVerbatim}[commandchars=\\\{\}]
\PYG{n+nx}{local}\PYG{p}{.}\PYG{n+nx}{FieldColor} \PYG{o}{=} \PYG{n+nx}{instance}\PYG{p}{.}\PYG{n+nx}{web}\PYG{p}{.}\PYG{n+nx}{form}\PYG{p}{.}\PYG{n+nx}{AbstractField}\PYG{p}{.}\PYG{n+nx}{extend}\PYG{p}{(}\PYG{p}{\PYGZob{}}
    \PYG{n+nx}{events}\PYG{o}{:} \PYG{p}{\PYGZob{}}
        \PYG{l+s+s1}{\PYGZsq{}change input\PYGZsq{}}\PYG{o}{:} \PYG{k+kd}{function} \PYG{p}{(}\PYG{n+nx}{e}\PYG{p}{)} \PYG{p}{\PYGZob{}}
            \PYG{k}{if} \PYG{p}{(}\PYG{o}{!}\PYG{k}{this}\PYG{p}{.}\PYG{n+nx}{get}\PYG{p}{(}\PYG{l+s+s1}{\PYGZsq{}effective\PYGZus{}readonly\PYGZsq{}}\PYG{p}{)}\PYG{p}{)} \PYG{p}{\PYGZob{}}
                \PYG{k}{this}\PYG{p}{.}\PYG{n+nx}{internal\PYGZus{}set\PYGZus{}value}\PYG{p}{(}\PYG{n+nx}{\PYGZdl{}}\PYG{p}{(}\PYG{n+nx}{e}\PYG{p}{.}\PYG{n+nx}{currentTarget}\PYG{p}{)}\PYG{p}{.}\PYG{n+nx}{val}\PYG{p}{(}\PYG{p}{)}\PYG{p}{)}\PYG{p}{;}
            \PYG{p}{\PYGZcb{}}
        \PYG{p}{\PYGZcb{}}
    \PYG{p}{\PYGZcb{}}\PYG{p}{,}
    \PYG{n+nx}{init}\PYG{o}{:} \PYG{k+kd}{function}\PYG{p}{(}\PYG{p}{)} \PYG{p}{\PYGZob{}}
        \PYG{k}{this}\PYG{p}{.}\PYG{n+nx}{\PYGZus{}super}\PYG{p}{.}\PYG{n+nx}{apply}\PYG{p}{(}\PYG{k}{this}\PYG{p}{,} \PYG{n+nx}{arguments}\PYG{p}{)}\PYG{p}{;}
        \PYG{k}{this}\PYG{p}{.}\PYG{n+nx}{set}\PYG{p}{(}\PYG{l+s+s2}{\PYGZdq{}value\PYGZdq{}}\PYG{p}{,} \PYG{l+s+s2}{\PYGZdq{}\PYGZdq{}}\PYG{p}{)}\PYG{p}{;}
    \PYG{p}{\PYGZcb{}}\PYG{p}{,}
    \PYG{n+nx}{start}\PYG{o}{:} \PYG{k+kd}{function}\PYG{p}{(}\PYG{p}{)} \PYG{p}{\PYGZob{}}
        \PYG{k}{this}\PYG{p}{.}\PYG{n+nx}{on}\PYG{p}{(}\PYG{l+s+s2}{\PYGZdq{}change:effective\PYGZus{}readonly\PYGZdq{}}\PYG{p}{,} \PYG{k}{this}\PYG{p}{,} \PYG{k+kd}{function}\PYG{p}{(}\PYG{p}{)} \PYG{p}{\PYGZob{}}
            \PYG{k}{this}\PYG{p}{.}\PYG{n+nx}{display\PYGZus{}field}\PYG{p}{(}\PYG{p}{)}\PYG{p}{;}
            \PYG{k}{this}\PYG{p}{.}\PYG{n+nx}{render\PYGZus{}value}\PYG{p}{(}\PYG{p}{)}\PYG{p}{;}
        \PYG{p}{\PYGZcb{}}\PYG{p}{)}\PYG{p}{;}
        \PYG{k}{this}\PYG{p}{.}\PYG{n+nx}{display\PYGZus{}field}\PYG{p}{(}\PYG{p}{)}\PYG{p}{;}
        \PYG{k}{return} \PYG{k}{this}\PYG{p}{.}\PYG{n+nx}{\PYGZus{}super}\PYG{p}{(}\PYG{p}{)}\PYG{p}{;}
    \PYG{p}{\PYGZcb{}}\PYG{p}{,}
    \PYG{n+nx}{display\PYGZus{}field}\PYG{o}{:} \PYG{k+kd}{function}\PYG{p}{(}\PYG{p}{)} \PYG{p}{\PYGZob{}}
        \PYG{k}{this}\PYG{p}{.}\PYG{n+nx}{\PYGZdl{}el}\PYG{p}{.}\PYG{n+nx}{html}\PYG{p}{(}\PYG{n+nx}{QWeb}\PYG{p}{.}\PYG{n+nx}{render}\PYG{p}{(}\PYG{l+s+s2}{\PYGZdq{}FieldColor\PYGZdq{}}\PYG{p}{,} \PYG{p}{\PYGZob{}}\PYG{n+nx}{widget}\PYG{o}{:} \PYG{k}{this}\PYG{p}{\PYGZcb{}}\PYG{p}{)}\PYG{p}{)}\PYG{p}{;}
    \PYG{p}{\PYGZcb{}}\PYG{p}{,}
    \PYG{n+nx}{render\PYGZus{}value}\PYG{o}{:} \PYG{k+kd}{function}\PYG{p}{(}\PYG{p}{)} \PYG{p}{\PYGZob{}}
        \PYG{k}{if} \PYG{p}{(}\PYG{k}{this}\PYG{p}{.}\PYG{n+nx}{get}\PYG{p}{(}\PYG{l+s+s2}{\PYGZdq{}effective\PYGZus{}readonly\PYGZdq{}}\PYG{p}{)}\PYG{p}{)} \PYG{p}{\PYGZob{}}
            \PYG{k}{this}\PYG{p}{.}\PYG{n+nx}{\PYGZdl{}}\PYG{p}{(}\PYG{l+s+s2}{\PYGZdq{}.oe\PYGZus{}field\PYGZus{}color\PYGZus{}content\PYGZdq{}}\PYG{p}{)}\PYG{p}{.}\PYG{n+nx}{css}\PYG{p}{(}\PYG{l+s+s2}{\PYGZdq{}background\PYGZhy{}color\PYGZdq{}}\PYG{p}{,} \PYG{k}{this}\PYG{p}{.}\PYG{n+nx}{get}\PYG{p}{(}\PYG{l+s+s2}{\PYGZdq{}value\PYGZdq{}}\PYG{p}{)} \PYG{o}{\textbar{}\textbar{}} \PYG{l+s+s2}{\PYGZdq{}\PYGZsh{}FFFFFF\PYGZdq{}}\PYG{p}{)}\PYG{p}{;}
        \PYG{p}{\PYGZcb{}} \PYG{k}{else} \PYG{p}{\PYGZob{}}
            \PYG{k}{this}\PYG{p}{.}\PYG{n+nx}{\PYGZdl{}}\PYG{p}{(}\PYG{l+s+s2}{\PYGZdq{}input\PYGZdq{}}\PYG{p}{)}\PYG{p}{.}\PYG{n+nx}{val}\PYG{p}{(}\PYG{k}{this}\PYG{p}{.}\PYG{n+nx}{get}\PYG{p}{(}\PYG{l+s+s2}{\PYGZdq{}value\PYGZdq{}}\PYG{p}{)} \PYG{o}{\textbar{}\textbar{}} \PYG{l+s+s2}{\PYGZdq{}\PYGZsh{}FFFFFF\PYGZdq{}}\PYG{p}{)}\PYG{p}{;}
        \PYG{p}{\PYGZcb{}}
    \PYG{p}{\PYGZcb{}}\PYG{p}{,}
\PYG{p}{\PYGZcb{}}\PYG{p}{)}\PYG{p}{;}
\PYG{n+nx}{instance}\PYG{p}{.}\PYG{n+nx}{web}\PYG{p}{.}\PYG{n+nx}{form}\PYG{p}{.}\PYG{n+nx}{widgets}\PYG{p}{.}\PYG{n+nx}{add}\PYG{p}{(}\PYG{l+s+s1}{\PYGZsq{}color\PYGZsq{}}\PYG{p}{,} \PYG{l+s+s1}{\PYGZsq{}instance.oepetstore.FieldColor\PYGZsq{}}\PYG{p}{)}\PYG{p}{;}
\end{sphinxVerbatim}

\fvset{hllines={, ,}}%
\begin{sphinxVerbatim}[commandchars=\\\{\}]
\PYG{n+nt}{\PYGZlt{}t} \PYG{n+na}{t\PYGZhy{}name=}\PYG{l+s}{\PYGZdq{}FieldColor\PYGZdq{}}\PYG{n+nt}{\PYGZgt{}}
    \PYG{n+nt}{\PYGZlt{}div} \PYG{n+na}{class=}\PYG{l+s}{\PYGZdq{}oe\PYGZus{}field\PYGZus{}color\PYGZdq{}}\PYG{n+nt}{\PYGZgt{}}
        \PYG{n+nt}{\PYGZlt{}t} \PYG{n+na}{t\PYGZhy{}if=}\PYG{l+s}{\PYGZdq{}widget.get(\PYGZsq{}effective\PYGZus{}readonly\PYGZsq{})\PYGZdq{}}\PYG{n+nt}{\PYGZgt{}}
            \PYG{n+nt}{\PYGZlt{}div} \PYG{n+na}{class=}\PYG{l+s}{\PYGZdq{}oe\PYGZus{}field\PYGZus{}color\PYGZus{}content\PYGZdq{}} \PYG{n+nt}{/\PYGZgt{}}
        \PYG{n+nt}{\PYGZlt{}/t\PYGZgt{}}
        \PYG{n+nt}{\PYGZlt{}t} \PYG{n+na}{t\PYGZhy{}if=}\PYG{l+s}{\PYGZdq{}! widget.get(\PYGZsq{}effective\PYGZus{}readonly\PYGZsq{})\PYGZdq{}}\PYG{n+nt}{\PYGZgt{}}
            \PYG{n+nt}{\PYGZlt{}input} \PYG{n+na}{type=}\PYG{l+s}{\PYGZdq{}color\PYGZdq{}}\PYG{n+nt}{\PYGZgt{}}\PYG{n+nt}{\PYGZlt{}/input\PYGZgt{}}
        \PYG{n+nt}{\PYGZlt{}/t\PYGZgt{}}
    \PYG{n+nt}{\PYGZlt{}/div\PYGZgt{}}
\PYG{n+nt}{\PYGZlt{}/t\PYGZgt{}}
\end{sphinxVerbatim}

\fvset{hllines={, ,}}%
\begin{sphinxVerbatim}[commandchars=\\\{\}]
\PYG{n+nc}{.oe\PYGZus{}field\PYGZus{}color\PYGZus{}content} \PYG{p}{\PYGZob{}}
    \PYG{n+nb}{height}\PYG{o}{:} \PYG{l+m}{20px}\PYG{p}{;}
    \PYG{n+nb}{width}\PYG{o}{:} \PYG{l+m}{50px}\PYG{p}{;}
    \PYG{n+nb}{border}\PYG{o}{:} \PYG{l+m}{1px} \PYG{n+nb}{solid} \PYG{n+nb}{black}\PYG{p}{;}
\PYG{p}{\PYGZcb{}}
\end{sphinxVerbatim}
\end{sphinxadmonition}


\subsubsection{The Form View Custom Widgets}
\label{\detokenize{howtos/web:the-form-view-custom-widgets}}
Form fields are used to edit a single field, and are intrinsically linked to
a field. Because this may be limiting, it is also possible to create
\sphinxstyleemphasis{form widgets} which are not so restricted and have less ties to a specific
lifecycle.

Custom form widgets can be added to a form view through the \sphinxcode{\sphinxupquote{widget}} tag:

\fvset{hllines={, ,}}%
\begin{sphinxVerbatim}[commandchars=\\\{\}]
\PYG{n+nt}{\PYGZlt{}widget} \PYG{n+na}{type=}\PYG{l+s}{\PYGZdq{}xxx\PYGZdq{}} \PYG{n+nt}{/\PYGZgt{}}
\end{sphinxVerbatim}

This type of widget will simply be created by the form view during the
creation of the HTML according to the XML definition. They have properties in
common with the fields (like the \sphinxcode{\sphinxupquote{effective\_readonly}} property) but they are
not assigned a precise field. And so they don’t have methods like
\sphinxcode{\sphinxupquote{get\_value()}} and \sphinxcode{\sphinxupquote{set\_value()}}. They must inherit from the \sphinxcode{\sphinxupquote{FormWidget}}
abstract class.

Form widgets can interact with form fields by listening for their changes and
fetching or altering their values. They can access form fields through
their \sphinxcode{\sphinxupquote{field\_manager}} attribute:

\fvset{hllines={, ,}}%
\begin{sphinxVerbatim}[commandchars=\\\{\}]
\PYG{n+nx}{local}\PYG{p}{.}\PYG{n+nx}{WidgetMultiplication} \PYG{o}{=} \PYG{n+nx}{instance}\PYG{p}{.}\PYG{n+nx}{web}\PYG{p}{.}\PYG{n+nx}{form}\PYG{p}{.}\PYG{n+nx}{FormWidget}\PYG{p}{.}\PYG{n+nx}{extend}\PYG{p}{(}\PYG{p}{\PYGZob{}}
    \PYG{n+nx}{start}\PYG{o}{:} \PYG{k+kd}{function}\PYG{p}{(}\PYG{p}{)} \PYG{p}{\PYGZob{}}
        \PYG{k}{this}\PYG{p}{.}\PYG{n+nx}{\PYGZus{}super}\PYG{p}{(}\PYG{p}{)}\PYG{p}{;}
        \PYG{k}{this}\PYG{p}{.}\PYG{n+nx}{field\PYGZus{}manager}\PYG{p}{.}\PYG{n+nx}{on}\PYG{p}{(}\PYG{l+s+s2}{\PYGZdq{}field\PYGZus{}changed:integer\PYGZus{}a\PYGZdq{}}\PYG{p}{,} \PYG{k}{this}\PYG{p}{,} \PYG{k}{this}\PYG{p}{.}\PYG{n+nx}{display\PYGZus{}result}\PYG{p}{)}\PYG{p}{;}
        \PYG{k}{this}\PYG{p}{.}\PYG{n+nx}{field\PYGZus{}manager}\PYG{p}{.}\PYG{n+nx}{on}\PYG{p}{(}\PYG{l+s+s2}{\PYGZdq{}field\PYGZus{}changed:integer\PYGZus{}b\PYGZdq{}}\PYG{p}{,} \PYG{k}{this}\PYG{p}{,} \PYG{k}{this}\PYG{p}{.}\PYG{n+nx}{display\PYGZus{}result}\PYG{p}{)}\PYG{p}{;}
        \PYG{k}{this}\PYG{p}{.}\PYG{n+nx}{display\PYGZus{}result}\PYG{p}{(}\PYG{p}{)}\PYG{p}{;}
    \PYG{p}{\PYGZcb{}}\PYG{p}{,}
    \PYG{n+nx}{display\PYGZus{}result}\PYG{o}{:} \PYG{k+kd}{function}\PYG{p}{(}\PYG{p}{)} \PYG{p}{\PYGZob{}}
        \PYG{k+kd}{var} \PYG{n+nx}{result} \PYG{o}{=} \PYG{k}{this}\PYG{p}{.}\PYG{n+nx}{field\PYGZus{}manager}\PYG{p}{.}\PYG{n+nx}{get\PYGZus{}field\PYGZus{}value}\PYG{p}{(}\PYG{l+s+s2}{\PYGZdq{}integer\PYGZus{}a\PYGZdq{}}\PYG{p}{)} \PYG{o}{*}
                     \PYG{k}{this}\PYG{p}{.}\PYG{n+nx}{field\PYGZus{}manager}\PYG{p}{.}\PYG{n+nx}{get\PYGZus{}field\PYGZus{}value}\PYG{p}{(}\PYG{l+s+s2}{\PYGZdq{}integer\PYGZus{}b\PYGZdq{}}\PYG{p}{)}\PYG{p}{;}
        \PYG{k}{this}\PYG{p}{.}\PYG{n+nx}{\PYGZdl{}el}\PYG{p}{.}\PYG{n+nx}{text}\PYG{p}{(}\PYG{l+s+s2}{\PYGZdq{}a*b = \PYGZdq{}} \PYG{o}{+} \PYG{n+nx}{result}\PYG{p}{)}\PYG{p}{;}
    \PYG{p}{\PYGZcb{}}
\PYG{p}{\PYGZcb{}}\PYG{p}{)}\PYG{p}{;}

\PYG{n+nx}{instance}\PYG{p}{.}\PYG{n+nx}{web}\PYG{p}{.}\PYG{n+nx}{form}\PYG{p}{.}\PYG{n+nx}{custom\PYGZus{}widgets}\PYG{p}{.}\PYG{n+nx}{add}\PYG{p}{(}\PYG{l+s+s1}{\PYGZsq{}multiplication\PYGZsq{}}\PYG{p}{,} \PYG{l+s+s1}{\PYGZsq{}instance.oepetstore.WidgetMultiplication\PYGZsq{}}\PYG{p}{)}\PYG{p}{;}
\end{sphinxVerbatim}

\sphinxcode{\sphinxupquote{FormWidget}} is generally the
\sphinxcode{\sphinxupquote{FormView()}} itself, but features used from it should
be limited to those defined by \sphinxcode{\sphinxupquote{FieldManagerMixin()}},
the most useful being:
\begin{itemize}
\item {} 
\sphinxcode{\sphinxupquote{get\_field\_value(field\_name)()}}
which returns the value of a field.

\item {} 
\sphinxcode{\sphinxupquote{set\_values(values)()}} sets multiple
field values, takes a mapping of \sphinxcode{\sphinxupquote{\{field\_name: value\_to\_set\}}}

\item {} 
An event \sphinxcode{\sphinxupquote{field\_changed:\sphinxstyleemphasis{field\_name}}} is triggered any time the value
of the field called \sphinxcode{\sphinxupquote{field\_name}} is changed

\end{itemize}

\begin{sphinxadmonition}{note}
Show Coordinates on Google Map

Add two fields to \sphinxcode{\sphinxupquote{product.product}} storing a latitude and a longitude,
then create a new form widget to display the latitude and longitude of
a product’s origin on a map

To display the map, use Google Map’s embedding:

\fvset{hllines={, ,}}%
\begin{sphinxVerbatim}[commandchars=\\\{\}]
\PYG{p}{\PYGZlt{}}\PYG{n+nt}{iframe} \PYG{n+na}{width}\PYG{o}{=}\PYG{l+s}{\PYGZdq{}400\PYGZdq{}} \PYG{n+na}{height}\PYG{o}{=}\PYG{l+s}{\PYGZdq{}300\PYGZdq{}} \PYG{n+na}{src}\PYG{o}{=}\PYG{l+s}{\PYGZdq{}https://maps.google.com/?ie=UTF8\PYGZam{}amp;ll=XXX,YYY\PYGZam{}amp;output=embed\PYGZdq{}}\PYG{p}{\PYGZgt{}}
\PYG{p}{\PYGZlt{}}\PYG{p}{/}\PYG{n+nt}{iframe}\PYG{p}{\PYGZgt{}}
\end{sphinxVerbatim}

where \sphinxcode{\sphinxupquote{XXX}} should be replaced by the latitude and \sphinxcode{\sphinxupquote{YYY}} by the
longitude.

Display the two position fields and a map widget using them in a new
notebook page of the product’s form view.

\fvset{hllines={, ,}}%
\begin{sphinxVerbatim}[commandchars=\\\{\}]
\PYG{n+nx}{local}\PYG{p}{.}\PYG{n+nx}{WidgetCoordinates} \PYG{o}{=} \PYG{n+nx}{instance}\PYG{p}{.}\PYG{n+nx}{web}\PYG{p}{.}\PYG{n+nx}{form}\PYG{p}{.}\PYG{n+nx}{FormWidget}\PYG{p}{.}\PYG{n+nx}{extend}\PYG{p}{(}\PYG{p}{\PYGZob{}}
    \PYG{n+nx}{start}\PYG{o}{:} \PYG{k+kd}{function}\PYG{p}{(}\PYG{p}{)} \PYG{p}{\PYGZob{}}
        \PYG{k}{this}\PYG{p}{.}\PYG{n+nx}{\PYGZus{}super}\PYG{p}{(}\PYG{p}{)}\PYG{p}{;}
        \PYG{k}{this}\PYG{p}{.}\PYG{n+nx}{field\PYGZus{}manager}\PYG{p}{.}\PYG{n+nx}{on}\PYG{p}{(}\PYG{l+s+s2}{\PYGZdq{}field\PYGZus{}changed:provider\PYGZus{}latitude\PYGZdq{}}\PYG{p}{,} \PYG{k}{this}\PYG{p}{,} \PYG{k}{this}\PYG{p}{.}\PYG{n+nx}{display\PYGZus{}map}\PYG{p}{)}\PYG{p}{;}
        \PYG{k}{this}\PYG{p}{.}\PYG{n+nx}{field\PYGZus{}manager}\PYG{p}{.}\PYG{n+nx}{on}\PYG{p}{(}\PYG{l+s+s2}{\PYGZdq{}field\PYGZus{}changed:provider\PYGZus{}longitude\PYGZdq{}}\PYG{p}{,} \PYG{k}{this}\PYG{p}{,} \PYG{k}{this}\PYG{p}{.}\PYG{n+nx}{display\PYGZus{}map}\PYG{p}{)}\PYG{p}{;}
        \PYG{k}{this}\PYG{p}{.}\PYG{n+nx}{display\PYGZus{}map}\PYG{p}{(}\PYG{p}{)}\PYG{p}{;}
    \PYG{p}{\PYGZcb{}}\PYG{p}{,}
    \PYG{n+nx}{display\PYGZus{}map}\PYG{o}{:} \PYG{k+kd}{function}\PYG{p}{(}\PYG{p}{)} \PYG{p}{\PYGZob{}}
        \PYG{k}{this}\PYG{p}{.}\PYG{n+nx}{\PYGZdl{}el}\PYG{p}{.}\PYG{n+nx}{html}\PYG{p}{(}\PYG{n+nx}{QWeb}\PYG{p}{.}\PYG{n+nx}{render}\PYG{p}{(}\PYG{l+s+s2}{\PYGZdq{}WidgetCoordinates\PYGZdq{}}\PYG{p}{,} \PYG{p}{\PYGZob{}}
            \PYG{l+s+s2}{\PYGZdq{}latitude\PYGZdq{}}\PYG{o}{:} \PYG{k}{this}\PYG{p}{.}\PYG{n+nx}{field\PYGZus{}manager}\PYG{p}{.}\PYG{n+nx}{get\PYGZus{}field\PYGZus{}value}\PYG{p}{(}\PYG{l+s+s2}{\PYGZdq{}provider\PYGZus{}latitude\PYGZdq{}}\PYG{p}{)} \PYG{o}{\textbar{}\textbar{}} \PYG{l+m+mi}{0}\PYG{p}{,}
            \PYG{l+s+s2}{\PYGZdq{}longitude\PYGZdq{}}\PYG{o}{:} \PYG{k}{this}\PYG{p}{.}\PYG{n+nx}{field\PYGZus{}manager}\PYG{p}{.}\PYG{n+nx}{get\PYGZus{}field\PYGZus{}value}\PYG{p}{(}\PYG{l+s+s2}{\PYGZdq{}provider\PYGZus{}longitude\PYGZdq{}}\PYG{p}{)} \PYG{o}{\textbar{}\textbar{}} \PYG{l+m+mi}{0}\PYG{p}{,}
        \PYG{p}{\PYGZcb{}}\PYG{p}{)}\PYG{p}{)}\PYG{p}{;}
    \PYG{p}{\PYGZcb{}}
\PYG{p}{\PYGZcb{}}\PYG{p}{)}\PYG{p}{;}

\PYG{n+nx}{instance}\PYG{p}{.}\PYG{n+nx}{web}\PYG{p}{.}\PYG{n+nx}{form}\PYG{p}{.}\PYG{n+nx}{custom\PYGZus{}widgets}\PYG{p}{.}\PYG{n+nx}{add}\PYG{p}{(}\PYG{l+s+s1}{\PYGZsq{}coordinates\PYGZsq{}}\PYG{p}{,} \PYG{l+s+s1}{\PYGZsq{}instance.oepetstore.WidgetCoordinates\PYGZsq{}}\PYG{p}{)}\PYG{p}{;}
\end{sphinxVerbatim}

\fvset{hllines={, ,}}%
\begin{sphinxVerbatim}[commandchars=\\\{\}]
\PYG{n+nt}{\PYGZlt{}t} \PYG{n+na}{t\PYGZhy{}name=}\PYG{l+s}{\PYGZdq{}WidgetCoordinates\PYGZdq{}}\PYG{n+nt}{\PYGZgt{}}
    \PYG{n+nt}{\PYGZlt{}iframe} \PYG{n+na}{width=}\PYG{l+s}{\PYGZdq{}400\PYGZdq{}} \PYG{n+na}{height=}\PYG{l+s}{\PYGZdq{}300\PYGZdq{}}
        \PYG{n+na}{t\PYGZhy{}attf\PYGZhy{}src=}\PYG{l+s}{\PYGZdq{}https://maps.google.com/?ie=UTF8\PYGZam{}amp;ll=\PYGZob{}\PYGZob{}latitude\PYGZcb{}\PYGZcb{},\PYGZob{}\PYGZob{}longitude\PYGZcb{}\PYGZcb{}\PYGZam{}amp;output=embed\PYGZdq{}}\PYG{n+nt}{\PYGZgt{}}
    \PYG{n+nt}{\PYGZlt{}/iframe\PYGZgt{}}
\PYG{n+nt}{\PYGZlt{}/t\PYGZgt{}}
\end{sphinxVerbatim}
\end{sphinxadmonition}

\begin{sphinxadmonition}{note}
Get the Current Coordinate

Add a button resetting the product’s coordinates to the location of the
user, you can get these coordinates using the
\sphinxhref{http://diveintohtml5.info/geolocation.html}{javascript geolocation API}.

Now we would like to display an additional button to automatically set the
coordinates to the location of the current user.

To get the coordinates of the user, an easy way is to use the geolocation
JavaScript API.  \sphinxhref{http://www.w3schools.com/html/html5\_geolocation.asp}{See the online documentation to know how to use it}.

Please also note that the user should not be able to
click on that button when the form view is in read-only mode. So, this
custom widget should handle correctly the \sphinxcode{\sphinxupquote{effective\_readonly}} property
just like any field. One way to do this would be to make the button
disappear when \sphinxcode{\sphinxupquote{effective\_readonly}} is true.

\fvset{hllines={, ,}}%
\begin{sphinxVerbatim}[commandchars=\\\{\}]
\PYG{n+nx}{local}\PYG{p}{.}\PYG{n+nx}{WidgetCoordinates} \PYG{o}{=} \PYG{n+nx}{instance}\PYG{p}{.}\PYG{n+nx}{web}\PYG{p}{.}\PYG{n+nx}{form}\PYG{p}{.}\PYG{n+nx}{FormWidget}\PYG{p}{.}\PYG{n+nx}{extend}\PYG{p}{(}\PYG{p}{\PYGZob{}}
    \PYG{n+nx}{events}\PYG{o}{:} \PYG{p}{\PYGZob{}}
        \PYG{l+s+s1}{\PYGZsq{}click button\PYGZsq{}}\PYG{o}{:} \PYG{k+kd}{function} \PYG{p}{(}\PYG{p}{)} \PYG{p}{\PYGZob{}}
            \PYG{n+nx}{navigator}\PYG{p}{.}\PYG{n+nx}{geolocation}\PYG{p}{.}\PYG{n+nx}{getCurrentPosition}\PYG{p}{(}
                \PYG{k}{this}\PYG{p}{.}\PYG{n+nx}{proxy}\PYG{p}{(}\PYG{l+s+s1}{\PYGZsq{}received\PYGZus{}position\PYGZsq{}}\PYG{p}{)}\PYG{p}{)}\PYG{p}{;}
        \PYG{p}{\PYGZcb{}}
    \PYG{p}{\PYGZcb{}}\PYG{p}{,}
    \PYG{n+nx}{start}\PYG{o}{:} \PYG{k+kd}{function}\PYG{p}{(}\PYG{p}{)} \PYG{p}{\PYGZob{}}
        \PYG{k+kd}{var} \PYG{n+nx}{sup} \PYG{o}{=} \PYG{k}{this}\PYG{p}{.}\PYG{n+nx}{\PYGZus{}super}\PYG{p}{(}\PYG{p}{)}\PYG{p}{;}
        \PYG{k}{this}\PYG{p}{.}\PYG{n+nx}{field\PYGZus{}manager}\PYG{p}{.}\PYG{n+nx}{on}\PYG{p}{(}\PYG{l+s+s2}{\PYGZdq{}field\PYGZus{}changed:provider\PYGZus{}latitude\PYGZdq{}}\PYG{p}{,} \PYG{k}{this}\PYG{p}{,} \PYG{k}{this}\PYG{p}{.}\PYG{n+nx}{display\PYGZus{}map}\PYG{p}{)}\PYG{p}{;}
        \PYG{k}{this}\PYG{p}{.}\PYG{n+nx}{field\PYGZus{}manager}\PYG{p}{.}\PYG{n+nx}{on}\PYG{p}{(}\PYG{l+s+s2}{\PYGZdq{}field\PYGZus{}changed:provider\PYGZus{}longitude\PYGZdq{}}\PYG{p}{,} \PYG{k}{this}\PYG{p}{,} \PYG{k}{this}\PYG{p}{.}\PYG{n+nx}{display\PYGZus{}map}\PYG{p}{)}\PYG{p}{;}
        \PYG{k}{this}\PYG{p}{.}\PYG{n+nx}{on}\PYG{p}{(}\PYG{l+s+s2}{\PYGZdq{}change:effective\PYGZus{}readonly\PYGZdq{}}\PYG{p}{,} \PYG{k}{this}\PYG{p}{,} \PYG{k}{this}\PYG{p}{.}\PYG{n+nx}{display\PYGZus{}map}\PYG{p}{)}\PYG{p}{;}
        \PYG{k}{this}\PYG{p}{.}\PYG{n+nx}{display\PYGZus{}map}\PYG{p}{(}\PYG{p}{)}\PYG{p}{;}
        \PYG{k}{return} \PYG{n+nx}{sup}\PYG{p}{;}
    \PYG{p}{\PYGZcb{}}\PYG{p}{,}
    \PYG{n+nx}{display\PYGZus{}map}\PYG{o}{:} \PYG{k+kd}{function}\PYG{p}{(}\PYG{p}{)} \PYG{p}{\PYGZob{}}
        \PYG{k}{this}\PYG{p}{.}\PYG{n+nx}{\PYGZdl{}el}\PYG{p}{.}\PYG{n+nx}{html}\PYG{p}{(}\PYG{n+nx}{QWeb}\PYG{p}{.}\PYG{n+nx}{render}\PYG{p}{(}\PYG{l+s+s2}{\PYGZdq{}WidgetCoordinates\PYGZdq{}}\PYG{p}{,} \PYG{p}{\PYGZob{}}
            \PYG{l+s+s2}{\PYGZdq{}latitude\PYGZdq{}}\PYG{o}{:} \PYG{k}{this}\PYG{p}{.}\PYG{n+nx}{field\PYGZus{}manager}\PYG{p}{.}\PYG{n+nx}{get\PYGZus{}field\PYGZus{}value}\PYG{p}{(}\PYG{l+s+s2}{\PYGZdq{}provider\PYGZus{}latitude\PYGZdq{}}\PYG{p}{)} \PYG{o}{\textbar{}\textbar{}} \PYG{l+m+mi}{0}\PYG{p}{,}
            \PYG{l+s+s2}{\PYGZdq{}longitude\PYGZdq{}}\PYG{o}{:} \PYG{k}{this}\PYG{p}{.}\PYG{n+nx}{field\PYGZus{}manager}\PYG{p}{.}\PYG{n+nx}{get\PYGZus{}field\PYGZus{}value}\PYG{p}{(}\PYG{l+s+s2}{\PYGZdq{}provider\PYGZus{}longitude\PYGZdq{}}\PYG{p}{)} \PYG{o}{\textbar{}\textbar{}} \PYG{l+m+mi}{0}\PYG{p}{,}
        \PYG{p}{\PYGZcb{}}\PYG{p}{)}\PYG{p}{)}\PYG{p}{;}
        \PYG{k}{this}\PYG{p}{.}\PYG{n+nx}{\PYGZdl{}}\PYG{p}{(}\PYG{l+s+s2}{\PYGZdq{}button\PYGZdq{}}\PYG{p}{)}\PYG{p}{.}\PYG{n+nx}{toggle}\PYG{p}{(}\PYG{o}{!} \PYG{k}{this}\PYG{p}{.}\PYG{n+nx}{get}\PYG{p}{(}\PYG{l+s+s2}{\PYGZdq{}effective\PYGZus{}readonly\PYGZdq{}}\PYG{p}{)}\PYG{p}{)}\PYG{p}{;}
    \PYG{p}{\PYGZcb{}}\PYG{p}{,}
    \PYG{n+nx}{received\PYGZus{}position}\PYG{o}{:} \PYG{k+kd}{function}\PYG{p}{(}\PYG{n+nx}{obj}\PYG{p}{)} \PYG{p}{\PYGZob{}}
        \PYG{k}{this}\PYG{p}{.}\PYG{n+nx}{field\PYGZus{}manager}\PYG{p}{.}\PYG{n+nx}{set\PYGZus{}values}\PYG{p}{(}\PYG{p}{\PYGZob{}}
            \PYG{l+s+s2}{\PYGZdq{}provider\PYGZus{}latitude\PYGZdq{}}\PYG{o}{:} \PYG{n+nx}{obj}\PYG{p}{.}\PYG{n+nx}{coords}\PYG{p}{.}\PYG{n+nx}{latitude}\PYG{p}{,}
            \PYG{l+s+s2}{\PYGZdq{}provider\PYGZus{}longitude\PYGZdq{}}\PYG{o}{:} \PYG{n+nx}{obj}\PYG{p}{.}\PYG{n+nx}{coords}\PYG{p}{.}\PYG{n+nx}{longitude}\PYG{p}{,}
        \PYG{p}{\PYGZcb{}}\PYG{p}{)}\PYG{p}{;}
    \PYG{p}{\PYGZcb{}}\PYG{p}{,}
\PYG{p}{\PYGZcb{}}\PYG{p}{)}\PYG{p}{;}

\PYG{n+nx}{instance}\PYG{p}{.}\PYG{n+nx}{web}\PYG{p}{.}\PYG{n+nx}{form}\PYG{p}{.}\PYG{n+nx}{custom\PYGZus{}widgets}\PYG{p}{.}\PYG{n+nx}{add}\PYG{p}{(}\PYG{l+s+s1}{\PYGZsq{}coordinates\PYGZsq{}}\PYG{p}{,} \PYG{l+s+s1}{\PYGZsq{}instance.oepetstore.WidgetCoordinates\PYGZsq{}}\PYG{p}{)}\PYG{p}{;}
\end{sphinxVerbatim}

\fvset{hllines={, ,}}%
\begin{sphinxVerbatim}[commandchars=\\\{\}]
\PYG{n+nt}{\PYGZlt{}t} \PYG{n+na}{t\PYGZhy{}name=}\PYG{l+s}{\PYGZdq{}WidgetCoordinates\PYGZdq{}}\PYG{n+nt}{\PYGZgt{}}
    \PYG{n+nt}{\PYGZlt{}iframe} \PYG{n+na}{width=}\PYG{l+s}{\PYGZdq{}400\PYGZdq{}} \PYG{n+na}{height=}\PYG{l+s}{\PYGZdq{}300\PYGZdq{}}
        \PYG{n+na}{t\PYGZhy{}attf\PYGZhy{}src=}\PYG{l+s}{\PYGZdq{}https://maps.google.com/?ie=UTF8\PYGZam{}amp;ll=\PYGZob{}\PYGZob{}latitude\PYGZcb{}\PYGZcb{},\PYGZob{}\PYGZob{}longitude\PYGZcb{}\PYGZcb{}\PYGZam{}amp;output=embed\PYGZdq{}}\PYG{n+nt}{\PYGZgt{}}
    \PYG{n+nt}{\PYGZlt{}/iframe\PYGZgt{}}
    \PYG{n+nt}{\PYGZlt{}button}\PYG{n+nt}{\PYGZgt{}}Get My Current Coordinate\PYG{n+nt}{\PYGZlt{}/button\PYGZgt{}}
\PYG{n+nt}{\PYGZlt{}/t\PYGZgt{}}
\end{sphinxVerbatim}
\end{sphinxadmonition}
\phantomsection\label{\detokenize{howtos/web:positional-arguments}}

\section{Profiling Odoo code}
\label{\detokenize{howtos/profilecode:profiling-odoo-code}}\label{\detokenize{howtos/profilecode::doc}}\label{\detokenize{howtos/profilecode:positional-arguments}}\label{\detokenize{howtos/profilecode:keyword-arguments}}
\begin{sphinxadmonition}{warning}{Warning:}
This tutorial requires {\hyperref[\detokenize{setup/install:setup-install}]{\sphinxcrossref{\DUrole{std,std-ref}{having installed Odoo}}}}
and {\hyperref[\detokenize{howtos/backend::doc}]{\sphinxcrossref{\DUrole{doc}{writing Odoo code}}}}
\end{sphinxadmonition}


\subsection{Graph a method}
\label{\detokenize{howtos/profilecode:graph-a-method}}
Odoo embeds a profiler of code. This embeded profiler output can be used to
generate a graph of calls triggered by the method, number of queries, percentage
of time taken in the method itself as well as time taken in method and it’s
sub-called methods.

\fvset{hllines={, ,}}%
\begin{sphinxVerbatim}[commandchars=\\\{\}]
\PYG{k+kn}{from} \PYG{n+nn}{odoo}\PYG{n+nn}{.}\PYG{n+nn}{tools}\PYG{n+nn}{.}\PYG{n+nn}{profiler} \PYG{k}{import} \PYG{n}{profile}
\PYG{p}{[}\PYG{o}{.}\PYG{o}{.}\PYG{o}{.}\PYG{p}{]}
\PYG{n+nd}{@profile}\PYG{p}{(}\PYG{l+s+s1}{\PYGZsq{}}\PYG{l+s+s1}{/temp/prof.profile}\PYG{l+s+s1}{\PYGZsq{}}\PYG{p}{)}
\PYG{n+nd}{@api}\PYG{o}{.}\PYG{n}{multi}
\PYG{k}{def} \PYG{n+nf}{mymethod}\PYG{p}{(}\PYG{o}{.}\PYG{o}{.}\PYG{o}{.}\PYG{p}{)}
\end{sphinxVerbatim}

This produce a file called /temp/prof.profile

A tool called \sphinxstyleemphasis{gprof2dot} will produce a graph with this result:

\fvset{hllines={, ,}}%
\begin{sphinxVerbatim}[commandchars=\\\{\}]
\PYG{n}{gprof2dot} \PYG{o}{\PYGZhy{}}\PYG{n}{f} \PYG{n}{pstats} \PYG{o}{\PYGZhy{}}\PYG{n}{o} \PYG{o}{/}\PYG{n}{temp}\PYG{o}{/}\PYG{n}{prof}\PYG{o}{.}\PYG{n}{xdot} \PYG{o}{/}\PYG{n}{temp}\PYG{o}{/}\PYG{n}{prof}\PYG{o}{.}\PYG{n}{profile}
\end{sphinxVerbatim}

A tool called \sphinxstyleemphasis{xdot} will display the resulting graph:

\fvset{hllines={, ,}}%
\begin{sphinxVerbatim}[commandchars=\\\{\}]
\PYG{n}{xdot} \PYG{o}{/}\PYG{n}{temp}\PYG{o}{/}\PYG{n}{prof}\PYG{o}{.}\PYG{n}{xdot}
\end{sphinxVerbatim}

The profiler can be also used without saving data in a file.

\fvset{hllines={, ,}}%
\begin{sphinxVerbatim}[commandchars=\\\{\}]
\PYG{n+nd}{@profile}
\PYG{n+nd}{@api}\PYG{o}{.}\PYG{n}{model}
\PYG{k}{def} \PYG{n+nf}{mymethod}\PYG{p}{(}\PYG{o}{.}\PYG{o}{.}\PYG{o}{.}\PYG{p}{)}\PYG{p}{:}
\end{sphinxVerbatim}

The statistics will be displayed into the logs once the method to be analysed is
completely reviewed.

\fvset{hllines={, ,}}%
\begin{sphinxVerbatim}[commandchars=\\\{\}]
\PYG{l+m+mi}{2018}\PYG{o}{\PYGZhy{}}\PYG{l+m+mi}{03}\PYG{o}{\PYGZhy{}}\PYG{l+m+mi}{28} \PYG{l+m+mi}{06}\PYG{p}{:}\PYG{l+m+mi}{18}\PYG{p}{:}\PYG{l+m+mi}{23}\PYG{p}{,}\PYG{l+m+mi}{196} \PYG{l+m+mi}{22878} \PYG{n}{INFO} \PYG{n}{openerp} \PYG{n}{odoo}\PYG{o}{.}\PYG{n}{tools}\PYG{o}{.}\PYG{n}{profiler}\PYG{p}{:}
\PYG{n}{calls}     \PYG{n}{queries}   \PYG{n}{ms}
\PYG{n}{project}\PYG{o}{.}\PYG{n}{task} \PYG{o}{\PYGZhy{}}\PYG{o}{\PYGZhy{}}\PYG{o}{\PYGZhy{}}\PYG{o}{\PYGZhy{}}\PYG{o}{\PYGZhy{}}\PYG{o}{\PYGZhy{}}\PYG{o}{\PYGZhy{}}\PYG{o}{\PYGZhy{}}\PYG{o}{\PYGZhy{}}\PYG{o}{\PYGZhy{}}\PYG{o}{\PYGZhy{}}\PYG{o}{\PYGZhy{}}\PYG{o}{\PYGZhy{}}\PYG{o}{\PYGZhy{}}\PYG{o}{\PYGZhy{}}\PYG{o}{\PYGZhy{}}\PYG{o}{\PYGZhy{}}\PYG{o}{\PYGZhy{}}\PYG{o}{\PYGZhy{}}\PYG{o}{\PYGZhy{}}\PYG{o}{\PYGZhy{}}\PYG{o}{\PYGZhy{}}\PYG{o}{\PYGZhy{}}\PYG{o}{\PYGZhy{}} \PYG{o}{/}\PYG{n}{home}\PYG{o}{/}\PYG{n}{odoo}\PYG{o}{/}\PYG{n}{src}\PYG{o}{/}\PYG{n}{odoo}\PYG{o}{/}\PYG{n}{addons}\PYG{o}{/}\PYG{n}{project}\PYG{o}{/}\PYG{n}{models}\PYG{o}{/}\PYG{n}{project}\PYG{o}{.}\PYG{n}{py}\PYG{p}{,} \PYG{l+m+mi}{638}

\PYG{l+m+mi}{1}         \PYG{l+m+mi}{0}         \PYG{l+m+mf}{0.02}          \PYG{n+nd}{@profile}
                                  \PYG{n+nd}{@api}\PYG{o}{.}\PYG{n}{model}
                                  \PYG{k}{def} \PYG{n+nf}{create}\PYG{p}{(}\PYG{n+nb+bp}{self}\PYG{p}{,} \PYG{n}{vals}\PYG{p}{)}\PYG{p}{:}
                                      \PYG{c+c1}{\PYGZsh{} context: no\PYGZus{}log, because subtype already handle this}
\PYG{l+m+mi}{1}         \PYG{l+m+mi}{0}         \PYG{l+m+mf}{0.01}              \PYG{n}{context} \PYG{o}{=} \PYG{n+nb}{dict}\PYG{p}{(}\PYG{n+nb+bp}{self}\PYG{o}{.}\PYG{n}{env}\PYG{o}{.}\PYG{n}{context}\PYG{p}{,} \PYG{n}{mail\PYGZus{}create\PYGZus{}nolog}\PYG{o}{=}\PYG{k+kc}{True}\PYG{p}{)}

                                      \PYG{c+c1}{\PYGZsh{} for default stage}
\PYG{l+m+mi}{1}         \PYG{l+m+mi}{0}         \PYG{l+m+mf}{0.01}              \PYG{k}{if} \PYG{n}{vals}\PYG{o}{.}\PYG{n}{get}\PYG{p}{(}\PYG{l+s+s1}{\PYGZsq{}}\PYG{l+s+s1}{project\PYGZus{}id}\PYG{l+s+s1}{\PYGZsq{}}\PYG{p}{)} \PYG{o+ow}{and} \PYG{o+ow}{not} \PYG{n}{context}\PYG{o}{.}\PYG{n}{get}\PYG{p}{(}\PYG{l+s+s1}{\PYGZsq{}}\PYG{l+s+s1}{default\PYGZus{}project\PYGZus{}id}\PYG{l+s+s1}{\PYGZsq{}}\PYG{p}{)}\PYG{p}{:}
                                          \PYG{n}{context}\PYG{p}{[}\PYG{l+s+s1}{\PYGZsq{}}\PYG{l+s+s1}{default\PYGZus{}project\PYGZus{}id}\PYG{l+s+s1}{\PYGZsq{}}\PYG{p}{]} \PYG{o}{=} \PYG{n}{vals}\PYG{o}{.}\PYG{n}{get}\PYG{p}{(}\PYG{l+s+s1}{\PYGZsq{}}\PYG{l+s+s1}{project\PYGZus{}id}\PYG{l+s+s1}{\PYGZsq{}}\PYG{p}{)}
                                      \PYG{c+c1}{\PYGZsh{} user\PYGZus{}id change: update date\PYGZus{}assign}
\PYG{l+m+mi}{1}         \PYG{l+m+mi}{0}         \PYG{l+m+mf}{0.01}              \PYG{k}{if} \PYG{n}{vals}\PYG{o}{.}\PYG{n}{get}\PYG{p}{(}\PYG{l+s+s1}{\PYGZsq{}}\PYG{l+s+s1}{user\PYGZus{}id}\PYG{l+s+s1}{\PYGZsq{}}\PYG{p}{)}\PYG{p}{:}
                                          \PYG{n}{vals}\PYG{p}{[}\PYG{l+s+s1}{\PYGZsq{}}\PYG{l+s+s1}{date\PYGZus{}assign}\PYG{l+s+s1}{\PYGZsq{}}\PYG{p}{]} \PYG{o}{=} \PYG{n}{fields}\PYG{o}{.}\PYG{n}{Datetime}\PYG{o}{.}\PYG{n}{now}\PYG{p}{(}\PYG{p}{)}
                                      \PYG{c+c1}{\PYGZsh{} Stage change: Update date\PYGZus{}end if folded stage}
\PYG{l+m+mi}{1}         \PYG{l+m+mi}{0}         \PYG{l+m+mf}{0.0}               \PYG{k}{if} \PYG{n}{vals}\PYG{o}{.}\PYG{n}{get}\PYG{p}{(}\PYG{l+s+s1}{\PYGZsq{}}\PYG{l+s+s1}{stage\PYGZus{}id}\PYG{l+s+s1}{\PYGZsq{}}\PYG{p}{)}\PYG{p}{:}
                                          \PYG{n}{vals}\PYG{o}{.}\PYG{n}{update}\PYG{p}{(}\PYG{n+nb+bp}{self}\PYG{o}{.}\PYG{n}{update\PYGZus{}date\PYGZus{}end}\PYG{p}{(}\PYG{n}{vals}\PYG{p}{[}\PYG{l+s+s1}{\PYGZsq{}}\PYG{l+s+s1}{stage\PYGZus{}id}\PYG{l+s+s1}{\PYGZsq{}}\PYG{p}{]}\PYG{p}{)}\PYG{p}{)}
\PYG{l+m+mi}{1}         \PYG{l+m+mi}{108}       \PYG{l+m+mf}{631.8}             \PYG{n}{task} \PYG{o}{=} \PYG{n+nb}{super}\PYG{p}{(}\PYG{n}{Task}\PYG{p}{,} \PYG{n+nb+bp}{self}\PYG{o}{.}\PYG{n}{with\PYGZus{}context}\PYG{p}{(}\PYG{n}{context}\PYG{p}{)}\PYG{p}{)}\PYG{o}{.}\PYG{n}{create}\PYG{p}{(}\PYG{n}{vals}\PYG{p}{)}
\PYG{l+m+mi}{1}         \PYG{l+m+mi}{0}         \PYG{l+m+mf}{0.01}              \PYG{k}{return} \PYG{n}{task}

\PYG{n}{Total}\PYG{p}{:}
\PYG{l+m+mi}{1}         \PYG{l+m+mi}{108}       \PYG{l+m+mf}{631.85}
\end{sphinxVerbatim}


\subsection{Dump stack}
\label{\detokenize{howtos/profilecode:dump-stack}}
Sending the SIGQUIT signal to an odoo process (only available on POSIX) makes
this process output the current stack trace to log, with info level. When an
odoo process seems stucked, sending this signal to the process permit to know
what the process is doing, and letting the process continue his job.


\subsection{Tracing code execution}
\label{\detokenize{howtos/profilecode:tracing-code-execution}}
Instead of sending the SIGQUIT signal to an odoo process often enough, to check
where processes is performing worse than expected, we can use pyflame tool to
do it for us.


\subsubsection{Install pyflame and flamegraph}
\label{\detokenize{howtos/profilecode:install-pyflame-and-flamegraph}}
\fvset{hllines={, ,}}%
\begin{sphinxVerbatim}[commandchars=\\\{\}]
\PYG{c+c1}{\PYGZsh{} These instructions are given for Debian/Ubuntu distributions}
\PYG{n}{sudo} \PYG{n}{apt} \PYG{n}{install} \PYG{n}{autoconf} \PYG{n}{automake} \PYG{n}{autotools}\PYG{o}{\PYGZhy{}}\PYG{n}{dev} \PYG{n}{g}\PYG{o}{+}\PYG{o}{+} \PYG{n}{pkg}\PYG{o}{\PYGZhy{}}\PYG{n}{config} \PYG{n}{python}\PYG{o}{\PYGZhy{}}\PYG{n}{dev} \PYG{n}{python3}\PYG{o}{\PYGZhy{}}\PYG{n}{dev} \PYG{n}{libtool} \PYG{n}{make}
\PYG{n}{git} \PYG{n}{clone} \PYG{n}{https}\PYG{p}{:}\PYG{o}{/}\PYG{o}{/}\PYG{n}{github}\PYG{o}{.}\PYG{n}{com}\PYG{o}{/}\PYG{n}{uber}\PYG{o}{/}\PYG{n}{pyflame}\PYG{o}{.}\PYG{n}{git}
\PYG{n}{git} \PYG{n}{clone} \PYG{n}{https}\PYG{p}{:}\PYG{o}{/}\PYG{o}{/}\PYG{n}{github}\PYG{o}{.}\PYG{n}{com}\PYG{o}{/}\PYG{n}{brendangregg}\PYG{o}{/}\PYG{n}{FlameGraph}\PYG{o}{.}\PYG{n}{git}
\PYG{n}{cd} \PYG{n}{pyflame}
\PYG{o}{.}\PYG{o}{/}\PYG{n}{autogen}\PYG{o}{.}\PYG{n}{sh}
\PYG{o}{.}\PYG{o}{/}\PYG{n}{configure}
\PYG{n}{make}
\PYG{n}{sudo} \PYG{n}{make} \PYG{n}{install}
\end{sphinxVerbatim}


\subsubsection{Record executed code}
\label{\detokenize{howtos/profilecode:record-executed-code}}
As pyflame is installed, we now record the executed code lines with pyflame.
This tool will record, multiple times a second, the stacktrace of the process.
Once done, we’ll display them as an execution graph.

\fvset{hllines={, ,}}%
\begin{sphinxVerbatim}[commandchars=\\\{\}]
\PYG{n}{pyflame} \PYG{o}{\PYGZhy{}}\PYG{o}{\PYGZhy{}}\PYG{n}{exclude}\PYG{o}{\PYGZhy{}}\PYG{n}{idle} \PYG{o}{\PYGZhy{}}\PYG{n}{s} \PYG{l+m+mi}{3600} \PYG{o}{\PYGZhy{}}\PYG{n}{r} \PYG{l+m+mf}{0.2} \PYG{o}{\PYGZhy{}}\PYG{n}{p} \PYG{o}{\PYGZlt{}}\PYG{n}{PID}\PYG{o}{\PYGZgt{}} \PYG{o}{\PYGZhy{}}\PYG{n}{o} \PYG{n}{test}\PYG{o}{.}\PYG{n}{flame}
\end{sphinxVerbatim}

where \textless{}PID\textgreater{} is the process ID of the odoo process you want to graph. This will
wait until the dead of the process, with a maximum of one hour, and and get 5
traces a second. With the output of pyflame, we can produce an svg graph with
the flamegraph tool:

\fvset{hllines={, ,}}%
\begin{sphinxVerbatim}[commandchars=\\\{\}]
\PYG{n}{flamegraph}\PYG{o}{.}\PYG{n}{pl} \PYG{o}{.}\PYG{o}{/}\PYG{n}{test}\PYG{o}{.}\PYG{n}{flame} \PYG{o}{\PYGZgt{}} \PYG{o}{\PYGZti{}}\PYG{o}{/}\PYG{n}{mycode}\PYG{o}{.}\PYG{n}{svg}
\end{sphinxVerbatim}

\noindent\sphinxincludegraphics{{flamegraph}.svg}


\chapter{Web Services}
\label{\detokenize{webservices::doc}}\label{\detokenize{webservices:web-services}}

\section{External API}
\label{\detokenize{webservices/odoo:external-api}}\label{\detokenize{webservices/odoo::doc}}
Odoo is usually extended internally via modules, but many of its features and
all of its data are also available from the outside for external analysis or
integration with various tools. Part of the {\hyperref[\detokenize{reference/orm:reference-orm-model}]{\sphinxcrossref{\DUrole{std,std-ref}{Model Reference}}}} API is
easily available over \sphinxhref{http://en.wikipedia.org/wiki/XML-RPC}{XML-RPC} and accessible from a variety of languages.


\subsection{Connection}
\label{\detokenize{webservices/odoo:connection}}

\subsubsection{Configuration}
\label{\detokenize{webservices/odoo:configuration}}
If you already have an Odoo server installed, you can just use its
parameters

\begin{sphinxadmonition}{warning}{Warning:}
For Odoo Online instances (\textless{}domain\textgreater{}.odoo.com), users are created without a
\sphinxstyleemphasis{local} password (as a person you are logged in via the Odoo Online
authentication system, not by the instance itself). To use XML-RPC on Odoo
Online instances, you will need to set a password on the user account you
want to use:
\begin{itemize}
\item {} 
Log in your instance with an administrator account

\item {} 
Go to \sphinxmenuselection{Settings \(\rightarrow\) Users \(\rightarrow\) Users}

\item {} 
Click on the user you want to use for XML-RPC access

\item {} 
Click the \sphinxmenuselection{Change Password} button

\item {} 
Set a \sphinxmenuselection{New Password} value then click
\sphinxmenuselection{Change Password}.

\end{itemize}

The \sphinxstyleemphasis{server url} is the instance’s domain (e.g.
\sphinxstyleemphasis{https://mycompany.odoo.com}), the \sphinxstyleemphasis{database name} is the name of the
instance (e.g. \sphinxstyleemphasis{mycompany}). The \sphinxstyleemphasis{username} is the configured user’s login
as shown by the \sphinxstyleemphasis{Change Password} screen.
\end{sphinxadmonition}
\begin{itemize}
\item {} Python 2
\item {} Ruby
\item {} PHP
\item {} Java
\end{itemize}

\fvset{hllines={, ,}}%
\begin{sphinxVerbatim}[commandchars=\\\{\}]
\PYG{n}{url} \PYG{o}{=} \PYG{o}{\PYGZlt{}}\PYG{n}{insert} \PYG{n}{server} \PYG{n}{URL}\PYG{o}{\PYGZgt{}}
\PYG{n}{db} \PYG{o}{=} \PYG{o}{\PYGZlt{}}\PYG{n}{insert} \PYG{n}{database} \PYG{n}{name}\PYG{o}{\PYGZgt{}}
\PYG{n}{username} \PYG{o}{=} \PYG{l+s+s1}{\PYGZsq{}}\PYG{l+s+s1}{admin}\PYG{l+s+s1}{\PYGZsq{}}
\PYG{n}{password} \PYG{o}{=} \PYG{o}{\PYGZlt{}}\PYG{n}{insert} \PYG{n}{password} \PYG{k}{for} \PYG{n}{your} \PYG{n}{admin} \PYG{n}{user} \PYG{p}{(}\PYG{n}{default}\PYG{p}{:} \PYG{n}{admin}\PYG{p}{)}\PYG{o}{\PYGZgt{}}
\end{sphinxVerbatim}

\fvset{hllines={, ,}}%
\begin{sphinxVerbatim}[commandchars=\\\{\}]
\PYG{n}{url} \PYG{o}{=} \PYG{o}{\PYGZlt{}}\PYG{n}{insert} \PYG{n}{server} \PYG{n+no}{URL}\PYG{o}{\PYGZgt{}}
\PYG{n}{db} \PYG{o}{=} \PYG{o}{\PYGZlt{}}\PYG{n}{insert} \PYG{n}{database} \PYG{n+nb}{name}\PYG{o}{\PYGZgt{}}
\PYG{n}{username} \PYG{o}{=} \PYG{l+s+s2}{\PYGZdq{}}\PYG{l+s+s2}{admin}\PYG{l+s+s2}{\PYGZdq{}}
\PYG{n}{password} \PYG{o}{=} \PYG{o}{\PYGZlt{}}\PYG{n}{insert} \PYG{n}{password} \PYG{k}{for} \PYG{n}{your} \PYG{n}{admin} \PYG{n}{user} \PYG{p}{(}\PYG{l+s+ss}{default}\PYG{p}{:} \PYG{n}{admin}\PYG{p}{)}\PYG{o}{\PYGZgt{}}
\end{sphinxVerbatim}

\fvset{hllines={, ,}}%
\begin{sphinxVerbatim}[commandchars=\\\{\}]
\PYG{n+nv}{\PYGZdl{}url} \PYG{o}{=} \PYG{o}{\PYGZlt{}}\PYG{n+nx}{insert} \PYG{n+nx}{server} \PYG{n+nx}{URL}\PYG{o}{\PYGZgt{}}\PYG{p}{;}
\PYG{n+nv}{\PYGZdl{}db} \PYG{o}{=} \PYG{o}{\PYGZlt{}}\PYG{n+nx}{insert} \PYG{n+nx}{database} \PYG{n+nx}{name}\PYG{o}{\PYGZgt{}}\PYG{p}{;}
\PYG{n+nv}{\PYGZdl{}username} \PYG{o}{=} \PYG{l+s+s2}{\PYGZdq{}}\PYG{l+s+s2}{admin}\PYG{l+s+s2}{\PYGZdq{}}\PYG{p}{;}
\PYG{n+nv}{\PYGZdl{}password} \PYG{o}{=} \PYG{o}{\PYGZlt{}}\PYG{n+nx}{insert} \PYG{n+nx}{password} \PYG{k}{for} \PYG{n+nx}{your} \PYG{n+nx}{admin} \PYG{n+nx}{user} \PYG{p}{(}\PYG{k}{default}\PYG{o}{:} \PYG{n+nx}{admin}\PYG{p}{)}\PYG{o}{\PYGZgt{}}\PYG{p}{;}
\end{sphinxVerbatim}

\fvset{hllines={, ,}}%
\begin{sphinxVerbatim}[commandchars=\\\{\}]
\PYG{k+kd}{final} \PYG{n}{String} \PYG{n}{url} \PYG{o}{=} \PYG{o}{\PYGZlt{}}\PYG{n}{insert} \PYG{n}{server} \PYG{n}{URL}\PYG{o}{\PYGZgt{}}\PYG{o}{,}
              \PYG{n}{db} \PYG{o}{=} \PYG{o}{\PYGZlt{}}\PYG{n}{insert} \PYG{n}{database} \PYG{n}{name}\PYG{o}{\PYGZgt{}}\PYG{o}{,}
        \PYG{n}{username} \PYG{o}{=} \PYG{l+s}{\PYGZdq{}admin\PYGZdq{}}\PYG{o}{,}
        \PYG{n}{password} \PYG{o}{=} \PYG{o}{\PYGZlt{}}\PYG{n}{insert} \PYG{n}{password} \PYG{k}{for} \PYG{n}{your} \PYG{n}{admin} \PYG{n+nf}{user} \PYG{o}{(}\PYG{k}{default}\PYG{o}{:} \PYG{n}{admin}\PYG{o}{)}\PYG{o}{\PYGZgt{}}\PYG{o}{;}
\end{sphinxVerbatim}


\paragraph{demo}
\label{\detokenize{webservices/odoo:demo}}
To make exploration simpler, you can also ask \sphinxurl{https://demo.odoo.com} for a test
database:
\begin{itemize}
\item {} Python 2
\item {} Ruby
\item {} PHP
\item {} Java
\end{itemize}

\fvset{hllines={, ,}}%
\begin{sphinxVerbatim}[commandchars=\\\{\}]
\PYG{k+kn}{import} \PYG{n+nn}{xmlrpclib}
\PYG{n}{info} \PYG{o}{=} \PYG{n}{xmlrpclib}\PYG{o}{.}\PYG{n}{ServerProxy}\PYG{p}{(}\PYG{l+s+s1}{\PYGZsq{}}\PYG{l+s+s1}{https://demo.odoo.com/start}\PYG{l+s+s1}{\PYGZsq{}}\PYG{p}{)}\PYG{o}{.}\PYG{n}{start}\PYG{p}{(}\PYG{p}{)}
\PYG{n}{url}\PYG{p}{,} \PYG{n}{db}\PYG{p}{,} \PYG{n}{username}\PYG{p}{,} \PYG{n}{password} \PYG{o}{=} \PYGZbs{}
    \PYG{n}{info}\PYG{p}{[}\PYG{l+s+s1}{\PYGZsq{}}\PYG{l+s+s1}{host}\PYG{l+s+s1}{\PYGZsq{}}\PYG{p}{]}\PYG{p}{,} \PYG{n}{info}\PYG{p}{[}\PYG{l+s+s1}{\PYGZsq{}}\PYG{l+s+s1}{database}\PYG{l+s+s1}{\PYGZsq{}}\PYG{p}{]}\PYG{p}{,} \PYG{n}{info}\PYG{p}{[}\PYG{l+s+s1}{\PYGZsq{}}\PYG{l+s+s1}{user}\PYG{l+s+s1}{\PYGZsq{}}\PYG{p}{]}\PYG{p}{,} \PYG{n}{info}\PYG{p}{[}\PYG{l+s+s1}{\PYGZsq{}}\PYG{l+s+s1}{password}\PYG{l+s+s1}{\PYGZsq{}}\PYG{p}{]}
\end{sphinxVerbatim}

\fvset{hllines={, ,}}%
\begin{sphinxVerbatim}[commandchars=\\\{\}]
\PYG{n+nb}{require} \PYG{l+s+s2}{\PYGZdq{}}\PYG{l+s+s2}{xmlrpc/client}\PYG{l+s+s2}{\PYGZdq{}}
\PYG{n}{info} \PYG{o}{=} \PYG{n+no}{XMLRPC}\PYG{o}{::}\PYG{n+no}{Client}\PYG{o}{.}\PYG{n}{new2}\PYG{p}{(}\PYG{l+s+s1}{\PYGZsq{}https://demo.odoo.com/start\PYGZsq{}}\PYG{p}{)}\PYG{o}{.}\PYG{n}{call}\PYG{p}{(}\PYG{l+s+s1}{\PYGZsq{}start\PYGZsq{}}\PYG{p}{)}
\PYG{n}{url}\PYG{p}{,} \PYG{n}{db}\PYG{p}{,} \PYG{n}{username}\PYG{p}{,} \PYG{n}{password} \PYG{o}{=} \PYG{p}{\PYGZbs{}}
    \PYG{n}{info}\PYG{o}{[}\PYG{l+s+s1}{\PYGZsq{}host\PYGZsq{}}\PYG{o}{]}\PYG{p}{,} \PYG{n}{info}\PYG{o}{[}\PYG{l+s+s1}{\PYGZsq{}database\PYGZsq{}}\PYG{o}{]}\PYG{p}{,} \PYG{n}{info}\PYG{o}{[}\PYG{l+s+s1}{\PYGZsq{}user\PYGZsq{}}\PYG{o}{]}\PYG{p}{,} \PYG{n}{info}\PYG{o}{[}\PYG{l+s+s1}{\PYGZsq{}password\PYGZsq{}}\PYG{o}{]}
\end{sphinxVerbatim}

\fvset{hllines={, ,}}%
\begin{sphinxVerbatim}[commandchars=\\\{\}]
\PYG{k}{require\PYGZus{}once}\PYG{p}{(}\PYG{l+s+s1}{\PYGZsq{}ripcord.php\PYGZsq{}}\PYG{p}{);}
\PYG{n+nv}{\PYGZdl{}info} \PYG{o}{=} \PYG{n+nx}{ripcord}\PYG{o}{::}\PYG{n+na}{client}\PYG{p}{(}\PYG{l+s+s1}{\PYGZsq{}https://demo.odoo.com/start\PYGZsq{}}\PYG{p}{)}\PYG{o}{\PYGZhy{}\PYGZgt{}}\PYG{n+na}{start}\PYG{p}{();}
\PYG{k}{list}\PYG{p}{(}\PYG{n+nv}{\PYGZdl{}url}\PYG{p}{,} \PYG{n+nv}{\PYGZdl{}db}\PYG{p}{,} \PYG{n+nv}{\PYGZdl{}username}\PYG{p}{,} \PYG{n+nv}{\PYGZdl{}password}\PYG{p}{)} \PYG{o}{=}
  \PYG{k}{array}\PYG{p}{(}\PYG{n+nv}{\PYGZdl{}info}\PYG{p}{[}\PYG{l+s+s1}{\PYGZsq{}host\PYGZsq{}}\PYG{p}{],} \PYG{n+nv}{\PYGZdl{}info}\PYG{p}{[}\PYG{l+s+s1}{\PYGZsq{}database\PYGZsq{}}\PYG{p}{],} \PYG{n+nv}{\PYGZdl{}info}\PYG{p}{[}\PYG{l+s+s1}{\PYGZsq{}user\PYGZsq{}}\PYG{p}{],} \PYG{n+nv}{\PYGZdl{}info}\PYG{p}{[}\PYG{l+s+s1}{\PYGZsq{}password\PYGZsq{}}\PYG{p}{]);}
\end{sphinxVerbatim}

\begin{sphinxadmonition}{note}{Note:}
These examples use the \sphinxhref{https://code.google.com/p/ripcord/}{Ripcord}
library, which provides a simple XML-RPC API. Ripcord requires that
\sphinxhref{http://php.net/manual/en/xmlrpc.installation.php}{XML-RPC support be enabled} in your PHP
installation.

Since calls are performed over
\sphinxhref{http://en.wikipedia.org/wiki/HTTP\_Secure}{HTTPS}, it also requires that
the \sphinxhref{http://php.net/manual/en/openssl.installation.php}{OpenSSL extension} be enabled.
\end{sphinxadmonition}

\fvset{hllines={, ,}}%
\begin{sphinxVerbatim}[commandchars=\\\{\}]
\PYG{k+kd}{final} \PYG{n}{XmlRpcClient} \PYG{n}{client} \PYG{o}{=} \PYG{k}{new} \PYG{n}{XmlRpcClient}\PYG{o}{(}\PYG{o}{)}\PYG{o}{;}

\PYG{k+kd}{final} \PYG{n}{XmlRpcClientConfigImpl} \PYG{n}{start\PYGZus{}config} \PYG{o}{=} \PYG{k}{new} \PYG{n}{XmlRpcClientConfigImpl}\PYG{o}{(}\PYG{o}{)}\PYG{o}{;}
\PYG{n}{start\PYGZus{}config}\PYG{o}{.}\PYG{n+na}{setServerURL}\PYG{o}{(}\PYG{k}{new} \PYG{n}{URL}\PYG{o}{(}\PYG{l+s}{\PYGZdq{}https://demo.odoo.com/start\PYGZdq{}}\PYG{o}{)}\PYG{o}{)}\PYG{o}{;}
\PYG{k+kd}{final} \PYG{n}{Map}\PYG{o}{\PYGZlt{}}\PYG{n}{String}\PYG{o}{,} \PYG{n}{String}\PYG{o}{\PYGZgt{}} \PYG{n}{info} \PYG{o}{=} \PYG{o}{(}\PYG{n}{Map}\PYG{o}{\PYGZlt{}}\PYG{n}{String}\PYG{o}{,} \PYG{n}{String}\PYG{o}{\PYGZgt{}}\PYG{o}{)}\PYG{n}{client}\PYG{o}{.}\PYG{n+na}{execute}\PYG{o}{(}
    \PYG{n}{start\PYGZus{}config}\PYG{o}{,} \PYG{l+s}{\PYGZdq{}start\PYGZdq{}}\PYG{o}{,} \PYG{n}{emptyList}\PYG{o}{(}\PYG{o}{)}\PYG{o}{)}\PYG{o}{;}

\PYG{k+kd}{final} \PYG{n}{String} \PYG{n}{url} \PYG{o}{=} \PYG{n}{info}\PYG{o}{.}\PYG{n+na}{get}\PYG{o}{(}\PYG{l+s}{\PYGZdq{}host\PYGZdq{}}\PYG{o}{)}\PYG{o}{,}
              \PYG{n}{db} \PYG{o}{=} \PYG{n}{info}\PYG{o}{.}\PYG{n+na}{get}\PYG{o}{(}\PYG{l+s}{\PYGZdq{}database\PYGZdq{}}\PYG{o}{)}\PYG{o}{,}
        \PYG{n}{username} \PYG{o}{=} \PYG{n}{info}\PYG{o}{.}\PYG{n+na}{get}\PYG{o}{(}\PYG{l+s}{\PYGZdq{}user\PYGZdq{}}\PYG{o}{)}\PYG{o}{,}
        \PYG{n}{password} \PYG{o}{=} \PYG{n}{info}\PYG{o}{.}\PYG{n+na}{get}\PYG{o}{(}\PYG{l+s}{\PYGZdq{}password\PYGZdq{}}\PYG{o}{)}\PYG{o}{;}
\end{sphinxVerbatim}

\begin{sphinxadmonition}{note}{Note:}
These examples use the \sphinxhref{https://ws.apache.org/xmlrpc/}{Apache XML-RPC library}

The examples do not include imports as these imports couldn’t be
pasted in the code.
\end{sphinxadmonition}


\subsubsection{Logging in}
\label{\detokenize{webservices/odoo:logging-in}}
Odoo requires users of the API to be authenticated before they can query most
data.

The \sphinxcode{\sphinxupquote{xmlrpc/2/common}} endpoint provides meta-calls which don’t require
authentication, such as the authentication itself or fetching version
information. To verify if the connection information is correct before trying
to authenticate, the simplest call is to ask for the server’s version. The
authentication itself is done through the \sphinxcode{\sphinxupquote{authenticate}} function and
returns a user identifier (\sphinxcode{\sphinxupquote{uid}}) used in authenticated calls instead of
the login.
\begin{itemize}
\item {} Python 2
\item {} Ruby
\item {} PHP
\item {} Java
\end{itemize}

\fvset{hllines={, ,}}%
\begin{sphinxVerbatim}[commandchars=\\\{\}]
\PYG{n}{common} \PYG{o}{=} \PYG{n}{xmlrpclib}\PYG{o}{.}\PYG{n}{ServerProxy}\PYG{p}{(}\PYG{l+s+s1}{\PYGZsq{}}\PYG{l+s+s1}{\PYGZob{}\PYGZcb{}/xmlrpc/2/common}\PYG{l+s+s1}{\PYGZsq{}}\PYG{o}{.}\PYG{n}{format}\PYG{p}{(}\PYG{n}{url}\PYG{p}{)}\PYG{p}{)}
\PYG{n}{common}\PYG{o}{.}\PYG{n}{version}\PYG{p}{(}\PYG{p}{)}
\end{sphinxVerbatim}

\fvset{hllines={, ,}}%
\begin{sphinxVerbatim}[commandchars=\\\{\}]
\PYG{n}{common} \PYG{o}{=} \PYG{n+no}{XMLRPC}\PYG{o}{::}\PYG{n+no}{Client}\PYG{o}{.}\PYG{n}{new2}\PYG{p}{(}\PYG{l+s+s2}{\PYGZdq{}}\PYG{l+s+si}{\PYGZsh{}\PYGZob{}}\PYG{n}{url}\PYG{l+s+si}{\PYGZcb{}}\PYG{l+s+s2}{/xmlrpc/2/common}\PYG{l+s+s2}{\PYGZdq{}}\PYG{p}{)}
\PYG{n}{common}\PYG{o}{.}\PYG{n}{call}\PYG{p}{(}\PYG{l+s+s1}{\PYGZsq{}version\PYGZsq{}}\PYG{p}{)}
\end{sphinxVerbatim}

\fvset{hllines={, ,}}%
\begin{sphinxVerbatim}[commandchars=\\\{\}]
\PYG{n+nv}{\PYGZdl{}common} \PYG{o}{=} \PYG{n+nx}{ripcord}\PYG{o}{::}\PYG{n+na}{client}\PYG{p}{(}\PYG{l+s+s2}{\PYGZdq{}}\PYG{l+s+si}{\PYGZdl{}url}\PYG{l+s+s2}{/xmlrpc/2/common}\PYG{l+s+s2}{\PYGZdq{}}\PYG{p}{);}
\PYG{n+nv}{\PYGZdl{}common}\PYG{o}{\PYGZhy{}\PYGZgt{}}\PYG{n+na}{version}\PYG{p}{();}
\end{sphinxVerbatim}

\fvset{hllines={, ,}}%
\begin{sphinxVerbatim}[commandchars=\\\{\}]
\PYG{k+kd}{final} \PYG{n}{XmlRpcClientConfigImpl} \PYG{n}{common\PYGZus{}config} \PYG{o}{=} \PYG{k}{new} \PYG{n}{XmlRpcClientConfigImpl}\PYG{o}{(}\PYG{o}{)}\PYG{o}{;}
\PYG{n}{common\PYGZus{}config}\PYG{o}{.}\PYG{n+na}{setServerURL}\PYG{o}{(}
    \PYG{k}{new} \PYG{n}{URL}\PYG{o}{(}\PYG{n}{String}\PYG{o}{.}\PYG{n+na}{format}\PYG{o}{(}\PYG{l+s}{\PYGZdq{}\PYGZpc{}s/xmlrpc/2/common\PYGZdq{}}\PYG{o}{,} \PYG{n}{url}\PYG{o}{)}\PYG{o}{)}\PYG{o}{)}\PYG{o}{;}
\PYG{n}{client}\PYG{o}{.}\PYG{n+na}{execute}\PYG{o}{(}\PYG{n}{common\PYGZus{}config}\PYG{o}{,} \PYG{l+s}{\PYGZdq{}version\PYGZdq{}}\PYG{o}{,} \PYG{n}{emptyList}\PYG{o}{(}\PYG{o}{)}\PYG{o}{)}\PYG{o}{;}
\end{sphinxVerbatim}

\fvset{hllines={, ,}}%
\begin{sphinxVerbatim}[commandchars=\\\{\}]
\PYG{p}{\PYGZob{}}
    \PYG{n+nt}{\PYGZdq{}server\PYGZus{}version\PYGZdq{}}\PYG{p}{:} \PYG{l+s+s2}{\PYGZdq{}8.0\PYGZdq{}}\PYG{p}{,}
    \PYG{n+nt}{\PYGZdq{}server\PYGZus{}version\PYGZus{}info\PYGZdq{}}\PYG{p}{:} \PYG{p}{[}\PYG{l+m+mi}{8}\PYG{p}{,} \PYG{l+m+mi}{0}\PYG{p}{,} \PYG{l+m+mi}{0}\PYG{p}{,} \PYG{l+s+s2}{\PYGZdq{}final\PYGZdq{}}\PYG{p}{,} \PYG{l+m+mi}{0}\PYG{p}{]}\PYG{p}{,}
    \PYG{n+nt}{\PYGZdq{}server\PYGZus{}serie\PYGZdq{}}\PYG{p}{:} \PYG{l+s+s2}{\PYGZdq{}8.0\PYGZdq{}}\PYG{p}{,}
    \PYG{n+nt}{\PYGZdq{}protocol\PYGZus{}version\PYGZdq{}}\PYG{p}{:} \PYG{l+m+mi}{1}\PYG{p}{,}
\PYG{p}{\PYGZcb{}}
\end{sphinxVerbatim}
\begin{itemize}
\item {} Python 2
\item {} Ruby
\item {} PHP
\item {} Java
\end{itemize}

\fvset{hllines={, ,}}%
\begin{sphinxVerbatim}[commandchars=\\\{\}]
\PYG{n}{uid} \PYG{o}{=} \PYG{n}{common}\PYG{o}{.}\PYG{n}{authenticate}\PYG{p}{(}\PYG{n}{db}\PYG{p}{,} \PYG{n}{username}\PYG{p}{,} \PYG{n}{password}\PYG{p}{,} \PYG{p}{\PYGZob{}}\PYG{p}{\PYGZcb{}}\PYG{p}{)}
\end{sphinxVerbatim}

\fvset{hllines={, ,}}%
\begin{sphinxVerbatim}[commandchars=\\\{\}]
\PYG{n}{uid} \PYG{o}{=} \PYG{n}{common}\PYG{o}{.}\PYG{n}{call}\PYG{p}{(}\PYG{l+s+s1}{\PYGZsq{}authenticate\PYGZsq{}}\PYG{p}{,} \PYG{n}{db}\PYG{p}{,} \PYG{n}{username}\PYG{p}{,} \PYG{n}{password}\PYG{p}{,} \PYG{p}{\PYGZob{}}\PYG{p}{\PYGZcb{}}\PYG{p}{)}
\end{sphinxVerbatim}

\fvset{hllines={, ,}}%
\begin{sphinxVerbatim}[commandchars=\\\{\}]
\PYG{n+nv}{\PYGZdl{}uid} \PYG{o}{=} \PYG{n+nv}{\PYGZdl{}common}\PYG{o}{\PYGZhy{}\PYGZgt{}}\PYG{n+na}{authenticate}\PYG{p}{(}\PYG{n+nv}{\PYGZdl{}db}\PYG{p}{,} \PYG{n+nv}{\PYGZdl{}username}\PYG{p}{,} \PYG{n+nv}{\PYGZdl{}password}\PYG{p}{,} \PYG{k}{array}\PYG{p}{());}
\end{sphinxVerbatim}

\fvset{hllines={, ,}}%
\begin{sphinxVerbatim}[commandchars=\\\{\}]
\PYG{k+kt}{int} \PYG{n}{uid} \PYG{o}{=} \PYG{o}{(}\PYG{k+kt}{int}\PYG{o}{)}\PYG{n}{client}\PYG{o}{.}\PYG{n+na}{execute}\PYG{o}{(}
    \PYG{n}{common\PYGZus{}config}\PYG{o}{,} \PYG{l+s}{\PYGZdq{}authenticate\PYGZdq{}}\PYG{o}{,} \PYG{n}{asList}\PYG{o}{(}
        \PYG{n}{db}\PYG{o}{,} \PYG{n}{username}\PYG{o}{,} \PYG{n}{password}\PYG{o}{,} \PYG{n}{emptyMap}\PYG{o}{(}\PYG{o}{)}\PYG{o}{)}\PYG{o}{)}\PYG{o}{;}
\end{sphinxVerbatim}


\subsection{Calling methods}
\label{\detokenize{webservices/odoo:calling-methods}}
The second endpoint is \sphinxcode{\sphinxupquote{xmlrpc/2/object}}, is used to call methods of odoo
models via the \sphinxcode{\sphinxupquote{execute\_kw}} RPC function.

Each call to \sphinxcode{\sphinxupquote{execute\_kw}} takes the following parameters:
\begin{itemize}
\item {} 
the database to use, a string

\item {} 
the user id (retrieved through \sphinxcode{\sphinxupquote{authenticate}}), an integer

\item {} 
the user’s password, a string

\item {} 
the model name, a string

\item {} 
the method name, a string

\item {} 
an array/list of parameters passed by position

\item {} 
a mapping/dict of parameters to pass by keyword (optional)

\end{itemize}

For instance to see if we can read the \sphinxcode{\sphinxupquote{res.partner}} model we can call
\sphinxcode{\sphinxupquote{check\_access\_rights}} with \sphinxcode{\sphinxupquote{operation}} passed by position and
\sphinxcode{\sphinxupquote{raise\_exception}} passed by keyword (in order to get a true/false result
rather than true/error):
\begin{itemize}
\item {} Python 2
\item {} Ruby
\item {} PHP
\item {} Java
\end{itemize}

\fvset{hllines={, ,}}%
\begin{sphinxVerbatim}[commandchars=\\\{\}]
\PYG{n}{models} \PYG{o}{=} \PYG{n}{xmlrpclib}\PYG{o}{.}\PYG{n}{ServerProxy}\PYG{p}{(}\PYG{l+s+s1}{\PYGZsq{}}\PYG{l+s+s1}{\PYGZob{}\PYGZcb{}/xmlrpc/2/object}\PYG{l+s+s1}{\PYGZsq{}}\PYG{o}{.}\PYG{n}{format}\PYG{p}{(}\PYG{n}{url}\PYG{p}{)}\PYG{p}{)}
\PYG{n}{models}\PYG{o}{.}\PYG{n}{execute\PYGZus{}kw}\PYG{p}{(}\PYG{n}{db}\PYG{p}{,} \PYG{n}{uid}\PYG{p}{,} \PYG{n}{password}\PYG{p}{,}
    \PYG{l+s+s1}{\PYGZsq{}}\PYG{l+s+s1}{res.partner}\PYG{l+s+s1}{\PYGZsq{}}\PYG{p}{,} \PYG{l+s+s1}{\PYGZsq{}}\PYG{l+s+s1}{check\PYGZus{}access\PYGZus{}rights}\PYG{l+s+s1}{\PYGZsq{}}\PYG{p}{,}
    \PYG{p}{[}\PYG{l+s+s1}{\PYGZsq{}}\PYG{l+s+s1}{read}\PYG{l+s+s1}{\PYGZsq{}}\PYG{p}{]}\PYG{p}{,} \PYG{p}{\PYGZob{}}\PYG{l+s+s1}{\PYGZsq{}}\PYG{l+s+s1}{raise\PYGZus{}exception}\PYG{l+s+s1}{\PYGZsq{}}\PYG{p}{:} \PYG{n+nb+bp}{False}\PYG{p}{\PYGZcb{}}\PYG{p}{)}
\end{sphinxVerbatim}

\fvset{hllines={, ,}}%
\begin{sphinxVerbatim}[commandchars=\\\{\}]
\PYG{n}{models} \PYG{o}{=} \PYG{n+no}{XMLRPC}\PYG{o}{::}\PYG{n+no}{Client}\PYG{o}{.}\PYG{n}{new2}\PYG{p}{(}\PYG{l+s+s2}{\PYGZdq{}}\PYG{l+s+si}{\PYGZsh{}\PYGZob{}}\PYG{n}{url}\PYG{l+s+si}{\PYGZcb{}}\PYG{l+s+s2}{/xmlrpc/2/object}\PYG{l+s+s2}{\PYGZdq{}}\PYG{p}{)}\PYG{o}{.}\PYG{n}{proxy}
\PYG{n}{models}\PYG{o}{.}\PYG{n}{execute\PYGZus{}kw}\PYG{p}{(}\PYG{n}{db}\PYG{p}{,} \PYG{n}{uid}\PYG{p}{,} \PYG{n}{password}\PYG{p}{,}
    \PYG{l+s+s1}{\PYGZsq{}res.partner\PYGZsq{}}\PYG{p}{,} \PYG{l+s+s1}{\PYGZsq{}check\PYGZus{}access\PYGZus{}rights\PYGZsq{}}\PYG{p}{,}
    \PYG{o}{[}\PYG{l+s+s1}{\PYGZsq{}read\PYGZsq{}}\PYG{o}{]}\PYG{p}{,} \PYG{p}{\PYGZob{}}\PYG{l+s+ss}{raise\PYGZus{}exception}\PYG{p}{:} \PYG{k+kp}{false}\PYG{p}{\PYGZcb{}}\PYG{p}{)}
\end{sphinxVerbatim}

\fvset{hllines={, ,}}%
\begin{sphinxVerbatim}[commandchars=\\\{\}]
\PYG{n+nv}{\PYGZdl{}models} \PYG{o}{=} \PYG{n+nx}{ripcord}\PYG{o}{::}\PYG{n+na}{client}\PYG{p}{(}\PYG{l+s+s2}{\PYGZdq{}}\PYG{l+s+si}{\PYGZdl{}url}\PYG{l+s+s2}{/xmlrpc/2/object}\PYG{l+s+s2}{\PYGZdq{}}\PYG{p}{);}
\PYG{n+nv}{\PYGZdl{}models}\PYG{o}{\PYGZhy{}\PYGZgt{}}\PYG{n+na}{execute\PYGZus{}kw}\PYG{p}{(}\PYG{n+nv}{\PYGZdl{}db}\PYG{p}{,} \PYG{n+nv}{\PYGZdl{}uid}\PYG{p}{,} \PYG{n+nv}{\PYGZdl{}password}\PYG{p}{,}
    \PYG{l+s+s1}{\PYGZsq{}res.partner\PYGZsq{}}\PYG{p}{,} \PYG{l+s+s1}{\PYGZsq{}check\PYGZus{}access\PYGZus{}rights\PYGZsq{}}\PYG{p}{,}
    \PYG{k}{array}\PYG{p}{(}\PYG{l+s+s1}{\PYGZsq{}read\PYGZsq{}}\PYG{p}{),} \PYG{k}{array}\PYG{p}{(}\PYG{l+s+s1}{\PYGZsq{}raise\PYGZus{}exception\PYGZsq{}} \PYG{o}{=\PYGZgt{}} \PYG{k}{false}\PYG{p}{));}
\end{sphinxVerbatim}

\fvset{hllines={, ,}}%
\begin{sphinxVerbatim}[commandchars=\\\{\}]
\PYG{k+kd}{final} \PYG{n}{XmlRpcClient} \PYG{n}{models} \PYG{o}{=} \PYG{k}{new} \PYG{n}{XmlRpcClient}\PYG{o}{(}\PYG{o}{)} \PYG{o}{\PYGZob{}}\PYG{o}{\PYGZob{}}
    \PYG{n}{setConfig}\PYG{o}{(}\PYG{k}{new} \PYG{n}{XmlRpcClientConfigImpl}\PYG{o}{(}\PYG{o}{)} \PYG{o}{\PYGZob{}}\PYG{o}{\PYGZob{}}
        \PYG{n}{setServerURL}\PYG{o}{(}\PYG{k}{new} \PYG{n}{URL}\PYG{o}{(}\PYG{n}{String}\PYG{o}{.}\PYG{n+na}{format}\PYG{o}{(}\PYG{l+s}{\PYGZdq{}\PYGZpc{}s/xmlrpc/2/object\PYGZdq{}}\PYG{o}{,} \PYG{n}{url}\PYG{o}{)}\PYG{o}{)}\PYG{o}{)}\PYG{o}{;}
    \PYG{o}{\PYGZcb{}}\PYG{o}{\PYGZcb{}}\PYG{o}{)}\PYG{o}{;}
\PYG{o}{\PYGZcb{}}\PYG{o}{\PYGZcb{}}\PYG{o}{;}
\PYG{n}{models}\PYG{o}{.}\PYG{n+na}{execute}\PYG{o}{(}\PYG{l+s}{\PYGZdq{}execute\PYGZus{}kw\PYGZdq{}}\PYG{o}{,} \PYG{n}{asList}\PYG{o}{(}
    \PYG{n}{db}\PYG{o}{,} \PYG{n}{uid}\PYG{o}{,} \PYG{n}{password}\PYG{o}{,}
    \PYG{l+s}{\PYGZdq{}res.partner\PYGZdq{}}\PYG{o}{,} \PYG{l+s}{\PYGZdq{}check\PYGZus{}access\PYGZus{}rights\PYGZdq{}}\PYG{o}{,}
    \PYG{n}{asList}\PYG{o}{(}\PYG{l+s}{\PYGZdq{}read\PYGZdq{}}\PYG{o}{)}\PYG{o}{,}
    \PYG{k}{new} \PYG{n}{HashMap}\PYG{o}{(}\PYG{o}{)} \PYG{o}{\PYGZob{}}\PYG{o}{\PYGZob{}} \PYG{n}{put}\PYG{o}{(}\PYG{l+s}{\PYGZdq{}raise\PYGZus{}exception\PYGZdq{}}\PYG{o}{,} \PYG{k+kc}{false}\PYG{o}{)}\PYG{o}{;} \PYG{o}{\PYGZcb{}}\PYG{o}{\PYGZcb{}}
\PYG{o}{)}\PYG{o}{)}\PYG{o}{;}
\end{sphinxVerbatim}

\fvset{hllines={, ,}}%
\begin{sphinxVerbatim}[commandchars=\\\{\}]
\PYG{k+kc}{true}
\end{sphinxVerbatim}


\subsubsection{List records}
\label{\detokenize{webservices/odoo:list-records}}
Records can be listed and filtered via {\hyperref[\detokenize{reference/orm:odoo.models.Model.search}]{\sphinxcrossref{\sphinxcode{\sphinxupquote{search()}}}}}.

{\hyperref[\detokenize{reference/orm:odoo.models.Model.search}]{\sphinxcrossref{\sphinxcode{\sphinxupquote{search()}}}}} takes a mandatory
{\hyperref[\detokenize{reference/orm:reference-orm-domains}]{\sphinxcrossref{\DUrole{std,std-ref}{domain}}}} filter (possibly empty), and returns the
database identifiers of all records matching the filter. To list customer
companies for instance:
\begin{itemize}
\item {} Python 2
\item {} Ruby
\item {} PHP
\item {} Java
\end{itemize}

\fvset{hllines={, ,}}%
\begin{sphinxVerbatim}[commandchars=\\\{\}]
\PYG{n}{models}\PYG{o}{.}\PYG{n}{execute\PYGZus{}kw}\PYG{p}{(}\PYG{n}{db}\PYG{p}{,} \PYG{n}{uid}\PYG{p}{,} \PYG{n}{password}\PYG{p}{,}
    \PYG{l+s+s1}{\PYGZsq{}}\PYG{l+s+s1}{res.partner}\PYG{l+s+s1}{\PYGZsq{}}\PYG{p}{,} \PYG{l+s+s1}{\PYGZsq{}}\PYG{l+s+s1}{search}\PYG{l+s+s1}{\PYGZsq{}}\PYG{p}{,}
    \PYG{p}{[}\PYG{p}{[}\PYG{p}{[}\PYG{l+s+s1}{\PYGZsq{}}\PYG{l+s+s1}{is\PYGZus{}company}\PYG{l+s+s1}{\PYGZsq{}}\PYG{p}{,} \PYG{l+s+s1}{\PYGZsq{}}\PYG{l+s+s1}{=}\PYG{l+s+s1}{\PYGZsq{}}\PYG{p}{,} \PYG{n+nb+bp}{True}\PYG{p}{]}\PYG{p}{,} \PYG{p}{[}\PYG{l+s+s1}{\PYGZsq{}}\PYG{l+s+s1}{customer}\PYG{l+s+s1}{\PYGZsq{}}\PYG{p}{,} \PYG{l+s+s1}{\PYGZsq{}}\PYG{l+s+s1}{=}\PYG{l+s+s1}{\PYGZsq{}}\PYG{p}{,} \PYG{n+nb+bp}{True}\PYG{p}{]}\PYG{p}{]}\PYG{p}{]}\PYG{p}{)}
\end{sphinxVerbatim}

\fvset{hllines={, ,}}%
\begin{sphinxVerbatim}[commandchars=\\\{\}]
\PYG{n}{models}\PYG{o}{.}\PYG{n}{execute\PYGZus{}kw}\PYG{p}{(}\PYG{n}{db}\PYG{p}{,} \PYG{n}{uid}\PYG{p}{,} \PYG{n}{password}\PYG{p}{,}
    \PYG{l+s+s1}{\PYGZsq{}res.partner\PYGZsq{}}\PYG{p}{,} \PYG{l+s+s1}{\PYGZsq{}search\PYGZsq{}}\PYG{p}{,}
    \PYG{o}{[}\PYG{o}{[}\PYG{o}{[}\PYG{l+s+s1}{\PYGZsq{}is\PYGZus{}company\PYGZsq{}}\PYG{p}{,} \PYG{l+s+s1}{\PYGZsq{}=\PYGZsq{}}\PYG{p}{,} \PYG{k+kp}{true}\PYG{o}{]}\PYG{p}{,} \PYG{o}{[}\PYG{l+s+s1}{\PYGZsq{}customer\PYGZsq{}}\PYG{p}{,} \PYG{l+s+s1}{\PYGZsq{}=\PYGZsq{}}\PYG{p}{,} \PYG{k+kp}{true}\PYG{o}{]}\PYG{o}{]}\PYG{o}{]}\PYG{p}{)}
\end{sphinxVerbatim}

\fvset{hllines={, ,}}%
\begin{sphinxVerbatim}[commandchars=\\\{\}]
\PYG{n+nv}{\PYGZdl{}models}\PYG{o}{\PYGZhy{}\PYGZgt{}}\PYG{n+na}{execute\PYGZus{}kw}\PYG{p}{(}\PYG{n+nv}{\PYGZdl{}db}\PYG{p}{,} \PYG{n+nv}{\PYGZdl{}uid}\PYG{p}{,} \PYG{n+nv}{\PYGZdl{}password}\PYG{p}{,}
    \PYG{l+s+s1}{\PYGZsq{}res.partner\PYGZsq{}}\PYG{p}{,} \PYG{l+s+s1}{\PYGZsq{}search\PYGZsq{}}\PYG{p}{,} \PYG{k}{array}\PYG{p}{(}
        \PYG{k}{array}\PYG{p}{(}\PYG{k}{array}\PYG{p}{(}\PYG{l+s+s1}{\PYGZsq{}is\PYGZus{}company\PYGZsq{}}\PYG{p}{,} \PYG{l+s+s1}{\PYGZsq{}=\PYGZsq{}}\PYG{p}{,} \PYG{k}{true}\PYG{p}{),}
              \PYG{k}{array}\PYG{p}{(}\PYG{l+s+s1}{\PYGZsq{}customer\PYGZsq{}}\PYG{p}{,} \PYG{l+s+s1}{\PYGZsq{}=\PYGZsq{}}\PYG{p}{,} \PYG{k}{true}\PYG{p}{))));}
\end{sphinxVerbatim}

\fvset{hllines={, ,}}%
\begin{sphinxVerbatim}[commandchars=\\\{\}]
\PYG{n}{asList}\PYG{o}{(}\PYG{o}{(}\PYG{n}{Object}\PYG{o}{[}\PYG{o}{]}\PYG{o}{)}\PYG{n}{models}\PYG{o}{.}\PYG{n+na}{execute}\PYG{o}{(}\PYG{l+s}{\PYGZdq{}execute\PYGZus{}kw\PYGZdq{}}\PYG{o}{,} \PYG{n}{asList}\PYG{o}{(}
    \PYG{n}{db}\PYG{o}{,} \PYG{n}{uid}\PYG{o}{,} \PYG{n}{password}\PYG{o}{,}
    \PYG{l+s}{\PYGZdq{}res.partner\PYGZdq{}}\PYG{o}{,} \PYG{l+s}{\PYGZdq{}search\PYGZdq{}}\PYG{o}{,}
    \PYG{n}{asList}\PYG{o}{(}\PYG{n}{asList}\PYG{o}{(}
        \PYG{n}{asList}\PYG{o}{(}\PYG{l+s}{\PYGZdq{}is\PYGZus{}company\PYGZdq{}}\PYG{o}{,} \PYG{l+s}{\PYGZdq{}=\PYGZdq{}}\PYG{o}{,} \PYG{k+kc}{true}\PYG{o}{)}\PYG{o}{,}
        \PYG{n}{asList}\PYG{o}{(}\PYG{l+s}{\PYGZdq{}customer\PYGZdq{}}\PYG{o}{,} \PYG{l+s}{\PYGZdq{}=\PYGZdq{}}\PYG{o}{,} \PYG{k+kc}{true}\PYG{o}{)}\PYG{o}{)}\PYG{o}{)}
\PYG{o}{)}\PYG{o}{)}\PYG{o}{)}\PYG{o}{;}
\end{sphinxVerbatim}

\fvset{hllines={, ,}}%
\begin{sphinxVerbatim}[commandchars=\\\{\}]
\PYG{p}{[}\PYG{l+m+mi}{7}\PYG{p}{,} \PYG{l+m+mi}{18}\PYG{p}{,} \PYG{l+m+mi}{12}\PYG{p}{,} \PYG{l+m+mi}{14}\PYG{p}{,} \PYG{l+m+mi}{17}\PYG{p}{,} \PYG{l+m+mi}{19}\PYG{p}{,} \PYG{l+m+mi}{8}\PYG{p}{,} \PYG{l+m+mi}{31}\PYG{p}{,} \PYG{l+m+mi}{26}\PYG{p}{,} \PYG{l+m+mi}{16}\PYG{p}{,} \PYG{l+m+mi}{13}\PYG{p}{,} \PYG{l+m+mi}{20}\PYG{p}{,} \PYG{l+m+mi}{30}\PYG{p}{,} \PYG{l+m+mi}{22}\PYG{p}{,} \PYG{l+m+mi}{29}\PYG{p}{,} \PYG{l+m+mi}{15}\PYG{p}{,} \PYG{l+m+mi}{23}\PYG{p}{,} \PYG{l+m+mi}{28}\PYG{p}{,} \PYG{l+m+mi}{74}\PYG{p}{]}
\end{sphinxVerbatim}


\paragraph{Pagination}
\label{\detokenize{webservices/odoo:pagination}}
By default a search will return the ids of all records matching the
condition, which may be a huge number. \sphinxcode{\sphinxupquote{offset}} and \sphinxcode{\sphinxupquote{limit}} parameters are
available to only retrieve a subset of all matched records.
\begin{itemize}
\item {} Python 2
\item {} Ruby
\item {} PHP
\item {} Java
\end{itemize}

\fvset{hllines={, ,}}%
\begin{sphinxVerbatim}[commandchars=\\\{\}]
\PYG{n}{models}\PYG{o}{.}\PYG{n}{execute\PYGZus{}kw}\PYG{p}{(}\PYG{n}{db}\PYG{p}{,} \PYG{n}{uid}\PYG{p}{,} \PYG{n}{password}\PYG{p}{,}
    \PYG{l+s+s1}{\PYGZsq{}}\PYG{l+s+s1}{res.partner}\PYG{l+s+s1}{\PYGZsq{}}\PYG{p}{,} \PYG{l+s+s1}{\PYGZsq{}}\PYG{l+s+s1}{search}\PYG{l+s+s1}{\PYGZsq{}}\PYG{p}{,}
    \PYG{p}{[}\PYG{p}{[}\PYG{p}{[}\PYG{l+s+s1}{\PYGZsq{}}\PYG{l+s+s1}{is\PYGZus{}company}\PYG{l+s+s1}{\PYGZsq{}}\PYG{p}{,} \PYG{l+s+s1}{\PYGZsq{}}\PYG{l+s+s1}{=}\PYG{l+s+s1}{\PYGZsq{}}\PYG{p}{,} \PYG{n+nb+bp}{True}\PYG{p}{]}\PYG{p}{,} \PYG{p}{[}\PYG{l+s+s1}{\PYGZsq{}}\PYG{l+s+s1}{customer}\PYG{l+s+s1}{\PYGZsq{}}\PYG{p}{,} \PYG{l+s+s1}{\PYGZsq{}}\PYG{l+s+s1}{=}\PYG{l+s+s1}{\PYGZsq{}}\PYG{p}{,} \PYG{n+nb+bp}{True}\PYG{p}{]}\PYG{p}{]}\PYG{p}{]}\PYG{p}{,}
    \PYG{p}{\PYGZob{}}\PYG{l+s+s1}{\PYGZsq{}}\PYG{l+s+s1}{offset}\PYG{l+s+s1}{\PYGZsq{}}\PYG{p}{:} \PYG{l+m+mi}{10}\PYG{p}{,} \PYG{l+s+s1}{\PYGZsq{}}\PYG{l+s+s1}{limit}\PYG{l+s+s1}{\PYGZsq{}}\PYG{p}{:} \PYG{l+m+mi}{5}\PYG{p}{\PYGZcb{}}\PYG{p}{)}
\end{sphinxVerbatim}

\fvset{hllines={, ,}}%
\begin{sphinxVerbatim}[commandchars=\\\{\}]
\PYG{n}{models}\PYG{o}{.}\PYG{n}{execute\PYGZus{}kw}\PYG{p}{(}\PYG{n}{db}\PYG{p}{,} \PYG{n}{uid}\PYG{p}{,} \PYG{n}{password}\PYG{p}{,}
    \PYG{l+s+s1}{\PYGZsq{}res.partner\PYGZsq{}}\PYG{p}{,} \PYG{l+s+s1}{\PYGZsq{}search\PYGZsq{}}\PYG{p}{,}
    \PYG{o}{[}\PYG{o}{[}\PYG{o}{[}\PYG{l+s+s1}{\PYGZsq{}is\PYGZus{}company\PYGZsq{}}\PYG{p}{,} \PYG{l+s+s1}{\PYGZsq{}=\PYGZsq{}}\PYG{p}{,} \PYG{k+kp}{true}\PYG{o}{]}\PYG{p}{,} \PYG{o}{[}\PYG{l+s+s1}{\PYGZsq{}customer\PYGZsq{}}\PYG{p}{,} \PYG{l+s+s1}{\PYGZsq{}=\PYGZsq{}}\PYG{p}{,} \PYG{k+kp}{true}\PYG{o}{]}\PYG{o}{]}\PYG{o}{]}\PYG{p}{,}
    \PYG{p}{\PYGZob{}}\PYG{l+s+ss}{offset}\PYG{p}{:} \PYG{l+m+mi}{10}\PYG{p}{,} \PYG{l+s+ss}{limit}\PYG{p}{:} \PYG{l+m+mi}{5}\PYG{p}{\PYGZcb{}}\PYG{p}{)}
\end{sphinxVerbatim}

\fvset{hllines={, ,}}%
\begin{sphinxVerbatim}[commandchars=\\\{\}]
\PYG{n+nv}{\PYGZdl{}models}\PYG{o}{\PYGZhy{}\PYGZgt{}}\PYG{n+na}{execute\PYGZus{}kw}\PYG{p}{(}\PYG{n+nv}{\PYGZdl{}db}\PYG{p}{,} \PYG{n+nv}{\PYGZdl{}uid}\PYG{p}{,} \PYG{n+nv}{\PYGZdl{}password}\PYG{p}{,}
    \PYG{l+s+s1}{\PYGZsq{}res.partner\PYGZsq{}}\PYG{p}{,} \PYG{l+s+s1}{\PYGZsq{}search\PYGZsq{}}\PYG{p}{,}
    \PYG{k}{array}\PYG{p}{(}\PYG{k}{array}\PYG{p}{(}\PYG{k}{array}\PYG{p}{(}\PYG{l+s+s1}{\PYGZsq{}is\PYGZus{}company\PYGZsq{}}\PYG{p}{,} \PYG{l+s+s1}{\PYGZsq{}=\PYGZsq{}}\PYG{p}{,} \PYG{k}{true}\PYG{p}{),}
                \PYG{k}{array}\PYG{p}{(}\PYG{l+s+s1}{\PYGZsq{}customer\PYGZsq{}}\PYG{p}{,} \PYG{l+s+s1}{\PYGZsq{}=\PYGZsq{}}\PYG{p}{,} \PYG{k}{true}\PYG{p}{))),}
    \PYG{k}{array}\PYG{p}{(}\PYG{l+s+s1}{\PYGZsq{}offset\PYGZsq{}}\PYG{o}{=\PYGZgt{}}\PYG{l+m+mi}{10}\PYG{p}{,} \PYG{l+s+s1}{\PYGZsq{}limit\PYGZsq{}}\PYG{o}{=\PYGZgt{}}\PYG{l+m+mi}{5}\PYG{p}{));}
\end{sphinxVerbatim}

\fvset{hllines={, ,}}%
\begin{sphinxVerbatim}[commandchars=\\\{\}]
\PYG{n}{asList}\PYG{o}{(}\PYG{o}{(}\PYG{n}{Object}\PYG{o}{[}\PYG{o}{]}\PYG{o}{)}\PYG{n}{models}\PYG{o}{.}\PYG{n+na}{execute}\PYG{o}{(}\PYG{l+s}{\PYGZdq{}execute\PYGZus{}kw\PYGZdq{}}\PYG{o}{,} \PYG{n}{asList}\PYG{o}{(}
    \PYG{n}{db}\PYG{o}{,} \PYG{n}{uid}\PYG{o}{,} \PYG{n}{password}\PYG{o}{,}
    \PYG{l+s}{\PYGZdq{}res.partner\PYGZdq{}}\PYG{o}{,} \PYG{l+s}{\PYGZdq{}search\PYGZdq{}}\PYG{o}{,}
    \PYG{n}{asList}\PYG{o}{(}\PYG{n}{asList}\PYG{o}{(}
        \PYG{n}{asList}\PYG{o}{(}\PYG{l+s}{\PYGZdq{}is\PYGZus{}company\PYGZdq{}}\PYG{o}{,} \PYG{l+s}{\PYGZdq{}=\PYGZdq{}}\PYG{o}{,} \PYG{k+kc}{true}\PYG{o}{)}\PYG{o}{,}
        \PYG{n}{asList}\PYG{o}{(}\PYG{l+s}{\PYGZdq{}customer\PYGZdq{}}\PYG{o}{,} \PYG{l+s}{\PYGZdq{}=\PYGZdq{}}\PYG{o}{,} \PYG{k+kc}{true}\PYG{o}{)}\PYG{o}{)}\PYG{o}{)}\PYG{o}{,}
    \PYG{k}{new} \PYG{n}{HashMap}\PYG{o}{(}\PYG{o}{)} \PYG{o}{\PYGZob{}}\PYG{o}{\PYGZob{}} \PYG{n}{put}\PYG{o}{(}\PYG{l+s}{\PYGZdq{}offset\PYGZdq{}}\PYG{o}{,} \PYG{l+m+mi}{10}\PYG{o}{)}\PYG{o}{;} \PYG{n}{put}\PYG{o}{(}\PYG{l+s}{\PYGZdq{}limit\PYGZdq{}}\PYG{o}{,} \PYG{l+m+mi}{5}\PYG{o}{)}\PYG{o}{;} \PYG{o}{\PYGZcb{}}\PYG{o}{\PYGZcb{}}
\PYG{o}{)}\PYG{o}{)}\PYG{o}{)}\PYG{o}{;}
\end{sphinxVerbatim}

\fvset{hllines={, ,}}%
\begin{sphinxVerbatim}[commandchars=\\\{\}]
\PYG{p}{[}\PYG{l+m+mi}{13}\PYG{p}{,} \PYG{l+m+mi}{20}\PYG{p}{,} \PYG{l+m+mi}{30}\PYG{p}{,} \PYG{l+m+mi}{22}\PYG{p}{,} \PYG{l+m+mi}{29}\PYG{p}{]}
\end{sphinxVerbatim}


\subsubsection{Count records}
\label{\detokenize{webservices/odoo:count-records}}
Rather than retrieve a possibly gigantic list of records and count them,
{\hyperref[\detokenize{reference/orm:odoo.models.Model.search_count}]{\sphinxcrossref{\sphinxcode{\sphinxupquote{search\_count()}}}}} can be used to retrieve
only the number of records matching the query. It takes the same
{\hyperref[\detokenize{reference/orm:reference-orm-domains}]{\sphinxcrossref{\DUrole{std,std-ref}{domain}}}} filter as
{\hyperref[\detokenize{reference/orm:odoo.models.Model.search}]{\sphinxcrossref{\sphinxcode{\sphinxupquote{search()}}}}} and no other parameter.
\begin{itemize}
\item {} Python 2
\item {} Ruby
\item {} PHP
\item {} Java
\end{itemize}

\fvset{hllines={, ,}}%
\begin{sphinxVerbatim}[commandchars=\\\{\}]
\PYG{n}{models}\PYG{o}{.}\PYG{n}{execute\PYGZus{}kw}\PYG{p}{(}\PYG{n}{db}\PYG{p}{,} \PYG{n}{uid}\PYG{p}{,} \PYG{n}{password}\PYG{p}{,}
    \PYG{l+s+s1}{\PYGZsq{}}\PYG{l+s+s1}{res.partner}\PYG{l+s+s1}{\PYGZsq{}}\PYG{p}{,} \PYG{l+s+s1}{\PYGZsq{}}\PYG{l+s+s1}{search\PYGZus{}count}\PYG{l+s+s1}{\PYGZsq{}}\PYG{p}{,}
    \PYG{p}{[}\PYG{p}{[}\PYG{p}{[}\PYG{l+s+s1}{\PYGZsq{}}\PYG{l+s+s1}{is\PYGZus{}company}\PYG{l+s+s1}{\PYGZsq{}}\PYG{p}{,} \PYG{l+s+s1}{\PYGZsq{}}\PYG{l+s+s1}{=}\PYG{l+s+s1}{\PYGZsq{}}\PYG{p}{,} \PYG{n+nb+bp}{True}\PYG{p}{]}\PYG{p}{,} \PYG{p}{[}\PYG{l+s+s1}{\PYGZsq{}}\PYG{l+s+s1}{customer}\PYG{l+s+s1}{\PYGZsq{}}\PYG{p}{,} \PYG{l+s+s1}{\PYGZsq{}}\PYG{l+s+s1}{=}\PYG{l+s+s1}{\PYGZsq{}}\PYG{p}{,} \PYG{n+nb+bp}{True}\PYG{p}{]}\PYG{p}{]}\PYG{p}{]}\PYG{p}{)}
\end{sphinxVerbatim}

\fvset{hllines={, ,}}%
\begin{sphinxVerbatim}[commandchars=\\\{\}]
\PYG{n}{models}\PYG{o}{.}\PYG{n}{execute\PYGZus{}kw}\PYG{p}{(}\PYG{n}{db}\PYG{p}{,} \PYG{n}{uid}\PYG{p}{,} \PYG{n}{password}\PYG{p}{,}
    \PYG{l+s+s1}{\PYGZsq{}res.partner\PYGZsq{}}\PYG{p}{,} \PYG{l+s+s1}{\PYGZsq{}search\PYGZus{}count\PYGZsq{}}\PYG{p}{,}
    \PYG{o}{[}\PYG{o}{[}\PYG{o}{[}\PYG{l+s+s1}{\PYGZsq{}is\PYGZus{}company\PYGZsq{}}\PYG{p}{,} \PYG{l+s+s1}{\PYGZsq{}=\PYGZsq{}}\PYG{p}{,} \PYG{k+kp}{true}\PYG{o}{]}\PYG{p}{,} \PYG{o}{[}\PYG{l+s+s1}{\PYGZsq{}customer\PYGZsq{}}\PYG{p}{,} \PYG{l+s+s1}{\PYGZsq{}=\PYGZsq{}}\PYG{p}{,} \PYG{k+kp}{true}\PYG{o}{]}\PYG{o}{]}\PYG{o}{]}\PYG{p}{)}
\end{sphinxVerbatim}

\fvset{hllines={, ,}}%
\begin{sphinxVerbatim}[commandchars=\\\{\}]
\PYG{n+nv}{\PYGZdl{}models}\PYG{o}{\PYGZhy{}\PYGZgt{}}\PYG{n+na}{execute\PYGZus{}kw}\PYG{p}{(}\PYG{n+nv}{\PYGZdl{}db}\PYG{p}{,} \PYG{n+nv}{\PYGZdl{}uid}\PYG{p}{,} \PYG{n+nv}{\PYGZdl{}password}\PYG{p}{,}
    \PYG{l+s+s1}{\PYGZsq{}res.partner\PYGZsq{}}\PYG{p}{,} \PYG{l+s+s1}{\PYGZsq{}search\PYGZus{}count\PYGZsq{}}\PYG{p}{,}
    \PYG{k}{array}\PYG{p}{(}\PYG{k}{array}\PYG{p}{(}\PYG{k}{array}\PYG{p}{(}\PYG{l+s+s1}{\PYGZsq{}is\PYGZus{}company\PYGZsq{}}\PYG{p}{,} \PYG{l+s+s1}{\PYGZsq{}=\PYGZsq{}}\PYG{p}{,} \PYG{k}{true}\PYG{p}{),}
                \PYG{k}{array}\PYG{p}{(}\PYG{l+s+s1}{\PYGZsq{}customer\PYGZsq{}}\PYG{p}{,} \PYG{l+s+s1}{\PYGZsq{}=\PYGZsq{}}\PYG{p}{,} \PYG{k}{true}\PYG{p}{))));}
\end{sphinxVerbatim}

\fvset{hllines={, ,}}%
\begin{sphinxVerbatim}[commandchars=\\\{\}]
\PYG{o}{(}\PYG{n}{Integer}\PYG{o}{)}\PYG{n}{models}\PYG{o}{.}\PYG{n+na}{execute}\PYG{o}{(}\PYG{l+s}{\PYGZdq{}execute\PYGZus{}kw\PYGZdq{}}\PYG{o}{,} \PYG{n}{asList}\PYG{o}{(}
    \PYG{n}{db}\PYG{o}{,} \PYG{n}{uid}\PYG{o}{,} \PYG{n}{password}\PYG{o}{,}
    \PYG{l+s}{\PYGZdq{}res.partner\PYGZdq{}}\PYG{o}{,} \PYG{l+s}{\PYGZdq{}search\PYGZus{}count\PYGZdq{}}\PYG{o}{,}
    \PYG{n}{asList}\PYG{o}{(}\PYG{n}{asList}\PYG{o}{(}
        \PYG{n}{asList}\PYG{o}{(}\PYG{l+s}{\PYGZdq{}is\PYGZus{}company\PYGZdq{}}\PYG{o}{,} \PYG{l+s}{\PYGZdq{}=\PYGZdq{}}\PYG{o}{,} \PYG{k+kc}{true}\PYG{o}{)}\PYG{o}{,}
        \PYG{n}{asList}\PYG{o}{(}\PYG{l+s}{\PYGZdq{}customer\PYGZdq{}}\PYG{o}{,} \PYG{l+s}{\PYGZdq{}=\PYGZdq{}}\PYG{o}{,} \PYG{k+kc}{true}\PYG{o}{)}\PYG{o}{)}\PYG{o}{)}
\PYG{o}{)}\PYG{o}{)}\PYG{o}{;}
\end{sphinxVerbatim}

\fvset{hllines={, ,}}%
\begin{sphinxVerbatim}[commandchars=\\\{\}]
\PYG{l+m+mi}{19}
\end{sphinxVerbatim}

\begin{sphinxadmonition}{warning}{Warning:}
calling \sphinxcode{\sphinxupquote{search}} then \sphinxcode{\sphinxupquote{search\_count}} (or the other way around) may not
yield coherent results if other users are using the server: stored data
could have changed between the calls
\end{sphinxadmonition}


\subsubsection{Read records}
\label{\detokenize{webservices/odoo:read-records}}
Record data is accessible via the {\hyperref[\detokenize{reference/orm:odoo.models.Model.read}]{\sphinxcrossref{\sphinxcode{\sphinxupquote{read()}}}}} method,
which takes a list of ids (as returned by
{\hyperref[\detokenize{reference/orm:odoo.models.Model.search}]{\sphinxcrossref{\sphinxcode{\sphinxupquote{search()}}}}}) and optionally a list of fields to
fetch. By default, it will fetch all the fields the current user can read,
which tends to be a huge amount.
\begin{itemize}
\item {} Python 2
\item {} Ruby
\item {} PHP
\item {} Java
\end{itemize}

\fvset{hllines={, ,}}%
\begin{sphinxVerbatim}[commandchars=\\\{\}]
\PYG{n}{ids} \PYG{o}{=} \PYG{n}{models}\PYG{o}{.}\PYG{n}{execute\PYGZus{}kw}\PYG{p}{(}\PYG{n}{db}\PYG{p}{,} \PYG{n}{uid}\PYG{p}{,} \PYG{n}{password}\PYG{p}{,}
    \PYG{l+s+s1}{\PYGZsq{}}\PYG{l+s+s1}{res.partner}\PYG{l+s+s1}{\PYGZsq{}}\PYG{p}{,} \PYG{l+s+s1}{\PYGZsq{}}\PYG{l+s+s1}{search}\PYG{l+s+s1}{\PYGZsq{}}\PYG{p}{,}
    \PYG{p}{[}\PYG{p}{[}\PYG{p}{[}\PYG{l+s+s1}{\PYGZsq{}}\PYG{l+s+s1}{is\PYGZus{}company}\PYG{l+s+s1}{\PYGZsq{}}\PYG{p}{,} \PYG{l+s+s1}{\PYGZsq{}}\PYG{l+s+s1}{=}\PYG{l+s+s1}{\PYGZsq{}}\PYG{p}{,} \PYG{n+nb+bp}{True}\PYG{p}{]}\PYG{p}{,} \PYG{p}{[}\PYG{l+s+s1}{\PYGZsq{}}\PYG{l+s+s1}{customer}\PYG{l+s+s1}{\PYGZsq{}}\PYG{p}{,} \PYG{l+s+s1}{\PYGZsq{}}\PYG{l+s+s1}{=}\PYG{l+s+s1}{\PYGZsq{}}\PYG{p}{,} \PYG{n+nb+bp}{True}\PYG{p}{]}\PYG{p}{]}\PYG{p}{]}\PYG{p}{,}
    \PYG{p}{\PYGZob{}}\PYG{l+s+s1}{\PYGZsq{}}\PYG{l+s+s1}{limit}\PYG{l+s+s1}{\PYGZsq{}}\PYG{p}{:} \PYG{l+m+mi}{1}\PYG{p}{\PYGZcb{}}\PYG{p}{)}
\PYG{p}{[}\PYG{n}{record}\PYG{p}{]} \PYG{o}{=} \PYG{n}{models}\PYG{o}{.}\PYG{n}{execute\PYGZus{}kw}\PYG{p}{(}\PYG{n}{db}\PYG{p}{,} \PYG{n}{uid}\PYG{p}{,} \PYG{n}{password}\PYG{p}{,}
    \PYG{l+s+s1}{\PYGZsq{}}\PYG{l+s+s1}{res.partner}\PYG{l+s+s1}{\PYGZsq{}}\PYG{p}{,} \PYG{l+s+s1}{\PYGZsq{}}\PYG{l+s+s1}{read}\PYG{l+s+s1}{\PYGZsq{}}\PYG{p}{,} \PYG{p}{[}\PYG{n}{ids}\PYG{p}{]}\PYG{p}{)}
\PYG{c+c1}{\PYGZsh{} count the number of fields fetched by default}
\PYG{n+nb}{len}\PYG{p}{(}\PYG{n}{record}\PYG{p}{)}
\end{sphinxVerbatim}

\fvset{hllines={, ,}}%
\begin{sphinxVerbatim}[commandchars=\\\{\}]
\PYG{n}{ids} \PYG{o}{=} \PYG{n}{models}\PYG{o}{.}\PYG{n}{execute\PYGZus{}kw}\PYG{p}{(}\PYG{n}{db}\PYG{p}{,} \PYG{n}{uid}\PYG{p}{,} \PYG{n}{password}\PYG{p}{,}
    \PYG{l+s+s1}{\PYGZsq{}res.partner\PYGZsq{}}\PYG{p}{,} \PYG{l+s+s1}{\PYGZsq{}search\PYGZsq{}}\PYG{p}{,}
    \PYG{o}{[}\PYG{o}{[}\PYG{o}{[}\PYG{l+s+s1}{\PYGZsq{}is\PYGZus{}company\PYGZsq{}}\PYG{p}{,} \PYG{l+s+s1}{\PYGZsq{}=\PYGZsq{}}\PYG{p}{,} \PYG{k+kp}{true}\PYG{o}{]}\PYG{p}{,} \PYG{o}{[}\PYG{l+s+s1}{\PYGZsq{}customer\PYGZsq{}}\PYG{p}{,} \PYG{l+s+s1}{\PYGZsq{}=\PYGZsq{}}\PYG{p}{,} \PYG{k+kp}{true}\PYG{o}{]}\PYG{o}{]}\PYG{o}{]}\PYG{p}{,}
    \PYG{p}{\PYGZob{}}\PYG{l+s+ss}{limit}\PYG{p}{:} \PYG{l+m+mi}{1}\PYG{p}{\PYGZcb{}}\PYG{p}{)}
\PYG{n}{record} \PYG{o}{=} \PYG{n}{models}\PYG{o}{.}\PYG{n}{execute\PYGZus{}kw}\PYG{p}{(}\PYG{n}{db}\PYG{p}{,} \PYG{n}{uid}\PYG{p}{,} \PYG{n}{password}\PYG{p}{,}
    \PYG{l+s+s1}{\PYGZsq{}res.partner\PYGZsq{}}\PYG{p}{,} \PYG{l+s+s1}{\PYGZsq{}read\PYGZsq{}}\PYG{p}{,} \PYG{o}{[}\PYG{n}{ids}\PYG{o}{]}\PYG{p}{)}\PYG{o}{.}\PYG{n}{first}
\PYG{c+c1}{\PYGZsh{} count the number of fields fetched by default}
\PYG{n}{record}\PYG{o}{.}\PYG{n}{length}
\end{sphinxVerbatim}

\fvset{hllines={, ,}}%
\begin{sphinxVerbatim}[commandchars=\\\{\}]
\PYG{n+nv}{\PYGZdl{}ids} \PYG{o}{=} \PYG{n+nv}{\PYGZdl{}models}\PYG{o}{\PYGZhy{}\PYGZgt{}}\PYG{n+na}{execute\PYGZus{}kw}\PYG{p}{(}\PYG{n+nv}{\PYGZdl{}db}\PYG{p}{,} \PYG{n+nv}{\PYGZdl{}uid}\PYG{p}{,} \PYG{n+nv}{\PYGZdl{}password}\PYG{p}{,}
    \PYG{l+s+s1}{\PYGZsq{}res.partner\PYGZsq{}}\PYG{p}{,} \PYG{l+s+s1}{\PYGZsq{}search\PYGZsq{}}\PYG{p}{,}
    \PYG{k}{array}\PYG{p}{(}\PYG{k}{array}\PYG{p}{(}\PYG{k}{array}\PYG{p}{(}\PYG{l+s+s1}{\PYGZsq{}is\PYGZus{}company\PYGZsq{}}\PYG{p}{,} \PYG{l+s+s1}{\PYGZsq{}=\PYGZsq{}}\PYG{p}{,} \PYG{k}{true}\PYG{p}{),}
                \PYG{k}{array}\PYG{p}{(}\PYG{l+s+s1}{\PYGZsq{}customer\PYGZsq{}}\PYG{p}{,} \PYG{l+s+s1}{\PYGZsq{}=\PYGZsq{}}\PYG{p}{,} \PYG{k}{true}\PYG{p}{))),}
    \PYG{k}{array}\PYG{p}{(}\PYG{l+s+s1}{\PYGZsq{}limit\PYGZsq{}}\PYG{o}{=\PYGZgt{}}\PYG{l+m+mi}{1}\PYG{p}{));}
\PYG{n+nv}{\PYGZdl{}records} \PYG{o}{=} \PYG{n+nv}{\PYGZdl{}models}\PYG{o}{\PYGZhy{}\PYGZgt{}}\PYG{n+na}{execute\PYGZus{}kw}\PYG{p}{(}\PYG{n+nv}{\PYGZdl{}db}\PYG{p}{,} \PYG{n+nv}{\PYGZdl{}uid}\PYG{p}{,} \PYG{n+nv}{\PYGZdl{}password}\PYG{p}{,}
    \PYG{l+s+s1}{\PYGZsq{}res.partner\PYGZsq{}}\PYG{p}{,} \PYG{l+s+s1}{\PYGZsq{}read\PYGZsq{}}\PYG{p}{,} \PYG{k}{array}\PYG{p}{(}\PYG{n+nv}{\PYGZdl{}ids}\PYG{p}{));}
\PYG{c+c1}{// count the number of fields fetched by default}
\PYG{n+nb}{count}\PYG{p}{(}\PYG{n+nv}{\PYGZdl{}records}\PYG{p}{[}\PYG{l+m+mi}{0}\PYG{p}{]);}
\end{sphinxVerbatim}

\fvset{hllines={, ,}}%
\begin{sphinxVerbatim}[commandchars=\\\{\}]
\PYG{k+kd}{final} \PYG{n}{List} \PYG{n}{ids} \PYG{o}{=} \PYG{n}{asList}\PYG{o}{(}\PYG{o}{(}\PYG{n}{Object}\PYG{o}{[}\PYG{o}{]}\PYG{o}{)}\PYG{n}{models}\PYG{o}{.}\PYG{n+na}{execute}\PYG{o}{(}
    \PYG{l+s}{\PYGZdq{}execute\PYGZus{}kw\PYGZdq{}}\PYG{o}{,} \PYG{n}{asList}\PYG{o}{(}
        \PYG{n}{db}\PYG{o}{,} \PYG{n}{uid}\PYG{o}{,} \PYG{n}{password}\PYG{o}{,}
        \PYG{l+s}{\PYGZdq{}res.partner\PYGZdq{}}\PYG{o}{,} \PYG{l+s}{\PYGZdq{}search\PYGZdq{}}\PYG{o}{,}
        \PYG{n}{asList}\PYG{o}{(}\PYG{n}{asList}\PYG{o}{(}
            \PYG{n}{asList}\PYG{o}{(}\PYG{l+s}{\PYGZdq{}is\PYGZus{}company\PYGZdq{}}\PYG{o}{,} \PYG{l+s}{\PYGZdq{}=\PYGZdq{}}\PYG{o}{,} \PYG{k+kc}{true}\PYG{o}{)}\PYG{o}{,}
            \PYG{n}{asList}\PYG{o}{(}\PYG{l+s}{\PYGZdq{}customer\PYGZdq{}}\PYG{o}{,} \PYG{l+s}{\PYGZdq{}=\PYGZdq{}}\PYG{o}{,} \PYG{k+kc}{true}\PYG{o}{)}\PYG{o}{)}\PYG{o}{)}\PYG{o}{,}
        \PYG{k}{new} \PYG{n}{HashMap}\PYG{o}{(}\PYG{o}{)} \PYG{o}{\PYGZob{}}\PYG{o}{\PYGZob{}} \PYG{n}{put}\PYG{o}{(}\PYG{l+s}{\PYGZdq{}limit\PYGZdq{}}\PYG{o}{,} \PYG{l+m+mi}{1}\PYG{o}{)}\PYG{o}{;} \PYG{o}{\PYGZcb{}}\PYG{o}{\PYGZcb{}}\PYG{o}{)}\PYG{o}{)}\PYG{o}{)}\PYG{o}{;}
\PYG{k+kd}{final} \PYG{n}{Map} \PYG{n}{record} \PYG{o}{=} \PYG{o}{(}\PYG{n}{Map}\PYG{o}{)}\PYG{o}{(}\PYG{o}{(}\PYG{n}{Object}\PYG{o}{[}\PYG{o}{]}\PYG{o}{)}\PYG{n}{models}\PYG{o}{.}\PYG{n+na}{execute}\PYG{o}{(}
    \PYG{l+s}{\PYGZdq{}execute\PYGZus{}kw\PYGZdq{}}\PYG{o}{,} \PYG{n}{asList}\PYG{o}{(}
        \PYG{n}{db}\PYG{o}{,} \PYG{n}{uid}\PYG{o}{,} \PYG{n}{password}\PYG{o}{,}
        \PYG{l+s}{\PYGZdq{}res.partner\PYGZdq{}}\PYG{o}{,} \PYG{l+s}{\PYGZdq{}read\PYGZdq{}}\PYG{o}{,}
        \PYG{n}{asList}\PYG{o}{(}\PYG{n}{ids}\PYG{o}{)}
    \PYG{o}{)}
\PYG{o}{)}\PYG{o}{)}\PYG{o}{[}\PYG{l+m+mi}{0}\PYG{o}{]}\PYG{o}{;}
\PYG{c+c1}{// count the number of fields fetched by default}
\PYG{n}{record}\PYG{o}{.}\PYG{n+na}{size}\PYG{o}{(}\PYG{o}{)}\PYG{o}{;}
\end{sphinxVerbatim}

\fvset{hllines={, ,}}%
\begin{sphinxVerbatim}[commandchars=\\\{\}]
\PYG{l+m+mi}{121}
\end{sphinxVerbatim}

Conversedly, picking only three fields deemed interesting.
\begin{itemize}
\item {} Python 2
\item {} Ruby
\item {} PHP
\item {} Java
\end{itemize}

\fvset{hllines={, ,}}%
\begin{sphinxVerbatim}[commandchars=\\\{\}]
\PYG{n}{models}\PYG{o}{.}\PYG{n}{execute\PYGZus{}kw}\PYG{p}{(}\PYG{n}{db}\PYG{p}{,} \PYG{n}{uid}\PYG{p}{,} \PYG{n}{password}\PYG{p}{,}
    \PYG{l+s+s1}{\PYGZsq{}}\PYG{l+s+s1}{res.partner}\PYG{l+s+s1}{\PYGZsq{}}\PYG{p}{,} \PYG{l+s+s1}{\PYGZsq{}}\PYG{l+s+s1}{read}\PYG{l+s+s1}{\PYGZsq{}}\PYG{p}{,}
    \PYG{p}{[}\PYG{n}{ids}\PYG{p}{]}\PYG{p}{,} \PYG{p}{\PYGZob{}}\PYG{l+s+s1}{\PYGZsq{}}\PYG{l+s+s1}{fields}\PYG{l+s+s1}{\PYGZsq{}}\PYG{p}{:} \PYG{p}{[}\PYG{l+s+s1}{\PYGZsq{}}\PYG{l+s+s1}{name}\PYG{l+s+s1}{\PYGZsq{}}\PYG{p}{,} \PYG{l+s+s1}{\PYGZsq{}}\PYG{l+s+s1}{country\PYGZus{}id}\PYG{l+s+s1}{\PYGZsq{}}\PYG{p}{,} \PYG{l+s+s1}{\PYGZsq{}}\PYG{l+s+s1}{comment}\PYG{l+s+s1}{\PYGZsq{}}\PYG{p}{]}\PYG{p}{\PYGZcb{}}\PYG{p}{)}
\end{sphinxVerbatim}

\fvset{hllines={, ,}}%
\begin{sphinxVerbatim}[commandchars=\\\{\}]
\PYG{n}{models}\PYG{o}{.}\PYG{n}{execute\PYGZus{}kw}\PYG{p}{(}\PYG{n}{db}\PYG{p}{,} \PYG{n}{uid}\PYG{p}{,} \PYG{n}{password}\PYG{p}{,}
    \PYG{l+s+s1}{\PYGZsq{}res.partner\PYGZsq{}}\PYG{p}{,} \PYG{l+s+s1}{\PYGZsq{}read\PYGZsq{}}\PYG{p}{,}
    \PYG{o}{[}\PYG{n}{ids}\PYG{o}{]}\PYG{p}{,} \PYG{p}{\PYGZob{}}\PYG{l+s+ss}{fields}\PYG{p}{:} \PYG{l+s+sx}{\PYGZpc{}w(}\PYG{l+s+sx}{name country\PYGZus{}id comment}\PYG{l+s+sx}{)}\PYG{p}{\PYGZcb{}}\PYG{p}{)}
\end{sphinxVerbatim}

\fvset{hllines={, ,}}%
\begin{sphinxVerbatim}[commandchars=\\\{\}]
\PYG{n+nv}{\PYGZdl{}models}\PYG{o}{\PYGZhy{}\PYGZgt{}}\PYG{n+na}{execute\PYGZus{}kw}\PYG{p}{(}\PYG{n+nv}{\PYGZdl{}db}\PYG{p}{,} \PYG{n+nv}{\PYGZdl{}uid}\PYG{p}{,} \PYG{n+nv}{\PYGZdl{}password}\PYG{p}{,}
    \PYG{l+s+s1}{\PYGZsq{}res.partner\PYGZsq{}}\PYG{p}{,} \PYG{l+s+s1}{\PYGZsq{}read\PYGZsq{}}\PYG{p}{,}
    \PYG{k}{array}\PYG{p}{(}\PYG{n+nv}{\PYGZdl{}ids}\PYG{p}{),}
    \PYG{k}{array}\PYG{p}{(}\PYG{l+s+s1}{\PYGZsq{}fields\PYGZsq{}}\PYG{o}{=\PYGZgt{}}\PYG{k}{array}\PYG{p}{(}\PYG{l+s+s1}{\PYGZsq{}name\PYGZsq{}}\PYG{p}{,} \PYG{l+s+s1}{\PYGZsq{}country\PYGZus{}id\PYGZsq{}}\PYG{p}{,} \PYG{l+s+s1}{\PYGZsq{}comment\PYGZsq{}}\PYG{p}{)));}
\end{sphinxVerbatim}

\fvset{hllines={, ,}}%
\begin{sphinxVerbatim}[commandchars=\\\{\}]
\PYG{n}{asList}\PYG{o}{(}\PYG{o}{(}\PYG{n}{Object}\PYG{o}{[}\PYG{o}{]}\PYG{o}{)}\PYG{n}{models}\PYG{o}{.}\PYG{n+na}{execute}\PYG{o}{(}\PYG{l+s}{\PYGZdq{}execute\PYGZus{}kw\PYGZdq{}}\PYG{o}{,} \PYG{n}{asList}\PYG{o}{(}
    \PYG{n}{db}\PYG{o}{,} \PYG{n}{uid}\PYG{o}{,} \PYG{n}{password}\PYG{o}{,}
    \PYG{l+s}{\PYGZdq{}res.partner\PYGZdq{}}\PYG{o}{,} \PYG{l+s}{\PYGZdq{}read\PYGZdq{}}\PYG{o}{,}
    \PYG{n}{asList}\PYG{o}{(}\PYG{n}{ids}\PYG{o}{)}\PYG{o}{,}
    \PYG{k}{new} \PYG{n}{HashMap}\PYG{o}{(}\PYG{o}{)} \PYG{o}{\PYGZob{}}\PYG{o}{\PYGZob{}}
        \PYG{n}{put}\PYG{o}{(}\PYG{l+s}{\PYGZdq{}fields\PYGZdq{}}\PYG{o}{,} \PYG{n}{asList}\PYG{o}{(}\PYG{l+s}{\PYGZdq{}name\PYGZdq{}}\PYG{o}{,} \PYG{l+s}{\PYGZdq{}country\PYGZus{}id\PYGZdq{}}\PYG{o}{,} \PYG{l+s}{\PYGZdq{}comment\PYGZdq{}}\PYG{o}{)}\PYG{o}{)}\PYG{o}{;}
    \PYG{o}{\PYGZcb{}}\PYG{o}{\PYGZcb{}}
\PYG{o}{)}\PYG{o}{)}\PYG{o}{)}\PYG{o}{;}
\end{sphinxVerbatim}

\fvset{hllines={, ,}}%
\begin{sphinxVerbatim}[commandchars=\\\{\}]
\PYG{p}{[}\PYG{p}{\PYGZob{}}\PYG{n+nt}{\PYGZdq{}comment\PYGZdq{}}\PYG{p}{:} \PYG{k+kc}{false}\PYG{p}{,} \PYG{n+nt}{\PYGZdq{}country\PYGZus{}id\PYGZdq{}}\PYG{p}{:} \PYG{p}{[}\PYG{l+m+mi}{21}\PYG{p}{,} \PYG{l+s+s2}{\PYGZdq{}Belgium\PYGZdq{}}\PYG{p}{]}\PYG{p}{,} \PYG{n+nt}{\PYGZdq{}id\PYGZdq{}}\PYG{p}{:} \PYG{l+m+mi}{7}\PYG{p}{,} \PYG{n+nt}{\PYGZdq{}name\PYGZdq{}}\PYG{p}{:} \PYG{l+s+s2}{\PYGZdq{}Agrolait\PYGZdq{}}\PYG{p}{\PYGZcb{}}\PYG{p}{]}
\end{sphinxVerbatim}

\begin{sphinxadmonition}{note}{Note:}
even if the \sphinxcode{\sphinxupquote{id}} field is not requested, it is always returned
\end{sphinxadmonition}


\subsubsection{Listing record fields}
\label{\detokenize{webservices/odoo:listing-record-fields}}
{\hyperref[\detokenize{reference/orm:odoo.models.Model.fields_get}]{\sphinxcrossref{\sphinxcode{\sphinxupquote{fields\_get()}}}}} can be used to inspect
a model’s fields and check which ones seem to be of interest.

Because it returns a large amount of meta-information (it is also used by client
programs) it should be filtered before printing, the most interesting items
for a human user are \sphinxcode{\sphinxupquote{string}} (the field’s label), \sphinxcode{\sphinxupquote{help}} (a help text if
available) and \sphinxcode{\sphinxupquote{type}} (to know which values to expect, or to send when
updating a record):
\begin{itemize}
\item {} Python 2
\item {} Ruby
\item {} PHP
\item {} Java
\end{itemize}

\fvset{hllines={, ,}}%
\begin{sphinxVerbatim}[commandchars=\\\{\}]
\PYG{n}{models}\PYG{o}{.}\PYG{n}{execute\PYGZus{}kw}\PYG{p}{(}
    \PYG{n}{db}\PYG{p}{,} \PYG{n}{uid}\PYG{p}{,} \PYG{n}{password}\PYG{p}{,} \PYG{l+s+s1}{\PYGZsq{}}\PYG{l+s+s1}{res.partner}\PYG{l+s+s1}{\PYGZsq{}}\PYG{p}{,} \PYG{l+s+s1}{\PYGZsq{}}\PYG{l+s+s1}{fields\PYGZus{}get}\PYG{l+s+s1}{\PYGZsq{}}\PYG{p}{,}
    \PYG{p}{[}\PYG{p}{]}\PYG{p}{,} \PYG{p}{\PYGZob{}}\PYG{l+s+s1}{\PYGZsq{}}\PYG{l+s+s1}{attributes}\PYG{l+s+s1}{\PYGZsq{}}\PYG{p}{:} \PYG{p}{[}\PYG{l+s+s1}{\PYGZsq{}}\PYG{l+s+s1}{string}\PYG{l+s+s1}{\PYGZsq{}}\PYG{p}{,} \PYG{l+s+s1}{\PYGZsq{}}\PYG{l+s+s1}{help}\PYG{l+s+s1}{\PYGZsq{}}\PYG{p}{,} \PYG{l+s+s1}{\PYGZsq{}}\PYG{l+s+s1}{type}\PYG{l+s+s1}{\PYGZsq{}}\PYG{p}{]}\PYG{p}{\PYGZcb{}}\PYG{p}{)}
\end{sphinxVerbatim}

\fvset{hllines={, ,}}%
\begin{sphinxVerbatim}[commandchars=\\\{\}]
\PYG{n}{models}\PYG{o}{.}\PYG{n}{execute\PYGZus{}kw}\PYG{p}{(}
    \PYG{n}{db}\PYG{p}{,} \PYG{n}{uid}\PYG{p}{,} \PYG{n}{password}\PYG{p}{,} \PYG{l+s+s1}{\PYGZsq{}res.partner\PYGZsq{}}\PYG{p}{,} \PYG{l+s+s1}{\PYGZsq{}fields\PYGZus{}get\PYGZsq{}}\PYG{p}{,}
    \PYG{o}{[}\PYG{o}{]}\PYG{p}{,} \PYG{p}{\PYGZob{}}\PYG{l+s+ss}{attributes}\PYG{p}{:} \PYG{l+s+sx}{\PYGZpc{}w(}\PYG{l+s+sx}{string help type}\PYG{l+s+sx}{)}\PYG{p}{\PYGZcb{}}\PYG{p}{)}
\end{sphinxVerbatim}

\fvset{hllines={, ,}}%
\begin{sphinxVerbatim}[commandchars=\\\{\}]
\PYG{n+nv}{\PYGZdl{}models}\PYG{o}{\PYGZhy{}\PYGZgt{}}\PYG{n+na}{execute\PYGZus{}kw}\PYG{p}{(}\PYG{n+nv}{\PYGZdl{}db}\PYG{p}{,} \PYG{n+nv}{\PYGZdl{}uid}\PYG{p}{,} \PYG{n+nv}{\PYGZdl{}password}\PYG{p}{,}
    \PYG{l+s+s1}{\PYGZsq{}res.partner\PYGZsq{}}\PYG{p}{,} \PYG{l+s+s1}{\PYGZsq{}fields\PYGZus{}get\PYGZsq{}}\PYG{p}{,}
    \PYG{k}{array}\PYG{p}{(),} \PYG{k}{array}\PYG{p}{(}\PYG{l+s+s1}{\PYGZsq{}attributes\PYGZsq{}} \PYG{o}{=\PYGZgt{}} \PYG{k}{array}\PYG{p}{(}\PYG{l+s+s1}{\PYGZsq{}string\PYGZsq{}}\PYG{p}{,} \PYG{l+s+s1}{\PYGZsq{}help\PYGZsq{}}\PYG{p}{,} \PYG{l+s+s1}{\PYGZsq{}type\PYGZsq{}}\PYG{p}{)));}
\end{sphinxVerbatim}

\fvset{hllines={, ,}}%
\begin{sphinxVerbatim}[commandchars=\\\{\}]
\PYG{o}{(}\PYG{n}{Map}\PYG{o}{\PYGZlt{}}\PYG{n}{String}\PYG{o}{,} \PYG{n}{Map}\PYG{o}{\PYGZlt{}}\PYG{n}{String}\PYG{o}{,} \PYG{n}{Object}\PYG{o}{\PYGZgt{}}\PYG{o}{\PYGZgt{}}\PYG{o}{)}\PYG{n}{models}\PYG{o}{.}\PYG{n+na}{execute}\PYG{o}{(}\PYG{l+s}{\PYGZdq{}execute\PYGZus{}kw\PYGZdq{}}\PYG{o}{,} \PYG{n}{asList}\PYG{o}{(}
    \PYG{n}{db}\PYG{o}{,} \PYG{n}{uid}\PYG{o}{,} \PYG{n}{password}\PYG{o}{,}
    \PYG{l+s}{\PYGZdq{}res.partner\PYGZdq{}}\PYG{o}{,} \PYG{l+s}{\PYGZdq{}fields\PYGZus{}get\PYGZdq{}}\PYG{o}{,}
    \PYG{n}{emptyList}\PYG{o}{(}\PYG{o}{)}\PYG{o}{,}
    \PYG{k}{new} \PYG{n}{HashMap}\PYG{o}{(}\PYG{o}{)} \PYG{o}{\PYGZob{}}\PYG{o}{\PYGZob{}}
        \PYG{n}{put}\PYG{o}{(}\PYG{l+s}{\PYGZdq{}attributes\PYGZdq{}}\PYG{o}{,} \PYG{n}{asList}\PYG{o}{(}\PYG{l+s}{\PYGZdq{}string\PYGZdq{}}\PYG{o}{,} \PYG{l+s}{\PYGZdq{}help\PYGZdq{}}\PYG{o}{,} \PYG{l+s}{\PYGZdq{}type\PYGZdq{}}\PYG{o}{)}\PYG{o}{)}\PYG{o}{;}
    \PYG{o}{\PYGZcb{}}\PYG{o}{\PYGZcb{}}
\PYG{o}{)}\PYG{o}{)}\PYG{o}{;}
\end{sphinxVerbatim}

\fvset{hllines={, ,}}%
\begin{sphinxVerbatim}[commandchars=\\\{\}]
\PYG{p}{\PYGZob{}}
    \PYG{n+nt}{\PYGZdq{}ean13\PYGZdq{}}\PYG{p}{:} \PYG{p}{\PYGZob{}}
        \PYG{n+nt}{\PYGZdq{}type\PYGZdq{}}\PYG{p}{:} \PYG{l+s+s2}{\PYGZdq{}char\PYGZdq{}}\PYG{p}{,}
        \PYG{n+nt}{\PYGZdq{}help\PYGZdq{}}\PYG{p}{:} \PYG{l+s+s2}{\PYGZdq{}BarCode\PYGZdq{}}\PYG{p}{,}
        \PYG{n+nt}{\PYGZdq{}string\PYGZdq{}}\PYG{p}{:} \PYG{l+s+s2}{\PYGZdq{}EAN13\PYGZdq{}}
    \PYG{p}{\PYGZcb{}}\PYG{p}{,}
    \PYG{n+nt}{\PYGZdq{}property\PYGZus{}account\PYGZus{}position\PYGZus{}id\PYGZdq{}}\PYG{p}{:} \PYG{p}{\PYGZob{}}
        \PYG{n+nt}{\PYGZdq{}type\PYGZdq{}}\PYG{p}{:} \PYG{l+s+s2}{\PYGZdq{}many2one\PYGZdq{}}\PYG{p}{,}
        \PYG{n+nt}{\PYGZdq{}help\PYGZdq{}}\PYG{p}{:} \PYG{l+s+s2}{\PYGZdq{}The fiscal position will determine taxes and accounts used for the partner.\PYGZdq{}}\PYG{p}{,}
        \PYG{n+nt}{\PYGZdq{}string\PYGZdq{}}\PYG{p}{:} \PYG{l+s+s2}{\PYGZdq{}Fiscal Position\PYGZdq{}}
    \PYG{p}{\PYGZcb{}}\PYG{p}{,}
    \PYG{n+nt}{\PYGZdq{}signup\PYGZus{}valid\PYGZdq{}}\PYG{p}{:} \PYG{p}{\PYGZob{}}
        \PYG{n+nt}{\PYGZdq{}type\PYGZdq{}}\PYG{p}{:} \PYG{l+s+s2}{\PYGZdq{}boolean\PYGZdq{}}\PYG{p}{,}
        \PYG{n+nt}{\PYGZdq{}help\PYGZdq{}}\PYG{p}{:} \PYG{l+s+s2}{\PYGZdq{}\PYGZdq{}}\PYG{p}{,}
        \PYG{n+nt}{\PYGZdq{}string\PYGZdq{}}\PYG{p}{:} \PYG{l+s+s2}{\PYGZdq{}Signup Token is Valid\PYGZdq{}}
    \PYG{p}{\PYGZcb{}}\PYG{p}{,}
    \PYG{n+nt}{\PYGZdq{}date\PYGZus{}localization\PYGZdq{}}\PYG{p}{:} \PYG{p}{\PYGZob{}}
        \PYG{n+nt}{\PYGZdq{}type\PYGZdq{}}\PYG{p}{:} \PYG{l+s+s2}{\PYGZdq{}date\PYGZdq{}}\PYG{p}{,}
        \PYG{n+nt}{\PYGZdq{}help\PYGZdq{}}\PYG{p}{:} \PYG{l+s+s2}{\PYGZdq{}\PYGZdq{}}\PYG{p}{,}
        \PYG{n+nt}{\PYGZdq{}string\PYGZdq{}}\PYG{p}{:} \PYG{l+s+s2}{\PYGZdq{}Geo Localization Date\PYGZdq{}}
    \PYG{p}{\PYGZcb{}}\PYG{p}{,}
    \PYG{n+nt}{\PYGZdq{}ref\PYGZus{}company\PYGZus{}ids\PYGZdq{}}\PYG{p}{:} \PYG{p}{\PYGZob{}}
        \PYG{n+nt}{\PYGZdq{}type\PYGZdq{}}\PYG{p}{:} \PYG{l+s+s2}{\PYGZdq{}one2many\PYGZdq{}}\PYG{p}{,}
        \PYG{n+nt}{\PYGZdq{}help\PYGZdq{}}\PYG{p}{:} \PYG{l+s+s2}{\PYGZdq{}\PYGZdq{}}\PYG{p}{,}
        \PYG{n+nt}{\PYGZdq{}string\PYGZdq{}}\PYG{p}{:} \PYG{l+s+s2}{\PYGZdq{}Companies that refers to partner\PYGZdq{}}
    \PYG{p}{\PYGZcb{}}\PYG{p}{,}
    \PYG{n+nt}{\PYGZdq{}sale\PYGZus{}order\PYGZus{}count\PYGZdq{}}\PYG{p}{:} \PYG{p}{\PYGZob{}}
        \PYG{n+nt}{\PYGZdq{}type\PYGZdq{}}\PYG{p}{:} \PYG{l+s+s2}{\PYGZdq{}integer\PYGZdq{}}\PYG{p}{,}
        \PYG{n+nt}{\PYGZdq{}help\PYGZdq{}}\PYG{p}{:} \PYG{l+s+s2}{\PYGZdq{}\PYGZdq{}}\PYG{p}{,}
        \PYG{n+nt}{\PYGZdq{}string\PYGZdq{}}\PYG{p}{:} \PYG{l+s+s2}{\PYGZdq{}\PYGZsh{} of Sales Order\PYGZdq{}}
    \PYG{p}{\PYGZcb{}}\PYG{p}{,}
    \PYG{n+nt}{\PYGZdq{}purchase\PYGZus{}order\PYGZus{}count\PYGZdq{}}\PYG{p}{:} \PYG{p}{\PYGZob{}}
        \PYG{n+nt}{\PYGZdq{}type\PYGZdq{}}\PYG{p}{:} \PYG{l+s+s2}{\PYGZdq{}integer\PYGZdq{}}\PYG{p}{,}
        \PYG{n+nt}{\PYGZdq{}help\PYGZdq{}}\PYG{p}{:} \PYG{l+s+s2}{\PYGZdq{}\PYGZdq{}}\PYG{p}{,}
        \PYG{n+nt}{\PYGZdq{}string\PYGZdq{}}\PYG{p}{:} \PYG{l+s+s2}{\PYGZdq{}\PYGZsh{} of Purchase Order\PYGZdq{}}
    \PYG{p}{\PYGZcb{}}\PYG{p}{,}
\end{sphinxVerbatim}


\subsubsection{Search and read}
\label{\detokenize{webservices/odoo:search-and-read}}
Because it is a very common task, Odoo provides a
\sphinxcode{\sphinxupquote{search\_read()}} shortcut which as its name suggests is
equivalent to a {\hyperref[\detokenize{reference/orm:odoo.models.Model.search}]{\sphinxcrossref{\sphinxcode{\sphinxupquote{search()}}}}} followed by a
{\hyperref[\detokenize{reference/orm:odoo.models.Model.read}]{\sphinxcrossref{\sphinxcode{\sphinxupquote{read()}}}}}, but avoids having to perform two requests
and keep ids around.

Its arguments are similar to {\hyperref[\detokenize{reference/orm:odoo.models.Model.search}]{\sphinxcrossref{\sphinxcode{\sphinxupquote{search()}}}}}’s, but it
can also take a list of \sphinxcode{\sphinxupquote{fields}} (like {\hyperref[\detokenize{reference/orm:odoo.models.Model.read}]{\sphinxcrossref{\sphinxcode{\sphinxupquote{read()}}}}},
if that list is not provided it will fetch all fields of matched records):
\begin{itemize}
\item {} Python 2
\item {} Ruby
\item {} PHP
\item {} Java
\end{itemize}

\fvset{hllines={, ,}}%
\begin{sphinxVerbatim}[commandchars=\\\{\}]
\PYG{n}{models}\PYG{o}{.}\PYG{n}{execute\PYGZus{}kw}\PYG{p}{(}\PYG{n}{db}\PYG{p}{,} \PYG{n}{uid}\PYG{p}{,} \PYG{n}{password}\PYG{p}{,}
    \PYG{l+s+s1}{\PYGZsq{}}\PYG{l+s+s1}{res.partner}\PYG{l+s+s1}{\PYGZsq{}}\PYG{p}{,} \PYG{l+s+s1}{\PYGZsq{}}\PYG{l+s+s1}{search\PYGZus{}read}\PYG{l+s+s1}{\PYGZsq{}}\PYG{p}{,}
    \PYG{p}{[}\PYG{p}{[}\PYG{p}{[}\PYG{l+s+s1}{\PYGZsq{}}\PYG{l+s+s1}{is\PYGZus{}company}\PYG{l+s+s1}{\PYGZsq{}}\PYG{p}{,} \PYG{l+s+s1}{\PYGZsq{}}\PYG{l+s+s1}{=}\PYG{l+s+s1}{\PYGZsq{}}\PYG{p}{,} \PYG{n+nb+bp}{True}\PYG{p}{]}\PYG{p}{,} \PYG{p}{[}\PYG{l+s+s1}{\PYGZsq{}}\PYG{l+s+s1}{customer}\PYG{l+s+s1}{\PYGZsq{}}\PYG{p}{,} \PYG{l+s+s1}{\PYGZsq{}}\PYG{l+s+s1}{=}\PYG{l+s+s1}{\PYGZsq{}}\PYG{p}{,} \PYG{n+nb+bp}{True}\PYG{p}{]}\PYG{p}{]}\PYG{p}{]}\PYG{p}{,}
    \PYG{p}{\PYGZob{}}\PYG{l+s+s1}{\PYGZsq{}}\PYG{l+s+s1}{fields}\PYG{l+s+s1}{\PYGZsq{}}\PYG{p}{:} \PYG{p}{[}\PYG{l+s+s1}{\PYGZsq{}}\PYG{l+s+s1}{name}\PYG{l+s+s1}{\PYGZsq{}}\PYG{p}{,} \PYG{l+s+s1}{\PYGZsq{}}\PYG{l+s+s1}{country\PYGZus{}id}\PYG{l+s+s1}{\PYGZsq{}}\PYG{p}{,} \PYG{l+s+s1}{\PYGZsq{}}\PYG{l+s+s1}{comment}\PYG{l+s+s1}{\PYGZsq{}}\PYG{p}{]}\PYG{p}{,} \PYG{l+s+s1}{\PYGZsq{}}\PYG{l+s+s1}{limit}\PYG{l+s+s1}{\PYGZsq{}}\PYG{p}{:} \PYG{l+m+mi}{5}\PYG{p}{\PYGZcb{}}\PYG{p}{)}
\end{sphinxVerbatim}

\fvset{hllines={, ,}}%
\begin{sphinxVerbatim}[commandchars=\\\{\}]
\PYG{n}{models}\PYG{o}{.}\PYG{n}{execute\PYGZus{}kw}\PYG{p}{(}\PYG{n}{db}\PYG{p}{,} \PYG{n}{uid}\PYG{p}{,} \PYG{n}{password}\PYG{p}{,}
    \PYG{l+s+s1}{\PYGZsq{}res.partner\PYGZsq{}}\PYG{p}{,} \PYG{l+s+s1}{\PYGZsq{}search\PYGZus{}read\PYGZsq{}}\PYG{p}{,}
    \PYG{o}{[}\PYG{o}{[}\PYG{o}{[}\PYG{l+s+s1}{\PYGZsq{}is\PYGZus{}company\PYGZsq{}}\PYG{p}{,} \PYG{l+s+s1}{\PYGZsq{}=\PYGZsq{}}\PYG{p}{,} \PYG{k+kp}{true}\PYG{o}{]}\PYG{p}{,} \PYG{o}{[}\PYG{l+s+s1}{\PYGZsq{}customer\PYGZsq{}}\PYG{p}{,} \PYG{l+s+s1}{\PYGZsq{}=\PYGZsq{}}\PYG{p}{,} \PYG{k+kp}{true}\PYG{o}{]}\PYG{o}{]}\PYG{o}{]}\PYG{p}{,}
    \PYG{p}{\PYGZob{}}\PYG{l+s+ss}{fields}\PYG{p}{:} \PYG{l+s+sx}{\PYGZpc{}w(}\PYG{l+s+sx}{name country\PYGZus{}id comment}\PYG{l+s+sx}{)}\PYG{p}{,} \PYG{l+s+ss}{limit}\PYG{p}{:} \PYG{l+m+mi}{5}\PYG{p}{\PYGZcb{}}\PYG{p}{)}
\end{sphinxVerbatim}

\fvset{hllines={, ,}}%
\begin{sphinxVerbatim}[commandchars=\\\{\}]
\PYG{n+nv}{\PYGZdl{}models}\PYG{o}{\PYGZhy{}\PYGZgt{}}\PYG{n+na}{execute\PYGZus{}kw}\PYG{p}{(}\PYG{n+nv}{\PYGZdl{}db}\PYG{p}{,} \PYG{n+nv}{\PYGZdl{}uid}\PYG{p}{,} \PYG{n+nv}{\PYGZdl{}password}\PYG{p}{,}
    \PYG{l+s+s1}{\PYGZsq{}res.partner\PYGZsq{}}\PYG{p}{,} \PYG{l+s+s1}{\PYGZsq{}search\PYGZus{}read\PYGZsq{}}\PYG{p}{,}
    \PYG{k}{array}\PYG{p}{(}\PYG{k}{array}\PYG{p}{(}\PYG{k}{array}\PYG{p}{(}\PYG{l+s+s1}{\PYGZsq{}is\PYGZus{}company\PYGZsq{}}\PYG{p}{,} \PYG{l+s+s1}{\PYGZsq{}=\PYGZsq{}}\PYG{p}{,} \PYG{k}{true}\PYG{p}{),}
                \PYG{k}{array}\PYG{p}{(}\PYG{l+s+s1}{\PYGZsq{}customer\PYGZsq{}}\PYG{p}{,} \PYG{l+s+s1}{\PYGZsq{}=\PYGZsq{}}\PYG{p}{,} \PYG{k}{true}\PYG{p}{))),}
    \PYG{k}{array}\PYG{p}{(}\PYG{l+s+s1}{\PYGZsq{}fields\PYGZsq{}}\PYG{o}{=\PYGZgt{}}\PYG{k}{array}\PYG{p}{(}\PYG{l+s+s1}{\PYGZsq{}name\PYGZsq{}}\PYG{p}{,} \PYG{l+s+s1}{\PYGZsq{}country\PYGZus{}id\PYGZsq{}}\PYG{p}{,} \PYG{l+s+s1}{\PYGZsq{}comment\PYGZsq{}}\PYG{p}{),} \PYG{l+s+s1}{\PYGZsq{}limit\PYGZsq{}}\PYG{o}{=\PYGZgt{}}\PYG{l+m+mi}{5}\PYG{p}{));}
\end{sphinxVerbatim}

\fvset{hllines={, ,}}%
\begin{sphinxVerbatim}[commandchars=\\\{\}]
\PYG{n}{asList}\PYG{o}{(}\PYG{o}{(}\PYG{n}{Object}\PYG{o}{[}\PYG{o}{]}\PYG{o}{)}\PYG{n}{models}\PYG{o}{.}\PYG{n+na}{execute}\PYG{o}{(}\PYG{l+s}{\PYGZdq{}execute\PYGZus{}kw\PYGZdq{}}\PYG{o}{,} \PYG{n}{asList}\PYG{o}{(}
    \PYG{n}{db}\PYG{o}{,} \PYG{n}{uid}\PYG{o}{,} \PYG{n}{password}\PYG{o}{,}
    \PYG{l+s}{\PYGZdq{}res.partner\PYGZdq{}}\PYG{o}{,} \PYG{l+s}{\PYGZdq{}search\PYGZus{}read\PYGZdq{}}\PYG{o}{,}
    \PYG{n}{asList}\PYG{o}{(}\PYG{n}{asList}\PYG{o}{(}
        \PYG{n}{asList}\PYG{o}{(}\PYG{l+s}{\PYGZdq{}is\PYGZus{}company\PYGZdq{}}\PYG{o}{,} \PYG{l+s}{\PYGZdq{}=\PYGZdq{}}\PYG{o}{,} \PYG{k+kc}{true}\PYG{o}{)}\PYG{o}{,}
        \PYG{n}{asList}\PYG{o}{(}\PYG{l+s}{\PYGZdq{}customer\PYGZdq{}}\PYG{o}{,} \PYG{l+s}{\PYGZdq{}=\PYGZdq{}}\PYG{o}{,} \PYG{k+kc}{true}\PYG{o}{)}\PYG{o}{)}\PYG{o}{)}\PYG{o}{,}
    \PYG{k}{new} \PYG{n}{HashMap}\PYG{o}{(}\PYG{o}{)} \PYG{o}{\PYGZob{}}\PYG{o}{\PYGZob{}}
        \PYG{n}{put}\PYG{o}{(}\PYG{l+s}{\PYGZdq{}fields\PYGZdq{}}\PYG{o}{,} \PYG{n}{asList}\PYG{o}{(}\PYG{l+s}{\PYGZdq{}name\PYGZdq{}}\PYG{o}{,} \PYG{l+s}{\PYGZdq{}country\PYGZus{}id\PYGZdq{}}\PYG{o}{,} \PYG{l+s}{\PYGZdq{}comment\PYGZdq{}}\PYG{o}{)}\PYG{o}{)}\PYG{o}{;}
        \PYG{n}{put}\PYG{o}{(}\PYG{l+s}{\PYGZdq{}limit\PYGZdq{}}\PYG{o}{,} \PYG{l+m+mi}{5}\PYG{o}{)}\PYG{o}{;}
    \PYG{o}{\PYGZcb{}}\PYG{o}{\PYGZcb{}}
\PYG{o}{)}\PYG{o}{)}\PYG{o}{)}\PYG{o}{;}
\end{sphinxVerbatim}

\fvset{hllines={, ,}}%
\begin{sphinxVerbatim}[commandchars=\\\{\}]
\PYG{p}{[}
    \PYG{p}{\PYGZob{}}
        \PYG{n+nt}{\PYGZdq{}comment\PYGZdq{}}\PYG{p}{:} \PYG{k+kc}{false}\PYG{p}{,}
        \PYG{n+nt}{\PYGZdq{}country\PYGZus{}id\PYGZdq{}}\PYG{p}{:} \PYG{p}{[} \PYG{l+m+mi}{21}\PYG{p}{,} \PYG{l+s+s2}{\PYGZdq{}Belgium\PYGZdq{}} \PYG{p}{]}\PYG{p}{,}
        \PYG{n+nt}{\PYGZdq{}id\PYGZdq{}}\PYG{p}{:} \PYG{l+m+mi}{7}\PYG{p}{,}
        \PYG{n+nt}{\PYGZdq{}name\PYGZdq{}}\PYG{p}{:} \PYG{l+s+s2}{\PYGZdq{}Agrolait\PYGZdq{}}
    \PYG{p}{\PYGZcb{}}\PYG{p}{,}
    \PYG{p}{\PYGZob{}}
        \PYG{n+nt}{\PYGZdq{}comment\PYGZdq{}}\PYG{p}{:} \PYG{k+kc}{false}\PYG{p}{,}
        \PYG{n+nt}{\PYGZdq{}country\PYGZus{}id\PYGZdq{}}\PYG{p}{:} \PYG{p}{[} \PYG{l+m+mi}{76}\PYG{p}{,} \PYG{l+s+s2}{\PYGZdq{}France\PYGZdq{}} \PYG{p}{]}\PYG{p}{,}
        \PYG{n+nt}{\PYGZdq{}id\PYGZdq{}}\PYG{p}{:} \PYG{l+m+mi}{18}\PYG{p}{,}
        \PYG{n+nt}{\PYGZdq{}name\PYGZdq{}}\PYG{p}{:} \PYG{l+s+s2}{\PYGZdq{}Axelor\PYGZdq{}}
    \PYG{p}{\PYGZcb{}}\PYG{p}{,}
    \PYG{p}{\PYGZob{}}
        \PYG{n+nt}{\PYGZdq{}comment\PYGZdq{}}\PYG{p}{:} \PYG{k+kc}{false}\PYG{p}{,}
        \PYG{n+nt}{\PYGZdq{}country\PYGZus{}id\PYGZdq{}}\PYG{p}{:} \PYG{p}{[} \PYG{l+m+mi}{233}\PYG{p}{,} \PYG{l+s+s2}{\PYGZdq{}United Kingdom\PYGZdq{}} \PYG{p}{]}\PYG{p}{,}
        \PYG{n+nt}{\PYGZdq{}id\PYGZdq{}}\PYG{p}{:} \PYG{l+m+mi}{12}\PYG{p}{,}
        \PYG{n+nt}{\PYGZdq{}name\PYGZdq{}}\PYG{p}{:} \PYG{l+s+s2}{\PYGZdq{}Bank Wealthy and sons\PYGZdq{}}
    \PYG{p}{\PYGZcb{}}\PYG{p}{,}
    \PYG{p}{\PYGZob{}}
        \PYG{n+nt}{\PYGZdq{}comment\PYGZdq{}}\PYG{p}{:} \PYG{k+kc}{false}\PYG{p}{,}
        \PYG{n+nt}{\PYGZdq{}country\PYGZus{}id\PYGZdq{}}\PYG{p}{:} \PYG{p}{[} \PYG{l+m+mi}{105}\PYG{p}{,} \PYG{l+s+s2}{\PYGZdq{}India\PYGZdq{}} \PYG{p}{]}\PYG{p}{,}
        \PYG{n+nt}{\PYGZdq{}id\PYGZdq{}}\PYG{p}{:} \PYG{l+m+mi}{14}\PYG{p}{,}
        \PYG{n+nt}{\PYGZdq{}name\PYGZdq{}}\PYG{p}{:} \PYG{l+s+s2}{\PYGZdq{}Best Designers\PYGZdq{}}
    \PYG{p}{\PYGZcb{}}\PYG{p}{,}
    \PYG{p}{\PYGZob{}}
        \PYG{n+nt}{\PYGZdq{}comment\PYGZdq{}}\PYG{p}{:} \PYG{k+kc}{false}\PYG{p}{,}
        \PYG{n+nt}{\PYGZdq{}country\PYGZus{}id\PYGZdq{}}\PYG{p}{:} \PYG{p}{[} \PYG{l+m+mi}{76}\PYG{p}{,} \PYG{l+s+s2}{\PYGZdq{}France\PYGZdq{}} \PYG{p}{]}\PYG{p}{,}
        \PYG{n+nt}{\PYGZdq{}id\PYGZdq{}}\PYG{p}{:} \PYG{l+m+mi}{17}\PYG{p}{,}
        \PYG{n+nt}{\PYGZdq{}name\PYGZdq{}}\PYG{p}{:} \PYG{l+s+s2}{\PYGZdq{}Camptocamp\PYGZdq{}}
    \PYG{p}{\PYGZcb{}}
\PYG{p}{]}
\end{sphinxVerbatim}


\subsubsection{Create records}
\label{\detokenize{webservices/odoo:create-records}}
Records of a model are created using {\hyperref[\detokenize{reference/orm:odoo.models.Model.create}]{\sphinxcrossref{\sphinxcode{\sphinxupquote{create()}}}}}. The
method will create a single record and return its database identifier.

{\hyperref[\detokenize{reference/orm:odoo.models.Model.create}]{\sphinxcrossref{\sphinxcode{\sphinxupquote{create()}}}}} takes a mapping of fields to values, used
to initialize the record. For any field which has a default value and is not
set through the mapping argument, the default value will be used.
\begin{itemize}
\item {} Python 2
\item {} Ruby
\item {} PHP
\item {} Java
\end{itemize}

\fvset{hllines={, ,}}%
\begin{sphinxVerbatim}[commandchars=\\\{\}]
\PYG{n+nb}{id} \PYG{o}{=} \PYG{n}{models}\PYG{o}{.}\PYG{n}{execute\PYGZus{}kw}\PYG{p}{(}\PYG{n}{db}\PYG{p}{,} \PYG{n}{uid}\PYG{p}{,} \PYG{n}{password}\PYG{p}{,} \PYG{l+s+s1}{\PYGZsq{}}\PYG{l+s+s1}{res.partner}\PYG{l+s+s1}{\PYGZsq{}}\PYG{p}{,} \PYG{l+s+s1}{\PYGZsq{}}\PYG{l+s+s1}{create}\PYG{l+s+s1}{\PYGZsq{}}\PYG{p}{,} \PYG{p}{[}\PYG{p}{\PYGZob{}}
    \PYG{l+s+s1}{\PYGZsq{}}\PYG{l+s+s1}{name}\PYG{l+s+s1}{\PYGZsq{}}\PYG{p}{:} \PYG{l+s+s2}{\PYGZdq{}}\PYG{l+s+s2}{New Partner}\PYG{l+s+s2}{\PYGZdq{}}\PYG{p}{,}
\PYG{p}{\PYGZcb{}}\PYG{p}{]}\PYG{p}{)}
\end{sphinxVerbatim}

\fvset{hllines={, ,}}%
\begin{sphinxVerbatim}[commandchars=\\\{\}]
\PYG{n+nb}{id} \PYG{o}{=} \PYG{n}{models}\PYG{o}{.}\PYG{n}{execute\PYGZus{}kw}\PYG{p}{(}\PYG{n}{db}\PYG{p}{,} \PYG{n}{uid}\PYG{p}{,} \PYG{n}{password}\PYG{p}{,} \PYG{l+s+s1}{\PYGZsq{}res.partner\PYGZsq{}}\PYG{p}{,} \PYG{l+s+s1}{\PYGZsq{}create\PYGZsq{}}\PYG{p}{,} \PYG{o}{[}\PYG{p}{\PYGZob{}}
    \PYG{n+nb}{name}\PYG{p}{:} \PYG{l+s+s2}{\PYGZdq{}}\PYG{l+s+s2}{New Partner}\PYG{l+s+s2}{\PYGZdq{}}\PYG{p}{,}
\PYG{p}{\PYGZcb{}}\PYG{o}{]}\PYG{p}{)}
\end{sphinxVerbatim}

\fvset{hllines={, ,}}%
\begin{sphinxVerbatim}[commandchars=\\\{\}]
\PYG{n+nv}{\PYGZdl{}id} \PYG{o}{=} \PYG{n+nv}{\PYGZdl{}models}\PYG{o}{\PYGZhy{}\PYGZgt{}}\PYG{n+na}{execute\PYGZus{}kw}\PYG{p}{(}\PYG{n+nv}{\PYGZdl{}db}\PYG{p}{,} \PYG{n+nv}{\PYGZdl{}uid}\PYG{p}{,} \PYG{n+nv}{\PYGZdl{}password}\PYG{p}{,}
    \PYG{l+s+s1}{\PYGZsq{}res.partner\PYGZsq{}}\PYG{p}{,} \PYG{l+s+s1}{\PYGZsq{}create\PYGZsq{}}\PYG{p}{,}
    \PYG{k}{array}\PYG{p}{(}\PYG{k}{array}\PYG{p}{(}\PYG{l+s+s1}{\PYGZsq{}name\PYGZsq{}}\PYG{o}{=\PYGZgt{}}\PYG{l+s+s2}{\PYGZdq{}}\PYG{l+s+s2}{New Partner}\PYG{l+s+s2}{\PYGZdq{}}\PYG{p}{)));}
\end{sphinxVerbatim}

\fvset{hllines={, ,}}%
\begin{sphinxVerbatim}[commandchars=\\\{\}]
\PYG{k+kd}{final} \PYG{n}{Integer} \PYG{n}{id} \PYG{o}{=} \PYG{o}{(}\PYG{n}{Integer}\PYG{o}{)}\PYG{n}{models}\PYG{o}{.}\PYG{n+na}{execute}\PYG{o}{(}\PYG{l+s}{\PYGZdq{}execute\PYGZus{}kw\PYGZdq{}}\PYG{o}{,} \PYG{n}{asList}\PYG{o}{(}
    \PYG{n}{db}\PYG{o}{,} \PYG{n}{uid}\PYG{o}{,} \PYG{n}{password}\PYG{o}{,}
    \PYG{l+s}{\PYGZdq{}res.partner\PYGZdq{}}\PYG{o}{,} \PYG{l+s}{\PYGZdq{}create\PYGZdq{}}\PYG{o}{,}
    \PYG{n}{asList}\PYG{o}{(}\PYG{k}{new} \PYG{n}{HashMap}\PYG{o}{(}\PYG{o}{)} \PYG{o}{\PYGZob{}}\PYG{o}{\PYGZob{}} \PYG{n}{put}\PYG{o}{(}\PYG{l+s}{\PYGZdq{}name\PYGZdq{}}\PYG{o}{,} \PYG{l+s}{\PYGZdq{}New Partner\PYGZdq{}}\PYG{o}{)}\PYG{o}{;} \PYG{o}{\PYGZcb{}}\PYG{o}{\PYGZcb{}}\PYG{o}{)}
\PYG{o}{)}\PYG{o}{)}\PYG{o}{;}
\end{sphinxVerbatim}

\fvset{hllines={, ,}}%
\begin{sphinxVerbatim}[commandchars=\\\{\}]
\PYG{l+m+mi}{78}
\end{sphinxVerbatim}

\begin{sphinxadmonition}{warning}{Warning:}
while most value types are what would be expected (integer for
{\hyperref[\detokenize{reference/orm:odoo.fields.Integer}]{\sphinxcrossref{\sphinxcode{\sphinxupquote{Integer}}}}}, string for {\hyperref[\detokenize{reference/orm:odoo.fields.Char}]{\sphinxcrossref{\sphinxcode{\sphinxupquote{Char}}}}}
or {\hyperref[\detokenize{reference/orm:odoo.fields.Text}]{\sphinxcrossref{\sphinxcode{\sphinxupquote{Text}}}}}),
\begin{itemize}
\item {} 
{\hyperref[\detokenize{reference/orm:odoo.fields.Date}]{\sphinxcrossref{\sphinxcode{\sphinxupquote{Date}}}}}, {\hyperref[\detokenize{reference/orm:odoo.fields.Datetime}]{\sphinxcrossref{\sphinxcode{\sphinxupquote{Datetime}}}}} and
\sphinxcode{\sphinxupquote{Binary}} fields use string values

\item {} 
{\hyperref[\detokenize{reference/orm:odoo.fields.One2many}]{\sphinxcrossref{\sphinxcode{\sphinxupquote{One2many}}}}} and {\hyperref[\detokenize{reference/orm:odoo.fields.Many2many}]{\sphinxcrossref{\sphinxcode{\sphinxupquote{Many2many}}}}}
use a special command protocol detailed in {\hyperref[\detokenize{reference/orm:odoo.models.Model.write}]{\sphinxcrossref{\sphinxcode{\sphinxupquote{the documentation to
the write method}}}}}.

\end{itemize}
\end{sphinxadmonition}


\subsubsection{Update records}
\label{\detokenize{webservices/odoo:update-records}}
Records can be updated using {\hyperref[\detokenize{reference/orm:odoo.models.Model.write}]{\sphinxcrossref{\sphinxcode{\sphinxupquote{write()}}}}}, it takes
a list of records to update and a mapping of updated fields to values similar
to {\hyperref[\detokenize{reference/orm:odoo.models.Model.create}]{\sphinxcrossref{\sphinxcode{\sphinxupquote{create()}}}}}.

Multiple records can be updated simultanously, but they will all get the same
values for the fields being set. It is not currently possible to perform
“computed” updates (where the value being set depends on an existing value of
a record).
\begin{itemize}
\item {} Python 2
\item {} Ruby
\item {} PHP
\item {} Java
\end{itemize}

\fvset{hllines={, ,}}%
\begin{sphinxVerbatim}[commandchars=\\\{\}]
\PYG{n}{models}\PYG{o}{.}\PYG{n}{execute\PYGZus{}kw}\PYG{p}{(}\PYG{n}{db}\PYG{p}{,} \PYG{n}{uid}\PYG{p}{,} \PYG{n}{password}\PYG{p}{,} \PYG{l+s+s1}{\PYGZsq{}}\PYG{l+s+s1}{res.partner}\PYG{l+s+s1}{\PYGZsq{}}\PYG{p}{,} \PYG{l+s+s1}{\PYGZsq{}}\PYG{l+s+s1}{write}\PYG{l+s+s1}{\PYGZsq{}}\PYG{p}{,} \PYG{p}{[}\PYG{p}{[}\PYG{n+nb}{id}\PYG{p}{]}\PYG{p}{,} \PYG{p}{\PYGZob{}}
    \PYG{l+s+s1}{\PYGZsq{}}\PYG{l+s+s1}{name}\PYG{l+s+s1}{\PYGZsq{}}\PYG{p}{:} \PYG{l+s+s2}{\PYGZdq{}}\PYG{l+s+s2}{Newer partner}\PYG{l+s+s2}{\PYGZdq{}}
\PYG{p}{\PYGZcb{}}\PYG{p}{]}\PYG{p}{)}
\PYG{c+c1}{\PYGZsh{} get record name after having changed it}
\PYG{n}{models}\PYG{o}{.}\PYG{n}{execute\PYGZus{}kw}\PYG{p}{(}\PYG{n}{db}\PYG{p}{,} \PYG{n}{uid}\PYG{p}{,} \PYG{n}{password}\PYG{p}{,} \PYG{l+s+s1}{\PYGZsq{}}\PYG{l+s+s1}{res.partner}\PYG{l+s+s1}{\PYGZsq{}}\PYG{p}{,} \PYG{l+s+s1}{\PYGZsq{}}\PYG{l+s+s1}{name\PYGZus{}get}\PYG{l+s+s1}{\PYGZsq{}}\PYG{p}{,} \PYG{p}{[}\PYG{p}{[}\PYG{n+nb}{id}\PYG{p}{]}\PYG{p}{]}\PYG{p}{)}
\end{sphinxVerbatim}

\fvset{hllines={, ,}}%
\begin{sphinxVerbatim}[commandchars=\\\{\}]
\PYG{n}{models}\PYG{o}{.}\PYG{n}{execute\PYGZus{}kw}\PYG{p}{(}\PYG{n}{db}\PYG{p}{,} \PYG{n}{uid}\PYG{p}{,} \PYG{n}{password}\PYG{p}{,} \PYG{l+s+s1}{\PYGZsq{}res.partner\PYGZsq{}}\PYG{p}{,} \PYG{l+s+s1}{\PYGZsq{}write\PYGZsq{}}\PYG{p}{,} \PYG{o}{[}\PYG{o}{[}\PYG{n+nb}{id}\PYG{o}{]}\PYG{p}{,} \PYG{p}{\PYGZob{}}
    \PYG{n+nb}{name}\PYG{p}{:} \PYG{l+s+s2}{\PYGZdq{}}\PYG{l+s+s2}{Newer partner}\PYG{l+s+s2}{\PYGZdq{}}
\PYG{p}{\PYGZcb{}}\PYG{o}{]}\PYG{p}{)}
\PYG{c+c1}{\PYGZsh{} get record name after having changed it}
\PYG{n}{models}\PYG{o}{.}\PYG{n}{execute\PYGZus{}kw}\PYG{p}{(}\PYG{n}{db}\PYG{p}{,} \PYG{n}{uid}\PYG{p}{,} \PYG{n}{password}\PYG{p}{,} \PYG{l+s+s1}{\PYGZsq{}res.partner\PYGZsq{}}\PYG{p}{,} \PYG{l+s+s1}{\PYGZsq{}name\PYGZus{}get\PYGZsq{}}\PYG{p}{,} \PYG{o}{[}\PYG{o}{[}\PYG{n+nb}{id}\PYG{o}{]}\PYG{o}{]}\PYG{p}{)}
\end{sphinxVerbatim}

\fvset{hllines={, ,}}%
\begin{sphinxVerbatim}[commandchars=\\\{\}]
\PYG{n+nv}{\PYGZdl{}models}\PYG{o}{\PYGZhy{}\PYGZgt{}}\PYG{n+na}{execute\PYGZus{}kw}\PYG{p}{(}\PYG{n+nv}{\PYGZdl{}db}\PYG{p}{,} \PYG{n+nv}{\PYGZdl{}uid}\PYG{p}{,} \PYG{n+nv}{\PYGZdl{}password}\PYG{p}{,} \PYG{l+s+s1}{\PYGZsq{}res.partner\PYGZsq{}}\PYG{p}{,} \PYG{l+s+s1}{\PYGZsq{}write\PYGZsq{}}\PYG{p}{,}
    \PYG{k}{array}\PYG{p}{(}\PYG{k}{array}\PYG{p}{(}\PYG{n+nv}{\PYGZdl{}id}\PYG{p}{),} \PYG{k}{array}\PYG{p}{(}\PYG{l+s+s1}{\PYGZsq{}name\PYGZsq{}}\PYG{o}{=\PYGZgt{}}\PYG{l+s+s2}{\PYGZdq{}}\PYG{l+s+s2}{Newer partner}\PYG{l+s+s2}{\PYGZdq{}}\PYG{p}{)));}
\PYG{c+c1}{// get record name after having changed it}
\PYG{n+nv}{\PYGZdl{}models}\PYG{o}{\PYGZhy{}\PYGZgt{}}\PYG{n+na}{execute\PYGZus{}kw}\PYG{p}{(}\PYG{n+nv}{\PYGZdl{}db}\PYG{p}{,} \PYG{n+nv}{\PYGZdl{}uid}\PYG{p}{,} \PYG{n+nv}{\PYGZdl{}password}\PYG{p}{,}
    \PYG{l+s+s1}{\PYGZsq{}res.partner\PYGZsq{}}\PYG{p}{,} \PYG{l+s+s1}{\PYGZsq{}name\PYGZus{}get\PYGZsq{}}\PYG{p}{,} \PYG{k}{array}\PYG{p}{(}\PYG{k}{array}\PYG{p}{(}\PYG{n+nv}{\PYGZdl{}id}\PYG{p}{)));}
\end{sphinxVerbatim}

\fvset{hllines={, ,}}%
\begin{sphinxVerbatim}[commandchars=\\\{\}]
\PYG{n}{models}\PYG{o}{.}\PYG{n+na}{execute}\PYG{o}{(}\PYG{l+s}{\PYGZdq{}execute\PYGZus{}kw\PYGZdq{}}\PYG{o}{,} \PYG{n}{asList}\PYG{o}{(}
    \PYG{n}{db}\PYG{o}{,} \PYG{n}{uid}\PYG{o}{,} \PYG{n}{password}\PYG{o}{,}
    \PYG{l+s}{\PYGZdq{}res.partner\PYGZdq{}}\PYG{o}{,} \PYG{l+s}{\PYGZdq{}write\PYGZdq{}}\PYG{o}{,}
    \PYG{n}{asList}\PYG{o}{(}
        \PYG{n}{asList}\PYG{o}{(}\PYG{n}{id}\PYG{o}{)}\PYG{o}{,}
        \PYG{k}{new} \PYG{n}{HashMap}\PYG{o}{(}\PYG{o}{)} \PYG{o}{\PYGZob{}}\PYG{o}{\PYGZob{}} \PYG{n}{put}\PYG{o}{(}\PYG{l+s}{\PYGZdq{}name\PYGZdq{}}\PYG{o}{,} \PYG{l+s}{\PYGZdq{}Newer Partner\PYGZdq{}}\PYG{o}{)}\PYG{o}{;} \PYG{o}{\PYGZcb{}}\PYG{o}{\PYGZcb{}}
    \PYG{o}{)}
\PYG{o}{)}\PYG{o}{)}\PYG{o}{;}
\PYG{c+c1}{// get record name after having changed it}
\PYG{n}{asList}\PYG{o}{(}\PYG{o}{(}\PYG{n}{Object}\PYG{o}{[}\PYG{o}{]}\PYG{o}{)}\PYG{n}{models}\PYG{o}{.}\PYG{n+na}{execute}\PYG{o}{(}\PYG{l+s}{\PYGZdq{}execute\PYGZus{}kw\PYGZdq{}}\PYG{o}{,} \PYG{n}{asList}\PYG{o}{(}
    \PYG{n}{db}\PYG{o}{,} \PYG{n}{uid}\PYG{o}{,} \PYG{n}{password}\PYG{o}{,}
    \PYG{l+s}{\PYGZdq{}res.partner\PYGZdq{}}\PYG{o}{,} \PYG{l+s}{\PYGZdq{}name\PYGZus{}get\PYGZdq{}}\PYG{o}{,}
    \PYG{n}{asList}\PYG{o}{(}\PYG{n}{asList}\PYG{o}{(}\PYG{n}{id}\PYG{o}{)}\PYG{o}{)}
\PYG{o}{)}\PYG{o}{)}\PYG{o}{)}\PYG{o}{;}
\end{sphinxVerbatim}

\fvset{hllines={, ,}}%
\begin{sphinxVerbatim}[commandchars=\\\{\}]
\PYG{p}{[}\PYG{p}{[}\PYG{l+m+mi}{78}\PYG{p}{,} \PYG{l+s+s2}{\PYGZdq{}Newer partner\PYGZdq{}}\PYG{p}{]}\PYG{p}{]}
\end{sphinxVerbatim}


\subsubsection{Delete records}
\label{\detokenize{webservices/odoo:delete-records}}
Records can be deleted in bulk by providing their ids to
{\hyperref[\detokenize{reference/orm:odoo.models.Model.unlink}]{\sphinxcrossref{\sphinxcode{\sphinxupquote{unlink()}}}}}.
\begin{itemize}
\item {} Python 2
\item {} Ruby
\item {} PHP
\item {} Java
\end{itemize}

\fvset{hllines={, ,}}%
\begin{sphinxVerbatim}[commandchars=\\\{\}]
\PYG{n}{models}\PYG{o}{.}\PYG{n}{execute\PYGZus{}kw}\PYG{p}{(}\PYG{n}{db}\PYG{p}{,} \PYG{n}{uid}\PYG{p}{,} \PYG{n}{password}\PYG{p}{,} \PYG{l+s+s1}{\PYGZsq{}}\PYG{l+s+s1}{res.partner}\PYG{l+s+s1}{\PYGZsq{}}\PYG{p}{,} \PYG{l+s+s1}{\PYGZsq{}}\PYG{l+s+s1}{unlink}\PYG{l+s+s1}{\PYGZsq{}}\PYG{p}{,} \PYG{p}{[}\PYG{p}{[}\PYG{n+nb}{id}\PYG{p}{]}\PYG{p}{]}\PYG{p}{)}
\PYG{c+c1}{\PYGZsh{} check if the deleted record is still in the database}
\PYG{n}{models}\PYG{o}{.}\PYG{n}{execute\PYGZus{}kw}\PYG{p}{(}\PYG{n}{db}\PYG{p}{,} \PYG{n}{uid}\PYG{p}{,} \PYG{n}{password}\PYG{p}{,}
    \PYG{l+s+s1}{\PYGZsq{}}\PYG{l+s+s1}{res.partner}\PYG{l+s+s1}{\PYGZsq{}}\PYG{p}{,} \PYG{l+s+s1}{\PYGZsq{}}\PYG{l+s+s1}{search}\PYG{l+s+s1}{\PYGZsq{}}\PYG{p}{,} \PYG{p}{[}\PYG{p}{[}\PYG{p}{[}\PYG{l+s+s1}{\PYGZsq{}}\PYG{l+s+s1}{id}\PYG{l+s+s1}{\PYGZsq{}}\PYG{p}{,} \PYG{l+s+s1}{\PYGZsq{}}\PYG{l+s+s1}{=}\PYG{l+s+s1}{\PYGZsq{}}\PYG{p}{,} \PYG{n+nb}{id}\PYG{p}{]}\PYG{p}{]}\PYG{p}{]}\PYG{p}{)}
\end{sphinxVerbatim}

\fvset{hllines={, ,}}%
\begin{sphinxVerbatim}[commandchars=\\\{\}]
\PYG{n}{models}\PYG{o}{.}\PYG{n}{execute\PYGZus{}kw}\PYG{p}{(}\PYG{n}{db}\PYG{p}{,} \PYG{n}{uid}\PYG{p}{,} \PYG{n}{password}\PYG{p}{,} \PYG{l+s+s1}{\PYGZsq{}res.partner\PYGZsq{}}\PYG{p}{,} \PYG{l+s+s1}{\PYGZsq{}unlink\PYGZsq{}}\PYG{p}{,} \PYG{o}{[}\PYG{o}{[}\PYG{n+nb}{id}\PYG{o}{]}\PYG{o}{]}\PYG{p}{)}
\PYG{c+c1}{\PYGZsh{} check if the deleted record is still in the database}
\PYG{n}{models}\PYG{o}{.}\PYG{n}{execute\PYGZus{}kw}\PYG{p}{(}\PYG{n}{db}\PYG{p}{,} \PYG{n}{uid}\PYG{p}{,} \PYG{n}{password}\PYG{p}{,}
    \PYG{l+s+s1}{\PYGZsq{}res.partner\PYGZsq{}}\PYG{p}{,} \PYG{l+s+s1}{\PYGZsq{}search\PYGZsq{}}\PYG{p}{,} \PYG{o}{[}\PYG{o}{[}\PYG{o}{[}\PYG{l+s+s1}{\PYGZsq{}id\PYGZsq{}}\PYG{p}{,} \PYG{l+s+s1}{\PYGZsq{}=\PYGZsq{}}\PYG{p}{,} \PYG{n+nb}{id}\PYG{o}{]}\PYG{o}{]}\PYG{o}{]}\PYG{p}{)}
\end{sphinxVerbatim}

\fvset{hllines={, ,}}%
\begin{sphinxVerbatim}[commandchars=\\\{\}]
\PYG{n+nv}{\PYGZdl{}models}\PYG{o}{\PYGZhy{}\PYGZgt{}}\PYG{n+na}{execute\PYGZus{}kw}\PYG{p}{(}\PYG{n+nv}{\PYGZdl{}db}\PYG{p}{,} \PYG{n+nv}{\PYGZdl{}uid}\PYG{p}{,} \PYG{n+nv}{\PYGZdl{}password}\PYG{p}{,}
    \PYG{l+s+s1}{\PYGZsq{}res.partner\PYGZsq{}}\PYG{p}{,} \PYG{l+s+s1}{\PYGZsq{}unlink\PYGZsq{}}\PYG{p}{,}
    \PYG{k}{array}\PYG{p}{(}\PYG{k}{array}\PYG{p}{(}\PYG{n+nv}{\PYGZdl{}id}\PYG{p}{)));}
\PYG{c+c1}{// check if the deleted record is still in the database}
\PYG{n+nv}{\PYGZdl{}models}\PYG{o}{\PYGZhy{}\PYGZgt{}}\PYG{n+na}{execute\PYGZus{}kw}\PYG{p}{(}\PYG{n+nv}{\PYGZdl{}db}\PYG{p}{,} \PYG{n+nv}{\PYGZdl{}uid}\PYG{p}{,} \PYG{n+nv}{\PYGZdl{}password}\PYG{p}{,}
    \PYG{l+s+s1}{\PYGZsq{}res.partner\PYGZsq{}}\PYG{p}{,} \PYG{l+s+s1}{\PYGZsq{}search\PYGZsq{}}\PYG{p}{,}
    \PYG{k}{array}\PYG{p}{(}\PYG{k}{array}\PYG{p}{(}\PYG{k}{array}\PYG{p}{(}\PYG{l+s+s1}{\PYGZsq{}id\PYGZsq{}}\PYG{p}{,} \PYG{l+s+s1}{\PYGZsq{}=\PYGZsq{}}\PYG{p}{,} \PYG{n+nv}{\PYGZdl{}id}\PYG{p}{))));}
\end{sphinxVerbatim}

\fvset{hllines={, ,}}%
\begin{sphinxVerbatim}[commandchars=\\\{\}]
\PYG{n}{models}\PYG{o}{.}\PYG{n+na}{execute}\PYG{o}{(}\PYG{l+s}{\PYGZdq{}execute\PYGZus{}kw\PYGZdq{}}\PYG{o}{,} \PYG{n}{asList}\PYG{o}{(}
    \PYG{n}{db}\PYG{o}{,} \PYG{n}{uid}\PYG{o}{,} \PYG{n}{password}\PYG{o}{,}
    \PYG{l+s}{\PYGZdq{}res.partner\PYGZdq{}}\PYG{o}{,} \PYG{l+s}{\PYGZdq{}unlink\PYGZdq{}}\PYG{o}{,}
    \PYG{n}{asList}\PYG{o}{(}\PYG{n}{asList}\PYG{o}{(}\PYG{n}{id}\PYG{o}{)}\PYG{o}{)}\PYG{o}{)}\PYG{o}{)}\PYG{o}{;}
\PYG{c+c1}{// check if the deleted record is still in the database}
\PYG{n}{asList}\PYG{o}{(}\PYG{o}{(}\PYG{n}{Object}\PYG{o}{[}\PYG{o}{]}\PYG{o}{)}\PYG{n}{models}\PYG{o}{.}\PYG{n+na}{execute}\PYG{o}{(}\PYG{l+s}{\PYGZdq{}execute\PYGZus{}kw\PYGZdq{}}\PYG{o}{,} \PYG{n}{asList}\PYG{o}{(}
    \PYG{n}{db}\PYG{o}{,} \PYG{n}{uid}\PYG{o}{,} \PYG{n}{password}\PYG{o}{,}
    \PYG{l+s}{\PYGZdq{}res.partner\PYGZdq{}}\PYG{o}{,} \PYG{l+s}{\PYGZdq{}search\PYGZdq{}}\PYG{o}{,}
    \PYG{n}{asList}\PYG{o}{(}\PYG{n}{asList}\PYG{o}{(}\PYG{n}{asList}\PYG{o}{(}\PYG{l+s}{\PYGZdq{}id\PYGZdq{}}\PYG{o}{,} \PYG{l+s}{\PYGZdq{}=\PYGZdq{}}\PYG{o}{,} \PYG{l+m+mi}{78}\PYG{o}{)}\PYG{o}{)}\PYG{o}{)}
\PYG{o}{)}\PYG{o}{)}\PYG{o}{)}\PYG{o}{;}
\end{sphinxVerbatim}

\fvset{hllines={, ,}}%
\begin{sphinxVerbatim}[commandchars=\\\{\}]
\PYG{p}{[}\PYG{p}{]}
\end{sphinxVerbatim}


\subsubsection{Inspection and introspection}
\label{\detokenize{webservices/odoo:inspection-and-introspection}}
While we previously used {\hyperref[\detokenize{reference/orm:odoo.models.Model.fields_get}]{\sphinxcrossref{\sphinxcode{\sphinxupquote{fields\_get()}}}}} to query a
model and have been using an arbitrary model from the start, Odoo stores
most model metadata inside a few meta-models which allow both querying the
system and altering models and fields (with some limitations) on the fly over
XML-RPC.


\paragraph{\sphinxstyleliteralintitle{\sphinxupquote{ir.model}}}
\label{\detokenize{webservices/odoo:ir-model}}\label{\detokenize{webservices/odoo:reference-webservice-inspection-models}}
Provides information about Odoo models via its various fields
\begin{description}
\item[{\sphinxcode{\sphinxupquote{name}}}] \leavevmode
a human-readable description of the model

\item[{\sphinxcode{\sphinxupquote{model}}}] \leavevmode
the name of each model in the system

\item[{\sphinxcode{\sphinxupquote{state}}}] \leavevmode
whether the model was generated in Python code (\sphinxcode{\sphinxupquote{base}}) or by creating
an \sphinxcode{\sphinxupquote{ir.model}} record (\sphinxcode{\sphinxupquote{manual}})

\item[{\sphinxcode{\sphinxupquote{field\_id}}}] \leavevmode
list of the model’s fields through a {\hyperref[\detokenize{reference/orm:odoo.fields.One2many}]{\sphinxcrossref{\sphinxcode{\sphinxupquote{One2many}}}}} to
{\hyperref[\detokenize{webservices/odoo:reference-webservice-inspection-fields}]{\sphinxcrossref{\DUrole{std,std-ref}{ir.model.fields}}}}

\item[{\sphinxcode{\sphinxupquote{view\_ids}}}] \leavevmode
{\hyperref[\detokenize{reference/orm:odoo.fields.One2many}]{\sphinxcrossref{\sphinxcode{\sphinxupquote{One2many}}}}} to the {\hyperref[\detokenize{reference/views:reference-views}]{\sphinxcrossref{\DUrole{std,std-ref}{Views}}}} defined
for the model

\item[{\sphinxcode{\sphinxupquote{access\_ids}}}] \leavevmode
{\hyperref[\detokenize{reference/orm:odoo.fields.One2many}]{\sphinxcrossref{\sphinxcode{\sphinxupquote{One2many}}}}} relation to the
{\hyperref[\detokenize{reference/security:reference-security-acl}]{\sphinxcrossref{\DUrole{std,std-ref}{Access Control}}}} set on the model

\end{description}

\sphinxcode{\sphinxupquote{ir.model}} can be used to
\begin{itemize}
\item {} 
query the system for installed models (as a precondition to operations
on the model or to explore the system’s content)

\item {} 
get information about a specific model (generally by listing the fields
associated with it)

\item {} 
create new models dynamically over RPC

\end{itemize}

\begin{sphinxadmonition}{warning}{Warning:}\begin{itemize}
\item {} 
“custom” model names must start with \sphinxcode{\sphinxupquote{x\_}}

\item {} 
the \sphinxcode{\sphinxupquote{state}} must be provided and \sphinxcode{\sphinxupquote{manual}}, otherwise the model will
not be loaded

\item {} 
it is not possible to add new \sphinxstyleemphasis{methods} to a custom model, only fields

\end{itemize}
\end{sphinxadmonition}

a custom model will initially contain only the “built-in” fields available
on all models:
\begin{itemize}
\item {} Python 2
\item {} PHP
\item {} Ruby
\item {} Java
\end{itemize}

\fvset{hllines={, ,}}%
\begin{sphinxVerbatim}[commandchars=\\\{\}]
\PYG{n}{models}\PYG{o}{.}\PYG{n}{execute\PYGZus{}kw}\PYG{p}{(}\PYG{n}{db}\PYG{p}{,} \PYG{n}{uid}\PYG{p}{,} \PYG{n}{password}\PYG{p}{,} \PYG{l+s+s1}{\PYGZsq{}}\PYG{l+s+s1}{ir.model}\PYG{l+s+s1}{\PYGZsq{}}\PYG{p}{,} \PYG{l+s+s1}{\PYGZsq{}}\PYG{l+s+s1}{create}\PYG{l+s+s1}{\PYGZsq{}}\PYG{p}{,} \PYG{p}{[}\PYG{p}{\PYGZob{}}
    \PYG{l+s+s1}{\PYGZsq{}}\PYG{l+s+s1}{name}\PYG{l+s+s1}{\PYGZsq{}}\PYG{p}{:} \PYG{l+s+s2}{\PYGZdq{}}\PYG{l+s+s2}{Custom Model}\PYG{l+s+s2}{\PYGZdq{}}\PYG{p}{,}
    \PYG{l+s+s1}{\PYGZsq{}}\PYG{l+s+s1}{model}\PYG{l+s+s1}{\PYGZsq{}}\PYG{p}{:} \PYG{l+s+s2}{\PYGZdq{}}\PYG{l+s+s2}{x\PYGZus{}custom\PYGZus{}model}\PYG{l+s+s2}{\PYGZdq{}}\PYG{p}{,}
    \PYG{l+s+s1}{\PYGZsq{}}\PYG{l+s+s1}{state}\PYG{l+s+s1}{\PYGZsq{}}\PYG{p}{:} \PYG{l+s+s1}{\PYGZsq{}}\PYG{l+s+s1}{manual}\PYG{l+s+s1}{\PYGZsq{}}\PYG{p}{,}
\PYG{p}{\PYGZcb{}}\PYG{p}{]}\PYG{p}{)}
\PYG{n}{models}\PYG{o}{.}\PYG{n}{execute\PYGZus{}kw}\PYG{p}{(}
    \PYG{n}{db}\PYG{p}{,} \PYG{n}{uid}\PYG{p}{,} \PYG{n}{password}\PYG{p}{,} \PYG{l+s+s1}{\PYGZsq{}}\PYG{l+s+s1}{x\PYGZus{}custom\PYGZus{}model}\PYG{l+s+s1}{\PYGZsq{}}\PYG{p}{,} \PYG{l+s+s1}{\PYGZsq{}}\PYG{l+s+s1}{fields\PYGZus{}get}\PYG{l+s+s1}{\PYGZsq{}}\PYG{p}{,}
    \PYG{p}{[}\PYG{p}{]}\PYG{p}{,} \PYG{p}{\PYGZob{}}\PYG{l+s+s1}{\PYGZsq{}}\PYG{l+s+s1}{attributes}\PYG{l+s+s1}{\PYGZsq{}}\PYG{p}{:} \PYG{p}{[}\PYG{l+s+s1}{\PYGZsq{}}\PYG{l+s+s1}{string}\PYG{l+s+s1}{\PYGZsq{}}\PYG{p}{,} \PYG{l+s+s1}{\PYGZsq{}}\PYG{l+s+s1}{help}\PYG{l+s+s1}{\PYGZsq{}}\PYG{p}{,} \PYG{l+s+s1}{\PYGZsq{}}\PYG{l+s+s1}{type}\PYG{l+s+s1}{\PYGZsq{}}\PYG{p}{]}\PYG{p}{\PYGZcb{}}\PYG{p}{)}
\end{sphinxVerbatim}

\fvset{hllines={, ,}}%
\begin{sphinxVerbatim}[commandchars=\\\{\}]
\PYG{n+nv}{\PYGZdl{}models}\PYG{o}{\PYGZhy{}\PYGZgt{}}\PYG{n+na}{execute\PYGZus{}kw}\PYG{p}{(}
    \PYG{n+nv}{\PYGZdl{}db}\PYG{p}{,} \PYG{n+nv}{\PYGZdl{}uid}\PYG{p}{,} \PYG{n+nv}{\PYGZdl{}password}\PYG{p}{,}
    \PYG{l+s+s1}{\PYGZsq{}ir.model\PYGZsq{}}\PYG{p}{,} \PYG{l+s+s1}{\PYGZsq{}create\PYGZsq{}}\PYG{p}{,} \PYG{k}{array}\PYG{p}{(}\PYG{k}{array}\PYG{p}{(}
        \PYG{l+s+s1}{\PYGZsq{}name\PYGZsq{}} \PYG{o}{=\PYGZgt{}} \PYG{l+s+s2}{\PYGZdq{}}\PYG{l+s+s2}{Custom Model}\PYG{l+s+s2}{\PYGZdq{}}\PYG{p}{,}
        \PYG{l+s+s1}{\PYGZsq{}model\PYGZsq{}} \PYG{o}{=\PYGZgt{}} \PYG{l+s+s1}{\PYGZsq{}x\PYGZus{}custom\PYGZus{}model\PYGZsq{}}\PYG{p}{,}
        \PYG{l+s+s1}{\PYGZsq{}state\PYGZsq{}} \PYG{o}{=\PYGZgt{}} \PYG{l+s+s1}{\PYGZsq{}manual\PYGZsq{}}
    \PYG{p}{))}
\PYG{p}{);}
\PYG{n+nv}{\PYGZdl{}models}\PYG{o}{\PYGZhy{}\PYGZgt{}}\PYG{n+na}{execute\PYGZus{}kw}\PYG{p}{(}
    \PYG{n+nv}{\PYGZdl{}db}\PYG{p}{,} \PYG{n+nv}{\PYGZdl{}uid}\PYG{p}{,} \PYG{n+nv}{\PYGZdl{}password}\PYG{p}{,}
    \PYG{l+s+s1}{\PYGZsq{}x\PYGZus{}custom\PYGZus{}model\PYGZsq{}}\PYG{p}{,} \PYG{l+s+s1}{\PYGZsq{}fields\PYGZus{}get\PYGZsq{}}\PYG{p}{,}
    \PYG{k}{array}\PYG{p}{(),}
    \PYG{k}{array}\PYG{p}{(}\PYG{l+s+s1}{\PYGZsq{}attributes\PYGZsq{}} \PYG{o}{=\PYGZgt{}} \PYG{k}{array}\PYG{p}{(}\PYG{l+s+s1}{\PYGZsq{}string\PYGZsq{}}\PYG{p}{,} \PYG{l+s+s1}{\PYGZsq{}help\PYGZsq{}}\PYG{p}{,} \PYG{l+s+s1}{\PYGZsq{}type\PYGZsq{}}\PYG{p}{))}
\PYG{p}{);}
\end{sphinxVerbatim}

\fvset{hllines={, ,}}%
\begin{sphinxVerbatim}[commandchars=\\\{\}]
\PYG{n}{models}\PYG{o}{.}\PYG{n}{execute\PYGZus{}kw}\PYG{p}{(}
    \PYG{n}{db}\PYG{p}{,} \PYG{n}{uid}\PYG{p}{,} \PYG{n}{password}\PYG{p}{,}
    \PYG{l+s+s1}{\PYGZsq{}ir.model\PYGZsq{}}\PYG{p}{,} \PYG{l+s+s1}{\PYGZsq{}create\PYGZsq{}}\PYG{p}{,} \PYG{o}{[}\PYG{p}{\PYGZob{}}
        \PYG{n+nb}{name}\PYG{p}{:} \PYG{l+s+s2}{\PYGZdq{}}\PYG{l+s+s2}{Custom Model}\PYG{l+s+s2}{\PYGZdq{}}\PYG{p}{,}
        \PYG{l+s+ss}{model}\PYG{p}{:} \PYG{l+s+s1}{\PYGZsq{}x\PYGZus{}custom\PYGZus{}model\PYGZsq{}}\PYG{p}{,}
        \PYG{l+s+ss}{state}\PYG{p}{:} \PYG{l+s+s1}{\PYGZsq{}manual\PYGZsq{}}
    \PYG{p}{\PYGZcb{}}\PYG{o}{]}\PYG{p}{)}
\PYG{n}{fields} \PYG{o}{=} \PYG{n}{models}\PYG{o}{.}\PYG{n}{execute\PYGZus{}kw}\PYG{p}{(}
    \PYG{n}{db}\PYG{p}{,} \PYG{n}{uid}\PYG{p}{,} \PYG{n}{password}\PYG{p}{,} \PYG{l+s+s1}{\PYGZsq{}x\PYGZus{}custom\PYGZus{}model\PYGZsq{}}\PYG{p}{,} \PYG{l+s+s1}{\PYGZsq{}fields\PYGZus{}get\PYGZsq{}}\PYG{p}{,}
    \PYG{o}{[}\PYG{o}{]}\PYG{p}{,} \PYG{p}{\PYGZob{}}\PYG{l+s+ss}{attributes}\PYG{p}{:} \PYG{l+s+sx}{\PYGZpc{}w(}\PYG{l+s+sx}{string help type}\PYG{l+s+sx}{)}\PYG{p}{\PYGZcb{}}\PYG{p}{)}
\end{sphinxVerbatim}

\fvset{hllines={, ,}}%
\begin{sphinxVerbatim}[commandchars=\\\{\}]
\PYG{n}{models}\PYG{o}{.}\PYG{n+na}{execute}\PYG{o}{(}
    \PYG{l+s}{\PYGZdq{}execute\PYGZus{}kw\PYGZdq{}}\PYG{o}{,} \PYG{n}{asList}\PYG{o}{(}
        \PYG{n}{db}\PYG{o}{,} \PYG{n}{uid}\PYG{o}{,} \PYG{n}{password}\PYG{o}{,}
        \PYG{l+s}{\PYGZdq{}ir.model\PYGZdq{}}\PYG{o}{,} \PYG{l+s}{\PYGZdq{}create\PYGZdq{}}\PYG{o}{,}
        \PYG{n}{asList}\PYG{o}{(}\PYG{k}{new} \PYG{n}{HashMap}\PYG{o}{\PYGZlt{}}\PYG{n}{String}\PYG{o}{,} \PYG{n}{Object}\PYG{o}{\PYGZgt{}}\PYG{o}{(}\PYG{o}{)} \PYG{o}{\PYGZob{}}\PYG{o}{\PYGZob{}}
            \PYG{n}{put}\PYG{o}{(}\PYG{l+s}{\PYGZdq{}name\PYGZdq{}}\PYG{o}{,} \PYG{l+s}{\PYGZdq{}Custom Model\PYGZdq{}}\PYG{o}{)}\PYG{o}{;}
            \PYG{n}{put}\PYG{o}{(}\PYG{l+s}{\PYGZdq{}model\PYGZdq{}}\PYG{o}{,} \PYG{l+s}{\PYGZdq{}x\PYGZus{}custom\PYGZus{}model\PYGZdq{}}\PYG{o}{)}\PYG{o}{;}
            \PYG{n}{put}\PYG{o}{(}\PYG{l+s}{\PYGZdq{}state\PYGZdq{}}\PYG{o}{,} \PYG{l+s}{\PYGZdq{}manual\PYGZdq{}}\PYG{o}{)}\PYG{o}{;}
        \PYG{o}{\PYGZcb{}}\PYG{o}{\PYGZcb{}}\PYG{o}{)}
\PYG{o}{)}\PYG{o}{)}\PYG{o}{;}
\PYG{k+kd}{final} \PYG{n}{Object} \PYG{n}{fields} \PYG{o}{=} \PYG{n}{models}\PYG{o}{.}\PYG{n+na}{execute}\PYG{o}{(}
    \PYG{l+s}{\PYGZdq{}execute\PYGZus{}kw\PYGZdq{}}\PYG{o}{,} \PYG{n}{asList}\PYG{o}{(}
        \PYG{n}{db}\PYG{o}{,} \PYG{n}{uid}\PYG{o}{,} \PYG{n}{password}\PYG{o}{,}
        \PYG{l+s}{\PYGZdq{}x\PYGZus{}custom\PYGZus{}model\PYGZdq{}}\PYG{o}{,} \PYG{l+s}{\PYGZdq{}fields\PYGZus{}get\PYGZdq{}}\PYG{o}{,}
        \PYG{n}{emptyList}\PYG{o}{(}\PYG{o}{)}\PYG{o}{,}
        \PYG{k}{new} \PYG{n}{HashMap}\PYG{o}{\PYGZlt{}}\PYG{n}{String}\PYG{o}{,} \PYG{n}{Object}\PYG{o}{\PYGZgt{}} \PYG{o}{(}\PYG{o}{)} \PYG{o}{\PYGZob{}}\PYG{o}{\PYGZob{}}
            \PYG{n}{put}\PYG{o}{(}\PYG{l+s}{\PYGZdq{}attributes\PYGZdq{}}\PYG{o}{,} \PYG{n}{asList}\PYG{o}{(}
                    \PYG{l+s}{\PYGZdq{}string\PYGZdq{}}\PYG{o}{,}
                    \PYG{l+s}{\PYGZdq{}help\PYGZdq{}}\PYG{o}{,}
                    \PYG{l+s}{\PYGZdq{}type\PYGZdq{}}\PYG{o}{)}\PYG{o}{)}\PYG{o}{;}
        \PYG{o}{\PYGZcb{}}\PYG{o}{\PYGZcb{}}
\PYG{o}{)}\PYG{o}{)}\PYG{o}{;}
\end{sphinxVerbatim}

\fvset{hllines={, ,}}%
\begin{sphinxVerbatim}[commandchars=\\\{\}]
\PYG{p}{\PYGZob{}}
    \PYG{n+nt}{\PYGZdq{}create\PYGZus{}uid\PYGZdq{}}\PYG{p}{:} \PYG{p}{\PYGZob{}}
        \PYG{n+nt}{\PYGZdq{}type\PYGZdq{}}\PYG{p}{:} \PYG{l+s+s2}{\PYGZdq{}many2one\PYGZdq{}}\PYG{p}{,}
        \PYG{n+nt}{\PYGZdq{}string\PYGZdq{}}\PYG{p}{:} \PYG{l+s+s2}{\PYGZdq{}Created by\PYGZdq{}}
    \PYG{p}{\PYGZcb{}}\PYG{p}{,}
    \PYG{n+nt}{\PYGZdq{}create\PYGZus{}date\PYGZdq{}}\PYG{p}{:} \PYG{p}{\PYGZob{}}
        \PYG{n+nt}{\PYGZdq{}type\PYGZdq{}}\PYG{p}{:} \PYG{l+s+s2}{\PYGZdq{}datetime\PYGZdq{}}\PYG{p}{,}
        \PYG{n+nt}{\PYGZdq{}string\PYGZdq{}}\PYG{p}{:} \PYG{l+s+s2}{\PYGZdq{}Created on\PYGZdq{}}
    \PYG{p}{\PYGZcb{}}\PYG{p}{,}
    \PYG{n+nt}{\PYGZdq{}\PYGZus{}\PYGZus{}last\PYGZus{}update\PYGZdq{}}\PYG{p}{:} \PYG{p}{\PYGZob{}}
        \PYG{n+nt}{\PYGZdq{}type\PYGZdq{}}\PYG{p}{:} \PYG{l+s+s2}{\PYGZdq{}datetime\PYGZdq{}}\PYG{p}{,}
        \PYG{n+nt}{\PYGZdq{}string\PYGZdq{}}\PYG{p}{:} \PYG{l+s+s2}{\PYGZdq{}Last Modified on\PYGZdq{}}
    \PYG{p}{\PYGZcb{}}\PYG{p}{,}
    \PYG{n+nt}{\PYGZdq{}write\PYGZus{}uid\PYGZdq{}}\PYG{p}{:} \PYG{p}{\PYGZob{}}
        \PYG{n+nt}{\PYGZdq{}type\PYGZdq{}}\PYG{p}{:} \PYG{l+s+s2}{\PYGZdq{}many2one\PYGZdq{}}\PYG{p}{,}
        \PYG{n+nt}{\PYGZdq{}string\PYGZdq{}}\PYG{p}{:} \PYG{l+s+s2}{\PYGZdq{}Last Updated by\PYGZdq{}}
    \PYG{p}{\PYGZcb{}}\PYG{p}{,}
    \PYG{n+nt}{\PYGZdq{}write\PYGZus{}date\PYGZdq{}}\PYG{p}{:} \PYG{p}{\PYGZob{}}
        \PYG{n+nt}{\PYGZdq{}type\PYGZdq{}}\PYG{p}{:} \PYG{l+s+s2}{\PYGZdq{}datetime\PYGZdq{}}\PYG{p}{,}
        \PYG{n+nt}{\PYGZdq{}string\PYGZdq{}}\PYG{p}{:} \PYG{l+s+s2}{\PYGZdq{}Last Updated on\PYGZdq{}}
    \PYG{p}{\PYGZcb{}}\PYG{p}{,}
    \PYG{n+nt}{\PYGZdq{}display\PYGZus{}name\PYGZdq{}}\PYG{p}{:} \PYG{p}{\PYGZob{}}
        \PYG{n+nt}{\PYGZdq{}type\PYGZdq{}}\PYG{p}{:} \PYG{l+s+s2}{\PYGZdq{}char\PYGZdq{}}\PYG{p}{,}
        \PYG{n+nt}{\PYGZdq{}string\PYGZdq{}}\PYG{p}{:} \PYG{l+s+s2}{\PYGZdq{}Display Name\PYGZdq{}}
    \PYG{p}{\PYGZcb{}}\PYG{p}{,}
    \PYG{n+nt}{\PYGZdq{}id\PYGZdq{}}\PYG{p}{:} \PYG{p}{\PYGZob{}}
        \PYG{n+nt}{\PYGZdq{}type\PYGZdq{}}\PYG{p}{:} \PYG{l+s+s2}{\PYGZdq{}integer\PYGZdq{}}\PYG{p}{,}
        \PYG{n+nt}{\PYGZdq{}string\PYGZdq{}}\PYG{p}{:} \PYG{l+s+s2}{\PYGZdq{}Id\PYGZdq{}}
    \PYG{p}{\PYGZcb{}}
\PYG{p}{\PYGZcb{}}
\end{sphinxVerbatim}


\paragraph{\sphinxstyleliteralintitle{\sphinxupquote{ir.model.fields}}}
\label{\detokenize{webservices/odoo:reference-webservice-inspection-fields}}\label{\detokenize{webservices/odoo:ir-model-fields}}
Provides information about the fields of Odoo models and allows adding
custom fields without using Python code
\begin{description}
\item[{\sphinxcode{\sphinxupquote{model\_id}}}] \leavevmode
{\hyperref[\detokenize{reference/orm:odoo.fields.Many2one}]{\sphinxcrossref{\sphinxcode{\sphinxupquote{Many2one}}}}} to
{\hyperref[\detokenize{webservices/odoo:reference-webservice-inspection-models}]{\sphinxcrossref{\DUrole{std,std-ref}{ir.model}}}} to which the field belongs

\item[{\sphinxcode{\sphinxupquote{name}}}] \leavevmode
the field’s technical name (used in \sphinxcode{\sphinxupquote{read}} or \sphinxcode{\sphinxupquote{write}})

\item[{\sphinxcode{\sphinxupquote{field\_description}}}] \leavevmode
the field’s user-readable label (e.g. \sphinxcode{\sphinxupquote{string}} in \sphinxcode{\sphinxupquote{fields\_get}})

\item[{\sphinxcode{\sphinxupquote{ttype}}}] \leavevmode
the {\hyperref[\detokenize{reference/orm:reference-orm-fields}]{\sphinxcrossref{\DUrole{std,std-ref}{type}}}} of field to create

\item[{\sphinxcode{\sphinxupquote{state}}}] \leavevmode
whether the field was created via Python code (\sphinxcode{\sphinxupquote{base}}) or via
\sphinxcode{\sphinxupquote{ir.model.fields}} (\sphinxcode{\sphinxupquote{manual}})

\item[{\sphinxcode{\sphinxupquote{required}}, \sphinxcode{\sphinxupquote{readonly}}, \sphinxcode{\sphinxupquote{translate}}}] \leavevmode
enables the corresponding flag on the field

\item[{\sphinxcode{\sphinxupquote{groups}}}] \leavevmode
{\hyperref[\detokenize{reference/security:reference-security-fields}]{\sphinxcrossref{\DUrole{std,std-ref}{field-level access control}}}}, a
{\hyperref[\detokenize{reference/orm:odoo.fields.Many2many}]{\sphinxcrossref{\sphinxcode{\sphinxupquote{Many2many}}}}} to \sphinxcode{\sphinxupquote{res.groups}}

\item[{\sphinxcode{\sphinxupquote{selection}}, \sphinxcode{\sphinxupquote{size}}, \sphinxcode{\sphinxupquote{on\_delete}}, \sphinxcode{\sphinxupquote{relation}}, \sphinxcode{\sphinxupquote{relation\_field}}, \sphinxcode{\sphinxupquote{domain}}}] \leavevmode
type-specific properties and customizations, see {\hyperref[\detokenize{reference/orm:reference-orm-fields}]{\sphinxcrossref{\DUrole{std,std-ref}{the fields
documentation}}}} for details

\end{description}

Like custom models, only new fields created with \sphinxcode{\sphinxupquote{state="manual"}} are
activated as actual fields on the model.

\begin{sphinxadmonition}{warning}{Warning:}
computed fields can not be added via \sphinxcode{\sphinxupquote{ir.model.fields}}, some
field meta-information (defaults, onchange) can not be set either
\end{sphinxadmonition}
\begin{itemize}
\item {} Python 2
\item {} PHP
\item {} Ruby
\item {} Java
\end{itemize}

\fvset{hllines={, ,}}%
\begin{sphinxVerbatim}[commandchars=\\\{\}]
\PYG{n+nb}{id} \PYG{o}{=} \PYG{n}{models}\PYG{o}{.}\PYG{n}{execute\PYGZus{}kw}\PYG{p}{(}\PYG{n}{db}\PYG{p}{,} \PYG{n}{uid}\PYG{p}{,} \PYG{n}{password}\PYG{p}{,} \PYG{l+s+s1}{\PYGZsq{}}\PYG{l+s+s1}{ir.model}\PYG{l+s+s1}{\PYGZsq{}}\PYG{p}{,} \PYG{l+s+s1}{\PYGZsq{}}\PYG{l+s+s1}{create}\PYG{l+s+s1}{\PYGZsq{}}\PYG{p}{,} \PYG{p}{[}\PYG{p}{\PYGZob{}}
    \PYG{l+s+s1}{\PYGZsq{}}\PYG{l+s+s1}{name}\PYG{l+s+s1}{\PYGZsq{}}\PYG{p}{:} \PYG{l+s+s2}{\PYGZdq{}}\PYG{l+s+s2}{Custom Model}\PYG{l+s+s2}{\PYGZdq{}}\PYG{p}{,}
    \PYG{l+s+s1}{\PYGZsq{}}\PYG{l+s+s1}{model}\PYG{l+s+s1}{\PYGZsq{}}\PYG{p}{:} \PYG{l+s+s2}{\PYGZdq{}}\PYG{l+s+s2}{x\PYGZus{}custom}\PYG{l+s+s2}{\PYGZdq{}}\PYG{p}{,}
    \PYG{l+s+s1}{\PYGZsq{}}\PYG{l+s+s1}{state}\PYG{l+s+s1}{\PYGZsq{}}\PYG{p}{:} \PYG{l+s+s1}{\PYGZsq{}}\PYG{l+s+s1}{manual}\PYG{l+s+s1}{\PYGZsq{}}\PYG{p}{,}
\PYG{p}{\PYGZcb{}}\PYG{p}{]}\PYG{p}{)}
\PYG{n}{models}\PYG{o}{.}\PYG{n}{execute\PYGZus{}kw}\PYG{p}{(}
    \PYG{n}{db}\PYG{p}{,} \PYG{n}{uid}\PYG{p}{,} \PYG{n}{password}\PYG{p}{,}
    \PYG{l+s+s1}{\PYGZsq{}}\PYG{l+s+s1}{ir.model.fields}\PYG{l+s+s1}{\PYGZsq{}}\PYG{p}{,} \PYG{l+s+s1}{\PYGZsq{}}\PYG{l+s+s1}{create}\PYG{l+s+s1}{\PYGZsq{}}\PYG{p}{,} \PYG{p}{[}\PYG{p}{\PYGZob{}}
        \PYG{l+s+s1}{\PYGZsq{}}\PYG{l+s+s1}{model\PYGZus{}id}\PYG{l+s+s1}{\PYGZsq{}}\PYG{p}{:} \PYG{n+nb}{id}\PYG{p}{,}
        \PYG{l+s+s1}{\PYGZsq{}}\PYG{l+s+s1}{name}\PYG{l+s+s1}{\PYGZsq{}}\PYG{p}{:} \PYG{l+s+s1}{\PYGZsq{}}\PYG{l+s+s1}{x\PYGZus{}name}\PYG{l+s+s1}{\PYGZsq{}}\PYG{p}{,}
        \PYG{l+s+s1}{\PYGZsq{}}\PYG{l+s+s1}{ttype}\PYG{l+s+s1}{\PYGZsq{}}\PYG{p}{:} \PYG{l+s+s1}{\PYGZsq{}}\PYG{l+s+s1}{char}\PYG{l+s+s1}{\PYGZsq{}}\PYG{p}{,}
        \PYG{l+s+s1}{\PYGZsq{}}\PYG{l+s+s1}{state}\PYG{l+s+s1}{\PYGZsq{}}\PYG{p}{:} \PYG{l+s+s1}{\PYGZsq{}}\PYG{l+s+s1}{manual}\PYG{l+s+s1}{\PYGZsq{}}\PYG{p}{,}
        \PYG{l+s+s1}{\PYGZsq{}}\PYG{l+s+s1}{required}\PYG{l+s+s1}{\PYGZsq{}}\PYG{p}{:} \PYG{n+nb+bp}{True}\PYG{p}{,}
    \PYG{p}{\PYGZcb{}}\PYG{p}{]}\PYG{p}{)}
\PYG{n}{record\PYGZus{}id} \PYG{o}{=} \PYG{n}{models}\PYG{o}{.}\PYG{n}{execute\PYGZus{}kw}\PYG{p}{(}
    \PYG{n}{db}\PYG{p}{,} \PYG{n}{uid}\PYG{p}{,} \PYG{n}{password}\PYG{p}{,}
    \PYG{l+s+s1}{\PYGZsq{}}\PYG{l+s+s1}{x\PYGZus{}custom}\PYG{l+s+s1}{\PYGZsq{}}\PYG{p}{,} \PYG{l+s+s1}{\PYGZsq{}}\PYG{l+s+s1}{create}\PYG{l+s+s1}{\PYGZsq{}}\PYG{p}{,} \PYG{p}{[}\PYG{p}{\PYGZob{}}
        \PYG{l+s+s1}{\PYGZsq{}}\PYG{l+s+s1}{x\PYGZus{}name}\PYG{l+s+s1}{\PYGZsq{}}\PYG{p}{:} \PYG{l+s+s2}{\PYGZdq{}}\PYG{l+s+s2}{test record}\PYG{l+s+s2}{\PYGZdq{}}\PYG{p}{,}
    \PYG{p}{\PYGZcb{}}\PYG{p}{]}\PYG{p}{)}
\PYG{n}{models}\PYG{o}{.}\PYG{n}{execute\PYGZus{}kw}\PYG{p}{(}\PYG{n}{db}\PYG{p}{,} \PYG{n}{uid}\PYG{p}{,} \PYG{n}{password}\PYG{p}{,} \PYG{l+s+s1}{\PYGZsq{}}\PYG{l+s+s1}{x\PYGZus{}custom}\PYG{l+s+s1}{\PYGZsq{}}\PYG{p}{,} \PYG{l+s+s1}{\PYGZsq{}}\PYG{l+s+s1}{read}\PYG{l+s+s1}{\PYGZsq{}}\PYG{p}{,} \PYG{p}{[}\PYG{p}{[}\PYG{n}{record\PYGZus{}id}\PYG{p}{]}\PYG{p}{]}\PYG{p}{)}
\end{sphinxVerbatim}

\fvset{hllines={, ,}}%
\begin{sphinxVerbatim}[commandchars=\\\{\}]
\PYG{n+nv}{\PYGZdl{}id} \PYG{o}{=} \PYG{n+nv}{\PYGZdl{}models}\PYG{o}{\PYGZhy{}\PYGZgt{}}\PYG{n+na}{execute\PYGZus{}kw}\PYG{p}{(}
    \PYG{n+nv}{\PYGZdl{}db}\PYG{p}{,} \PYG{n+nv}{\PYGZdl{}uid}\PYG{p}{,} \PYG{n+nv}{\PYGZdl{}password}\PYG{p}{,}
    \PYG{l+s+s1}{\PYGZsq{}ir.model\PYGZsq{}}\PYG{p}{,} \PYG{l+s+s1}{\PYGZsq{}create\PYGZsq{}}\PYG{p}{,} \PYG{k}{array}\PYG{p}{(}\PYG{k}{array}\PYG{p}{(}
        \PYG{l+s+s1}{\PYGZsq{}name\PYGZsq{}} \PYG{o}{=\PYGZgt{}} \PYG{l+s+s2}{\PYGZdq{}}\PYG{l+s+s2}{Custom Model}\PYG{l+s+s2}{\PYGZdq{}}\PYG{p}{,}
        \PYG{l+s+s1}{\PYGZsq{}model\PYGZsq{}} \PYG{o}{=\PYGZgt{}} \PYG{l+s+s1}{\PYGZsq{}x\PYGZus{}custom\PYGZsq{}}\PYG{p}{,}
        \PYG{l+s+s1}{\PYGZsq{}state\PYGZsq{}} \PYG{o}{=\PYGZgt{}} \PYG{l+s+s1}{\PYGZsq{}manual\PYGZsq{}}
    \PYG{p}{))}
\PYG{p}{);}
\PYG{n+nv}{\PYGZdl{}models}\PYG{o}{\PYGZhy{}\PYGZgt{}}\PYG{n+na}{execute\PYGZus{}kw}\PYG{p}{(}
    \PYG{n+nv}{\PYGZdl{}db}\PYG{p}{,} \PYG{n+nv}{\PYGZdl{}uid}\PYG{p}{,} \PYG{n+nv}{\PYGZdl{}password}\PYG{p}{,}
    \PYG{l+s+s1}{\PYGZsq{}ir.model.fields\PYGZsq{}}\PYG{p}{,} \PYG{l+s+s1}{\PYGZsq{}create\PYGZsq{}}\PYG{p}{,} \PYG{k}{array}\PYG{p}{(}\PYG{k}{array}\PYG{p}{(}
        \PYG{l+s+s1}{\PYGZsq{}model\PYGZus{}id\PYGZsq{}} \PYG{o}{=\PYGZgt{}} \PYG{n+nv}{\PYGZdl{}id}\PYG{p}{,}
        \PYG{l+s+s1}{\PYGZsq{}name\PYGZsq{}} \PYG{o}{=\PYGZgt{}} \PYG{l+s+s1}{\PYGZsq{}x\PYGZus{}name\PYGZsq{}}\PYG{p}{,}
        \PYG{l+s+s1}{\PYGZsq{}ttype\PYGZsq{}} \PYG{o}{=\PYGZgt{}} \PYG{l+s+s1}{\PYGZsq{}char\PYGZsq{}}\PYG{p}{,}
        \PYG{l+s+s1}{\PYGZsq{}state\PYGZsq{}} \PYG{o}{=\PYGZgt{}} \PYG{l+s+s1}{\PYGZsq{}manual\PYGZsq{}}\PYG{p}{,}
        \PYG{l+s+s1}{\PYGZsq{}required\PYGZsq{}} \PYG{o}{=\PYGZgt{}} \PYG{k}{true}
    \PYG{p}{))}
\PYG{p}{);}
\PYG{n+nv}{\PYGZdl{}record\PYGZus{}id} \PYG{o}{=} \PYG{n+nv}{\PYGZdl{}models}\PYG{o}{\PYGZhy{}\PYGZgt{}}\PYG{n+na}{execute\PYGZus{}kw}\PYG{p}{(}
    \PYG{n+nv}{\PYGZdl{}db}\PYG{p}{,} \PYG{n+nv}{\PYGZdl{}uid}\PYG{p}{,} \PYG{n+nv}{\PYGZdl{}password}\PYG{p}{,}
    \PYG{l+s+s1}{\PYGZsq{}x\PYGZus{}custom\PYGZsq{}}\PYG{p}{,} \PYG{l+s+s1}{\PYGZsq{}create\PYGZsq{}}\PYG{p}{,} \PYG{k}{array}\PYG{p}{(}\PYG{k}{array}\PYG{p}{(}
        \PYG{l+s+s1}{\PYGZsq{}x\PYGZus{}name\PYGZsq{}} \PYG{o}{=\PYGZgt{}} \PYG{l+s+s2}{\PYGZdq{}}\PYG{l+s+s2}{test record}\PYG{l+s+s2}{\PYGZdq{}}
    \PYG{p}{))}
\PYG{p}{);}
\PYG{n+nv}{\PYGZdl{}models}\PYG{o}{\PYGZhy{}\PYGZgt{}}\PYG{n+na}{execute\PYGZus{}kw}\PYG{p}{(}
    \PYG{n+nv}{\PYGZdl{}db}\PYG{p}{,} \PYG{n+nv}{\PYGZdl{}uid}\PYG{p}{,} \PYG{n+nv}{\PYGZdl{}password}\PYG{p}{,}
    \PYG{l+s+s1}{\PYGZsq{}x\PYGZus{}custom\PYGZsq{}}\PYG{p}{,} \PYG{l+s+s1}{\PYGZsq{}read\PYGZsq{}}\PYG{p}{,}
    \PYG{k}{array}\PYG{p}{(}\PYG{k}{array}\PYG{p}{(}\PYG{n+nv}{\PYGZdl{}record\PYGZus{}id}\PYG{p}{)));}
\end{sphinxVerbatim}

\fvset{hllines={, ,}}%
\begin{sphinxVerbatim}[commandchars=\\\{\}]
\PYG{n+nb}{id} \PYG{o}{=} \PYG{n}{models}\PYG{o}{.}\PYG{n}{execute\PYGZus{}kw}\PYG{p}{(}
    \PYG{n}{db}\PYG{p}{,} \PYG{n}{uid}\PYG{p}{,} \PYG{n}{password}\PYG{p}{,}
    \PYG{l+s+s1}{\PYGZsq{}ir.model\PYGZsq{}}\PYG{p}{,} \PYG{l+s+s1}{\PYGZsq{}create\PYGZsq{}}\PYG{p}{,} \PYG{o}{[}\PYG{p}{\PYGZob{}}
        \PYG{n+nb}{name}\PYG{p}{:} \PYG{l+s+s2}{\PYGZdq{}}\PYG{l+s+s2}{Custom Model}\PYG{l+s+s2}{\PYGZdq{}}\PYG{p}{,}
        \PYG{l+s+ss}{model}\PYG{p}{:} \PYG{l+s+s2}{\PYGZdq{}}\PYG{l+s+s2}{x\PYGZus{}custom}\PYG{l+s+s2}{\PYGZdq{}}\PYG{p}{,}
        \PYG{l+s+ss}{state}\PYG{p}{:} \PYG{l+s+s1}{\PYGZsq{}manual\PYGZsq{}}
    \PYG{p}{\PYGZcb{}}\PYG{o}{]}\PYG{p}{)}
\PYG{n}{models}\PYG{o}{.}\PYG{n}{execute\PYGZus{}kw}\PYG{p}{(}
    \PYG{n}{db}\PYG{p}{,} \PYG{n}{uid}\PYG{p}{,} \PYG{n}{password}\PYG{p}{,}
    \PYG{l+s+s1}{\PYGZsq{}ir.model.fields\PYGZsq{}}\PYG{p}{,} \PYG{l+s+s1}{\PYGZsq{}create\PYGZsq{}}\PYG{p}{,} \PYG{o}{[}\PYG{p}{\PYGZob{}}
        \PYG{l+s+ss}{model\PYGZus{}id}\PYG{p}{:} \PYG{n+nb}{id}\PYG{p}{,}
        \PYG{n+nb}{name}\PYG{p}{:} \PYG{l+s+s2}{\PYGZdq{}}\PYG{l+s+s2}{x\PYGZus{}name}\PYG{l+s+s2}{\PYGZdq{}}\PYG{p}{,}
        \PYG{l+s+ss}{ttype}\PYG{p}{:} \PYG{l+s+s2}{\PYGZdq{}}\PYG{l+s+s2}{char}\PYG{l+s+s2}{\PYGZdq{}}\PYG{p}{,}
        \PYG{l+s+ss}{state}\PYG{p}{:} \PYG{l+s+s2}{\PYGZdq{}}\PYG{l+s+s2}{manual}\PYG{l+s+s2}{\PYGZdq{}}\PYG{p}{,}
        \PYG{l+s+ss}{required}\PYG{p}{:} \PYG{k+kp}{true}
    \PYG{p}{\PYGZcb{}}\PYG{o}{]}\PYG{p}{)}
\PYG{n}{record\PYGZus{}id} \PYG{o}{=} \PYG{n}{models}\PYG{o}{.}\PYG{n}{execute\PYGZus{}kw}\PYG{p}{(}
    \PYG{n}{db}\PYG{p}{,} \PYG{n}{uid}\PYG{p}{,} \PYG{n}{password}\PYG{p}{,}
    \PYG{l+s+s1}{\PYGZsq{}x\PYGZus{}custom\PYGZsq{}}\PYG{p}{,} \PYG{l+s+s1}{\PYGZsq{}create\PYGZsq{}}\PYG{p}{,} \PYG{o}{[}\PYG{p}{\PYGZob{}}
        \PYG{l+s+ss}{x\PYGZus{}name}\PYG{p}{:} \PYG{l+s+s2}{\PYGZdq{}}\PYG{l+s+s2}{test record}\PYG{l+s+s2}{\PYGZdq{}}
    \PYG{p}{\PYGZcb{}}\PYG{o}{]}\PYG{p}{)}
\PYG{n}{models}\PYG{o}{.}\PYG{n}{execute\PYGZus{}kw}\PYG{p}{(}
    \PYG{n}{db}\PYG{p}{,} \PYG{n}{uid}\PYG{p}{,} \PYG{n}{password}\PYG{p}{,}
    \PYG{l+s+s1}{\PYGZsq{}x\PYGZus{}custom\PYGZsq{}}\PYG{p}{,} \PYG{l+s+s1}{\PYGZsq{}read\PYGZsq{}}\PYG{p}{,} \PYG{o}{[}\PYG{o}{[}\PYG{n}{record\PYGZus{}id}\PYG{o}{]}\PYG{o}{]}\PYG{p}{)}
\end{sphinxVerbatim}

\fvset{hllines={, ,}}%
\begin{sphinxVerbatim}[commandchars=\\\{\}]
\PYG{k+kd}{final} \PYG{n}{Integer} \PYG{n}{id} \PYG{o}{=} \PYG{o}{(}\PYG{n}{Integer}\PYG{o}{)}\PYG{n}{models}\PYG{o}{.}\PYG{n+na}{execute}\PYG{o}{(}
    \PYG{l+s}{\PYGZdq{}execute\PYGZus{}kw\PYGZdq{}}\PYG{o}{,} \PYG{n}{asList}\PYG{o}{(}
        \PYG{n}{db}\PYG{o}{,} \PYG{n}{uid}\PYG{o}{,} \PYG{n}{password}\PYG{o}{,}
        \PYG{l+s}{\PYGZdq{}ir.model\PYGZdq{}}\PYG{o}{,} \PYG{l+s}{\PYGZdq{}create\PYGZdq{}}\PYG{o}{,}
        \PYG{n}{asList}\PYG{o}{(}\PYG{k}{new} \PYG{n}{HashMap}\PYG{o}{\PYGZlt{}}\PYG{n}{String}\PYG{o}{,} \PYG{n}{Object}\PYG{o}{\PYGZgt{}}\PYG{o}{(}\PYG{o}{)} \PYG{o}{\PYGZob{}}\PYG{o}{\PYGZob{}}
            \PYG{n}{put}\PYG{o}{(}\PYG{l+s}{\PYGZdq{}name\PYGZdq{}}\PYG{o}{,} \PYG{l+s}{\PYGZdq{}Custom Model\PYGZdq{}}\PYG{o}{)}\PYG{o}{;}
            \PYG{n}{put}\PYG{o}{(}\PYG{l+s}{\PYGZdq{}model\PYGZdq{}}\PYG{o}{,} \PYG{l+s}{\PYGZdq{}x\PYGZus{}custom\PYGZdq{}}\PYG{o}{)}\PYG{o}{;}
            \PYG{n}{put}\PYG{o}{(}\PYG{l+s}{\PYGZdq{}state\PYGZdq{}}\PYG{o}{,} \PYG{l+s}{\PYGZdq{}manual\PYGZdq{}}\PYG{o}{)}\PYG{o}{;}
        \PYG{o}{\PYGZcb{}}\PYG{o}{\PYGZcb{}}\PYG{o}{)}
\PYG{o}{)}\PYG{o}{)}\PYG{o}{;}
\PYG{n}{models}\PYG{o}{.}\PYG{n+na}{execute}\PYG{o}{(}
    \PYG{l+s}{\PYGZdq{}execute\PYGZus{}kw\PYGZdq{}}\PYG{o}{,} \PYG{n}{asList}\PYG{o}{(}
        \PYG{n}{db}\PYG{o}{,} \PYG{n}{uid}\PYG{o}{,} \PYG{n}{password}\PYG{o}{,}
        \PYG{l+s}{\PYGZdq{}ir.model.fields\PYGZdq{}}\PYG{o}{,} \PYG{l+s}{\PYGZdq{}create\PYGZdq{}}\PYG{o}{,}
        \PYG{n}{asList}\PYG{o}{(}\PYG{k}{new} \PYG{n}{HashMap}\PYG{o}{\PYGZlt{}}\PYG{n}{String}\PYG{o}{,} \PYG{n}{Object}\PYG{o}{\PYGZgt{}}\PYG{o}{(}\PYG{o}{)} \PYG{o}{\PYGZob{}}\PYG{o}{\PYGZob{}}
            \PYG{n}{put}\PYG{o}{(}\PYG{l+s}{\PYGZdq{}model\PYGZus{}id\PYGZdq{}}\PYG{o}{,} \PYG{n}{id}\PYG{o}{)}\PYG{o}{;}
            \PYG{n}{put}\PYG{o}{(}\PYG{l+s}{\PYGZdq{}name\PYGZdq{}}\PYG{o}{,} \PYG{l+s}{\PYGZdq{}x\PYGZus{}name\PYGZdq{}}\PYG{o}{)}\PYG{o}{;}
            \PYG{n}{put}\PYG{o}{(}\PYG{l+s}{\PYGZdq{}ttype\PYGZdq{}}\PYG{o}{,} \PYG{l+s}{\PYGZdq{}char\PYGZdq{}}\PYG{o}{)}\PYG{o}{;}
            \PYG{n}{put}\PYG{o}{(}\PYG{l+s}{\PYGZdq{}state\PYGZdq{}}\PYG{o}{,} \PYG{l+s}{\PYGZdq{}manual\PYGZdq{}}\PYG{o}{)}\PYG{o}{;}
            \PYG{n}{put}\PYG{o}{(}\PYG{l+s}{\PYGZdq{}required\PYGZdq{}}\PYG{o}{,} \PYG{k+kc}{true}\PYG{o}{)}\PYG{o}{;}
        \PYG{o}{\PYGZcb{}}\PYG{o}{\PYGZcb{}}\PYG{o}{)}
\PYG{o}{)}\PYG{o}{)}\PYG{o}{;}
\PYG{k+kd}{final} \PYG{n}{Integer} \PYG{n}{record\PYGZus{}id} \PYG{o}{=} \PYG{o}{(}\PYG{n}{Integer}\PYG{o}{)}\PYG{n}{models}\PYG{o}{.}\PYG{n+na}{execute}\PYG{o}{(}
    \PYG{l+s}{\PYGZdq{}execute\PYGZus{}kw\PYGZdq{}}\PYG{o}{,} \PYG{n}{asList}\PYG{o}{(}
        \PYG{n}{db}\PYG{o}{,} \PYG{n}{uid}\PYG{o}{,} \PYG{n}{password}\PYG{o}{,}
        \PYG{l+s}{\PYGZdq{}x\PYGZus{}custom\PYGZdq{}}\PYG{o}{,} \PYG{l+s}{\PYGZdq{}create\PYGZdq{}}\PYG{o}{,}
        \PYG{n}{asList}\PYG{o}{(}\PYG{k}{new} \PYG{n}{HashMap}\PYG{o}{\PYGZlt{}}\PYG{n}{String}\PYG{o}{,} \PYG{n}{Object}\PYG{o}{\PYGZgt{}}\PYG{o}{(}\PYG{o}{)} \PYG{o}{\PYGZob{}}\PYG{o}{\PYGZob{}}
            \PYG{n}{put}\PYG{o}{(}\PYG{l+s}{\PYGZdq{}x\PYGZus{}name\PYGZdq{}}\PYG{o}{,} \PYG{l+s}{\PYGZdq{}test record\PYGZdq{}}\PYG{o}{)}\PYG{o}{;}
        \PYG{o}{\PYGZcb{}}\PYG{o}{\PYGZcb{}}\PYG{o}{)}
\PYG{o}{)}\PYG{o}{)}\PYG{o}{;}

\PYG{n}{client}\PYG{o}{.}\PYG{n+na}{execute}\PYG{o}{(}
    \PYG{l+s}{\PYGZdq{}execute\PYGZus{}kw\PYGZdq{}}\PYG{o}{,} \PYG{n}{asList}\PYG{o}{(}
        \PYG{n}{db}\PYG{o}{,} \PYG{n}{uid}\PYG{o}{,} \PYG{n}{password}\PYG{o}{,}
        \PYG{l+s}{\PYGZdq{}x\PYGZus{}custom\PYGZdq{}}\PYG{o}{,} \PYG{l+s}{\PYGZdq{}read\PYGZdq{}}\PYG{o}{,}
        \PYG{n}{asList}\PYG{o}{(}\PYG{n}{asList}\PYG{o}{(}\PYG{n}{record\PYGZus{}id}\PYG{o}{)}\PYG{o}{)}
\PYG{o}{)}\PYG{o}{)}\PYG{o}{;}
\end{sphinxVerbatim}

\fvset{hllines={, ,}}%
\begin{sphinxVerbatim}[commandchars=\\\{\}]
\PYG{p}{[}
    \PYG{p}{\PYGZob{}}
        \PYG{n+nt}{\PYGZdq{}create\PYGZus{}uid\PYGZdq{}}\PYG{p}{:} \PYG{p}{[}\PYG{l+m+mi}{1}\PYG{p}{,} \PYG{l+s+s2}{\PYGZdq{}Administrator\PYGZdq{}}\PYG{p}{]}\PYG{p}{,}
        \PYG{n+nt}{\PYGZdq{}x\PYGZus{}name\PYGZdq{}}\PYG{p}{:} \PYG{l+s+s2}{\PYGZdq{}test record\PYGZdq{}}\PYG{p}{,}
        \PYG{n+nt}{\PYGZdq{}\PYGZus{}\PYGZus{}last\PYGZus{}update\PYGZdq{}}\PYG{p}{:} \PYG{l+s+s2}{\PYGZdq{}2014\PYGZhy{}11\PYGZhy{}12 16:32:13\PYGZdq{}}\PYG{p}{,}
        \PYG{n+nt}{\PYGZdq{}write\PYGZus{}uid\PYGZdq{}}\PYG{p}{:} \PYG{p}{[}\PYG{l+m+mi}{1}\PYG{p}{,} \PYG{l+s+s2}{\PYGZdq{}Administrator\PYGZdq{}}\PYG{p}{]}\PYG{p}{,}
        \PYG{n+nt}{\PYGZdq{}write\PYGZus{}date\PYGZdq{}}\PYG{p}{:} \PYG{l+s+s2}{\PYGZdq{}2014\PYGZhy{}11\PYGZhy{}12 16:32:13\PYGZdq{}}\PYG{p}{,}
        \PYG{n+nt}{\PYGZdq{}create\PYGZus{}date\PYGZdq{}}\PYG{p}{:} \PYG{l+s+s2}{\PYGZdq{}2014\PYGZhy{}11\PYGZhy{}12 16:32:13\PYGZdq{}}\PYG{p}{,}
        \PYG{n+nt}{\PYGZdq{}id\PYGZdq{}}\PYG{p}{:} \PYG{l+m+mi}{1}\PYG{p}{,}
        \PYG{n+nt}{\PYGZdq{}display\PYGZus{}name\PYGZdq{}}\PYG{p}{:} \PYG{l+s+s2}{\PYGZdq{}test record\PYGZdq{}}
    \PYG{p}{\PYGZcb{}}
\PYG{p}{]}
\end{sphinxVerbatim}
\phantomsection\label{\detokenize{webservices/iap:webservices-iap}}

\section{In-App Purchases}
\label{\detokenize{webservices/iap:in-app-purchases}}\label{\detokenize{webservices/iap:base64}}\label{\detokenize{webservices/iap::doc}}
In-App Purchase (IAP) allow providers of ongoing services through Odoo apps to
be compensated for ongoing service use rather than — and possibly instead of
— a sole initial purchase.

In that context, Odoo acts mostly as a \sphinxstyleemphasis{broker} between a client and an Odoo
App Developer:
\begin{itemize}
\item {} 
users purchase service tokens from Odoo

\item {} 
service providers draw tokens from the user’s Odoo account when service
is requested

\end{itemize}

\begin{sphinxadmonition}{attention}{Attention:}
This document is intended for \sphinxstyleemphasis{service providers} and presents the latter,
which can be done either via direct \sphinxhref{http://www.jsonrpc.org/specification}{JSON-RPC2} or if you are using Odoo
using the convenience helpers it provides.
\end{sphinxadmonition}


\subsection{Overview}
\label{\detokenize{webservices/iap:overview}}
\begin{figure}[htbp]
\centering
\capstart

\noindent\sphinxincludegraphics{{players}.png}
\caption{The Players}
\begin{sphinxlegend}\begin{itemize}
\item {} 
The Service Provider is (probably) you the reader, you will be providing
value to the client in the form of a service paid per-use

\item {} 
The Client installed your Odoo App, and from there will request services

\item {} 
Odoo brokers crediting, the Client adds credit to their account, and you
can draw credits from there to provide services

\item {} 
The External Service is an optional player: \sphinxstyleemphasis{you} can either provide a
service directly, or you can delegate the actual service acting as a
bridge/translator between an Odoo system and the actual service

\end{itemize}
\end{sphinxlegend}
\label{\detokenize{webservices/iap:id2}}\end{figure}

\begin{figure}[htbp]
\centering
\capstart

\noindent\sphinxincludegraphics{{credits}.jpg}
\caption{The Credits}
\begin{sphinxlegend}
Every service provided through the In-App platform can be used by the
clients with tokens or \sphinxstyleemphasis{credits}. The credits are an integer unit and
their monetary value depends on the service and is decided by the
provider. This could be:
\begin{itemize}
\item {} 
for an sms service: 1 credit = 1 sms,

\item {} 
for an add service: 1 credit = 1 add,

\item {} 
for a postage service: 1 credit = 1 post stamp.

\end{itemize}

A credit can also simply be associated with a fixed amount of money
to palliate the variations of price (e.g. the prices of sms and stamps
may vary following the countries).

The value of the credits is fixed with the help of prepaid credit packs
that the clients can buy on \sphinxurl{https://iap.odoo.com} (see {\hyperref[\detokenize{webservices/iap:iap-packages}]{\sphinxcrossref{\DUrole{std,std-ref}{Packages}}}}).
\end{sphinxlegend}
\label{\detokenize{webservices/iap:id3}}\end{figure}

\begin{sphinxadmonition}{note}{Note:}
in the following explanations we will ignore the External Service,
they’re just a detail of the service you provide
\end{sphinxadmonition}

\begin{figure}[htbp]
\centering
\capstart

\noindent\sphinxincludegraphics{{normal}.png}
\caption{A “normal” service flow}
\begin{sphinxlegend}
If everything goes well, the normal flow is:
\begin{enumerate}
\item {} 
the Client requests a service of some sort

\item {} 
the Service Provider asks Odoo if there are enough credits for the
service in the Client’s account, and creates a transaction over that
amount

\item {} 
the Service Provider provides the service (either on their own or
calling to External Services)

\item {} 
the Service Provider goes back to Odoo to capture (if the service could
be provided) or cancel (if the service could not be provided) the
transaction created at step 2

\item {} 
finally the Service Provider notifies the Client that the service has
been rendered, possibly (depending on the service) displaying or
storing its results in the client’s system

\end{enumerate}
\end{sphinxlegend}
\label{\detokenize{webservices/iap:id4}}\end{figure}

\begin{figure}[htbp]
\centering
\capstart

\noindent\sphinxincludegraphics{{no-credit}.png}
\caption{Insufficient Credits}
\begin{sphinxlegend}
If the Client’s account lacks credits for the service, however
\begin{enumerate}
\item {} 
the Client requests a service as previously

\item {} 
the Service Provider asks Odoo if there are enough credits on the
Client’s account and gets a negative reply

\item {} 
this is signaled back to the Client

\item {} 
who is redirected to their Odoo account to credit it and re-try

\end{enumerate}
\end{sphinxlegend}
\label{\detokenize{webservices/iap:id5}}\end{figure}


\subsection{Building your service}
\label{\detokenize{webservices/iap:building-your-service}}
For this example, the service we will provide is \textasciitilde{}\textasciitilde{}mining dogecoins\textasciitilde{}\textasciitilde{} burning
10 seconds of CPU for a credit. For your own services, you could for example:
\begin{itemize}
\item {} 
provide an online service yourself (e.g. convert quotations to faxes for
business in Japan)

\item {} 
provide an \sphinxstyleemphasis{offline} service yourself (e.g. provide accountancy service)

\item {} 
act as intermediary to an other service provider (e.g. bridge to an MMS
gateway)

\end{itemize}


\subsubsection{Register the service on Odoo}
\label{\detokenize{webservices/iap:register-the-service-on-odoo}}
The first step is to register your service on the IAP endpoint (production
and/or test) before you can actually query user accounts. To create a service,
go to your \sphinxstyleemphasis{Portal Account} on the IAP endpoint (\sphinxurl{https://iap.odoo.com} for
production, \sphinxurl{https://iap-sandbox.odoo.com} for testing, the endpoints are
\sphinxstyleemphasis{independent} and \sphinxstyleemphasis{not synchronized}).

\begin{sphinxadmonition}{note}{Note:}
On production, there is a manual validation step before the service
can be used to manage real transactions. This step is automatically passed when
on sandbox to ease the tests.
\end{sphinxadmonition}

Log in then go to \sphinxmenuselection{My Account \(\rightarrow\) Your In-App Services}, click
Create and provide the name of your service.

The now created service has \sphinxstyleemphasis{two} important fields:
\begin{itemize}
\item {} 
\sphinxcode{\sphinxupquote{name}} - {\hyperref[\detokenize{webservices/iap:ServiceName}]{\sphinxcrossref{\sphinxcode{\sphinxupquote{ServiceName}}}}}: this will identify your service in the
client’s {\hyperref[\detokenize{webservices/iap:iap-odoo-app}]{\sphinxcrossref{\DUrole{std,std-ref}{app}}}} communicates directly with IAP.

\item {} 
\sphinxcode{\sphinxupquote{key}} - {\hyperref[\detokenize{webservices/iap:ServiceKey}]{\sphinxcrossref{\sphinxcode{\sphinxupquote{ServiceKey}}}}}: the developer key that identifies you in
IAP (see {\hyperref[\detokenize{webservices/iap:iap-service}]{\sphinxcrossref{\DUrole{std,std-ref}{your service}}}}) and allows to draw credits from
the client’s account.

\end{itemize}

\begin{sphinxadmonition}{warning}{Warning:}
The {\hyperref[\detokenize{webservices/iap:ServiceName}]{\sphinxcrossref{\sphinxcode{\sphinxupquote{ServiceName}}}}} is unique and should usually match the name of your
Odoo App.
\end{sphinxadmonition}

\begin{sphinxadmonition}{danger}{Danger:}
Your {\hyperref[\detokenize{webservices/iap:ServiceKey}]{\sphinxcrossref{\sphinxcode{\sphinxupquote{ServiceKey}}}}} \sphinxstyleemphasis{is a secret}, leaking your service key
allows other application developers to draw credits bought for
your service(s).
\end{sphinxadmonition}

\noindent{\hspace*{\fill}\sphinxincludegraphics{{service_select}.png}\hspace*{\fill}}

\noindent{\hspace*{\fill}\sphinxincludegraphics{{service_create}.png}\hspace*{\fill}}

\noindent{\hspace*{\fill}\sphinxincludegraphics{{service_packs}.png}\hspace*{\fill}}

You can then create \sphinxstyleemphasis{credit packs} which clients can purchase in order to
use your service.


\subsubsection{Packages}
\label{\detokenize{webservices/iap:packages}}\label{\detokenize{webservices/iap:iap-packages}}
The credit packages are essentially a product with 4 characteristics.
\begin{itemize}
\item {} 
Name: the name of the package,

\item {} 
Description: details on the package that will appear on the shop page as
well as the invoice,

\item {} 
Credits: the amount of credits the client is entitled to when buying the package,

\item {} 
Price: the price in \sphinxstyleemphasis{EUROS} for the time being (USD support is planned).

\end{itemize}

\begin{sphinxadmonition}{note}{Note:}
Odoo takes a 25\% commission on all package sales. Adjust your selling price accordingly.
\end{sphinxadmonition}

\begin{sphinxadmonition}{note}{Note:}
Depending on the strategy, the price per credit can vary from one
package to another.
\end{sphinxadmonition}

\noindent{\hspace*{\fill}\sphinxincludegraphics{{package}.png}\hspace*{\fill}}


\subsubsection{Odoo App}
\label{\detokenize{webservices/iap:odoo-app}}\label{\detokenize{webservices/iap:iap-odoo-app}}
The second step is to develop an \sphinxhref{https://www.odoo.com/apps}{Odoo App} which clients can install in their
Odoo instance and through which they can \sphinxstyleemphasis{request} services you will provide.
Our app will just add a button to the Partners form which lets a user request
burning some CPU time on the server.

First, we’ll create an \sphinxstyleemphasis{odoo module} depending on \sphinxcode{\sphinxupquote{iap}}. IAP is a standard
V11 module and the dependency ensures a local account is properly set up and
we will have access to some necessary views and useful helpers
\sphinxstyleemphasis{coalroller/\_\_manifest\_\_.py}
\fvset{hllines={, 1, 2, 3, 4, 5,}}%
\begin{sphinxVerbatim}[commandchars=\\\{\}]
\PYG{p}{\PYGZob{}}
    \PYG{l+s+s1}{\PYGZsq{}}\PYG{l+s+s1}{name}\PYG{l+s+s1}{\PYGZsq{}}\PYG{p}{:} \PYG{l+s+s2}{\PYGZdq{}}\PYG{l+s+s2}{Coal Roller}\PYG{l+s+s2}{\PYGZdq{}}\PYG{p}{,}
    \PYG{l+s+s1}{\PYGZsq{}}\PYG{l+s+s1}{category}\PYG{l+s+s1}{\PYGZsq{}}\PYG{p}{:} \PYG{l+s+s1}{\PYGZsq{}}\PYG{l+s+s1}{Tools}\PYG{l+s+s1}{\PYGZsq{}}\PYG{p}{,}
    \PYG{l+s+s1}{\PYGZsq{}}\PYG{l+s+s1}{depends}\PYG{l+s+s1}{\PYGZsq{}}\PYG{p}{:} \PYG{p}{[}\PYG{l+s+s1}{\PYGZsq{}}\PYG{l+s+s1}{iap}\PYG{l+s+s1}{\PYGZsq{}}\PYG{p}{]}\PYG{p}{,}
\PYG{p}{\PYGZcb{}}
\end{sphinxVerbatim}
\sphinxstyleemphasis{coalroller/\_\_init\_\_.py}
\fvset{hllines={, 1,}}%
\begin{sphinxVerbatim}[commandchars=\\\{\}]
\PYG{c+c1}{\PYGZsh{} \PYGZhy{}*\PYGZhy{} coding: utf\PYGZhy{}8 \PYGZhy{}*\PYGZhy{}}
\end{sphinxVerbatim}

Second, the “local” side of the integration, here we will only be adding an
action button to the partners view, but you can of course provide significant
local value via your application and additional parts via a remote service.
\sphinxstyleemphasis{coalroller/\_\_manifest\_\_.py}
\fvset{hllines={, 4, 5, 6,}}%
\begin{sphinxVerbatim}[commandchars=\\\{\}]
    \PYG{l+s+s1}{\PYGZsq{}}\PYG{l+s+s1}{name}\PYG{l+s+s1}{\PYGZsq{}}\PYG{p}{:} \PYG{l+s+s2}{\PYGZdq{}}\PYG{l+s+s2}{Coal Roller}\PYG{l+s+s2}{\PYGZdq{}}\PYG{p}{,}
    \PYG{l+s+s1}{\PYGZsq{}}\PYG{l+s+s1}{category}\PYG{l+s+s1}{\PYGZsq{}}\PYG{p}{:} \PYG{l+s+s1}{\PYGZsq{}}\PYG{l+s+s1}{Tools}\PYG{l+s+s1}{\PYGZsq{}}\PYG{p}{,}
    \PYG{l+s+s1}{\PYGZsq{}}\PYG{l+s+s1}{depends}\PYG{l+s+s1}{\PYGZsq{}}\PYG{p}{:} \PYG{p}{[}\PYG{l+s+s1}{\PYGZsq{}}\PYG{l+s+s1}{iap}\PYG{l+s+s1}{\PYGZsq{}}\PYG{p}{]}\PYG{p}{,}
    \PYG{l+s+s1}{\PYGZsq{}}\PYG{l+s+s1}{data}\PYG{l+s+s1}{\PYGZsq{}}\PYG{p}{:} \PYG{p}{[}
        \PYG{l+s+s1}{\PYGZsq{}}\PYG{l+s+s1}{views/views.xml}\PYG{l+s+s1}{\PYGZsq{}}\PYG{p}{,}
    \PYG{p}{]}\PYG{p}{,}
\PYG{p}{\PYGZcb{}}
\end{sphinxVerbatim}
\sphinxstyleemphasis{coalroller/views/views.xml}
\fvset{hllines={, 1, 2, 3, 4, 5, 6, 7, 8, 9, 10, 11, 12, 13, 14, 15, 16, 17,}}%
\begin{sphinxVerbatim}[commandchars=\\\{\}]
\PYG{n+nt}{\PYGZlt{}odoo}\PYG{n+nt}{\PYGZgt{}}
  \PYG{n+nt}{\PYGZlt{}record} \PYG{n+na}{model=}\PYG{l+s}{\PYGZdq{}ir.ui.view\PYGZdq{}} \PYG{n+na}{id=}\PYG{l+s}{\PYGZdq{}partner\PYGZus{}form\PYGZus{}coalroll\PYGZdq{}}\PYG{n+nt}{\PYGZgt{}}
    \PYG{n+nt}{\PYGZlt{}field} \PYG{n+na}{name=}\PYG{l+s}{\PYGZdq{}name\PYGZdq{}}\PYG{n+nt}{\PYGZgt{}}partner.form.coalroll\PYG{n+nt}{\PYGZlt{}/field\PYGZgt{}}
    \PYG{n+nt}{\PYGZlt{}field} \PYG{n+na}{name=}\PYG{l+s}{\PYGZdq{}model\PYGZdq{}}\PYG{n+nt}{\PYGZgt{}}res.partner\PYG{n+nt}{\PYGZlt{}/field\PYGZgt{}}
    \PYG{n+nt}{\PYGZlt{}field} \PYG{n+na}{name=}\PYG{l+s}{\PYGZdq{}inherit\PYGZus{}id\PYGZdq{}} \PYG{n+na}{ref=}\PYG{l+s}{\PYGZdq{}base.view\PYGZus{}partner\PYGZus{}form\PYGZdq{}} \PYG{n+nt}{/\PYGZgt{}}
    \PYG{n+nt}{\PYGZlt{}field} \PYG{n+na}{name=}\PYG{l+s}{\PYGZdq{}arch\PYGZdq{}} \PYG{n+na}{type=}\PYG{l+s}{\PYGZdq{}xml\PYGZdq{}}\PYG{n+nt}{\PYGZgt{}}
      \PYG{n+nt}{\PYGZlt{}xpath} \PYG{n+na}{expr=}\PYG{l+s}{\PYGZdq{}//div[@name=\PYGZsq{}button\PYGZus{}box\PYGZsq{}]\PYGZdq{}}\PYG{n+nt}{\PYGZgt{}}
        \PYG{n+nt}{\PYGZlt{}button} \PYG{n+na}{type=}\PYG{l+s}{\PYGZdq{}object\PYGZdq{}} \PYG{n+na}{name=}\PYG{l+s}{\PYGZdq{}action\PYGZus{}partner\PYGZus{}coalroll\PYGZdq{}}
                \PYG{n+na}{class=}\PYG{l+s}{\PYGZdq{}oe\PYGZus{}stat\PYGZus{}button\PYGZdq{}} \PYG{n+na}{icon=}\PYG{l+s}{\PYGZdq{}fa\PYGZhy{}gears\PYGZdq{}}\PYG{n+nt}{\PYGZgt{}}
          \PYG{n+nt}{\PYGZlt{}div} \PYG{n+na}{class=}\PYG{l+s}{\PYGZdq{}o\PYGZus{}form\PYGZus{}field o\PYGZus{}stat\PYGZus{}info\PYGZdq{}}\PYG{n+nt}{\PYGZgt{}}
            \PYG{n+nt}{\PYGZlt{}span} \PYG{n+na}{class=}\PYG{l+s}{\PYGZdq{}o\PYGZus{}stat\PYGZus{}text\PYGZdq{}}\PYG{n+nt}{\PYGZgt{}}Roll Coal\PYG{n+nt}{\PYGZlt{}/span\PYGZgt{}}
          \PYG{n+nt}{\PYGZlt{}/div\PYGZgt{}}
        \PYG{n+nt}{\PYGZlt{}/button\PYGZgt{}}
      \PYG{n+nt}{\PYGZlt{}/xpath\PYGZgt{}}
    \PYG{n+nt}{\PYGZlt{}/field\PYGZgt{}}
  \PYG{n+nt}{\PYGZlt{}/record\PYGZgt{}}
\PYG{n+nt}{\PYGZlt{}/odoo\PYGZgt{}}
\end{sphinxVerbatim}

\noindent{\hspace*{\fill}\sphinxincludegraphics{{button}.png}\hspace*{\fill}}

We can now implement the action method/callback. This will \sphinxstyleemphasis{call our own
server}.

There are no requirements when it comes to the server or the communication
protocol between the app and our server, but \sphinxcode{\sphinxupquote{iap}} provides a
\sphinxcode{\sphinxupquote{jsonrpc()}} helper to call a \sphinxhref{http://www.jsonrpc.org/specification}{JSON-RPC2} endpoint on an
other Odoo instance and transparently re-raise relevant Odoo exceptions
({\hyperref[\detokenize{webservices/iap:odoo.addons.iap.models.iap.InsufficientCreditError}]{\sphinxcrossref{\sphinxcode{\sphinxupquote{InsufficientCreditError}}}}},
{\hyperref[\detokenize{webservices/iap:odoo.exceptions.AccessError}]{\sphinxcrossref{\sphinxcode{\sphinxupquote{odoo.exceptions.AccessError}}}}} and {\hyperref[\detokenize{webservices/iap:odoo.exceptions.UserError}]{\sphinxcrossref{\sphinxcode{\sphinxupquote{odoo.exceptions.UserError}}}}}).

In that call, we will need to provide:
\begin{itemize}
\item {} 
any relevant client parameter (none here)

\item {} 
the {\hyperref[\detokenize{webservices/iap:UserToken}]{\sphinxcrossref{\sphinxcode{\sphinxupquote{token}}}}} of the current client, this is provided by
the \sphinxcode{\sphinxupquote{iap.account}} model’s \sphinxcode{\sphinxupquote{account\_token}} field. You can retrieve the
account for your service by calling \sphinxcode{\sphinxupquote{env{[}'iap.account'{]}.get(\sphinxstyleemphasis{service\_name})}}
where {\hyperref[\detokenize{webservices/iap:ServiceName}]{\sphinxcrossref{\sphinxcode{\sphinxupquote{service\_name}}}}} is the name of the service registered
on IAP endpoint.

\end{itemize}
\sphinxstyleemphasis{coalroller/\_\_init\_\_.py}
\fvset{hllines={, 2,}}%
\begin{sphinxVerbatim}[commandchars=\\\{\}]
\PYG{c+c1}{\PYGZsh{} \PYGZhy{}*\PYGZhy{} coding: utf\PYGZhy{}8 \PYGZhy{}*\PYGZhy{}}
\PYG{k+kn}{from} \PYG{n+nn}{.} \PYG{k+kn}{import} \PYG{n}{models}
\end{sphinxVerbatim}
\sphinxstyleemphasis{coalroller/models/\_\_init\_\_.py}
\fvset{hllines={, 1,}}%
\begin{sphinxVerbatim}[commandchars=\\\{\}]
\PYG{k+kn}{from} \PYG{n+nn}{.} \PYG{k+kn}{import} \PYG{n}{res\PYGZus{}partner}
\end{sphinxVerbatim}
\sphinxstyleemphasis{coalroller/models/res\_partner.py}
\fvset{hllines={, 1, 2, 3, 4, 5, 6, 7, 8, 9, 10, 11, 12, 13, 14, 15, 16, 17, 18, 19, 20, 21, 22,}}%
\begin{sphinxVerbatim}[commandchars=\\\{\}]
\PYG{c+c1}{\PYGZsh{} \PYGZhy{}*\PYGZhy{} coding: utf\PYGZhy{}8 \PYGZhy{}*\PYGZhy{}}
\PYG{k+kn}{from} \PYG{n+nn}{odoo} \PYG{k+kn}{import} \PYG{n}{api}\PYG{p}{,} \PYG{n}{models}
\PYG{k+kn}{from} \PYG{n+nn}{odoo.addons.iap} \PYG{k+kn}{import} \PYG{n}{jsonrpc}\PYG{p}{,} \PYG{n}{InsufficientCreditError}

\PYG{c+c1}{\PYGZsh{} whichever URL you deploy the service at, here we will run the remote}
\PYG{c+c1}{\PYGZsh{} service in a local Odoo bound to the port 8070}
\PYG{n}{DEFAULT\PYGZus{}ENDPOINT} \PYG{o}{=} \PYG{l+s+s1}{\PYGZsq{}}\PYG{l+s+s1}{http://localhost:8070}\PYG{l+s+s1}{\PYGZsq{}}
\PYG{k}{class} \PYG{n+nc}{Partner}\PYG{p}{(}\PYG{n}{models}\PYG{o}{.}\PYG{n}{Model}\PYG{p}{)}\PYG{p}{:}
    \PYG{n}{\PYGZus{}inherit} \PYG{o}{=} \PYG{l+s+s1}{\PYGZsq{}}\PYG{l+s+s1}{res.partner}\PYG{l+s+s1}{\PYGZsq{}}
    \PYG{n+nd}{@api.multi}
    \PYG{k}{def} \PYG{n+nf}{action\PYGZus{}partner\PYGZus{}coalroll}\PYG{p}{(}\PYG{n+nb+bp}{self}\PYG{p}{)}\PYG{p}{:}
        \PYG{c+c1}{\PYGZsh{} fetch the user\PYGZsq{}s token for our service}
        \PYG{n}{user\PYGZus{}token} \PYG{o}{=} \PYG{n+nb+bp}{self}\PYG{o}{.}\PYG{n}{env}\PYG{p}{[}\PYG{l+s+s1}{\PYGZsq{}}\PYG{l+s+s1}{iap.account}\PYG{l+s+s1}{\PYGZsq{}}\PYG{p}{]}\PYG{o}{.}\PYG{n}{get}\PYG{p}{(}\PYG{l+s+s1}{\PYGZsq{}}\PYG{l+s+s1}{coalroller}\PYG{l+s+s1}{\PYGZsq{}}\PYG{p}{)}
        \PYG{n}{params} \PYG{o}{=} \PYG{p}{\PYGZob{}}
            \PYG{c+c1}{\PYGZsh{} we don\PYGZsq{}t have any parameter to provide}
            \PYG{l+s+s1}{\PYGZsq{}}\PYG{l+s+s1}{account\PYGZus{}token}\PYG{l+s+s1}{\PYGZsq{}}\PYG{p}{:} \PYG{n}{user\PYGZus{}token}\PYG{o}{.}\PYG{n}{account\PYGZus{}token}
        \PYG{p}{\PYGZcb{}}
        \PYG{c+c1}{\PYGZsh{} ir.config\PYGZus{}parameter allows locally overriding the endpoint}
        \PYG{c+c1}{\PYGZsh{} for testing \PYGZam{} al}
        \PYG{n}{endpoint} \PYG{o}{=} \PYG{n+nb+bp}{self}\PYG{o}{.}\PYG{n}{env}\PYG{p}{[}\PYG{l+s+s1}{\PYGZsq{}}\PYG{l+s+s1}{ir.config\PYGZus{}parameter}\PYG{l+s+s1}{\PYGZsq{}}\PYG{p}{]}\PYG{o}{.}\PYG{n}{sudo}\PYG{p}{(}\PYG{p}{)}\PYG{o}{.}\PYG{n}{get\PYGZus{}param}\PYG{p}{(}\PYG{l+s+s1}{\PYGZsq{}}\PYG{l+s+s1}{coalroller.endpoint}\PYG{l+s+s1}{\PYGZsq{}}\PYG{p}{,} \PYG{n}{DEFAULT\PYGZus{}ENDPOINT}\PYG{p}{)}
        \PYG{n}{jsonrpc}\PYG{p}{(}\PYG{n}{endpoint} \PYG{o}{+} \PYG{l+s+s1}{\PYGZsq{}}\PYG{l+s+s1}{/roll}\PYG{l+s+s1}{\PYGZsq{}}\PYG{p}{,} \PYG{n}{params}\PYG{o}{=}\PYG{n}{params}\PYG{p}{)}
        \PYG{k}{return} \PYG{n+nb+bp}{True}
\end{sphinxVerbatim}

\begin{sphinxadmonition}{note}{Note:}
\sphinxcode{\sphinxupquote{iap}} automatically handles
{\hyperref[\detokenize{webservices/iap:odoo.addons.iap.models.iap.InsufficientCreditError}]{\sphinxcrossref{\sphinxcode{\sphinxupquote{InsufficientCreditError}}}}} coming from the action
and prompts the user to add credits to their account.

\sphinxcode{\sphinxupquote{jsonrpc()}} takes care of re-raising
{\hyperref[\detokenize{webservices/iap:odoo.addons.iap.models.iap.InsufficientCreditError}]{\sphinxcrossref{\sphinxcode{\sphinxupquote{InsufficientCreditError}}}}} for you.
\end{sphinxadmonition}

\begin{sphinxadmonition}{danger}{Danger:}
If you are not using \sphinxcode{\sphinxupquote{jsonrpc()}} you \sphinxstyleemphasis{must} be
careful to re-raise
{\hyperref[\detokenize{webservices/iap:odoo.addons.iap.models.iap.InsufficientCreditError}]{\sphinxcrossref{\sphinxcode{\sphinxupquote{InsufficientCreditError}}}}} in your handler
otherwise the user will not be prompted to credit their account, and the
next call will fail the same way.
\end{sphinxadmonition}


\subsubsection{Service}
\label{\detokenize{webservices/iap:service}}\label{\detokenize{webservices/iap:iap-service}}
Though that is not \sphinxstyleemphasis{required}, since \sphinxcode{\sphinxupquote{iap}} provides both a client helper
for \sphinxhref{http://www.jsonrpc.org/specification}{JSON-RPC2} calls (\sphinxcode{\sphinxupquote{jsonrpc()}}) and a service helper
for transactions ({\hyperref[\detokenize{webservices/iap:odoo.addons.iap.models.iap.charge}]{\sphinxcrossref{\sphinxcode{\sphinxupquote{charge}}}}}) we will also be
implementing the service side as an Odoo module:
\sphinxstyleemphasis{coalroller\_service/\_\_init\_\_.py}
\fvset{hllines={, 1,}}%
\begin{sphinxVerbatim}[commandchars=\\\{\}]
\PYG{c+c1}{\PYGZsh{} \PYGZhy{}*\PYGZhy{} encoding: utf\PYGZhy{}8 \PYGZhy{}*\PYGZhy{}}
\end{sphinxVerbatim}
\sphinxstyleemphasis{coalroller\_service/\_\_manifest\_\_.py}
\fvset{hllines={, 1, 2, 3, 4, 5,}}%
\begin{sphinxVerbatim}[commandchars=\\\{\}]
\PYG{p}{\PYGZob{}}
    \PYG{l+s+s1}{\PYGZsq{}}\PYG{l+s+s1}{name}\PYG{l+s+s1}{\PYGZsq{}}\PYG{p}{:} \PYG{l+s+s2}{\PYGZdq{}}\PYG{l+s+s2}{Coal Roller Service}\PYG{l+s+s2}{\PYGZdq{}}\PYG{p}{,}
    \PYG{l+s+s1}{\PYGZsq{}}\PYG{l+s+s1}{category}\PYG{l+s+s1}{\PYGZsq{}}\PYG{p}{:} \PYG{l+s+s1}{\PYGZsq{}}\PYG{l+s+s1}{Tools}\PYG{l+s+s1}{\PYGZsq{}}\PYG{p}{,}
    \PYG{l+s+s1}{\PYGZsq{}}\PYG{l+s+s1}{depends}\PYG{l+s+s1}{\PYGZsq{}}\PYG{p}{:} \PYG{p}{[}\PYG{l+s+s1}{\PYGZsq{}}\PYG{l+s+s1}{iap}\PYG{l+s+s1}{\PYGZsq{}}\PYG{p}{]}\PYG{p}{,}
\PYG{p}{\PYGZcb{}}
\end{sphinxVerbatim}

Since the query from the client comes as \sphinxhref{http://www.jsonrpc.org/specification}{JSON-RPC2} we will need the
corresponding controller which can call {\hyperref[\detokenize{webservices/iap:odoo.addons.iap.models.iap.charge}]{\sphinxcrossref{\sphinxcode{\sphinxupquote{charge}}}}} and
perform the service within:
\sphinxstyleemphasis{coalroller\_service/controllers/main.py}
\fvset{hllines={, 1, 2, 3, 4, 5, 6, 7, 8, 9, 10, 11, 12, 13, 14, 15, 16, 17, 18, 19, 20, 21, 22, 23, 24, 25, 26, 27, 28, 29,}}%
\begin{sphinxVerbatim}[commandchars=\\\{\}]
\PYG{k+kn}{import} \PYG{n+nn}{time}

\PYG{k+kn}{from} \PYG{n+nn}{passlib} \PYG{k+kn}{import} \PYG{n}{pwd}\PYG{p}{,} \PYG{n+nb}{hash}

\PYG{k+kn}{from} \PYG{n+nn}{odoo} \PYG{k+kn}{import} \PYG{n}{http}
\PYG{k+kn}{from} \PYG{n+nn}{odoo.addons.iap} \PYG{k+kn}{import} \PYG{n}{charge}

\PYG{k}{class} \PYG{n+nc}{CoalBurnerController}\PYG{p}{(}\PYG{n}{http}\PYG{o}{.}\PYG{n}{Controller}\PYG{p}{)}\PYG{p}{:}
    \PYG{n+nd}{@http.route}\PYG{p}{(}\PYG{l+s+s1}{\PYGZsq{}}\PYG{l+s+s1}{/roll}\PYG{l+s+s1}{\PYGZsq{}}\PYG{p}{,} \PYG{n+nb}{type}\PYG{o}{=}\PYG{l+s+s1}{\PYGZsq{}}\PYG{l+s+s1}{json}\PYG{l+s+s1}{\PYGZsq{}}\PYG{p}{,} \PYG{n}{auth}\PYG{o}{=}\PYG{l+s+s1}{\PYGZsq{}}\PYG{l+s+s1}{none}\PYG{l+s+s1}{\PYGZsq{}}\PYG{p}{,} \PYG{n}{csrf}\PYG{o}{=}\PYG{l+s+s1}{\PYGZsq{}}\PYG{l+s+s1}{false}\PYG{l+s+s1}{\PYGZsq{}}\PYG{p}{)}
    \PYG{k}{def} \PYG{n+nf}{roll}\PYG{p}{(}\PYG{n+nb+bp}{self}\PYG{p}{,} \PYG{n}{account\PYGZus{}token}\PYG{p}{)}\PYG{p}{:}
        \PYG{c+c1}{\PYGZsh{} the service key *is a secret*, it should not be committed in}
        \PYG{c+c1}{\PYGZsh{} the source}
        \PYG{n}{service\PYGZus{}key} \PYG{o}{=} \PYG{n+nb+bp}{self}\PYG{o}{.}\PYG{n}{env}\PYG{p}{[}\PYG{l+s+s1}{\PYGZsq{}}\PYG{l+s+s1}{ir.config\PYGZus{}parameter}\PYG{l+s+s1}{\PYGZsq{}}\PYG{p}{]}\PYG{o}{.}\PYG{n}{sudo}\PYG{p}{(}\PYG{p}{)}\PYG{o}{.}\PYG{n}{get\PYGZus{}param}\PYG{p}{(}\PYG{l+s+s1}{\PYGZsq{}}\PYG{l+s+s1}{coalroller.service\PYGZus{}key}\PYG{l+s+s1}{\PYGZsq{}}\PYG{p}{)}

        \PYG{c+c1}{\PYGZsh{} we charge 1 credit for 10 seconds of CPU}
        \PYG{n}{cost} \PYG{o}{=} \PYG{l+m+mi}{1}
        \PYG{c+c1}{\PYGZsh{} TODO: allow the user to specify how many (tens of seconds) of CPU they want to use}
        \PYG{k}{with} \PYG{n}{charge}\PYG{p}{(}\PYG{n}{http}\PYG{o}{.}\PYG{n}{request}\PYG{o}{.}\PYG{n}{env}\PYG{p}{,} \PYG{n}{service\PYGZus{}key}\PYG{p}{,} \PYG{n}{account\PYGZus{}token}\PYG{p}{,} \PYG{n}{cost}\PYG{p}{)}\PYG{p}{:}

            \PYG{c+c1}{\PYGZsh{} 10 seconds of CPU per credit}
            \PYG{n}{end} \PYG{o}{=} \PYG{n}{time}\PYG{o}{.}\PYG{n}{time}\PYG{p}{(}\PYG{p}{)} \PYG{o}{+} \PYG{p}{(}\PYG{l+m+mi}{10} \PYG{o}{*} \PYG{n}{cost}\PYG{p}{)}
            \PYG{k}{while} \PYG{n}{time}\PYG{o}{.}\PYG{n}{time}\PYG{p}{(}\PYG{p}{)} \PYG{o}{\PYGZlt{}} \PYG{n}{end}\PYG{p}{:}
                \PYG{c+c1}{\PYGZsh{} we will use CPU doing useful things: generating and}
                \PYG{c+c1}{\PYGZsh{} hashing passphrases}
                \PYG{n}{p} \PYG{o}{=} \PYG{n}{pwd}\PYG{o}{.}\PYG{n}{genphrase}\PYG{p}{(}\PYG{p}{)}
                \PYG{n}{h} \PYG{o}{=} \PYG{n+nb}{hash}\PYG{o}{.}\PYG{n}{pbkdf2\PYGZus{}sha512}\PYG{o}{.}\PYG{n}{hash}\PYG{p}{(}\PYG{n}{p}\PYG{p}{)}
        \PYG{c+c1}{\PYGZsh{} here we don\PYGZsq{}t have anything useful to the client, an error}
        \PYG{c+c1}{\PYGZsh{} will be raised \PYGZam{} transmitted in case of issue, if no error}
        \PYG{c+c1}{\PYGZsh{} is raised we did the job}
\end{sphinxVerbatim}
\sphinxstyleemphasis{coalroller\_service/controllers/\_\_init\_\_.py}
\fvset{hllines={, 1, 2,}}%
\begin{sphinxVerbatim}[commandchars=\\\{\}]
\PYG{c+c1}{\PYGZsh{} \PYGZhy{}*\PYGZhy{} encoding: utf\PYGZhy{}8 \PYGZhy{}*\PYGZhy{}}
\PYG{k+kn}{from} \PYG{n+nn}{.} \PYG{k+kn}{import} \PYG{n}{main}
\end{sphinxVerbatim}
\sphinxstyleemphasis{coalroller\_service/\_\_init\_\_.py}
\fvset{hllines={, 2,}}%
\begin{sphinxVerbatim}[commandchars=\\\{\}]
\PYG{c+c1}{\PYGZsh{} \PYGZhy{}*\PYGZhy{} encoding: utf\PYGZhy{}8 \PYGZhy{}*\PYGZhy{}}
\PYG{k+kn}{from} \PYG{n+nn}{.} \PYG{k+kn}{import} \PYG{n}{controllers}
\end{sphinxVerbatim}

The {\hyperref[\detokenize{webservices/iap:odoo.addons.iap.models.iap.charge}]{\sphinxcrossref{\sphinxcode{\sphinxupquote{charge}}}}} helper will:

\begin{sphinxadmonition}{note}{Note:}
Since the 15th of January 2018, a new functionality that allows one to capture a different amount than autorized has been added.
See {\hyperref[\detokenize{webservices/iap:iap-charging}]{\sphinxcrossref{\DUrole{std,std-ref}{Charging}}}}
\end{sphinxadmonition}
\begin{enumerate}
\item {} 
authorize (create) a transaction with the specified number of credits,
if the account does not have enough credits it will raise the relevant
error

\item {} 
execute the body of the \sphinxcode{\sphinxupquote{with}} statement

\item {} 
(NEW) if the body of the \sphinxcode{\sphinxupquote{with}} executes succesfully, update the price
of the transaction if needed

\item {} 
capture (confirm) the transaction

\item {} 
otherwise if an error is raised from the body of the \sphinxcode{\sphinxupquote{with}} cancel the
transaction (and release the hold on the credits)

\end{enumerate}

\begin{sphinxadmonition}{danger}{Danger:}
By default, {\hyperref[\detokenize{webservices/iap:odoo.addons.iap.models.iap.charge}]{\sphinxcrossref{\sphinxcode{\sphinxupquote{charge}}}}} contacts the \sphinxstyleemphasis{production}
IAP endpoint, \sphinxurl{https://iap.odoo.com}. While developing and testing your
service you may want to point it towards the \sphinxstyleemphasis{development} IAP endpoint
\sphinxurl{https://iap-sandbox.odoo.com}.

To do so, set the \sphinxcode{\sphinxupquote{iap.endpoint}} config parameter in your service
Odoo: in debug/developer mode, \sphinxmenuselection{Setting \(\rightarrow\) Technical \(\rightarrow\)
Parameters \(\rightarrow\) System Parameters}, just define an entry for the key
\sphinxcode{\sphinxupquote{iap.endpoint}} if none already exists).
\end{sphinxadmonition}

The {\hyperref[\detokenize{webservices/iap:odoo.addons.iap.models.iap.charge}]{\sphinxcrossref{\sphinxcode{\sphinxupquote{charge}}}}} helper has two additional optional
parameters we can use to make things clearer to the end-user:
\begin{description}
\item[{\sphinxcode{\sphinxupquote{description}}}] \leavevmode
is a message which will be associated with the transaction and will be
displayed in the user’s dashboard, it is useful to remind the user why
the charge exists

\item[{\sphinxcode{\sphinxupquote{credit\_template}}}] \leavevmode
is the name of a {\hyperref[\detokenize{reference/qweb:reference-qweb}]{\sphinxcrossref{\DUrole{std,std-ref}{QWeb}}}} template which will be rendered
and shown to the user if their account has less credit available than the
service provider is requesting, its purpose is to tell your users why
they should be interested in your IAP offers

\end{description}
\sphinxstyleemphasis{coalroller\_service/controllers/main.py}
\fvset{hllines={, 4, 9, 10, 11,}}%
\begin{sphinxVerbatim}[commandchars=\\\{\}]
    \PYG{k}{def} \PYG{n+nf}{roll}\PYG{p}{(}\PYG{n+nb+bp}{self}\PYG{p}{,} \PYG{n}{account\PYGZus{}token}\PYG{p}{)}\PYG{p}{:}
        \PYG{c+c1}{\PYGZsh{} the service key *is a secret*, it should not be committed in}
        \PYG{c+c1}{\PYGZsh{} the source}
        \PYG{n}{service\PYGZus{}key} \PYG{o}{=} \PYG{n}{http}\PYG{o}{.}\PYG{n}{request}\PYG{o}{.}\PYG{n}{env}\PYG{p}{[}\PYG{l+s+s1}{\PYGZsq{}}\PYG{l+s+s1}{ir.config\PYGZus{}parameter}\PYG{l+s+s1}{\PYGZsq{}}\PYG{p}{]}\PYG{o}{.}\PYG{n}{sudo}\PYG{p}{(}\PYG{p}{)}\PYG{o}{.}\PYG{n}{get\PYGZus{}param}\PYG{p}{(}\PYG{l+s+s1}{\PYGZsq{}}\PYG{l+s+s1}{coalroller.service\PYGZus{}key}\PYG{l+s+s1}{\PYGZsq{}}\PYG{p}{)}

        \PYG{c+c1}{\PYGZsh{} we charge 1 credit for 10 seconds of CPU}
        \PYG{n}{cost} \PYG{o}{=} \PYG{l+m+mi}{1}
        \PYG{c+c1}{\PYGZsh{} TODO: allow the user to specify how many (tens of seconds) of CPU they want to use}
        \PYG{k}{with} \PYG{n}{charge}\PYG{p}{(}\PYG{n}{http}\PYG{o}{.}\PYG{n}{request}\PYG{o}{.}\PYG{n}{env}\PYG{p}{,} \PYG{n}{service\PYGZus{}key}\PYG{p}{,} \PYG{n}{account\PYGZus{}token}\PYG{p}{,} \PYG{n}{cost}\PYG{p}{,}
                    \PYG{n}{description}\PYG{o}{=}\PYG{l+s+s2}{\PYGZdq{}}\PYG{l+s+s2}{We}\PYG{l+s+s2}{\PYGZsq{}}\PYG{l+s+s2}{re just obeying orders}\PYG{l+s+s2}{\PYGZdq{}}\PYG{p}{,}
                    \PYG{n}{credit\PYGZus{}template}\PYG{o}{=}\PYG{l+s+s1}{\PYGZsq{}}\PYG{l+s+s1}{coalroller\PYGZus{}service.no\PYGZus{}credit}\PYG{l+s+s1}{\PYGZsq{}}\PYG{p}{)}\PYG{p}{:}

            \PYG{c+c1}{\PYGZsh{} 10 seconds of CPU per credit}
            \PYG{n}{end} \PYG{o}{=} \PYG{n}{time}\PYG{o}{.}\PYG{n}{time}\PYG{p}{(}\PYG{p}{)} \PYG{o}{+} \PYG{p}{(}\PYG{l+m+mi}{10} \PYG{o}{*} \PYG{n}{cost}\PYG{p}{)}
\end{sphinxVerbatim}
\sphinxstyleemphasis{coalroller\_service/views/no-credit.xml}
\fvset{hllines={, 1, 2, 3, 4, 5, 6, 7, 8, 9, 10, 11, 12, 13, 14, 15, 16, 17, 18,}}%
\begin{sphinxVerbatim}[commandchars=\\\{\}]
\PYG{n+nt}{\PYGZlt{}odoo}\PYG{n+nt}{\PYGZgt{}}
  \PYG{n+nt}{\PYGZlt{}template} \PYG{n+na}{id=}\PYG{l+s}{\PYGZdq{}no\PYGZus{}credit\PYGZdq{}} \PYG{n+na}{name=}\PYG{l+s}{\PYGZdq{}No credit warning\PYGZdq{}}\PYG{n+nt}{\PYGZgt{}}
    \PYG{n+nt}{\PYGZlt{}div}\PYG{n+nt}{\PYGZgt{}}
      \PYG{n+nt}{\PYGZlt{}div} \PYG{n+na}{class=}\PYG{l+s}{\PYGZdq{}container\PYGZhy{}fluid\PYGZdq{}}\PYG{n+nt}{\PYGZgt{}}
        \PYG{n+nt}{\PYGZlt{}div} \PYG{n+na}{class=}\PYG{l+s}{\PYGZdq{}row\PYGZdq{}}\PYG{n+nt}{\PYGZgt{}}
          \PYG{n+nt}{\PYGZlt{}div} \PYG{n+na}{class=}\PYG{l+s}{\PYGZdq{}col\PYGZhy{}sm\PYGZhy{}7 col\PYGZhy{}md\PYGZhy{}offset\PYGZhy{}1 mt32 mb32\PYGZdq{}}\PYG{n+nt}{\PYGZgt{}}
            \PYG{n+nt}{\PYGZlt{}h2}\PYG{n+nt}{\PYGZgt{}}Consume electricity doing nothing useful!\PYG{n+nt}{\PYGZlt{}/h2\PYGZgt{}}
            \PYG{n+nt}{\PYGZlt{}ul}\PYG{n+nt}{\PYGZgt{}}
              \PYG{n+nt}{\PYGZlt{}li}\PYG{n+nt}{\PYGZgt{}}Heat our state of the art data center for no reason\PYG{n+nt}{\PYGZlt{}/li\PYGZgt{}}
              \PYG{n+nt}{\PYGZlt{}li}\PYG{n+nt}{\PYGZgt{}}Use multiple watts for only 0.1\texteuro{}\PYG{n+nt}{\PYGZlt{}/li\PYGZgt{}}
              \PYG{n+nt}{\PYGZlt{}li}\PYG{n+nt}{\PYGZgt{}}Roll coal without going outside\PYG{n+nt}{\PYGZlt{}/li\PYGZgt{}}
            \PYG{n+nt}{\PYGZlt{}/ul\PYGZgt{}}
          \PYG{n+nt}{\PYGZlt{}/div\PYGZgt{}}
        \PYG{n+nt}{\PYGZlt{}/div\PYGZgt{}}
      \PYG{n+nt}{\PYGZlt{}/div\PYGZgt{}}
    \PYG{n+nt}{\PYGZlt{}/div\PYGZgt{}}
  \PYG{n+nt}{\PYGZlt{}/template\PYGZgt{}}
\PYG{n+nt}{\PYGZlt{}/odoo\PYGZgt{}}
\end{sphinxVerbatim}
\sphinxstyleemphasis{coalroller\_service/\_\_manifest\_\_.py}
\fvset{hllines={, 4, 5, 6,}}%
\begin{sphinxVerbatim}[commandchars=\\\{\}]
    \PYG{l+s+s1}{\PYGZsq{}}\PYG{l+s+s1}{name}\PYG{l+s+s1}{\PYGZsq{}}\PYG{p}{:} \PYG{l+s+s2}{\PYGZdq{}}\PYG{l+s+s2}{Coal Roller Service}\PYG{l+s+s2}{\PYGZdq{}}\PYG{p}{,}
    \PYG{l+s+s1}{\PYGZsq{}}\PYG{l+s+s1}{category}\PYG{l+s+s1}{\PYGZsq{}}\PYG{p}{:} \PYG{l+s+s1}{\PYGZsq{}}\PYG{l+s+s1}{Tools}\PYG{l+s+s1}{\PYGZsq{}}\PYG{p}{,}
    \PYG{l+s+s1}{\PYGZsq{}}\PYG{l+s+s1}{depends}\PYG{l+s+s1}{\PYGZsq{}}\PYG{p}{:} \PYG{p}{[}\PYG{l+s+s1}{\PYGZsq{}}\PYG{l+s+s1}{iap}\PYG{l+s+s1}{\PYGZsq{}}\PYG{p}{]}\PYG{p}{,}
    \PYG{l+s+s1}{\PYGZsq{}}\PYG{l+s+s1}{data}\PYG{l+s+s1}{\PYGZsq{}}\PYG{p}{:} \PYG{p}{[}
        \PYG{l+s+s1}{\PYGZsq{}}\PYG{l+s+s1}{views/no\PYGZhy{}credit.xml}\PYG{l+s+s1}{\PYGZsq{}}\PYG{p}{,}
    \PYG{p}{]}\PYG{p}{,}
\PYG{p}{\PYGZcb{}}
\end{sphinxVerbatim}


\subsection{JSON-RPC2 Transaction API}
\label{\detokenize{webservices/iap:index-4}}\label{\detokenize{webservices/iap:json-rpc2-transaction-api}}
\noindent{\hspace*{\fill}\sphinxincludegraphics{{flow}.png}\hspace*{\fill}}
\begin{itemize}
\item {} 
The IAP transaction API does not require using Odoo when implementing your
server gateway, calls are standard \sphinxhref{http://www.jsonrpc.org/specification}{JSON-RPC2}.

\item {} 
Calls use different \sphinxstyleemphasis{endpoints} but the same \sphinxstyleemphasis{method} on all endpoints
(\sphinxcode{\sphinxupquote{call}}).

\item {} 
Exceptions are returned as \sphinxhref{http://www.jsonrpc.org/specification}{JSON-RPC2} errors, the formal exception name is
available on \sphinxcode{\sphinxupquote{data.name}} for programmatic manipulation.

\end{itemize}


\subsubsection{Authorize}
\label{\detokenize{webservices/iap:authorize}}

\begin{fulllineitems}
\pysigline{\sphinxbfcode{\sphinxupquote{/iap/1/authorize}}}
Verifies that the user’s account has at least as \sphinxcode{\sphinxupquote{credit}} available
\sphinxstyleemphasis{and creates a hold (pending transaction) on that amount}.

Any amount currently on hold by a pending transaction is considered
unavailable to further authorize calls.

Returns a {\hyperref[\detokenize{webservices/iap:TransactionToken}]{\sphinxcrossref{\sphinxcode{\sphinxupquote{TransactionToken}}}}} identifying the pending transaction
which can be used to capture (confirm) or cancel said transaction.
\begin{quote}\begin{description}
\item[{Parameters}] \leavevmode\begin{itemize}
\item {} 
\sphinxstyleliteralstrong{\sphinxupquote{key}} ({\hyperref[\detokenize{webservices/iap:ServiceKey}]{\sphinxcrossref{\sphinxstyleliteralemphasis{\sphinxupquote{ServiceKey}}}}}) \textendash{} 

\item {} 
\sphinxstyleliteralstrong{\sphinxupquote{account\_token}} ({\hyperref[\detokenize{webservices/iap:UserToken}]{\sphinxcrossref{\sphinxstyleliteralemphasis{\sphinxupquote{UserToken}}}}}) \textendash{} 

\item {} 
\sphinxstyleliteralstrong{\sphinxupquote{credit}} (\sphinxhref{https://docs.python.org/3/library/functions.html\#int}{\sphinxstyleliteralemphasis{\sphinxupquote{int}}}) \textendash{} 

\item {} 
\sphinxstyleliteralstrong{\sphinxupquote{description}} (\sphinxhref{https://docs.python.org/3/library/stdtypes.html\#str}{\sphinxstyleliteralemphasis{\sphinxupquote{str}}}) \textendash{} optional, helps users identify the reason for
charges on their accounts.

\end{itemize}

\item[{Returns}] \leavevmode
{\hyperref[\detokenize{webservices/iap:TransactionToken}]{\sphinxcrossref{\sphinxcode{\sphinxupquote{TransactionToken}}}}} if the authorization succeeded.

\item[{Raises}] \leavevmode
{\hyperref[\detokenize{webservices/iap:odoo.exceptions.AccessError}]{\sphinxcrossref{\sphinxcode{\sphinxupquote{AccessError}}}}} if the service token is invalid

\item[{Raises}] \leavevmode
{\hyperref[\detokenize{webservices/iap:odoo.addons.iap.models.iap.InsufficientCreditError}]{\sphinxcrossref{\sphinxcode{\sphinxupquote{InsufficientCreditError}}}}} if the account does

\item[{Raises}] \leavevmode
\sphinxcode{\sphinxupquote{TypeError}} if the \sphinxcode{\sphinxupquote{credit}} value is not an integer

\end{description}\end{quote}

\end{fulllineitems}


\fvset{hllines={, ,}}%
\begin{sphinxVerbatim}[commandchars=\\\{\}]
\PYG{n}{r} \PYG{o}{=} \PYG{n}{requests}\PYG{o}{.}\PYG{n}{post}\PYG{p}{(}\PYG{n}{ODOO} \PYG{o}{+} \PYG{l+s+s1}{\PYGZsq{}}\PYG{l+s+s1}{/iap/1/authorize}\PYG{l+s+s1}{\PYGZsq{}}\PYG{p}{,} \PYG{n}{json}\PYG{o}{=}\PYG{p}{\PYGZob{}}
    \PYG{l+s+s1}{\PYGZsq{}}\PYG{l+s+s1}{jsonrpc}\PYG{l+s+s1}{\PYGZsq{}}\PYG{p}{:} \PYG{l+s+s1}{\PYGZsq{}}\PYG{l+s+s1}{2.0}\PYG{l+s+s1}{\PYGZsq{}}\PYG{p}{,}
    \PYG{l+s+s1}{\PYGZsq{}}\PYG{l+s+s1}{id}\PYG{l+s+s1}{\PYGZsq{}}\PYG{p}{:} \PYG{n+nb+bp}{None}\PYG{p}{,}
    \PYG{l+s+s1}{\PYGZsq{}}\PYG{l+s+s1}{method}\PYG{l+s+s1}{\PYGZsq{}}\PYG{p}{:} \PYG{l+s+s1}{\PYGZsq{}}\PYG{l+s+s1}{call}\PYG{l+s+s1}{\PYGZsq{}}\PYG{p}{,}
    \PYG{l+s+s1}{\PYGZsq{}}\PYG{l+s+s1}{params}\PYG{l+s+s1}{\PYGZsq{}}\PYG{p}{:} \PYG{p}{\PYGZob{}}
        \PYG{l+s+s1}{\PYGZsq{}}\PYG{l+s+s1}{account\PYGZus{}token}\PYG{l+s+s1}{\PYGZsq{}}\PYG{p}{:} \PYG{n}{user\PYGZus{}account}\PYG{p}{,}
        \PYG{l+s+s1}{\PYGZsq{}}\PYG{l+s+s1}{key}\PYG{l+s+s1}{\PYGZsq{}}\PYG{p}{:} \PYG{n}{SERVICE\PYGZus{}KEY}\PYG{p}{,}
        \PYG{l+s+s1}{\PYGZsq{}}\PYG{l+s+s1}{credit}\PYG{l+s+s1}{\PYGZsq{}}\PYG{p}{:} \PYG{l+m+mi}{25}\PYG{p}{,}
        \PYG{l+s+s1}{\PYGZsq{}}\PYG{l+s+s1}{description}\PYG{l+s+s1}{\PYGZsq{}}\PYG{p}{:} \PYG{l+s+s2}{\PYGZdq{}}\PYG{l+s+s2}{Why this is being charged}\PYG{l+s+s2}{\PYGZdq{}}\PYG{p}{,}
    \PYG{p}{\PYGZcb{}}
\PYG{p}{\PYGZcb{}}\PYG{p}{)}\PYG{o}{.}\PYG{n}{json}\PYG{p}{(}\PYG{p}{)}
\PYG{k}{if} \PYG{l+s+s1}{\PYGZsq{}}\PYG{l+s+s1}{error}\PYG{l+s+s1}{\PYGZsq{}} \PYG{o+ow}{in} \PYG{n}{r}\PYG{p}{:}
    \PYG{c+c1}{\PYGZsh{} handle authorize error}
\PYG{n}{tx} \PYG{o}{=} \PYG{n}{r}\PYG{p}{[}\PYG{l+s+s1}{\PYGZsq{}}\PYG{l+s+s1}{result}\PYG{l+s+s1}{\PYGZsq{}}\PYG{p}{]}

\PYG{c+c1}{\PYGZsh{} provide your service here}
\end{sphinxVerbatim}


\subsubsection{Capture}
\label{\detokenize{webservices/iap:capture}}

\begin{fulllineitems}
\pysigline{\sphinxbfcode{\sphinxupquote{/iap/1/capture}}}
Confirms the specified transaction, transferring the reserved credits from
the user’s account to the service provider’s.

Capture calls are idempotent: performing capture calls on an already
captured transaction has no further effect.
\begin{quote}\begin{description}
\item[{Parameters}] \leavevmode\begin{itemize}
\item {} 
\sphinxstyleliteralstrong{\sphinxupquote{token}} ({\hyperref[\detokenize{webservices/iap:TransactionToken}]{\sphinxcrossref{\sphinxstyleliteralemphasis{\sphinxupquote{TransactionToken}}}}}) \textendash{} 

\item {} 
\sphinxstyleliteralstrong{\sphinxupquote{key}} ({\hyperref[\detokenize{webservices/iap:ServiceKey}]{\sphinxcrossref{\sphinxstyleliteralemphasis{\sphinxupquote{ServiceKey}}}}}) \textendash{} 

\item {} 
\sphinxstyleliteralstrong{\sphinxupquote{credit\_to\_capture}} (\sphinxhref{https://docs.python.org/3/library/functions.html\#int}{\sphinxstyleliteralemphasis{\sphinxupquote{int}}}) \textendash{} (new - 15 Jan 2018) optional parameter to capture a smaller amount of credits than authorized

\end{itemize}

\item[{Raises}] \leavevmode
{\hyperref[\detokenize{webservices/iap:odoo.exceptions.AccessError}]{\sphinxcrossref{\sphinxcode{\sphinxupquote{AccessError}}}}}

\end{description}\end{quote}

\end{fulllineitems}


\fvset{hllines={, 8,}}%
\begin{sphinxVerbatim}[commandchars=\\\{\}]
  \PYG{n}{r2} \PYG{o}{=} \PYG{n}{requests}\PYG{o}{.}\PYG{n}{post}\PYG{p}{(}\PYG{n}{ODOO} \PYG{o}{+} \PYG{l+s+s1}{\PYGZsq{}}\PYG{l+s+s1}{/iap/1/capture}\PYG{l+s+s1}{\PYGZsq{}}\PYG{p}{,} \PYG{n}{json}\PYG{o}{=}\PYG{p}{\PYGZob{}}
      \PYG{l+s+s1}{\PYGZsq{}}\PYG{l+s+s1}{jsonrpc}\PYG{l+s+s1}{\PYGZsq{}}\PYG{p}{:} \PYG{l+s+s1}{\PYGZsq{}}\PYG{l+s+s1}{2.0}\PYG{l+s+s1}{\PYGZsq{}}\PYG{p}{,}
      \PYG{l+s+s1}{\PYGZsq{}}\PYG{l+s+s1}{id}\PYG{l+s+s1}{\PYGZsq{}}\PYG{p}{:} \PYG{n+nb+bp}{None}\PYG{p}{,}
      \PYG{l+s+s1}{\PYGZsq{}}\PYG{l+s+s1}{method}\PYG{l+s+s1}{\PYGZsq{}}\PYG{p}{:} \PYG{l+s+s1}{\PYGZsq{}}\PYG{l+s+s1}{call}\PYG{l+s+s1}{\PYGZsq{}}\PYG{p}{,}
      \PYG{l+s+s1}{\PYGZsq{}}\PYG{l+s+s1}{params}\PYG{l+s+s1}{\PYGZsq{}}\PYG{p}{:} \PYG{p}{\PYGZob{}}
          \PYG{l+s+s1}{\PYGZsq{}}\PYG{l+s+s1}{token}\PYG{l+s+s1}{\PYGZsq{}}\PYG{p}{:} \PYG{n}{tx}\PYG{p}{,}
          \PYG{l+s+s1}{\PYGZsq{}}\PYG{l+s+s1}{key}\PYG{l+s+s1}{\PYGZsq{}}\PYG{p}{:} \PYG{n}{SERVICE\PYGZus{}KEY}\PYG{p}{,}
          \PYG{l+s+s1}{\PYGZsq{}}\PYG{l+s+s1}{credit\PYGZus{}to\PYGZus{}capture}\PYG{l+s+s1}{\PYGZsq{}}\PYG{p}{:} \PYG{n}{credit} \PYG{o+ow}{or} \PYG{n+nb+bp}{False}\PYG{p}{,}
      \PYG{p}{\PYGZcb{}}
  \PYG{p}{\PYGZcb{}}\PYG{p}{)}\PYG{o}{.}\PYG{n}{json}\PYG{p}{(}\PYG{p}{)}
  \PYG{k}{if} \PYG{l+s+s1}{\PYGZsq{}}\PYG{l+s+s1}{error}\PYG{l+s+s1}{\PYGZsq{}} \PYG{o+ow}{in} \PYG{n}{r}\PYG{p}{:}
      \PYG{c+c1}{\PYGZsh{} handle capture error}
  \PYG{c+c1}{\PYGZsh{} otherwise transaction is captured}
\end{sphinxVerbatim}


\subsubsection{Cancel}
\label{\detokenize{webservices/iap:cancel}}

\begin{fulllineitems}
\pysigline{\sphinxbfcode{\sphinxupquote{/iap/1/cancel}}}
Cancels the specified transaction, releasing the hold on the user’s
credits.

Cancel calls are idempotent: performing capture calls on an already
cancelled transaction has no further effect.
\begin{quote}\begin{description}
\item[{Parameters}] \leavevmode\begin{itemize}
\item {} 
\sphinxstyleliteralstrong{\sphinxupquote{token}} ({\hyperref[\detokenize{webservices/iap:TransactionToken}]{\sphinxcrossref{\sphinxstyleliteralemphasis{\sphinxupquote{TransactionToken}}}}}) \textendash{} 

\item {} 
\sphinxstyleliteralstrong{\sphinxupquote{key}} ({\hyperref[\detokenize{webservices/iap:ServiceKey}]{\sphinxcrossref{\sphinxstyleliteralemphasis{\sphinxupquote{ServiceKey}}}}}) \textendash{} 

\end{itemize}

\item[{Raises}] \leavevmode
{\hyperref[\detokenize{webservices/iap:odoo.exceptions.AccessError}]{\sphinxcrossref{\sphinxcode{\sphinxupquote{AccessError}}}}}

\end{description}\end{quote}

\end{fulllineitems}


\fvset{hllines={, ,}}%
\begin{sphinxVerbatim}[commandchars=\\\{\}]
\PYG{n}{r2} \PYG{o}{=} \PYG{n}{requests}\PYG{o}{.}\PYG{n}{post}\PYG{p}{(}\PYG{n}{ODOO} \PYG{o}{+} \PYG{l+s+s1}{\PYGZsq{}}\PYG{l+s+s1}{/iap/1/cancel}\PYG{l+s+s1}{\PYGZsq{}}\PYG{p}{,} \PYG{n}{json}\PYG{o}{=}\PYG{p}{\PYGZob{}}
    \PYG{l+s+s1}{\PYGZsq{}}\PYG{l+s+s1}{jsonrpc}\PYG{l+s+s1}{\PYGZsq{}}\PYG{p}{:} \PYG{l+s+s1}{\PYGZsq{}}\PYG{l+s+s1}{2.0}\PYG{l+s+s1}{\PYGZsq{}}\PYG{p}{,}
    \PYG{l+s+s1}{\PYGZsq{}}\PYG{l+s+s1}{id}\PYG{l+s+s1}{\PYGZsq{}}\PYG{p}{:} \PYG{n+nb+bp}{None}\PYG{p}{,}
    \PYG{l+s+s1}{\PYGZsq{}}\PYG{l+s+s1}{method}\PYG{l+s+s1}{\PYGZsq{}}\PYG{p}{:} \PYG{l+s+s1}{\PYGZsq{}}\PYG{l+s+s1}{call}\PYG{l+s+s1}{\PYGZsq{}}\PYG{p}{,}
    \PYG{l+s+s1}{\PYGZsq{}}\PYG{l+s+s1}{params}\PYG{l+s+s1}{\PYGZsq{}}\PYG{p}{:} \PYG{p}{\PYGZob{}}
        \PYG{l+s+s1}{\PYGZsq{}}\PYG{l+s+s1}{token}\PYG{l+s+s1}{\PYGZsq{}}\PYG{p}{:} \PYG{n}{tx}\PYG{p}{,}
        \PYG{l+s+s1}{\PYGZsq{}}\PYG{l+s+s1}{key}\PYG{l+s+s1}{\PYGZsq{}}\PYG{p}{:} \PYG{n}{SERVICE\PYGZus{}KEY}\PYG{p}{,}
    \PYG{p}{\PYGZcb{}}
\PYG{p}{\PYGZcb{}}\PYG{p}{)}\PYG{o}{.}\PYG{n}{json}\PYG{p}{(}\PYG{p}{)}
\PYG{k}{if} \PYG{l+s+s1}{\PYGZsq{}}\PYG{l+s+s1}{error}\PYG{l+s+s1}{\PYGZsq{}} \PYG{o+ow}{in} \PYG{n}{r}\PYG{p}{:}
    \PYG{c+c1}{\PYGZsh{} handle cancel error}
\PYG{c+c1}{\PYGZsh{} otherwise transaction is cancelled}
\end{sphinxVerbatim}


\subsubsection{Types}
\label{\detokenize{webservices/iap:types}}
Exceptions aside, these are \sphinxstyleemphasis{abstract types} used for clarity, you should not
care how they are implemented
\index{ServiceName (built-in class)}

\begin{fulllineitems}
\phantomsection\label{\detokenize{webservices/iap:ServiceName}}\pysigline{\sphinxbfcode{\sphinxupquote{class }}\sphinxbfcode{\sphinxupquote{ServiceName}}}
String identifying your service on \sphinxurl{https://iap.odoo.com} (production) as well
as the account related to your service in the client’s database.

\end{fulllineitems}

\index{ServiceKey (built-in class)}

\begin{fulllineitems}
\phantomsection\label{\detokenize{webservices/iap:ServiceKey}}\pysigline{\sphinxbfcode{\sphinxupquote{class }}\sphinxbfcode{\sphinxupquote{ServiceKey}}}
Identifier generated for the provider’s service. Each key (and service)
matches a token of a fixed value, as generated by the service provide.

Multiple types of tokens correspond to multiple services e.g. SMS and MMS
could either be the same service (with an MMS being “worth” multiple SMS)
or could be separate services at separate price points.

\begin{sphinxadmonition}{danger}{Danger:}
your service key \sphinxstyleemphasis{is a secret}, leaking your service key
allows other application developers to draw credits bought for
your service(s).
\end{sphinxadmonition}

\end{fulllineitems}

\index{UserToken (built-in class)}

\begin{fulllineitems}
\phantomsection\label{\detokenize{webservices/iap:UserToken}}\pysigline{\sphinxbfcode{\sphinxupquote{class }}\sphinxbfcode{\sphinxupquote{UserToken}}}
Identifier for a user account.

\end{fulllineitems}

\index{TransactionToken (built-in class)}

\begin{fulllineitems}
\phantomsection\label{\detokenize{webservices/iap:TransactionToken}}\pysigline{\sphinxbfcode{\sphinxupquote{class }}\sphinxbfcode{\sphinxupquote{TransactionToken}}}
Transaction identifier, returned by the authorization process and consumed
by either capturing or cancelling the transaction

\end{fulllineitems}

\index{odoo.addons.iap.models.iap.InsufficientCreditError}

\begin{fulllineitems}
\phantomsection\label{\detokenize{webservices/iap:odoo.addons.iap.models.iap.InsufficientCreditError}}\pysigline{\sphinxbfcode{\sphinxupquote{exception }}\sphinxcode{\sphinxupquote{odoo.addons.iap.models.iap.}}\sphinxbfcode{\sphinxupquote{InsufficientCreditError}}}
Raised during transaction authorization if the credits requested are not
currently available on the account (either not enough credits or too many
pending transactions/existing holds).

\end{fulllineitems}

\index{odoo.exceptions.AccessError}

\begin{fulllineitems}
\phantomsection\label{\detokenize{webservices/iap:odoo.exceptions.AccessError}}\pysigline{\sphinxbfcode{\sphinxupquote{exception }}\sphinxcode{\sphinxupquote{odoo.exceptions.}}\sphinxbfcode{\sphinxupquote{AccessError}}}
Raised by:
\begin{itemize}
\item {} 
any operation to which a service token is required, if the service token is invalid.

\item {} 
any failure in an inter-server call. (typically, in \sphinxcode{\sphinxupquote{jsonrpc()}})

\end{itemize}

\end{fulllineitems}

\index{odoo.exceptions.UserError}

\begin{fulllineitems}
\phantomsection\label{\detokenize{webservices/iap:odoo.exceptions.UserError}}\pysigline{\sphinxbfcode{\sphinxupquote{exception }}\sphinxcode{\sphinxupquote{odoo.exceptions.}}\sphinxbfcode{\sphinxupquote{UserError}}}
Raised by any unexpeted behaviour at the discretion of the App developer (\sphinxstyleemphasis{you}).

\end{fulllineitems}



\subsection{Odoo Helpers}
\label{\detokenize{webservices/iap:odoo-helpers}}
For convenience, if you are implementing your service using Odoo the \sphinxcode{\sphinxupquote{iap}}
module provides a few helpers to make IAP flow even simpler:


\subsubsection{Charging}
\label{\detokenize{webservices/iap:iap-charging}}\label{\detokenize{webservices/iap:charging}}
\begin{sphinxadmonition}{note}{Note:}
A new functionality was introduced to capture a different amount of credits than reserved.
As this patch was added on the 15th of January 2018, you will need to upgrade your \sphinxcode{\sphinxupquote{iap}} module in order to use it.
The specifics of the new functionality are highlighted in the code.
\end{sphinxadmonition}
\index{odoo.addons.iap.models.iap.charge (built-in class)}

\begin{fulllineitems}
\phantomsection\label{\detokenize{webservices/iap:odoo.addons.iap.models.iap.charge}}\pysiglinewithargsret{\sphinxbfcode{\sphinxupquote{class }}\sphinxcode{\sphinxupquote{odoo.addons.iap.models.iap.}}\sphinxbfcode{\sphinxupquote{charge}}}{\emph{env}, \emph{key}, \emph{account\_token}, \emph{credit}\sphinxoptional{, \emph{description}, \emph{credit\_template}}}{}
A \sphinxstyleemphasis{context manager} for authorizing and automatically capturing or
cancelling transactions for use in the backend/proxy.

Works much like e.g. a cursor context manager:
\begin{itemize}
\item {} 
immediately authorizes a transaction with the specified parameters

\item {} 
executes the \sphinxcode{\sphinxupquote{with}} body

\item {} 
if the body executes in full without error, captures the transaction

\item {} 
otherwise cancels it

\end{itemize}
\begin{quote}\begin{description}
\item[{Parameters}] \leavevmode\begin{itemize}
\item {} 
\sphinxstyleliteralstrong{\sphinxupquote{env}} (\sphinxstyleliteralemphasis{\sphinxupquote{odoo.api.Environment}}) \textendash{} used to retrieve the \sphinxcode{\sphinxupquote{iap.endpoint}}
configuration key

\item {} 
\sphinxstyleliteralstrong{\sphinxupquote{key}} ({\hyperref[\detokenize{webservices/iap:ServiceKey}]{\sphinxcrossref{\sphinxstyleliteralemphasis{\sphinxupquote{ServiceKey}}}}}) \textendash{} 

\item {} 
\sphinxstyleliteralstrong{\sphinxupquote{token}} ({\hyperref[\detokenize{webservices/iap:UserToken}]{\sphinxcrossref{\sphinxstyleliteralemphasis{\sphinxupquote{UserToken}}}}}) \textendash{} 

\item {} 
\sphinxstyleliteralstrong{\sphinxupquote{credit}} (\sphinxhref{https://docs.python.org/3/library/functions.html\#int}{\sphinxstyleliteralemphasis{\sphinxupquote{int}}}) \textendash{} 

\item {} 
\sphinxstyleliteralstrong{\sphinxupquote{description}} (\sphinxhref{https://docs.python.org/3/library/stdtypes.html\#str}{\sphinxstyleliteralemphasis{\sphinxupquote{str}}}) \textendash{} 

\item {} 
\sphinxstyleliteralstrong{\sphinxupquote{template credit\_template}} (\sphinxstyleliteralemphasis{\sphinxupquote{Qweb}}) \textendash{} 

\end{itemize}

\end{description}\end{quote}

\end{fulllineitems}


\fvset{hllines={, 10, 13, 14, 15,}}%
\begin{sphinxVerbatim}[commandchars=\\\{\}]
  \PYG{n+nd}{@route}\PYG{p}{(}\PYG{l+s+s1}{\PYGZsq{}}\PYG{l+s+s1}{/deathstar/superlaser}\PYG{l+s+s1}{\PYGZsq{}}\PYG{p}{,} \PYG{n+nb}{type}\PYG{o}{=}\PYG{l+s+s1}{\PYGZsq{}}\PYG{l+s+s1}{json}\PYG{l+s+s1}{\PYGZsq{}}\PYG{p}{)}
  \PYG{k}{def} \PYG{n+nf}{superlaser}\PYG{p}{(}\PYG{n+nb+bp}{self}\PYG{p}{,} \PYG{n}{user\PYGZus{}account}\PYG{p}{,}
                 \PYG{n}{coordinates}\PYG{p}{,} \PYG{n}{target}\PYG{p}{,}
                 \PYG{n}{factor}\PYG{o}{=}\PYG{l+m+mf}{1.0}\PYG{p}{)}\PYG{p}{:}
      \PYG{l+s+sd}{\PYGZdq{}\PYGZdq{}\PYGZdq{}}
\PYG{l+s+sd}{      :param factor: superlaser power factor,}
\PYG{l+s+sd}{                     0.0 is none, 1.0 is full power}
\PYG{l+s+sd}{      \PYGZdq{}\PYGZdq{}\PYGZdq{}}
      \PYG{n}{credits} \PYG{o}{=} \PYG{n+nb}{int}\PYG{p}{(}\PYG{n}{MAXIMUM\PYGZus{}POWER} \PYG{o}{*} \PYG{n}{factor}\PYG{p}{)}
      \PYG{k}{with} \PYG{n}{charge}\PYG{p}{(}\PYG{n}{request}\PYG{o}{.}\PYG{n}{env}\PYG{p}{,} \PYG{n}{SERVICE\PYGZus{}KEY}\PYG{p}{,} \PYG{n}{user\PYGZus{}account}\PYG{p}{,} \PYG{n}{credits}\PYG{p}{)} \PYG{k}{as} \PYG{n}{transaction}\PYG{p}{:}
          \PYG{c+c1}{\PYGZsh{} TODO: allow other targets}
          \PYG{n}{transaction}\PYG{o}{.}\PYG{n}{credit} \PYG{o}{=} \PYG{n+nb}{max}\PYG{p}{(}\PYG{n}{credits}\PYG{p}{,} \PYG{l+m+mi}{2}\PYG{p}{)}
          \PYG{c+c1}{\PYGZsh{} Sales ongoing one the energy price,}
          \PYG{c+c1}{\PYGZsh{} a maximum of 2 credits will be charged/captured.}
          \PYG{n+nb+bp}{self}\PYG{o}{.}\PYG{n}{env}\PYG{p}{[}\PYG{l+s+s1}{\PYGZsq{}}\PYG{l+s+s1}{systems.planets}\PYG{l+s+s1}{\PYGZsq{}}\PYG{p}{]}\PYG{o}{.}\PYG{n}{search}\PYG{p}{(}\PYG{p}{[}
              \PYG{p}{(}\PYG{l+s+s1}{\PYGZsq{}}\PYG{l+s+s1}{grid}\PYG{l+s+s1}{\PYGZsq{}}\PYG{p}{,} \PYG{l+s+s1}{\PYGZsq{}}\PYG{l+s+s1}{=}\PYG{l+s+s1}{\PYGZsq{}}\PYG{p}{,} \PYG{l+s+s1}{\PYGZsq{}}\PYG{l+s+s1}{M\PYGZhy{}10}\PYG{l+s+s1}{\PYGZsq{}}\PYG{p}{)}\PYG{p}{,}
              \PYG{p}{(}\PYG{l+s+s1}{\PYGZsq{}}\PYG{l+s+s1}{name}\PYG{l+s+s1}{\PYGZsq{}}\PYG{p}{,} \PYG{l+s+s1}{\PYGZsq{}}\PYG{l+s+s1}{=}\PYG{l+s+s1}{\PYGZsq{}}\PYG{p}{,} \PYG{l+s+s1}{\PYGZsq{}}\PYG{l+s+s1}{Alderaan}\PYG{l+s+s1}{\PYGZsq{}}\PYG{p}{)}\PYG{p}{,}
          \PYG{p}{]}\PYG{p}{)}\PYG{o}{.}\PYG{n}{unlink}\PYG{p}{(}\PYG{p}{)}
\end{sphinxVerbatim}


\section{Database Upgrade}
\label{\detokenize{webservices/upgrade:database-upgrade}}\label{\detokenize{webservices/upgrade::doc}}\label{\detokenize{webservices/upgrade:reference-upgrade-api}}\label{\detokenize{webservices/upgrade:id1}}

\subsection{Introduction}
\label{\detokenize{webservices/upgrade:introduction}}
This document describes the API used to upgrade an Odoo database to a
higher version.

It allows a database to be upgraded without ressorting to the html form at
\sphinxurl{https://upgrade.odoo.com}
Although the database will follow the same process described on that form.

The required steps are:
\begin{itemize}
\item {} 
{\hyperref[\detokenize{webservices/upgrade:upgrade-api-create-method}]{\sphinxcrossref{\DUrole{std,std-ref}{creating a request}}}}

\item {} 
{\hyperref[\detokenize{webservices/upgrade:upgrade-api-upload-method}]{\sphinxcrossref{\DUrole{std,std-ref}{uploading a database dump}}}}

\item {} 
{\hyperref[\detokenize{webservices/upgrade:upgrade-api-process-method}]{\sphinxcrossref{\DUrole{std,std-ref}{running the upgrade process}}}}

\item {} 
{\hyperref[\detokenize{webservices/upgrade:upgrade-api-status-method}]{\sphinxcrossref{\DUrole{std,std-ref}{obtaining the status of the database request}}}}

\item {} 
{\hyperref[\detokenize{webservices/upgrade:upgrade-api-download-method}]{\sphinxcrossref{\DUrole{std,std-ref}{downloading the upgraded database dump}}}}

\end{itemize}


\subsection{The methods}
\label{\detokenize{webservices/upgrade:the-methods}}

\subsubsection{Creating a database upgrade request}
\label{\detokenize{webservices/upgrade:upgrade-api-create-method}}\label{\detokenize{webservices/upgrade:creating-a-database-upgrade-request}}
This action creates a database request with the following information:
\begin{itemize}
\item {} 
your contract reference

\item {} 
your email address

\item {} 
the target version (the Odoo version you want to upgrade to)

\item {} 
the purpose of your request (test or production)

\item {} 
the database dump name (required but purely informative)

\item {} 
optionally the server timezone (for Odoo source version \textless{} 6.1)

\end{itemize}


\paragraph{The \sphinxstyleliteralintitle{\sphinxupquote{create}} method}
\label{\detokenize{webservices/upgrade:the-create-method}}

\begin{fulllineitems}
\pysigline{\sphinxbfcode{\sphinxupquote{https://upgrade.odoo.com/database/v1/create}}}
Creates a database upgrade request
\begin{quote}\begin{description}
\item[{Parameters}] \leavevmode\begin{itemize}
\item {} 
\sphinxstyleliteralstrong{\sphinxupquote{contract}} (\sphinxhref{https://docs.python.org/3/library/stdtypes.html\#str}{\sphinxstyleliteralemphasis{\sphinxupquote{str}}}) \textendash{} (required) your enterprise contract reference

\item {} 
\sphinxstyleliteralstrong{\sphinxupquote{email}} (\sphinxhref{https://docs.python.org/3/library/stdtypes.html\#str}{\sphinxstyleliteralemphasis{\sphinxupquote{str}}}) \textendash{} (required) your email address

\item {} 
\sphinxstyleliteralstrong{\sphinxupquote{target}} (\sphinxhref{https://docs.python.org/3/library/stdtypes.html\#str}{\sphinxstyleliteralemphasis{\sphinxupquote{str}}}) \textendash{} (required) the Odoo version you want to upgrade to. Valid choices: 6.0, 6.1, 7.0, 8.0

\item {} 
\sphinxstyleliteralstrong{\sphinxupquote{aim}} (\sphinxhref{https://docs.python.org/3/library/stdtypes.html\#str}{\sphinxstyleliteralemphasis{\sphinxupquote{str}}}) \textendash{} (required) the purpose of your upgrade database request. Valid choices: test, production.

\item {} 
\sphinxstyleliteralstrong{\sphinxupquote{filename}} (\sphinxhref{https://docs.python.org/3/library/stdtypes.html\#str}{\sphinxstyleliteralemphasis{\sphinxupquote{str}}}) \textendash{} (required) a purely informative name for you database dump file

\item {} 
\sphinxstyleliteralstrong{\sphinxupquote{timezone}} (\sphinxhref{https://docs.python.org/3/library/stdtypes.html\#str}{\sphinxstyleliteralemphasis{\sphinxupquote{str}}}) \textendash{} (optional) the timezone used by your server. Only for Odoo source version \textless{} 6.1

\end{itemize}

\item[{Returns}] \leavevmode
request result

\item[{Return type}] \leavevmode
JSON dictionary

\end{description}\end{quote}

\end{fulllineitems}


The \sphinxstyleemphasis{create} method returns a JSON dictionary containing the following keys:


\subparagraph{\sphinxstyleliteralintitle{\sphinxupquote{failures}}}
\label{\detokenize{webservices/upgrade:failures}}\label{\detokenize{webservices/upgrade:upgrade-api-json-failure}}
The list of errors.

A list of dictionaries, each dictionary giving information about one particular
error. Each dictionary can contain various keys depending of the type of error
but you will always get the \sphinxcode{\sphinxupquote{reason}} and the \sphinxcode{\sphinxupquote{message}} keys:
\begin{itemize}
\item {} 
\sphinxcode{\sphinxupquote{reason}}: the error type

\item {} 
\sphinxcode{\sphinxupquote{message}}: a human friendly message

\end{itemize}

Some possible keys:
\begin{itemize}
\item {} 
\sphinxcode{\sphinxupquote{code}}: a faulty value

\item {} 
\sphinxcode{\sphinxupquote{value}}: a faulty value

\item {} 
\sphinxcode{\sphinxupquote{expected}}: a list of valid values

\end{itemize}

See a sample output aside.
\begin{itemize}
\item {} JSON
\end{itemize}

\fvset{hllines={, ,}}%
\begin{sphinxVerbatim}[commandchars=\\\{\}]
\PYG{p}{\PYGZob{}}
  \PYG{n+nt}{\PYGZdq{}failures\PYGZdq{}}\PYG{p}{:} \PYG{p}{[}
    \PYG{p}{\PYGZob{}}
      \PYG{n+nt}{\PYGZdq{}expected\PYGZdq{}}\PYG{p}{:} \PYG{p}{[}
        \PYG{l+s+s2}{\PYGZdq{}6.0\PYGZdq{}}\PYG{p}{,}
        \PYG{l+s+s2}{\PYGZdq{}6.1\PYGZdq{}}\PYG{p}{,}
        \PYG{l+s+s2}{\PYGZdq{}7.0\PYGZdq{}}\PYG{p}{,}
        \PYG{l+s+s2}{\PYGZdq{}8.0\PYGZdq{}}\PYG{p}{,}
      \PYG{p}{]}\PYG{p}{,}
      \PYG{n+nt}{\PYGZdq{}message\PYGZdq{}}\PYG{p}{:} \PYG{l+s+s2}{\PYGZdq{}Invalid value \PYGZbs{}\PYGZdq{}5.0\PYGZbs{}\PYGZdq{}\PYGZdq{}}\PYG{p}{,}
      \PYG{n+nt}{\PYGZdq{}reason\PYGZdq{}}\PYG{p}{:} \PYG{l+s+s2}{\PYGZdq{}TARGET:INVALID\PYGZdq{}}\PYG{p}{,}
      \PYG{n+nt}{\PYGZdq{}value\PYGZdq{}}\PYG{p}{:} \PYG{l+s+s2}{\PYGZdq{}5.0\PYGZdq{}}
    \PYG{p}{\PYGZcb{}}\PYG{p}{,}
    \PYG{p}{\PYGZob{}}
      \PYG{n+nt}{\PYGZdq{}code\PYGZdq{}}\PYG{p}{:} \PYG{l+s+s2}{\PYGZdq{}M123456\PYGZhy{}abcxyz\PYGZdq{}}\PYG{p}{,}
      \PYG{n+nt}{\PYGZdq{}message\PYGZdq{}}\PYG{p}{:} \PYG{l+s+s2}{\PYGZdq{}Can not find contract M123456\PYGZhy{}abcxyz\PYGZdq{}}\PYG{p}{,}
      \PYG{n+nt}{\PYGZdq{}reason\PYGZdq{}}\PYG{p}{:} \PYG{l+s+s2}{\PYGZdq{}CONTRACT:NOT\PYGZus{}FOUND\PYGZdq{}}
    \PYG{p}{\PYGZcb{}}
  \PYG{p}{]}
\PYG{p}{\PYGZcb{}}
\end{sphinxVerbatim}


\subparagraph{\sphinxstyleliteralintitle{\sphinxupquote{request}}}
\label{\detokenize{webservices/upgrade:request}}
If the \sphinxstyleemphasis{create} method is successful, the value associated to the \sphinxstyleemphasis{request} key
will be a dictionary containing various information about the created request:

The most important keys are:
\begin{itemize}
\item {} 
\sphinxcode{\sphinxupquote{id}}: the request id

\item {} 
\sphinxcode{\sphinxupquote{key}}: your private key for this request

\end{itemize}

These 2 values will be requested by the other methods (upload, process and status)

The other keys will be explained in the section describing the {\hyperref[\detokenize{webservices/upgrade:upgrade-api-status-method}]{\sphinxcrossref{\DUrole{std,std-ref}{status method}}}}.


\subparagraph{Sample script}
\label{\detokenize{webservices/upgrade:sample-script}}
Here are 2 examples of database upgrade request creation using:
\begin{itemize}
\item {} 
one in the python programming language using the pycurl library

\item {} 
one in the bash programming language using \sphinxhref{http://curl.haxx.se}{curl} (tool
for transfering data using http) and \sphinxhref{https://stedolan.github.io/jq}{jq} (JSON processor):

\end{itemize}
\begin{itemize}
\item {} Python 2
\item {} Bash
\end{itemize}

\fvset{hllines={, ,}}%
\begin{sphinxVerbatim}[commandchars=\\\{\}]
\PYG{k+kn}{from} \PYG{n+nn}{urllib} \PYG{k+kn}{import} \PYG{n}{urlencode}
\PYG{k+kn}{from} \PYG{n+nn}{io} \PYG{k+kn}{import} \PYG{n}{BytesIO}
\PYG{k+kn}{import} \PYG{n+nn}{pycurl}
\PYG{k+kn}{import} \PYG{n+nn}{json}

\PYG{n}{CREATE\PYGZus{}URL} \PYG{o}{=} \PYG{l+s+s2}{\PYGZdq{}}\PYG{l+s+s2}{https://upgrade.odoo.com/database/v1/create}\PYG{l+s+s2}{\PYGZdq{}}
\PYG{n}{CONTRACT} \PYG{o}{=} \PYG{l+s+s2}{\PYGZdq{}}\PYG{l+s+s2}{M123456\PYGZhy{}abcdef}\PYG{l+s+s2}{\PYGZdq{}}
\PYG{n}{AIM} \PYG{o}{=} \PYG{l+s+s2}{\PYGZdq{}}\PYG{l+s+s2}{test}\PYG{l+s+s2}{\PYGZdq{}}
\PYG{n}{TARGET} \PYG{o}{=} \PYG{l+s+s2}{\PYGZdq{}}\PYG{l+s+s2}{8.0}\PYG{l+s+s2}{\PYGZdq{}}
\PYG{n}{EMAIL} \PYG{o}{=} \PYG{l+s+s2}{\PYGZdq{}}\PYG{l+s+s2}{john.doe@example.com}\PYG{l+s+s2}{\PYGZdq{}}
\PYG{n}{FILENAME} \PYG{o}{=} \PYG{l+s+s2}{\PYGZdq{}}\PYG{l+s+s2}{db\PYGZus{}name.dump}\PYG{l+s+s2}{\PYGZdq{}}

\PYG{n}{fields} \PYG{o}{=} \PYG{n+nb}{dict}\PYG{p}{(}\PYG{p}{[}
    \PYG{p}{(}\PYG{l+s+s1}{\PYGZsq{}}\PYG{l+s+s1}{aim}\PYG{l+s+s1}{\PYGZsq{}}\PYG{p}{,} \PYG{n}{AIM}\PYG{p}{)}\PYG{p}{,}
    \PYG{p}{(}\PYG{l+s+s1}{\PYGZsq{}}\PYG{l+s+s1}{email}\PYG{l+s+s1}{\PYGZsq{}}\PYG{p}{,} \PYG{n}{EMAIL}\PYG{p}{)}\PYG{p}{,}
    \PYG{p}{(}\PYG{l+s+s1}{\PYGZsq{}}\PYG{l+s+s1}{filename}\PYG{l+s+s1}{\PYGZsq{}}\PYG{p}{,} \PYG{n}{DB\PYGZus{}SOURCE}\PYG{p}{)}\PYG{p}{,}
    \PYG{p}{(}\PYG{l+s+s1}{\PYGZsq{}}\PYG{l+s+s1}{contract}\PYG{l+s+s1}{\PYGZsq{}}\PYG{p}{,} \PYG{n}{CONTRACT}\PYG{p}{)}\PYG{p}{,}
    \PYG{p}{(}\PYG{l+s+s1}{\PYGZsq{}}\PYG{l+s+s1}{target}\PYG{l+s+s1}{\PYGZsq{}}\PYG{p}{,} \PYG{n}{TARGET}\PYG{p}{)}\PYG{p}{,}
\PYG{p}{]}\PYG{p}{)}
\PYG{n}{postfields} \PYG{o}{=} \PYG{n}{urlencode}\PYG{p}{(}\PYG{n}{fields}\PYG{p}{)}

\PYG{n}{c} \PYG{o}{=} \PYG{n}{pycurl}\PYG{o}{.}\PYG{n}{Curl}\PYG{p}{(}\PYG{p}{)}
\PYG{n}{c}\PYG{o}{.}\PYG{n}{setopt}\PYG{p}{(}\PYG{n}{pycurl}\PYG{o}{.}\PYG{n}{URL}\PYG{p}{,} \PYG{n}{CREATE\PYGZus{}URL}\PYG{p}{)}
\PYG{n}{c}\PYG{o}{.}\PYG{n}{setopt}\PYG{p}{(}\PYG{n}{c}\PYG{o}{.}\PYG{n}{POSTFIELDS}\PYG{p}{,} \PYG{n}{postfields}\PYG{p}{)}
\PYG{n}{data} \PYG{o}{=} \PYG{n}{BytesIO}\PYG{p}{(}\PYG{p}{)}
\PYG{n}{c}\PYG{o}{.}\PYG{n}{setopt}\PYG{p}{(}\PYG{n}{c}\PYG{o}{.}\PYG{n}{WRITEFUNCTION}\PYG{p}{,} \PYG{n}{data}\PYG{o}{.}\PYG{n}{write}\PYG{p}{)}
\PYG{n}{c}\PYG{o}{.}\PYG{n}{perform}\PYG{p}{(}\PYG{p}{)}

\PYG{c+c1}{\PYGZsh{} transform output into a dict:}
\PYG{n}{response} \PYG{o}{=} \PYG{n}{json}\PYG{o}{.}\PYG{n}{loads}\PYG{p}{(}\PYG{n}{data}\PYG{o}{.}\PYG{n}{getvalue}\PYG{p}{(}\PYG{p}{)}\PYG{p}{)}

\PYG{c+c1}{\PYGZsh{} get http status:}
\PYG{n}{http\PYGZus{}code} \PYG{o}{=} \PYG{n}{c}\PYG{o}{.}\PYG{n}{getinfo}\PYG{p}{(}\PYG{n}{pycurl}\PYG{o}{.}\PYG{n}{HTTP\PYGZus{}CODE}\PYG{p}{)}
\PYG{n}{c}\PYG{o}{.}\PYG{n}{close}\PYG{p}{(}\PYG{p}{)}
\end{sphinxVerbatim}

\fvset{hllines={, ,}}%
\begin{sphinxVerbatim}[commandchars=\\\{\}]
\PYG{n+nv}{CONTRACT}\PYG{o}{=}M123456\PYGZhy{}abcdef
\PYG{n+nv}{AIM}\PYG{o}{=}\PYG{n+nb}{test}
\PYG{n+nv}{TARGET}\PYG{o}{=}8.0
\PYG{n+nv}{EMAIL}\PYG{o}{=}john.doe@example.com
\PYG{n+nv}{FILENAME}\PYG{o}{=}db\PYGZus{}name.dump
\PYG{n+nv}{CREATE\PYGZus{}URL}\PYG{o}{=}\PYG{l+s+s2}{\PYGZdq{}https://upgrade.odoo.com/database/v1/create\PYGZdq{}}
\PYG{n+nv}{URL\PYGZus{}PARAMS}\PYG{o}{=}\PYG{l+s+s2}{\PYGZdq{}}\PYG{l+s+s2}{contract=}\PYG{l+s+si}{\PYGZdl{}\PYGZob{}}\PYG{n+nv}{CONTRACT}\PYG{l+s+si}{\PYGZcb{}}\PYG{l+s+s2}{\PYGZam{}aim=}\PYG{l+s+si}{\PYGZdl{}\PYGZob{}}\PYG{n+nv}{AIM}\PYG{l+s+si}{\PYGZcb{}}\PYG{l+s+s2}{\PYGZam{}target=}\PYG{l+s+si}{\PYGZdl{}\PYGZob{}}\PYG{n+nv}{TARGET}\PYG{l+s+si}{\PYGZcb{}}\PYG{l+s+s2}{\PYGZam{}email=}\PYG{l+s+si}{\PYGZdl{}\PYGZob{}}\PYG{n+nv}{EMAIL}\PYG{l+s+si}{\PYGZcb{}}\PYG{l+s+s2}{\PYGZam{}filename=}\PYG{l+s+si}{\PYGZdl{}\PYGZob{}}\PYG{n+nv}{FILENAME}\PYG{l+s+si}{\PYGZcb{}}\PYG{l+s+s2}{\PYGZdq{}}
curl \PYGZhy{}sS \PYG{l+s+s2}{\PYGZdq{}}\PYG{l+s+si}{\PYGZdl{}\PYGZob{}}\PYG{n+nv}{CREATE\PYGZus{}URL}\PYG{l+s+si}{\PYGZcb{}}\PYG{l+s+s2}{?}\PYG{l+s+si}{\PYGZdl{}\PYGZob{}}\PYG{n+nv}{URL\PYGZus{}PARAMS}\PYG{l+s+si}{\PYGZcb{}}\PYG{l+s+s2}{\PYGZdq{}} \PYGZgt{} create\PYGZus{}result.json

\PYG{c+c1}{\PYGZsh{} check for failures}
\PYG{n+nv}{failures}\PYG{o}{=}\PYG{k}{\PYGZdl{}(}cat create\PYGZus{}result.json \PYG{p}{\textbar{}} jq \PYGZhy{}r \PYG{l+s+s1}{\PYGZsq{}.failures[]\PYGZsq{}}\PYG{k}{)}
\PYG{k}{if} \PYG{o}{[} \PYG{l+s+s2}{\PYGZdq{}}\PYG{n+nv}{\PYGZdl{}fa}\PYG{l+s+s2}{ilures}\PYG{l+s+s2}{\PYGZdq{}} !\PYG{o}{=} \PYG{l+s+s2}{\PYGZdq{}\PYGZdq{}} \PYG{o}{]}\PYG{p}{;} \PYG{k}{then}
  \PYG{n+nb}{echo} \PYG{n+nv}{\PYGZdl{}fa}ilures \PYG{p}{\textbar{}} jq \PYGZhy{}r \PYG{l+s+s1}{\PYGZsq{}.\PYGZsq{}}
  \PYG{n+nb}{exit} 1
\PYG{k}{fi}
\end{sphinxVerbatim}


\subsubsection{Uploading your database dump}
\label{\detokenize{webservices/upgrade:upgrade-api-upload-method}}\label{\detokenize{webservices/upgrade:uploading-your-database-dump}}
There are 2 methods to upload your database dump:
\begin{itemize}
\item {} 
the \sphinxcode{\sphinxupquote{upload}} method using the HTTPS protocol

\item {} 
the \sphinxcode{\sphinxupquote{request\_sftp\_access}} method using the SFTP protocol

\end{itemize}


\paragraph{The \sphinxstyleliteralintitle{\sphinxupquote{upload}} method}
\label{\detokenize{webservices/upgrade:the-upload-method}}
It’s the most simple and most straightforward way of uploading your database dump.
It uses the HTTPS protocol.


\begin{fulllineitems}
\pysigline{\sphinxbfcode{\sphinxupquote{https://upgrade.odoo.com/database/v1/upload}}}
Uploads a database dump
\begin{quote}\begin{description}
\item[{Parameters}] \leavevmode\begin{itemize}
\item {} 
\sphinxstyleliteralstrong{\sphinxupquote{key}} (\sphinxhref{https://docs.python.org/3/library/stdtypes.html\#str}{\sphinxstyleliteralemphasis{\sphinxupquote{str}}}) \textendash{} (required) your private key

\item {} 
\sphinxstyleliteralstrong{\sphinxupquote{request}} (\sphinxhref{https://docs.python.org/3/library/stdtypes.html\#str}{\sphinxstyleliteralemphasis{\sphinxupquote{str}}}) \textendash{} (required) your request id

\end{itemize}

\item[{Returns}] \leavevmode
request result

\item[{Return type}] \leavevmode
JSON dictionary

\end{description}\end{quote}

\end{fulllineitems}


The request id and the private key are obtained using the {\hyperref[\detokenize{webservices/upgrade:upgrade-api-create-method}]{\sphinxcrossref{\DUrole{std,std-ref}{create method}}}}

The result is a JSON dictionary containing the list of \sphinxcode{\sphinxupquote{failures}}, which
should be empty if everything went fine.
\begin{itemize}
\item {} Python 2
\item {} Bash
\end{itemize}

\fvset{hllines={, ,}}%
\begin{sphinxVerbatim}[commandchars=\\\{\}]
\PYG{k+kn}{import} \PYG{n+nn}{os}
\PYG{k+kn}{import} \PYG{n+nn}{pycurl}
\PYG{k+kn}{from} \PYG{n+nn}{urllib} \PYG{k+kn}{import} \PYG{n}{urlencode}

\PYG{n}{UPLOAD\PYGZus{}URL} \PYG{o}{=} \PYG{l+s+s2}{\PYGZdq{}}\PYG{l+s+s2}{https://upgrade.odoo.com/database/v1/upload}\PYG{l+s+s2}{\PYGZdq{}}
\PYG{n}{DUMPFILE} \PYG{o}{=} \PYG{l+s+s2}{\PYGZdq{}}\PYG{l+s+s2}{openchs.70.cdump}\PYG{l+s+s2}{\PYGZdq{}}

\PYG{n}{fields} \PYG{o}{=} \PYG{n+nb}{dict}\PYG{p}{(}\PYG{p}{[}
    \PYG{p}{(}\PYG{l+s+s1}{\PYGZsq{}}\PYG{l+s+s1}{request}\PYG{l+s+s1}{\PYGZsq{}}\PYG{p}{,} \PYG{l+s+s1}{\PYGZsq{}}\PYG{l+s+s1}{10534}\PYG{l+s+s1}{\PYGZsq{}}\PYG{p}{)}\PYG{p}{,}
    \PYG{p}{(}\PYG{l+s+s1}{\PYGZsq{}}\PYG{l+s+s1}{key}\PYG{l+s+s1}{\PYGZsq{}}\PYG{p}{,} \PYG{l+s+s1}{\PYGZsq{}}\PYG{l+s+s1}{Aw7pItGVKFuZ\PYGZus{}FOR3U8VFQ==}\PYG{l+s+s1}{\PYGZsq{}}\PYG{p}{)}\PYG{p}{,}
\PYG{p}{]}\PYG{p}{)}
\PYG{n}{headers} \PYG{o}{=} \PYG{p}{\PYGZob{}}\PYG{l+s+s2}{\PYGZdq{}}\PYG{l+s+s2}{Content\PYGZhy{}Type}\PYG{l+s+s2}{\PYGZdq{}}\PYG{p}{:} \PYG{l+s+s2}{\PYGZdq{}}\PYG{l+s+s2}{application/octet\PYGZhy{}stream}\PYG{l+s+s2}{\PYGZdq{}}\PYG{p}{\PYGZcb{}}
\PYG{n}{postfields} \PYG{o}{=} \PYG{n}{urlencode}\PYG{p}{(}\PYG{n}{fields}\PYG{p}{)}

\PYG{n}{c} \PYG{o}{=} \PYG{n}{pycurl}\PYG{o}{.}\PYG{n}{Curl}\PYG{p}{(}\PYG{p}{)}
\PYG{n}{c}\PYG{o}{.}\PYG{n}{setopt}\PYG{p}{(}\PYG{n}{pycurl}\PYG{o}{.}\PYG{n}{URL}\PYG{p}{,} \PYG{n}{UPLOAD\PYGZus{}URL}\PYG{o}{+}\PYG{l+s+s2}{\PYGZdq{}}\PYG{l+s+s2}{?}\PYG{l+s+s2}{\PYGZdq{}}\PYG{o}{+}\PYG{n}{postfields}\PYG{p}{)}
\PYG{n}{c}\PYG{o}{.}\PYG{n}{setopt}\PYG{p}{(}\PYG{n}{pycurl}\PYG{o}{.}\PYG{n}{POST}\PYG{p}{,} \PYG{l+m+mi}{1}\PYG{p}{)}
\PYG{n}{filesize} \PYG{o}{=} \PYG{n}{os}\PYG{o}{.}\PYG{n}{path}\PYG{o}{.}\PYG{n}{getsize}\PYG{p}{(}\PYG{n}{DUMPFILE}\PYG{p}{)}
\PYG{n}{c}\PYG{o}{.}\PYG{n}{setopt}\PYG{p}{(}\PYG{n}{pycurl}\PYG{o}{.}\PYG{n}{POSTFIELDSIZE}\PYG{p}{,} \PYG{n}{filesize}\PYG{p}{)}
\PYG{n}{fp} \PYG{o}{=} \PYG{n+nb}{open}\PYG{p}{(}\PYG{n}{DUMPFILE}\PYG{p}{,} \PYG{l+s+s1}{\PYGZsq{}}\PYG{l+s+s1}{rb}\PYG{l+s+s1}{\PYGZsq{}}\PYG{p}{)}
\PYG{n}{c}\PYG{o}{.}\PYG{n}{setopt}\PYG{p}{(}\PYG{n}{pycurl}\PYG{o}{.}\PYG{n}{READFUNCTION}\PYG{p}{,} \PYG{n}{fp}\PYG{o}{.}\PYG{n}{read}\PYG{p}{)}
\PYG{n}{c}\PYG{o}{.}\PYG{n}{setopt}\PYG{p}{(}
    \PYG{n}{pycurl}\PYG{o}{.}\PYG{n}{HTTPHEADER}\PYG{p}{,}
    \PYG{p}{[}\PYG{l+s+s1}{\PYGZsq{}}\PYG{l+s+si}{\PYGZpc{}s}\PYG{l+s+s1}{: }\PYG{l+s+si}{\PYGZpc{}s}\PYG{l+s+s1}{\PYGZsq{}} \PYG{o}{\PYGZpc{}} \PYG{p}{(}\PYG{n}{k}\PYG{p}{,} \PYG{n}{headers}\PYG{p}{[}\PYG{n}{k}\PYG{p}{]}\PYG{p}{)} \PYG{k}{for} \PYG{n}{k} \PYG{o+ow}{in} \PYG{n}{headers}\PYG{p}{]}\PYG{p}{)}

\PYG{n}{c}\PYG{o}{.}\PYG{n}{perform}\PYG{p}{(}\PYG{p}{)}
\PYG{n}{c}\PYG{o}{.}\PYG{n}{close}\PYG{p}{(}\PYG{p}{)}
\end{sphinxVerbatim}

\fvset{hllines={, ,}}%
\begin{sphinxVerbatim}[commandchars=\\\{\}]
\PYG{n+nv}{UPLOAD\PYGZus{}URL}\PYG{o}{=}\PYG{l+s+s2}{\PYGZdq{}https://upgrade.odoo.com/database/v1/upload\PYGZdq{}}
\PYG{n+nv}{DUMPFILE}\PYG{o}{=}\PYG{l+s+s2}{\PYGZdq{}openchs.70.cdump\PYGZdq{}}
\PYG{n+nv}{KEY}\PYG{o}{=}\PYG{l+s+s2}{\PYGZdq{}Aw7pItGVKFuZ\PYGZus{}FOR3U8VFQ==\PYGZdq{}}
\PYG{n+nv}{REQUEST\PYGZus{}ID}\PYG{o}{=}\PYG{l+s+s2}{\PYGZdq{}10534\PYGZdq{}}
\PYG{n+nv}{URL\PYGZus{}PARAMS}\PYG{o}{=}\PYG{l+s+s2}{\PYGZdq{}}\PYG{l+s+s2}{key=}\PYG{l+s+si}{\PYGZdl{}\PYGZob{}}\PYG{n+nv}{KEY}\PYG{l+s+si}{\PYGZcb{}}\PYG{l+s+s2}{\PYGZam{}request=}\PYG{l+s+si}{\PYGZdl{}\PYGZob{}}\PYG{n+nv}{REQUEST\PYGZus{}ID}\PYG{l+s+si}{\PYGZcb{}}\PYG{l+s+s2}{\PYGZdq{}}
\PYG{n+nv}{HEADER}\PYG{o}{=}\PYG{l+s+s2}{\PYGZdq{}Content\PYGZhy{}Type: application/octet\PYGZhy{}stream\PYGZdq{}}
curl \PYGZhy{}H \PYGZdl{}HEADER \PYGZhy{}\PYGZhy{}data\PYGZhy{}binary \PYG{l+s+s2}{\PYGZdq{}}\PYG{l+s+s2}{@}\PYG{l+s+si}{\PYGZdl{}\PYGZob{}}\PYG{n+nv}{DUMPFILE}\PYG{l+s+si}{\PYGZcb{}}\PYG{l+s+s2}{\PYGZdq{}} \PYG{l+s+s2}{\PYGZdq{}}\PYG{l+s+si}{\PYGZdl{}\PYGZob{}}\PYG{n+nv}{UPLOAD\PYGZus{}URL}\PYG{l+s+si}{\PYGZcb{}}\PYG{l+s+s2}{?}\PYG{l+s+si}{\PYGZdl{}\PYGZob{}}\PYG{n+nv}{URL\PYGZus{}PARAMS}\PYG{l+s+si}{\PYGZcb{}}\PYG{l+s+s2}{\PYGZdq{}}
\end{sphinxVerbatim}


\paragraph{The \sphinxstyleliteralintitle{\sphinxupquote{request\_sftp\_access}} method}
\label{\detokenize{webservices/upgrade:upgrade-api-request-sftp-access-method}}\label{\detokenize{webservices/upgrade:the-request-sftp-access-method}}
This method is recommanded for big database dumps.
It uses the SFTP protocol and supports resuming.

It will create a temporary SFTP server where you can connect to and allow you
to upload your database dump using an SFTP client.


\begin{fulllineitems}
\pysigline{\sphinxbfcode{\sphinxupquote{https://upgrade.odoo.com/database/v1/request\_sftp\_access}}}
Creates an SFTP server
\begin{quote}\begin{description}
\item[{Parameters}] \leavevmode\begin{itemize}
\item {} 
\sphinxstyleliteralstrong{\sphinxupquote{key}} (\sphinxhref{https://docs.python.org/3/library/stdtypes.html\#str}{\sphinxstyleliteralemphasis{\sphinxupquote{str}}}) \textendash{} (required) your private key

\item {} 
\sphinxstyleliteralstrong{\sphinxupquote{request}} (\sphinxhref{https://docs.python.org/3/library/stdtypes.html\#str}{\sphinxstyleliteralemphasis{\sphinxupquote{str}}}) \textendash{} (required) your request id

\item {} 
\sphinxstyleliteralstrong{\sphinxupquote{ssh\_keys}} (\sphinxhref{https://docs.python.org/3/library/stdtypes.html\#str}{\sphinxstyleliteralemphasis{\sphinxupquote{str}}}) \textendash{} (required) the path to a file listing the ssh public keys you’d like to use

\end{itemize}

\item[{Returns}] \leavevmode
request result

\item[{Return type}] \leavevmode
JSON dictionary

\end{description}\end{quote}

\end{fulllineitems}


The request id and the private key are obtained using the {\hyperref[\detokenize{webservices/upgrade:upgrade-api-create-method}]{\sphinxcrossref{\DUrole{std,std-ref}{create method}}}}

The file listing your ssh public keys should be roughly similar to a standard \sphinxcode{\sphinxupquote{authorized\_keys}} file.
This file should only contains public keys, blank lines or comments (lines starting with the \sphinxcode{\sphinxupquote{\#}} character)

Your database upgrade request should be in the \sphinxcode{\sphinxupquote{draft}} state.

The \sphinxcode{\sphinxupquote{request\_sftp\_access}} method returns a JSON dictionary containing the following keys:
\begin{itemize}
\item {} Python 2
\item {} Bash
\end{itemize}

\fvset{hllines={, ,}}%
\begin{sphinxVerbatim}[commandchars=\\\{\}]
\PYG{k+kn}{import} \PYG{n+nn}{os}
\PYG{k+kn}{import} \PYG{n+nn}{pycurl}
\PYG{k+kn}{from} \PYG{n+nn}{urllib} \PYG{k+kn}{import} \PYG{n}{urlencode}

\PYG{n}{UPLOAD\PYGZus{}URL} \PYG{o}{=} \PYG{l+s+s2}{\PYGZdq{}}\PYG{l+s+s2}{https://upgrade.odoo.com/database/v1/request\PYGZus{}sftp\PYGZus{}access}\PYG{l+s+s2}{\PYGZdq{}}
\PYG{n}{SSH\PYGZus{}KEYS}\PYG{o}{=}\PYG{l+s+s2}{\PYGZdq{}}\PYG{l+s+s2}{/path/to/your/authorized\PYGZus{}keys}\PYG{l+s+s2}{\PYGZdq{}}

\PYG{n}{fields} \PYG{o}{=} \PYG{n+nb}{dict}\PYG{p}{(}\PYG{p}{[}
    \PYG{p}{(}\PYG{l+s+s1}{\PYGZsq{}}\PYG{l+s+s1}{request}\PYG{l+s+s1}{\PYGZsq{}}\PYG{p}{,} \PYG{l+s+s1}{\PYGZsq{}}\PYG{l+s+s1}{10534}\PYG{l+s+s1}{\PYGZsq{}}\PYG{p}{)}\PYG{p}{,}
    \PYG{p}{(}\PYG{l+s+s1}{\PYGZsq{}}\PYG{l+s+s1}{key}\PYG{l+s+s1}{\PYGZsq{}}\PYG{p}{,} \PYG{l+s+s1}{\PYGZsq{}}\PYG{l+s+s1}{Aw7pItGVKFuZ\PYGZus{}FOR3U8VFQ==}\PYG{l+s+s1}{\PYGZsq{}}\PYG{p}{)}\PYG{p}{,}
\PYG{p}{]}\PYG{p}{)}
\PYG{n}{postfields} \PYG{o}{=} \PYG{n}{urlencode}\PYG{p}{(}\PYG{n}{fields}\PYG{p}{)}

\PYG{n}{c} \PYG{o}{=} \PYG{n}{pycurl}\PYG{o}{.}\PYG{n}{Curl}\PYG{p}{(}\PYG{p}{)}
\PYG{n}{c}\PYG{o}{.}\PYG{n}{setopt}\PYG{p}{(}\PYG{n}{pycurl}\PYG{o}{.}\PYG{n}{URL}\PYG{p}{,} \PYG{n}{UPLOAD\PYGZus{}URL}\PYG{o}{+}\PYG{l+s+s2}{\PYGZdq{}}\PYG{l+s+s2}{?}\PYG{l+s+s2}{\PYGZdq{}}\PYG{o}{+}\PYG{n}{postfields}\PYG{p}{)}
\PYG{n}{c}\PYG{o}{.}\PYG{n}{setopt}\PYG{p}{(}\PYG{n}{pycurl}\PYG{o}{.}\PYG{n}{POST}\PYG{p}{,} \PYG{l+m+mi}{1}\PYG{p}{)}
\PYG{n}{c}\PYG{o}{.}\PYG{n}{setopt}\PYG{p}{(}\PYG{n}{c}\PYG{o}{.}\PYG{n}{HTTPPOST}\PYG{p}{,}\PYG{p}{[}\PYG{p}{(}\PYG{l+s+s2}{\PYGZdq{}}\PYG{l+s+s2}{ssh\PYGZus{}keys}\PYG{l+s+s2}{\PYGZdq{}}\PYG{p}{,}
                        \PYG{p}{(}\PYG{n}{c}\PYG{o}{.}\PYG{n}{FORM\PYGZus{}FILE}\PYG{p}{,} \PYG{n}{SSH\PYGZus{}KEYS}\PYG{p}{,}
                        \PYG{n}{c}\PYG{o}{.}\PYG{n}{FORM\PYGZus{}CONTENTTYPE}\PYG{p}{,} \PYG{l+s+s2}{\PYGZdq{}}\PYG{l+s+s2}{text/plain}\PYG{l+s+s2}{\PYGZdq{}}\PYG{p}{)}\PYG{p}{)}
                    \PYG{p}{]}\PYG{p}{)}

\PYG{n}{c}\PYG{o}{.}\PYG{n}{perform}\PYG{p}{(}\PYG{p}{)}
\PYG{n}{c}\PYG{o}{.}\PYG{n}{close}\PYG{p}{(}\PYG{p}{)}
\end{sphinxVerbatim}

\fvset{hllines={, ,}}%
\begin{sphinxVerbatim}[commandchars=\\\{\}]
\PYG{n+nv}{REQUEST\PYGZus{}SFTP\PYGZus{}ACCESS\PYGZus{}URL}\PYG{o}{=}\PYG{l+s+s2}{\PYGZdq{}https://upgrade.odoo.com/database/v1/request\PYGZus{}sftp\PYGZus{}access\PYGZdq{}}
\PYG{n+nv}{SSH\PYGZus{}KEYS}\PYG{o}{=}/path/to/your/authorized\PYGZus{}keys
\PYG{n+nv}{KEY}\PYG{o}{=}\PYG{l+s+s2}{\PYGZdq{}Aw7pItGVKFuZ\PYGZus{}FOR3U8VFQ==\PYGZdq{}}
\PYG{n+nv}{REQUEST\PYGZus{}ID}\PYG{o}{=}\PYG{l+s+s2}{\PYGZdq{}10534\PYGZdq{}}
\PYG{n+nv}{URL\PYGZus{}PARAMS}\PYG{o}{=}\PYG{l+s+s2}{\PYGZdq{}}\PYG{l+s+s2}{key=}\PYG{l+s+si}{\PYGZdl{}\PYGZob{}}\PYG{n+nv}{KEY}\PYG{l+s+si}{\PYGZcb{}}\PYG{l+s+s2}{\PYGZam{}request=}\PYG{l+s+si}{\PYGZdl{}\PYGZob{}}\PYG{n+nv}{REQUEST\PYGZus{}ID}\PYG{l+s+si}{\PYGZcb{}}\PYG{l+s+s2}{\PYGZdq{}}

curl \PYGZhy{}sS \PYG{l+s+s2}{\PYGZdq{}}\PYG{l+s+si}{\PYGZdl{}\PYGZob{}}\PYG{n+nv}{REQUEST\PYGZus{}SFTP\PYGZus{}ACCESS\PYGZus{}URL}\PYG{l+s+si}{\PYGZcb{}}\PYG{l+s+s2}{?}\PYG{l+s+si}{\PYGZdl{}\PYGZob{}}\PYG{n+nv}{URL\PYGZus{}PARAMS}\PYG{l+s+si}{\PYGZcb{}}\PYG{l+s+s2}{\PYGZdq{}} \PYGZhy{}F \PYG{n+nv}{ssh\PYGZus{}keys}\PYG{o}{=}@\PYG{l+s+si}{\PYGZdl{}\PYGZob{}}\PYG{n+nv}{SSH\PYGZus{}KEYS}\PYG{l+s+si}{\PYGZcb{}} \PYGZgt{} request\PYGZus{}sftp\PYGZus{}result.json

\PYG{c+c1}{\PYGZsh{} check for failures}
\PYG{n+nv}{failures}\PYG{o}{=}\PYG{k}{\PYGZdl{}(}cat request\PYGZus{}sftp\PYGZus{}result.json \PYG{p}{\textbar{}} jq \PYGZhy{}r \PYG{l+s+s1}{\PYGZsq{}.failures[]\PYGZsq{}}\PYG{k}{)}
\PYG{k}{if} \PYG{o}{[} \PYG{l+s+s2}{\PYGZdq{}}\PYG{n+nv}{\PYGZdl{}fa}\PYG{l+s+s2}{ilures}\PYG{l+s+s2}{\PYGZdq{}} !\PYG{o}{=} \PYG{l+s+s2}{\PYGZdq{}\PYGZdq{}} \PYG{o}{]}\PYG{p}{;} \PYG{k}{then}
  \PYG{n+nb}{echo} \PYG{n+nv}{\PYGZdl{}fa}ilures \PYG{p}{\textbar{}} jq \PYGZhy{}r \PYG{l+s+s1}{\PYGZsq{}.\PYGZsq{}}
  \PYG{n+nb}{exit} 1
\PYG{k}{fi}
\end{sphinxVerbatim}


\subparagraph{\sphinxstyleliteralintitle{\sphinxupquote{failures}}}
\label{\detokenize{webservices/upgrade:id1}}
The list of errors. See {\hyperref[\detokenize{webservices/upgrade:upgrade-api-json-failure}]{\sphinxcrossref{\DUrole{std,std-ref}{failures}}}} for an
explanation about the JSON dictionary returned in case of failure.


\subparagraph{\sphinxstyleliteralintitle{\sphinxupquote{request}}}
\label{\detokenize{webservices/upgrade:id2}}
If the call is successful, the value associated to the \sphinxstyleemphasis{request} key
will be a dictionary containing your SFTP connexion parameters:
\begin{itemize}
\item {} 
\sphinxcode{\sphinxupquote{hostname}}: the host address to connect to

\item {} 
\sphinxcode{\sphinxupquote{sftp\_port}}: the port to connect to

\item {} 
\sphinxcode{\sphinxupquote{sftp\_user}}: the SFTP user to use for connecting

\item {} 
\sphinxcode{\sphinxupquote{shared\_file}}: the filename you need to use (identical to the \sphinxcode{\sphinxupquote{filename}} value you have used when creating the request in the {\hyperref[\detokenize{webservices/upgrade:upgrade-api-create-method}]{\sphinxcrossref{\DUrole{std,std-ref}{create method}}}}.)

\item {} 
\sphinxcode{\sphinxupquote{request\_id}}: the related upgrade request id (only informative, ,not required for the connection)

\item {} 
\sphinxcode{\sphinxupquote{sample\_command}}: a sample command using the ‘sftp’ client

\end{itemize}

You should normally be able to connect using the sample command as is.

You will only have access to the \sphinxcode{\sphinxupquote{shared\_file}}. No other files will be
accessible and you will not be able to create new files in your shared
environment on the SFTP server.


\subparagraph{Using the ‘sftp’ client}
\label{\detokenize{webservices/upgrade:using-the-sftp-client}}
Once you have successfully connected using your SFTP client, you can upload
your database dump. Here is a sample session using the ‘sftp’ client:

\fvset{hllines={, ,}}%
\begin{sphinxVerbatim}[commandchars=\\\{\}]
\PYGZdl{} sftp \PYGZhy{}P 2200 user\PYGZus{}10534@upgrade.odoo.com
Connected to upgrade.odoo.com.
sftp\PYGZgt{} put /path/to/openchs.70.cdump openchs.70.cdump
Uploading /path/to/openchs.70.cdump to /openchs.70.cdump
sftp\PYGZgt{} ls \PYGZhy{}l openchs.70.cdump
\PYGZhy{}rw\PYGZhy{}rw\PYGZhy{}rw\PYGZhy{}    0 0        0          849920 Aug 30 15:58 openchs.70.cdump
\end{sphinxVerbatim}

If your connection is interrupted, you can continue your file transfer using
the \sphinxcode{\sphinxupquote{-a}} command line switch:

\fvset{hllines={, ,}}%
\begin{sphinxVerbatim}[commandchars=\\\{\}]
sftp\PYGZgt{} put \PYGZhy{}a /path/to/openchs.70.cdump openchs.70.cdump
Resuming upload of /path/to/openchs.70.cdump to /openchs.70.cdump
\end{sphinxVerbatim}

If you don’t want to manually type the command and need to automate your
database upgrade using a script, you can use a batch file or pipe your commands to ‘sftp’:

\fvset{hllines={, ,}}%
\begin{sphinxVerbatim}[commandchars=\\\{\}]
\PYG{n}{echo} \PYG{l+s+s2}{\PYGZdq{}}\PYG{l+s+s2}{put /path/to/openchs.70.cdump openchs.70.cdump}\PYG{l+s+s2}{\PYGZdq{}} \PYG{o}{\textbar{}} \PYG{n}{sftp} \PYG{o}{\PYGZhy{}}\PYG{n}{b} \PYG{o}{\PYGZhy{}} \PYG{o}{\PYGZhy{}}\PYG{n}{P} \PYG{l+m+mi}{2200} \PYG{n}{user\PYGZus{}10534}\PYG{n+nd}{@upgrade}\PYG{o}{.}\PYG{n}{odoo}\PYG{o}{.}\PYG{n}{com}
\end{sphinxVerbatim}

The \sphinxcode{\sphinxupquote{-b}} parameter takes a filename. If the filename is \sphinxcode{\sphinxupquote{-}}, it reads the commands from standard input.


\subsubsection{Asking to process your request}
\label{\detokenize{webservices/upgrade:asking-to-process-your-request}}\label{\detokenize{webservices/upgrade:upgrade-api-process-method}}
This action ask the Upgrade Platform to process your database dump.


\paragraph{The \sphinxstyleliteralintitle{\sphinxupquote{process}} method}
\label{\detokenize{webservices/upgrade:the-process-method}}

\begin{fulllineitems}
\pysigline{\sphinxbfcode{\sphinxupquote{https://upgrade.odoo.com/database/v1/process}}}
Process a database dump
\begin{quote}\begin{description}
\item[{Parameters}] \leavevmode\begin{itemize}
\item {} 
\sphinxstyleliteralstrong{\sphinxupquote{key}} (\sphinxhref{https://docs.python.org/3/library/stdtypes.html\#str}{\sphinxstyleliteralemphasis{\sphinxupquote{str}}}) \textendash{} (required) your private key

\item {} 
\sphinxstyleliteralstrong{\sphinxupquote{request}} (\sphinxhref{https://docs.python.org/3/library/stdtypes.html\#str}{\sphinxstyleliteralemphasis{\sphinxupquote{str}}}) \textendash{} (required) your request id

\end{itemize}

\item[{Returns}] \leavevmode
request result

\item[{Return type}] \leavevmode
JSON dictionary

\end{description}\end{quote}

\end{fulllineitems}


The request id and the private key are obtained using the {\hyperref[\detokenize{webservices/upgrade:upgrade-api-create-method}]{\sphinxcrossref{\DUrole{std,std-ref}{create method}}}}

The result is a JSON dictionary containing the list of \sphinxcode{\sphinxupquote{failures}}, which
should be empty if everything went fine.
\begin{itemize}
\item {} Python 2
\item {} Bash
\end{itemize}

\fvset{hllines={, ,}}%
\begin{sphinxVerbatim}[commandchars=\\\{\}]
\PYG{k+kn}{from} \PYG{n+nn}{urllib} \PYG{k+kn}{import} \PYG{n}{urlencode}
\PYG{k+kn}{from} \PYG{n+nn}{io} \PYG{k+kn}{import} \PYG{n}{BytesIO}
\PYG{k+kn}{import} \PYG{n+nn}{pycurl}
\PYG{k+kn}{import} \PYG{n+nn}{json}

\PYG{n}{PROCESS\PYGZus{}URL} \PYG{o}{=} \PYG{l+s+s2}{\PYGZdq{}}\PYG{l+s+s2}{https://upgrade.odoo.com/database/v1/process}\PYG{l+s+s2}{\PYGZdq{}}

\PYG{n}{fields} \PYG{o}{=} \PYG{n+nb}{dict}\PYG{p}{(}\PYG{p}{[}
    \PYG{p}{(}\PYG{l+s+s1}{\PYGZsq{}}\PYG{l+s+s1}{request}\PYG{l+s+s1}{\PYGZsq{}}\PYG{p}{,} \PYG{l+s+s1}{\PYGZsq{}}\PYG{l+s+s1}{10534}\PYG{l+s+s1}{\PYGZsq{}}\PYG{p}{)}\PYG{p}{,}
    \PYG{p}{(}\PYG{l+s+s1}{\PYGZsq{}}\PYG{l+s+s1}{key}\PYG{l+s+s1}{\PYGZsq{}}\PYG{p}{,} \PYG{l+s+s1}{\PYGZsq{}}\PYG{l+s+s1}{Aw7pItGVKFuZ\PYGZus{}FOR3U8VFQ==}\PYG{l+s+s1}{\PYGZsq{}}\PYG{p}{)}\PYG{p}{,}
\PYG{p}{]}\PYG{p}{)}
\PYG{n}{postfields} \PYG{o}{=} \PYG{n}{urlencode}\PYG{p}{(}\PYG{n}{fields}\PYG{p}{)}

\PYG{n}{c} \PYG{o}{=} \PYG{n}{pycurl}\PYG{o}{.}\PYG{n}{Curl}\PYG{p}{(}\PYG{p}{)}
\PYG{n}{c}\PYG{o}{.}\PYG{n}{setopt}\PYG{p}{(}\PYG{n}{pycurl}\PYG{o}{.}\PYG{n}{URL}\PYG{p}{,} \PYG{n}{PROCESS\PYGZus{}URL}\PYG{p}{)}
\PYG{n}{c}\PYG{o}{.}\PYG{n}{setopt}\PYG{p}{(}\PYG{n}{c}\PYG{o}{.}\PYG{n}{POSTFIELDS}\PYG{p}{,} \PYG{n}{postfields}\PYG{p}{)}
\PYG{n}{data} \PYG{o}{=} \PYG{n}{BytesIO}\PYG{p}{(}\PYG{p}{)}
\PYG{n}{c}\PYG{o}{.}\PYG{n}{setopt}\PYG{p}{(}\PYG{n}{c}\PYG{o}{.}\PYG{n}{WRITEFUNCTION}\PYG{p}{,} \PYG{n}{data}\PYG{o}{.}\PYG{n}{write}\PYG{p}{)}
\PYG{n}{c}\PYG{o}{.}\PYG{n}{perform}\PYG{p}{(}\PYG{p}{)}

\PYG{c+c1}{\PYGZsh{} transform output into a dict:}
\PYG{n}{response} \PYG{o}{=} \PYG{n}{json}\PYG{o}{.}\PYG{n}{loads}\PYG{p}{(}\PYG{n}{data}\PYG{o}{.}\PYG{n}{getvalue}\PYG{p}{(}\PYG{p}{)}\PYG{p}{)}

\PYG{c+c1}{\PYGZsh{} get http status:}
\PYG{n}{http\PYGZus{}code} \PYG{o}{=} \PYG{n}{c}\PYG{o}{.}\PYG{n}{getinfo}\PYG{p}{(}\PYG{n}{pycurl}\PYG{o}{.}\PYG{n}{HTTP\PYGZus{}CODE}\PYG{p}{)}
\PYG{n}{c}\PYG{o}{.}\PYG{n}{close}\PYG{p}{(}\PYG{p}{)}
\end{sphinxVerbatim}

\fvset{hllines={, ,}}%
\begin{sphinxVerbatim}[commandchars=\\\{\}]
\PYG{n+nv}{PROCESS\PYGZus{}URL}\PYG{o}{=}\PYG{l+s+s2}{\PYGZdq{}https://upgrade.odoo.com/database/v1/process\PYGZdq{}}
\PYG{n+nv}{KEY}\PYG{o}{=}\PYG{l+s+s2}{\PYGZdq{}Aw7pItGVKFuZ\PYGZus{}FOR3U8VFQ==\PYGZdq{}}
\PYG{n+nv}{REQUEST\PYGZus{}ID}\PYG{o}{=}\PYG{l+s+s2}{\PYGZdq{}10534\PYGZdq{}}
\PYG{n+nv}{URL\PYGZus{}PARAMS}\PYG{o}{=}\PYG{l+s+s2}{\PYGZdq{}}\PYG{l+s+s2}{key=}\PYG{l+s+si}{\PYGZdl{}\PYGZob{}}\PYG{n+nv}{KEY}\PYG{l+s+si}{\PYGZcb{}}\PYG{l+s+s2}{\PYGZam{}request=}\PYG{l+s+si}{\PYGZdl{}\PYGZob{}}\PYG{n+nv}{REQUEST\PYGZus{}ID}\PYG{l+s+si}{\PYGZcb{}}\PYG{l+s+s2}{\PYGZdq{}}
curl \PYGZhy{}sS \PYG{l+s+s2}{\PYGZdq{}}\PYG{l+s+si}{\PYGZdl{}\PYGZob{}}\PYG{n+nv}{PROCESS\PYGZus{}URL}\PYG{l+s+si}{\PYGZcb{}}\PYG{l+s+s2}{?}\PYG{l+s+si}{\PYGZdl{}\PYGZob{}}\PYG{n+nv}{URL\PYGZus{}PARAMS}\PYG{l+s+si}{\PYGZcb{}}\PYG{l+s+s2}{\PYGZdq{}}
\end{sphinxVerbatim}


\subsubsection{Obtaining your request status}
\label{\detokenize{webservices/upgrade:obtaining-your-request-status}}\label{\detokenize{webservices/upgrade:upgrade-api-status-method}}
This action ask the status of your database upgrade request.


\paragraph{The \sphinxstyleliteralintitle{\sphinxupquote{status}} method}
\label{\detokenize{webservices/upgrade:the-status-method}}

\begin{fulllineitems}
\pysigline{\sphinxbfcode{\sphinxupquote{https://upgrade.odoo.com/database/v1/status}}}
Ask the status of a database upgrade request
\begin{quote}\begin{description}
\item[{Parameters}] \leavevmode\begin{itemize}
\item {} 
\sphinxstyleliteralstrong{\sphinxupquote{key}} (\sphinxhref{https://docs.python.org/3/library/stdtypes.html\#str}{\sphinxstyleliteralemphasis{\sphinxupquote{str}}}) \textendash{} (required) your private key

\item {} 
\sphinxstyleliteralstrong{\sphinxupquote{request}} (\sphinxhref{https://docs.python.org/3/library/stdtypes.html\#str}{\sphinxstyleliteralemphasis{\sphinxupquote{str}}}) \textendash{} (required) your request id

\end{itemize}

\item[{Returns}] \leavevmode
request result

\item[{Return type}] \leavevmode
JSON dictionary

\end{description}\end{quote}

\end{fulllineitems}


The request id and the private key are obtained using the {\hyperref[\detokenize{webservices/upgrade:upgrade-api-create-method}]{\sphinxcrossref{\DUrole{std,std-ref}{create method}}}}

The result is a JSON dictionary containing various information about your
database upgrade request.
\begin{itemize}
\item {} Python 2
\item {} Bash
\end{itemize}

\fvset{hllines={, ,}}%
\begin{sphinxVerbatim}[commandchars=\\\{\}]
\PYG{k+kn}{from} \PYG{n+nn}{urllib} \PYG{k+kn}{import} \PYG{n}{urlencode}
\PYG{k+kn}{from} \PYG{n+nn}{io} \PYG{k+kn}{import} \PYG{n}{BytesIO}
\PYG{k+kn}{import} \PYG{n+nn}{pycurl}
\PYG{k+kn}{import} \PYG{n+nn}{json}

\PYG{n}{STATUS\PYGZus{}URL} \PYG{o}{=} \PYG{l+s+s2}{\PYGZdq{}}\PYG{l+s+s2}{https://upgrade.odoo.com/database/v1/status}\PYG{l+s+s2}{\PYGZdq{}}

\PYG{n}{fields} \PYG{o}{=} \PYG{n+nb}{dict}\PYG{p}{(}\PYG{p}{[}
    \PYG{p}{(}\PYG{l+s+s1}{\PYGZsq{}}\PYG{l+s+s1}{request}\PYG{l+s+s1}{\PYGZsq{}}\PYG{p}{,} \PYG{l+s+s1}{\PYGZsq{}}\PYG{l+s+s1}{10534}\PYG{l+s+s1}{\PYGZsq{}}\PYG{p}{)}\PYG{p}{,}
    \PYG{p}{(}\PYG{l+s+s1}{\PYGZsq{}}\PYG{l+s+s1}{key}\PYG{l+s+s1}{\PYGZsq{}}\PYG{p}{,} \PYG{l+s+s1}{\PYGZsq{}}\PYG{l+s+s1}{Aw7pItGVKFuZ\PYGZus{}FOR3U8VFQ==}\PYG{l+s+s1}{\PYGZsq{}}\PYG{p}{)}\PYG{p}{,}
\PYG{p}{]}\PYG{p}{)}
\PYG{n}{postfields} \PYG{o}{=} \PYG{n}{urlencode}\PYG{p}{(}\PYG{n}{fields}\PYG{p}{)}

\PYG{n}{c} \PYG{o}{=} \PYG{n}{pycurl}\PYG{o}{.}\PYG{n}{Curl}\PYG{p}{(}\PYG{p}{)}
\PYG{n}{c}\PYG{o}{.}\PYG{n}{setopt}\PYG{p}{(}\PYG{n}{pycurl}\PYG{o}{.}\PYG{n}{URL}\PYG{p}{,} \PYG{n}{PROCESS\PYGZus{}URL}\PYG{p}{)}
\PYG{n}{c}\PYG{o}{.}\PYG{n}{setopt}\PYG{p}{(}\PYG{n}{c}\PYG{o}{.}\PYG{n}{POSTFIELDS}\PYG{p}{,} \PYG{n}{postfields}\PYG{p}{)}
\PYG{n}{data} \PYG{o}{=} \PYG{n}{BytesIO}\PYG{p}{(}\PYG{p}{)}
\PYG{n}{c}\PYG{o}{.}\PYG{n}{setopt}\PYG{p}{(}\PYG{n}{c}\PYG{o}{.}\PYG{n}{WRITEFUNCTION}\PYG{p}{,} \PYG{n}{data}\PYG{o}{.}\PYG{n}{write}\PYG{p}{)}
\PYG{n}{c}\PYG{o}{.}\PYG{n}{perform}\PYG{p}{(}\PYG{p}{)}

\PYG{c+c1}{\PYGZsh{} transform output into a dict:}
\PYG{n}{response} \PYG{o}{=} \PYG{n}{json}\PYG{o}{.}\PYG{n}{loads}\PYG{p}{(}\PYG{n}{data}\PYG{o}{.}\PYG{n}{getvalue}\PYG{p}{(}\PYG{p}{)}\PYG{p}{)}

\PYG{n}{c}\PYG{o}{.}\PYG{n}{close}\PYG{p}{(}\PYG{p}{)}
\end{sphinxVerbatim}

\fvset{hllines={, ,}}%
\begin{sphinxVerbatim}[commandchars=\\\{\}]
\PYG{n+nv}{STATUS\PYGZus{}URL}\PYG{o}{=}\PYG{l+s+s2}{\PYGZdq{}https://upgrade.odoo.com/database/v1/status\PYGZdq{}}
\PYG{n+nv}{KEY}\PYG{o}{=}\PYG{l+s+s2}{\PYGZdq{}Aw7pItGVKFuZ\PYGZus{}FOR3U8VFQ==\PYGZdq{}}
\PYG{n+nv}{REQUEST\PYGZus{}ID}\PYG{o}{=}\PYG{l+s+s2}{\PYGZdq{}10534\PYGZdq{}}
\PYG{n+nv}{URL\PYGZus{}PARAMS}\PYG{o}{=}\PYG{l+s+s2}{\PYGZdq{}}\PYG{l+s+s2}{key=}\PYG{l+s+si}{\PYGZdl{}\PYGZob{}}\PYG{n+nv}{KEY}\PYG{l+s+si}{\PYGZcb{}}\PYG{l+s+s2}{\PYGZam{}request=}\PYG{l+s+si}{\PYGZdl{}\PYGZob{}}\PYG{n+nv}{REQUEST\PYGZus{}ID}\PYG{l+s+si}{\PYGZcb{}}\PYG{l+s+s2}{\PYGZdq{}}
curl \PYGZhy{}sS \PYG{l+s+s2}{\PYGZdq{}}\PYG{l+s+si}{\PYGZdl{}\PYGZob{}}\PYG{n+nv}{STATUS\PYGZus{}URL}\PYG{l+s+si}{\PYGZcb{}}\PYG{l+s+s2}{?}\PYG{l+s+si}{\PYGZdl{}\PYGZob{}}\PYG{n+nv}{URL\PYGZus{}PARAMS}\PYG{l+s+si}{\PYGZcb{}}\PYG{l+s+s2}{\PYGZdq{}}
\end{sphinxVerbatim}


\paragraph{Sample output}
\label{\detokenize{webservices/upgrade:sample-output}}
The \sphinxcode{\sphinxupquote{request}} key contains various useful information about your request:
\begin{description}
\item[{\sphinxcode{\sphinxupquote{id}}}] \leavevmode
the request id

\item[{\sphinxcode{\sphinxupquote{key}}}] \leavevmode
your private key

\item[{\sphinxcode{\sphinxupquote{email}}}] \leavevmode
the email address you supplied when creating the request

\item[{\sphinxcode{\sphinxupquote{target}}}] \leavevmode
the target Odoo version you supplied when creating the request

\item[{\sphinxcode{\sphinxupquote{aim}}}] \leavevmode
the purpose (test, production) of your database upgrade request you supplied when creating the request

\item[{\sphinxcode{\sphinxupquote{filename}}}] \leavevmode
the filename you supplied when creating the request

\item[{\sphinxcode{\sphinxupquote{timezone}}}] \leavevmode
the timezone you supplied when creating the request

\item[{\sphinxcode{\sphinxupquote{state}}}] \leavevmode
the state of your request

\item[{\sphinxcode{\sphinxupquote{issue\_stage}}}] \leavevmode
the stage of the issue we have create on Odoo main server

\item[{\sphinxcode{\sphinxupquote{issue}}}] \leavevmode
the id of the issue we have create on Odoo main server

\item[{\sphinxcode{\sphinxupquote{status\_url}}}] \leavevmode
the URL to access your database upgrade request html page

\item[{\sphinxcode{\sphinxupquote{notes\_url}}}] \leavevmode
the URL to get the notes about your database upgrade

\item[{\sphinxcode{\sphinxupquote{original\_sql\_url}}}] \leavevmode
the URL used to get your uploaded (not upgraded) database as an SQL stream

\item[{\sphinxcode{\sphinxupquote{original\_dump\_url}}}] \leavevmode
the URL used to get your uploaded (not upgraded) database as an archive file

\item[{\sphinxcode{\sphinxupquote{upgraded\_sql\_url}}}] \leavevmode
the URL used to get your upgraded database as an SQL stream

\item[{\sphinxcode{\sphinxupquote{upgraded\_dump\_url}}}] \leavevmode
the URL used to get your upgraded database as an archive file

\item[{\sphinxcode{\sphinxupquote{modules\_url}}}] \leavevmode
the URL used to get your custom modules

\item[{\sphinxcode{\sphinxupquote{filesize}}}] \leavevmode
the size of your uploaded database file

\item[{\sphinxcode{\sphinxupquote{database\_uuid}}}] \leavevmode
the Unique ID of your database

\item[{\sphinxcode{\sphinxupquote{created\_at}}}] \leavevmode
the date when you created the request

\item[{\sphinxcode{\sphinxupquote{estimated\_time}}}] \leavevmode
an estimation of the time it takes to upgrade your database

\item[{\sphinxcode{\sphinxupquote{processed\_at}}}] \leavevmode
time when your database upgrade was started

\item[{\sphinxcode{\sphinxupquote{elapsed}}}] \leavevmode
the time it takes to upgrade your database

\item[{\sphinxcode{\sphinxupquote{filestore}}}] \leavevmode
your attachments were converted to the filestore

\item[{\sphinxcode{\sphinxupquote{customer\_message}}}] \leavevmode
an important message related to your request

\item[{\sphinxcode{\sphinxupquote{database\_version}}}] \leavevmode
the guessed Odoo version of your uploaded (not upgraded) database

\item[{\sphinxcode{\sphinxupquote{postgresql}}}] \leavevmode
the guessed Postgresql version of your uploaded (not upgraded) database

\item[{\sphinxcode{\sphinxupquote{compressions}}}] \leavevmode
the compression methods used by your uploaded database

\end{description}
\begin{itemize}
\item {} JSON
\end{itemize}

\fvset{hllines={, ,}}%
\begin{sphinxVerbatim}[commandchars=\\\{\}]
\PYG{p}{\PYGZob{}}
  \PYG{n+nt}{\PYGZdq{}failures\PYGZdq{}}\PYG{p}{:} \PYG{p}{[}\PYG{p}{]}\PYG{p}{,}
  \PYG{n+nt}{\PYGZdq{}request\PYGZdq{}}\PYG{p}{:} \PYG{p}{\PYGZob{}}
    \PYG{n+nt}{\PYGZdq{}id\PYGZdq{}}\PYG{p}{:} \PYG{l+m+mi}{10534}\PYG{p}{,}
    \PYG{n+nt}{\PYGZdq{}key\PYGZdq{}}\PYG{p}{:} \PYG{l+s+s2}{\PYGZdq{}Aw7pItGVKFuZ\PYGZus{}FOR3U8VFQ==\PYGZdq{}}\PYG{p}{,}
    \PYG{n+nt}{\PYGZdq{}email\PYGZdq{}}\PYG{p}{:} \PYG{l+s+s2}{\PYGZdq{}john.doe@example.com\PYGZdq{}}\PYG{p}{,}
    \PYG{n+nt}{\PYGZdq{}target\PYGZdq{}}\PYG{p}{:} \PYG{l+s+s2}{\PYGZdq{}8.0\PYGZdq{}}\PYG{p}{,}
    \PYG{n+nt}{\PYGZdq{}aim\PYGZdq{}}\PYG{p}{:} \PYG{l+s+s2}{\PYGZdq{}test\PYGZdq{}}\PYG{p}{,}
    \PYG{n+nt}{\PYGZdq{}filename\PYGZdq{}}\PYG{p}{:} \PYG{l+s+s2}{\PYGZdq{}db\PYGZus{}name.dump\PYGZdq{}}\PYG{p}{,}
    \PYG{n+nt}{\PYGZdq{}timezone\PYGZdq{}}\PYG{p}{:} \PYG{k+kc}{null}\PYG{p}{,}
    \PYG{n+nt}{\PYGZdq{}state\PYGZdq{}}\PYG{p}{:} \PYG{l+s+s2}{\PYGZdq{}draft\PYGZdq{}}\PYG{p}{,}
    \PYG{n+nt}{\PYGZdq{}issue\PYGZus{}stage\PYGZdq{}}\PYG{p}{:} \PYG{l+s+s2}{\PYGZdq{}new\PYGZdq{}}\PYG{p}{,}
    \PYG{n+nt}{\PYGZdq{}issue\PYGZdq{}}\PYG{p}{:} \PYG{l+m+mi}{648398}\PYG{p}{,}
    \PYG{n+nt}{\PYGZdq{}status\PYGZus{}url\PYGZdq{}}\PYG{p}{:} \PYG{l+s+s2}{\PYGZdq{}https://upgrade.odoo.com/database/eu1/10534/Aw7pItGVKFuZ\PYGZus{}FOR3U8VFQ==/status\PYGZdq{}}\PYG{p}{,}
    \PYG{n+nt}{\PYGZdq{}notes\PYGZus{}url\PYGZdq{}}\PYG{p}{:} \PYG{l+s+s2}{\PYGZdq{}https://upgrade.odoo.com/database/eu1/10534/Aw7pItGVKFuZ\PYGZus{}FOR3U8VFQ==/upgraded/notes\PYGZdq{}}\PYG{p}{,}
    \PYG{n+nt}{\PYGZdq{}original\PYGZus{}sql\PYGZus{}url\PYGZdq{}}\PYG{p}{:} \PYG{l+s+s2}{\PYGZdq{}https://upgrade.odoo.com/database/eu1/10534/Aw7pItGVKFuZ\PYGZus{}FOR3U8VFQ==/original/sql\PYGZdq{}}\PYG{p}{,}
    \PYG{n+nt}{\PYGZdq{}original\PYGZus{}dump\PYGZus{}url\PYGZdq{}}\PYG{p}{:} \PYG{l+s+s2}{\PYGZdq{}https://upgrade.odoo.com/database/eu1/10534/Aw7pItGVKFuZ\PYGZus{}FOR3U8VFQ==/original/archive\PYGZdq{}}\PYG{p}{,}
    \PYG{n+nt}{\PYGZdq{}upgraded\PYGZus{}sql\PYGZus{}url\PYGZdq{}}\PYG{p}{:} \PYG{l+s+s2}{\PYGZdq{}https://upgrade.odoo.com/database/eu1/10534/Aw7pItGVKFuZ\PYGZus{}FOR3U8VFQ==/upgraded/sql\PYGZdq{}}\PYG{p}{,}
    \PYG{n+nt}{\PYGZdq{}upgraded\PYGZus{}dump\PYGZus{}url\PYGZdq{}}\PYG{p}{:} \PYG{l+s+s2}{\PYGZdq{}https://upgrade.odoo.com/database/eu1/10534/Aw7pItGVKFuZ\PYGZus{}FOR3U8VFQ==/upgraded/archive\PYGZdq{}}\PYG{p}{,}
    \PYG{n+nt}{\PYGZdq{}modules\PYGZus{}url\PYGZdq{}}\PYG{p}{:} \PYG{l+s+s2}{\PYGZdq{}https://upgrade.odoo.com/database/eu1/10534/Aw7pItGVKFuZ\PYGZus{}FOR3U8VFQ==/modules/archive\PYGZdq{}}\PYG{p}{,}
    \PYG{n+nt}{\PYGZdq{}filesize\PYGZdq{}}\PYG{p}{:} \PYG{l+s+s2}{\PYGZdq{}912.99 Kb\PYGZdq{}}\PYG{p}{,}
    \PYG{n+nt}{\PYGZdq{}database\PYGZus{}uuid\PYGZdq{}}\PYG{p}{:} \PYG{k+kc}{null}\PYG{p}{,}
    \PYG{n+nt}{\PYGZdq{}created\PYGZus{}at\PYGZdq{}}\PYG{p}{:} \PYG{l+s+s2}{\PYGZdq{}2015\PYGZhy{}09\PYGZhy{}09 07:13:49\PYGZdq{}}\PYG{p}{,}
    \PYG{n+nt}{\PYGZdq{}estimated\PYGZus{}time\PYGZdq{}}\PYG{p}{:} \PYG{k+kc}{null}\PYG{p}{,}
    \PYG{n+nt}{\PYGZdq{}processed\PYGZus{}at\PYGZdq{}}\PYG{p}{:} \PYG{k+kc}{null}\PYG{p}{,}
    \PYG{n+nt}{\PYGZdq{}elapsed\PYGZdq{}}\PYG{p}{:} \PYG{l+s+s2}{\PYGZdq{}00:00\PYGZdq{}}\PYG{p}{,}
    \PYG{n+nt}{\PYGZdq{}filestore\PYGZdq{}}\PYG{p}{:} \PYG{k+kc}{false}\PYG{p}{,}
    \PYG{n+nt}{\PYGZdq{}customer\PYGZus{}message\PYGZdq{}}\PYG{p}{:} \PYG{k+kc}{null}\PYG{p}{,}
    \PYG{n+nt}{\PYGZdq{}database\PYGZus{}version\PYGZdq{}}\PYG{p}{:} \PYG{k+kc}{null}\PYG{p}{,}
    \PYG{n+nt}{\PYGZdq{}postgresql\PYGZdq{}}\PYG{p}{:} \PYG{l+s+s2}{\PYGZdq{}9.4\PYGZdq{}}\PYG{p}{,}
    \PYG{n+nt}{\PYGZdq{}compressions\PYGZdq{}}\PYG{p}{:} \PYG{p}{[}
      \PYG{l+s+s2}{\PYGZdq{}pgdmp\PYGZus{}custom\PYGZdq{}}\PYG{p}{,}
      \PYG{l+s+s2}{\PYGZdq{}sql\PYGZdq{}}
    \PYG{p}{]}
  \PYG{p}{\PYGZcb{}}
\PYG{p}{\PYGZcb{}}
\end{sphinxVerbatim}


\subsubsection{Downloading your database dump}
\label{\detokenize{webservices/upgrade:upgrade-api-download-method}}\label{\detokenize{webservices/upgrade:downloading-your-database-dump}}
Beside downloading your migrated database using the URL provided by the
{\hyperref[\detokenize{webservices/upgrade:upgrade-api-status-method}]{\sphinxcrossref{\DUrole{std,std-ref}{status method}}}}, you can also use the SFTP
protocol as described in the {\hyperref[\detokenize{webservices/upgrade:upgrade-api-request-sftp-access-method}]{\sphinxcrossref{\DUrole{std,std-ref}{request\_sftp\_access method}}}}

The diffence is that you’ll only be able to download the migrated database. No
uploading will be possible.

Your database upgrade request should be in the \sphinxcode{\sphinxupquote{done}} state.

Once you have successfully connected using your SFTP client, you can download
your database dump. Here is a sample session using the ‘sftp’ client:

\fvset{hllines={, ,}}%
\begin{sphinxVerbatim}[commandchars=\\\{\}]
\PYGZdl{} sftp \PYGZhy{}P 2200 user\PYGZus{}10534@upgrade.odoo.com
Connected to upgrade.odoo.com.
sftp\PYGZgt{} get upgraded\PYGZus{}openchs.70.cdump /path/to/upgraded\PYGZus{}openchs.70.cdump
Downloading /upgraded\PYGZus{}openchs.70.cdump to /path/to/upgraded\PYGZus{}openchs.70.cdump
\end{sphinxVerbatim}


\section{Creating a Localization}
\label{\detokenize{webservices/localization:creating-a-localization}}\label{\detokenize{webservices/localization::doc}}
\begin{sphinxadmonition}{warning}{Warning:}
This tutorial requires knowledges about how to build a module in Odoo (see
{\hyperref[\detokenize{howtos/backend::doc}]{\sphinxcrossref{\DUrole{doc}{Building a Module}}}}).
\end{sphinxadmonition}


\subsection{Building a localization module}
\label{\detokenize{webservices/localization:building-a-localization-module}}
When installing the \sphinxcode{\sphinxupquote{accounting}} module, the localization module corresponding to the country code of the company is installed automatically.
In case of no country code set or no localization module found, the \sphinxcode{\sphinxupquote{l10n\_generic\_coa}} (US) localization module is installed by default.

For example, \sphinxcode{\sphinxupquote{l10n\_be}} will be installed if the company has \sphinxcode{\sphinxupquote{Belgium}} as country.

This behavior is allowed by the presence of a \sphinxstyleemphasis{.yml} file containing the following code:

\fvset{hllines={, ,}}%
\begin{sphinxVerbatim}[commandchars=\\\{\}]
\PYGZhy{}
  !python \PYGZob{}model: account.chart.template, id: pl\PYGZus{}chart\PYGZus{}template\PYGZcb{}: \textbar{}
    self[0].try\PYGZus{}loading\PYGZus{}for\PYGZus{}current\PYGZus{}company()
\end{sphinxVerbatim}

Usually located in the \sphinxcode{\sphinxupquote{data}} folder, it must be loaded at the very last in the \sphinxcode{\sphinxupquote{\_\_manifest\_\_.py}} file.

\begin{sphinxadmonition}{danger}{Danger:}
If the \sphinxstyleemphasis{.yml} file is missing, the right chart of accounts won’t be loaded on time!
\end{sphinxadmonition}


\subsection{Configuring my own Chart of Accounts?}
\label{\detokenize{webservices/localization:configuring-my-own-chart-of-accounts}}
First of all, before I proceed, we need to talk about the templates. A template is a record that allows replica of itself.
This mechanism is needed when working in multi-companies. For example, the creation of a new account is done using the \sphinxcode{\sphinxupquote{account.account.template}} model.
However, each company using this chart of accounts will be linked to a replica having \sphinxcode{\sphinxupquote{account.account}} as model.
So, the templates are never used directly by the company.

Then, when a chart of accounts needs to be installed, all templates dependent of this one will create a replica and link this newly generated record to the company’s user.
It means all such templates must be linked to the chart of accounts in some way. To do so, each one must reference the desired chart of accounts using the \sphinxcode{\sphinxupquote{chart\_template\_id}} field.
For this reason, we need to define an instance of the \sphinxcode{\sphinxupquote{account.chart.template}} model before creating its templates.

\fvset{hllines={, ,}}%
\begin{sphinxVerbatim}[commandchars=\\\{\}]
\PYG{n+nt}{\PYGZlt{}record} \PYG{n+na}{id=}\PYG{l+s}{\PYGZdq{}...\PYGZdq{}} \PYG{n+na}{model=}\PYG{l+s}{\PYGZdq{}account.chart.template\PYGZdq{}}\PYG{n+nt}{\PYGZgt{}}
    \PYG{c}{\PYGZlt{}!\PYGZhy{}\PYGZhy{}}\PYG{c}{ [Required] Specify the name to display for this CoA. }\PYG{c}{\PYGZhy{}\PYGZhy{}\PYGZgt{}}
    \PYG{n+nt}{\PYGZlt{}field} \PYG{n+na}{name=}\PYG{l+s}{\PYGZdq{}name\PYGZdq{}}\PYG{n+nt}{\PYGZgt{}}...\PYG{n+nt}{\PYGZlt{}/field\PYGZgt{}}

    \PYG{c}{\PYGZlt{}!\PYGZhy{}\PYGZhy{}}\PYG{c}{ [Required] Specify the currency. E.g. \PYGZdq{}base.USD\PYGZdq{}. }\PYG{c}{\PYGZhy{}\PYGZhy{}\PYGZgt{}}
    \PYG{n+nt}{\PYGZlt{}field} \PYG{n+na}{name=}\PYG{l+s}{\PYGZdq{}currency\PYGZus{}id\PYGZdq{}} \PYG{n+na}{ref=}\PYG{l+s}{\PYGZdq{}...\PYGZdq{}}\PYG{n+nt}{/\PYGZgt{}}

    \PYG{c}{\PYGZlt{}!\PYGZhy{}\PYGZhy{}}\PYG{c}{ [Required] Specify a prefix of the bank accounts. }\PYG{c}{\PYGZhy{}\PYGZhy{}\PYGZgt{}}
    \PYG{n+nt}{\PYGZlt{}field} \PYG{n+na}{name=}\PYG{l+s}{\PYGZdq{}bank\PYGZus{}account\PYGZus{}code\PYGZus{}prefix\PYGZdq{}}\PYG{n+nt}{\PYGZgt{}}...\PYG{n+nt}{\PYGZlt{}/field\PYGZgt{}}

    \PYG{c}{\PYGZlt{}!\PYGZhy{}\PYGZhy{}}\PYG{c}{ [Required] Specify a prefix of the cash accounts. }\PYG{c}{\PYGZhy{}\PYGZhy{}\PYGZgt{}}
    \PYG{n+nt}{\PYGZlt{}field} \PYG{n+na}{name=}\PYG{l+s}{\PYGZdq{}cash\PYGZus{}account\PYGZus{}code\PYGZus{}prefix\PYGZdq{}}\PYG{n+nt}{\PYGZgt{}}...\PYG{n+nt}{\PYGZlt{}/field\PYGZgt{}}

    \PYG{c}{\PYGZlt{}!\PYGZhy{}\PYGZhy{}}\PYG{c}{ [Optional] Define a parent chart of accounts that will be installed just before this one. }\PYG{c}{\PYGZhy{}\PYGZhy{}\PYGZgt{}}
    \PYG{n+nt}{\PYGZlt{}field} \PYG{n+na}{name=}\PYG{l+s}{\PYGZdq{}parent\PYGZus{}id\PYGZdq{}} \PYG{n+na}{ref=}\PYG{l+s}{\PYGZdq{}...\PYGZdq{}}\PYG{n+nt}{/\PYGZgt{}}

    \PYG{c}{\PYGZlt{}!\PYGZhy{}\PYGZhy{}}\PYG{c}{ [Optional] Define the number of digits of account codes. By default, the value is 6. }\PYG{c}{\PYGZhy{}\PYGZhy{}\PYGZgt{}}
    \PYG{n+nt}{\PYGZlt{}field} \PYG{n+na}{name=}\PYG{l+s}{\PYGZdq{}code\PYGZus{}digits\PYGZdq{}}\PYG{n+nt}{\PYGZgt{}}...\PYG{n+nt}{\PYGZlt{}/field\PYGZgt{}}

    \PYG{c}{\PYGZlt{}!\PYGZhy{}\PYGZhy{}}\PYG{c}{ [Optional] Boolean to show or not this CoA on the list. By default, the CoA is visible.}
\PYG{c}{     This field is mostly used when this CoA has some children (see parent\PYGZus{}id field). }\PYG{c}{\PYGZhy{}\PYGZhy{}\PYGZgt{}}
    \PYG{n+nt}{\PYGZlt{}field} \PYG{n+na}{name=}\PYG{l+s}{\PYGZdq{}visible\PYGZdq{}} \PYG{n+na}{eval=}\PYG{l+s}{\PYGZdq{}...\PYGZdq{}}\PYG{n+nt}{/\PYGZgt{}}

    \PYG{c}{\PYGZlt{}!\PYGZhy{}\PYGZhy{}}\PYG{c}{ [Optional] Boolean to enable the Anglo}\PYG{c}{\PYGZhy{}}\PYG{c}{Saxon accounting. By default, this field is False. }\PYG{c}{\PYGZhy{}\PYGZhy{}\PYGZgt{}}
    \PYG{n+nt}{\PYGZlt{}field} \PYG{n+na}{name=}\PYG{l+s}{\PYGZdq{}use\PYGZus{}anglo\PYGZus{}saxon\PYGZdq{}} \PYG{n+na}{eval=}\PYG{l+s}{\PYGZdq{}...\PYGZdq{}}\PYG{n+nt}{/\PYGZgt{}}

    \PYG{c}{\PYGZlt{}!\PYGZhy{}\PYGZhy{}}\PYG{c}{ [Optional] Boolean to enable the complete set of taxes. By default, this field is True.}
\PYG{c}{    This boolean helps you to choose if you want to propose to the user to encode the sale and purchase rates or choose from list of taxes.}
\PYG{c}{    This last choice assumes that the set of tax defined on this template is complete. }\PYG{c}{\PYGZhy{}\PYGZhy{}\PYGZgt{}}
    \PYG{n+nt}{\PYGZlt{}field} \PYG{n+na}{name=}\PYG{l+s}{\PYGZdq{}complete\PYGZus{}tax\PYGZus{}set\PYGZdq{}} \PYG{n+na}{eval=}\PYG{l+s}{\PYGZdq{}...\PYGZdq{}}\PYG{n+nt}{/\PYGZgt{}}

    \PYG{c}{\PYGZlt{}!\PYGZhy{}\PYGZhy{}}\PYG{c}{ [Optional] Specify the spoken languages.}
\PYG{c}{    /!\PYGZbs{} This option is only available if your module depends of l10n\PYGZus{}multilang.}
\PYG{c}{    You must provide the language codes separated by \PYGZsq{};\PYGZsq{}, e.g. eval=\PYGZdq{}\PYGZsq{}en\PYGZus{}US;ar\PYGZus{}EG;ar\PYGZus{}SY\PYGZsq{}\PYGZdq{}. }\PYG{c}{\PYGZhy{}\PYGZhy{}\PYGZgt{}}
    \PYG{n+nt}{\PYGZlt{}field} \PYG{n+na}{name=}\PYG{l+s}{\PYGZdq{}spoken\PYGZus{}languages\PYGZdq{}} \PYG{n+na}{eval=}\PYG{l+s}{\PYGZdq{}...\PYGZdq{}}\PYG{n+nt}{/\PYGZgt{}}
\PYG{n+nt}{\PYGZlt{}/record\PYGZgt{}}
\end{sphinxVerbatim}

For example, let’s take a look to the Belgium chart of accounts.

\fvset{hllines={, ,}}%
\begin{sphinxVerbatim}[commandchars=\\\{\}]
\PYG{n+nt}{\PYGZlt{}record} \PYG{n+na}{id=}\PYG{l+s}{\PYGZdq{}l10nbe\PYGZus{}chart\PYGZus{}template\PYGZdq{}} \PYG{n+na}{model=}\PYG{l+s}{\PYGZdq{}account.chart.template\PYGZdq{}}\PYG{n+nt}{\PYGZgt{}}
    \PYG{n+nt}{\PYGZlt{}field} \PYG{n+na}{name=}\PYG{l+s}{\PYGZdq{}name\PYGZdq{}}\PYG{n+nt}{\PYGZgt{}}Belgian PCMN\PYG{n+nt}{\PYGZlt{}/field\PYGZgt{}}
    \PYG{n+nt}{\PYGZlt{}field} \PYG{n+na}{name=}\PYG{l+s}{\PYGZdq{}currency\PYGZus{}id\PYGZdq{}} \PYG{n+na}{ref=}\PYG{l+s}{\PYGZdq{}base.EUR\PYGZdq{}}\PYG{n+nt}{/\PYGZgt{}}
    \PYG{n+nt}{\PYGZlt{}field} \PYG{n+na}{name=}\PYG{l+s}{\PYGZdq{}bank\PYGZus{}account\PYGZus{}code\PYGZus{}prefix\PYGZdq{}}\PYG{n+nt}{\PYGZgt{}}550\PYG{n+nt}{\PYGZlt{}/field\PYGZgt{}}
    \PYG{n+nt}{\PYGZlt{}field} \PYG{n+na}{name=}\PYG{l+s}{\PYGZdq{}cash\PYGZus{}account\PYGZus{}code\PYGZus{}prefix\PYGZdq{}}\PYG{n+nt}{\PYGZgt{}}570\PYG{n+nt}{\PYGZlt{}/field\PYGZgt{}}
    \PYG{n+nt}{\PYGZlt{}field} \PYG{n+na}{name=}\PYG{l+s}{\PYGZdq{}spoken\PYGZus{}languages\PYGZdq{}} \PYG{n+na}{eval=}\PYG{l+s}{\PYGZdq{}\PYGZsq{}nl\PYGZus{}BE\PYGZsq{}\PYGZdq{}}\PYG{n+nt}{/\PYGZgt{}}
\PYG{n+nt}{\PYGZlt{}/record\PYGZgt{}}
\end{sphinxVerbatim}

Now that the chart of accounts is created, we can focus on the creation of the templates.
As said previously, each record must reference this record through the \sphinxcode{\sphinxupquote{chart\_template\_id}} field.
If not, the template will be ignored. The following sections show in details how to create these templates.


\subsubsection{Adding a new account to my Chart of Accounts}
\label{\detokenize{webservices/localization:adding-a-new-account-to-my-chart-of-accounts}}
It’s time to create our accounts. It consists to creating records of \sphinxcode{\sphinxupquote{account.account.template}} type.
Each \sphinxcode{\sphinxupquote{account.account.template}} is able to create an \sphinxcode{\sphinxupquote{account.account}} for each company.

\fvset{hllines={, ,}}%
\begin{sphinxVerbatim}[commandchars=\\\{\}]
\PYG{n+nt}{\PYGZlt{}record} \PYG{n+na}{id=}\PYG{l+s}{\PYGZdq{}...\PYGZdq{}} \PYG{n+na}{model=}\PYG{l+s}{\PYGZdq{}account.account.template\PYGZdq{}}\PYG{n+nt}{\PYGZgt{}}
    \PYG{c}{\PYGZlt{}!\PYGZhy{}\PYGZhy{}}\PYG{c}{ [Required] Specify the name to display for this account. }\PYG{c}{\PYGZhy{}\PYGZhy{}\PYGZgt{}}
    \PYG{n+nt}{\PYGZlt{}field} \PYG{n+na}{name=}\PYG{l+s}{\PYGZdq{}name\PYGZdq{}}\PYG{n+nt}{\PYGZgt{}}...\PYG{n+nt}{\PYGZlt{}/field\PYGZgt{}}

    \PYG{c}{\PYGZlt{}!\PYGZhy{}\PYGZhy{}}\PYG{c}{ [Required] Specify a code. }\PYG{c}{\PYGZhy{}\PYGZhy{}\PYGZgt{}}
    \PYG{n+nt}{\PYGZlt{}field} \PYG{n+na}{name=}\PYG{l+s}{\PYGZdq{}code\PYGZdq{}}\PYG{n+nt}{\PYGZgt{}}...\PYG{n+nt}{\PYGZlt{}/field\PYGZgt{}}

    \PYG{c}{\PYGZlt{}!\PYGZhy{}\PYGZhy{}}\PYG{c}{ [Required] Specify a type. }\PYG{c}{\PYGZhy{}\PYGZhy{}\PYGZgt{}}
    \PYG{n+nt}{\PYGZlt{}field} \PYG{n+na}{name=}\PYG{l+s}{\PYGZdq{}user\PYGZus{}type\PYGZus{}id\PYGZdq{}}\PYG{n+nt}{\PYGZgt{}}...\PYG{n+nt}{\PYGZlt{}/field\PYGZgt{}}

    \PYG{c}{\PYGZlt{}!\PYGZhy{}\PYGZhy{}}\PYG{c}{ [Required] Set the CoA owning this account. }\PYG{c}{\PYGZhy{}\PYGZhy{}\PYGZgt{}}
    \PYG{n+nt}{\PYGZlt{}field} \PYG{n+na}{name=}\PYG{l+s}{\PYGZdq{}chart\PYGZus{}template\PYGZus{}id\PYGZdq{}} \PYG{n+na}{ref=}\PYG{l+s}{\PYGZdq{}...\PYGZdq{}}\PYG{n+nt}{/\PYGZgt{}}

    \PYG{c}{\PYGZlt{}!\PYGZhy{}\PYGZhy{}}\PYG{c}{ [Optional] Specify a secondary currency for each account.move.line linked to this account. }\PYG{c}{\PYGZhy{}\PYGZhy{}\PYGZgt{}}
    \PYG{n+nt}{\PYGZlt{}field} \PYG{n+na}{name=}\PYG{l+s}{\PYGZdq{}currency\PYGZus{}id\PYGZdq{}} \PYG{n+na}{ref=}\PYG{l+s}{\PYGZdq{}...\PYGZdq{}}\PYG{n+nt}{/\PYGZgt{}}

    \PYG{c}{\PYGZlt{}!\PYGZhy{}\PYGZhy{}}\PYG{c}{ [Optional] Boolean to allow the user to reconcile entries in this account. True by default. }\PYG{c}{\PYGZhy{}\PYGZhy{}\PYGZgt{}}
    \PYG{n+nt}{\PYGZlt{}field} \PYG{n+na}{name=}\PYG{l+s}{\PYGZdq{}reconcile\PYGZdq{}} \PYG{n+na}{eval=}\PYG{l+s}{\PYGZdq{}...\PYGZdq{}}\PYG{n+nt}{/\PYGZgt{}}

    \PYG{c}{\PYGZlt{}!\PYGZhy{}\PYGZhy{}}\PYG{c}{ [Optional] Specify a group for this account. }\PYG{c}{\PYGZhy{}\PYGZhy{}\PYGZgt{}}
    \PYG{n+nt}{\PYGZlt{}field} \PYG{n+na}{name=}\PYG{l+s}{\PYGZdq{}group\PYGZus{}id\PYGZdq{}} \PYG{n+na}{ref=}\PYG{l+s}{\PYGZdq{}...\PYGZdq{}}\PYG{n+nt}{\PYGZgt{}}

    \PYG{c}{\PYGZlt{}!\PYGZhy{}\PYGZhy{}}\PYG{c}{ [Optional] Specify some tags. }\PYG{c}{\PYGZhy{}\PYGZhy{}\PYGZgt{}}
    \PYG{n+nt}{\PYGZlt{}field} \PYG{n+na}{name=}\PYG{l+s}{\PYGZdq{}tag\PYGZus{}ids\PYGZdq{}} \PYG{n+na}{eval=}\PYG{l+s}{\PYGZdq{}...\PYGZdq{}}\PYG{n+nt}{\PYGZgt{}}
\PYG{n+nt}{\PYGZlt{}/record\PYGZgt{}}
\end{sphinxVerbatim}

Some of the described fields above deserve a bit more explanation.

The \sphinxcode{\sphinxupquote{user\_type\_id}} field requires a value of type \sphinxcode{\sphinxupquote{account.account.type}}.
Although some additional types could be created in a localization module, we encourage the usage of the existing types in the \sphinxhref{https://github.com/odoo/odoo/blob/11.0/addons/account/data/data\_account\_type.xml}{account/data/data\_account\_type.xml} file.
The usage of these generic types ensures the generic reports working correctly in addition to those that you could create in your localization module.

\begin{sphinxadmonition}{warning}{Warning:}
Avoid the usage of liquidity \sphinxcode{\sphinxupquote{account.account.type}}!
Indeed, the bank \& cash accounts are created directly at the installation of the localization module and then, are linked to an \sphinxcode{\sphinxupquote{account.journal}}.
\end{sphinxadmonition}

\begin{sphinxadmonition}{warning}{Warning:}
Only one account of type payable/receivable is enough.
\end{sphinxadmonition}

Although the \sphinxcode{\sphinxupquote{tag\_ids}} field is optional, this one remains a very powerful feature.
Indeed, this one allows you to define some tags for your accounts to spread them correctly on your reports.
For example, suppose you want to create a financial report having multiple lines but you have no way to find a rule to dispatch the accounts according their \sphinxcode{\sphinxupquote{code}} or \sphinxcode{\sphinxupquote{name}}.
The solution is the usage of tags, one for each report line, to spread and aggregate your accounts like you want.

Like any other record, a tag can be created with the following xml structure:

\fvset{hllines={, ,}}%
\begin{sphinxVerbatim}[commandchars=\\\{\}]
\PYG{n+nt}{\PYGZlt{}record} \PYG{n+na}{id=}\PYG{l+s}{\PYGZdq{}...\PYGZdq{}} \PYG{n+na}{model=}\PYG{l+s}{\PYGZdq{}account.account.tag\PYGZdq{}}\PYG{n+nt}{\PYGZgt{}}
    \PYG{c}{\PYGZlt{}!\PYGZhy{}\PYGZhy{}}\PYG{c}{ [Required] Specify the name to display for this tag. }\PYG{c}{\PYGZhy{}\PYGZhy{}\PYGZgt{}}
    \PYG{n+nt}{\PYGZlt{}field} \PYG{n+na}{name=}\PYG{l+s}{\PYGZdq{}name\PYGZdq{}}\PYG{n+nt}{\PYGZgt{}}...\PYG{n+nt}{\PYGZlt{}/field\PYGZgt{}}

    \PYG{c}{\PYGZlt{}!\PYGZhy{}\PYGZhy{}}\PYG{c}{ [Optional] Define a scope for this applicability.}
\PYG{c}{    The available keys are \PYGZsq{}accounts\PYGZsq{} and \PYGZsq{}taxes\PYGZsq{} but \PYGZsq{}accounts\PYGZsq{} is the default value. }\PYG{c}{\PYGZhy{}\PYGZhy{}\PYGZgt{}}
    \PYG{n+nt}{\PYGZlt{}field} \PYG{n+na}{name=}\PYG{l+s}{\PYGZdq{}applicability\PYGZdq{}}\PYG{n+nt}{\PYGZgt{}}...\PYG{n+nt}{\PYGZlt{}/field\PYGZgt{}}
\PYG{n+nt}{\PYGZlt{}/record\PYGZgt{}}
\end{sphinxVerbatim}

As you can well imagine with the usage of tags, this feature can also be used with taxes.

An examples coming from the \sphinxcode{\sphinxupquote{l10n\_be}} module:

\fvset{hllines={, ,}}%
\begin{sphinxVerbatim}[commandchars=\\\{\}]
\PYG{n+nt}{\PYGZlt{}record} \PYG{n+na}{id=}\PYG{l+s}{\PYGZdq{}a4000\PYGZdq{}} \PYG{n+na}{model=}\PYG{l+s}{\PYGZdq{}account.account.template\PYGZdq{}}\PYG{n+nt}{\PYGZgt{}}
    \PYG{n+nt}{\PYGZlt{}field} \PYG{n+na}{name=}\PYG{l+s}{\PYGZdq{}name\PYGZdq{}}\PYG{n+nt}{\PYGZgt{}}Clients\PYG{n+nt}{\PYGZlt{}/field\PYGZgt{}}
    \PYG{n+nt}{\PYGZlt{}field} \PYG{n+na}{name=}\PYG{l+s}{\PYGZdq{}code\PYGZdq{}}\PYG{n+nt}{\PYGZgt{}}4000\PYG{n+nt}{\PYGZlt{}/field\PYGZgt{}}
    \PYG{n+nt}{\PYGZlt{}field} \PYG{n+na}{name=}\PYG{l+s}{\PYGZdq{}user\PYGZus{}type\PYGZus{}id\PYGZdq{}} \PYG{n+na}{ref=}\PYG{l+s}{\PYGZdq{}account.data\PYGZus{}account\PYGZus{}type\PYGZus{}receivable\PYGZdq{}}\PYG{n+nt}{/\PYGZgt{}}
    \PYG{n+nt}{\PYGZlt{}field} \PYG{n+na}{name=}\PYG{l+s}{\PYGZdq{}chart\PYGZus{}template\PYGZus{}id\PYGZdq{}} \PYG{n+na}{ref=}\PYG{l+s}{\PYGZdq{}l10nbe\PYGZus{}chart\PYGZus{}template\PYGZdq{}}\PYG{n+nt}{/\PYGZgt{}}
\PYG{n+nt}{\PYGZlt{}/record\PYGZgt{}}
\end{sphinxVerbatim}

\begin{sphinxadmonition}{warning}{Warning:}
Don’t create too much accounts: 200-300 is enough.
\end{sphinxadmonition}


\subsubsection{Adding a new tax to my Chart of Accounts}
\label{\detokenize{webservices/localization:adding-a-new-tax-to-my-chart-of-accounts}}
To create a new tax record, you just need to follow the same process as the creation of accounts.
The only difference being that you must use the \sphinxcode{\sphinxupquote{account.tax.template}} model.

\fvset{hllines={, ,}}%
\begin{sphinxVerbatim}[commandchars=\\\{\}]
\PYG{n+nt}{\PYGZlt{}record} \PYG{n+na}{id=}\PYG{l+s}{\PYGZdq{}...\PYGZdq{}} \PYG{n+na}{model=}\PYG{l+s}{\PYGZdq{}account.tax.template\PYGZdq{}}\PYG{n+nt}{\PYGZgt{}}
    \PYG{c}{\PYGZlt{}!\PYGZhy{}\PYGZhy{}}\PYG{c}{ [Required] Specify the name to display for this tax. }\PYG{c}{\PYGZhy{}\PYGZhy{}\PYGZgt{}}
    \PYG{n+nt}{\PYGZlt{}field} \PYG{n+na}{name=}\PYG{l+s}{\PYGZdq{}name\PYGZdq{}}\PYG{n+nt}{\PYGZgt{}}...\PYG{n+nt}{\PYGZlt{}/field\PYGZgt{}}

    \PYG{c}{\PYGZlt{}!\PYGZhy{}\PYGZhy{}}\PYG{c}{ [Required] Specify the amount.}
\PYG{c}{    E.g. 7 with fixed amount\PYGZus{}type means v + 7 if v is the amount on which the tax is applied.}
\PYG{c}{     If amount\PYGZus{}type is \PYGZsq{}percent\PYGZsq{}, the tax amount is v * 0.07. }\PYG{c}{\PYGZhy{}\PYGZhy{}\PYGZgt{}}
    \PYG{n+nt}{\PYGZlt{}field} \PYG{n+na}{name=}\PYG{l+s}{\PYGZdq{}amount\PYGZdq{}} \PYG{n+na}{eval=}\PYG{l+s}{\PYGZdq{}...\PYGZdq{}}\PYG{n+nt}{/\PYGZgt{}}

    \PYG{c}{\PYGZlt{}!\PYGZhy{}\PYGZhy{}}\PYG{c}{ [Required] Set the CoA owning this tax. }\PYG{c}{\PYGZhy{}\PYGZhy{}\PYGZgt{}}
    \PYG{n+nt}{\PYGZlt{}field} \PYG{n+na}{name=}\PYG{l+s}{\PYGZdq{}chart\PYGZus{}template\PYGZus{}id\PYGZdq{}} \PYG{n+na}{ref=}\PYG{l+s}{\PYGZdq{}...\PYGZdq{}}\PYG{n+nt}{/\PYGZgt{}}

    \PYG{c}{\PYGZlt{}!\PYGZhy{}\PYGZhy{}}\PYG{c}{ [Required/Optional] Define an account if the tax is not a group of taxes. }\PYG{c}{\PYGZhy{}\PYGZhy{}\PYGZgt{}}
    \PYG{n+nt}{\PYGZlt{}field} \PYG{n+na}{name=}\PYG{l+s}{\PYGZdq{}account\PYGZus{}id\PYGZdq{}} \PYG{n+na}{ref=}\PYG{l+s}{\PYGZdq{}...\PYGZdq{}}\PYG{n+nt}{/\PYGZgt{}}

    \PYG{c}{\PYGZlt{}!\PYGZhy{}\PYGZhy{}}\PYG{c}{ [Required/Optional] Define an refund account if the tax is not a group of taxes. }\PYG{c}{\PYGZhy{}\PYGZhy{}\PYGZgt{}}
    \PYG{n+nt}{\PYGZlt{}field} \PYG{n+na}{name=}\PYG{l+s}{\PYGZdq{}refund\PYGZus{}account\PYGZus{}id\PYGZdq{}} \PYG{n+na}{ref=}\PYG{l+s}{\PYGZdq{}...\PYGZdq{}}\PYG{n+nt}{/\PYGZgt{}}

    \PYG{c}{\PYGZlt{}!\PYGZhy{}\PYGZhy{}}\PYG{c}{ [Optional] Define the tax\PYGZsq{}s type.}
\PYG{c}{    \PYGZsq{}sale\PYGZsq{}, \PYGZsq{}purchase\PYGZsq{} or \PYGZsq{}none\PYGZsq{} are the allowed values. \PYGZsq{}sale\PYGZsq{} is the default value.}
\PYG{c}{    \PYGZsq{}adjustment\PYGZsq{} is also available to do some tax adjustments.}
\PYG{c}{    Note: \PYGZsq{}none\PYGZsq{} means a tax can\PYGZsq{}t be used by itself, however it can still be used in a group. }\PYG{c}{\PYGZhy{}\PYGZhy{}\PYGZgt{}}
    \PYG{n+nt}{\PYGZlt{}field} \PYG{n+na}{name=}\PYG{l+s}{\PYGZdq{}type\PYGZus{}tax\PYGZus{}use\PYGZdq{}}\PYG{n+nt}{\PYGZgt{}}...\PYG{n+nt}{\PYGZlt{}/field\PYGZgt{}}

    \PYG{c}{\PYGZlt{}!\PYGZhy{}\PYGZhy{}}\PYG{c}{ [Optional] Define the type of amount:}
\PYG{c}{    \PYGZsq{}group\PYGZsq{} for a group of taxes, \PYGZsq{}fixed\PYGZsq{} for a tax with a fixed amount or \PYGZsq{}percent\PYGZsq{} for a classic percentage of price.}
\PYG{c}{    By default, the type of amount is percentage. }\PYG{c}{\PYGZhy{}\PYGZhy{}\PYGZgt{}}
    \PYG{n+nt}{\PYGZlt{}field} \PYG{n+na}{name=}\PYG{l+s}{\PYGZdq{}amount\PYGZus{}type\PYGZdq{}} \PYG{n+na}{eval=}\PYG{l+s}{\PYGZdq{}...\PYGZdq{}}\PYG{n+nt}{/\PYGZgt{}}

    \PYG{c}{\PYGZlt{}!\PYGZhy{}\PYGZhy{}}\PYG{c}{ [Optional] Define some children taxes.}
\PYG{c}{    /!\PYGZbs{} Should be used only with an amount\PYGZus{}type with \PYGZsq{}group\PYGZsq{} set. }\PYG{c}{\PYGZhy{}\PYGZhy{}\PYGZgt{}}
    \PYG{n+nt}{\PYGZlt{}field} \PYG{n+na}{name=}\PYG{l+s}{\PYGZdq{}children\PYGZus{}tax\PYGZus{}ids\PYGZdq{}} \PYG{n+na}{eval=}\PYG{l+s}{\PYGZdq{}...\PYGZdq{}}\PYG{n+nt}{/\PYGZgt{}}

    \PYG{c}{\PYGZlt{}!\PYGZhy{}\PYGZhy{}}\PYG{c}{ [Optional] The sequence field is used to define order in which the tax lines are applied.}
\PYG{c}{    By default, sequence = 1. }\PYG{c}{\PYGZhy{}\PYGZhy{}\PYGZgt{}}
    \PYG{n+nt}{\PYGZlt{}field} \PYG{n+na}{name=}\PYG{l+s}{\PYGZdq{}sequence\PYGZdq{}} \PYG{n+na}{eval=}\PYG{l+s}{\PYGZdq{}...\PYGZdq{}}\PYG{n+nt}{/\PYGZgt{}}

    \PYG{c}{\PYGZlt{}!\PYGZhy{}\PYGZhy{}}\PYG{c}{ [Optional] Specify a short text to be displayed on invoices.}
\PYG{c}{    For example, a tax named \PYGZdq{}15\PYGZpc{} on Services\PYGZdq{} can have the following label on invoice \PYGZdq{}15\PYGZpc{}\PYGZdq{}. }\PYG{c}{\PYGZhy{}\PYGZhy{}\PYGZgt{}}
    \PYG{n+nt}{\PYGZlt{}field} \PYG{n+na}{name=}\PYG{l+s}{\PYGZdq{}description\PYGZdq{}}\PYG{n+nt}{\PYGZgt{}}...\PYG{n+nt}{\PYGZlt{}/field\PYGZgt{}}

    \PYG{c}{\PYGZlt{}!\PYGZhy{}\PYGZhy{}}\PYG{c}{ [Optional] Boolean that indicates if the amount should be considered as included in price. False by default.}
\PYG{c}{    E.g. Suppose v = 132 and a tax amount of 20.}
\PYG{c}{    If price\PYGZus{}include = False, the computed amount will be 132 * 0.2 = 26.4.}
\PYG{c}{    If price\PYGZus{}include = True, the computed amount will be 132 }\PYG{c}{\PYGZhy{}}\PYG{c}{ (132 / 1.2) = 132 }\PYG{c}{\PYGZhy{}}\PYG{c}{ 110 = 22. }\PYG{c}{\PYGZhy{}\PYGZhy{}\PYGZgt{}}
    \PYG{n+nt}{\PYGZlt{}field} \PYG{n+na}{name=}\PYG{l+s}{\PYGZdq{}price\PYGZus{}include\PYGZdq{}} \PYG{n+na}{eval=}\PYG{l+s}{\PYGZdq{}...\PYGZdq{}}\PYG{n+nt}{/\PYGZgt{}}

    \PYG{c}{\PYGZlt{}!\PYGZhy{}\PYGZhy{}}\PYG{c}{ [Optional] Boolean to set to include the amount to the base. False by default.}
\PYG{c}{     If True, the subsequent taxes will be computed based on the base tax amount plus the amount of this tax.}
\PYG{c}{     E.g. suppose v = 100, t1, a tax of 10\PYGZpc{} and another tax t2 with 20\PYGZpc{}.}
\PYG{c}{     If t1 doesn\PYGZsq{}t affects the base,}
\PYG{c}{     t1 amount = 100 * 0.1 = 10 and t2 amount = 100 * 0.2 = 20.}
\PYG{c}{     If t1 affects the base,}
\PYG{c}{     t1 amount = 100 * 0.1 = 10 and t2 amount = 110 * 0.2 = 22.  }\PYG{c}{\PYGZhy{}\PYGZhy{}\PYGZgt{}}
    \PYG{n+nt}{\PYGZlt{}field} \PYG{n+na}{name=}\PYG{l+s}{\PYGZdq{}include\PYGZus{}base\PYGZus{}amount\PYGZdq{}} \PYG{n+na}{eval=}\PYG{l+s}{\PYGZdq{}...\PYGZdq{}}\PYG{n+nt}{/\PYGZgt{}}

    \PYG{c}{\PYGZlt{}!\PYGZhy{}\PYGZhy{}}\PYG{c}{ [Optional] Boolean false by default.}
\PYG{c}{     If set, the amount computed by this tax will be assigned to the same analytic account as the invoice line (if any). }\PYG{c}{\PYGZhy{}\PYGZhy{}\PYGZgt{}}
    \PYG{n+nt}{\PYGZlt{}field} \PYG{n+na}{name=}\PYG{l+s}{\PYGZdq{}analytic\PYGZdq{}} \PYG{n+na}{eval=}\PYG{l+s}{\PYGZdq{}...\PYGZdq{}}\PYG{n+nt}{/\PYGZgt{}}

    \PYG{c}{\PYGZlt{}!\PYGZhy{}\PYGZhy{}}\PYG{c}{ [Optional] Specify some tags.}
\PYG{c}{    These tags must have \PYGZsq{}taxes\PYGZsq{} as applicability.}
\PYG{c}{    See the previous section for more details. }\PYG{c}{\PYGZhy{}\PYGZhy{}\PYGZgt{}}
    \PYG{n+nt}{\PYGZlt{}field} \PYG{n+na}{name=}\PYG{l+s}{\PYGZdq{}tag\PYGZus{}ids\PYGZdq{}} \PYG{n+na}{eval=}\PYG{l+s}{\PYGZdq{}...\PYGZdq{}}\PYG{n+nt}{\PYGZgt{}}

    \PYG{c}{\PYGZlt{}!\PYGZhy{}\PYGZhy{}}\PYG{c}{ [Optional] Define a tax group used to display the sums of taxes in the invoices. }\PYG{c}{\PYGZhy{}\PYGZhy{}\PYGZgt{}}
    \PYG{n+nt}{\PYGZlt{}field} \PYG{n+na}{name=}\PYG{l+s}{\PYGZdq{}tax\PYGZus{}group\PYGZus{}id\PYGZdq{}} \PYG{n+na}{ref=}\PYG{l+s}{\PYGZdq{}...\PYGZdq{}}\PYG{n+nt}{/\PYGZgt{}}

    \PYG{c}{\PYGZlt{}!\PYGZhy{}\PYGZhy{}}\PYG{c}{ [Optional] Define the tax exigibility either based on invoice (\PYGZsq{}on\PYGZus{}invoice\PYGZsq{} value) or}
\PYG{c}{    either based on payment using the \PYGZsq{}on\PYGZus{}payment\PYGZsq{} key.}
\PYG{c}{    The default value is \PYGZsq{}on\PYGZus{}invoice\PYGZsq{}. }\PYG{c}{\PYGZhy{}\PYGZhy{}\PYGZgt{}}
    \PYG{n+nt}{\PYGZlt{}field} \PYG{n+na}{name=}\PYG{l+s}{\PYGZdq{}tax\PYGZus{}exigibility\PYGZdq{}}\PYG{n+nt}{\PYGZgt{}}...\PYG{n+nt}{\PYGZlt{}/field\PYGZgt{}}

    \PYG{c}{\PYGZlt{}!\PYGZhy{}\PYGZhy{}}\PYG{c}{ [Optional] Define a cash basis account in case of tax exigibility \PYGZsq{}on\PYGZus{}payment\PYGZsq{}. }\PYG{c}{\PYGZhy{}\PYGZhy{}\PYGZgt{}}
    \PYG{n+nt}{\PYGZlt{}field} \PYG{n+na}{name=}\PYG{l+s}{\PYGZdq{}cash\PYGZus{}basis\PYGZus{}account\PYGZdq{}} \PYG{n+na}{red=}\PYG{l+s}{\PYGZdq{}...\PYGZdq{}}\PYG{n+nt}{/\PYGZgt{}}
\PYG{n+nt}{\PYGZlt{}/record\PYGZgt{}}
\end{sphinxVerbatim}

An example found in the \sphinxcode{\sphinxupquote{l10n\_pl}} module:

\fvset{hllines={, ,}}%
\begin{sphinxVerbatim}[commandchars=\\\{\}]
\PYG{n+nt}{\PYGZlt{}record} \PYG{n+na}{id=}\PYG{l+s}{\PYGZdq{}vp\PYGZus{}leas\PYGZus{}sale\PYGZdq{}} \PYG{n+na}{model=}\PYG{l+s}{\PYGZdq{}account.tax.template\PYGZdq{}}\PYG{n+nt}{\PYGZgt{}}
    \PYG{n+nt}{\PYGZlt{}field} \PYG{n+na}{name=}\PYG{l+s}{\PYGZdq{}chart\PYGZus{}template\PYGZus{}id\PYGZdq{}} \PYG{n+na}{ref=}\PYG{l+s}{\PYGZdq{}pl\PYGZus{}chart\PYGZus{}template\PYGZdq{}}\PYG{n+nt}{/\PYGZgt{}}
    \PYG{n+nt}{\PYGZlt{}field} \PYG{n+na}{name=}\PYG{l+s}{\PYGZdq{}name\PYGZdq{}}\PYG{n+nt}{\PYGZgt{}}VAT \PYGZhy{} leasing pojazdu(sale)\PYG{n+nt}{\PYGZlt{}/field\PYGZgt{}}
    \PYG{n+nt}{\PYGZlt{}field} \PYG{n+na}{name=}\PYG{l+s}{\PYGZdq{}description\PYGZdq{}}\PYG{n+nt}{\PYGZgt{}}VLP\PYG{n+nt}{\PYGZlt{}/field\PYGZgt{}}
    \PYG{n+nt}{\PYGZlt{}field} \PYG{n+na}{name=}\PYG{l+s}{\PYGZdq{}amount\PYGZdq{}}\PYG{n+nt}{\PYGZgt{}}1.00\PYG{n+nt}{\PYGZlt{}/field\PYGZgt{}}
    \PYG{n+nt}{\PYGZlt{}field} \PYG{n+na}{name=}\PYG{l+s}{\PYGZdq{}sequence\PYGZdq{}} \PYG{n+na}{eval=}\PYG{l+s}{\PYGZdq{}1\PYGZdq{}}\PYG{n+nt}{/\PYGZgt{}}
    \PYG{n+nt}{\PYGZlt{}field} \PYG{n+na}{name=}\PYG{l+s}{\PYGZdq{}amount\PYGZus{}type\PYGZdq{}}\PYG{n+nt}{\PYGZgt{}}group\PYG{n+nt}{\PYGZlt{}/field\PYGZgt{}}
    \PYG{n+nt}{\PYGZlt{}field} \PYG{n+na}{name=}\PYG{l+s}{\PYGZdq{}type\PYGZus{}tax\PYGZus{}use\PYGZdq{}}\PYG{n+nt}{\PYGZgt{}}sale\PYG{n+nt}{\PYGZlt{}/field\PYGZgt{}}
    \PYG{n+nt}{\PYGZlt{}field} \PYG{n+na}{name=}\PYG{l+s}{\PYGZdq{}children\PYGZus{}tax\PYGZus{}ids\PYGZdq{}} \PYG{n+na}{eval=}\PYG{l+s}{\PYGZdq{}[(6, 0, [ref(\PYGZsq{}vp\PYGZus{}leas\PYGZus{}sale\PYGZus{}1\PYGZsq{}), ref(\PYGZsq{}vp\PYGZus{}leas\PYGZus{}sale\PYGZus{}2\PYGZsq{})])]\PYGZdq{}}\PYG{n+nt}{/\PYGZgt{}}
    \PYG{n+nt}{\PYGZlt{}field} \PYG{n+na}{name=}\PYG{l+s}{\PYGZdq{}tag\PYGZus{}ids\PYGZdq{}} \PYG{n+na}{eval=}\PYG{l+s}{\PYGZdq{}[(6,0,[ref(\PYGZsq{}l10n\PYGZus{}pl.tag\PYGZus{}pl\PYGZus{}21\PYGZsq{})])]\PYGZdq{}}\PYG{n+nt}{/\PYGZgt{}}
    \PYG{n+nt}{\PYGZlt{}field} \PYG{n+na}{name=}\PYG{l+s}{\PYGZdq{}tax\PYGZus{}group\PYGZus{}id\PYGZdq{}} \PYG{n+na}{ref=}\PYG{l+s}{\PYGZdq{}tax\PYGZus{}group\PYGZus{}vat\PYGZus{}23\PYGZdq{}}\PYG{n+nt}{/\PYGZgt{}}
\PYG{n+nt}{\PYGZlt{}/record\PYGZgt{}}
\end{sphinxVerbatim}


\subsubsection{Adding a new fiscal position to my Chart of Accounts}
\label{\detokenize{webservices/localization:adding-a-new-fiscal-position-to-my-chart-of-accounts}}
\begin{sphinxadmonition}{note}{Note:}
If you need more information about what is a fiscal position and how it works in Odoo, please refer to \sphinxhref{https://www.odoo.com/documentation/user/online/accounting/others/taxes/application.html}{How to adapt taxes to my customer status or localization}.
\end{sphinxadmonition}

To create a new fiscal position, simply use the \sphinxcode{\sphinxupquote{account.fiscal.position.template}} model:

\fvset{hllines={, ,}}%
\begin{sphinxVerbatim}[commandchars=\\\{\}]
\PYG{n+nt}{\PYGZlt{}record} \PYG{n+na}{id=}\PYG{l+s}{\PYGZdq{}...\PYGZdq{}} \PYG{n+na}{model=}\PYG{l+s}{\PYGZdq{}account.fiscal.position.template\PYGZdq{}}\PYG{n+nt}{\PYGZgt{}}
    \PYG{c}{\PYGZlt{}!\PYGZhy{}\PYGZhy{}}\PYG{c}{ [Required] Specify the name to display for this fiscal position. }\PYG{c}{\PYGZhy{}\PYGZhy{}\PYGZgt{}}
    \PYG{n+nt}{\PYGZlt{}field} \PYG{n+na}{name=}\PYG{l+s}{\PYGZdq{}name\PYGZdq{}}\PYG{n+nt}{\PYGZgt{}}...\PYG{n+nt}{\PYGZlt{}/field\PYGZgt{}}

    \PYG{c}{\PYGZlt{}!\PYGZhy{}\PYGZhy{}}\PYG{c}{ [Required] Set the CoA owning this fiscal position. }\PYG{c}{\PYGZhy{}\PYGZhy{}\PYGZgt{}}
    \PYG{n+nt}{\PYGZlt{}field} \PYG{n+na}{name=}\PYG{l+s}{\PYGZdq{}chart\PYGZus{}template\PYGZus{}id\PYGZdq{}} \PYG{n+na}{ref=}\PYG{l+s}{\PYGZdq{}...\PYGZdq{}}\PYG{n+nt}{/\PYGZgt{}}

    \PYG{c}{\PYGZlt{}!\PYGZhy{}\PYGZhy{}}\PYG{c}{ [Optional] Add some additional notes. }\PYG{c}{\PYGZhy{}\PYGZhy{}\PYGZgt{}}
    \PYG{n+nt}{\PYGZlt{}field} \PYG{n+na}{name=}\PYG{l+s}{\PYGZdq{}note\PYGZdq{}}\PYG{n+nt}{\PYGZgt{}}...\PYG{n+nt}{\PYGZlt{}/field\PYGZgt{}}
\PYG{n+nt}{\PYGZlt{}/record\PYGZgt{}}
\end{sphinxVerbatim}


\subsubsection{Adding the properties to my Chart of Accounts}
\label{\detokenize{webservices/localization:adding-the-properties-to-my-chart-of-accounts}}
When the whole accounts are generated, you have the possibility to override the newly generated chart of accounts by adding some properties that correspond to default accounts used in certain situations.
This must be done after the creation of accounts before each one must be linked to the chart of accounts.

\fvset{hllines={, ,}}%
\begin{sphinxVerbatim}[commandchars=\\\{\}]
\PYG{c+cp}{\PYGZlt{}?xml version=\PYGZdq{}1.0\PYGZdq{} encoding=\PYGZdq{}utf\PYGZhy{}8\PYGZdq{}?\PYGZgt{}}
\PYG{n+nt}{\PYGZlt{}odoo}\PYG{n+nt}{\PYGZgt{}}
    \PYG{n+nt}{\PYGZlt{}record} \PYG{n+na}{id=}\PYG{l+s}{\PYGZdq{}l10n\PYGZus{}xx\PYGZus{}chart\PYGZus{}template\PYGZdq{}} \PYG{n+na}{model=}\PYG{l+s}{\PYGZdq{}account.chart.template\PYGZdq{}}\PYG{n+nt}{\PYGZgt{}}

        \PYG{c}{\PYGZlt{}!\PYGZhy{}\PYGZhy{}}\PYG{c}{ Define receivable/payable accounts. }\PYG{c}{\PYGZhy{}\PYGZhy{}\PYGZgt{}}
        \PYG{n+nt}{\PYGZlt{}field} \PYG{n+na}{name=}\PYG{l+s}{\PYGZdq{}property\PYGZus{}account\PYGZus{}receivable\PYGZus{}id\PYGZdq{}} \PYG{n+na}{ref=}\PYG{l+s}{\PYGZdq{}...\PYGZdq{}}\PYG{n+nt}{/\PYGZgt{}}
        \PYG{n+nt}{\PYGZlt{}field} \PYG{n+na}{name=}\PYG{l+s}{\PYGZdq{}property\PYGZus{}account\PYGZus{}payable\PYGZus{}id\PYGZdq{}} \PYG{n+na}{ref=}\PYG{l+s}{\PYGZdq{}...\PYGZdq{}}\PYG{n+nt}{/\PYGZgt{}}

        \PYG{c}{\PYGZlt{}!\PYGZhy{}\PYGZhy{}}\PYG{c}{ Define categories of expense/income account. }\PYG{c}{\PYGZhy{}\PYGZhy{}\PYGZgt{}}
        \PYG{n+nt}{\PYGZlt{}field} \PYG{n+na}{name=}\PYG{l+s}{\PYGZdq{}property\PYGZus{}account\PYGZus{}expense\PYGZus{}categ\PYGZus{}id\PYGZdq{}} \PYG{n+na}{ref=}\PYG{l+s}{\PYGZdq{}...\PYGZdq{}}\PYG{n+nt}{/\PYGZgt{}}
        \PYG{n+nt}{\PYGZlt{}field} \PYG{n+na}{name=}\PYG{l+s}{\PYGZdq{}property\PYGZus{}account\PYGZus{}income\PYGZus{}categ\PYGZus{}id\PYGZdq{}} \PYG{n+na}{ref=}\PYG{l+s}{\PYGZdq{}...\PYGZdq{}}\PYG{n+nt}{/\PYGZgt{}}

        \PYG{c}{\PYGZlt{}!\PYGZhy{}\PYGZhy{}}\PYG{c}{ Define input/output accounts for stock valuation. }\PYG{c}{\PYGZhy{}\PYGZhy{}\PYGZgt{}}
        \PYG{n+nt}{\PYGZlt{}field} \PYG{n+na}{name=}\PYG{l+s}{\PYGZdq{}property\PYGZus{}stock\PYGZus{}account\PYGZus{}input\PYGZus{}categ\PYGZus{}id\PYGZdq{}} \PYG{n+na}{ref=}\PYG{l+s}{\PYGZdq{}...\PYGZdq{}}\PYG{n+nt}{/\PYGZgt{}}
        \PYG{n+nt}{\PYGZlt{}field} \PYG{n+na}{name=}\PYG{l+s}{\PYGZdq{}property\PYGZus{}stock\PYGZus{}account\PYGZus{}output\PYGZus{}categ\PYGZus{}id\PYGZdq{}} \PYG{n+na}{ref=}\PYG{l+s}{\PYGZdq{}...\PYGZdq{}}\PYG{n+nt}{/\PYGZgt{}}

        \PYG{c}{\PYGZlt{}!\PYGZhy{}\PYGZhy{}}\PYG{c}{ Define an account template for stock valuation. }\PYG{c}{\PYGZhy{}\PYGZhy{}\PYGZgt{}}
        \PYG{n+nt}{\PYGZlt{}field} \PYG{n+na}{name=}\PYG{l+s}{\PYGZdq{}property\PYGZus{}stock\PYGZus{}valuation\PYGZus{}account\PYGZus{}id\PYGZdq{}} \PYG{n+na}{ref=}\PYG{l+s}{\PYGZdq{}...\PYGZdq{}}\PYG{n+nt}{/\PYGZgt{}}

        \PYG{c}{\PYGZlt{}!\PYGZhy{}\PYGZhy{}}\PYG{c}{ Define loss/gain exchange rate accounts. }\PYG{c}{\PYGZhy{}\PYGZhy{}\PYGZgt{}}
        \PYG{n+nt}{\PYGZlt{}field} \PYG{n+na}{name=}\PYG{l+s}{\PYGZdq{}expense\PYGZus{}currency\PYGZus{}exchange\PYGZus{}account\PYGZus{}id\PYGZdq{}} \PYG{n+na}{ref=}\PYG{l+s}{\PYGZdq{}...\PYGZdq{}}\PYG{n+nt}{/\PYGZgt{}}
        \PYG{n+nt}{\PYGZlt{}field} \PYG{n+na}{name=}\PYG{l+s}{\PYGZdq{}income\PYGZus{}currency\PYGZus{}exchange\PYGZus{}account\PYGZus{}id\PYGZdq{}} \PYG{n+na}{ref=}\PYG{l+s}{\PYGZdq{}...\PYGZdq{}}\PYG{n+nt}{/\PYGZgt{}}

        \PYG{c}{\PYGZlt{}!\PYGZhy{}\PYGZhy{}}\PYG{c}{ Define a transfer account. }\PYG{c}{\PYGZhy{}\PYGZhy{}\PYGZgt{}}
        \PYG{n+nt}{\PYGZlt{}field} \PYG{n+na}{name=}\PYG{l+s}{\PYGZdq{}transfer\PYGZus{}account\PYGZus{}id\PYGZdq{}} \PYG{n+na}{ref=}\PYG{l+s}{\PYGZdq{}...\PYGZdq{}}\PYG{n+nt}{/\PYGZgt{}}
    \PYG{n+nt}{\PYGZlt{}/record\PYGZgt{}}
\PYG{n+nt}{\PYGZlt{}/odoo\PYGZgt{}}
\end{sphinxVerbatim}

For example, let’s come back to the Belgium PCMN. This chart of accounts is override in this way to add some properties.

\fvset{hllines={, ,}}%
\begin{sphinxVerbatim}[commandchars=\\\{\}]
\PYG{n+nt}{\PYGZlt{}record} \PYG{n+na}{id=}\PYG{l+s}{\PYGZdq{}l10nbe\PYGZus{}chart\PYGZus{}template\PYGZdq{}} \PYG{n+na}{model=}\PYG{l+s}{\PYGZdq{}account.chart.template\PYGZdq{}}\PYG{n+nt}{\PYGZgt{}}
    \PYG{n+nt}{\PYGZlt{}field} \PYG{n+na}{name=}\PYG{l+s}{\PYGZdq{}property\PYGZus{}account\PYGZus{}receivable\PYGZus{}id\PYGZdq{}} \PYG{n+na}{ref=}\PYG{l+s}{\PYGZdq{}a4000\PYGZdq{}}\PYG{n+nt}{/\PYGZgt{}}
    \PYG{n+nt}{\PYGZlt{}field} \PYG{n+na}{name=}\PYG{l+s}{\PYGZdq{}property\PYGZus{}account\PYGZus{}payable\PYGZus{}id\PYGZdq{}} \PYG{n+na}{ref=}\PYG{l+s}{\PYGZdq{}a440\PYGZdq{}}\PYG{n+nt}{/\PYGZgt{}}
    \PYG{n+nt}{\PYGZlt{}field} \PYG{n+na}{name=}\PYG{l+s}{\PYGZdq{}property\PYGZus{}account\PYGZus{}expense\PYGZus{}categ\PYGZus{}id\PYGZdq{}} \PYG{n+na}{ref=}\PYG{l+s}{\PYGZdq{}a600\PYGZdq{}}\PYG{n+nt}{/\PYGZgt{}}
    \PYG{n+nt}{\PYGZlt{}field} \PYG{n+na}{name=}\PYG{l+s}{\PYGZdq{}property\PYGZus{}account\PYGZus{}income\PYGZus{}categ\PYGZus{}id\PYGZdq{}} \PYG{n+na}{ref=}\PYG{l+s}{\PYGZdq{}a7010\PYGZdq{}}\PYG{n+nt}{/\PYGZgt{}}
    \PYG{n+nt}{\PYGZlt{}field} \PYG{n+na}{name=}\PYG{l+s}{\PYGZdq{}expense\PYGZus{}currency\PYGZus{}exchange\PYGZus{}account\PYGZus{}id\PYGZdq{}} \PYG{n+na}{ref=}\PYG{l+s}{\PYGZdq{}a654\PYGZdq{}}\PYG{n+nt}{/\PYGZgt{}}
    \PYG{n+nt}{\PYGZlt{}field} \PYG{n+na}{name=}\PYG{l+s}{\PYGZdq{}income\PYGZus{}currency\PYGZus{}exchange\PYGZus{}account\PYGZus{}id\PYGZdq{}} \PYG{n+na}{ref=}\PYG{l+s}{\PYGZdq{}a754\PYGZdq{}}\PYG{n+nt}{/\PYGZgt{}}
    \PYG{n+nt}{\PYGZlt{}field} \PYG{n+na}{name=}\PYG{l+s}{\PYGZdq{}transfer\PYGZus{}account\PYGZus{}id\PYGZdq{}} \PYG{n+na}{ref=}\PYG{l+s}{\PYGZdq{}trans\PYGZdq{}}\PYG{n+nt}{/\PYGZgt{}}
\PYG{n+nt}{\PYGZlt{}/record\PYGZgt{}}
\end{sphinxVerbatim}


\subsection{How to create a new bank operation model?}
\label{\detokenize{webservices/localization:how-to-create-a-new-bank-operation-model}}
\begin{sphinxadmonition}{note}{Note:}
How a bank operation model works exactly in Odoo? See \sphinxhref{https://www.odoo.com/documentation/user/online/accounting/bank/reconciliation/configure.html}{Configure model of entries}.
\end{sphinxadmonition}

Since \sphinxcode{\sphinxupquote{V10}}, a new feature is available in the bank statement reconciliation widget: the bank operation model.
This allows the user to pre-fill some accounting entries with a single click.
The creation of an \sphinxcode{\sphinxupquote{account.reconcile.model.template}} record is quite easy:

\fvset{hllines={, ,}}%
\begin{sphinxVerbatim}[commandchars=\\\{\}]
 \PYG{n+nt}{\PYGZlt{}record} \PYG{n+na}{id=}\PYG{l+s}{\PYGZdq{}...\PYGZdq{}} \PYG{n+na}{model=}\PYG{l+s}{\PYGZdq{}account.reconcile.model.template\PYGZdq{}}\PYG{n+nt}{\PYGZgt{}}
     \PYG{c}{\PYGZlt{}!\PYGZhy{}\PYGZhy{}}\PYG{c}{ [Required] Specify the name to display. }\PYG{c}{\PYGZhy{}\PYGZhy{}\PYGZgt{}}
     \PYG{n+nt}{\PYGZlt{}field} \PYG{n+na}{name=}\PYG{l+s}{\PYGZdq{}name\PYGZdq{}}\PYG{n+nt}{\PYGZgt{}}...\PYG{n+nt}{\PYGZlt{}/field\PYGZgt{}}

     \PYG{c}{\PYGZlt{}!\PYGZhy{}\PYGZhy{}}\PYG{c}{ [Required] Set the CoA owning this. }\PYG{c}{\PYGZhy{}\PYGZhy{}\PYGZgt{}}
     \PYG{n+nt}{\PYGZlt{}field} \PYG{n+na}{name=}\PYG{l+s}{\PYGZdq{}chart\PYGZus{}template\PYGZus{}id\PYGZdq{}} \PYG{n+na}{ref=}\PYG{l+s}{\PYGZdq{}...\PYGZdq{}}\PYG{n+nt}{/\PYGZgt{}}

     \PYG{c}{\PYGZlt{}!\PYGZhy{}\PYGZhy{}}\PYG{c}{ [Optional] Set a sequence number defining the order in which it will be displayed.}
\PYG{c}{     By default, the sequence is 10. }\PYG{c}{\PYGZhy{}\PYGZhy{}\PYGZgt{}}
     \PYG{n+nt}{\PYGZlt{}field} \PYG{n+na}{name=}\PYG{l+s}{\PYGZdq{}sequence\PYGZdq{}} \PYG{n+na}{eval=}\PYG{l+s}{\PYGZdq{}...\PYGZdq{}}\PYG{n+nt}{/\PYGZgt{}}

     \PYG{c}{\PYGZlt{}!\PYGZhy{}\PYGZhy{}}\PYG{c}{ [Optional] Define an account. }\PYG{c}{\PYGZhy{}\PYGZhy{}\PYGZgt{}}
     \PYG{n+nt}{\PYGZlt{}field} \PYG{n+na}{name=}\PYG{l+s}{\PYGZdq{}account\PYGZus{}id\PYGZdq{}} \PYG{n+na}{ref=}\PYG{l+s}{\PYGZdq{}...\PYGZdq{}}\PYG{n+nt}{/\PYGZgt{}}

     \PYG{c}{\PYGZlt{}!\PYGZhy{}\PYGZhy{}}\PYG{c}{ [Optional] Define a label to be added to the journal item. }\PYG{c}{\PYGZhy{}\PYGZhy{}\PYGZgt{}}
     \PYG{n+nt}{\PYGZlt{}field} \PYG{n+na}{name=}\PYG{l+s}{\PYGZdq{}label\PYGZdq{}}\PYG{n+nt}{\PYGZgt{}}...\PYG{n+nt}{\PYGZlt{}/field\PYGZgt{}}

     \PYG{c}{\PYGZlt{}!\PYGZhy{}\PYGZhy{}}\PYG{c}{ [Optional] Define the type of amount\PYGZus{}type, either \PYGZsq{}fixed\PYGZsq{} or \PYGZsq{}percentage\PYGZsq{}.}
\PYG{c}{     The last one is the default value. }\PYG{c}{\PYGZhy{}\PYGZhy{}\PYGZgt{}}
     \PYG{n+nt}{\PYGZlt{}field} \PYG{n+na}{name=}\PYG{l+s}{\PYGZdq{}amount\PYGZus{}type\PYGZdq{}}\PYG{n+nt}{\PYGZgt{}}...\PYG{n+nt}{\PYGZlt{}/field\PYGZgt{}}

     \PYG{c}{\PYGZlt{}!\PYGZhy{}\PYGZhy{}}\PYG{c}{ [Optional] Define the balance amount on which this model will be applied to (100 by default).}
\PYG{c}{     Fixed amount will count as a debit if it is negative, as a credit if it is positive. }\PYG{c}{\PYGZhy{}\PYGZhy{}\PYGZgt{}}
     \PYG{n+nt}{\PYGZlt{}field} \PYG{n+na}{name=}\PYG{l+s}{\PYGZdq{}amount\PYGZdq{}}\PYG{n+nt}{\PYGZgt{}}...\PYG{n+nt}{\PYGZlt{}/field\PYGZgt{}}

     \PYG{c}{\PYGZlt{}!\PYGZhy{}\PYGZhy{}}\PYG{c}{ [Optional] Define eventually a tax. }\PYG{c}{\PYGZhy{}\PYGZhy{}\PYGZgt{}}
     \PYG{n+nt}{\PYGZlt{}field} \PYG{n+na}{name=}\PYG{l+s}{\PYGZdq{}tax\PYGZus{}id\PYGZdq{}} \PYG{n+na}{ref=}\PYG{l+s}{\PYGZdq{}...\PYGZdq{}}\PYG{n+nt}{/\PYGZgt{}}

     \PYG{c}{\PYGZlt{}!\PYGZhy{}\PYGZhy{}}\PYG{c}{ [Optional] The sames fields are available twice.}
\PYG{c}{     To enable a second journal line, you can set this field to true and}
\PYG{c}{     fill the fields accordingly. }\PYG{c}{\PYGZhy{}\PYGZhy{}\PYGZgt{}}
     \PYG{n+nt}{\PYGZlt{}field} \PYG{n+na}{name=}\PYG{l+s}{\PYGZdq{}has\PYGZus{}second\PYGZus{}line\PYGZdq{}} \PYG{n+na}{eval=}\PYG{l+s}{\PYGZdq{}...\PYGZdq{}}\PYG{n+nt}{/\PYGZgt{}}
     \PYG{n+nt}{\PYGZlt{}field} \PYG{n+na}{name=}\PYG{l+s}{\PYGZdq{}second\PYGZus{}account\PYGZus{}id\PYGZdq{}} \PYG{n+na}{ref=}\PYG{l+s}{\PYGZdq{}...\PYGZdq{}}\PYG{n+nt}{/\PYGZgt{}}
     \PYG{n+nt}{\PYGZlt{}field} \PYG{n+na}{name=}\PYG{l+s}{\PYGZdq{}second\PYGZus{}label\PYGZdq{}}\PYG{n+nt}{\PYGZgt{}}...\PYG{n+nt}{\PYGZlt{}/field\PYGZgt{}}
     \PYG{n+nt}{\PYGZlt{}field} \PYG{n+na}{name=}\PYG{l+s}{\PYGZdq{}second\PYGZus{}amount\PYGZus{}type\PYGZdq{}}\PYG{n+nt}{\PYGZgt{}}...\PYG{n+nt}{\PYGZlt{}/field\PYGZgt{}}
     \PYG{n+nt}{\PYGZlt{}field} \PYG{n+na}{name=}\PYG{l+s}{\PYGZdq{}second\PYGZus{}amount\PYGZdq{}}\PYG{n+nt}{\PYGZgt{}}...\PYG{n+nt}{\PYGZlt{}/field\PYGZgt{}}
     \PYG{n+nt}{\PYGZlt{}field} \PYG{n+na}{name=}\PYG{l+s}{\PYGZdq{}second\PYGZus{}tax\PYGZus{}id\PYGZdq{}} \PYG{n+na}{ref=}\PYG{l+s}{\PYGZdq{}...\PYGZdq{}}\PYG{n+nt}{/\PYGZgt{}}
\PYG{n+nt}{\PYGZlt{}/record\PYGZgt{}}
\end{sphinxVerbatim}


\subsection{How to create a new dynamic report?}
\label{\detokenize{webservices/localization:how-to-create-a-new-dynamic-report}}
If you need to add some reports on your localization, you need to create a new module named \sphinxstylestrong{l10n\_xx\_reports}.
Furthermore, this additional module must be present in the \sphinxcode{\sphinxupquote{enterprise}} repository and must have at least two dependencies,
one to bring all the stuff for your localization module and one more, \sphinxcode{\sphinxupquote{account\_reports}}, to design dynamic reports.

\fvset{hllines={, ,}}%
\begin{sphinxVerbatim}[commandchars=\\\{\}]
\PYG{l+s+s1}{\PYGZsq{}}\PYG{l+s+s1}{depends}\PYG{l+s+s1}{\PYGZsq{}}\PYG{p}{:} \PYG{p}{[}\PYG{l+s+s1}{\PYGZsq{}}\PYG{l+s+s1}{l10n\PYGZus{}xx}\PYG{l+s+s1}{\PYGZsq{}}\PYG{p}{,} \PYG{l+s+s1}{\PYGZsq{}}\PYG{l+s+s1}{account\PYGZus{}reports}\PYG{l+s+s1}{\PYGZsq{}}\PYG{p}{]}\PYG{p}{,}
\end{sphinxVerbatim}

Once it’s done, you can start the creation of your report statements. The documentation is available in the following \sphinxhref{https://www.odoo.com/slides/slide/how-to-create-custom-accounting-report-415}{slides}.


\chapter{Setting Up}
\label{\detokenize{setup:setting-up}}\label{\detokenize{setup::doc}}

\section{Installing Odoo}
\label{\detokenize{setup/install:setup-install}}\label{\detokenize{setup/install::doc}}\label{\detokenize{setup/install:installing-odoo}}
There are mutliple ways to install Odoo, or not install it at all, depending
on the intended use case.

This documents attempts to describe most of the installation options.
\begin{description}
\item[{{\hyperref[\detokenize{setup/install:setup-install-online}]{\sphinxcrossref{\DUrole{std,std-ref}{Online}}}}}] \leavevmode
The easiest way to use Odoo in production or to try it.

\item[{{\hyperref[\detokenize{setup/install:setup-install-packaged}]{\sphinxcrossref{\DUrole{std,std-ref}{Packaged installers}}}}}] \leavevmode
Suitable for testing Odoo, developing modules and can be used for
long-term production use with additional deployment and maintenance work.

\item[{{\hyperref[\detokenize{setup/install:setup-install-source}]{\sphinxcrossref{\DUrole{std,std-ref}{Source Install}}}}}] \leavevmode
Provides greater flexibility:  e.g. allow multiple running Odoo versions on
the same system. Good for developing modules, can be used as base for
production deployment.

\item[{{\hyperref[\detokenize{setup/install:setup-install-docker}]{\sphinxcrossref{\DUrole{std,std-ref}{Docker}}}}}] \leavevmode
If you usually use \sphinxhref{https://www.docker.com}{docker} for development or deployment, an official
\sphinxhref{https://www.docker.com}{docker} base image is available.

\end{description}


\subsection{Editions}
\label{\detokenize{setup/install:editions}}\label{\detokenize{setup/install:setup-install-editions}}
There are two different \sphinxhref{https://www.odoo.com/pricing\#pricing\_table\_features}{Editions} of Odoo: the Community and Enterprise versions.
Using the Enterprise version is possible on our \sphinxhref{https://www.odoo.com/page/start}{SaaS} and accessing the code is
restricted to Enterprise customers and partners. The Community version is freely
available to anyone.

If you already use the Community version and wish to upgrade to Enterprise, please
refer to {\hyperref[\detokenize{setup/enterprise:setup-enterprise}]{\sphinxcrossref{\DUrole{std,std-ref}{From Community to Enterprise}}}} (except for {\hyperref[\detokenize{setup/install:setup-install-source}]{\sphinxcrossref{\DUrole{std,std-ref}{Source Install}}}}).


\subsection{Online}
\label{\detokenize{setup/install:online}}\label{\detokenize{setup/install:setup-install-online}}

\subsubsection{Demo}
\label{\detokenize{setup/install:demo}}
To simply get a quick idea of Odoo, \sphinxhref{https://demo.odoo.com}{demo} instances are available. They are
shared instances which only live for a few hours, and can be used to browse
around and try things out with no commitment.

\sphinxhref{https://demo.odoo.com}{Demo} instances require no local installation, just a web browser.


\subsubsection{SaaS}
\label{\detokenize{setup/install:saas}}
Trivial to start with, fully managed and migrated by Odoo S.A., Odoo’s \sphinxhref{https://www.odoo.com/page/start}{SaaS}
provides private instances and starts out free. It can be used to discover and
test Odoo and do non-code customizations (i.e. incompatible with custom modules
or the Odoo Apps Store) without having to install it locally.

Can be used for both testing Odoo and long-term production use.

Like \sphinxhref{https://demo.odoo.com}{demo} instances, \sphinxhref{https://www.odoo.com/page/start}{SaaS} instances require no local installation, a web
browser is sufficient.


\subsection{Packaged installers}
\label{\detokenize{setup/install:setup-install-packaged}}\label{\detokenize{setup/install:packaged-installers}}
Odoo provides packaged installers for Windows, deb-based distributions
(Debian, Ubuntu, …) and RPM-based distributions (Fedora, CentOS, RHEL, …) for
both the Community and Enterprise versions.

These packages automatically set up all dependencies (for the Community version),
but may be difficult to keep up-to-date.

Official Community packages with all relevant dependency requirements are
available on our \sphinxhref{https://nightly.odoo.com/11.0/nightly/}{nightly} server. Both Communtiy and Enterprise packages can
be downloaded from our \sphinxhref{https://www.odoo.com/page/download}{Download} page (you must to be logged in as a paying
customer or partner to download the Enterprise packages).


\subsubsection{Windows}
\label{\detokenize{setup/install:windows}}\begin{itemize}
\item {} 
Download the installer from our \sphinxhref{https://nightly.odoo.com/11.0/nightly/}{nightly} server (Community only)
or the Windows installer from the \sphinxhref{https://www.odoo.com/page/download}{Download} page (any edition)

\item {} 
Run the downloaded file

\begin{sphinxadmonition}{warning}{Warning:}
on Windows 8, you may see a warning titled “Windows protected
your PC”. Click \sphinxmenuselection{More Info} then
\sphinxmenuselection{Run anyway}
\end{sphinxadmonition}

\item {} 
Accept the \sphinxhref{http://en.wikipedia.org/wiki/User\_Account\_Control}{UAC} prompt

\item {} 
Go through the various installation steps

\end{itemize}

Odoo will automatically be started at the end of the installation.


\subsubsection{Linux}
\label{\detokenize{setup/install:linux}}

\paragraph{Debian/Ubuntu}
\label{\detokenize{setup/install:debian-ubuntu}}
Odoo 11.0 ‘deb’ package currently supports \sphinxhref{https://www.debian.org/releases/stretch/}{Debian Stretch}, \sphinxhref{http://releases.ubuntu.com/16.04/}{Ubuntu Xenial},
\sphinxhref{http://releases.ubuntu.com/17.04/}{Ubuntu Zesty} and \sphinxhref{http://releases.ubuntu.com/17.10/}{Ubuntu Artful}.


\subparagraph{Prepare}
\label{\detokenize{setup/install:prepare}}
Odoo needs a \sphinxhref{http://www.postgresql.org}{PostgreSQL} server to run properly. The default configuration for
the Odoo ‘deb’ package is to use the PostgreSQL server on the same host as your
Odoo instance. Execute the following command as root in order to install
PostgreSQL server :

\fvset{hllines={, ,}}%
\begin{sphinxVerbatim}[commandchars=\\\{\}]
\PYG{g+gp}{\PYGZsh{}} apt\PYGZhy{}get install postgresql \PYGZhy{}y
\end{sphinxVerbatim}

In order to print PDF reports, you must install \sphinxhref{http://wkhtmltopdf.org}{wkhtmltopdf} yourself:
the version of \sphinxhref{http://wkhtmltopdf.org}{wkhtmltopdf} available in debian repositories does not support
headers and footers so it can not be installed automatically.
The recommended version is 0.12.1 and is available on \sphinxhref{https://github.com/wkhtmltopdf/wkhtmltopdf/releases/tag/0.12.1}{the wkhtmltopdf download page},
in the archive section.


\subparagraph{Repository}
\label{\detokenize{setup/install:repository}}
Odoo S.A. provides a repository that can be used with  Debian and Ubuntu
distributions. It can be used to install Odoo Community Edition by executing the
following commands as root:

\fvset{hllines={, ,}}%
\begin{sphinxVerbatim}[commandchars=\\\{\}]
\PYG{g+gp}{\PYGZsh{}} wget \PYGZhy{}O \PYGZhy{} https://nightly.odoo.com/odoo.key \PYG{p}{\textbar{}} apt\PYGZhy{}key add \PYGZhy{}
\PYG{g+gp}{\PYGZsh{}} \PYG{n+nb}{echo} \PYG{l+s+s2}{\PYGZdq{}deb http://nightly.odoo.com/11.0/nightly/deb/ ./\PYGZdq{}} \PYGZgt{}\PYGZgt{} /etc/apt/sources.list.d/odoo.list
\PYG{g+gp}{\PYGZsh{}} apt\PYGZhy{}get update \PYG{o}{\PYGZam{}\PYGZam{}} apt\PYGZhy{}get install odoo
\end{sphinxVerbatim}

You can then use the usual \sphinxcode{\sphinxupquote{apt-get upgrade}} command to keep your installation up-to-date.

At this moment, there is no repository for the Enterprise Edition.


\subparagraph{Deb Package}
\label{\detokenize{setup/install:deb-package}}
Instead of using the repository as described above, the ‘deb’ package can be
downloaded here:
\begin{itemize}
\item {} 
Community Edition: \sphinxhref{https://nightly.odoo.com/11.0/nightly/}{nightly}

\item {} 
Enterprise Edition \sphinxhref{https://www.odoo.com/page/download}{Download}

\end{itemize}

You can then use \sphinxcode{\sphinxupquote{gdebi}}:

\fvset{hllines={, ,}}%
\begin{sphinxVerbatim}[commandchars=\\\{\}]
\PYG{g+gp}{\PYGZsh{}} gdebi \PYGZlt{}path\PYGZus{}to\PYGZus{}installation\PYGZus{}package\PYGZgt{}
\end{sphinxVerbatim}

Or \sphinxcode{\sphinxupquote{dpkg}} (handles less dependencies automatically):

\fvset{hllines={, ,}}%
\begin{sphinxVerbatim}[commandchars=\\\{\}]
\PYG{g+gp}{\PYGZsh{}} dpkg \PYGZhy{}i \PYGZlt{}path\PYGZus{}to\PYGZus{}installation\PYGZus{}package\PYGZgt{} \PYG{c+c1}{\PYGZsh{} this probably fails with missing dependencies}
\PYG{g+gp}{\PYGZsh{}} apt\PYGZhy{}get install \PYGZhy{}f \PYG{c+c1}{\PYGZsh{} should install the missing dependencies}
\PYG{g+gp}{\PYGZsh{}} dpkg \PYGZhy{}i \PYGZlt{}path\PYGZus{}to\PYGZus{}installation\PYGZus{}package\PYGZgt{}
\end{sphinxVerbatim}

This will install Odoo as a service, create the necessary \sphinxhref{http://www.postgresql.org}{PostgreSQL} user
and automatically start the server.

\begin{sphinxadmonition}{warning}{Warning:}
The 3 following python packages are only suggested by the Debian package.
Those packages are not available in Ubuntu Xenial (16.04).
\end{sphinxadmonition}
\begin{itemize}
\item {} 
python3-vobject: Used in calendars to produce ical files.

\item {} 
python3-pyldap: Used to authenticat users with LDAP.

\item {} 
python3-qrcode: Used by the hardware driver for ESC/POS

\end{itemize}

If you need one or all of the packages mentioned in the above warning, you can install them manually.
One way to do it, is simply using pip3 like this:

\fvset{hllines={, ,}}%
\begin{sphinxVerbatim}[commandchars=\\\{\}]
\PYG{g+gp}{\PYGZdl{}} sudo pip3 install vobject qrcode
\PYG{g+gp}{\PYGZdl{}} sudo apt install libldap2\PYGZhy{}dev libsasl2\PYGZhy{}dev
\PYG{g+gp}{\PYGZdl{}} sudo pip3 install pyldap
\end{sphinxVerbatim}

\begin{sphinxadmonition}{warning}{Warning:}
Debian 9 and Ubuntu do not provide a package for the python module
num2words.
Textual amounts will not be rendered by Odoo and this could cause
problems with the “l10n\_mx\_edi” module.
\end{sphinxadmonition}

If you need this feature, you can install the python module like this:

\fvset{hllines={, ,}}%
\begin{sphinxVerbatim}[commandchars=\\\{\}]
\PYG{g+gp}{\PYGZdl{}} sudo pip3 install num2words
\end{sphinxVerbatim}


\paragraph{Fedora}
\label{\detokenize{setup/install:fedora}}
Odoo 11.0 ‘rpm’ package supports Fedora 26.
As of 2017, CentOS does not have the minimum Python requirements (3.5) for
Odoo 11.0.


\subparagraph{Prepare}
\label{\detokenize{setup/install:id1}}
Odoo needs a \sphinxhref{http://www.postgresql.org}{PostgreSQL} server to run properly. Assuming that the ‘sudo’
command is available and configured properly, run the following commands :

\fvset{hllines={, ,}}%
\begin{sphinxVerbatim}[commandchars=\\\{\}]
\PYG{g+gp}{\PYGZdl{}} sudo dnf install \PYGZhy{}y postgresql\PYGZhy{}server
\PYG{g+gp}{\PYGZdl{}} sudo postgresql\PYGZhy{}setup \PYGZhy{}\PYGZhy{}initdb \PYGZhy{}\PYGZhy{}unit postgresql
\PYG{g+gp}{\PYGZdl{}} sudo systemctl \PYG{n+nb}{enable} postgresql
\PYG{g+gp}{\PYGZdl{}} sudo systemctl start postgresql
\end{sphinxVerbatim}

In order to print PDF reports, you must install \sphinxhref{http://wkhtmltopdf.org}{wkhtmltopdf} yourself:
the version of \sphinxhref{http://wkhtmltopdf.org}{wkhtmltopdf} available in debian repositories does not support
headers and footers so it can not be installed automatically.
The recommended version is 0.12.1 and is available on \sphinxhref{https://github.com/wkhtmltopdf/wkhtmltopdf/releases/tag/0.12.1}{the wkhtmltopdf download page},
in the archive section.


\subparagraph{Repository}
\label{\detokenize{setup/install:id2}}
Odoo S.A. provides a repository that can be used with the Fedora distibutions.
It can be used to install Odoo Community Edition by executing the following
commands:

\fvset{hllines={, ,}}%
\begin{sphinxVerbatim}[commandchars=\\\{\}]
\PYG{g+gp}{\PYGZdl{}} sudo dnf config\PYGZhy{}manager \PYGZhy{}\PYGZhy{}add\PYGZhy{}repo\PYG{o}{=}https://nightly.odoo.com/11.0/nightly/rpm/odoo.repo
\PYG{g+gp}{\PYGZdl{}} sudo dnf install \PYGZhy{}y odoo
\PYG{g+gp}{\PYGZdl{}} sudo systemctl \PYG{n+nb}{enable} odoo
\PYG{g+gp}{\PYGZdl{}} sudo systemctl start odoo
\end{sphinxVerbatim}


\subparagraph{RPM package}
\label{\detokenize{setup/install:rpm-package}}
Instead of using the repository as described above, the ‘rpm’ package can be
downloaded here:
\begin{itemize}
\item {} 
Community Edition: \sphinxhref{https://nightly.odoo.com/11.0/nightly/}{nightly}

\item {} 
Enterprise Edition \sphinxhref{https://www.odoo.com/page/download}{Download}

\end{itemize}

Once downloaded, the package can be installed using the ‘dnf’ package manager:

\fvset{hllines={, ,}}%
\begin{sphinxVerbatim}[commandchars=\\\{\}]
\PYG{g+gp}{\PYGZdl{}} sudo dnf localinstall odoo\PYGZus{}11.0.latest.noarch.rpm
\PYG{g+gp}{\PYGZdl{}} sudo systemctl \PYG{n+nb}{enable} odoo
\PYG{g+gp}{\PYGZdl{}} sudo systemctl start odoo
\end{sphinxVerbatim}


\subsection{Source Install}
\label{\detokenize{setup/install:setup-install-source}}\label{\detokenize{setup/install:source-install}}
The source “installation” really is about not installing Odoo, and running
it directly from source instead.

This can be more convenient for module developers as the Odoo source is
more easily accessible than using packaged installation (for information or
to build this documentation and have it available offline).

It also makes starting and stopping Odoo more flexible and explicit than the
services set up by the packaged installations, and allows overriding settings
using {\hyperref[\detokenize{reference/cmdline:reference-cmdline}]{\sphinxcrossref{\DUrole{std,std-ref}{command-line parameters}}}} without needing to
edit a configuration file.

Finally it provides greater control over the system’s set up, and allows more
easily keeping (and running) multiple versions of Odoo side-by-side.


\subsubsection{Prepare}
\label{\detokenize{setup/install:id3}}
Source installation requires manually installing dependencies:
\begin{itemize}
\item {} 
Python 3.5+.
\begin{itemize}
\item {} 
on Linux and OS X, using your package manager if not installed by default

\begin{sphinxadmonition}{note}{Note:}
on some system, \sphinxcode{\sphinxupquote{python}} command refers to Python 2 (outdated)
or to Python 3 (supported). Make sure you are using the right
version and that the alias \sphinxcode{\sphinxupquote{python3}} is present in your
\index{PATH}\index{environment variable!PATH}\sphinxcode{\sphinxupquote{PATH}}
\end{sphinxadmonition}

\item {} 
on Windows, use \sphinxhref{https://www.python.org/downloads/windows/}{the official Python 3 installer}.

\begin{sphinxadmonition}{warning}{Warning:}
select “add python.exe to Path” during installation, and
reboot afterwards to ensure the \index{PATH}\index{environment variable!PATH}\sphinxcode{\sphinxupquote{PATH}} is updated
\end{sphinxadmonition}

\begin{sphinxadmonition}{note}{Note:}
if Python is already installed, make sure it is 3.5 or above,
previous versions are not compatible with Odoo.
\end{sphinxadmonition}

\end{itemize}

\item {} 
PostgreSQL, to use a local database

After installation you will need to create a postgres user: by default the
only user is \sphinxcode{\sphinxupquote{postgres}}, and Odoo forbids connecting as \sphinxcode{\sphinxupquote{postgres}}.
\begin{itemize}
\item {} 
on Linux, use your distribution’s package, then create a postgres user
named like your login:

\fvset{hllines={, ,}}%
\begin{sphinxVerbatim}[commandchars=\\\{\}]
\PYG{g+gp}{\PYGZdl{}} sudo su \PYGZhy{} postgres \PYGZhy{}c \PYG{l+s+s2}{\PYGZdq{}}\PYG{l+s+s2}{createuser \PYGZhy{}s }\PYGZdl{}\PYG{l+s+s2}{USER}\PYG{l+s+s2}{\PYGZdq{}}
\end{sphinxVerbatim}

Because the role login is the same as your unix login unix sockets can be
use without a password.

\item {} 
on OS X, \sphinxhref{http://postgresapp.com}{postgres.app} is the simplest way to
get started, then create a postgres user as on Linux

\item {} 
on Windows, use \sphinxhref{http://www.enterprisedb.com/products-services-training/pgdownload}{PostgreSQL for windows} then
\begin{itemize}
\item {} 
add PostgreSQL’s \sphinxcode{\sphinxupquote{bin}} directory (default:
\sphinxcode{\sphinxupquote{C:\textbackslash{}Program Files\textbackslash{}PostgreSQL\textbackslash{}9.4\textbackslash{}bin}}) to your \index{PATH}\index{environment variable!PATH}\sphinxcode{\sphinxupquote{PATH}}

\item {} 
create a postgres user with a password using the pg admin gui: open
pgAdminIII, double-click the server to create a connection, select
\sphinxmenuselection{Edit \(\rightarrow\) New Object \(\rightarrow\) New Login Role}, enter the
usename in the \sphinxmenuselection{Role Name} field (e.g. \sphinxcode{\sphinxupquote{odoo}}), then open
the \sphinxmenuselection{Definition} tab and enter the password (e.g. \sphinxcode{\sphinxupquote{odoo}}),
then click \sphinxmenuselection{OK}.

The user and password must be passed to Odoo using either the
{\hyperref[\detokenize{reference/cmdline:cmdoption-odoo-bin-w}]{\sphinxcrossref{\sphinxcode{\sphinxupquote{-w}}}}} and {\hyperref[\detokenize{reference/cmdline:cmdoption-odoo-bin-r}]{\sphinxcrossref{\sphinxcode{\sphinxupquote{-r}}}}} options or
{\hyperref[\detokenize{reference/cmdline:reference-cmdline-config}]{\sphinxcrossref{\DUrole{std,std-ref}{the configuration file}}}}

\end{itemize}

\end{itemize}

\item {} 
Python dependencies listed in the \sphinxcode{\sphinxupquote{requirements.txt}} file.
\begin{itemize}
\item {} 
on Linux, python dependencies may be installable with the system’s package
manager or using pip.

For libraries using native code (Pillow, lxml, greenlet, gevent, psycopg2,
ldap) it may be necessary to install development tools and native
dependencies before pip is able to install the dependencies themselves.
These are available in \sphinxcode{\sphinxupquote{-dev}} or \sphinxcode{\sphinxupquote{-devel}} packages for Python,
Postgres, libxml2, libxslt, libevent, libsasl2 and libldap2. Then the Python
dependecies can themselves be installed:

\fvset{hllines={, ,}}%
\begin{sphinxVerbatim}[commandchars=\\\{\}]
\PYG{g+gp}{\PYGZdl{}} pip3 install \PYGZhy{}r requirements.txt
\end{sphinxVerbatim}

\item {} 
on OS X, you will need to install the Command Line Tools
(\sphinxcode{\sphinxupquote{xcode-select -{-}install}}) then download and install a package manager
of your choice (\sphinxhref{http://brew.sh}{homebrew}, \sphinxhref{https://www.macports.org}{macports}) to install non-Python dependencies.
pip can then be used to install the Python dependencies as on Linux:

\fvset{hllines={, ,}}%
\begin{sphinxVerbatim}[commandchars=\\\{\}]
\PYG{g+gp}{\PYGZdl{}} pip3 install \PYGZhy{}r requirements.txt
\end{sphinxVerbatim}

\item {} 
on Windows you need to install some of the dependencies manually, tweak the
requirements.txt file, then run pip to install the remaning ones.

Install \sphinxcode{\sphinxupquote{psycopg}} using the installer here
\sphinxurl{http://www.stickpeople.com/projects/python/win-psycopg/}

Then use pip to install the dependencies using the following
command from a cmd.exe prompt (replace \sphinxcode{\sphinxupquote{\textbackslash{}YourOdooPath}} by the actual
path where you downloaded Odoo):

\fvset{hllines={, ,}}%
\begin{sphinxVerbatim}[commandchars=\\\{\}]
\PYG{g+gp}{C:\PYGZbs{}\PYGZgt{}} \PYG{k}{cd} \PYGZbs{}YourOdooPath
\PYG{g+gp}{C:\PYGZbs{}YourOdooPath\PYGZgt{}} C:\PYGZbs{}Python35\PYGZbs{}Scripts\PYGZbs{}pip.exe install \PYGZhy{}r requirements.txt
\end{sphinxVerbatim}

\end{itemize}

\item {} 
\sphinxstyleemphasis{Less CSS} via nodejs
\begin{itemize}
\item {} 
on Linux, use your distribution’s package manager to install nodejs and
npm.

\begin{sphinxadmonition}{warning}{Warning:}
In debian wheezy and Ubuntu 13.10 and before you need to install
nodejs manually:

\fvset{hllines={, ,}}%
\begin{sphinxVerbatim}[commandchars=\\\{\}]
\PYG{g+gp}{\PYGZdl{}} wget \PYGZhy{}qO\PYGZhy{} https://deb.nodesource.com/setup \PYG{p}{\textbar{}} bash \PYGZhy{}
\PYG{g+gp}{\PYGZdl{}} apt\PYGZhy{}get install \PYGZhy{}y nodejs
\end{sphinxVerbatim}

In later debian (\textgreater{}jessie) and ubuntu (\textgreater{}14.04) you may need to add a
symlink as npm packages call \sphinxcode{\sphinxupquote{node}} but debian calls the binary
\sphinxcode{\sphinxupquote{nodejs}}

\fvset{hllines={, ,}}%
\begin{sphinxVerbatim}[commandchars=\\\{\}]
\PYG{g+gp}{\PYGZdl{}} apt\PYGZhy{}get install \PYGZhy{}y npm
\PYG{g+gp}{\PYGZdl{}} sudo ln \PYGZhy{}s /usr/bin/nodejs /usr/bin/node
\end{sphinxVerbatim}
\end{sphinxadmonition}

Once npm is installed, use it to install less:

\fvset{hllines={, ,}}%
\begin{sphinxVerbatim}[commandchars=\\\{\}]
\PYG{g+gp}{\PYGZdl{}} sudo npm install \PYGZhy{}g less
\end{sphinxVerbatim}

\item {} 
on OS X, install nodejs via your preferred package manager (\sphinxhref{http://brew.sh}{homebrew},
\sphinxhref{https://www.macports.org}{macports}) then install less:

\fvset{hllines={, ,}}%
\begin{sphinxVerbatim}[commandchars=\\\{\}]
\PYG{g+gp}{\PYGZdl{}} sudo npm install \PYGZhy{}g less
\end{sphinxVerbatim}

\item {} 
on Windows, \sphinxhref{https://nodejs.org/en/download/}{install nodejs}, reboot (to
update the \index{PATH}\index{environment variable!PATH}\sphinxcode{\sphinxupquote{PATH}}) and install less:

\fvset{hllines={, ,}}%
\begin{sphinxVerbatim}[commandchars=\\\{\}]
\PYG{g+gp}{C:\PYGZbs{}\PYGZgt{}} npm install \PYGZhy{}g less
\end{sphinxVerbatim}

\end{itemize}

\end{itemize}


\subsubsection{Fetch the sources}
\label{\detokenize{setup/install:fetch-the-sources}}
There are two ways to obtain the Odoo source code: zip or git.
\begin{itemize}
\item {} 
Odoo zip can be downloaded from  our \sphinxhref{https://nightly.odoo.com/11.0/nightly/}{nightly} server or our \sphinxhref{https://www.odoo.com/page/download}{Download}  page,
the zip file then needs to be uncompressed to use its content

\item {} 
git allows simpler update and easier switching between different versions
of Odoo. It also simplifies maintaining non-module patches and
contributions.  The primary drawback of git is that it is significantly
larger than a tarball as it contains the entire history of the Odoo project.

\end{itemize}


\paragraph{Community Edition}
\label{\detokenize{setup/install:community-edition}}
The git repository is \sphinxurl{https://github.com/odoo/odoo.git} for the Community
edition.

Downloading it requires a \sphinxhref{http://git-scm.com/download/}{git client}
(which may be available via your distribution on linux) and can be performed
using the following command:

\fvset{hllines={, ,}}%
\begin{sphinxVerbatim}[commandchars=\\\{\}]
\PYG{g+gp}{\PYGZdl{}} git clone https://github.com/odoo/odoo.git
\end{sphinxVerbatim}


\paragraph{Enterprise Edition}
\label{\detokenize{setup/install:enterprise-edition}}
If you have access to the Enterprise repository (see {\hyperref[\detokenize{setup/install:setup-install-editions}]{\sphinxcrossref{\DUrole{std,std-ref}{Editions}}}}
if you wish to get access), you can use this command to fetch the addons:

\fvset{hllines={, ,}}%
\begin{sphinxVerbatim}[commandchars=\\\{\}]
\PYG{g+gp}{\PYGZdl{}} git clone https://github.com/odoo/enterprise.git
\end{sphinxVerbatim}

\begin{sphinxadmonition}{note}{Note:}
The Enterprise git repository \sphinxstylestrong{does not contain the full Odoo
source code}. It is only a collection of extra add-ons. The main server
code is in the Community version.  Running the Enterprise version actually
means running the server from the Community version with the addons-path option
set to the folder with the Enterprise version.

You need to clone both the Community and Enterprise repository to have a working
Odoo installation
\end{sphinxadmonition}


\subsubsection{Running Odoo}
\label{\detokenize{setup/install:running-odoo}}
Once all dependencies are set up, Odoo can be launched by running \sphinxcode{\sphinxupquote{odoo-bin}}.

\begin{sphinxadmonition}{tip}{Tip:}
For the Enterprise edition, you must specify the \sphinxcode{\sphinxupquote{enterprise}}
addons folder when starting your server. You can do so by providing the path
to your \sphinxcode{\sphinxupquote{enterprise}} folder in the \sphinxcode{\sphinxupquote{addons-path}} parameter. Please
note that the \sphinxcode{\sphinxupquote{enterprise}} folder must come before the default
\sphinxcode{\sphinxupquote{addons}} folder in the  list for the addons to be loaded correctly.
\end{sphinxadmonition}

{\hyperref[\detokenize{reference/cmdline:reference-cmdline}]{\sphinxcrossref{\DUrole{std,std-ref}{Configuration}}}} can be provided either through
{\hyperref[\detokenize{reference/cmdline:reference-cmdline}]{\sphinxcrossref{\DUrole{std,std-ref}{command-line arguments}}}} or through a
{\hyperref[\detokenize{reference/cmdline:reference-cmdline-config}]{\sphinxcrossref{\DUrole{std,std-ref}{configuration file}}}}.

Common necessary configurations are:
\begin{itemize}
\item {} 
PostgreSQL host, port, user and password.

Odoo has no defaults beyond
\sphinxhref{http://initd.org/psycopg/docs/module.html}{psycopg2’s defaults}: connects
over a UNIX socket on port 5432 with the current user and no password. By
default this should work on Linux and OS X, but it \sphinxstyleemphasis{will not work} on
windows as it does not support UNIX sockets.

\item {} 
Custom addons path beyond the defaults, to load your own modules

\end{itemize}

Under Windows a typical way to execute odoo would be:

\fvset{hllines={, ,}}%
\begin{sphinxVerbatim}[commandchars=\\\{\}]
\PYG{g+gp}{C:\PYGZbs{}YourOdooPath\PYGZgt{}} python3 odoo\PYGZhy{}bin \PYGZhy{}w odoo \PYGZhy{}r odoo \PYGZhy{}\PYGZhy{}addons\PYGZhy{}path=addons,../mymodules \PYGZhy{}\PYGZhy{}db\PYGZhy{}filter=mydb\PYGZdl{}
\end{sphinxVerbatim}

Where \sphinxcode{\sphinxupquote{odoo}}, \sphinxcode{\sphinxupquote{odoo}} are the postgresql login and password,
\sphinxcode{\sphinxupquote{../mymodules}} a directory with additional addons and \sphinxcode{\sphinxupquote{mydb}} the default
db to serve on localhost:8069

Under Unix a typical way to execute odoo would be:

\fvset{hllines={, ,}}%
\begin{sphinxVerbatim}[commandchars=\\\{\}]
\PYG{g+gp}{\PYGZdl{}} ./odoo\PYGZhy{}bin \PYGZhy{}\PYGZhy{}addons\PYGZhy{}path\PYG{o}{=}addons,../mymodules \PYGZhy{}\PYGZhy{}db\PYGZhy{}filter\PYG{o}{=}mydb\PYGZdl{}
\end{sphinxVerbatim}

Where \sphinxcode{\sphinxupquote{../mymodules}} is a directory with additional addons and \sphinxcode{\sphinxupquote{mydb}} the
default db to serve on localhost:8069


\subsubsection{Virtualenv}
\label{\detokenize{setup/install:virtualenv}}
\sphinxhref{https://pypi.python.org/pypi/virtualenv}{Virtualenv} is a tool to create Python isolated environments because it’s
sometimes preferable to not mix your distribution python modules packages
with globally installed python modules with pip.

This section will explain how to run Odoo in a such isolated Python environment.

Here we are going to use \sphinxhref{https://virtualenvwrapper.readthedocs.io/en/latest/}{virtualenvwrapper} which is a set of shell scripts that
makes the use of virtualenv easier.

The examples below are based on a Debian 9 distribution but could be adapted on
any platform where \sphinxhref{https://virtualenvwrapper.readthedocs.io/en/latest/}{virtualenvwrapper} and \sphinxhref{https://pypi.python.org/pypi/virtualenv}{virtualenv} are able to run.

This section assumes that you obtained the Odoo sources from the zip file or the
git repository as explained above. The same apply for postgresql installation
and configuration.


\paragraph{Install virtualenvwrapper}
\label{\detokenize{setup/install:install-virtualenvwrapper}}
\fvset{hllines={, ,}}%
\begin{sphinxVerbatim}[commandchars=\\\{\}]
\PYG{g+gp}{\PYGZdl{}} sudo apt install virtualenvwrapper
\PYG{g+gp}{\PYGZdl{}} \PYG{n+nb}{source} /usr/share/virtualenvwrapper/virtualenvwrapper.sh
\end{sphinxVerbatim}

This will install \sphinxhref{https://virtualenvwrapper.readthedocs.io/en/latest/}{virtualenvwrapper} and activate it immediately.
Now, let’s install the tools required to build Odoo dependencies if needed:

\fvset{hllines={, ,}}%
\begin{sphinxVerbatim}[commandchars=\\\{\}]
\PYG{g+gp}{\PYGZdl{}} sudo apt install build\PYGZhy{}essential python3\PYGZhy{}dev libxslt\PYGZhy{}dev libzip\PYGZhy{}dev libldap2\PYGZhy{}dev libsasl2\PYGZhy{}dev
\end{sphinxVerbatim}


\paragraph{Create an isolated environment}
\label{\detokenize{setup/install:create-an-isolated-environment}}
Now we can create a virtual environment for Odoo like this:

\fvset{hllines={, ,}}%
\begin{sphinxVerbatim}[commandchars=\\\{\}]
\PYG{g+gp}{\PYGZdl{}} mkvirtualenv \PYGZhy{}p /usr/bin/python3 odoo\PYGZhy{}venv
\end{sphinxVerbatim}

With this command, we ask for an isolated Python3 environment that will be named
“odoo-env”. If the command works as expected, your shell is now using this
environment. Your prompt should have changed to remind you that you are using
an isolated environment. You can verify with this command:

\fvset{hllines={, ,}}%
\begin{sphinxVerbatim}[commandchars=\\\{\}]
\PYG{g+gp}{\PYGZdl{}} which python3
\end{sphinxVerbatim}

This command should show you the path to the Python interpreter located in the
isolated environment directory.

Now let’s install the Odoo required python packages:

\fvset{hllines={, ,}}%
\begin{sphinxVerbatim}[commandchars=\\\{\}]
\PYG{g+gp}{\PYGZdl{}} \PYG{n+nb}{cd} your\PYGZus{}odoo\PYGZus{}sources\PYGZus{}path
\PYG{g+gp}{\PYGZdl{}} pip install \PYGZhy{}r requirements.txt
\end{sphinxVerbatim}

After a little while, you should be ready to run odoo from the command line as
explained above.

When you you want to leave the virtual environment, just issue this command:

\fvset{hllines={, ,}}%
\begin{sphinxVerbatim}[commandchars=\\\{\}]
\PYG{g+gp}{\PYGZdl{}} deactivate
\end{sphinxVerbatim}

Whenever you want to work again with your ‘odoo-venv’ environment:

\fvset{hllines={, ,}}%
\begin{sphinxVerbatim}[commandchars=\\\{\}]
\PYG{g+gp}{\PYGZdl{}} workon odoo\PYGZhy{}venv
\end{sphinxVerbatim}


\subsection{Docker}
\label{\detokenize{setup/install:docker}}\label{\detokenize{setup/install:setup-install-docker}}
The full documentation on how to use Odoo with Docker can be found on the
offcial Odoo \sphinxhref{https://registry.hub.docker.com/\_/odoo/}{docker image} page.
\phantomsection\label{\detokenize{setup/install:the-official-installer}}

\section{Deploying Odoo}
\label{\detokenize{setup/deploy::doc}}\label{\detokenize{setup/deploy:deploying-odoo}}\label{\detokenize{setup/deploy:extra}}
This document describes basic steps to set up Odoo in production or on an
internet-facing server. It follows {\hyperref[\detokenize{setup/install:setup-install}]{\sphinxcrossref{\DUrole{std,std-ref}{installation}}}}, and is
not generally necessary for a development systems that is not exposed on the
internet.

\begin{sphinxadmonition}{warning}{Warning:}
If you are setting up a public server, be sure to check our {\hyperref[\detokenize{setup/deploy:security}]{\sphinxcrossref{\DUrole{std,std-ref}{Security}}}} recommandations!
\end{sphinxadmonition}


\subsection{dbfilter}
\label{\detokenize{setup/deploy:dbfilter}}\label{\detokenize{setup/deploy:db-filter}}
Odoo is a multi-tenant system: a single Odoo system may run and serve a number
of database instances. It is also highly customizable, with customizations
(starting from the modules being loaded) depending on the “current database”.

This is not an issue when working with the backend (web client) as a logged-in
company user: the database can be selected when logging in, and customizations
loaded afterwards.

However it is an issue for non-logged users (portal, website) which aren’t
bound to a database: Odoo needs to know which database should be used to load
the website page or perform the operation. If multi-tenancy is not used that is not an
issue, there’s only one database to use, but if there are multiple databases
accessible Odoo needs a rule to know which one it should use.

That is one of the purposes of {\hyperref[\detokenize{reference/cmdline:cmdoption-odoo-bin-db-filter}]{\sphinxcrossref{\sphinxcode{\sphinxupquote{-{-}db-filter}}}}}:
it specifies how the database should be selected based on the hostname (domain)
that is being requested. The value is a \sphinxhref{https://docs.python.org/3/library/re.html}{regular expression}, possibly
including the dynamically injected hostname (\sphinxcode{\sphinxupquote{\%h}}) or the first subdomain
(\sphinxcode{\sphinxupquote{\%d}}) through which the system is being accessed.

For servers hosting multiple databases in production, especially if \sphinxcode{\sphinxupquote{website}}
is used, dbfilter \sphinxstylestrong{must} be set, otherwise a number of features will not work
correctly.


\subsubsection{Configuration samples}
\label{\detokenize{setup/deploy:configuration-samples}}\begin{itemize}
\item {} 
Show only databases with names beginning with ‘mycompany’

\end{itemize}

in \sphinxcode{\sphinxupquote{/etc/odoo.conf}} set:

\fvset{hllines={, ,}}%
\begin{sphinxVerbatim}[commandchars=\\\{\}]
\PYG{k}{[options]}
\PYG{n+na}{dbfilter} \PYG{o}{=} \PYG{l+s}{\PYGZca{}mycompany.*\PYGZdl{}}
\end{sphinxVerbatim}
\begin{itemize}
\item {} 
Show only databases matching the first subdomain after \sphinxcode{\sphinxupquote{www}}: for example
the database “mycompany” will be shown if the incoming request
was sent to \sphinxcode{\sphinxupquote{www.mycompany.com}} or \sphinxcode{\sphinxupquote{mycompany.co.uk}}, but not
for \sphinxcode{\sphinxupquote{www2.mycompany.com}} or \sphinxcode{\sphinxupquote{helpdesk.mycompany.com}}.

\end{itemize}

in \sphinxcode{\sphinxupquote{/etc/odoo.conf}} set:

\fvset{hllines={, ,}}%
\begin{sphinxVerbatim}[commandchars=\\\{\}]
\PYG{k}{[options]}
\PYG{n+na}{dbfilter} \PYG{o}{=} \PYG{l+s}{\PYGZca{}\PYGZpc{}d\PYGZdl{}}
\end{sphinxVerbatim}

\begin{sphinxadmonition}{note}{Note:}
Setting a proper {\hyperref[\detokenize{reference/cmdline:cmdoption-odoo-bin-db-filter}]{\sphinxcrossref{\sphinxcode{\sphinxupquote{-{-}db-filter}}}}} is an important part
of securing your deployment.
Once it is correctly working and only matching a single database per hostname, it
is strongly recommended to block access to the database manager screens,
and to use the \sphinxcode{\sphinxupquote{-{-}no-database-list}} startup paramater to prevent listing
your databases, and to block access to the database management screens.
See also {\hyperref[\detokenize{setup/deploy:security}]{\sphinxcrossref{security}}}.
\end{sphinxadmonition}


\subsection{PostgreSQL}
\label{\detokenize{setup/deploy:postgresql}}
By default, PostgreSQL only allows connection over UNIX sockets and loopback
connections (from “localhost”, the same machine the PostgreSQL server is
installed on).

UNIX socket is fine if you want Odoo and PostgreSQL to execute on the same
machine, and is the default when no host is provided, but if you want Odoo and
PostgreSQL to execute on different machines %
\begin{footnote}[1]\sphinxAtStartFootnote
to have multiple Odoo installations use the same PostgreSQL database,
or to provide more computing resources to both software.
%
\end{footnote} it will
need to \sphinxhref{http://www.postgresql.org/docs/9.6/static/runtime-config-connection.html}{listen to network interfaces} %
\begin{footnote}[2]\sphinxAtStartFootnote
technically a tool like \sphinxhref{http://www.dest-unreach.org/socat/}{socat} can be used to proxy UNIX sockets across
networks, but that is mostly for software which can only be used over
UNIX sockets
%
\end{footnote}, either:
\begin{itemize}
\item {} 
Only accept loopback connections and \sphinxhref{http://www.postgresql.org/docs/9.6/static/ssh-tunnels.html}{use an SSH tunnel} between the
machine on which Odoo runs and the one on which PostgreSQL runs, then
configure Odoo to connect to its end of the tunnel

\item {} 
Accept connections to the machine on which Odoo is installed, possibly
over ssl (see \sphinxhref{http://www.postgresql.org/docs/9.6/static/runtime-config-connection.html}{PostgreSQL connection settings} for details), then configure
Odoo to connect over the network

\end{itemize}


\subsubsection{Configuration sample}
\label{\detokenize{setup/deploy:configuration-sample}}\begin{itemize}
\item {} 
Allow tcp connection on localhost

\item {} 
Allow tcp connection from 192.168.1.x network

\end{itemize}

in \sphinxcode{\sphinxupquote{/etc/postgresql/9.5/main/pg\_hba.conf}} set:

\fvset{hllines={, ,}}%
\begin{sphinxVerbatim}[commandchars=\\\{\}]
\PYGZsh{} IPv4 local connections:
host    all             all             127.0.0.1/32            md5
host    all             all             192.168.1.0/24          md5
\end{sphinxVerbatim}

in \sphinxcode{\sphinxupquote{/etc/postgresql/9.5/main/postgresql.conf}} set:

\fvset{hllines={, ,}}%
\begin{sphinxVerbatim}[commandchars=\\\{\}]
listen\PYGZus{}addresses = \PYGZsq{}localhost,192.168.1.2\PYGZsq{}
port = 5432
max\PYGZus{}connections = 80
\end{sphinxVerbatim}


\subsubsection{Configuring Odoo}
\label{\detokenize{setup/deploy:configuring-odoo}}\label{\detokenize{setup/deploy:setup-deploy-odoo}}
Out of the box, Odoo connects to a local postgres over UNIX socket via port
5432. This can be overridden using {\hyperref[\detokenize{reference/cmdline:reference-cmdline-server-database}]{\sphinxcrossref{\DUrole{std,std-ref}{the database options}}}} when your Postgres deployment is not
local and/or does not use the installation defaults.

The {\hyperref[\detokenize{setup/install:setup-install-packaged}]{\sphinxcrossref{\DUrole{std,std-ref}{packaged installers}}}} will automatically
create a new user (\sphinxcode{\sphinxupquote{odoo}}) and set it as the database user.
\begin{itemize}
\item {} 
The database management screens are protected by the \sphinxcode{\sphinxupquote{admin\_passwd}}
setting. This setting can only be set using configuration files, and is
simply checked before performing database alterations. It should be set to
a randomly generated value to ensure third parties can not use this
interface.

\item {} 
All database operations use the {\hyperref[\detokenize{reference/cmdline:reference-cmdline-server-database}]{\sphinxcrossref{\DUrole{std,std-ref}{database options}}}}, including the database management
screen. For the database management screen to work requires that the PostgreSQL user
have \sphinxcode{\sphinxupquote{createdb}} right.

\item {} 
Users can always drop databases they own. For the database management screen
to be completely non-functional, the PostgreSQL user needs to be created with
\sphinxcode{\sphinxupquote{no-createdb}} and the database must be owned by a different PostgreSQL user.

\begin{sphinxadmonition}{warning}{Warning:}
the PostgreSQL user \sphinxstyleemphasis{must not} be a superuser
\end{sphinxadmonition}

\end{itemize}


\paragraph{Configuration sample}
\label{\detokenize{setup/deploy:id3}}\begin{itemize}
\item {} 
connect to a PostgreSQL server on 192.168.1.2

\item {} 
port 5432

\item {} 
using an ‘odoo’ user account,

\item {} 
with ‘pwd’ as a password

\item {} 
filtering only db with a name beginning with ‘mycompany’

\end{itemize}

in \sphinxcode{\sphinxupquote{/etc/odoo.conf}} set:

\fvset{hllines={, ,}}%
\begin{sphinxVerbatim}[commandchars=\\\{\}]
\PYG{k}{[options]}
\PYG{n+na}{admin\PYGZus{}passwd} \PYG{o}{=} \PYG{l+s}{mysupersecretpassword}
\PYG{n+na}{db\PYGZus{}host} \PYG{o}{=} \PYG{l+s}{192.168.1.2}
\PYG{n+na}{db\PYGZus{}port} \PYG{o}{=} \PYG{l+s}{5432}
\PYG{n+na}{db\PYGZus{}user} \PYG{o}{=} \PYG{l+s}{odoo}
\PYG{n+na}{db\PYGZus{}password} \PYG{o}{=} \PYG{l+s}{pwd}
\PYG{n+na}{dbfilter} \PYG{o}{=} \PYG{l+s}{\PYGZca{}mycompany.*\PYGZdl{}}
\end{sphinxVerbatim}


\subsubsection{SSL Between Odoo and PostgreSQL}
\label{\detokenize{setup/deploy:ssl-between-odoo-and-postgresql}}\label{\detokenize{setup/deploy:postgresql-ssl-connect}}
Since Odoo 11.0, you can enforce ssl connection between Odoo and PostgreSQL.
in Odoo the db\_sslmode control the ssl security of the connection
with value choosed out of ‘disable’, ‘allow’, ‘prefer’, ‘require’, ‘verify-ca’
or ‘verify-full’

\sphinxhref{https://www.postgresql.org/docs/current/static/libpq-ssl.html}{PostgreSQL Doc}


\subsection{Builtin server}
\label{\detokenize{setup/deploy:id4}}\label{\detokenize{setup/deploy:builtin-server}}
Odoo includes built-in HTTP servers, using either multithreading or
multiprocessing.

For production use, it is recommended to use the multiprocessing server as it
increases stability, makes somewhat better use of computing resources and can
be better monitored and resource-restricted.
\begin{itemize}
\item {} 
Multiprocessing is enabled by configuring {\hyperref[\detokenize{reference/cmdline:cmdoption-odoo-bin-workers}]{\sphinxcrossref{\sphinxcode{\sphinxupquote{a non-zero number of
worker processes}}}}}, the number of workers should be based
on the number of cores in the machine (possibly with some room for cron
workers depending on how much cron work is predicted)

\item {} 
Worker limits can be configured based on the hardware configuration to avoid
resources exhaustion

\end{itemize}

\begin{sphinxadmonition}{warning}{Warning:}
multiprocessing mode currently isn’t available on Windows
\end{sphinxadmonition}


\subsubsection{Worker number calculation}
\label{\detokenize{setup/deploy:worker-number-calculation}}\begin{itemize}
\item {} 
Rule of thumb : (\#CPU * 2) + 1

\item {} 
Cron workers need CPU

\item {} 
1 worker \textasciitilde{}= 6 concurrent users

\end{itemize}


\subsubsection{memory size calculation}
\label{\detokenize{setup/deploy:memory-size-calculation}}\begin{itemize}
\item {} 
We consider 20\% of the requests are heavy requests, while 80\% are simpler ones

\item {} 
A heavy worker, when all computed field are well designed, SQL requests are well designed, … is estimated to consume around 1Go of RAM

\item {} 
A lighter worker, in the same scenario, is estimated to consume around 150MB of RAM

\end{itemize}

Needed RAM = \#worker * ( (light\_worker\_ratio * light\_worker\_ram\_estimation) + (heavy\_worker\_ratio * heavy\_worker\_ram\_estimation) )


\subsubsection{LiveChat}
\label{\detokenize{setup/deploy:livechat}}
In multiprocessing, a dedicated LiveChat worker is automatically started and
listening on {\hyperref[\detokenize{reference/cmdline:cmdoption-odoo-bin-longpolling-port}]{\sphinxcrossref{\sphinxcode{\sphinxupquote{the longpolling port}}}}} but
the client will not connect to it.

Instead you must have a proxy redirecting requests whose URL starts with
\sphinxcode{\sphinxupquote{/longpolling/}} to the longpolling port. Other request should be proxied to
the {\hyperref[\detokenize{reference/cmdline:cmdoption-odoo-bin-http-port}]{\sphinxcrossref{\sphinxcode{\sphinxupquote{normal HTTP port}}}}}

To achieve such a thing, you’ll need to deploy a reverse proxy in front of Odoo,
like nginx or apache. When doing so, you’ll need to forward some more http Headers
to Odoo, and activate the proxy\_mode in Odoo configuration to have Odoo read those
headers.


\subsubsection{Configuration sample}
\label{\detokenize{setup/deploy:id5}}\begin{itemize}
\item {} 
Server with 4 CPU, 8 Thread

\item {} 
60 concurrent users

\item {} 
60 users / 6 = 10 \textless{}- theorical number of worker needed

\item {} 
(4 * 2) + 1 = 9 \textless{}- theorical maximal number of worker

\item {} 
We’ll use 8 workers + 1 for cron. We’ll also use a monitoring system to measure cpu load, and check if it’s between 7 and 7.5 .

\item {} 
RAM = 9 * ((0.8*150) + (0.2*1024)) \textasciitilde{}= 3Go RAM for Odoo

\end{itemize}

in \sphinxcode{\sphinxupquote{/etc/odoo.conf}}:

\fvset{hllines={, ,}}%
\begin{sphinxVerbatim}[commandchars=\\\{\}]
\PYG{k}{[options]}
\PYG{n+na}{limit\PYGZus{}memory\PYGZus{}hard} \PYG{o}{=} \PYG{l+s}{1677721600}
\PYG{n+na}{limit\PYGZus{}memory\PYGZus{}soft} \PYG{o}{=} \PYG{l+s}{629145600}
\PYG{n+na}{limit\PYGZus{}request} \PYG{o}{=} \PYG{l+s}{8192}
\PYG{n+na}{limit\PYGZus{}time\PYGZus{}cpu} \PYG{o}{=} \PYG{l+s}{600}
\PYG{n+na}{limit\PYGZus{}time\PYGZus{}real} \PYG{o}{=} \PYG{l+s}{1200}
\PYG{n+na}{max\PYGZus{}cron\PYGZus{}threads} \PYG{o}{=} \PYG{l+s}{1}
\PYG{n+na}{workers} \PYG{o}{=} \PYG{l+s}{8}
\end{sphinxVerbatim}


\subsection{HTTPS}
\label{\detokenize{setup/deploy:https}}\label{\detokenize{setup/deploy:https-proxy}}
Whether it’s accessed via website/web client or web service, Odoo transmits
authentication information in cleartext. This means a secure deployment of
Odoo must use HTTPS%
\begin{footnote}[3]\sphinxAtStartFootnote
or be accessible only over an internal packet-switched network, but that
requires secured switches, protections against \sphinxhref{http://en.wikipedia.org/wiki/ARP\_spoofing}{ARP spoofing} and
precludes usage of WiFi. Even over secure packet-switched networks,
deployment over HTTPS is recommended, and possible costs are lowered as
“self-signed” certificates are easier to deploy on a controlled
environment than over the internet.
%
\end{footnote}. SSL termination can be implemented via
just about any SSL termination proxy, but requires the following setup:
\begin{itemize}
\item {} 
Enable Odoo’s {\hyperref[\detokenize{reference/cmdline:cmdoption-odoo-bin-proxy-mode}]{\sphinxcrossref{\sphinxcode{\sphinxupquote{proxy mode}}}}}. This should only be enabled when Odoo is behind a reverse proxy

\item {} 
Set up the SSL termination proxy (\sphinxhref{http://nginx.com/resources/admin-guide/nginx-ssl-termination/}{Nginx termination example})

\item {} 
Set up the proxying itself (\sphinxhref{http://nginx.com/resources/admin-guide/reverse-proxy/}{Nginx proxying example})

\item {} 
Your SSL termination proxy should also automatically redirect non-secure
connections to the secure port

\end{itemize}

\begin{sphinxadmonition}{warning}{Warning:}
In case you are using the Point of Sale module in combination with a \sphinxhref{https://www.odoo.com/page/point-of-sale-hardware\#part\_2}{POSBox},
you must disable the HTTPS configuration for the route \sphinxcode{\sphinxupquote{/pos/web}} to avoid
mixed-content errors.
\end{sphinxadmonition}


\subsubsection{Configuration sample}
\label{\detokenize{setup/deploy:id7}}\begin{itemize}
\item {} 
Redirect http requests to https

\item {} 
Proxy requests to odoo

\end{itemize}

in \sphinxcode{\sphinxupquote{/etc/odoo.conf}} set:

\fvset{hllines={, ,}}%
\begin{sphinxVerbatim}[commandchars=\\\{\}]
\PYG{n+na}{proxy\PYGZus{}mode} \PYG{o}{=} \PYG{l+s}{True}
\end{sphinxVerbatim}

in \sphinxcode{\sphinxupquote{/etc/nginx/sites-enabled/odoo.conf}} set:

\fvset{hllines={, ,}}%
\begin{sphinxVerbatim}[commandchars=\\\{\}]
\PYG{c+c1}{\PYGZsh{}odoo server}
\PYG{k}{upstream} \PYG{l+s}{odoo} \PYG{p}{\PYGZob{}}
 \PYG{k+kn}{server} \PYG{n}{127.0.0.1}\PYG{p}{:}\PYG{l+m+mi}{8069}\PYG{p}{;}
\PYG{p}{\PYGZcb{}}
\PYG{k}{upstream} \PYG{l+s}{odoochat} \PYG{p}{\PYGZob{}}
 \PYG{k+kn}{server} \PYG{n}{127.0.0.1}\PYG{p}{:}\PYG{l+m+mi}{8072}\PYG{p}{;}
\PYG{p}{\PYGZcb{}}

\PYG{c+c1}{\PYGZsh{} http \PYGZhy{}\PYGZgt{} https}
\PYG{k}{server} \PYG{p}{\PYGZob{}}
   \PYG{k+kn}{listen} \PYG{l+m+mi}{80}\PYG{p}{;}
   \PYG{k+kn}{server\PYGZus{}name} \PYG{l+s}{odoo.mycompany.com}\PYG{p}{;}
   \PYG{k+kn}{rewrite} \PYG{l+s}{\PYGZca{}(.*)} \PYG{l+s}{https://}\PYG{n+nv}{\PYGZdl{}host\PYGZdl{}1} \PYG{l+s}{permanent}\PYG{p}{;}
\PYG{p}{\PYGZcb{}}

\PYG{k}{server} \PYG{p}{\PYGZob{}}
 \PYG{k+kn}{listen} \PYG{l+m+mi}{443}\PYG{p}{;}
 \PYG{k+kn}{server\PYGZus{}name} \PYG{l+s}{odoo.mycompany.com}\PYG{p}{;}
 \PYG{k+kn}{proxy\PYGZus{}read\PYGZus{}timeout} \PYG{l+s}{720s}\PYG{p}{;}
 \PYG{k+kn}{proxy\PYGZus{}connect\PYGZus{}timeout} \PYG{l+s}{720s}\PYG{p}{;}
 \PYG{k+kn}{proxy\PYGZus{}send\PYGZus{}timeout} \PYG{l+s}{720s}\PYG{p}{;}

 \PYG{c+c1}{\PYGZsh{} Add Headers for odoo proxy mode}
 \PYG{k+kn}{proxy\PYGZus{}set\PYGZus{}header} \PYG{l+s}{X\PYGZhy{}Forwarded\PYGZhy{}Host} \PYG{n+nv}{\PYGZdl{}host}\PYG{p}{;}
 \PYG{k+kn}{proxy\PYGZus{}set\PYGZus{}header} \PYG{l+s}{X\PYGZhy{}Forwarded\PYGZhy{}For} \PYG{n+nv}{\PYGZdl{}proxy\PYGZus{}add\PYGZus{}x\PYGZus{}forwarded\PYGZus{}for}\PYG{p}{;}
 \PYG{k+kn}{proxy\PYGZus{}set\PYGZus{}header} \PYG{l+s}{X\PYGZhy{}Forwarded\PYGZhy{}Proto} \PYG{n+nv}{\PYGZdl{}scheme}\PYG{p}{;}
 \PYG{k+kn}{proxy\PYGZus{}set\PYGZus{}header} \PYG{l+s}{X\PYGZhy{}Real\PYGZhy{}IP} \PYG{n+nv}{\PYGZdl{}remote\PYGZus{}addr}\PYG{p}{;}

 \PYG{c+c1}{\PYGZsh{} SSL parameters}
 \PYG{k+kn}{ssl} \PYG{n+no}{on}\PYG{p}{;}
 \PYG{k+kn}{ssl\PYGZus{}certificate} \PYG{l+s}{/etc/ssl/nginx/server.crt}\PYG{p}{;}
 \PYG{k+kn}{ssl\PYGZus{}certificate\PYGZus{}key} \PYG{l+s}{/etc/ssl/nginx/server.key}\PYG{p}{;}
 \PYG{k+kn}{ssl\PYGZus{}session\PYGZus{}timeout} \PYG{l+m+mi}{30m}\PYG{p}{;}
 \PYG{k+kn}{ssl\PYGZus{}protocols} \PYG{l+s}{TLSv1} \PYG{l+s}{TLSv1.1} \PYG{l+s}{TLSv1.2}\PYG{p}{;}
 \PYG{k+kn}{ssl\PYGZus{}ciphers} \PYG{l+s}{\PYGZsq{}ECDHE\PYGZhy{}RSA\PYGZhy{}AES128\PYGZhy{}GCM\PYGZhy{}SHA256:ECDHE\PYGZhy{}ECDSA\PYGZhy{}AES128\PYGZhy{}GCM\PYGZhy{}SHA256:ECDHE\PYGZhy{}RSA\PYGZhy{}AES256\PYGZhy{}GCM\PYGZhy{}SHA384:ECDHE\PYGZhy{}ECDSA\PYGZhy{}AES256\PYGZhy{}GCM\PYGZhy{}SHA384:DHE\PYGZhy{}RSA\PYGZhy{}AES128\PYGZhy{}GCM\PYGZhy{}SHA256:DHE\PYGZhy{}DSS\PYGZhy{}AES128\PYGZhy{}GCM\PYGZhy{}SHA256:kEDH+AESGCM:ECDHE\PYGZhy{}RSA\PYGZhy{}AES128\PYGZhy{}SHA256:ECDHE\PYGZhy{}ECDSA\PYGZhy{}AES128\PYGZhy{}SHA256:ECDHE\PYGZhy{}RSA\PYGZhy{}AES128\PYGZhy{}SHA:ECDHE\PYGZhy{}ECDSA\PYGZhy{}AES128\PYGZhy{}SHA:ECDHE\PYGZhy{}RSA\PYGZhy{}AES256\PYGZhy{}SHA384:ECDHE\PYGZhy{}ECDSA\PYGZhy{}AES256\PYGZhy{}SHA384:ECDHE\PYGZhy{}RSA\PYGZhy{}AES256\PYGZhy{}SHA:ECDHE\PYGZhy{}ECDSA\PYGZhy{}AES256\PYGZhy{}SHA:DHE\PYGZhy{}RSA\PYGZhy{}AES128\PYGZhy{}SHA256:DHE\PYGZhy{}RSA\PYGZhy{}AES128\PYGZhy{}SHA:DHE\PYGZhy{}DSS\PYGZhy{}AES128\PYGZhy{}SHA256:DHE\PYGZhy{}RSA\PYGZhy{}AES256\PYGZhy{}SHA256:DHE\PYGZhy{}DSS\PYGZhy{}AES256\PYGZhy{}SHA:DHE\PYGZhy{}RSA\PYGZhy{}AES256\PYGZhy{}SHA:AES128\PYGZhy{}GCM\PYGZhy{}SHA256:AES256\PYGZhy{}GCM\PYGZhy{}SHA384:AES128\PYGZhy{}SHA256:AES256\PYGZhy{}SHA256:AES128\PYGZhy{}SHA:AES256\PYGZhy{}SHA:AES:CAMELLIA:DES\PYGZhy{}CBC3\PYGZhy{}SHA:!aNULL:!eNULL:!EXPORT:!DES:!RC4:!MD5:!PSK:!aECDH:!EDH\PYGZhy{}DSS\PYGZhy{}DES\PYGZhy{}CBC3\PYGZhy{}SHA:!EDH\PYGZhy{}RSA\PYGZhy{}DES\PYGZhy{}CBC3\PYGZhy{}SHA:!KRB5\PYGZhy{}DES\PYGZhy{}CBC3\PYGZhy{}SHA\PYGZsq{}}\PYG{p}{;}
 \PYG{k+kn}{ssl\PYGZus{}prefer\PYGZus{}server\PYGZus{}ciphers} \PYG{n+no}{on}\PYG{p}{;}

 \PYG{c+c1}{\PYGZsh{} log}
 \PYG{k+kn}{access\PYGZus{}log} \PYG{l+s}{/var/log/nginx/odoo.access.log}\PYG{p}{;}
 \PYG{k+kn}{error\PYGZus{}log} \PYG{l+s}{/var/log/nginx/odoo.error.log}\PYG{p}{;}

 \PYG{c+c1}{\PYGZsh{} Redirect longpoll requests to odoo longpolling port}
 \PYG{k+kn}{location} \PYG{l+s}{/longpolling} \PYG{p}{\PYGZob{}}
 \PYG{k+kn}{proxy\PYGZus{}pass} \PYG{l+s}{http://odoochat}\PYG{p}{;}
 \PYG{p}{\PYGZcb{}}

 \PYG{c+c1}{\PYGZsh{} Redirect requests to odoo backend server}
 \PYG{k+kn}{location} \PYG{l+s}{/} \PYG{p}{\PYGZob{}}
   \PYG{k+kn}{proxy\PYGZus{}redirect} \PYG{n+no}{off}\PYG{p}{;}
   \PYG{k+kn}{proxy\PYGZus{}pass} \PYG{l+s}{http://odoo}\PYG{p}{;}
 \PYG{p}{\PYGZcb{}}

 \PYG{c+c1}{\PYGZsh{} common gzip}
 \PYG{k+kn}{gzip\PYGZus{}types} \PYG{l+s}{text/css} \PYG{l+s}{text/less} \PYG{l+s}{text/plain} \PYG{l+s}{text/xml} \PYG{l+s}{application/xml} \PYG{l+s}{application/json} \PYG{l+s}{application/javascript}\PYG{p}{;}
 \PYG{k+kn}{gzip} \PYG{n+no}{on}\PYG{p}{;}
\PYG{p}{\PYGZcb{}}
\end{sphinxVerbatim}


\subsection{Odoo as a WSGI Application}
\label{\detokenize{setup/deploy:odoo-as-a-wsgi-application}}
It is also possible to mount Odoo as a standard \sphinxhref{http://wsgi.readthedocs.org/}{WSGI} application. Odoo
provides the base for a WSGI launcher script as \sphinxcode{\sphinxupquote{odoo-wsgi.example.py}}. That
script should be customized (possibly after copying it from the setup directory) to correctly set the
configuration directly in \sphinxcode{\sphinxupquote{odoo.tools.config}} rather than through the
command-line or a configuration file.

However the WSGI server will only expose the main HTTP endpoint for the web
client, website and webservice API. Because Odoo does not control the creation
of workers anymore it can not setup cron or livechat workers


\subsubsection{Cron Workers}
\label{\detokenize{setup/deploy:cron-workers}}
To run cron jobs for an Odoo deployment as a WSGI application requires
\begin{itemize}
\item {} 
A classical Odoo (run via \sphinxcode{\sphinxupquote{odoo-bin}})

\item {} 
Connected to the database in which cron jobs have to be run (via
{\hyperref[\detokenize{reference/cmdline:cmdoption-odoo-bin-d}]{\sphinxcrossref{\sphinxcode{\sphinxupquote{odoo-bin -d}}}}})

\item {} 
Which should not be exposed to the network. To ensure cron runners are not
network-accessible, it is possible to disable the built-in HTTP server
entirely with {\hyperref[\detokenize{reference/cmdline:cmdoption-odoo-bin-no-http}]{\sphinxcrossref{\sphinxcode{\sphinxupquote{odoo-bin -{-}no-http}}}}} or setting \sphinxcode{\sphinxupquote{http\_enable = False}}
in the configuration file

\end{itemize}


\subsubsection{LiveChat}
\label{\detokenize{setup/deploy:id8}}
The second problematic subsystem for WSGI deployments is the LiveChat: where
most HTTP connections are relatively short and quickly free up their worker
process for the next request, LiveChat require a long-lived connection for
each client in order to implement near-real-time notifications.

This is in conflict with the process-based worker model, as it will tie
up worker processes and prevent new users from accessing the system. However,
those long-lived connections do very little and mostly stay parked waiting for
notifications.

The solutions to support livechat/motifications in a WSGI application are:
\begin{itemize}
\item {} 
Deploy a threaded version of Odoo (instread of a process-based preforking
one) and redirect only requests to URLs starting with \sphinxcode{\sphinxupquote{/longpolling/}} to
that Odoo, this is the simplest and the longpolling URL can double up as
the cron instance.

\item {} 
Deploy an evented Odoo via \sphinxcode{\sphinxupquote{odoo-gevent}} and proxy requests starting
with \sphinxcode{\sphinxupquote{/longpolling/}} to
{\hyperref[\detokenize{reference/cmdline:cmdoption-odoo-bin-longpolling-port}]{\sphinxcrossref{\sphinxcode{\sphinxupquote{the longpolling port}}}}}.

\end{itemize}


\subsection{Serving Static Files}
\label{\detokenize{setup/deploy:serving-static-files}}
For development convenience, Odoo directly serves all static files in its
modules. This may not be ideal when it comes to performances, and static
files should generally be served by a static HTTP server.

Odoo static files live in each module’s \sphinxcode{\sphinxupquote{static/}} folder, so static files
can be served by intercepting all requests to \sphinxcode{\sphinxupquote{/\sphinxstyleemphasis{MODULE}/static/\sphinxstyleemphasis{FILE}}},
and looking up the right module (and file) in the various addons paths.


\subsection{Security}
\label{\detokenize{setup/deploy:security}}\label{\detokenize{setup/deploy:id9}}
For starters, keep in mind that securing an information system is a continuous process,
not a one-shot operation. At any moment, you will only be as secure as the weakest link
in your environment.

So please do not take this section as the ultimate list of measures that will prevent
all security problems. It’s only intended as a summary of the first important things
you should be sure to include in your security action plan. The rest will come
from best security practices for your operating system and distribution,
best practices in terms of users, passwords, and access control management, etc.

When deploying an internet-facing server, please be sure to consider the following
security-related topics:
\begin{itemize}
\item {} 
Always set a strong super-admin admin password, and restrict access to the database
management pages as soon as the system is set up. See {\hyperref[\detokenize{setup/deploy:db-manager-security}]{\sphinxcrossref{\DUrole{std,std-ref}{Database Manager Security}}}}.

\item {} 
Choose unique logins and strong passwords for all administrator accounts on all databases.
Do not use ‘admin’ as the login. Do not use those logins for day-to-day operations,
only for controlling/managing the installation.
\sphinxstyleemphasis{Never} use any default passwords like admin/admin, even for test/staging databases.

\item {} 
Do \sphinxstylestrong{not} install demo data on internet-facing servers. Databases with demo data contain
default logins and passwords that can be used to get into your systems and cause significant
trouble, even on staging/dev systems.

\item {} 
Use appropriate database filters ( {\hyperref[\detokenize{reference/cmdline:cmdoption-odoo-bin-db-filter}]{\sphinxcrossref{\sphinxcode{\sphinxupquote{-{-}db-filter}}}}})
to restrict the visibility of your databases according to the hostname.
See {\hyperref[\detokenize{setup/deploy:db-filter}]{\sphinxcrossref{\DUrole{std,std-ref}{dbfilter}}}}.
You may also use {\hyperref[\detokenize{reference/cmdline:cmdoption-odoo-bin-d}]{\sphinxcrossref{\sphinxcode{\sphinxupquote{-d}}}}} to provide your own (comma-separated)
list of available databases to filter from, instead of letting the system fetch
them all from the database backend.

\item {} 
Once your \sphinxcode{\sphinxupquote{db\_name}} and \sphinxcode{\sphinxupquote{db\_filter}} are configured and only match a single database
per hostname, you should set \sphinxcode{\sphinxupquote{list\_db}} configuration option to \sphinxcode{\sphinxupquote{False}}, to prevent
listing databases entirely, and to block access to the database management screens
(this is also exposed as the {\hyperref[\detokenize{reference/cmdline:cmdoption-odoo-bin-no-database-list}]{\sphinxcrossref{\sphinxcode{\sphinxupquote{-{-}no-database-list}}}}}
command-line option)

\item {} 
Make sure the PostgreSQL user ({\hyperref[\detokenize{reference/cmdline:cmdoption-odoo-bin-r}]{\sphinxcrossref{\sphinxcode{\sphinxupquote{-{-}db\_user}}}}}) is \sphinxstyleemphasis{not} a super-user,
and that your databases are owned by a different user. For example they could be owned by
the \sphinxcode{\sphinxupquote{postgres}} super-user if you are using a dedicated non-privileged \sphinxcode{\sphinxupquote{db\_user}}.
See also {\hyperref[\detokenize{setup/deploy:setup-deploy-odoo}]{\sphinxcrossref{\DUrole{std,std-ref}{Configuring Odoo}}}}.

\item {} 
Keep installations updated by regularly installing the latest builds,
either via GitHub or by downloading the latest version from
\sphinxurl{https://www.odoo.com/page/download} or \sphinxurl{http://nightly.odoo.com}

\item {} 
Configure your server in multi-process mode with proper limits matching your typical
usage (memory/CPU/timeouts). See also {\hyperref[\detokenize{setup/deploy:builtin-server}]{\sphinxcrossref{\DUrole{std,std-ref}{Builtin server}}}}.

\item {} 
Run Odoo behind a web server providing HTTPS termination with a valid SSL certificate,
in order to prevent eavesdropping on cleartext communications. SSL certificates are
cheap, and many free options exist.
Configure the web proxy to limit the size of requests, set appropriate timeouts,
and then enable the {\hyperref[\detokenize{reference/cmdline:cmdoption-odoo-bin-proxy-mode}]{\sphinxcrossref{\sphinxcode{\sphinxupquote{proxy mode}}}}} option.
See also {\hyperref[\detokenize{setup/deploy:https-proxy}]{\sphinxcrossref{\DUrole{std,std-ref}{HTTPS}}}}.

\item {} 
If you need to allow remote SSH access to your servers, make sure to set a strong password
for \sphinxstylestrong{all} accounts, not just \sphinxcode{\sphinxupquote{root}}. It is strongly recommended to entirely disable
password-based authentication, and only allow public key authentication. Also consider
restricting access via a VPN, allowing only trusted IPs in the firewall, and/or
running a brute-force detection system such as \sphinxcode{\sphinxupquote{fail2ban}} or equivalent.

\item {} 
Whenever possible, host your public-facing demo/test/staging instances on different
machines than the production ones. And apply the same security precautions as for
production.

\item {} 
If you are hosting multiple customers, isolate customer data and files from each other
using containers or appropriate “jail” techniques.

\item {} 
Setup daily backups of your databases and filestore data, and copy them to a remote
archiving server that is not accessible from the server itself.

\end{itemize}


\subsubsection{Database Manager Security}
\label{\detokenize{setup/deploy:database-manager-security}}\label{\detokenize{setup/deploy:db-manager-security}}
{\hyperref[\detokenize{setup/deploy:setup-deploy-odoo}]{\sphinxcrossref{\DUrole{std,std-ref}{Configuring Odoo}}}} mentioned \sphinxcode{\sphinxupquote{admin\_passwd}} in passing.

This setting is used on all database management screens (to create, delete,
dump or restore databases).

If the management screens must not be accessible at all, you should set \sphinxcode{\sphinxupquote{list\_db}}
configuration option to \sphinxcode{\sphinxupquote{False}}, to block access to all the database selection and
management screens. But be sure to setup an appropriate \sphinxcode{\sphinxupquote{db\_name}} parameter
(and optionally, \sphinxcode{\sphinxupquote{db\_filter}} too) so that the system can determine the target database
for each request, otherwise users will be blocked as they won’t be allowed to choose the
database themselves.

If the management screens must only be accessible from a selected set of machines,
use the proxy server’s features to block access to all routes starting with \sphinxcode{\sphinxupquote{/web/database}}
except (maybe) \sphinxcode{\sphinxupquote{/web/database/selector}} which displays the database-selection screen.

If the database-management screen should be left accessible, the
\sphinxcode{\sphinxupquote{admin\_passwd}} setting must be changed from its \sphinxcode{\sphinxupquote{admin}} default: this
password is checked before allowing database-alteration operations.

It should be stored securely, and should be generated randomly e.g.

\fvset{hllines={, ,}}%
\begin{sphinxVerbatim}[commandchars=\\\{\}]
\PYG{g+gp}{\PYGZdl{}} python3 \PYGZhy{}c \PYG{l+s+s1}{\PYGZsq{}import base64, os; print(base64.b64encode(os.urandom(24)))\PYGZsq{}}
\end{sphinxVerbatim}

which will generate a 32 characters pseudorandom printable string.


\subsection{Supported Browsers}
\label{\detokenize{setup/deploy:supported-browsers}}
Odoo is supported by multiple browsers for each of its versions. No
distinction is made according to the browser version in order to be
up-to-date. Odoo is supported on the current browser version. The list
of the supported browsers by Odoo version is the following:
\begin{itemize}
\item {} 
\sphinxstylestrong{Odoo 9:} IE11, Mozilla Firefox, Google Chrome, Safari, Microsoft Edge

\item {} 
\sphinxstylestrong{Odoo 10+:} Mozilla Firefox, Google Chrome, Safari, Microsoft Edge

\end{itemize}
\phantomsection\label{\detokenize{setup/deploy:postgresql-connection-settings}}

\section{Deploying with Content Delivery Networks}
\label{\detokenize{setup/cdn:posbox}}\label{\detokenize{setup/cdn::doc}}\label{\detokenize{setup/cdn:deploying-with-content-delivery-networks}}

\subsection{Deploying with KeyCDN}
\label{\detokenize{setup/cdn:deploying-with-keycdn}}\label{\detokenize{setup/cdn:reference-cdn-keycdn}}
This document will guide you through the setup of a \sphinxhref{https://www.keycdn.com}{KeyCDN} account with your
Odoo powered website.


\subsubsection{Step 1: Create a pull zone in the KeyCDN dashboard}
\label{\detokenize{setup/cdn:step-1-create-a-pull-zone-in-the-keycdn-dashboard}}
\noindent\sphinxincludegraphics{{keycdn_create_a_pull_zone}.png}

When creating the zone, enable the CORS option in the
\sphinxmenuselection{advanced features} submenu. (more on that later)

\noindent\sphinxincludegraphics{{keycdn_enable_CORS}.png}

Once done, you’ll have to wait a bit while \sphinxhref{https://www.keycdn.com}{KeyCDN} is crawling your website.

\noindent\sphinxincludegraphics{{keycdn_progressbar}.png}

\begin{sphinxadmonition}{note}{Note:}
a new URL has been generated for your Zone, in this case it is
\sphinxcode{\sphinxupquote{http://pulltest-b49.kxcdn.com}}
\end{sphinxadmonition}


\subsubsection{Step 2: Configure the odoo instance with your zone}
\label{\detokenize{setup/cdn:step-2-configure-the-odoo-instance-with-your-zone}}
In the Odoo back end, go to the \sphinxmenuselection{Website Settings}: menu, then
activate the CDN support and copy/paste your zone URL in the
\sphinxmenuselection{CDN Base URL} field. This field is only visible and configurable if
you have developer mode activated.

\noindent\sphinxincludegraphics{{odoo_cdn_base_url}.png}

Now your website is using the CDN for the resources matching the
\sphinxmenuselection{CDN filters} regular expressions.

You can have a look to the HTML of your website in order to check if the CDN
integration is properly working.

\noindent\sphinxincludegraphics{{odoo_check_your_html}.png}


\subsubsection{Why should I activate CORS?}
\label{\detokenize{setup/cdn:why-should-i-activate-cors}}
A security restriction in some browsers (Firefox and Chrome at time of writing)
prevents a remotely linked CSS file to fetch relative resources on this same
external server.

If you don’t activate the CORS option in the CDN zone, the more obvious
resulting problem on a default Odoo website will be the lack of font-awesome
icons because the font file declared in the font-awesome CSS won’t be loaded on
the remote server.

Here’s what you would see on your homepage in such a case:

\noindent\sphinxincludegraphics{{odoo_font_file_not_loaded}.png}

A security error message will also appear in the browser’s console:

\noindent\sphinxincludegraphics{{odoo_security_message}.png}

Enabling the CORS option in the CDN fixes this issue.


\section{From Community to Enterprise}
\label{\detokenize{setup/enterprise:keycdn}}\label{\detokenize{setup/enterprise:from-community-to-enterprise}}\label{\detokenize{setup/enterprise::doc}}\label{\detokenize{setup/enterprise:setup-enterprise}}
Depending on your current installation, there are multiple ways to upgrade
your community version.
In any case the basic guidelines are:
\begin{itemize}
\item {} 
Backup your community database

\noindent\sphinxincludegraphics{{db_manager}.png}

\item {} 
Shutdown your server

\item {} 
Install the web\_enterprise module

\item {} 
Restart your server

\item {} 
Enter your Odoo Enterprise Subscription code

\end{itemize}

\noindent\sphinxincludegraphics{{enterprise_code}.png}


\subsection{On Linux, using an installer}
\label{\detokenize{setup/enterprise:on-linux-using-an-installer}}\begin{itemize}
\item {} 
Backup your community database

\item {} 
Stop the odoo service

\fvset{hllines={, ,}}%
\begin{sphinxVerbatim}[commandchars=\\\{\}]
\PYG{g+gp}{\PYGZdl{}} sudo service odoo stop
\end{sphinxVerbatim}

\item {} 
Install the enterprise .deb (it should install over the community package)

\fvset{hllines={, ,}}%
\begin{sphinxVerbatim}[commandchars=\\\{\}]
\PYG{g+gp}{\PYGZdl{}} sudo dpkg \PYGZhy{}i \PYGZlt{}path\PYGZus{}to\PYGZus{}enterprise\PYGZus{}deb\PYGZgt{}
\end{sphinxVerbatim}

\item {} 
Update your database to the enterprise packages using

\fvset{hllines={, ,}}%
\begin{sphinxVerbatim}[commandchars=\\\{\}]
\PYG{g+gp}{\PYGZdl{}} python3 /usr/bin/odoo.py \PYGZhy{}d \PYGZlt{}database\PYGZus{}name\PYGZgt{} \PYGZhy{}i web\PYGZus{}enterprise \PYGZhy{}\PYGZhy{}stop\PYGZhy{}after\PYGZhy{}init
\end{sphinxVerbatim}

\item {} 
You should be able to connect to your Odoo Enterprise instance using your usual mean of identification.
You can then link your database with your Odoo Enterprise Subscription by entering the code you received
by e-mail in the form input

\end{itemize}


\subsection{On Linux, using the source code}
\label{\detokenize{setup/enterprise:on-linux-using-the-source-code}}
There are many ways to launch your server when using sources, and you probably
have your own favourite. You may need to adapt sections to your usual workflow.
\begin{itemize}
\item {} 
Shutdown your server

\item {} 
Backup your community database

\item {} 
Update the \sphinxcode{\sphinxupquote{-{-}addons-path}} parameter of your launch command (see {\hyperref[\detokenize{setup/install:setup-install-source}]{\sphinxcrossref{\DUrole{std,std-ref}{Source Install}}}})

\item {} 
Install the web\_enterprise module by using

\fvset{hllines={, ,}}%
\begin{sphinxVerbatim}[commandchars=\\\{\}]
\PYG{g+gp}{\PYGZdl{}} \PYGZhy{}d \PYGZlt{}database\PYGZus{}name\PYGZgt{} \PYGZhy{}i web\PYGZus{}enterprise \PYGZhy{}\PYGZhy{}stop\PYGZhy{}after\PYGZhy{}init
\end{sphinxVerbatim}

Depending on the size of your database, this may take some time.

\item {} 
Restart your server with the updated addons path of point 3.
You should be able to connect to your instance. You can then link your database with your
Odoo Enterprise Subscription by entering the code you received by e-mail in the form input

\end{itemize}


\subsection{On Windows}
\label{\detokenize{setup/enterprise:on-windows}}\begin{itemize}
\item {} 
Backup your community database

\item {} 
Uninstall Odoo Community (using the Uninstall executable in the installation folder) -
PostgreSQL will remain installed

\noindent\sphinxincludegraphics{{windows_uninstall}.png}

\item {} 
Launch the Odoo Enterprise Installer and follow the steps normally. When choosing
the installation path, you can set the folder of the Community installation
(this folder still contains the PostgreSQL installation).
Uncheck \sphinxcode{\sphinxupquote{Start Odoo}} at the end of the installation

\noindent\sphinxincludegraphics{{windows_setup}.png}

\item {} 
Using a command window, update your Odoo Database using this command (from the Odoo
installation path, in the server subfolder)

\fvset{hllines={, ,}}%
\begin{sphinxVerbatim}[commandchars=\\\{\}]
\PYG{g+gp}{\PYGZdl{}} odoo.exe \PYGZhy{}d \PYGZlt{}database\PYGZus{}name\PYGZgt{} \PYGZhy{}i web\PYGZus{}enterprise \PYGZhy{}\PYGZhy{}stop\PYGZhy{}after\PYGZhy{}init
\end{sphinxVerbatim}

\item {} 
No need to manually launch the server, the service is running.
You should be able to connect to your Odoo Enterprise instance using your usual
mean of identification. You can then link your database with your Odoo Enterprise
Subscription by entering the code you received by e-mail in the form input

\end{itemize}


\chapter{Reference}
\label{\detokenize{reference::doc}}\label{\detokenize{reference:reference}}

\section{ORM API}
\label{\detokenize{reference/orm:orm-api}}\label{\detokenize{reference/orm::doc}}\label{\detokenize{reference/orm:reference-orm}}

\subsection{Recordsets}
\label{\detokenize{reference/orm:recordsets}}
\DUrole{versionmodified}{New in version 8.0: }This page documents the New API added in Odoo 8.0 which should be the
primary development API going forward. It also provides information about
porting from or bridging with the “old API” of versions 7 and earlier, but
does not explicitly document that API. See the old documentation for that.

Interaction with models and records is performed through recordsets, a sorted
set of records of the same model.

\begin{sphinxadmonition}{warning}{Warning:}
contrary to what the name implies, it is currently possible for
recordsets to contain duplicates. This may change in the future.
\end{sphinxadmonition}

Methods defined on a model are executed on a recordset, and their \sphinxcode{\sphinxupquote{self}} is
a recordset:

\fvset{hllines={, ,}}%
\begin{sphinxVerbatim}[commandchars=\\\{\}]
\PYG{k}{class} \PYG{n+nc}{AModel}\PYG{p}{(}\PYG{n}{models}\PYG{o}{.}\PYG{n}{Model}\PYG{p}{)}\PYG{p}{:}
    \PYG{n}{\PYGZus{}name} \PYG{o}{=} \PYG{l+s+s1}{\PYGZsq{}}\PYG{l+s+s1}{a.model}\PYG{l+s+s1}{\PYGZsq{}}
    \PYG{k}{def} \PYG{n+nf}{a\PYGZus{}method}\PYG{p}{(}\PYG{n+nb+bp}{self}\PYG{p}{)}\PYG{p}{:}
        \PYG{c+c1}{\PYGZsh{} self can be anywhere between 0 records and all records in the}
        \PYG{c+c1}{\PYGZsh{} database}
        \PYG{n+nb+bp}{self}\PYG{o}{.}\PYG{n}{do\PYGZus{}operation}\PYG{p}{(}\PYG{p}{)}
\end{sphinxVerbatim}

Iterating on a recordset will yield new sets of \sphinxstyleemphasis{a single record}
(“singletons”), much like iterating on a Python string yields strings of a
single characters:

\fvset{hllines={, ,}}%
\begin{sphinxVerbatim}[commandchars=\\\{\}]
\PYG{k}{def} \PYG{n+nf}{do\PYGZus{}operation}\PYG{p}{(}\PYG{n+nb+bp}{self}\PYG{p}{)}\PYG{p}{:}
    \PYG{n+nb}{print} \PYG{n+nb+bp}{self} \PYG{c+c1}{\PYGZsh{} =\PYGZgt{} a.model(1, 2, 3, 4, 5)}
    \PYG{k}{for} \PYG{n}{record} \PYG{o+ow}{in} \PYG{n+nb+bp}{self}\PYG{p}{:}
        \PYG{n+nb}{print} \PYG{n}{record} \PYG{c+c1}{\PYGZsh{} =\PYGZgt{} a.model(1), then a.model(2), then a.model(3), ...}
\end{sphinxVerbatim}


\subsubsection{Field access}
\label{\detokenize{reference/orm:field-access}}
Recordsets provide an “Active Record” interface: model fields can be read and
written directly from the record as attributes, but only on singletons
(single-record recordsets).
Field values can also be accessed like dict items, which is more elegant and
safer than \sphinxcode{\sphinxupquote{getattr()}} for dynamic field names.
Setting a field’s value triggers an update to the database:

\fvset{hllines={, ,}}%
\begin{sphinxVerbatim}[commandchars=\\\{\}]
\PYG{g+gp}{\PYGZgt{}\PYGZgt{}\PYGZgt{} }\PYG{n}{record}\PYG{o}{.}\PYG{n}{name}
\PYG{g+go}{Example Name}
\PYG{g+gp}{\PYGZgt{}\PYGZgt{}\PYGZgt{} }\PYG{n}{record}\PYG{o}{.}\PYG{n}{company\PYGZus{}id}\PYG{o}{.}\PYG{n}{name}
\PYG{g+go}{Company Name}
\PYG{g+gp}{\PYGZgt{}\PYGZgt{}\PYGZgt{} }\PYG{n}{record}\PYG{o}{.}\PYG{n}{name} \PYG{o}{=} \PYG{l+s+s2}{\PYGZdq{}}\PYG{l+s+s2}{Bob}\PYG{l+s+s2}{\PYGZdq{}}
\PYG{g+gp}{\PYGZgt{}\PYGZgt{}\PYGZgt{} }\PYG{n}{field} \PYG{o}{=} \PYG{l+s+s2}{\PYGZdq{}}\PYG{l+s+s2}{name}\PYG{l+s+s2}{\PYGZdq{}}
\PYG{g+gp}{\PYGZgt{}\PYGZgt{}\PYGZgt{} }\PYG{n}{record}\PYG{p}{[}\PYG{n}{field}\PYG{p}{]}
\PYG{g+go}{Bob}
\end{sphinxVerbatim}

Trying to read or write a field on multiple records will raise an error.

Accessing a relational field ({\hyperref[\detokenize{reference/orm:odoo.fields.Many2one}]{\sphinxcrossref{\sphinxcode{\sphinxupquote{Many2one}}}}},
{\hyperref[\detokenize{reference/orm:odoo.fields.One2many}]{\sphinxcrossref{\sphinxcode{\sphinxupquote{One2many}}}}}, {\hyperref[\detokenize{reference/orm:odoo.fields.Many2many}]{\sphinxcrossref{\sphinxcode{\sphinxupquote{Many2many}}}}})
\sphinxstyleemphasis{always} returns a recordset, empty if the field is not set.

\begin{sphinxadmonition}{danger}{Danger:}
each assignment to a field triggers a database update, when setting
multiple fields at the same time or setting fields on multiple records
(to the same value), use {\hyperref[\detokenize{reference/orm:odoo.models.Model.write}]{\sphinxcrossref{\sphinxcode{\sphinxupquote{write()}}}}}:

\fvset{hllines={, ,}}%
\begin{sphinxVerbatim}[commandchars=\\\{\}]
\PYG{c+c1}{\PYGZsh{} 3 * len(records) database updates}
\PYG{k}{for} \PYG{n}{record} \PYG{o+ow}{in} \PYG{n}{records}\PYG{p}{:}
    \PYG{n}{record}\PYG{o}{.}\PYG{n}{a} \PYG{o}{=} \PYG{l+m+mi}{1}
    \PYG{n}{record}\PYG{o}{.}\PYG{n}{b} \PYG{o}{=} \PYG{l+m+mi}{2}
    \PYG{n}{record}\PYG{o}{.}\PYG{n}{c} \PYG{o}{=} \PYG{l+m+mi}{3}

\PYG{c+c1}{\PYGZsh{} len(records) database updates}
\PYG{k}{for} \PYG{n}{record} \PYG{o+ow}{in} \PYG{n}{records}\PYG{p}{:}
    \PYG{n}{record}\PYG{o}{.}\PYG{n}{write}\PYG{p}{(}\PYG{p}{\PYGZob{}}\PYG{l+s+s1}{\PYGZsq{}}\PYG{l+s+s1}{a}\PYG{l+s+s1}{\PYGZsq{}}\PYG{p}{:} \PYG{l+m+mi}{1}\PYG{p}{,} \PYG{l+s+s1}{\PYGZsq{}}\PYG{l+s+s1}{b}\PYG{l+s+s1}{\PYGZsq{}}\PYG{p}{:} \PYG{l+m+mi}{2}\PYG{p}{,} \PYG{l+s+s1}{\PYGZsq{}}\PYG{l+s+s1}{c}\PYG{l+s+s1}{\PYGZsq{}}\PYG{p}{:} \PYG{l+m+mi}{3}\PYG{p}{\PYGZcb{}}\PYG{p}{)}

\PYG{c+c1}{\PYGZsh{} 1 database update}
\PYG{n}{records}\PYG{o}{.}\PYG{n}{write}\PYG{p}{(}\PYG{p}{\PYGZob{}}\PYG{l+s+s1}{\PYGZsq{}}\PYG{l+s+s1}{a}\PYG{l+s+s1}{\PYGZsq{}}\PYG{p}{:} \PYG{l+m+mi}{1}\PYG{p}{,} \PYG{l+s+s1}{\PYGZsq{}}\PYG{l+s+s1}{b}\PYG{l+s+s1}{\PYGZsq{}}\PYG{p}{:} \PYG{l+m+mi}{2}\PYG{p}{,} \PYG{l+s+s1}{\PYGZsq{}}\PYG{l+s+s1}{c}\PYG{l+s+s1}{\PYGZsq{}}\PYG{p}{:} \PYG{l+m+mi}{3}\PYG{p}{\PYGZcb{}}\PYG{p}{)}
\end{sphinxVerbatim}
\end{sphinxadmonition}


\subsubsection{Record cache and prefetching}
\label{\detokenize{reference/orm:record-cache-and-prefetching}}
Odoo maintains a cache for the fields of the records, so that not every field
access issues a database request, which would be terrible for performance. The
following example queries the database only for the first statement:

\fvset{hllines={, ,}}%
\begin{sphinxVerbatim}[commandchars=\\\{\}]
\PYG{n}{record}\PYG{o}{.}\PYG{n}{name}             \PYG{c+c1}{\PYGZsh{} first access reads value from database}
\PYG{n}{record}\PYG{o}{.}\PYG{n}{name}             \PYG{c+c1}{\PYGZsh{} second access gets value from cache}
\end{sphinxVerbatim}

To avoid reading one field on one record at a time, Odoo \sphinxstyleemphasis{prefetches} records
and fields following some heuristics to get good performance. Once a field must
be read on a given record, the ORM actually reads that field on a larger
recordset, and stores the returned values in cache for later use. The prefetched
recordset is usually the recordset from which the record comes by iteration.
Moreover, all simple stored fields (boolean, integer, float, char, text, date,
datetime, selection, many2one) are fetched altogether; they correspond to the
columns of the model’s table, and are fetched efficiently in the same query.

Consider the following example, where \sphinxcode{\sphinxupquote{partners}} is a recordset of 1000
records. Without prefetching, the loop would make 2000 queries to the database.
With prefetching, only one query is made:

\fvset{hllines={, ,}}%
\begin{sphinxVerbatim}[commandchars=\\\{\}]
\PYG{k}{for} \PYG{n}{partner} \PYG{o+ow}{in} \PYG{n}{partners}\PYG{p}{:}
    \PYG{n+nb}{print} \PYG{n}{partner}\PYG{o}{.}\PYG{n}{name}          \PYG{c+c1}{\PYGZsh{} first pass prefetches \PYGZsq{}name\PYGZsq{} and \PYGZsq{}lang\PYGZsq{}}
                                \PYG{c+c1}{\PYGZsh{} (and other fields) on all \PYGZsq{}partners\PYGZsq{}}
    \PYG{n+nb}{print} \PYG{n}{partner}\PYG{o}{.}\PYG{n}{lang}
\end{sphinxVerbatim}

The prefetching also works on \sphinxstyleemphasis{secondary records}: when relational fields are
read, their values (which are records) are  subscribed for future prefetching.
Accessing one of those secondary records prefetches all secondary records from
the same model. This makes the following example generate only two queries, one
for partners and one for countries:

\fvset{hllines={, ,}}%
\begin{sphinxVerbatim}[commandchars=\\\{\}]
\PYG{n}{countries} \PYG{o}{=} \PYG{n+nb}{set}\PYG{p}{(}\PYG{p}{)}
\PYG{k}{for} \PYG{n}{partner} \PYG{o+ow}{in} \PYG{n}{partners}\PYG{p}{:}
    \PYG{n}{country} \PYG{o}{=} \PYG{n}{partner}\PYG{o}{.}\PYG{n}{country\PYGZus{}id}        \PYG{c+c1}{\PYGZsh{} first pass prefetches all partners}
    \PYG{n}{countries}\PYG{o}{.}\PYG{n}{add}\PYG{p}{(}\PYG{n}{country}\PYG{o}{.}\PYG{n}{name}\PYG{p}{)}         \PYG{c+c1}{\PYGZsh{} first pass prefetches all countries}
\end{sphinxVerbatim}


\subsubsection{Set operations}
\label{\detokenize{reference/orm:set-operations}}
Recordsets are immutable, but sets of the same model can be combined using
various set operations, returning new recordsets. Set operations do \sphinxstyleemphasis{not}
preserve order.
\begin{itemize}
\item {} 
\sphinxcode{\sphinxupquote{record in set}} returns whether \sphinxcode{\sphinxupquote{record}} (which must be a 1-element
recordset) is present in \sphinxcode{\sphinxupquote{set}}. \sphinxcode{\sphinxupquote{record not in set}} is the inverse
operation

\item {} 
\sphinxcode{\sphinxupquote{set1 \textless{}= set2}} and \sphinxcode{\sphinxupquote{set1 \textless{} set2}} return whether \sphinxcode{\sphinxupquote{set1}} is a subset
of \sphinxcode{\sphinxupquote{set2}} (resp. strict)

\item {} 
\sphinxcode{\sphinxupquote{set1 \textgreater{}= set2}} and \sphinxcode{\sphinxupquote{set1 \textgreater{} set2}} return whether \sphinxcode{\sphinxupquote{set1}} is a superset
of \sphinxcode{\sphinxupquote{set2}} (resp. strict)

\item {} 
\sphinxcode{\sphinxupquote{set1 \textbar{} set2}} returns the union of the two recordsets, a new recordset
containing all records present in either source

\item {} 
\sphinxcode{\sphinxupquote{set1 \& set2}} returns the intersection of two recordsets, a new recordset
containing only records present in both sources

\item {} 
\sphinxcode{\sphinxupquote{set1 - set2}} returns a new recordset containing only records of \sphinxcode{\sphinxupquote{set1}}
which are \sphinxstyleemphasis{not} in \sphinxcode{\sphinxupquote{set2}}

\end{itemize}


\subsubsection{Other recordset operations}
\label{\detokenize{reference/orm:other-recordset-operations}}
Recordsets are iterable so the usual Python tools are available for
transformation (\sphinxhref{https://docs.python.org/3/library/functions.html\#map}{\sphinxcode{\sphinxupquote{map()}}}, \sphinxhref{https://docs.python.org/3/library/functions.html\#sorted}{\sphinxcode{\sphinxupquote{sorted()}}},
\sphinxcode{\sphinxupquote{itertools.ifilter}}, …) however these return either a
\sphinxhref{https://docs.python.org/3/library/stdtypes.html\#list}{\sphinxcode{\sphinxupquote{list}}} or an \sphinxhref{https://docs.python.org/3/glossary.html\#term-iterator}{iterator}, removing the ability to
call methods on their result, or to use set operations.

Recordsets therefore provide these operations returning recordsets themselves
(when possible):
\begin{description}
\item[{{\hyperref[\detokenize{reference/orm:odoo.models.Model.filtered}]{\sphinxcrossref{\sphinxcode{\sphinxupquote{filtered()}}}}}}] \leavevmode
returns a recordset containing only records satisfying the provided
predicate function. The predicate can also be a string to filter by a
field being true or false:

\fvset{hllines={, ,}}%
\begin{sphinxVerbatim}[commandchars=\\\{\}]
\PYG{c+c1}{\PYGZsh{} only keep records whose company is the current user\PYGZsq{}s}
\PYG{n}{records}\PYG{o}{.}\PYG{n}{filtered}\PYG{p}{(}\PYG{k}{lambda} \PYG{n}{r}\PYG{p}{:} \PYG{n}{r}\PYG{o}{.}\PYG{n}{company\PYGZus{}id} \PYG{o}{==} \PYG{n}{user}\PYG{o}{.}\PYG{n}{company\PYGZus{}id}\PYG{p}{)}

\PYG{c+c1}{\PYGZsh{} only keep records whose partner is a company}
\PYG{n}{records}\PYG{o}{.}\PYG{n}{filtered}\PYG{p}{(}\PYG{l+s+s2}{\PYGZdq{}}\PYG{l+s+s2}{partner\PYGZus{}id.is\PYGZus{}company}\PYG{l+s+s2}{\PYGZdq{}}\PYG{p}{)}
\end{sphinxVerbatim}

\item[{{\hyperref[\detokenize{reference/orm:odoo.models.Model.sorted}]{\sphinxcrossref{\sphinxcode{\sphinxupquote{sorted()}}}}}}] \leavevmode
returns a recordset sorted by the provided key function. If no key
is provided, use the model’s default sort order:

\fvset{hllines={, ,}}%
\begin{sphinxVerbatim}[commandchars=\\\{\}]
\PYG{c+c1}{\PYGZsh{} sort records by name}
\PYG{n}{records}\PYG{o}{.}\PYG{n}{sorted}\PYG{p}{(}\PYG{n}{key}\PYG{o}{=}\PYG{k}{lambda} \PYG{n}{r}\PYG{p}{:} \PYG{n}{r}\PYG{o}{.}\PYG{n}{name}\PYG{p}{)}
\end{sphinxVerbatim}

\item[{{\hyperref[\detokenize{reference/orm:odoo.models.Model.mapped}]{\sphinxcrossref{\sphinxcode{\sphinxupquote{mapped()}}}}}}] \leavevmode
applies the provided function to each record in the recordset, returns
a recordset if the results are recordsets:

\fvset{hllines={, ,}}%
\begin{sphinxVerbatim}[commandchars=\\\{\}]
\PYG{c+c1}{\PYGZsh{} returns a list of summing two fields for each record in the set}
\PYG{n}{records}\PYG{o}{.}\PYG{n}{mapped}\PYG{p}{(}\PYG{k}{lambda} \PYG{n}{r}\PYG{p}{:} \PYG{n}{r}\PYG{o}{.}\PYG{n}{field1} \PYG{o}{+} \PYG{n}{r}\PYG{o}{.}\PYG{n}{field2}\PYG{p}{)}
\end{sphinxVerbatim}

The provided function can be a string to get field values:

\fvset{hllines={, ,}}%
\begin{sphinxVerbatim}[commandchars=\\\{\}]
\PYG{c+c1}{\PYGZsh{} returns a list of names}
\PYG{n}{records}\PYG{o}{.}\PYG{n}{mapped}\PYG{p}{(}\PYG{l+s+s1}{\PYGZsq{}}\PYG{l+s+s1}{name}\PYG{l+s+s1}{\PYGZsq{}}\PYG{p}{)}

\PYG{c+c1}{\PYGZsh{} returns a recordset of partners}
\PYG{n}{record}\PYG{o}{.}\PYG{n}{mapped}\PYG{p}{(}\PYG{l+s+s1}{\PYGZsq{}}\PYG{l+s+s1}{partner\PYGZus{}id}\PYG{l+s+s1}{\PYGZsq{}}\PYG{p}{)}

\PYG{c+c1}{\PYGZsh{} returns the union of all partner banks, with duplicates removed}
\PYG{n}{record}\PYG{o}{.}\PYG{n}{mapped}\PYG{p}{(}\PYG{l+s+s1}{\PYGZsq{}}\PYG{l+s+s1}{partner\PYGZus{}id.bank\PYGZus{}ids}\PYG{l+s+s1}{\PYGZsq{}}\PYG{p}{)}
\end{sphinxVerbatim}

\end{description}


\subsection{Environment}
\label{\detokenize{reference/orm:environment}}
The \sphinxcode{\sphinxupquote{Environment}} stores various contextual data used by
the ORM: the database cursor (for database queries), the current user
(for access rights checking) and the current context (storing arbitrary
metadata). The environment also stores caches.

All recordsets have an environment, which is immutable, can be accessed
using \sphinxcode{\sphinxupquote{env}} and gives access to the current user
(\sphinxcode{\sphinxupquote{user}}), the cursor
(\sphinxcode{\sphinxupquote{cr}}) or the context
(\sphinxcode{\sphinxupquote{context}}):

\fvset{hllines={, ,}}%
\begin{sphinxVerbatim}[commandchars=\\\{\}]
\PYG{g+gp}{\PYGZgt{}\PYGZgt{}\PYGZgt{} }\PYG{n}{records}\PYG{o}{.}\PYG{n}{env}
\PYG{g+go}{\PYGZlt{}Environment object ...\PYGZgt{}}
\PYG{g+gp}{\PYGZgt{}\PYGZgt{}\PYGZgt{} }\PYG{n}{records}\PYG{o}{.}\PYG{n}{env}\PYG{o}{.}\PYG{n}{user}
\PYG{g+go}{res.user(3)}
\PYG{g+gp}{\PYGZgt{}\PYGZgt{}\PYGZgt{} }\PYG{n}{records}\PYG{o}{.}\PYG{n}{env}\PYG{o}{.}\PYG{n}{cr}
\PYG{g+go}{\PYGZlt{}Cursor object ...)}
\end{sphinxVerbatim}

When creating a recordset from an other recordset, the environment is
inherited. The environment can be used to get an empty recordset in an
other model, and query that model:

\fvset{hllines={, ,}}%
\begin{sphinxVerbatim}[commandchars=\\\{\}]
\PYG{g+gp}{\PYGZgt{}\PYGZgt{}\PYGZgt{} }\PYG{n+nb+bp}{self}\PYG{o}{.}\PYG{n}{env}\PYG{p}{[}\PYG{l+s+s1}{\PYGZsq{}}\PYG{l+s+s1}{res.partner}\PYG{l+s+s1}{\PYGZsq{}}\PYG{p}{]}
\PYG{g+go}{res.partner}
\PYG{g+gp}{\PYGZgt{}\PYGZgt{}\PYGZgt{} }\PYG{n+nb+bp}{self}\PYG{o}{.}\PYG{n}{env}\PYG{p}{[}\PYG{l+s+s1}{\PYGZsq{}}\PYG{l+s+s1}{res.partner}\PYG{l+s+s1}{\PYGZsq{}}\PYG{p}{]}\PYG{o}{.}\PYG{n}{search}\PYG{p}{(}\PYG{p}{[}\PYG{p}{[}\PYG{l+s+s1}{\PYGZsq{}}\PYG{l+s+s1}{is\PYGZus{}company}\PYG{l+s+s1}{\PYGZsq{}}\PYG{p}{,} \PYG{l+s+s1}{\PYGZsq{}}\PYG{l+s+s1}{=}\PYG{l+s+s1}{\PYGZsq{}}\PYG{p}{,} \PYG{k+kc}{True}\PYG{p}{]}\PYG{p}{,} \PYG{p}{[}\PYG{l+s+s1}{\PYGZsq{}}\PYG{l+s+s1}{customer}\PYG{l+s+s1}{\PYGZsq{}}\PYG{p}{,} \PYG{l+s+s1}{\PYGZsq{}}\PYG{l+s+s1}{=}\PYG{l+s+s1}{\PYGZsq{}}\PYG{p}{,} \PYG{k+kc}{True}\PYG{p}{]}\PYG{p}{]}\PYG{p}{)}
\PYG{g+go}{res.partner(7, 18, 12, 14, 17, 19, 8, 31, 26, 16, 13, 20, 30, 22, 29, 15, 23, 28, 74)}
\end{sphinxVerbatim}


\subsubsection{Altering the environment}
\label{\detokenize{reference/orm:altering-the-environment}}
The environment can be customized from a recordset. This returns a new
version of the recordset using the altered environment.
\begin{description}
\item[{{\hyperref[\detokenize{reference/orm:odoo.models.Model.sudo}]{\sphinxcrossref{\sphinxcode{\sphinxupquote{sudo()}}}}}}] \leavevmode
creates a new environment with the provided user set, uses the
administrator if none is provided (to bypass access rights/rules in safe
contexts), returns a copy of the recordset it is called on using the
new environment:

\fvset{hllines={, ,}}%
\begin{sphinxVerbatim}[commandchars=\\\{\}]
\PYG{c+c1}{\PYGZsh{} create partner object as administrator}
\PYG{n}{env}\PYG{p}{[}\PYG{l+s+s1}{\PYGZsq{}}\PYG{l+s+s1}{res.partner}\PYG{l+s+s1}{\PYGZsq{}}\PYG{p}{]}\PYG{o}{.}\PYG{n}{sudo}\PYG{p}{(}\PYG{p}{)}\PYG{o}{.}\PYG{n}{create}\PYG{p}{(}\PYG{p}{\PYGZob{}}\PYG{l+s+s1}{\PYGZsq{}}\PYG{l+s+s1}{name}\PYG{l+s+s1}{\PYGZsq{}}\PYG{p}{:} \PYG{l+s+s2}{\PYGZdq{}}\PYG{l+s+s2}{A Partner}\PYG{l+s+s2}{\PYGZdq{}}\PYG{p}{\PYGZcb{}}\PYG{p}{)}

\PYG{c+c1}{\PYGZsh{} list partners visible by the \PYGZdq{}public\PYGZdq{} user}
\PYG{n}{public} \PYG{o}{=} \PYG{n}{env}\PYG{o}{.}\PYG{n}{ref}\PYG{p}{(}\PYG{l+s+s1}{\PYGZsq{}}\PYG{l+s+s1}{base.public\PYGZus{}user}\PYG{l+s+s1}{\PYGZsq{}}\PYG{p}{)}
\PYG{n}{env}\PYG{p}{[}\PYG{l+s+s1}{\PYGZsq{}}\PYG{l+s+s1}{res.partner}\PYG{l+s+s1}{\PYGZsq{}}\PYG{p}{]}\PYG{o}{.}\PYG{n}{sudo}\PYG{p}{(}\PYG{n}{public}\PYG{p}{)}\PYG{o}{.}\PYG{n}{search}\PYG{p}{(}\PYG{p}{[}\PYG{p}{]}\PYG{p}{)}
\end{sphinxVerbatim}

\item[{{\hyperref[\detokenize{reference/orm:odoo.models.Model.with_context}]{\sphinxcrossref{\sphinxcode{\sphinxupquote{with\_context()}}}}}}] \leavevmode\begin{enumerate}
\item {} 
can take a single positional parameter, which replaces the current
environment’s context

\item {} 
can take any number of parameters by keyword, which are added to either
the current environment’s context or the context set during step 1

\end{enumerate}

\fvset{hllines={, ,}}%
\begin{sphinxVerbatim}[commandchars=\\\{\}]
\PYG{c+c1}{\PYGZsh{} look for partner, or create one with specified timezone if none is}
\PYG{c+c1}{\PYGZsh{} found}
\PYG{n}{env}\PYG{p}{[}\PYG{l+s+s1}{\PYGZsq{}}\PYG{l+s+s1}{res.partner}\PYG{l+s+s1}{\PYGZsq{}}\PYG{p}{]}\PYG{o}{.}\PYG{n}{with\PYGZus{}context}\PYG{p}{(}\PYG{n}{tz}\PYG{o}{=}\PYG{n}{a\PYGZus{}tz}\PYG{p}{)}\PYG{o}{.}\PYG{n}{find\PYGZus{}or\PYGZus{}create}\PYG{p}{(}\PYG{n}{email\PYGZus{}address}\PYG{p}{)}
\end{sphinxVerbatim}

\item[{{\hyperref[\detokenize{reference/orm:odoo.models.Model.with_env}]{\sphinxcrossref{\sphinxcode{\sphinxupquote{with\_env()}}}}}}] \leavevmode
replaces the existing environment entirely

\end{description}


\subsection{Common ORM methods}
\label{\detokenize{reference/orm:common-orm-methods}}\begin{description}
\item[{{\hyperref[\detokenize{reference/orm:odoo.models.Model.search}]{\sphinxcrossref{\sphinxcode{\sphinxupquote{search()}}}}}}] \leavevmode
Takes a {\hyperref[\detokenize{reference/orm:reference-orm-domains}]{\sphinxcrossref{\DUrole{std,std-ref}{search domain}}}}, returns a recordset
of matching records. Can return a subset of matching records (\sphinxcode{\sphinxupquote{offset}}
and \sphinxcode{\sphinxupquote{limit}} parameters) and be ordered (\sphinxcode{\sphinxupquote{order}} parameter):

\fvset{hllines={, ,}}%
\begin{sphinxVerbatim}[commandchars=\\\{\}]
\PYG{g+gp}{\PYGZgt{}\PYGZgt{}\PYGZgt{} }\PYG{c+c1}{\PYGZsh{} searches the current model}
\PYG{g+gp}{\PYGZgt{}\PYGZgt{}\PYGZgt{} }\PYG{n+nb+bp}{self}\PYG{o}{.}\PYG{n}{search}\PYG{p}{(}\PYG{p}{[}\PYG{p}{(}\PYG{l+s+s1}{\PYGZsq{}}\PYG{l+s+s1}{is\PYGZus{}company}\PYG{l+s+s1}{\PYGZsq{}}\PYG{p}{,} \PYG{l+s+s1}{\PYGZsq{}}\PYG{l+s+s1}{=}\PYG{l+s+s1}{\PYGZsq{}}\PYG{p}{,} \PYG{k+kc}{True}\PYG{p}{)}\PYG{p}{,} \PYG{p}{(}\PYG{l+s+s1}{\PYGZsq{}}\PYG{l+s+s1}{customer}\PYG{l+s+s1}{\PYGZsq{}}\PYG{p}{,} \PYG{l+s+s1}{\PYGZsq{}}\PYG{l+s+s1}{=}\PYG{l+s+s1}{\PYGZsq{}}\PYG{p}{,} \PYG{k+kc}{True}\PYG{p}{)}\PYG{p}{]}\PYG{p}{)}
\PYG{g+go}{res.partner(7, 18, 12, 14, 17, 19, 8, 31, 26, 16, 13, 20, 30, 22, 29, 15, 23, 28, 74)}
\PYG{g+gp}{\PYGZgt{}\PYGZgt{}\PYGZgt{} }\PYG{n+nb+bp}{self}\PYG{o}{.}\PYG{n}{search}\PYG{p}{(}\PYG{p}{[}\PYG{p}{(}\PYG{l+s+s1}{\PYGZsq{}}\PYG{l+s+s1}{is\PYGZus{}company}\PYG{l+s+s1}{\PYGZsq{}}\PYG{p}{,} \PYG{l+s+s1}{\PYGZsq{}}\PYG{l+s+s1}{=}\PYG{l+s+s1}{\PYGZsq{}}\PYG{p}{,} \PYG{k+kc}{True}\PYG{p}{)}\PYG{p}{]}\PYG{p}{,} \PYG{n}{limit}\PYG{o}{=}\PYG{l+m+mi}{1}\PYG{p}{)}\PYG{o}{.}\PYG{n}{name}
\PYG{g+go}{\PYGZsq{}Agrolait\PYGZsq{}}
\end{sphinxVerbatim}

\begin{sphinxadmonition}{tip}{Tip:}
to just check if any record matches a domain, or count the number
of records which do, use
{\hyperref[\detokenize{reference/orm:odoo.models.Model.search_count}]{\sphinxcrossref{\sphinxcode{\sphinxupquote{search\_count()}}}}}
\end{sphinxadmonition}

\item[{{\hyperref[\detokenize{reference/orm:odoo.models.Model.create}]{\sphinxcrossref{\sphinxcode{\sphinxupquote{create()}}}}}}] \leavevmode
Takes a number of field values, and returns a recordset containing the
record created:

\fvset{hllines={, ,}}%
\begin{sphinxVerbatim}[commandchars=\\\{\}]
\PYG{g+gp}{\PYGZgt{}\PYGZgt{}\PYGZgt{} }\PYG{n+nb+bp}{self}\PYG{o}{.}\PYG{n}{create}\PYG{p}{(}\PYG{p}{\PYGZob{}}\PYG{l+s+s1}{\PYGZsq{}}\PYG{l+s+s1}{name}\PYG{l+s+s1}{\PYGZsq{}}\PYG{p}{:} \PYG{l+s+s2}{\PYGZdq{}}\PYG{l+s+s2}{New Name}\PYG{l+s+s2}{\PYGZdq{}}\PYG{p}{\PYGZcb{}}\PYG{p}{)}
\PYG{g+go}{res.partner(78)}
\end{sphinxVerbatim}

\item[{{\hyperref[\detokenize{reference/orm:odoo.models.Model.write}]{\sphinxcrossref{\sphinxcode{\sphinxupquote{write()}}}}}}] \leavevmode
Takes a number of field values, writes them to all the records in its
recordset. Does not return anything:

\fvset{hllines={, ,}}%
\begin{sphinxVerbatim}[commandchars=\\\{\}]
\PYG{n+nb+bp}{self}\PYG{o}{.}\PYG{n}{write}\PYG{p}{(}\PYG{p}{\PYGZob{}}\PYG{l+s+s1}{\PYGZsq{}}\PYG{l+s+s1}{name}\PYG{l+s+s1}{\PYGZsq{}}\PYG{p}{:} \PYG{l+s+s2}{\PYGZdq{}}\PYG{l+s+s2}{Newer Name}\PYG{l+s+s2}{\PYGZdq{}}\PYG{p}{\PYGZcb{}}\PYG{p}{)}
\end{sphinxVerbatim}

\item[{{\hyperref[\detokenize{reference/orm:odoo.models.Model.browse}]{\sphinxcrossref{\sphinxcode{\sphinxupquote{browse()}}}}}}] \leavevmode
Takes a database id or a list of ids and returns a recordset, useful when
record ids are obtained from outside Odoo (e.g. round-trip through
external system) or {\hyperref[\detokenize{reference/orm:reference-orm-oldapi}]{\sphinxcrossref{\DUrole{std,std-ref}{when calling methods in the old API}}}}:

\fvset{hllines={, ,}}%
\begin{sphinxVerbatim}[commandchars=\\\{\}]
\PYG{g+gp}{\PYGZgt{}\PYGZgt{}\PYGZgt{} }\PYG{n+nb+bp}{self}\PYG{o}{.}\PYG{n}{browse}\PYG{p}{(}\PYG{p}{[}\PYG{l+m+mi}{7}\PYG{p}{,} \PYG{l+m+mi}{18}\PYG{p}{,} \PYG{l+m+mi}{12}\PYG{p}{]}\PYG{p}{)}
\PYG{g+go}{res.partner(7, 18, 12)}
\end{sphinxVerbatim}

\item[{{\hyperref[\detokenize{reference/orm:odoo.models.Model.exists}]{\sphinxcrossref{\sphinxcode{\sphinxupquote{exists()}}}}}}] \leavevmode
Returns a new recordset containing only the records which exist in the
database. Can be used to check whether a record (e.g. obtained externally)
still exists:

\fvset{hllines={, ,}}%
\begin{sphinxVerbatim}[commandchars=\\\{\}]
\PYG{k}{if} \PYG{o+ow}{not} \PYG{n}{record}\PYG{o}{.}\PYG{n}{exists}\PYG{p}{(}\PYG{p}{)}\PYG{p}{:}
    \PYG{k}{raise} \PYG{n+ne}{Exception}\PYG{p}{(}\PYG{l+s+s2}{\PYGZdq{}}\PYG{l+s+s2}{The record has been deleted}\PYG{l+s+s2}{\PYGZdq{}}\PYG{p}{)}
\end{sphinxVerbatim}

or after calling a method which could have removed some records:

\fvset{hllines={, ,}}%
\begin{sphinxVerbatim}[commandchars=\\\{\}]
\PYG{n}{records}\PYG{o}{.}\PYG{n}{may\PYGZus{}remove\PYGZus{}some}\PYG{p}{(}\PYG{p}{)}
\PYG{c+c1}{\PYGZsh{} only keep records which were not deleted}
\PYG{n}{records} \PYG{o}{=} \PYG{n}{records}\PYG{o}{.}\PYG{n}{exists}\PYG{p}{(}\PYG{p}{)}
\end{sphinxVerbatim}

\item[{\sphinxcode{\sphinxupquote{ref()}}}] \leavevmode
Environment method returning the record matching a provided
\DUrole{xref,std,std-term}{external id}:

\fvset{hllines={, ,}}%
\begin{sphinxVerbatim}[commandchars=\\\{\}]
\PYG{g+gp}{\PYGZgt{}\PYGZgt{}\PYGZgt{} }\PYG{n}{env}\PYG{o}{.}\PYG{n}{ref}\PYG{p}{(}\PYG{l+s+s1}{\PYGZsq{}}\PYG{l+s+s1}{base.group\PYGZus{}public}\PYG{l+s+s1}{\PYGZsq{}}\PYG{p}{)}
\PYG{g+go}{res.groups(2)}
\end{sphinxVerbatim}

\item[{{\hyperref[\detokenize{reference/orm:odoo.models.Model.ensure_one}]{\sphinxcrossref{\sphinxcode{\sphinxupquote{ensure\_one()}}}}}}] \leavevmode
checks that the recordset is a singleton (only contains a single record),
raises an error otherwise:

\fvset{hllines={, ,}}%
\begin{sphinxVerbatim}[commandchars=\\\{\}]
\PYG{n}{records}\PYG{o}{.}\PYG{n}{ensure\PYGZus{}one}\PYG{p}{(}\PYG{p}{)}
\PYG{c+c1}{\PYGZsh{} is equivalent to but clearer than:}
\PYG{k}{assert} \PYG{n+nb}{len}\PYG{p}{(}\PYG{n}{records}\PYG{p}{)} \PYG{o}{==} \PYG{l+m+mi}{1}\PYG{p}{,} \PYG{l+s+s2}{\PYGZdq{}}\PYG{l+s+s2}{Expected singleton}\PYG{l+s+s2}{\PYGZdq{}}
\end{sphinxVerbatim}

\end{description}


\subsection{Creating Models}
\label{\detokenize{reference/orm:creating-models}}
Model fields are defined as attributes on the model itself:

\fvset{hllines={, ,}}%
\begin{sphinxVerbatim}[commandchars=\\\{\}]
\PYG{k+kn}{from} \PYG{n+nn}{odoo} \PYG{k}{import} \PYG{n}{models}\PYG{p}{,} \PYG{n}{fields}
\PYG{k}{class} \PYG{n+nc}{AModel}\PYG{p}{(}\PYG{n}{models}\PYG{o}{.}\PYG{n}{Model}\PYG{p}{)}\PYG{p}{:}
    \PYG{n}{\PYGZus{}name} \PYG{o}{=} \PYG{l+s+s1}{\PYGZsq{}}\PYG{l+s+s1}{a.model.name}\PYG{l+s+s1}{\PYGZsq{}}

    \PYG{n}{field1} \PYG{o}{=} \PYG{n}{fields}\PYG{o}{.}\PYG{n}{Char}\PYG{p}{(}\PYG{p}{)}
\end{sphinxVerbatim}

\begin{sphinxadmonition}{warning}{Warning:}
this means you can not define a field and a method with the same
name, they will conflict
\end{sphinxadmonition}

By default, the field’s label (user-visible name) is a capitalized version of
the field name, this can be overridden with the \sphinxcode{\sphinxupquote{string}} parameter:

\fvset{hllines={, ,}}%
\begin{sphinxVerbatim}[commandchars=\\\{\}]
\PYG{n}{field2} \PYG{o}{=} \PYG{n}{fields}\PYG{o}{.}\PYG{n}{Integer}\PYG{p}{(}\PYG{n}{string}\PYG{o}{=}\PYG{l+s+s2}{\PYGZdq{}}\PYG{l+s+s2}{an other field}\PYG{l+s+s2}{\PYGZdq{}}\PYG{p}{)}
\end{sphinxVerbatim}

For the various field types and parameters, see {\hyperref[\detokenize{reference/orm:reference-orm-fields}]{\sphinxcrossref{\DUrole{std,std-ref}{the fields reference}}}}.

Default values are defined as parameters on fields, either a value:

\fvset{hllines={, ,}}%
\begin{sphinxVerbatim}[commandchars=\\\{\}]
\PYG{n}{a\PYGZus{}field} \PYG{o}{=} \PYG{n}{fields}\PYG{o}{.}\PYG{n}{Char}\PYG{p}{(}\PYG{n}{default}\PYG{o}{=}\PYG{l+s+s2}{\PYGZdq{}}\PYG{l+s+s2}{a value}\PYG{l+s+s2}{\PYGZdq{}}\PYG{p}{)}
\end{sphinxVerbatim}

or a function called to compute the default value, which should return that
value:

\fvset{hllines={, ,}}%
\begin{sphinxVerbatim}[commandchars=\\\{\}]
\PYG{k}{def} \PYG{n+nf}{compute\PYGZus{}default\PYGZus{}value}\PYG{p}{(}\PYG{n+nb+bp}{self}\PYG{p}{)}\PYG{p}{:}
    \PYG{k}{return} \PYG{n+nb+bp}{self}\PYG{o}{.}\PYG{n}{get\PYGZus{}value}\PYG{p}{(}\PYG{p}{)}
\PYG{n}{a\PYGZus{}field} \PYG{o}{=} \PYG{n}{fields}\PYG{o}{.}\PYG{n}{Char}\PYG{p}{(}\PYG{n}{default}\PYG{o}{=}\PYG{n}{compute\PYGZus{}default\PYGZus{}value}\PYG{p}{)}
\end{sphinxVerbatim}


\subsubsection{Computed fields}
\label{\detokenize{reference/orm:computed-fields}}
Fields can be computed (instead of read straight from the database) using the
\sphinxcode{\sphinxupquote{compute}} parameter. \sphinxstylestrong{It must assign the computed value to the field}. If
it uses the values of other \sphinxstyleemphasis{fields}, it should specify those fields using
{\hyperref[\detokenize{reference/orm:odoo.api.depends}]{\sphinxcrossref{\sphinxcode{\sphinxupquote{depends()}}}}}:

\fvset{hllines={, ,}}%
\begin{sphinxVerbatim}[commandchars=\\\{\}]
\PYG{k+kn}{from} \PYG{n+nn}{odoo} \PYG{k}{import} \PYG{n}{api}
\PYG{n}{total} \PYG{o}{=} \PYG{n}{fields}\PYG{o}{.}\PYG{n}{Float}\PYG{p}{(}\PYG{n}{compute}\PYG{o}{=}\PYG{l+s+s1}{\PYGZsq{}}\PYG{l+s+s1}{\PYGZus{}compute\PYGZus{}total}\PYG{l+s+s1}{\PYGZsq{}}\PYG{p}{)}

\PYG{n+nd}{@api}\PYG{o}{.}\PYG{n}{depends}\PYG{p}{(}\PYG{l+s+s1}{\PYGZsq{}}\PYG{l+s+s1}{value}\PYG{l+s+s1}{\PYGZsq{}}\PYG{p}{,} \PYG{l+s+s1}{\PYGZsq{}}\PYG{l+s+s1}{tax}\PYG{l+s+s1}{\PYGZsq{}}\PYG{p}{)}
\PYG{k}{def} \PYG{n+nf}{\PYGZus{}compute\PYGZus{}total}\PYG{p}{(}\PYG{n+nb+bp}{self}\PYG{p}{)}\PYG{p}{:}
    \PYG{k}{for} \PYG{n}{record} \PYG{o+ow}{in} \PYG{n+nb+bp}{self}\PYG{p}{:}
        \PYG{n}{record}\PYG{o}{.}\PYG{n}{total} \PYG{o}{=} \PYG{n}{record}\PYG{o}{.}\PYG{n}{value} \PYG{o}{+} \PYG{n}{record}\PYG{o}{.}\PYG{n}{value} \PYG{o}{*} \PYG{n}{record}\PYG{o}{.}\PYG{n}{tax}
\end{sphinxVerbatim}
\begin{itemize}
\item {} 
dependencies can be dotted paths when using sub-fields:

\fvset{hllines={, ,}}%
\begin{sphinxVerbatim}[commandchars=\\\{\}]
\PYG{n+nd}{@api}\PYG{o}{.}\PYG{n}{depends}\PYG{p}{(}\PYG{l+s+s1}{\PYGZsq{}}\PYG{l+s+s1}{line\PYGZus{}ids.value}\PYG{l+s+s1}{\PYGZsq{}}\PYG{p}{)}
\PYG{k}{def} \PYG{n+nf}{\PYGZus{}compute\PYGZus{}total}\PYG{p}{(}\PYG{n+nb+bp}{self}\PYG{p}{)}\PYG{p}{:}
    \PYG{k}{for} \PYG{n}{record} \PYG{o+ow}{in} \PYG{n+nb+bp}{self}\PYG{p}{:}
        \PYG{n}{record}\PYG{o}{.}\PYG{n}{total} \PYG{o}{=} \PYG{n+nb}{sum}\PYG{p}{(}\PYG{n}{line}\PYG{o}{.}\PYG{n}{value} \PYG{k}{for} \PYG{n}{line} \PYG{o+ow}{in} \PYG{n}{record}\PYG{o}{.}\PYG{n}{line\PYGZus{}ids}\PYG{p}{)}
\end{sphinxVerbatim}

\item {} 
computed fields are not stored by default, they are computed and
returned when requested. Setting \sphinxcode{\sphinxupquote{store=True}} will store them in the
database and automatically enable searching

\item {} 
searching on a computed field can also be enabled by setting the \sphinxcode{\sphinxupquote{search}}
parameter. The value is a method name returning a
{\hyperref[\detokenize{reference/orm:reference-orm-domains}]{\sphinxcrossref{\DUrole{std,std-ref}{Domains}}}}:

\fvset{hllines={, ,}}%
\begin{sphinxVerbatim}[commandchars=\\\{\}]
\PYG{n}{upper\PYGZus{}name} \PYG{o}{=} \PYG{n}{field}\PYG{o}{.}\PYG{n}{Char}\PYG{p}{(}\PYG{n}{compute}\PYG{o}{=}\PYG{l+s+s1}{\PYGZsq{}}\PYG{l+s+s1}{\PYGZus{}compute\PYGZus{}upper}\PYG{l+s+s1}{\PYGZsq{}}\PYG{p}{,} \PYG{n}{search}\PYG{o}{=}\PYG{l+s+s1}{\PYGZsq{}}\PYG{l+s+s1}{\PYGZus{}search\PYGZus{}upper}\PYG{l+s+s1}{\PYGZsq{}}\PYG{p}{)}

\PYG{k}{def} \PYG{n+nf}{\PYGZus{}search\PYGZus{}upper}\PYG{p}{(}\PYG{n+nb+bp}{self}\PYG{p}{,} \PYG{n}{operator}\PYG{p}{,} \PYG{n}{value}\PYG{p}{)}\PYG{p}{:}
    \PYG{k}{if} \PYG{n}{operator} \PYG{o}{==} \PYG{l+s+s1}{\PYGZsq{}}\PYG{l+s+s1}{like}\PYG{l+s+s1}{\PYGZsq{}}\PYG{p}{:}
        \PYG{n}{operator} \PYG{o}{=} \PYG{l+s+s1}{\PYGZsq{}}\PYG{l+s+s1}{ilike}\PYG{l+s+s1}{\PYGZsq{}}
    \PYG{k}{return} \PYG{p}{[}\PYG{p}{(}\PYG{l+s+s1}{\PYGZsq{}}\PYG{l+s+s1}{name}\PYG{l+s+s1}{\PYGZsq{}}\PYG{p}{,} \PYG{n}{operator}\PYG{p}{,} \PYG{n}{value}\PYG{p}{)}\PYG{p}{]}
\end{sphinxVerbatim}

\item {} 
to allow \sphinxstyleemphasis{setting} values on a computed field, use the \sphinxcode{\sphinxupquote{inverse}}
parameter. It is the name of a function reversing the computation and
setting the relevant fields:

\fvset{hllines={, ,}}%
\begin{sphinxVerbatim}[commandchars=\\\{\}]
\PYG{n}{document} \PYG{o}{=} \PYG{n}{fields}\PYG{o}{.}\PYG{n}{Char}\PYG{p}{(}\PYG{n}{compute}\PYG{o}{=}\PYG{l+s+s1}{\PYGZsq{}}\PYG{l+s+s1}{\PYGZus{}get\PYGZus{}document}\PYG{l+s+s1}{\PYGZsq{}}\PYG{p}{,} \PYG{n}{inverse}\PYG{o}{=}\PYG{l+s+s1}{\PYGZsq{}}\PYG{l+s+s1}{\PYGZus{}set\PYGZus{}document}\PYG{l+s+s1}{\PYGZsq{}}\PYG{p}{)}

\PYG{k}{def} \PYG{n+nf}{\PYGZus{}get\PYGZus{}document}\PYG{p}{(}\PYG{n+nb+bp}{self}\PYG{p}{)}\PYG{p}{:}
    \PYG{k}{for} \PYG{n}{record} \PYG{o+ow}{in} \PYG{n+nb+bp}{self}\PYG{p}{:}
        \PYG{k}{with} \PYG{n+nb}{open}\PYG{p}{(}\PYG{n}{record}\PYG{o}{.}\PYG{n}{get\PYGZus{}document\PYGZus{}path}\PYG{p}{)} \PYG{k}{as} \PYG{n}{f}\PYG{p}{:}
            \PYG{n}{record}\PYG{o}{.}\PYG{n}{document} \PYG{o}{=} \PYG{n}{f}\PYG{o}{.}\PYG{n}{read}\PYG{p}{(}\PYG{p}{)}
\PYG{k}{def} \PYG{n+nf}{\PYGZus{}set\PYGZus{}document}\PYG{p}{(}\PYG{n+nb+bp}{self}\PYG{p}{)}\PYG{p}{:}
    \PYG{k}{for} \PYG{n}{record} \PYG{o+ow}{in} \PYG{n+nb+bp}{self}\PYG{p}{:}
        \PYG{k}{if} \PYG{o+ow}{not} \PYG{n}{record}\PYG{o}{.}\PYG{n}{document}\PYG{p}{:} \PYG{k}{continue}
        \PYG{k}{with} \PYG{n+nb}{open}\PYG{p}{(}\PYG{n}{record}\PYG{o}{.}\PYG{n}{get\PYGZus{}document\PYGZus{}path}\PYG{p}{(}\PYG{p}{)}\PYG{p}{)} \PYG{k}{as} \PYG{n}{f}\PYG{p}{:}
            \PYG{n}{f}\PYG{o}{.}\PYG{n}{write}\PYG{p}{(}\PYG{n}{record}\PYG{o}{.}\PYG{n}{document}\PYG{p}{)}
\end{sphinxVerbatim}

\item {} 
multiple fields can be computed at the same time by the same method, just
use the same method on all fields and set all of them:

\fvset{hllines={, ,}}%
\begin{sphinxVerbatim}[commandchars=\\\{\}]
\PYG{n}{discount\PYGZus{}value} \PYG{o}{=} \PYG{n}{fields}\PYG{o}{.}\PYG{n}{Float}\PYG{p}{(}\PYG{n}{compute}\PYG{o}{=}\PYG{l+s+s1}{\PYGZsq{}}\PYG{l+s+s1}{\PYGZus{}apply\PYGZus{}discount}\PYG{l+s+s1}{\PYGZsq{}}\PYG{p}{)}
\PYG{n}{total} \PYG{o}{=} \PYG{n}{fields}\PYG{o}{.}\PYG{n}{Float}\PYG{p}{(}\PYG{n}{compute}\PYG{o}{=}\PYG{l+s+s1}{\PYGZsq{}}\PYG{l+s+s1}{\PYGZus{}apply\PYGZus{}discount}\PYG{l+s+s1}{\PYGZsq{}}\PYG{p}{)}

\PYG{n+nd}{@depends}\PYG{p}{(}\PYG{l+s+s1}{\PYGZsq{}}\PYG{l+s+s1}{value}\PYG{l+s+s1}{\PYGZsq{}}\PYG{p}{,} \PYG{l+s+s1}{\PYGZsq{}}\PYG{l+s+s1}{discount}\PYG{l+s+s1}{\PYGZsq{}}\PYG{p}{)}
\PYG{k}{def} \PYG{n+nf}{\PYGZus{}apply\PYGZus{}discount}\PYG{p}{(}\PYG{n+nb+bp}{self}\PYG{p}{)}\PYG{p}{:}
    \PYG{k}{for} \PYG{n}{record} \PYG{o+ow}{in} \PYG{n+nb+bp}{self}\PYG{p}{:}
        \PYG{c+c1}{\PYGZsh{} compute actual discount from discount percentage}
        \PYG{n}{discount} \PYG{o}{=} \PYG{n}{record}\PYG{o}{.}\PYG{n}{value} \PYG{o}{*} \PYG{n}{record}\PYG{o}{.}\PYG{n}{discount}
        \PYG{n}{record}\PYG{o}{.}\PYG{n}{discount\PYGZus{}value} \PYG{o}{=} \PYG{n}{discount}
        \PYG{n}{record}\PYG{o}{.}\PYG{n}{total} \PYG{o}{=} \PYG{n}{record}\PYG{o}{.}\PYG{n}{value} \PYG{o}{\PYGZhy{}} \PYG{n}{discount}
\end{sphinxVerbatim}

\end{itemize}


\paragraph{Related fields}
\label{\detokenize{reference/orm:related-fields}}
A special case of computed fields are \sphinxstyleemphasis{related} (proxy) fields, which provide
the value of a sub-field on the current record. They are defined by setting
the \sphinxcode{\sphinxupquote{related}} parameter and like regular computed fields they can be
stored:

\fvset{hllines={, ,}}%
\begin{sphinxVerbatim}[commandchars=\\\{\}]
\PYG{n}{nickname} \PYG{o}{=} \PYG{n}{fields}\PYG{o}{.}\PYG{n}{Char}\PYG{p}{(}\PYG{n}{related}\PYG{o}{=}\PYG{l+s+s1}{\PYGZsq{}}\PYG{l+s+s1}{user\PYGZus{}id.partner\PYGZus{}id.name}\PYG{l+s+s1}{\PYGZsq{}}\PYG{p}{,} \PYG{n}{store}\PYG{o}{=}\PYG{k+kc}{True}\PYG{p}{)}
\end{sphinxVerbatim}


\subsubsection{onchange: updating UI on the fly}
\label{\detokenize{reference/orm:onchange-updating-ui-on-the-fly}}
When a user changes a field’s value in a form (but hasn’t saved the form yet),
it can be useful to automatically update other fields based on that value
e.g. updating a final total when the tax is changed or a new invoice line is
added.
\begin{itemize}
\item {} 
computed fields are automatically checked and recomputed, they do not need
an \sphinxcode{\sphinxupquote{onchange}}

\item {} 
for non-computed fields, the {\hyperref[\detokenize{reference/orm:odoo.api.onchange}]{\sphinxcrossref{\sphinxcode{\sphinxupquote{onchange()}}}}} decorator is used
to provide new field values:

\fvset{hllines={, ,}}%
\begin{sphinxVerbatim}[commandchars=\\\{\}]
\PYG{n+nd}{@api}\PYG{o}{.}\PYG{n}{onchange}\PYG{p}{(}\PYG{l+s+s1}{\PYGZsq{}}\PYG{l+s+s1}{field1}\PYG{l+s+s1}{\PYGZsq{}}\PYG{p}{,} \PYG{l+s+s1}{\PYGZsq{}}\PYG{l+s+s1}{field2}\PYG{l+s+s1}{\PYGZsq{}}\PYG{p}{)} \PYG{c+c1}{\PYGZsh{} if these fields are changed, call method}
\PYG{k}{def} \PYG{n+nf}{check\PYGZus{}change}\PYG{p}{(}\PYG{n+nb+bp}{self}\PYG{p}{)}\PYG{p}{:}
    \PYG{k}{if} \PYG{n+nb+bp}{self}\PYG{o}{.}\PYG{n}{field1} \PYG{o}{\PYGZlt{}} \PYG{n+nb+bp}{self}\PYG{o}{.}\PYG{n}{field2}\PYG{p}{:}
        \PYG{n+nb+bp}{self}\PYG{o}{.}\PYG{n}{field3} \PYG{o}{=} \PYG{k+kc}{True}
\end{sphinxVerbatim}

the changes performed during the method are then sent to the client program
and become visible to the user

\item {} 
Both computed fields and new-API onchanges are automatically called by the
client without having to add them in views

\item {} 
It is possible to suppress the trigger from a specific field by adding
\sphinxcode{\sphinxupquote{on\_change="0"}} in a view:

\fvset{hllines={, ,}}%
\begin{sphinxVerbatim}[commandchars=\\\{\}]
\PYG{o}{\PYGZlt{}}\PYG{n}{field} \PYG{n}{name}\PYG{o}{=}\PYG{l+s+s2}{\PYGZdq{}}\PYG{l+s+s2}{name}\PYG{l+s+s2}{\PYGZdq{}} \PYG{n}{on\PYGZus{}change}\PYG{o}{=}\PYG{l+s+s2}{\PYGZdq{}}\PYG{l+s+s2}{0}\PYG{l+s+s2}{\PYGZdq{}}\PYG{o}{/}\PYG{o}{\PYGZgt{}}
\end{sphinxVerbatim}

will not trigger any interface update when the field is edited by the user,
even if there are function fields or explicit onchange depending on that
field.

\end{itemize}

\begin{sphinxadmonition}{note}{Note:}
\sphinxcode{\sphinxupquote{onchange}} methods work on virtual records assignment on these records
is not written to the database, just used to know which value to send back
to the client
\end{sphinxadmonition}

\begin{sphinxadmonition}{warning}{Warning:}
It is not possible for a \sphinxcode{\sphinxupquote{one2many}} or \sphinxcode{\sphinxupquote{many2many}} field to modify
itself via onchange. This is a webclient limitation - see \sphinxhref{https://github.com/odoo/odoo/issues/2693}{\#2693}.
\end{sphinxadmonition}


\subsubsection{Low-level SQL}
\label{\detokenize{reference/orm:low-level-sql}}
The \sphinxcode{\sphinxupquote{cr}} attribute on environments is the
cursor for the current database transaction and allows executing SQL directly,
either for queries which are difficult to express using the ORM (e.g. complex
joins) or for performance reasons:

\fvset{hllines={, ,}}%
\begin{sphinxVerbatim}[commandchars=\\\{\}]
\PYG{n+nb+bp}{self}\PYG{o}{.}\PYG{n}{env}\PYG{o}{.}\PYG{n}{cr}\PYG{o}{.}\PYG{n}{execute}\PYG{p}{(}\PYG{l+s+s2}{\PYGZdq{}}\PYG{l+s+s2}{some\PYGZus{}sql}\PYG{l+s+s2}{\PYGZdq{}}\PYG{p}{,} \PYG{n}{param1}\PYG{p}{,} \PYG{n}{param2}\PYG{p}{,} \PYG{n}{param3}\PYG{p}{)}
\end{sphinxVerbatim}

Because models use the same cursor and the \sphinxcode{\sphinxupquote{Environment}}
holds various caches, these caches must be invalidated when \sphinxstyleemphasis{altering} the
database in raw SQL, or further uses of models may become incoherent. It is
necessary to clear caches when using \sphinxcode{\sphinxupquote{CREATE}}, \sphinxcode{\sphinxupquote{UPDATE}} or \sphinxcode{\sphinxupquote{DELETE}} in
SQL, but not \sphinxcode{\sphinxupquote{SELECT}} (which simply reads the database).

Clearing caches can be performed using the
\sphinxcode{\sphinxupquote{invalidate\_all()}} method of the
\sphinxcode{\sphinxupquote{Environment}} object.


\subsection{Compatibility between new API and old API}
\label{\detokenize{reference/orm:compatibility-between-new-api-and-old-api}}\label{\detokenize{reference/orm:reference-orm-oldapi}}
Odoo is currently transitioning from an older (less regular) API, it can be
necessary to manually bridge from one to the other manually:
\begin{itemize}
\item {} 
RPC layers (both XML-RPC and JSON-RPC) are expressed in terms of the old
API, methods expressed purely in the new API are not available over RPC

\item {} 
overridable methods may be called from older pieces of code still written
in the old API style

\end{itemize}

The big differences between the old and new APIs are:
\begin{itemize}
\item {} 
values of the \sphinxcode{\sphinxupquote{Environment}} (cursor, user id and
context) are passed explicitly to methods instead

\item {} 
record data ({\hyperref[\detokenize{reference/orm:odoo.models.Model.ids}]{\sphinxcrossref{\sphinxcode{\sphinxupquote{ids}}}}}) are passed explicitly to
methods, and possibly not passed at all

\item {} 
methods tend to work on lists of ids instead of recordsets

\end{itemize}

By default, methods are assumed to use the new API style and are not callable
from the old API style.

\begin{sphinxadmonition}{tip}{Tip:}
calls from the new API to the old API are bridged

when using the new API style, calls to methods defined using the old API
are automatically converted on-the-fly, there should be no need to do
anything special:

\fvset{hllines={, ,}}%
\begin{sphinxVerbatim}[commandchars=\\\{\}]
\PYG{g+gp}{\PYGZgt{}\PYGZgt{}\PYGZgt{} }\PYG{c+c1}{\PYGZsh{} method in the old API style}
\PYG{g+gp}{\PYGZgt{}\PYGZgt{}\PYGZgt{} }\PYG{k}{def} \PYG{n+nf}{old\PYGZus{}method}\PYG{p}{(}\PYG{n+nb+bp}{self}\PYG{p}{,} \PYG{n}{cr}\PYG{p}{,} \PYG{n}{uid}\PYG{p}{,} \PYG{n}{ids}\PYG{p}{,} \PYG{n}{context}\PYG{o}{=}\PYG{k+kc}{None}\PYG{p}{)}\PYG{p}{:}
\PYG{g+gp}{... }   \PYG{n+nb}{print} \PYG{n}{ids}

\PYG{g+gp}{\PYGZgt{}\PYGZgt{}\PYGZgt{} }\PYG{c+c1}{\PYGZsh{} method in the new API style}
\PYG{g+gp}{\PYGZgt{}\PYGZgt{}\PYGZgt{} }\PYG{k}{def} \PYG{n+nf}{new\PYGZus{}method}\PYG{p}{(}\PYG{n+nb+bp}{self}\PYG{p}{)}\PYG{p}{:}
\PYG{g+gp}{... }    \PYG{c+c1}{\PYGZsh{} system automatically infers how to call the old\PYGZhy{}style}
\PYG{g+gp}{... }    \PYG{c+c1}{\PYGZsh{} method from the new\PYGZhy{}style method}
\PYG{g+gp}{... }    \PYG{n+nb+bp}{self}\PYG{o}{.}\PYG{n}{old\PYGZus{}method}\PYG{p}{(}\PYG{p}{)}

\PYG{g+gp}{\PYGZgt{}\PYGZgt{}\PYGZgt{} }\PYG{n}{env}\PYG{p}{[}\PYG{n}{model}\PYG{p}{]}\PYG{o}{.}\PYG{n}{browse}\PYG{p}{(}\PYG{p}{[}\PYG{l+m+mi}{1}\PYG{p}{,} \PYG{l+m+mi}{2}\PYG{p}{,} \PYG{l+m+mi}{3}\PYG{p}{,} \PYG{l+m+mi}{4}\PYG{p}{]}\PYG{p}{)}\PYG{o}{.}\PYG{n}{new\PYGZus{}method}\PYG{p}{(}\PYG{p}{)}
\PYG{g+go}{[1, 2, 3, 4]}
\end{sphinxVerbatim}
\end{sphinxadmonition}

Two decorators can expose a new-style method to the old API:
\begin{description}
\item[{{\hyperref[\detokenize{reference/orm:odoo.api.model}]{\sphinxcrossref{\sphinxcode{\sphinxupquote{model()}}}}}}] \leavevmode
the method is exposed as not using ids, its recordset will generally be
empty. Its “old API” signature is \sphinxcode{\sphinxupquote{cr, uid, *arguments, context}}:

\fvset{hllines={, ,}}%
\begin{sphinxVerbatim}[commandchars=\\\{\}]
\PYG{n+nd}{@api}\PYG{o}{.}\PYG{n}{model}
\PYG{k}{def} \PYG{n+nf}{some\PYGZus{}method}\PYG{p}{(}\PYG{n+nb+bp}{self}\PYG{p}{,} \PYG{n}{a\PYGZus{}value}\PYG{p}{)}\PYG{p}{:}
    \PYG{k}{pass}
\PYG{c+c1}{\PYGZsh{} can be called as}
\PYG{n}{old\PYGZus{}style\PYGZus{}model}\PYG{o}{.}\PYG{n}{some\PYGZus{}method}\PYG{p}{(}\PYG{n}{cr}\PYG{p}{,} \PYG{n}{uid}\PYG{p}{,} \PYG{n}{a\PYGZus{}value}\PYG{p}{,} \PYG{n}{context}\PYG{o}{=}\PYG{n}{context}\PYG{p}{)}
\end{sphinxVerbatim}

\item[{{\hyperref[\detokenize{reference/orm:odoo.api.multi}]{\sphinxcrossref{\sphinxcode{\sphinxupquote{multi()}}}}}}] \leavevmode
the method is exposed as taking a list of ids (possibly empty), its
“old API” signature is \sphinxcode{\sphinxupquote{cr, uid, ids, *arguments, context}}:

\fvset{hllines={, ,}}%
\begin{sphinxVerbatim}[commandchars=\\\{\}]
\PYG{n+nd}{@api}\PYG{o}{.}\PYG{n}{multi}
\PYG{k}{def} \PYG{n+nf}{some\PYGZus{}method}\PYG{p}{(}\PYG{n+nb+bp}{self}\PYG{p}{,} \PYG{n}{a\PYGZus{}value}\PYG{p}{)}\PYG{p}{:}
    \PYG{k}{pass}
\PYG{c+c1}{\PYGZsh{} can be called as}
\PYG{n}{old\PYGZus{}style\PYGZus{}model}\PYG{o}{.}\PYG{n}{some\PYGZus{}method}\PYG{p}{(}\PYG{n}{cr}\PYG{p}{,} \PYG{n}{uid}\PYG{p}{,} \PYG{p}{[}\PYG{n}{id1}\PYG{p}{,} \PYG{n}{id2}\PYG{p}{]}\PYG{p}{,} \PYG{n}{a\PYGZus{}value}\PYG{p}{,} \PYG{n}{context}\PYG{o}{=}\PYG{n}{context}\PYG{p}{)}
\end{sphinxVerbatim}

\end{description}

Because new-style APIs tend to return recordsets and old-style APIs tend to
return lists of ids, there is also a decorator managing this:
\begin{description}
\item[{{\hyperref[\detokenize{reference/orm:odoo.api.returns}]{\sphinxcrossref{\sphinxcode{\sphinxupquote{returns()}}}}}}] \leavevmode
the function is assumed to return a recordset, the first parameter should
be the name of the recordset’s model or \sphinxcode{\sphinxupquote{self}} (for the current model).

No effect if the method is called in new API style, but transforms the
recordset into a list of ids when called from the old API style:

\fvset{hllines={, ,}}%
\begin{sphinxVerbatim}[commandchars=\\\{\}]
\PYG{g+gp}{\PYGZgt{}\PYGZgt{}\PYGZgt{} }\PYG{n+nd}{@api}\PYG{o}{.}\PYG{n}{multi}
\PYG{g+gp}{... }\PYG{n+nd}{@api}\PYG{o}{.}\PYG{n}{returns}\PYG{p}{(}\PYG{l+s+s1}{\PYGZsq{}}\PYG{l+s+s1}{self}\PYG{l+s+s1}{\PYGZsq{}}\PYG{p}{)}
\PYG{g+gp}{... }\PYG{k}{def} \PYG{n+nf}{some\PYGZus{}method}\PYG{p}{(}\PYG{n+nb+bp}{self}\PYG{p}{)}\PYG{p}{:}
\PYG{g+gp}{... }    \PYG{k}{return} \PYG{n+nb+bp}{self}
\PYG{g+gp}{\PYGZgt{}\PYGZgt{}\PYGZgt{} }\PYG{n}{new\PYGZus{}style\PYGZus{}model} \PYG{o}{=} \PYG{n}{env}\PYG{p}{[}\PYG{l+s+s1}{\PYGZsq{}}\PYG{l+s+s1}{a.model}\PYG{l+s+s1}{\PYGZsq{}}\PYG{p}{]}\PYG{o}{.}\PYG{n}{browse}\PYG{p}{(}\PYG{l+m+mi}{1}\PYG{p}{,} \PYG{l+m+mi}{2}\PYG{p}{,} \PYG{l+m+mi}{3}\PYG{p}{)}
\PYG{g+gp}{\PYGZgt{}\PYGZgt{}\PYGZgt{} }\PYG{n}{new\PYGZus{}style\PYGZus{}model}\PYG{o}{.}\PYG{n}{some\PYGZus{}method}\PYG{p}{(}\PYG{p}{)}
\PYG{g+go}{a.model(1, 2, 3)}
\PYG{g+gp}{\PYGZgt{}\PYGZgt{}\PYGZgt{} }\PYG{n}{old\PYGZus{}style\PYGZus{}model} \PYG{o}{=} \PYG{n}{pool}\PYG{p}{[}\PYG{l+s+s1}{\PYGZsq{}}\PYG{l+s+s1}{a.model}\PYG{l+s+s1}{\PYGZsq{}}\PYG{p}{]}
\PYG{g+gp}{\PYGZgt{}\PYGZgt{}\PYGZgt{} }\PYG{n}{old\PYGZus{}style\PYGZus{}model}\PYG{o}{.}\PYG{n}{some\PYGZus{}method}\PYG{p}{(}\PYG{n}{cr}\PYG{p}{,} \PYG{n}{uid}\PYG{p}{,} \PYG{p}{[}\PYG{l+m+mi}{1}\PYG{p}{,} \PYG{l+m+mi}{2}\PYG{p}{,} \PYG{l+m+mi}{3}\PYG{p}{]}\PYG{p}{,} \PYG{n}{context}\PYG{o}{=}\PYG{n}{context}\PYG{p}{)}
\PYG{g+go}{[1, 2, 3]}
\end{sphinxVerbatim}

\end{description}


\subsection{Model Reference}
\label{\detokenize{reference/orm:model-reference}}\label{\detokenize{reference/orm:reference-orm-model}}\index{Model (class in odoo.models)}

\begin{fulllineitems}
\phantomsection\label{\detokenize{reference/orm:odoo.models.Model}}\pysiglinewithargsret{\sphinxbfcode{\sphinxupquote{class }}\sphinxcode{\sphinxupquote{odoo.models.}}\sphinxbfcode{\sphinxupquote{Model}}}{\emph{pool}, \emph{cr}}{}
Main super-class for regular database-persisted Odoo models.

Odoo models are created by inheriting from this class:

\fvset{hllines={, ,}}%
\begin{sphinxVerbatim}[commandchars=\\\{\}]
\PYG{k}{class} \PYG{n+nc}{user}\PYG{p}{(}\PYG{n}{Model}\PYG{p}{)}\PYG{p}{:}
    \PYG{o}{.}\PYG{o}{.}\PYG{o}{.}
\end{sphinxVerbatim}

The system will later instantiate the class once per database (on
which the class’ module is installed).
\paragraph{Structural attributes}
\index{\_name (odoo.models.Model attribute)}

\begin{fulllineitems}
\phantomsection\label{\detokenize{reference/orm:odoo.models.Model._name}}\pysigline{\sphinxbfcode{\sphinxupquote{\_name}}}
business object name, in dot-notation (in module namespace)

\end{fulllineitems}

\index{\_rec\_name (odoo.models.Model attribute)}

\begin{fulllineitems}
\phantomsection\label{\detokenize{reference/orm:odoo.models.Model._rec_name}}\pysigline{\sphinxbfcode{\sphinxupquote{\_rec\_name}}}
Alternative field to use as name, used by osv’s name\_get()
(default: \sphinxcode{\sphinxupquote{'name'}})

\end{fulllineitems}

\index{\_inherit (odoo.models.Model attribute)}

\begin{fulllineitems}
\phantomsection\label{\detokenize{reference/orm:odoo.models.Model._inherit}}\pysigline{\sphinxbfcode{\sphinxupquote{\_inherit}}}~\begin{itemize}
\item {} 
If {\hyperref[\detokenize{reference/orm:odoo.models.Model._name}]{\sphinxcrossref{\sphinxcode{\sphinxupquote{\_name}}}}} is set, names of parent models to inherit from.
Can be a \sphinxcode{\sphinxupquote{str}} if inheriting from a single parent

\item {} 
If {\hyperref[\detokenize{reference/orm:odoo.models.Model._name}]{\sphinxcrossref{\sphinxcode{\sphinxupquote{\_name}}}}} is unset, name of a single model to extend
in-place

\end{itemize}

See {\hyperref[\detokenize{reference/orm:reference-orm-inheritance}]{\sphinxcrossref{\DUrole{std,std-ref}{Inheritance and extension}}}}.

\end{fulllineitems}

\index{\_order (odoo.models.Model attribute)}

\begin{fulllineitems}
\phantomsection\label{\detokenize{reference/orm:odoo.models.Model._order}}\pysigline{\sphinxbfcode{\sphinxupquote{\_order}}}
Ordering field when searching without an ordering specified (default:
\sphinxcode{\sphinxupquote{'id'}})
\begin{quote}\begin{description}
\item[{Type}] \leavevmode
str

\end{description}\end{quote}

\end{fulllineitems}

\index{\_auto (odoo.models.Model attribute)}

\begin{fulllineitems}
\phantomsection\label{\detokenize{reference/orm:odoo.models.Model._auto}}\pysigline{\sphinxbfcode{\sphinxupquote{\_auto}}}~\begin{quote}

Whether a database table should be created (default: \sphinxcode{\sphinxupquote{True}})

If set to \sphinxcode{\sphinxupquote{False}}, override \sphinxcode{\sphinxupquote{init()}} to create the database
table
\end{quote}

\begin{sphinxadmonition}{tip}{Tip:}
To create a model without any table, inherit
from \sphinxcode{\sphinxupquote{odoo.models.AbstractModel}}
\end{sphinxadmonition}

\end{fulllineitems}

\index{\_table (odoo.models.Model attribute)}

\begin{fulllineitems}
\phantomsection\label{\detokenize{reference/orm:odoo.models.Model._table}}\pysigline{\sphinxbfcode{\sphinxupquote{\_table}}}
Name of the table backing the model created when
{\hyperref[\detokenize{reference/orm:odoo.models.Model._auto}]{\sphinxcrossref{\sphinxcode{\sphinxupquote{\_auto}}}}}, automatically generated by
default.

\end{fulllineitems}

\index{\_inherits (odoo.models.Model attribute)}

\begin{fulllineitems}
\phantomsection\label{\detokenize{reference/orm:odoo.models.Model._inherits}}\pysigline{\sphinxbfcode{\sphinxupquote{\_inherits}}}
dictionary mapping the \_name of the parent business objects to the
names of the corresponding foreign key fields to use:

\fvset{hllines={, ,}}%
\begin{sphinxVerbatim}[commandchars=\\\{\}]
\PYG{n}{\PYGZus{}inherits} \PYG{o}{=} \PYG{p}{\PYGZob{}}
    \PYG{l+s+s1}{\PYGZsq{}}\PYG{l+s+s1}{a.model}\PYG{l+s+s1}{\PYGZsq{}}\PYG{p}{:} \PYG{l+s+s1}{\PYGZsq{}}\PYG{l+s+s1}{a\PYGZus{}field\PYGZus{}id}\PYG{l+s+s1}{\PYGZsq{}}\PYG{p}{,}
    \PYG{l+s+s1}{\PYGZsq{}}\PYG{l+s+s1}{b.model}\PYG{l+s+s1}{\PYGZsq{}}\PYG{p}{:} \PYG{l+s+s1}{\PYGZsq{}}\PYG{l+s+s1}{b\PYGZus{}field\PYGZus{}id}\PYG{l+s+s1}{\PYGZsq{}}
\PYG{p}{\PYGZcb{}}
\end{sphinxVerbatim}

implements composition-based inheritance: the new model exposes all
the fields of the {\hyperref[\detokenize{reference/orm:odoo.models.Model._inherits}]{\sphinxcrossref{\sphinxcode{\sphinxupquote{\_inherits}}}}}-ed model but
stores none of them: the values themselves remain stored on the linked
record.

\begin{sphinxadmonition}{warning}{Warning:}
if the same field is defined on multiple
{\hyperref[\detokenize{reference/orm:odoo.models.Model._inherits}]{\sphinxcrossref{\sphinxcode{\sphinxupquote{\_inherits}}}}}-ed
\end{sphinxadmonition}

\end{fulllineitems}

\index{\_constraints (odoo.models.Model attribute)}

\begin{fulllineitems}
\phantomsection\label{\detokenize{reference/orm:odoo.models.Model._constraints}}\pysigline{\sphinxbfcode{\sphinxupquote{\_constraints}}}
list of \sphinxcode{\sphinxupquote{(constraint\_function, message, fields)}} defining Python
constraints. The fields list is indicative

\DUrole{versionmodified}{Deprecated since version 8.0: }use {\hyperref[\detokenize{reference/orm:odoo.api.constrains}]{\sphinxcrossref{\sphinxcode{\sphinxupquote{constrains()}}}}}

\end{fulllineitems}

\index{\_sql\_constraints (odoo.models.Model attribute)}

\begin{fulllineitems}
\phantomsection\label{\detokenize{reference/orm:odoo.models.Model._sql_constraints}}\pysigline{\sphinxbfcode{\sphinxupquote{\_sql\_constraints}}}
list of \sphinxcode{\sphinxupquote{(name, sql\_definition, message)}} triples defining SQL
constraints to execute when generating the backing table

\end{fulllineitems}

\index{\_parent\_store (odoo.models.Model attribute)}

\begin{fulllineitems}
\phantomsection\label{\detokenize{reference/orm:odoo.models.Model._parent_store}}\pysigline{\sphinxbfcode{\sphinxupquote{\_parent\_store}}}
Alongside {\hyperref[\detokenize{reference/orm:odoo.models.Model.parent_left}]{\sphinxcrossref{\sphinxcode{\sphinxupquote{parent\_left}}}}} and {\hyperref[\detokenize{reference/orm:odoo.models.Model.parent_right}]{\sphinxcrossref{\sphinxcode{\sphinxupquote{parent\_right}}}}}, sets up a
\sphinxhref{http://en.wikipedia.org/wiki/Nested\_set\_model}{nested set}  to
enable fast hierarchical queries on the records of the current model
(default: \sphinxcode{\sphinxupquote{False}})
\begin{quote}\begin{description}
\item[{Type}] \leavevmode
bool

\end{description}\end{quote}

\end{fulllineitems}

\paragraph{CRUD}
\index{create() (odoo.models.Model method)}

\begin{fulllineitems}
\phantomsection\label{\detokenize{reference/orm:odoo.models.Model.create}}\pysiglinewithargsret{\sphinxbfcode{\sphinxupquote{create}}}{\emph{vals}}{{ $\rightarrow$ record}}
Creates a new record for the model.

The new record is initialized using the values from \sphinxcode{\sphinxupquote{vals}} and
if necessary those from {\hyperref[\detokenize{reference/orm:odoo.models.Model.default_get}]{\sphinxcrossref{\sphinxcode{\sphinxupquote{default\_get()}}}}}.
\begin{quote}\begin{description}
\item[{Parameters}] \leavevmode
\sphinxstyleliteralstrong{\sphinxupquote{vals}} (\sphinxhref{https://docs.python.org/3/library/stdtypes.html\#dict}{\sphinxstyleliteralemphasis{\sphinxupquote{dict}}}) \textendash{} 
values for the model’s fields, as a dictionary:

\fvset{hllines={, ,}}%
\begin{sphinxVerbatim}[commandchars=\\\{\}]
\PYG{p}{\PYGZob{}}\PYG{l+s+s1}{\PYGZsq{}}\PYG{l+s+s1}{field\PYGZus{}name}\PYG{l+s+s1}{\PYGZsq{}}\PYG{p}{:} \PYG{n}{field\PYGZus{}value}\PYG{p}{,} \PYG{o}{.}\PYG{o}{.}\PYG{o}{.}\PYG{p}{\PYGZcb{}}
\end{sphinxVerbatim}

see {\hyperref[\detokenize{reference/orm:odoo.models.Model.write}]{\sphinxcrossref{\sphinxcode{\sphinxupquote{write()}}}}} for details


\item[{Returns}] \leavevmode
new record created

\item[{Raises}] \leavevmode\begin{itemize}
\item {} 
{\hyperref[\detokenize{webservices/iap:odoo.exceptions.AccessError}]{\sphinxcrossref{\sphinxstyleliteralstrong{\sphinxupquote{AccessError}}}}} \textendash{} \begin{itemize}
\item {} 
if user has no create rights on the requested object

\item {} 
if user tries to bypass access rules for create on the requested object

\end{itemize}


\item {} 
\sphinxstyleliteralstrong{\sphinxupquote{ValidateError}} \textendash{} if user tries to enter invalid value for a field that is not in selection

\item {} 
{\hyperref[\detokenize{webservices/iap:odoo.exceptions.UserError}]{\sphinxcrossref{\sphinxstyleliteralstrong{\sphinxupquote{UserError}}}}} \textendash{} if a loop would be created in a hierarchy of objects a result of the operation (such as setting an object as its own parent)

\end{itemize}

\end{description}\end{quote}

\end{fulllineitems}

\index{browse() (odoo.models.Model method)}

\begin{fulllineitems}
\phantomsection\label{\detokenize{reference/orm:odoo.models.Model.browse}}\pysiglinewithargsret{\sphinxbfcode{\sphinxupquote{browse}}}{\sphinxoptional{\emph{ids}}}{{ $\rightarrow$ records}}
Returns a recordset for the ids provided as parameter in the current
environment.

Can take no ids, a single id or a sequence of ids.

\end{fulllineitems}

\index{unlink() (odoo.models.Model method)}

\begin{fulllineitems}
\phantomsection\label{\detokenize{reference/orm:odoo.models.Model.unlink}}\pysiglinewithargsret{\sphinxbfcode{\sphinxupquote{unlink}}}{}{}
Deletes the records of the current set
\begin{quote}\begin{description}
\item[{Raises}] \leavevmode\begin{itemize}
\item {} 
{\hyperref[\detokenize{webservices/iap:odoo.exceptions.AccessError}]{\sphinxcrossref{\sphinxstyleliteralstrong{\sphinxupquote{AccessError}}}}} \textendash{} \begin{itemize}
\item {} 
if user has no unlink rights on the requested object

\item {} 
if user tries to bypass access rules for unlink on the requested object

\end{itemize}


\item {} 
{\hyperref[\detokenize{webservices/iap:odoo.exceptions.UserError}]{\sphinxcrossref{\sphinxstyleliteralstrong{\sphinxupquote{UserError}}}}} \textendash{} if the record is default property for other records

\end{itemize}

\end{description}\end{quote}

\end{fulllineitems}

\index{write() (odoo.models.Model method)}

\begin{fulllineitems}
\phantomsection\label{\detokenize{reference/orm:odoo.models.Model.write}}\pysiglinewithargsret{\sphinxbfcode{\sphinxupquote{write}}}{\emph{vals}}{}
Updates all records in the current set with the provided values.
\begin{quote}\begin{description}
\item[{Parameters}] \leavevmode
\sphinxstyleliteralstrong{\sphinxupquote{vals}} (\sphinxhref{https://docs.python.org/3/library/stdtypes.html\#dict}{\sphinxstyleliteralemphasis{\sphinxupquote{dict}}}) \textendash{} 
fields to update and the value to set on them e.g:

\fvset{hllines={, ,}}%
\begin{sphinxVerbatim}[commandchars=\\\{\}]
\PYG{p}{\PYGZob{}}\PYG{l+s+s1}{\PYGZsq{}}\PYG{l+s+s1}{foo}\PYG{l+s+s1}{\PYGZsq{}}\PYG{p}{:} \PYG{l+m+mi}{1}\PYG{p}{,} \PYG{l+s+s1}{\PYGZsq{}}\PYG{l+s+s1}{bar}\PYG{l+s+s1}{\PYGZsq{}}\PYG{p}{:} \PYG{l+s+s2}{\PYGZdq{}}\PYG{l+s+s2}{Qux}\PYG{l+s+s2}{\PYGZdq{}}\PYG{p}{\PYGZcb{}}
\end{sphinxVerbatim}

will set the field \sphinxcode{\sphinxupquote{foo}} to \sphinxcode{\sphinxupquote{1}} and the field \sphinxcode{\sphinxupquote{bar}} to
\sphinxcode{\sphinxupquote{"Qux"}} if those are valid (otherwise it will trigger an error).


\item[{Raises}] \leavevmode\begin{itemize}
\item {} 
{\hyperref[\detokenize{webservices/iap:odoo.exceptions.AccessError}]{\sphinxcrossref{\sphinxstyleliteralstrong{\sphinxupquote{AccessError}}}}} \textendash{} \begin{itemize}
\item {} 
if user has no write rights on the requested object

\item {} 
if user tries to bypass access rules for write on the requested object

\end{itemize}


\item {} 
\sphinxstyleliteralstrong{\sphinxupquote{ValidateError}} \textendash{} if user tries to enter invalid value for a field that is not in selection

\item {} 
{\hyperref[\detokenize{webservices/iap:odoo.exceptions.UserError}]{\sphinxcrossref{\sphinxstyleliteralstrong{\sphinxupquote{UserError}}}}} \textendash{} if a loop would be created in a hierarchy of objects a result of the operation (such as setting an object as its own parent)

\end{itemize}

\end{description}\end{quote}
\begin{itemize}
\item {} 
For numeric fields ({\hyperref[\detokenize{reference/orm:odoo.fields.Integer}]{\sphinxcrossref{\sphinxcode{\sphinxupquote{Integer}}}}},
{\hyperref[\detokenize{reference/orm:odoo.fields.Float}]{\sphinxcrossref{\sphinxcode{\sphinxupquote{Float}}}}}) the value should be of the
corresponding type

\item {} 
For {\hyperref[\detokenize{reference/orm:odoo.fields.Boolean}]{\sphinxcrossref{\sphinxcode{\sphinxupquote{Boolean}}}}}, the value should be a
\sphinxhref{https://docs.python.org/3/library/functions.html\#bool}{\sphinxcode{\sphinxupquote{bool}}}

\item {} 
For {\hyperref[\detokenize{reference/orm:odoo.fields.Selection}]{\sphinxcrossref{\sphinxcode{\sphinxupquote{Selection}}}}}, the value should match the
selection values (generally \sphinxhref{https://docs.python.org/3/library/stdtypes.html\#str}{\sphinxcode{\sphinxupquote{str}}}, sometimes
\sphinxhref{https://docs.python.org/3/library/functions.html\#int}{\sphinxcode{\sphinxupquote{int}}})

\item {} 
For {\hyperref[\detokenize{reference/orm:odoo.fields.Many2one}]{\sphinxcrossref{\sphinxcode{\sphinxupquote{Many2one}}}}}, the value should be the
database identifier of the record to set

\item {} 
Other non-relational fields use a string for value

\begin{sphinxadmonition}{danger}{Danger:}
for historical and compatibility reasons,
{\hyperref[\detokenize{reference/orm:odoo.fields.Date}]{\sphinxcrossref{\sphinxcode{\sphinxupquote{Date}}}}} and
{\hyperref[\detokenize{reference/orm:odoo.fields.Datetime}]{\sphinxcrossref{\sphinxcode{\sphinxupquote{Datetime}}}}} fields use strings as values
(written and read) rather than \sphinxhref{https://docs.python.org/3/library/datetime.html\#datetime.date}{\sphinxcode{\sphinxupquote{date}}} or
\sphinxhref{https://docs.python.org/3/library/datetime.html\#datetime.datetime}{\sphinxcode{\sphinxupquote{datetime}}}. These date strings are
UTC-only and formatted according to
\sphinxcode{\sphinxupquote{odoo.tools.misc.DEFAULT\_SERVER\_DATE\_FORMAT}} and
\sphinxcode{\sphinxupquote{odoo.tools.misc.DEFAULT\_SERVER\_DATETIME\_FORMAT}}
\end{sphinxadmonition}

\item {} \phantomsection\label{\detokenize{reference/orm:openerp-models-relationals-format}}
{\hyperref[\detokenize{reference/orm:odoo.fields.One2many}]{\sphinxcrossref{\sphinxcode{\sphinxupquote{One2many}}}}} and
{\hyperref[\detokenize{reference/orm:odoo.fields.Many2many}]{\sphinxcrossref{\sphinxcode{\sphinxupquote{Many2many}}}}} use a special “commands” format to
manipulate the set of records stored in/associated with the field.

This format is a list of triplets executed sequentially, where each
triplet is a command to execute on the set of records. Not all
commands apply in all situations. Possible commands are:
\begin{description}
\item[{\sphinxcode{\sphinxupquote{(0, \_, values)}}}] \leavevmode
adds a new record created from the provided \sphinxcode{\sphinxupquote{value}} dict.

\item[{\sphinxcode{\sphinxupquote{(1, id, values)}}}] \leavevmode
updates an existing record of id \sphinxcode{\sphinxupquote{id}} with the values in
\sphinxcode{\sphinxupquote{values}}. Can not be used in {\hyperref[\detokenize{reference/orm:odoo.models.Model.create}]{\sphinxcrossref{\sphinxcode{\sphinxupquote{create()}}}}}.

\item[{\sphinxcode{\sphinxupquote{(2, id, \_)}}}] \leavevmode
removes the record of id \sphinxcode{\sphinxupquote{id}} from the set, then deletes it
(from the database). Can not be used in {\hyperref[\detokenize{reference/orm:odoo.models.Model.create}]{\sphinxcrossref{\sphinxcode{\sphinxupquote{create()}}}}}.

\item[{\sphinxcode{\sphinxupquote{(3, id, \_)}}}] \leavevmode
removes the record of id \sphinxcode{\sphinxupquote{id}} from the set, but does not
delete it. Can not be used on
{\hyperref[\detokenize{reference/orm:odoo.fields.One2many}]{\sphinxcrossref{\sphinxcode{\sphinxupquote{One2many}}}}}. Can not be used in
{\hyperref[\detokenize{reference/orm:odoo.models.Model.create}]{\sphinxcrossref{\sphinxcode{\sphinxupquote{create()}}}}}.

\item[{\sphinxcode{\sphinxupquote{(4, id, \_)}}}] \leavevmode
adds an existing record of id \sphinxcode{\sphinxupquote{id}} to the set. Can not be
used on {\hyperref[\detokenize{reference/orm:odoo.fields.One2many}]{\sphinxcrossref{\sphinxcode{\sphinxupquote{One2many}}}}}.

\item[{\sphinxcode{\sphinxupquote{(5, \_, \_)}}}] \leavevmode
removes all records from the set, equivalent to using the
command \sphinxcode{\sphinxupquote{3}} on every record explicitly. Can not be used on
{\hyperref[\detokenize{reference/orm:odoo.fields.One2many}]{\sphinxcrossref{\sphinxcode{\sphinxupquote{One2many}}}}}. Can not be used in
{\hyperref[\detokenize{reference/orm:odoo.models.Model.create}]{\sphinxcrossref{\sphinxcode{\sphinxupquote{create()}}}}}.

\item[{\sphinxcode{\sphinxupquote{(6, \_, ids)}}}] \leavevmode
replaces all existing records in the set by the \sphinxcode{\sphinxupquote{ids}} list,
equivalent to using the command \sphinxcode{\sphinxupquote{5}} followed by a command
\sphinxcode{\sphinxupquote{4}} for each \sphinxcode{\sphinxupquote{id}} in \sphinxcode{\sphinxupquote{ids}}.

\end{description}

\begin{sphinxadmonition}{note}{Note:}
Values marked as \sphinxcode{\sphinxupquote{\_}} in the list above are ignored and
can be anything, generally \sphinxcode{\sphinxupquote{0}} or \sphinxcode{\sphinxupquote{False}}.
\end{sphinxadmonition}

\end{itemize}

\end{fulllineitems}

\index{read() (odoo.models.Model method)}

\begin{fulllineitems}
\phantomsection\label{\detokenize{reference/orm:odoo.models.Model.read}}\pysiglinewithargsret{\sphinxbfcode{\sphinxupquote{read}}}{\sphinxoptional{\emph{fields}}}{}
Reads the requested fields for the records in \sphinxcode{\sphinxupquote{self}}, low-level/RPC
method. In Python code, prefer {\hyperref[\detokenize{reference/orm:odoo.models.Model.browse}]{\sphinxcrossref{\sphinxcode{\sphinxupquote{browse()}}}}}.
\begin{quote}\begin{description}
\item[{Parameters}] \leavevmode
\sphinxstyleliteralstrong{\sphinxupquote{fields}} \textendash{} list of field names to return (default is all fields)

\item[{Returns}] \leavevmode
a list of dictionaries mapping field names to their values,
with one dictionary per record

\item[{Raises}] \leavevmode
{\hyperref[\detokenize{webservices/iap:odoo.exceptions.AccessError}]{\sphinxcrossref{\sphinxstyleliteralstrong{\sphinxupquote{AccessError}}}}} \textendash{} if user has no read rights on some of the given
records

\end{description}\end{quote}

\end{fulllineitems}

\index{read\_group() (odoo.models.Model method)}

\begin{fulllineitems}
\phantomsection\label{\detokenize{reference/orm:odoo.models.Model.read_group}}\pysiglinewithargsret{\sphinxbfcode{\sphinxupquote{read\_group}}}{\emph{domain}, \emph{fields}, \emph{groupby}, \emph{offset=0}, \emph{limit=None}, \emph{orderby=False}, \emph{lazy=True}}{}
Get the list of records in list view grouped by the given \sphinxcode{\sphinxupquote{groupby}} fields
\begin{quote}\begin{description}
\item[{Parameters}] \leavevmode\begin{itemize}
\item {} 
\sphinxstyleliteralstrong{\sphinxupquote{domain}} \textendash{} list specifying search criteria {[}{[}‘field\_name’, ‘operator’, ‘value’{]}, …{]}

\item {} 
\sphinxstyleliteralstrong{\sphinxupquote{fields}} (\sphinxhref{https://docs.python.org/3/library/stdtypes.html\#list}{\sphinxstyleliteralemphasis{\sphinxupquote{list}}}) \textendash{} list of fields present in the list view specified on the object

\item {} 
\sphinxstyleliteralstrong{\sphinxupquote{groupby}} (\sphinxhref{https://docs.python.org/3/library/stdtypes.html\#list}{\sphinxstyleliteralemphasis{\sphinxupquote{list}}}) \textendash{} list of groupby descriptions by which the records will be grouped.  
A groupby description is either a field (then it will be grouped by that field)
or a string ‘field:groupby\_function’.  Right now, the only functions supported
are ‘day’, ‘week’, ‘month’, ‘quarter’ or ‘year’, and they only make sense for 
date/datetime fields.

\item {} 
\sphinxstyleliteralstrong{\sphinxupquote{offset}} (\sphinxhref{https://docs.python.org/3/library/functions.html\#int}{\sphinxstyleliteralemphasis{\sphinxupquote{int}}}) \textendash{} optional number of records to skip

\item {} 
\sphinxstyleliteralstrong{\sphinxupquote{limit}} (\sphinxhref{https://docs.python.org/3/library/functions.html\#int}{\sphinxstyleliteralemphasis{\sphinxupquote{int}}}) \textendash{} optional max number of records to return

\item {} 
\sphinxstyleliteralstrong{\sphinxupquote{orderby}} (\sphinxhref{https://docs.python.org/3/library/stdtypes.html\#list}{\sphinxstyleliteralemphasis{\sphinxupquote{list}}}) \textendash{} optional \sphinxcode{\sphinxupquote{order by}} specification, for
overriding the natural sort ordering of the
groups, see also \sphinxcode{\sphinxupquote{search()}}
(supported only for many2one fields currently)

\item {} 
\sphinxstyleliteralstrong{\sphinxupquote{lazy}} (\sphinxhref{https://docs.python.org/3/library/functions.html\#bool}{\sphinxstyleliteralemphasis{\sphinxupquote{bool}}}) \textendash{} if true, the results are only grouped by the first groupby and the 
remaining groupbys are put in the \_\_context key.  If false, all the groupbys are
done in one call.

\end{itemize}

\item[{Returns}] \leavevmode

list of dictionaries(one dictionary for each record) containing:
\begin{itemize}
\item {} 
the values of fields grouped by the fields in \sphinxcode{\sphinxupquote{groupby}} argument

\item {} 
\_\_domain: list of tuples specifying the search criteria

\item {} 
\_\_context: dictionary with argument like \sphinxcode{\sphinxupquote{groupby}}

\end{itemize}


\item[{Return type}] \leavevmode
{[}\{‘field\_name\_1’: value, ..{]}

\item[{Raises}] \leavevmode
{\hyperref[\detokenize{webservices/iap:odoo.exceptions.AccessError}]{\sphinxcrossref{\sphinxstyleliteralstrong{\sphinxupquote{AccessError}}}}} \textendash{} \begin{itemize}
\item {} 
if user has no read rights on the requested object

\item {} 
if user tries to bypass access rules for read on the requested object

\end{itemize}


\end{description}\end{quote}

\end{fulllineitems}

\paragraph{Searching}
\index{search() (odoo.models.Model method)}

\begin{fulllineitems}
\phantomsection\label{\detokenize{reference/orm:odoo.models.Model.search}}\pysiglinewithargsret{\sphinxbfcode{\sphinxupquote{search}}}{\emph{args{[}, offset=0{]}{[}, limit=None{]}{[}, order=None{]}{[}, count=False{]}}}{}
Searches for records based on the \sphinxcode{\sphinxupquote{args}}
{\hyperref[\detokenize{reference/orm:reference-orm-domains}]{\sphinxcrossref{\DUrole{std,std-ref}{search domain}}}}.
\begin{quote}\begin{description}
\item[{Parameters}] \leavevmode\begin{itemize}
\item {} 
\sphinxstyleliteralstrong{\sphinxupquote{args}} \textendash{} {\hyperref[\detokenize{reference/orm:reference-orm-domains}]{\sphinxcrossref{\DUrole{std,std-ref}{A search domain}}}}. Use an empty
list to match all records.

\item {} 
\sphinxstyleliteralstrong{\sphinxupquote{offset}} (\sphinxhref{https://docs.python.org/3/library/functions.html\#int}{\sphinxstyleliteralemphasis{\sphinxupquote{int}}}) \textendash{} number of results to ignore (default: none)

\item {} 
\sphinxstyleliteralstrong{\sphinxupquote{limit}} (\sphinxhref{https://docs.python.org/3/library/functions.html\#int}{\sphinxstyleliteralemphasis{\sphinxupquote{int}}}) \textendash{} maximum number of records to return (default: all)

\item {} 
\sphinxstyleliteralstrong{\sphinxupquote{order}} (\sphinxhref{https://docs.python.org/3/library/stdtypes.html\#str}{\sphinxstyleliteralemphasis{\sphinxupquote{str}}}) \textendash{} sort string

\item {} 
\sphinxstyleliteralstrong{\sphinxupquote{count}} (\sphinxhref{https://docs.python.org/3/library/functions.html\#bool}{\sphinxstyleliteralemphasis{\sphinxupquote{bool}}}) \textendash{} if True, only counts and returns the number of matching records (default: False)

\end{itemize}

\item[{Returns}] \leavevmode
at most \sphinxcode{\sphinxupquote{limit}} records matching the search criteria

\item[{Raises}] \leavevmode
{\hyperref[\detokenize{webservices/iap:odoo.exceptions.AccessError}]{\sphinxcrossref{\sphinxstyleliteralstrong{\sphinxupquote{AccessError}}}}} \textendash{} \begin{itemize}
\item {} 
if user tries to bypass access rules for read on the requested object.

\end{itemize}


\end{description}\end{quote}

\end{fulllineitems}

\index{search\_count() (odoo.models.Model method)}

\begin{fulllineitems}
\phantomsection\label{\detokenize{reference/orm:odoo.models.Model.search_count}}\pysiglinewithargsret{\sphinxbfcode{\sphinxupquote{search\_count}}}{\emph{args}}{{ $\rightarrow$ int}}
Returns the number of records in the current model matching {\hyperref[\detokenize{reference/orm:reference-orm-domains}]{\sphinxcrossref{\DUrole{std,std-ref}{the
provided domain}}}}.

\end{fulllineitems}

\index{name\_search() (odoo.models.Model method)}

\begin{fulllineitems}
\phantomsection\label{\detokenize{reference/orm:odoo.models.Model.name_search}}\pysiglinewithargsret{\sphinxbfcode{\sphinxupquote{name\_search}}}{\emph{name=''}, \emph{args=None}, \emph{operator='ilike'}, \emph{limit=100}}{{ $\rightarrow$ records}}
Search for records that have a display name matching the given
\sphinxcode{\sphinxupquote{name}} pattern when compared with the given \sphinxcode{\sphinxupquote{operator}}, while also
matching the optional search domain (\sphinxcode{\sphinxupquote{args}}).

This is used for example to provide suggestions based on a partial
value for a relational field. Sometimes be seen as the inverse
function of {\hyperref[\detokenize{reference/orm:odoo.models.Model.name_get}]{\sphinxcrossref{\sphinxcode{\sphinxupquote{name\_get()}}}}}, but it is not guaranteed to be.

This method is equivalent to calling {\hyperref[\detokenize{reference/orm:odoo.models.Model.search}]{\sphinxcrossref{\sphinxcode{\sphinxupquote{search()}}}}} with a search
domain based on \sphinxcode{\sphinxupquote{display\_name}} and then {\hyperref[\detokenize{reference/orm:odoo.models.Model.name_get}]{\sphinxcrossref{\sphinxcode{\sphinxupquote{name\_get()}}}}} on the
result of the search.
\begin{quote}\begin{description}
\item[{Parameters}] \leavevmode\begin{itemize}
\item {} 
\sphinxstyleliteralstrong{\sphinxupquote{name}} (\sphinxhref{https://docs.python.org/3/library/stdtypes.html\#str}{\sphinxstyleliteralemphasis{\sphinxupquote{str}}}) \textendash{} the name pattern to match

\item {} 
\sphinxstyleliteralstrong{\sphinxupquote{args}} (\sphinxhref{https://docs.python.org/3/library/stdtypes.html\#list}{\sphinxstyleliteralemphasis{\sphinxupquote{list}}}) \textendash{} optional search domain (see {\hyperref[\detokenize{reference/orm:odoo.models.Model.search}]{\sphinxcrossref{\sphinxcode{\sphinxupquote{search()}}}}} for
syntax), specifying further restrictions

\item {} 
\sphinxstyleliteralstrong{\sphinxupquote{operator}} (\sphinxhref{https://docs.python.org/3/library/stdtypes.html\#str}{\sphinxstyleliteralemphasis{\sphinxupquote{str}}}) \textendash{} domain operator for matching \sphinxcode{\sphinxupquote{name}}, such as
\sphinxcode{\sphinxupquote{'like'}} or \sphinxcode{\sphinxupquote{'='}}.

\item {} 
\sphinxstyleliteralstrong{\sphinxupquote{limit}} (\sphinxhref{https://docs.python.org/3/library/functions.html\#int}{\sphinxstyleliteralemphasis{\sphinxupquote{int}}}) \textendash{} optional max number of records to return

\end{itemize}

\item[{Return type}] \leavevmode
\sphinxhref{https://docs.python.org/3/library/stdtypes.html\#list}{list}

\item[{Returns}] \leavevmode
list of pairs \sphinxcode{\sphinxupquote{(id, text\_repr)}} for all matching records.

\end{description}\end{quote}

\end{fulllineitems}

\paragraph{Recordset operations}
\index{ids (odoo.models.Model attribute)}

\begin{fulllineitems}
\phantomsection\label{\detokenize{reference/orm:odoo.models.Model.ids}}\pysigline{\sphinxbfcode{\sphinxupquote{ids}}}
List of actual record ids in this recordset (ignores placeholder
ids for records to create)

\end{fulllineitems}

\index{ensure\_one() (odoo.models.Model method)}

\begin{fulllineitems}
\phantomsection\label{\detokenize{reference/orm:odoo.models.Model.ensure_one}}\pysiglinewithargsret{\sphinxbfcode{\sphinxupquote{ensure\_one}}}{}{}
Verifies that the current recorset holds a single record. Raises
an exception otherwise.

\end{fulllineitems}

\index{exists() (odoo.models.Model method)}

\begin{fulllineitems}
\phantomsection\label{\detokenize{reference/orm:odoo.models.Model.exists}}\pysiglinewithargsret{\sphinxbfcode{\sphinxupquote{exists}}}{}{{ $\rightarrow$ records}}
Returns the subset of records in \sphinxcode{\sphinxupquote{self}} that exist, and marks deleted
records as such in cache. It can be used as a test on records:

\fvset{hllines={, ,}}%
\begin{sphinxVerbatim}[commandchars=\\\{\}]
\PYG{k}{if} \PYG{n}{record}\PYG{o}{.}\PYG{n}{exists}\PYG{p}{(}\PYG{p}{)}\PYG{p}{:}
    \PYG{o}{.}\PYG{o}{.}\PYG{o}{.}
\end{sphinxVerbatim}

By convention, new records are returned as existing.

\end{fulllineitems}

\index{filtered() (odoo.models.Model method)}

\begin{fulllineitems}
\phantomsection\label{\detokenize{reference/orm:odoo.models.Model.filtered}}\pysiglinewithargsret{\sphinxbfcode{\sphinxupquote{filtered}}}{\emph{func}}{}
Select the records in \sphinxcode{\sphinxupquote{self}} such that \sphinxcode{\sphinxupquote{func(rec)}} is true, and
return them as a recordset.
\begin{quote}\begin{description}
\item[{Parameters}] \leavevmode
\sphinxstyleliteralstrong{\sphinxupquote{func}} \textendash{} a function or a dot-separated sequence of field names

\end{description}\end{quote}

\end{fulllineitems}

\index{sorted() (odoo.models.Model method)}

\begin{fulllineitems}
\phantomsection\label{\detokenize{reference/orm:odoo.models.Model.sorted}}\pysiglinewithargsret{\sphinxbfcode{\sphinxupquote{sorted}}}{\emph{key=None}, \emph{reverse=False}}{}
Return the recordset \sphinxcode{\sphinxupquote{self}} ordered by \sphinxcode{\sphinxupquote{key}}.
\begin{quote}\begin{description}
\item[{Parameters}] \leavevmode\begin{itemize}
\item {} 
\sphinxstyleliteralstrong{\sphinxupquote{key}} \textendash{} either a function of one argument that returns a
comparison key for each record, or a field name, or \sphinxcode{\sphinxupquote{None}}, in
which case records are ordered according the default model’s order

\item {} 
\sphinxstyleliteralstrong{\sphinxupquote{reverse}} \textendash{} if \sphinxcode{\sphinxupquote{True}}, return the result in reverse order

\end{itemize}

\end{description}\end{quote}

\end{fulllineitems}

\index{mapped() (odoo.models.Model method)}

\begin{fulllineitems}
\phantomsection\label{\detokenize{reference/orm:odoo.models.Model.mapped}}\pysiglinewithargsret{\sphinxbfcode{\sphinxupquote{mapped}}}{\emph{func}}{}
Apply \sphinxcode{\sphinxupquote{func}} on all records in \sphinxcode{\sphinxupquote{self}}, and return the result as a
list or a recordset (if \sphinxcode{\sphinxupquote{func}} return recordsets). In the latter
case, the order of the returned recordset is arbitrary.
\begin{quote}\begin{description}
\item[{Parameters}] \leavevmode
\sphinxstyleliteralstrong{\sphinxupquote{func}} \textendash{} a function or a dot-separated sequence of field names
(string); any falsy value simply returns the recordset \sphinxcode{\sphinxupquote{self}}

\end{description}\end{quote}

\end{fulllineitems}

\paragraph{Environment swapping}
\index{sudo() (odoo.models.Model method)}

\begin{fulllineitems}
\phantomsection\label{\detokenize{reference/orm:odoo.models.Model.sudo}}\pysiglinewithargsret{\sphinxbfcode{\sphinxupquote{sudo}}}{\sphinxoptional{\emph{user=SUPERUSER}}}{}
Returns a new version of this recordset attached to the provided
user.

By default this returns a \sphinxcode{\sphinxupquote{SUPERUSER}} recordset, where access
control and record rules are bypassed.

\begin{sphinxadmonition}{note}{Note:}
Using \sphinxcode{\sphinxupquote{sudo}} could cause data access to cross the
boundaries of record rules, possibly mixing records that
are meant to be isolated (e.g. records from different
companies in multi-company environments).

It may lead to un-intuitive results in methods which select one
record among many - for example getting the default company, or
selecting a Bill of Materials.
\end{sphinxadmonition}

\begin{sphinxadmonition}{note}{Note:}
Because the record rules and access control will have to be
re-evaluated, the new recordset will not benefit from the current
environment’s data cache, so later data access may incur extra
delays while re-fetching from the database.
The returned recordset has the same prefetch object as \sphinxcode{\sphinxupquote{self}}.
\end{sphinxadmonition}

\end{fulllineitems}

\index{with\_context() (odoo.models.Model method)}

\begin{fulllineitems}
\phantomsection\label{\detokenize{reference/orm:odoo.models.Model.with_context}}\pysiglinewithargsret{\sphinxbfcode{\sphinxupquote{with\_context}}}{\emph{{[}context{]}{[}, **overrides{]}}}{{ $\rightarrow$ records}}
Returns a new version of this recordset attached to an extended
context.

The extended context is either the provided \sphinxcode{\sphinxupquote{context}} in which
\sphinxcode{\sphinxupquote{overrides}} are merged or the \sphinxstyleemphasis{current} context in which
\sphinxcode{\sphinxupquote{overrides}} are merged e.g.:

\fvset{hllines={, ,}}%
\begin{sphinxVerbatim}[commandchars=\\\{\}]
\PYG{c+c1}{\PYGZsh{} current context is \PYGZob{}\PYGZsq{}key1\PYGZsq{}: True\PYGZcb{}}
\PYG{n}{r2} \PYG{o}{=} \PYG{n}{records}\PYG{o}{.}\PYG{n}{with\PYGZus{}context}\PYG{p}{(}\PYG{p}{\PYGZob{}}\PYG{p}{\PYGZcb{}}\PYG{p}{,} \PYG{n}{key2}\PYG{o}{=}\PYG{k+kc}{True}\PYG{p}{)}
\PYG{c+c1}{\PYGZsh{} \PYGZhy{}\PYGZgt{} r2.\PYGZus{}context is \PYGZob{}\PYGZsq{}key2\PYGZsq{}: True\PYGZcb{}}
\PYG{n}{r2} \PYG{o}{=} \PYG{n}{records}\PYG{o}{.}\PYG{n}{with\PYGZus{}context}\PYG{p}{(}\PYG{n}{key2}\PYG{o}{=}\PYG{k+kc}{True}\PYG{p}{)}
\PYG{c+c1}{\PYGZsh{} \PYGZhy{}\PYGZgt{} r2.\PYGZus{}context is \PYGZob{}\PYGZsq{}key1\PYGZsq{}: True, \PYGZsq{}key2\PYGZsq{}: True\PYGZcb{}}
\end{sphinxVerbatim}

\end{fulllineitems}

\index{with\_env() (odoo.models.Model method)}

\begin{fulllineitems}
\phantomsection\label{\detokenize{reference/orm:odoo.models.Model.with_env}}\pysiglinewithargsret{\sphinxbfcode{\sphinxupquote{with\_env}}}{\emph{env}}{}
Returns a new version of this recordset attached to the provided
environment

\begin{sphinxadmonition}{warning}{Warning:}
The new environment will not benefit from the current
environment’s data cache, so later data access may incur extra
delays while re-fetching from the database.
The returned recordset has the same prefetch object as \sphinxcode{\sphinxupquote{self}}.
\end{sphinxadmonition}
\begin{quote}\begin{description}
\end{description}\end{quote}

\end{fulllineitems}

\paragraph{Fields and views querying}
\index{fields\_get() (odoo.models.Model method)}

\begin{fulllineitems}
\phantomsection\label{\detokenize{reference/orm:odoo.models.Model.fields_get}}\pysiglinewithargsret{\sphinxbfcode{\sphinxupquote{fields\_get}}}{\emph{{[}fields{]}{[}, attributes{]}}}{}
Return the definition of each field.

The returned value is a dictionary (indiced by field name) of
dictionaries. The \_inherits’d fields are included. The string, help,
and selection (if present) attributes are translated.
\begin{quote}\begin{description}
\item[{Parameters}] \leavevmode\begin{itemize}
\item {} 
\sphinxstyleliteralstrong{\sphinxupquote{allfields}} \textendash{} list of fields to document, all if empty or not provided

\item {} 
\sphinxstyleliteralstrong{\sphinxupquote{attributes}} \textendash{} list of description attributes to return for each field, all if empty or not provided

\end{itemize}

\end{description}\end{quote}

\end{fulllineitems}

\index{fields\_view\_get() (odoo.models.Model method)}

\begin{fulllineitems}
\phantomsection\label{\detokenize{reference/orm:odoo.models.Model.fields_view_get}}\pysiglinewithargsret{\sphinxbfcode{\sphinxupquote{fields\_view\_get}}}{\sphinxoptional{\emph{view\_id \textbar{} view\_type='form'}}}{}
Get the detailed composition of the requested view like fields, model, view architecture
\begin{quote}\begin{description}
\item[{Parameters}] \leavevmode\begin{itemize}
\item {} 
\sphinxstyleliteralstrong{\sphinxupquote{view\_id}} \textendash{} id of the view or None

\item {} 
\sphinxstyleliteralstrong{\sphinxupquote{view\_type}} \textendash{} type of the view to return if view\_id is None (‘form’, ‘tree’, …)

\item {} 
\sphinxstyleliteralstrong{\sphinxupquote{toolbar}} \textendash{} true to include contextual actions

\item {} 
\sphinxstyleliteralstrong{\sphinxupquote{submenu}} \textendash{} deprecated

\end{itemize}

\item[{Returns}] \leavevmode
dictionary describing the composition of the requested view (including inherited views and extensions)

\item[{Raises}] \leavevmode\begin{itemize}
\item {} 
\sphinxhref{https://docs.python.org/3/library/exceptions.html\#AttributeError}{\sphinxstyleliteralstrong{\sphinxupquote{AttributeError}}} \textendash{} \begin{itemize}
\item {} 
if the inherited view has unknown position to work with other than ‘before’, ‘after’, ‘inside’, ‘replace’

\item {} 
if some tag other than ‘position’ is found in parent view

\end{itemize}


\item {} 
\sphinxstyleliteralstrong{\sphinxupquote{Invalid ArchitectureError}} \textendash{} if there is view type other than form, tree, calendar, search etc defined on the structure

\end{itemize}

\end{description}\end{quote}

\end{fulllineitems}

\paragraph{Miscellaneous methods}
\index{default\_get() (odoo.models.Model method)}

\begin{fulllineitems}
\phantomsection\label{\detokenize{reference/orm:odoo.models.Model.default_get}}\pysiglinewithargsret{\sphinxbfcode{\sphinxupquote{default\_get}}}{\emph{fields}}{{ $\rightarrow$ default\_values}}
Return default values for the fields in \sphinxcode{\sphinxupquote{fields\_list}}. Default
values are determined by the context, user defaults, and the model
itself.
\begin{quote}\begin{description}
\item[{Parameters}] \leavevmode
\sphinxstyleliteralstrong{\sphinxupquote{fields\_list}} \textendash{} a list of field names

\item[{Returns}] \leavevmode
a dictionary mapping each field name to its corresponding
default value, if it has one.

\end{description}\end{quote}

\end{fulllineitems}

\index{copy() (odoo.models.Model method)}

\begin{fulllineitems}
\phantomsection\label{\detokenize{reference/orm:odoo.models.Model.copy}}\pysiglinewithargsret{\sphinxbfcode{\sphinxupquote{copy}}}{\emph{default=None}}{}
Duplicate record \sphinxcode{\sphinxupquote{self}} updating it with default values
\begin{quote}\begin{description}
\item[{Parameters}] \leavevmode
\sphinxstyleliteralstrong{\sphinxupquote{default}} (\sphinxhref{https://docs.python.org/3/library/stdtypes.html\#dict}{\sphinxstyleliteralemphasis{\sphinxupquote{dict}}}) \textendash{} dictionary of field values to override in the
original values of the copied record, e.g: \sphinxcode{\sphinxupquote{\{'field\_name': overridden\_value, ...\}}}

\item[{Returns}] \leavevmode
new record

\end{description}\end{quote}

\end{fulllineitems}

\index{name\_get() (odoo.models.Model method)}

\begin{fulllineitems}
\phantomsection\label{\detokenize{reference/orm:odoo.models.Model.name_get}}\pysiglinewithargsret{\sphinxbfcode{\sphinxupquote{name\_get}}}{}{{ $\rightarrow$ {[}(id, name), ...{]}}}
Returns a textual representation for the records in \sphinxcode{\sphinxupquote{self}}.
By default this is the value of the \sphinxcode{\sphinxupquote{display\_name}} field.
\begin{quote}\begin{description}
\item[{Returns}] \leavevmode
list of pairs \sphinxcode{\sphinxupquote{(id, text\_repr)}} for each records

\item[{Return type}] \leavevmode
\sphinxhref{https://docs.python.org/3/library/stdtypes.html\#list}{list}(\sphinxhref{https://docs.python.org/3/library/stdtypes.html\#tuple}{tuple})

\end{description}\end{quote}

\end{fulllineitems}

\index{name\_create() (odoo.models.Model method)}

\begin{fulllineitems}
\phantomsection\label{\detokenize{reference/orm:odoo.models.Model.name_create}}\pysiglinewithargsret{\sphinxbfcode{\sphinxupquote{name\_create}}}{\emph{name}}{{ $\rightarrow$ record}}
Create a new record by calling {\hyperref[\detokenize{reference/orm:odoo.models.Model.create}]{\sphinxcrossref{\sphinxcode{\sphinxupquote{create()}}}}} with only one value
provided: the display name of the new record.

The new record will be initialized with any default values
applicable to this model, or provided through the context. The usual
behavior of {\hyperref[\detokenize{reference/orm:odoo.models.Model.create}]{\sphinxcrossref{\sphinxcode{\sphinxupquote{create()}}}}} applies.
\begin{quote}\begin{description}
\item[{Parameters}] \leavevmode
\sphinxstyleliteralstrong{\sphinxupquote{name}} \textendash{} display name of the record to create

\item[{Return type}] \leavevmode
\sphinxhref{https://docs.python.org/3/library/stdtypes.html\#tuple}{tuple}

\item[{Returns}] \leavevmode
the {\hyperref[\detokenize{reference/orm:odoo.models.Model.name_get}]{\sphinxcrossref{\sphinxcode{\sphinxupquote{name\_get()}}}}} pair value of the created record

\end{description}\end{quote}

\end{fulllineitems}

\phantomsection\label{\detokenize{reference/orm:reference-orm-model-automatic}}\paragraph{Automatic fields}
\index{id (odoo.models.Model attribute)}

\begin{fulllineitems}
\phantomsection\label{\detokenize{reference/orm:odoo.models.Model.id}}\pysigline{\sphinxbfcode{\sphinxupquote{id}}}
Identifier {\hyperref[\detokenize{reference/orm:odoo.fields.Field}]{\sphinxcrossref{\sphinxcode{\sphinxupquote{field}}}}}

\end{fulllineitems}

\index{\_log\_access (odoo.models.Model attribute)}

\begin{fulllineitems}
\phantomsection\label{\detokenize{reference/orm:odoo.models.Model._log_access}}\pysigline{\sphinxbfcode{\sphinxupquote{\_log\_access}}}
Whether log access fields (\sphinxcode{\sphinxupquote{create\_date}}, \sphinxcode{\sphinxupquote{write\_uid}}, …) should
be generated (default: \sphinxcode{\sphinxupquote{True}})

\end{fulllineitems}

\index{create\_date (odoo.models.Model attribute)}

\begin{fulllineitems}
\phantomsection\label{\detokenize{reference/orm:odoo.models.Model.create_date}}\pysigline{\sphinxbfcode{\sphinxupquote{create\_date}}}
Date at which the record was created
\begin{quote}\begin{description}
\item[{Type}] \leavevmode
\sphinxcode{\sphinxupquote{Datetime}}

\end{description}\end{quote}

\end{fulllineitems}

\index{create\_uid (odoo.models.Model attribute)}

\begin{fulllineitems}
\phantomsection\label{\detokenize{reference/orm:odoo.models.Model.create_uid}}\pysigline{\sphinxbfcode{\sphinxupquote{create\_uid}}}
Relational field to the user who created the record
\begin{quote}\begin{description}
\item[{Type}] \leavevmode
\sphinxcode{\sphinxupquote{res.users}}

\end{description}\end{quote}

\end{fulllineitems}

\index{write\_date (odoo.models.Model attribute)}

\begin{fulllineitems}
\phantomsection\label{\detokenize{reference/orm:odoo.models.Model.write_date}}\pysigline{\sphinxbfcode{\sphinxupquote{write\_date}}}
Date at which the record was last modified
\begin{quote}\begin{description}
\item[{Type}] \leavevmode
\sphinxcode{\sphinxupquote{Datetime}}

\end{description}\end{quote}

\end{fulllineitems}

\index{write\_uid (odoo.models.Model attribute)}

\begin{fulllineitems}
\phantomsection\label{\detokenize{reference/orm:odoo.models.Model.write_uid}}\pysigline{\sphinxbfcode{\sphinxupquote{write\_uid}}}
Relational field to the last user who modified the record
\begin{quote}\begin{description}
\item[{Type}] \leavevmode
\sphinxcode{\sphinxupquote{res.users}}

\end{description}\end{quote}

\end{fulllineitems}

\paragraph{Reserved field names}

A few field names are reserved for pre-defined behaviors beyond that of
automated fields. They should be defined on a model when the related
behavior is desired:
\index{name (odoo.models.Model attribute)}

\begin{fulllineitems}
\phantomsection\label{\detokenize{reference/orm:odoo.models.Model.name}}\pysigline{\sphinxbfcode{\sphinxupquote{name}}}
default value for {\hyperref[\detokenize{reference/orm:odoo.models.Model._rec_name}]{\sphinxcrossref{\sphinxcode{\sphinxupquote{\_rec\_name}}}}}, used to
display records in context where a representative “naming” is
necessary.
\begin{quote}\begin{description}
\item[{Type}] \leavevmode
{\hyperref[\detokenize{reference/orm:odoo.fields.Char}]{\sphinxcrossref{\sphinxcode{\sphinxupquote{Char}}}}}

\end{description}\end{quote}

\end{fulllineitems}

\index{active (odoo.models.Model attribute)}

\begin{fulllineitems}
\phantomsection\label{\detokenize{reference/orm:odoo.models.Model.active}}\pysigline{\sphinxbfcode{\sphinxupquote{active}}}
toggles the global visibility of the record, if \sphinxcode{\sphinxupquote{active}} is set to
\sphinxcode{\sphinxupquote{False}} the record is invisible in most searches and listing
\begin{quote}\begin{description}
\item[{Type}] \leavevmode
{\hyperref[\detokenize{reference/orm:odoo.fields.Boolean}]{\sphinxcrossref{\sphinxcode{\sphinxupquote{Boolean}}}}}

\end{description}\end{quote}

\end{fulllineitems}

\index{sequence (odoo.models.Model attribute)}

\begin{fulllineitems}
\phantomsection\label{\detokenize{reference/orm:odoo.models.Model.sequence}}\pysigline{\sphinxbfcode{\sphinxupquote{sequence}}}
Alterable ordering criteria, allows drag-and-drop reordering of models
in list views
\begin{quote}\begin{description}
\item[{Type}] \leavevmode
{\hyperref[\detokenize{reference/orm:odoo.fields.Integer}]{\sphinxcrossref{\sphinxcode{\sphinxupquote{Integer}}}}}

\end{description}\end{quote}

\end{fulllineitems}

\index{state (odoo.models.Model attribute)}

\begin{fulllineitems}
\phantomsection\label{\detokenize{reference/orm:odoo.models.Model.state}}\pysigline{\sphinxbfcode{\sphinxupquote{state}}}
lifecycle stages of the object, used by the \sphinxcode{\sphinxupquote{states}} attribute on
{\hyperref[\detokenize{reference/orm:odoo.fields.Field}]{\sphinxcrossref{\sphinxcode{\sphinxupquote{fields}}}}}
\begin{quote}\begin{description}
\item[{Type}] \leavevmode
{\hyperref[\detokenize{reference/orm:odoo.fields.Selection}]{\sphinxcrossref{\sphinxcode{\sphinxupquote{Selection}}}}}

\end{description}\end{quote}

\end{fulllineitems}

\index{parent\_id (odoo.models.Model attribute)}

\begin{fulllineitems}
\phantomsection\label{\detokenize{reference/orm:odoo.models.Model.parent_id}}\pysigline{\sphinxbfcode{\sphinxupquote{parent\_id}}}
used to order records in a tree structure and enables the \sphinxcode{\sphinxupquote{child\_of}}
operator in domains
\begin{quote}\begin{description}
\item[{Type}] \leavevmode
{\hyperref[\detokenize{reference/orm:odoo.fields.Many2one}]{\sphinxcrossref{\sphinxcode{\sphinxupquote{Many2one}}}}}

\end{description}\end{quote}

\end{fulllineitems}

\index{parent\_left (odoo.models.Model attribute)}

\begin{fulllineitems}
\phantomsection\label{\detokenize{reference/orm:odoo.models.Model.parent_left}}\pysigline{\sphinxbfcode{\sphinxupquote{parent\_left}}}
used with {\hyperref[\detokenize{reference/orm:odoo.models.Model._parent_store}]{\sphinxcrossref{\sphinxcode{\sphinxupquote{\_parent\_store}}}}}, allows faster tree structure access

\end{fulllineitems}

\index{parent\_right (odoo.models.Model attribute)}

\begin{fulllineitems}
\phantomsection\label{\detokenize{reference/orm:odoo.models.Model.parent_right}}\pysigline{\sphinxbfcode{\sphinxupquote{parent\_right}}}
see {\hyperref[\detokenize{reference/orm:odoo.models.Model.parent_left}]{\sphinxcrossref{\sphinxcode{\sphinxupquote{parent\_left}}}}}

\end{fulllineitems}


\end{fulllineitems}



\subsection{Method decorators}
\label{\detokenize{reference/orm:reference-orm-decorators}}\label{\detokenize{reference/orm:method-decorators}}\label{\detokenize{reference/orm:module-odoo.api}}\index{odoo.api (module)}
This module provides the elements for managing two different API styles,
namely the “traditional” and “record” styles.

In the “traditional” style, parameters like the database cursor, user id,
context dictionary and record ids (usually denoted as \sphinxcode{\sphinxupquote{cr}}, \sphinxcode{\sphinxupquote{uid}},
\sphinxcode{\sphinxupquote{context}}, \sphinxcode{\sphinxupquote{ids}}) are passed explicitly to all methods. In the “record”
style, those parameters are hidden into model instances, which gives it a
more object-oriented feel.

For instance, the statements:

\fvset{hllines={, ,}}%
\begin{sphinxVerbatim}[commandchars=\\\{\}]
\PYG{n}{model} \PYG{o}{=} \PYG{n+nb+bp}{self}\PYG{o}{.}\PYG{n}{pool}\PYG{o}{.}\PYG{n}{get}\PYG{p}{(}\PYG{n}{MODEL}\PYG{p}{)}
\PYG{n}{ids} \PYG{o}{=} \PYG{n}{model}\PYG{o}{.}\PYG{n}{search}\PYG{p}{(}\PYG{n}{cr}\PYG{p}{,} \PYG{n}{uid}\PYG{p}{,} \PYG{n}{DOMAIN}\PYG{p}{,} \PYG{n}{context}\PYG{o}{=}\PYG{n}{context}\PYG{p}{)}
\PYG{k}{for} \PYG{n}{rec} \PYG{o+ow}{in} \PYG{n}{model}\PYG{o}{.}\PYG{n}{browse}\PYG{p}{(}\PYG{n}{cr}\PYG{p}{,} \PYG{n}{uid}\PYG{p}{,} \PYG{n}{ids}\PYG{p}{,} \PYG{n}{context}\PYG{o}{=}\PYG{n}{context}\PYG{p}{)}\PYG{p}{:}
    \PYG{n+nb}{print} \PYG{n}{rec}\PYG{o}{.}\PYG{n}{name}
\PYG{n}{model}\PYG{o}{.}\PYG{n}{write}\PYG{p}{(}\PYG{n}{cr}\PYG{p}{,} \PYG{n}{uid}\PYG{p}{,} \PYG{n}{ids}\PYG{p}{,} \PYG{n}{VALUES}\PYG{p}{,} \PYG{n}{context}\PYG{o}{=}\PYG{n}{context}\PYG{p}{)}
\end{sphinxVerbatim}

may also be written as:

\fvset{hllines={, ,}}%
\begin{sphinxVerbatim}[commandchars=\\\{\}]
\PYG{n}{env} \PYG{o}{=} \PYG{n}{Environment}\PYG{p}{(}\PYG{n}{cr}\PYG{p}{,} \PYG{n}{uid}\PYG{p}{,} \PYG{n}{context}\PYG{p}{)} \PYG{c+c1}{\PYGZsh{} cr, uid, context wrapped in env}
\PYG{n}{model} \PYG{o}{=} \PYG{n}{env}\PYG{p}{[}\PYG{n}{MODEL}\PYG{p}{]}                  \PYG{c+c1}{\PYGZsh{} retrieve an instance of MODEL}
\PYG{n}{recs} \PYG{o}{=} \PYG{n}{model}\PYG{o}{.}\PYG{n}{search}\PYG{p}{(}\PYG{n}{DOMAIN}\PYG{p}{)}         \PYG{c+c1}{\PYGZsh{} search returns a recordset}
\PYG{k}{for} \PYG{n}{rec} \PYG{o+ow}{in} \PYG{n}{recs}\PYG{p}{:}                    \PYG{c+c1}{\PYGZsh{} iterate over the records}
    \PYG{n+nb}{print} \PYG{n}{rec}\PYG{o}{.}\PYG{n}{name}
\PYG{n}{recs}\PYG{o}{.}\PYG{n}{write}\PYG{p}{(}\PYG{n}{VALUES}\PYG{p}{)}                  \PYG{c+c1}{\PYGZsh{} update all records in recs}
\end{sphinxVerbatim}

Methods written in the “traditional” style are automatically decorated,
following some heuristics based on parameter names.
\index{multi() (in module odoo.api)}

\begin{fulllineitems}
\phantomsection\label{\detokenize{reference/orm:odoo.api.multi}}\pysiglinewithargsret{\sphinxcode{\sphinxupquote{odoo.api.}}\sphinxbfcode{\sphinxupquote{multi}}}{\emph{method}}{}
Decorate a record-style method where \sphinxcode{\sphinxupquote{self}} is a recordset. The method
typically defines an operation on records. Such a method:

\fvset{hllines={, ,}}%
\begin{sphinxVerbatim}[commandchars=\\\{\}]
\PYG{n+nd}{@api}\PYG{o}{.}\PYG{n}{multi}
\PYG{k}{def} \PYG{n+nf}{method}\PYG{p}{(}\PYG{n+nb+bp}{self}\PYG{p}{,} \PYG{n}{args}\PYG{p}{)}\PYG{p}{:}
    \PYG{o}{.}\PYG{o}{.}\PYG{o}{.}
\end{sphinxVerbatim}

may be called in both record and traditional styles, like:

\fvset{hllines={, ,}}%
\begin{sphinxVerbatim}[commandchars=\\\{\}]
\PYG{c+c1}{\PYGZsh{} recs = model.browse(cr, uid, ids, context)}
\PYG{n}{recs}\PYG{o}{.}\PYG{n}{method}\PYG{p}{(}\PYG{n}{args}\PYG{p}{)}

\PYG{n}{model}\PYG{o}{.}\PYG{n}{method}\PYG{p}{(}\PYG{n}{cr}\PYG{p}{,} \PYG{n}{uid}\PYG{p}{,} \PYG{n}{ids}\PYG{p}{,} \PYG{n}{args}\PYG{p}{,} \PYG{n}{context}\PYG{o}{=}\PYG{n}{context}\PYG{p}{)}
\end{sphinxVerbatim}

\end{fulllineitems}

\index{model() (in module odoo.api)}

\begin{fulllineitems}
\phantomsection\label{\detokenize{reference/orm:odoo.api.model}}\pysiglinewithargsret{\sphinxcode{\sphinxupquote{odoo.api.}}\sphinxbfcode{\sphinxupquote{model}}}{\emph{method}}{}
Decorate a record-style method where \sphinxcode{\sphinxupquote{self}} is a recordset, but its
contents is not relevant, only the model is. Such a method:

\fvset{hllines={, ,}}%
\begin{sphinxVerbatim}[commandchars=\\\{\}]
\PYG{n+nd}{@api}\PYG{o}{.}\PYG{n}{model}
\PYG{k}{def} \PYG{n+nf}{method}\PYG{p}{(}\PYG{n+nb+bp}{self}\PYG{p}{,} \PYG{n}{args}\PYG{p}{)}\PYG{p}{:}
    \PYG{o}{.}\PYG{o}{.}\PYG{o}{.}
\end{sphinxVerbatim}

may be called in both record and traditional styles, like:

\fvset{hllines={, ,}}%
\begin{sphinxVerbatim}[commandchars=\\\{\}]
\PYG{c+c1}{\PYGZsh{} recs = model.browse(cr, uid, ids, context)}
\PYG{n}{recs}\PYG{o}{.}\PYG{n}{method}\PYG{p}{(}\PYG{n}{args}\PYG{p}{)}

\PYG{n}{model}\PYG{o}{.}\PYG{n}{method}\PYG{p}{(}\PYG{n}{cr}\PYG{p}{,} \PYG{n}{uid}\PYG{p}{,} \PYG{n}{args}\PYG{p}{,} \PYG{n}{context}\PYG{o}{=}\PYG{n}{context}\PYG{p}{)}
\end{sphinxVerbatim}

Notice that no \sphinxcode{\sphinxupquote{ids}} are passed to the method in the traditional style.

\end{fulllineitems}

\index{depends() (in module odoo.api)}

\begin{fulllineitems}
\phantomsection\label{\detokenize{reference/orm:odoo.api.depends}}\pysiglinewithargsret{\sphinxcode{\sphinxupquote{odoo.api.}}\sphinxbfcode{\sphinxupquote{depends}}}{\emph{*args}}{}
Return a decorator that specifies the field dependencies of a “compute”
method (for new-style function fields). Each argument must be a string
that consists in a dot-separated sequence of field names:

\fvset{hllines={, ,}}%
\begin{sphinxVerbatim}[commandchars=\\\{\}]
\PYG{n}{pname} \PYG{o}{=} \PYG{n}{fields}\PYG{o}{.}\PYG{n}{Char}\PYG{p}{(}\PYG{n}{compute}\PYG{o}{=}\PYG{l+s+s1}{\PYGZsq{}}\PYG{l+s+s1}{\PYGZus{}compute\PYGZus{}pname}\PYG{l+s+s1}{\PYGZsq{}}\PYG{p}{)}

\PYG{n+nd}{@api}\PYG{o}{.}\PYG{n}{one}
\PYG{n+nd}{@api}\PYG{o}{.}\PYG{n}{depends}\PYG{p}{(}\PYG{l+s+s1}{\PYGZsq{}}\PYG{l+s+s1}{partner\PYGZus{}id.name}\PYG{l+s+s1}{\PYGZsq{}}\PYG{p}{,} \PYG{l+s+s1}{\PYGZsq{}}\PYG{l+s+s1}{partner\PYGZus{}id.is\PYGZus{}company}\PYG{l+s+s1}{\PYGZsq{}}\PYG{p}{)}
\PYG{k}{def} \PYG{n+nf}{\PYGZus{}compute\PYGZus{}pname}\PYG{p}{(}\PYG{n+nb+bp}{self}\PYG{p}{)}\PYG{p}{:}
    \PYG{k}{if} \PYG{n+nb+bp}{self}\PYG{o}{.}\PYG{n}{partner\PYGZus{}id}\PYG{o}{.}\PYG{n}{is\PYGZus{}company}\PYG{p}{:}
        \PYG{n+nb+bp}{self}\PYG{o}{.}\PYG{n}{pname} \PYG{o}{=} \PYG{p}{(}\PYG{n+nb+bp}{self}\PYG{o}{.}\PYG{n}{partner\PYGZus{}id}\PYG{o}{.}\PYG{n}{name} \PYG{o+ow}{or} \PYG{l+s+s2}{\PYGZdq{}}\PYG{l+s+s2}{\PYGZdq{}}\PYG{p}{)}\PYG{o}{.}\PYG{n}{upper}\PYG{p}{(}\PYG{p}{)}
    \PYG{k}{else}\PYG{p}{:}
        \PYG{n+nb+bp}{self}\PYG{o}{.}\PYG{n}{pname} \PYG{o}{=} \PYG{n+nb+bp}{self}\PYG{o}{.}\PYG{n}{partner\PYGZus{}id}\PYG{o}{.}\PYG{n}{name}
\end{sphinxVerbatim}

One may also pass a single function as argument. In that case, the
dependencies are given by calling the function with the field’s model.

\end{fulllineitems}

\index{constrains() (in module odoo.api)}

\begin{fulllineitems}
\phantomsection\label{\detokenize{reference/orm:odoo.api.constrains}}\pysiglinewithargsret{\sphinxcode{\sphinxupquote{odoo.api.}}\sphinxbfcode{\sphinxupquote{constrains}}}{\emph{*args}}{}
Decorates a constraint checker. Each argument must be a field name
used in the check:

\fvset{hllines={, ,}}%
\begin{sphinxVerbatim}[commandchars=\\\{\}]
\PYG{n+nd}{@api}\PYG{o}{.}\PYG{n}{one}
\PYG{n+nd}{@api}\PYG{o}{.}\PYG{n}{constrains}\PYG{p}{(}\PYG{l+s+s1}{\PYGZsq{}}\PYG{l+s+s1}{name}\PYG{l+s+s1}{\PYGZsq{}}\PYG{p}{,} \PYG{l+s+s1}{\PYGZsq{}}\PYG{l+s+s1}{description}\PYG{l+s+s1}{\PYGZsq{}}\PYG{p}{)}
\PYG{k}{def} \PYG{n+nf}{\PYGZus{}check\PYGZus{}description}\PYG{p}{(}\PYG{n+nb+bp}{self}\PYG{p}{)}\PYG{p}{:}
    \PYG{k}{if} \PYG{n+nb+bp}{self}\PYG{o}{.}\PYG{n}{name} \PYG{o}{==} \PYG{n+nb+bp}{self}\PYG{o}{.}\PYG{n}{description}\PYG{p}{:}
        \PYG{k}{raise} \PYG{n}{ValidationError}\PYG{p}{(}\PYG{l+s+s2}{\PYGZdq{}}\PYG{l+s+s2}{Fields name and description must be different}\PYG{l+s+s2}{\PYGZdq{}}\PYG{p}{)}
\end{sphinxVerbatim}

Invoked on the records on which one of the named fields has been modified.

Should raise \sphinxcode{\sphinxupquote{ValidationError}} if the
validation failed.

\begin{sphinxadmonition}{warning}{Warning:}
\sphinxcode{\sphinxupquote{@constrains}} only supports simple field names, dotted names
(fields of relational fields e.g. \sphinxcode{\sphinxupquote{partner\_id.customer}}) are not
supported and will be ignored

\sphinxcode{\sphinxupquote{@constrains}} will be triggered only if the declared fields in the
decorated method are included in the \sphinxcode{\sphinxupquote{create}} or \sphinxcode{\sphinxupquote{write}} call.
It implies that fields not present in a view will not trigger a call
during a record creation. A override of \sphinxcode{\sphinxupquote{create}} is necessary to make
sure a constraint will always be triggered (e.g. to test the absence of
value).
\end{sphinxadmonition}

\end{fulllineitems}

\index{onchange() (in module odoo.api)}

\begin{fulllineitems}
\phantomsection\label{\detokenize{reference/orm:odoo.api.onchange}}\pysiglinewithargsret{\sphinxcode{\sphinxupquote{odoo.api.}}\sphinxbfcode{\sphinxupquote{onchange}}}{\emph{*args}}{}
Return a decorator to decorate an onchange method for given fields.
Each argument must be a field name:

\fvset{hllines={, ,}}%
\begin{sphinxVerbatim}[commandchars=\\\{\}]
\PYG{n+nd}{@api}\PYG{o}{.}\PYG{n}{onchange}\PYG{p}{(}\PYG{l+s+s1}{\PYGZsq{}}\PYG{l+s+s1}{partner\PYGZus{}id}\PYG{l+s+s1}{\PYGZsq{}}\PYG{p}{)}
\PYG{k}{def} \PYG{n+nf}{\PYGZus{}onchange\PYGZus{}partner}\PYG{p}{(}\PYG{n+nb+bp}{self}\PYG{p}{)}\PYG{p}{:}
    \PYG{n+nb+bp}{self}\PYG{o}{.}\PYG{n}{message} \PYG{o}{=} \PYG{l+s+s2}{\PYGZdq{}}\PYG{l+s+s2}{Dear }\PYG{l+s+si}{\PYGZpc{}s}\PYG{l+s+s2}{\PYGZdq{}} \PYG{o}{\PYGZpc{}} \PYG{p}{(}\PYG{n+nb+bp}{self}\PYG{o}{.}\PYG{n}{partner\PYGZus{}id}\PYG{o}{.}\PYG{n}{name} \PYG{o+ow}{or} \PYG{l+s+s2}{\PYGZdq{}}\PYG{l+s+s2}{\PYGZdq{}}\PYG{p}{)}
\end{sphinxVerbatim}

In the form views where the field appears, the method will be called
when one of the given fields is modified. The method is invoked on a
pseudo-record that contains the values present in the form. Field
assignments on that record are automatically sent back to the client.

The method may return a dictionary for changing field domains and pop up
a warning message, like in the old API:

\fvset{hllines={, ,}}%
\begin{sphinxVerbatim}[commandchars=\\\{\}]
\PYG{k}{return} \PYG{p}{\PYGZob{}}
    \PYG{l+s+s1}{\PYGZsq{}}\PYG{l+s+s1}{domain}\PYG{l+s+s1}{\PYGZsq{}}\PYG{p}{:} \PYG{p}{\PYGZob{}}\PYG{l+s+s1}{\PYGZsq{}}\PYG{l+s+s1}{other\PYGZus{}id}\PYG{l+s+s1}{\PYGZsq{}}\PYG{p}{:} \PYG{p}{[}\PYG{p}{(}\PYG{l+s+s1}{\PYGZsq{}}\PYG{l+s+s1}{partner\PYGZus{}id}\PYG{l+s+s1}{\PYGZsq{}}\PYG{p}{,} \PYG{l+s+s1}{\PYGZsq{}}\PYG{l+s+s1}{=}\PYG{l+s+s1}{\PYGZsq{}}\PYG{p}{,} \PYG{n}{partner\PYGZus{}id}\PYG{p}{)}\PYG{p}{]}\PYG{p}{\PYGZcb{}}\PYG{p}{,}
    \PYG{l+s+s1}{\PYGZsq{}}\PYG{l+s+s1}{warning}\PYG{l+s+s1}{\PYGZsq{}}\PYG{p}{:} \PYG{p}{\PYGZob{}}\PYG{l+s+s1}{\PYGZsq{}}\PYG{l+s+s1}{title}\PYG{l+s+s1}{\PYGZsq{}}\PYG{p}{:} \PYG{l+s+s2}{\PYGZdq{}}\PYG{l+s+s2}{Warning}\PYG{l+s+s2}{\PYGZdq{}}\PYG{p}{,} \PYG{l+s+s1}{\PYGZsq{}}\PYG{l+s+s1}{message}\PYG{l+s+s1}{\PYGZsq{}}\PYG{p}{:} \PYG{l+s+s2}{\PYGZdq{}}\PYG{l+s+s2}{What is this?}\PYG{l+s+s2}{\PYGZdq{}}\PYG{p}{\PYGZcb{}}\PYG{p}{,}
\PYG{p}{\PYGZcb{}}
\end{sphinxVerbatim}

\begin{sphinxadmonition}{warning}{Warning:}
\sphinxcode{\sphinxupquote{@onchange}} only supports simple field names, dotted names
(fields of relational fields e.g. \sphinxcode{\sphinxupquote{partner\_id.tz}}) are not
supported and will be ignored
\end{sphinxadmonition}

\end{fulllineitems}

\index{returns() (in module odoo.api)}

\begin{fulllineitems}
\phantomsection\label{\detokenize{reference/orm:odoo.api.returns}}\pysiglinewithargsret{\sphinxcode{\sphinxupquote{odoo.api.}}\sphinxbfcode{\sphinxupquote{returns}}}{\emph{model}, \emph{downgrade=None}, \emph{upgrade=None}}{}
Return a decorator for methods that return instances of \sphinxcode{\sphinxupquote{model}}.
\begin{quote}\begin{description}
\item[{Parameters}] \leavevmode\begin{itemize}
\item {} 
\sphinxstyleliteralstrong{\sphinxupquote{model}} \textendash{} a model name, or \sphinxcode{\sphinxupquote{'self'}} for the current model

\item {} 
\sphinxstyleliteralstrong{\sphinxupquote{downgrade}} \textendash{} a function \sphinxcode{\sphinxupquote{downgrade(self, value, *args, **kwargs)}}
to convert the record-style \sphinxcode{\sphinxupquote{value}} to a traditional-style output

\item {} 
\sphinxstyleliteralstrong{\sphinxupquote{upgrade}} \textendash{} a function \sphinxcode{\sphinxupquote{upgrade(self, value, *args, **kwargs)}}
to convert the traditional-style \sphinxcode{\sphinxupquote{value}} to a record-style output

\end{itemize}

\end{description}\end{quote}

The arguments \sphinxcode{\sphinxupquote{self}}, \sphinxcode{\sphinxupquote{*args}} and \sphinxcode{\sphinxupquote{**kwargs}} are the ones passed
to the method in the record-style.

The decorator adapts the method output to the api style: \sphinxcode{\sphinxupquote{id}}, \sphinxcode{\sphinxupquote{ids}} or
\sphinxcode{\sphinxupquote{False}} for the traditional style, and recordset for the record style:

\fvset{hllines={, ,}}%
\begin{sphinxVerbatim}[commandchars=\\\{\}]
\PYG{n+nd}{@model}
\PYG{n+nd}{@returns}\PYG{p}{(}\PYG{l+s+s1}{\PYGZsq{}}\PYG{l+s+s1}{res.partner}\PYG{l+s+s1}{\PYGZsq{}}\PYG{p}{)}
\PYG{k}{def} \PYG{n+nf}{find\PYGZus{}partner}\PYG{p}{(}\PYG{n+nb+bp}{self}\PYG{p}{,} \PYG{n}{arg}\PYG{p}{)}\PYG{p}{:}
    \PYG{o}{.}\PYG{o}{.}\PYG{o}{.}     \PYG{c+c1}{\PYGZsh{} return some record}

\PYG{c+c1}{\PYGZsh{} output depends on call style: traditional vs record style}
\PYG{n}{partner\PYGZus{}id} \PYG{o}{=} \PYG{n}{model}\PYG{o}{.}\PYG{n}{find\PYGZus{}partner}\PYG{p}{(}\PYG{n}{cr}\PYG{p}{,} \PYG{n}{uid}\PYG{p}{,} \PYG{n}{arg}\PYG{p}{,} \PYG{n}{context}\PYG{o}{=}\PYG{n}{context}\PYG{p}{)}

\PYG{c+c1}{\PYGZsh{} recs = model.browse(cr, uid, ids, context)}
\PYG{n}{partner\PYGZus{}record} \PYG{o}{=} \PYG{n}{recs}\PYG{o}{.}\PYG{n}{find\PYGZus{}partner}\PYG{p}{(}\PYG{n}{arg}\PYG{p}{)}
\end{sphinxVerbatim}

Note that the decorated method must satisfy that convention.

Those decorators are automatically \sphinxstyleemphasis{inherited}: a method that overrides
a decorated existing method will be decorated with the same
\sphinxcode{\sphinxupquote{@returns(model)}}.

\end{fulllineitems}

\index{one() (in module odoo.api)}

\begin{fulllineitems}
\phantomsection\label{\detokenize{reference/orm:odoo.api.one}}\pysiglinewithargsret{\sphinxcode{\sphinxupquote{odoo.api.}}\sphinxbfcode{\sphinxupquote{one}}}{\emph{method}}{}
Decorate a record-style method where \sphinxcode{\sphinxupquote{self}} is expected to be a
singleton instance. The decorated method automatically loops on records,
and makes a list with the results. In case the method is decorated with
{\hyperref[\detokenize{reference/orm:odoo.api.returns}]{\sphinxcrossref{\sphinxcode{\sphinxupquote{returns()}}}}}, it concatenates the resulting instances. Such a
method:

\fvset{hllines={, ,}}%
\begin{sphinxVerbatim}[commandchars=\\\{\}]
\PYG{n+nd}{@api}\PYG{o}{.}\PYG{n}{one}
\PYG{k}{def} \PYG{n+nf}{method}\PYG{p}{(}\PYG{n+nb+bp}{self}\PYG{p}{,} \PYG{n}{args}\PYG{p}{)}\PYG{p}{:}
    \PYG{k}{return} \PYG{n+nb+bp}{self}\PYG{o}{.}\PYG{n}{name}
\end{sphinxVerbatim}

may be called in both record and traditional styles, like:

\fvset{hllines={, ,}}%
\begin{sphinxVerbatim}[commandchars=\\\{\}]
\PYG{c+c1}{\PYGZsh{} recs = model.browse(cr, uid, ids, context)}
\PYG{n}{names} \PYG{o}{=} \PYG{n}{recs}\PYG{o}{.}\PYG{n}{method}\PYG{p}{(}\PYG{n}{args}\PYG{p}{)}

\PYG{n}{names} \PYG{o}{=} \PYG{n}{model}\PYG{o}{.}\PYG{n}{method}\PYG{p}{(}\PYG{n}{cr}\PYG{p}{,} \PYG{n}{uid}\PYG{p}{,} \PYG{n}{ids}\PYG{p}{,} \PYG{n}{args}\PYG{p}{,} \PYG{n}{context}\PYG{o}{=}\PYG{n}{context}\PYG{p}{)}
\end{sphinxVerbatim}

\DUrole{versionmodified}{Deprecated since version 9.0: }{\hyperref[\detokenize{reference/orm:odoo.api.one}]{\sphinxcrossref{\sphinxcode{\sphinxupquote{one()}}}}} often makes the code less clear and behaves in ways
developers and readers may not expect.

It is strongly recommended to use {\hyperref[\detokenize{reference/orm:odoo.api.multi}]{\sphinxcrossref{\sphinxcode{\sphinxupquote{multi()}}}}} and either
iterate on the \sphinxcode{\sphinxupquote{self}} recordset or ensure that the recordset
is a single record with {\hyperref[\detokenize{reference/orm:odoo.models.Model.ensure_one}]{\sphinxcrossref{\sphinxcode{\sphinxupquote{ensure\_one()}}}}}.

\end{fulllineitems}

\index{v7() (in module odoo.api)}

\begin{fulllineitems}
\phantomsection\label{\detokenize{reference/orm:odoo.api.v7}}\pysiglinewithargsret{\sphinxcode{\sphinxupquote{odoo.api.}}\sphinxbfcode{\sphinxupquote{v7}}}{\emph{method\_v7}}{}
Decorate a method that supports the old-style api only. A new-style api
may be provided by redefining a method with the same name and decorated
with {\hyperref[\detokenize{reference/orm:odoo.api.v8}]{\sphinxcrossref{\sphinxcode{\sphinxupquote{v8()}}}}}:

\fvset{hllines={, ,}}%
\begin{sphinxVerbatim}[commandchars=\\\{\}]
\PYG{n+nd}{@api}\PYG{o}{.}\PYG{n}{v7}
\PYG{k}{def} \PYG{n+nf}{foo}\PYG{p}{(}\PYG{n+nb+bp}{self}\PYG{p}{,} \PYG{n}{cr}\PYG{p}{,} \PYG{n}{uid}\PYG{p}{,} \PYG{n}{ids}\PYG{p}{,} \PYG{n}{context}\PYG{o}{=}\PYG{k+kc}{None}\PYG{p}{)}\PYG{p}{:}
    \PYG{o}{.}\PYG{o}{.}\PYG{o}{.}

\PYG{n+nd}{@api}\PYG{o}{.}\PYG{n}{v8}
\PYG{k}{def} \PYG{n+nf}{foo}\PYG{p}{(}\PYG{n+nb+bp}{self}\PYG{p}{)}\PYG{p}{:}
    \PYG{o}{.}\PYG{o}{.}\PYG{o}{.}
\end{sphinxVerbatim}

Special care must be taken if one method calls the other one, because
the method may be overridden! In that case, one should call the method
from the current class (say \sphinxcode{\sphinxupquote{MyClass}}), for instance:

\fvset{hllines={, ,}}%
\begin{sphinxVerbatim}[commandchars=\\\{\}]
\PYG{n+nd}{@api}\PYG{o}{.}\PYG{n}{v7}
\PYG{k}{def} \PYG{n+nf}{foo}\PYG{p}{(}\PYG{n+nb+bp}{self}\PYG{p}{,} \PYG{n}{cr}\PYG{p}{,} \PYG{n}{uid}\PYG{p}{,} \PYG{n}{ids}\PYG{p}{,} \PYG{n}{context}\PYG{o}{=}\PYG{k+kc}{None}\PYG{p}{)}\PYG{p}{:}
    \PYG{c+c1}{\PYGZsh{} Beware: records.foo() may call an overriding of foo()}
    \PYG{n}{records} \PYG{o}{=} \PYG{n+nb+bp}{self}\PYG{o}{.}\PYG{n}{browse}\PYG{p}{(}\PYG{n}{cr}\PYG{p}{,} \PYG{n}{uid}\PYG{p}{,} \PYG{n}{ids}\PYG{p}{,} \PYG{n}{context}\PYG{p}{)}
    \PYG{k}{return} \PYG{n}{MyClass}\PYG{o}{.}\PYG{n}{foo}\PYG{p}{(}\PYG{n}{records}\PYG{p}{)}
\end{sphinxVerbatim}

Note that the wrapper method uses the docstring of the first method.

\end{fulllineitems}

\index{v8() (in module odoo.api)}

\begin{fulllineitems}
\phantomsection\label{\detokenize{reference/orm:odoo.api.v8}}\pysiglinewithargsret{\sphinxcode{\sphinxupquote{odoo.api.}}\sphinxbfcode{\sphinxupquote{v8}}}{\emph{method\_v8}}{}
Decorate a method that supports the new-style api only. An old-style api
may be provided by redefining a method with the same name and decorated
with {\hyperref[\detokenize{reference/orm:odoo.api.v7}]{\sphinxcrossref{\sphinxcode{\sphinxupquote{v7()}}}}}:

\fvset{hllines={, ,}}%
\begin{sphinxVerbatim}[commandchars=\\\{\}]
\PYG{n+nd}{@api}\PYG{o}{.}\PYG{n}{v8}
\PYG{k}{def} \PYG{n+nf}{foo}\PYG{p}{(}\PYG{n+nb+bp}{self}\PYG{p}{)}\PYG{p}{:}
    \PYG{o}{.}\PYG{o}{.}\PYG{o}{.}

\PYG{n+nd}{@api}\PYG{o}{.}\PYG{n}{v7}
\PYG{k}{def} \PYG{n+nf}{foo}\PYG{p}{(}\PYG{n+nb+bp}{self}\PYG{p}{,} \PYG{n}{cr}\PYG{p}{,} \PYG{n}{uid}\PYG{p}{,} \PYG{n}{ids}\PYG{p}{,} \PYG{n}{context}\PYG{o}{=}\PYG{k+kc}{None}\PYG{p}{)}\PYG{p}{:}
    \PYG{o}{.}\PYG{o}{.}\PYG{o}{.}
\end{sphinxVerbatim}

Note that the wrapper method uses the docstring of the first method.

\end{fulllineitems}



\subsection{Fields}
\label{\detokenize{reference/orm:fields}}\label{\detokenize{reference/orm:reference-orm-fields}}

\subsubsection{Basic fields}
\label{\detokenize{reference/orm:reference-orm-fields-basic}}\label{\detokenize{reference/orm:basic-fields}}\index{Field (class in odoo.fields)}

\begin{fulllineitems}
\phantomsection\label{\detokenize{reference/orm:odoo.fields.Field}}\pysiglinewithargsret{\sphinxbfcode{\sphinxupquote{class }}\sphinxcode{\sphinxupquote{odoo.fields.}}\sphinxbfcode{\sphinxupquote{Field}}}{\emph{string=\textless{}object object\textgreater{}}, \emph{**kwargs}}{}
The field descriptor contains the field definition, and manages accesses
and assignments of the corresponding field on records. The following
attributes may be provided when instanciating a field:
\begin{quote}\begin{description}
\item[{Parameters}] \leavevmode\begin{itemize}
\item {} 
\sphinxstyleliteralstrong{\sphinxupquote{string}} \textendash{} the label of the field seen by users (string); if not
set, the ORM takes the field name in the class (capitalized).

\item {} 
\sphinxstyleliteralstrong{\sphinxupquote{help}} \textendash{} the tooltip of the field seen by users (string)

\item {} 
\sphinxstyleliteralstrong{\sphinxupquote{readonly}} \textendash{} whether the field is readonly (boolean, by default \sphinxcode{\sphinxupquote{False}})

\item {} 
\sphinxstyleliteralstrong{\sphinxupquote{required}} \textendash{} whether the value of the field is required (boolean, by
default \sphinxcode{\sphinxupquote{False}})

\item {} 
\sphinxstyleliteralstrong{\sphinxupquote{index}} \textendash{} whether the field is indexed in database (boolean, by
default \sphinxcode{\sphinxupquote{False}})

\item {} 
\sphinxstyleliteralstrong{\sphinxupquote{default}} \textendash{} the default value for the field; this is either a static
value, or a function taking a recordset and returning a value; use
\sphinxcode{\sphinxupquote{default=None}} to discard default values for the field

\item {} 
\sphinxstyleliteralstrong{\sphinxupquote{states}} \textendash{} a dictionary mapping state values to lists of UI attribute-value
pairs; possible attributes are: ‘readonly’, ‘required’, ‘invisible’.
Note: Any state-based condition requires the \sphinxcode{\sphinxupquote{state}} field value to be
available on the client-side UI. This is typically done by including it in
the relevant views, possibly made invisible if not relevant for the
end-user.

\item {} 
\sphinxstyleliteralstrong{\sphinxupquote{groups}} \textendash{} comma-separated list of group xml ids (string); this
restricts the field access to the users of the given groups only

\item {} 
\sphinxstyleliteralstrong{\sphinxupquote{copy}} (\sphinxhref{https://docs.python.org/3/library/functions.html\#bool}{\sphinxstyleliteralemphasis{\sphinxupquote{bool}}}) \textendash{} whether the field value should be copied when the record
is duplicated (default: \sphinxcode{\sphinxupquote{True}} for normal fields, \sphinxcode{\sphinxupquote{False}} for
\sphinxcode{\sphinxupquote{one2many}} and computed fields, including property fields and
related fields)

\item {} 
\sphinxstyleliteralstrong{\sphinxupquote{oldname}} (\sphinxstyleliteralemphasis{\sphinxupquote{string}}) \textendash{} the previous name of this field, so that ORM can rename
it automatically at migration

\end{itemize}

\end{description}\end{quote}
\phantomsection\label{\detokenize{reference/orm:field-computed}}\paragraph{Computed fields}

One can define a field whose value is computed instead of simply being
read from the database. The attributes that are specific to computed
fields are given below. To define such a field, simply provide a value
for the attribute \sphinxcode{\sphinxupquote{compute}}.
\begin{quote}\begin{description}
\item[{Parameters}] \leavevmode\begin{itemize}
\item {} 
\sphinxstyleliteralstrong{\sphinxupquote{compute}} \textendash{} name of a method that computes the field

\item {} 
\sphinxstyleliteralstrong{\sphinxupquote{inverse}} \textendash{} name of a method that inverses the field (optional)

\item {} 
\sphinxstyleliteralstrong{\sphinxupquote{search}} \textendash{} name of a method that implement search on the field (optional)

\item {} 
\sphinxstyleliteralstrong{\sphinxupquote{store}} \textendash{} whether the field is stored in database (boolean, by
default \sphinxcode{\sphinxupquote{False}} on computed fields)

\item {} 
\sphinxstyleliteralstrong{\sphinxupquote{compute\_sudo}} \textendash{} whether the field should be recomputed as superuser
to bypass access rights (boolean, by default \sphinxcode{\sphinxupquote{False}})

\end{itemize}

\end{description}\end{quote}

The methods given for \sphinxcode{\sphinxupquote{compute}}, \sphinxcode{\sphinxupquote{inverse}} and \sphinxcode{\sphinxupquote{search}} are model
methods. Their signature is shown in the following example:

\fvset{hllines={, ,}}%
\begin{sphinxVerbatim}[commandchars=\\\{\}]
\PYG{n}{upper} \PYG{o}{=} \PYG{n}{fields}\PYG{o}{.}\PYG{n}{Char}\PYG{p}{(}\PYG{n}{compute}\PYG{o}{=}\PYG{l+s+s1}{\PYGZsq{}}\PYG{l+s+s1}{\PYGZus{}compute\PYGZus{}upper}\PYG{l+s+s1}{\PYGZsq{}}\PYG{p}{,}
                    \PYG{n}{inverse}\PYG{o}{=}\PYG{l+s+s1}{\PYGZsq{}}\PYG{l+s+s1}{\PYGZus{}inverse\PYGZus{}upper}\PYG{l+s+s1}{\PYGZsq{}}\PYG{p}{,}
                    \PYG{n}{search}\PYG{o}{=}\PYG{l+s+s1}{\PYGZsq{}}\PYG{l+s+s1}{\PYGZus{}search\PYGZus{}upper}\PYG{l+s+s1}{\PYGZsq{}}\PYG{p}{)}

\PYG{n+nd}{@api}\PYG{o}{.}\PYG{n}{depends}\PYG{p}{(}\PYG{l+s+s1}{\PYGZsq{}}\PYG{l+s+s1}{name}\PYG{l+s+s1}{\PYGZsq{}}\PYG{p}{)}
\PYG{k}{def} \PYG{n+nf}{\PYGZus{}compute\PYGZus{}upper}\PYG{p}{(}\PYG{n+nb+bp}{self}\PYG{p}{)}\PYG{p}{:}
    \PYG{k}{for} \PYG{n}{rec} \PYG{o+ow}{in} \PYG{n+nb+bp}{self}\PYG{p}{:}
        \PYG{n}{rec}\PYG{o}{.}\PYG{n}{upper} \PYG{o}{=} \PYG{n}{rec}\PYG{o}{.}\PYG{n}{name}\PYG{o}{.}\PYG{n}{upper}\PYG{p}{(}\PYG{p}{)} \PYG{k}{if} \PYG{n}{rec}\PYG{o}{.}\PYG{n}{name} \PYG{k}{else} \PYG{k+kc}{False}

\PYG{k}{def} \PYG{n+nf}{\PYGZus{}inverse\PYGZus{}upper}\PYG{p}{(}\PYG{n+nb+bp}{self}\PYG{p}{)}\PYG{p}{:}
    \PYG{k}{for} \PYG{n}{rec} \PYG{o+ow}{in} \PYG{n+nb+bp}{self}\PYG{p}{:}
        \PYG{n}{rec}\PYG{o}{.}\PYG{n}{name} \PYG{o}{=} \PYG{n}{rec}\PYG{o}{.}\PYG{n}{upper}\PYG{o}{.}\PYG{n}{lower}\PYG{p}{(}\PYG{p}{)} \PYG{k}{if} \PYG{n}{rec}\PYG{o}{.}\PYG{n}{upper} \PYG{k}{else} \PYG{k+kc}{False}

\PYG{k}{def} \PYG{n+nf}{\PYGZus{}search\PYGZus{}upper}\PYG{p}{(}\PYG{n+nb+bp}{self}\PYG{p}{,} \PYG{n}{operator}\PYG{p}{,} \PYG{n}{value}\PYG{p}{)}\PYG{p}{:}
    \PYG{k}{if} \PYG{n}{operator} \PYG{o}{==} \PYG{l+s+s1}{\PYGZsq{}}\PYG{l+s+s1}{like}\PYG{l+s+s1}{\PYGZsq{}}\PYG{p}{:}
        \PYG{n}{operator} \PYG{o}{=} \PYG{l+s+s1}{\PYGZsq{}}\PYG{l+s+s1}{ilike}\PYG{l+s+s1}{\PYGZsq{}}
    \PYG{k}{return} \PYG{p}{[}\PYG{p}{(}\PYG{l+s+s1}{\PYGZsq{}}\PYG{l+s+s1}{name}\PYG{l+s+s1}{\PYGZsq{}}\PYG{p}{,} \PYG{n}{operator}\PYG{p}{,} \PYG{n}{value}\PYG{p}{)}\PYG{p}{]}
\end{sphinxVerbatim}

The compute method has to assign the field on all records of the invoked
recordset. The decorator {\hyperref[\detokenize{reference/orm:odoo.api.depends}]{\sphinxcrossref{\sphinxcode{\sphinxupquote{odoo.api.depends()}}}}} must be applied on
the compute method to specify the field dependencies; those dependencies
are used to determine when to recompute the field; recomputation is
automatic and guarantees cache/database consistency. Note that the same
method can be used for several fields, you simply have to assign all the
given fields in the method; the method will be invoked once for all
those fields.

By default, a computed field is not stored to the database, and is
computed on-the-fly. Adding the attribute \sphinxcode{\sphinxupquote{store=True}} will store the
field’s values in the database. The advantage of a stored field is that
searching on that field is done by the database itself. The disadvantage
is that it requires database updates when the field must be recomputed.

The inverse method, as its name says, does the inverse of the compute
method: the invoked records have a value for the field, and you must
apply the necessary changes on the field dependencies such that the
computation gives the expected value. Note that a computed field without
an inverse method is readonly by default.

The search method is invoked when processing domains before doing an
actual search on the model. It must return a domain equivalent to the
condition: \sphinxcode{\sphinxupquote{field operator value}}.

\phantomsection\label{\detokenize{reference/orm:field-related}}\paragraph{Related fields}

The value of a related field is given by following a sequence of
relational fields and reading a field on the reached model. The complete
sequence of fields to traverse is specified by the attribute
\begin{quote}\begin{description}
\item[{Parameters}] \leavevmode
\sphinxstyleliteralstrong{\sphinxupquote{related}} \textendash{} sequence of field names

\end{description}\end{quote}

Some field attributes are automatically copied from the source field if
they are not redefined: \sphinxcode{\sphinxupquote{string}}, \sphinxcode{\sphinxupquote{help}}, \sphinxcode{\sphinxupquote{readonly}}, \sphinxcode{\sphinxupquote{required}} (only
if all fields in the sequence are required), \sphinxcode{\sphinxupquote{groups}}, \sphinxcode{\sphinxupquote{digits}}, \sphinxcode{\sphinxupquote{size}},
\sphinxcode{\sphinxupquote{translate}}, \sphinxcode{\sphinxupquote{sanitize}}, \sphinxcode{\sphinxupquote{selection}}, \sphinxcode{\sphinxupquote{comodel\_name}}, \sphinxcode{\sphinxupquote{domain}},
\sphinxcode{\sphinxupquote{context}}. All semantic-free attributes are copied from the source
field.

By default, the values of related fields are not stored to the database.
Add the attribute \sphinxcode{\sphinxupquote{store=True}} to make it stored, just like computed
fields. Related fields are automatically recomputed when their
dependencies are modified.

\phantomsection\label{\detokenize{reference/orm:field-company-dependent}}\paragraph{Company-dependent fields}

Formerly known as ‘property’ fields, the value of those fields depends
on the company. In other words, users that belong to different companies
may see different values for the field on a given record.
\begin{quote}\begin{description}
\item[{Parameters}] \leavevmode
\sphinxstyleliteralstrong{\sphinxupquote{company\_dependent}} \textendash{} whether the field is company-dependent (boolean)

\end{description}\end{quote}
\phantomsection\label{\detokenize{reference/orm:field-incremental-definition}}\paragraph{Incremental definition}

A field is defined as class attribute on a model class. If the model
is extended (see {\hyperref[\detokenize{reference/orm:odoo.models.Model}]{\sphinxcrossref{\sphinxcode{\sphinxupquote{Model}}}}}), one can also extend
the field definition by redefining a field with the same name and same
type on the subclass. In that case, the attributes of the field are
taken from the parent class and overridden by the ones given in
subclasses.

For instance, the second class below only adds a tooltip on the field
\sphinxcode{\sphinxupquote{state}}:

\fvset{hllines={, ,}}%
\begin{sphinxVerbatim}[commandchars=\\\{\}]
\PYG{k}{class} \PYG{n+nc}{First}\PYG{p}{(}\PYG{n}{models}\PYG{o}{.}\PYG{n}{Model}\PYG{p}{)}\PYG{p}{:}
    \PYG{n}{\PYGZus{}name} \PYG{o}{=} \PYG{l+s+s1}{\PYGZsq{}}\PYG{l+s+s1}{foo}\PYG{l+s+s1}{\PYGZsq{}}
    \PYG{n}{state} \PYG{o}{=} \PYG{n}{fields}\PYG{o}{.}\PYG{n}{Selection}\PYG{p}{(}\PYG{p}{[}\PYG{o}{.}\PYG{o}{.}\PYG{o}{.}\PYG{p}{]}\PYG{p}{,} \PYG{n}{required}\PYG{o}{=}\PYG{k+kc}{True}\PYG{p}{)}

\PYG{k}{class} \PYG{n+nc}{Second}\PYG{p}{(}\PYG{n}{models}\PYG{o}{.}\PYG{n}{Model}\PYG{p}{)}\PYG{p}{:}
    \PYG{n}{\PYGZus{}inherit} \PYG{o}{=} \PYG{l+s+s1}{\PYGZsq{}}\PYG{l+s+s1}{foo}\PYG{l+s+s1}{\PYGZsq{}}
    \PYG{n}{state} \PYG{o}{=} \PYG{n}{fields}\PYG{o}{.}\PYG{n}{Selection}\PYG{p}{(}\PYG{n}{help}\PYG{o}{=}\PYG{l+s+s2}{\PYGZdq{}}\PYG{l+s+s2}{Blah blah blah}\PYG{l+s+s2}{\PYGZdq{}}\PYG{p}{)}
\end{sphinxVerbatim}

\end{fulllineitems}

\index{Char (class in odoo.fields)}

\begin{fulllineitems}
\phantomsection\label{\detokenize{reference/orm:odoo.fields.Char}}\pysiglinewithargsret{\sphinxbfcode{\sphinxupquote{class }}\sphinxcode{\sphinxupquote{odoo.fields.}}\sphinxbfcode{\sphinxupquote{Char}}}{\emph{string=\textless{}object object\textgreater{}}, \emph{**kwargs}}{}
Bases: \sphinxcode{\sphinxupquote{odoo.fields.\_String}}

Basic string field, can be length-limited, usually displayed as a
single-line string in clients.
\begin{quote}\begin{description}
\item[{Parameters}] \leavevmode\begin{itemize}
\item {} 
\sphinxstyleliteralstrong{\sphinxupquote{size}} (\sphinxhref{https://docs.python.org/3/library/functions.html\#int}{\sphinxstyleliteralemphasis{\sphinxupquote{int}}}) \textendash{} the maximum size of values stored for that field

\item {} 
\sphinxstyleliteralstrong{\sphinxupquote{translate}} \textendash{} enable the translation of the field’s values; use
\sphinxcode{\sphinxupquote{translate=True}} to translate field values as a whole; \sphinxcode{\sphinxupquote{translate}}
may also be a callable such that \sphinxcode{\sphinxupquote{translate(callback, value)}}
translates \sphinxcode{\sphinxupquote{value}} by using \sphinxcode{\sphinxupquote{callback(term)}} to retrieve the
translation of terms.

\end{itemize}

\end{description}\end{quote}

\end{fulllineitems}

\index{Boolean (class in odoo.fields)}

\begin{fulllineitems}
\phantomsection\label{\detokenize{reference/orm:odoo.fields.Boolean}}\pysiglinewithargsret{\sphinxbfcode{\sphinxupquote{class }}\sphinxcode{\sphinxupquote{odoo.fields.}}\sphinxbfcode{\sphinxupquote{Boolean}}}{\emph{string=\textless{}object object\textgreater{}}, \emph{**kwargs}}{}
Bases: {\hyperref[\detokenize{reference/orm:odoo.fields.Field}]{\sphinxcrossref{\sphinxcode{\sphinxupquote{odoo.fields.Field}}}}}

\end{fulllineitems}

\index{Integer (class in odoo.fields)}

\begin{fulllineitems}
\phantomsection\label{\detokenize{reference/orm:odoo.fields.Integer}}\pysiglinewithargsret{\sphinxbfcode{\sphinxupquote{class }}\sphinxcode{\sphinxupquote{odoo.fields.}}\sphinxbfcode{\sphinxupquote{Integer}}}{\emph{string=\textless{}object object\textgreater{}}, \emph{**kwargs}}{}
Bases: {\hyperref[\detokenize{reference/orm:odoo.fields.Field}]{\sphinxcrossref{\sphinxcode{\sphinxupquote{odoo.fields.Field}}}}}

\end{fulllineitems}

\index{Float (class in odoo.fields)}

\begin{fulllineitems}
\phantomsection\label{\detokenize{reference/orm:odoo.fields.Float}}\pysiglinewithargsret{\sphinxbfcode{\sphinxupquote{class }}\sphinxcode{\sphinxupquote{odoo.fields.}}\sphinxbfcode{\sphinxupquote{Float}}}{\emph{string=\textless{}object object\textgreater{}}, \emph{digits=\textless{}object object\textgreater{}}, \emph{**kwargs}}{}
Bases: {\hyperref[\detokenize{reference/orm:odoo.fields.Field}]{\sphinxcrossref{\sphinxcode{\sphinxupquote{odoo.fields.Field}}}}}

The precision digits are given by the attribute
\begin{quote}\begin{description}
\item[{Parameters}] \leavevmode
\sphinxstyleliteralstrong{\sphinxupquote{digits}} \textendash{} a pair (total, decimal), or a function taking a database
cursor and returning a pair (total, decimal)

\end{description}\end{quote}

\end{fulllineitems}

\index{Text (class in odoo.fields)}

\begin{fulllineitems}
\phantomsection\label{\detokenize{reference/orm:odoo.fields.Text}}\pysiglinewithargsret{\sphinxbfcode{\sphinxupquote{class }}\sphinxcode{\sphinxupquote{odoo.fields.}}\sphinxbfcode{\sphinxupquote{Text}}}{\emph{string=\textless{}object object\textgreater{}}, \emph{**kwargs}}{}
Bases: \sphinxcode{\sphinxupquote{odoo.fields.\_String}}

Very similar to {\hyperref[\detokenize{reference/orm:odoo.fields.Char}]{\sphinxcrossref{\sphinxcode{\sphinxupquote{Char}}}}} but used for longer contents, does not
have a size and usually displayed as a multiline text box.
\begin{quote}\begin{description}
\item[{Parameters}] \leavevmode
\sphinxstyleliteralstrong{\sphinxupquote{translate}} \textendash{} enable the translation of the field’s values; use
\sphinxcode{\sphinxupquote{translate=True}} to translate field values as a whole; \sphinxcode{\sphinxupquote{translate}}
may also be a callable such that \sphinxcode{\sphinxupquote{translate(callback, value)}}
translates \sphinxcode{\sphinxupquote{value}} by using \sphinxcode{\sphinxupquote{callback(term)}} to retrieve the
translation of terms.

\end{description}\end{quote}

\end{fulllineitems}

\index{Selection (class in odoo.fields)}

\begin{fulllineitems}
\phantomsection\label{\detokenize{reference/orm:odoo.fields.Selection}}\pysiglinewithargsret{\sphinxbfcode{\sphinxupquote{class }}\sphinxcode{\sphinxupquote{odoo.fields.}}\sphinxbfcode{\sphinxupquote{Selection}}}{\emph{selection=\textless{}object object\textgreater{}}, \emph{string=\textless{}object object\textgreater{}}, \emph{**kwargs}}{}
Bases: {\hyperref[\detokenize{reference/orm:odoo.fields.Field}]{\sphinxcrossref{\sphinxcode{\sphinxupquote{odoo.fields.Field}}}}}
\begin{quote}\begin{description}
\item[{Parameters}] \leavevmode\begin{itemize}
\item {} 
\sphinxstyleliteralstrong{\sphinxupquote{selection}} \textendash{} specifies the possible values for this field.
It is given as either a list of pairs (\sphinxcode{\sphinxupquote{value}}, \sphinxcode{\sphinxupquote{string}}), or a
model method, or a method name.

\item {} 
\sphinxstyleliteralstrong{\sphinxupquote{selection\_add}} \textendash{} provides an extension of the selection in the case
of an overridden field. It is a list of pairs (\sphinxcode{\sphinxupquote{value}}, \sphinxcode{\sphinxupquote{string}}).

\end{itemize}

\end{description}\end{quote}

The attribute \sphinxcode{\sphinxupquote{selection}} is mandatory except in the case of
{\hyperref[\detokenize{reference/orm:field-related}]{\sphinxcrossref{\DUrole{std,std-ref}{related fields}}}} or {\hyperref[\detokenize{reference/orm:field-incremental-definition}]{\sphinxcrossref{\DUrole{std,std-ref}{field extensions}}}}.

\end{fulllineitems}

\index{Html (class in odoo.fields)}

\begin{fulllineitems}
\phantomsection\label{\detokenize{reference/orm:odoo.fields.Html}}\pysiglinewithargsret{\sphinxbfcode{\sphinxupquote{class }}\sphinxcode{\sphinxupquote{odoo.fields.}}\sphinxbfcode{\sphinxupquote{Html}}}{\emph{string=\textless{}object object\textgreater{}}, \emph{**kwargs}}{}
Bases: \sphinxcode{\sphinxupquote{odoo.fields.\_String}}

\end{fulllineitems}

\index{Date (class in odoo.fields)}

\begin{fulllineitems}
\phantomsection\label{\detokenize{reference/orm:odoo.fields.Date}}\pysiglinewithargsret{\sphinxbfcode{\sphinxupquote{class }}\sphinxcode{\sphinxupquote{odoo.fields.}}\sphinxbfcode{\sphinxupquote{Date}}}{\emph{string=\textless{}object object\textgreater{}}, \emph{**kwargs}}{}
Bases: {\hyperref[\detokenize{reference/orm:odoo.fields.Field}]{\sphinxcrossref{\sphinxcode{\sphinxupquote{odoo.fields.Field}}}}}
\index{context\_today() (odoo.fields.Date static method)}

\begin{fulllineitems}
\phantomsection\label{\detokenize{reference/orm:odoo.fields.Date.context_today}}\pysiglinewithargsret{\sphinxbfcode{\sphinxupquote{static }}\sphinxbfcode{\sphinxupquote{context\_today}}}{\emph{record}, \emph{timestamp=None}}{}
Return the current date as seen in the client’s timezone in a format
fit for date fields. This method may be used to compute default
values.
\begin{quote}\begin{description}
\item[{Parameters}] \leavevmode
\sphinxstyleliteralstrong{\sphinxupquote{timestamp}} (\sphinxstyleliteralemphasis{\sphinxupquote{datetime}}) \textendash{} optional datetime value to use instead of
the current date and time (must be a datetime, regular dates
can’t be converted between timezones.)

\item[{Return type}] \leavevmode
\sphinxhref{https://docs.python.org/3/library/stdtypes.html\#str}{str}

\end{description}\end{quote}

\end{fulllineitems}

\index{from\_string() (odoo.fields.Date static method)}

\begin{fulllineitems}
\phantomsection\label{\detokenize{reference/orm:odoo.fields.Date.from_string}}\pysiglinewithargsret{\sphinxbfcode{\sphinxupquote{static }}\sphinxbfcode{\sphinxupquote{from\_string}}}{\emph{value}}{}
Convert an ORM \sphinxcode{\sphinxupquote{value}} into a \sphinxcode{\sphinxupquote{date}} value.

\end{fulllineitems}

\index{to\_string() (odoo.fields.Date static method)}

\begin{fulllineitems}
\phantomsection\label{\detokenize{reference/orm:odoo.fields.Date.to_string}}\pysiglinewithargsret{\sphinxbfcode{\sphinxupquote{static }}\sphinxbfcode{\sphinxupquote{to\_string}}}{\emph{value}}{}
Convert a \sphinxcode{\sphinxupquote{date}} value into the format expected by the ORM.

\end{fulllineitems}

\index{today() (odoo.fields.Date static method)}

\begin{fulllineitems}
\phantomsection\label{\detokenize{reference/orm:odoo.fields.Date.today}}\pysiglinewithargsret{\sphinxbfcode{\sphinxupquote{static }}\sphinxbfcode{\sphinxupquote{today}}}{\emph{*args}}{}
Return the current day in the format expected by the ORM.
This function may be used to compute default values.

\end{fulllineitems}


\end{fulllineitems}

\index{Datetime (class in odoo.fields)}

\begin{fulllineitems}
\phantomsection\label{\detokenize{reference/orm:odoo.fields.Datetime}}\pysiglinewithargsret{\sphinxbfcode{\sphinxupquote{class }}\sphinxcode{\sphinxupquote{odoo.fields.}}\sphinxbfcode{\sphinxupquote{Datetime}}}{\emph{string=\textless{}object object\textgreater{}}, \emph{**kwargs}}{}
Bases: {\hyperref[\detokenize{reference/orm:odoo.fields.Field}]{\sphinxcrossref{\sphinxcode{\sphinxupquote{odoo.fields.Field}}}}}
\index{context\_timestamp() (odoo.fields.Datetime static method)}

\begin{fulllineitems}
\phantomsection\label{\detokenize{reference/orm:odoo.fields.Datetime.context_timestamp}}\pysiglinewithargsret{\sphinxbfcode{\sphinxupquote{static }}\sphinxbfcode{\sphinxupquote{context\_timestamp}}}{\emph{record}, \emph{timestamp}}{}
Returns the given timestamp converted to the client’s timezone.
This method is \sphinxstyleemphasis{not} meant for use as a default initializer,
because datetime fields are automatically converted upon
display on client side. For default values \sphinxcode{\sphinxupquote{fields.datetime.now()}}
should be used instead.
\begin{quote}\begin{description}
\item[{Parameters}] \leavevmode
\sphinxstyleliteralstrong{\sphinxupquote{timestamp}} (\sphinxstyleliteralemphasis{\sphinxupquote{datetime}}) \textendash{} naive datetime value (expressed in UTC)
to be converted to the client timezone

\item[{Return type}] \leavevmode
datetime

\item[{Returns}] \leavevmode
timestamp converted to timezone-aware datetime in context
timezone

\end{description}\end{quote}

\end{fulllineitems}

\index{from\_string() (odoo.fields.Datetime static method)}

\begin{fulllineitems}
\phantomsection\label{\detokenize{reference/orm:odoo.fields.Datetime.from_string}}\pysiglinewithargsret{\sphinxbfcode{\sphinxupquote{static }}\sphinxbfcode{\sphinxupquote{from\_string}}}{\emph{value}}{}
Convert an ORM \sphinxcode{\sphinxupquote{value}} into a \sphinxcode{\sphinxupquote{datetime}} value.

\end{fulllineitems}

\index{now() (odoo.fields.Datetime static method)}

\begin{fulllineitems}
\phantomsection\label{\detokenize{reference/orm:odoo.fields.Datetime.now}}\pysiglinewithargsret{\sphinxbfcode{\sphinxupquote{static }}\sphinxbfcode{\sphinxupquote{now}}}{\emph{*args}}{}
Return the current day and time in the format expected by the ORM.
This function may be used to compute default values.

\end{fulllineitems}

\index{to\_string() (odoo.fields.Datetime static method)}

\begin{fulllineitems}
\phantomsection\label{\detokenize{reference/orm:odoo.fields.Datetime.to_string}}\pysiglinewithargsret{\sphinxbfcode{\sphinxupquote{static }}\sphinxbfcode{\sphinxupquote{to\_string}}}{\emph{value}}{}
Convert a \sphinxcode{\sphinxupquote{datetime}} value into the format expected by the ORM.

\end{fulllineitems}


\end{fulllineitems}



\subsubsection{Relational fields}
\label{\detokenize{reference/orm:relational-fields}}\label{\detokenize{reference/orm:reference-orm-fields-relational}}\index{Many2one (class in odoo.fields)}

\begin{fulllineitems}
\phantomsection\label{\detokenize{reference/orm:odoo.fields.Many2one}}\pysiglinewithargsret{\sphinxbfcode{\sphinxupquote{class }}\sphinxcode{\sphinxupquote{odoo.fields.}}\sphinxbfcode{\sphinxupquote{Many2one}}}{\emph{comodel\_name=\textless{}object object\textgreater{}}, \emph{string=\textless{}object object\textgreater{}}, \emph{**kwargs}}{}
Bases: \sphinxcode{\sphinxupquote{odoo.fields.\_Relational}}

The value of such a field is a recordset of size 0 (no
record) or 1 (a single record).
\begin{quote}\begin{description}
\item[{Parameters}] \leavevmode\begin{itemize}
\item {} 
\sphinxstyleliteralstrong{\sphinxupquote{comodel\_name}} \textendash{} name of the target model (string)

\item {} 
\sphinxstyleliteralstrong{\sphinxupquote{domain}} \textendash{} an optional domain to set on candidate values on the
client side (domain or string)

\item {} 
\sphinxstyleliteralstrong{\sphinxupquote{context}} \textendash{} an optional context to use on the client side when
handling that field (dictionary)

\item {} 
\sphinxstyleliteralstrong{\sphinxupquote{ondelete}} \textendash{} what to do when the referred record is deleted;
possible values are: \sphinxcode{\sphinxupquote{'set null'}}, \sphinxcode{\sphinxupquote{'restrict'}}, \sphinxcode{\sphinxupquote{'cascade'}}

\item {} 
\sphinxstyleliteralstrong{\sphinxupquote{auto\_join}} \textendash{} whether JOINs are generated upon search through that
field (boolean, by default \sphinxcode{\sphinxupquote{False}})

\item {} 
\sphinxstyleliteralstrong{\sphinxupquote{delegate}} \textendash{} set it to \sphinxcode{\sphinxupquote{True}} to make fields of the target model
accessible from the current model (corresponds to \sphinxcode{\sphinxupquote{\_inherits}})

\end{itemize}

\end{description}\end{quote}

The attribute \sphinxcode{\sphinxupquote{comodel\_name}} is mandatory except in the case of related
fields or field extensions.

\end{fulllineitems}

\index{One2many (class in odoo.fields)}

\begin{fulllineitems}
\phantomsection\label{\detokenize{reference/orm:odoo.fields.One2many}}\pysiglinewithargsret{\sphinxbfcode{\sphinxupquote{class }}\sphinxcode{\sphinxupquote{odoo.fields.}}\sphinxbfcode{\sphinxupquote{One2many}}}{\emph{comodel\_name=\textless{}object object\textgreater{}}, \emph{inverse\_name=\textless{}object object\textgreater{}}, \emph{string=\textless{}object object\textgreater{}}, \emph{**kwargs}}{}
Bases: \sphinxcode{\sphinxupquote{odoo.fields.\_RelationalMulti}}

One2many field; the value of such a field is the recordset of all the
records in \sphinxcode{\sphinxupquote{comodel\_name}} such that the field \sphinxcode{\sphinxupquote{inverse\_name}} is equal to
the current record.
\begin{quote}\begin{description}
\item[{Parameters}] \leavevmode\begin{itemize}
\item {} 
\sphinxstyleliteralstrong{\sphinxupquote{comodel\_name}} \textendash{} name of the target model (string)

\item {} 
\sphinxstyleliteralstrong{\sphinxupquote{inverse\_name}} \textendash{} name of the inverse \sphinxcode{\sphinxupquote{Many2one}} field in
\sphinxcode{\sphinxupquote{comodel\_name}} (string)

\item {} 
\sphinxstyleliteralstrong{\sphinxupquote{domain}} \textendash{} an optional domain to set on candidate values on the
client side (domain or string)

\item {} 
\sphinxstyleliteralstrong{\sphinxupquote{context}} \textendash{} an optional context to use on the client side when
handling that field (dictionary)

\item {} 
\sphinxstyleliteralstrong{\sphinxupquote{auto\_join}} \textendash{} whether JOINs are generated upon search through that
field (boolean, by default \sphinxcode{\sphinxupquote{False}})

\item {} 
\sphinxstyleliteralstrong{\sphinxupquote{limit}} \textendash{} optional limit to use upon read (integer)

\end{itemize}

\end{description}\end{quote}

The attributes \sphinxcode{\sphinxupquote{comodel\_name}} and \sphinxcode{\sphinxupquote{inverse\_name}} are mandatory except in
the case of related fields or field extensions.

\end{fulllineitems}

\index{Many2many (class in odoo.fields)}

\begin{fulllineitems}
\phantomsection\label{\detokenize{reference/orm:odoo.fields.Many2many}}\pysiglinewithargsret{\sphinxbfcode{\sphinxupquote{class }}\sphinxcode{\sphinxupquote{odoo.fields.}}\sphinxbfcode{\sphinxupquote{Many2many}}}{\emph{comodel\_name=\textless{}object object\textgreater{}}, \emph{relation=\textless{}object object\textgreater{}}, \emph{column1=\textless{}object object\textgreater{}}, \emph{column2=\textless{}object object\textgreater{}}, \emph{string=\textless{}object object\textgreater{}}, \emph{**kwargs}}{}
Bases: \sphinxcode{\sphinxupquote{odoo.fields.\_RelationalMulti}}

Many2many field; the value of such a field is the recordset.
\begin{quote}\begin{description}
\item[{Parameters}] \leavevmode
\sphinxstyleliteralstrong{\sphinxupquote{comodel\_name}} \textendash{} name of the target model (string)

\end{description}\end{quote}

The attribute \sphinxcode{\sphinxupquote{comodel\_name}} is mandatory except in the case of related
fields or field extensions.
\begin{quote}\begin{description}
\item[{Parameters}] \leavevmode\begin{itemize}
\item {} 
\sphinxstyleliteralstrong{\sphinxupquote{relation}} \textendash{} optional name of the table that stores the relation in
the database (string)

\item {} 
\sphinxstyleliteralstrong{\sphinxupquote{column1}} \textendash{} optional name of the column referring to “these” records
in the table \sphinxcode{\sphinxupquote{relation}} (string)

\item {} 
\sphinxstyleliteralstrong{\sphinxupquote{column2}} \textendash{} optional name of the column referring to “those” records
in the table \sphinxcode{\sphinxupquote{relation}} (string)

\end{itemize}

\end{description}\end{quote}

The attributes \sphinxcode{\sphinxupquote{relation}}, \sphinxcode{\sphinxupquote{column1}} and \sphinxcode{\sphinxupquote{column2}} are optional. If not
given, names are automatically generated from model names, provided
\sphinxcode{\sphinxupquote{model\_name}} and \sphinxcode{\sphinxupquote{comodel\_name}} are different!
\begin{quote}\begin{description}
\item[{Parameters}] \leavevmode\begin{itemize}
\item {} 
\sphinxstyleliteralstrong{\sphinxupquote{domain}} \textendash{} an optional domain to set on candidate values on the
client side (domain or string)

\item {} 
\sphinxstyleliteralstrong{\sphinxupquote{context}} \textendash{} an optional context to use on the client side when
handling that field (dictionary)

\item {} 
\sphinxstyleliteralstrong{\sphinxupquote{limit}} \textendash{} optional limit to use upon read (integer)

\end{itemize}

\end{description}\end{quote}

\end{fulllineitems}

\index{Reference (class in odoo.fields)}

\begin{fulllineitems}
\phantomsection\label{\detokenize{reference/orm:odoo.fields.Reference}}\pysiglinewithargsret{\sphinxbfcode{\sphinxupquote{class }}\sphinxcode{\sphinxupquote{odoo.fields.}}\sphinxbfcode{\sphinxupquote{Reference}}}{\emph{selection=\textless{}object object\textgreater{}}, \emph{string=\textless{}object object\textgreater{}}, \emph{**kwargs}}{}
Bases: {\hyperref[\detokenize{reference/orm:odoo.fields.Selection}]{\sphinxcrossref{\sphinxcode{\sphinxupquote{odoo.fields.Selection}}}}}

\end{fulllineitems}



\subsection{Inheritance and extension}
\label{\detokenize{reference/orm:reference-orm-inheritance}}\label{\detokenize{reference/orm:inheritance-and-extension}}
Odoo provides three different mechanisms to extend models in a modular way:
\begin{itemize}
\item {} 
creating a new model from an existing one, adding new information to the
copy but leaving the original module as-is

\item {} 
extending models defined in other modules in-place, replacing the previous
version

\item {} 
delegating some of the model’s fields to records it contains

\end{itemize}

\noindent{\hspace*{\fill}\sphinxincludegraphics{{inheritance_methods1}.png}\hspace*{\fill}}


\subsubsection{Classical inheritance}
\label{\detokenize{reference/orm:classical-inheritance}}
When using the {\hyperref[\detokenize{reference/orm:odoo.models.Model._inherit}]{\sphinxcrossref{\sphinxcode{\sphinxupquote{\_inherit}}}}} and
{\hyperref[\detokenize{reference/orm:odoo.models.Model._name}]{\sphinxcrossref{\sphinxcode{\sphinxupquote{\_name}}}}} attributes together, Odoo creates a new
model using the existing one (provided via
{\hyperref[\detokenize{reference/orm:odoo.models.Model._inherit}]{\sphinxcrossref{\sphinxcode{\sphinxupquote{\_inherit}}}}}) as a base. The new model gets all the
fields, methods and meta-information (defaults \& al) from its base.

\fvset{hllines={, ,}}%
\begin{sphinxVerbatim}[commandchars=\\\{\}]

\PYG{k}{class} \PYG{n+nc}{Inheritance0}\PYG{p}{(}\PYG{n}{models}\PYG{o}{.}\PYG{n}{Model}\PYG{p}{)}\PYG{p}{:}
    \PYG{n}{\PYGZus{}name} \PYG{o}{=} \PYG{l+s+s1}{\PYGZsq{}}\PYG{l+s+s1}{inheritance.0}\PYG{l+s+s1}{\PYGZsq{}}

    \PYG{n}{name} \PYG{o}{=} \PYG{n}{fields}\PYG{o}{.}\PYG{n}{Char}\PYG{p}{(}\PYG{p}{)}

    \PYG{k}{def} \PYG{n+nf}{call}\PYG{p}{(}\PYG{n+nb+bp}{self}\PYG{p}{)}\PYG{p}{:}
        \PYG{k}{return} \PYG{n+nb+bp}{self}\PYG{o}{.}\PYG{n}{check}\PYG{p}{(}\PYG{l+s+s2}{\PYGZdq{}}\PYG{l+s+s2}{model 0}\PYG{l+s+s2}{\PYGZdq{}}\PYG{p}{)}

    \PYG{k}{def} \PYG{n+nf}{check}\PYG{p}{(}\PYG{n+nb+bp}{self}\PYG{p}{,} \PYG{n}{s}\PYG{p}{)}\PYG{p}{:}
        \PYG{k}{return} \PYG{l+s+s2}{\PYGZdq{}}\PYG{l+s+s2}{This is \PYGZob{}\PYGZcb{} record \PYGZob{}\PYGZcb{}}\PYG{l+s+s2}{\PYGZdq{}}\PYG{o}{.}\PYG{n}{format}\PYG{p}{(}\PYG{n}{s}\PYG{p}{,} \PYG{n+nb+bp}{self}\PYG{o}{.}\PYG{n}{name}\PYG{p}{)}

\PYG{k}{class} \PYG{n+nc}{Inheritance1}\PYG{p}{(}\PYG{n}{models}\PYG{o}{.}\PYG{n}{Model}\PYG{p}{)}\PYG{p}{:}
    \PYG{n}{\PYGZus{}name} \PYG{o}{=} \PYG{l+s+s1}{\PYGZsq{}}\PYG{l+s+s1}{inheritance.1}\PYG{l+s+s1}{\PYGZsq{}}
    \PYG{n}{\PYGZus{}inherit} \PYG{o}{=} \PYG{l+s+s1}{\PYGZsq{}}\PYG{l+s+s1}{inheritance.0}\PYG{l+s+s1}{\PYGZsq{}}

    \PYG{k}{def} \PYG{n+nf}{call}\PYG{p}{(}\PYG{n+nb+bp}{self}\PYG{p}{)}\PYG{p}{:}
        \PYG{k}{return} \PYG{n+nb+bp}{self}\PYG{o}{.}\PYG{n}{check}\PYG{p}{(}\PYG{l+s+s2}{\PYGZdq{}}\PYG{l+s+s2}{model 1}\PYG{l+s+s2}{\PYGZdq{}}\PYG{p}{)}
\end{sphinxVerbatim}

and using them:

\fvset{hllines={, ,}}%
\begin{sphinxVerbatim}[commandchars=\\\{\}]
        \PYG{n}{a} \PYG{o}{=} \PYG{n}{env}\PYG{p}{[}\PYG{l+s+s1}{\PYGZsq{}}\PYG{l+s+s1}{inheritance.0}\PYG{l+s+s1}{\PYGZsq{}}\PYG{p}{]}\PYG{o}{.}\PYG{n}{create}\PYG{p}{(}\PYG{p}{\PYGZob{}}\PYG{l+s+s1}{\PYGZsq{}}\PYG{l+s+s1}{name}\PYG{l+s+s1}{\PYGZsq{}}\PYG{p}{:} \PYG{l+s+s1}{\PYGZsq{}}\PYG{l+s+s1}{A}\PYG{l+s+s1}{\PYGZsq{}}\PYG{p}{\PYGZcb{}}\PYG{p}{)}
        \PYG{n}{b} \PYG{o}{=} \PYG{n}{env}\PYG{p}{[}\PYG{l+s+s1}{\PYGZsq{}}\PYG{l+s+s1}{inheritance.1}\PYG{l+s+s1}{\PYGZsq{}}\PYG{p}{]}\PYG{o}{.}\PYG{n}{create}\PYG{p}{(}\PYG{p}{\PYGZob{}}\PYG{l+s+s1}{\PYGZsq{}}\PYG{l+s+s1}{name}\PYG{l+s+s1}{\PYGZsq{}}\PYG{p}{:} \PYG{l+s+s1}{\PYGZsq{}}\PYG{l+s+s1}{B}\PYG{l+s+s1}{\PYGZsq{}}\PYG{p}{\PYGZcb{}}\PYG{p}{)}
            \PYG{n}{a}\PYG{o}{.}\PYG{n}{call}\PYG{p}{(}\PYG{p}{)}
            \PYG{n}{b}\PYG{o}{.}\PYG{n}{call}\PYG{p}{(}\PYG{p}{)}
\end{sphinxVerbatim}

will yield:

\fvset{hllines={, ,}}%
\begin{sphinxVerbatim}[commandchars=\\\{\}]
            \PYGZdq{}This is model 0 record A\PYGZdq{}
            \PYGZdq{}This is model 1 record B\PYGZdq{}
\end{sphinxVerbatim}

the second model has inherited from the first model’s \sphinxcode{\sphinxupquote{check}} method and its
\sphinxcode{\sphinxupquote{name}} field, but overridden the \sphinxcode{\sphinxupquote{call}} method, as when using standard
\sphinxhref{https://docs.python.org/3/tutorial/classes.html\#tut-inheritance}{\DUrole{xref,std,std-ref}{Python inheritance}}.


\subsubsection{Extension}
\label{\detokenize{reference/orm:extension}}
When using {\hyperref[\detokenize{reference/orm:odoo.models.Model._inherit}]{\sphinxcrossref{\sphinxcode{\sphinxupquote{\_inherit}}}}} but leaving out
{\hyperref[\detokenize{reference/orm:odoo.models.Model._name}]{\sphinxcrossref{\sphinxcode{\sphinxupquote{\_name}}}}}, the new model replaces the existing one,
essentially extending it in-place. This is useful to add new fields or methods
to existing models (created in other modules), or to customize or reconfigure
them (e.g. to change their default sort order):

\fvset{hllines={, ,}}%
\begin{sphinxVerbatim}[commandchars=\\\{\}]

\PYG{k}{class} \PYG{n+nc}{Extension0}\PYG{p}{(}\PYG{n}{models}\PYG{o}{.}\PYG{n}{Model}\PYG{p}{)}\PYG{p}{:}
    \PYG{n}{\PYGZus{}name} \PYG{o}{=} \PYG{l+s+s1}{\PYGZsq{}}\PYG{l+s+s1}{extension.0}\PYG{l+s+s1}{\PYGZsq{}}

    \PYG{n}{name} \PYG{o}{=} \PYG{n}{fields}\PYG{o}{.}\PYG{n}{Char}\PYG{p}{(}\PYG{n}{default}\PYG{o}{=}\PYG{l+s+s2}{\PYGZdq{}}\PYG{l+s+s2}{A}\PYG{l+s+s2}{\PYGZdq{}}\PYG{p}{)}

\PYG{k}{class} \PYG{n+nc}{Extension1}\PYG{p}{(}\PYG{n}{models}\PYG{o}{.}\PYG{n}{Model}\PYG{p}{)}\PYG{p}{:}
    \PYG{n}{\PYGZus{}inherit} \PYG{o}{=} \PYG{l+s+s1}{\PYGZsq{}}\PYG{l+s+s1}{extension.0}\PYG{l+s+s1}{\PYGZsq{}}

    \PYG{n}{description} \PYG{o}{=} \PYG{n}{fields}\PYG{o}{.}\PYG{n}{Char}\PYG{p}{(}\PYG{n}{default}\PYG{o}{=}\PYG{l+s+s2}{\PYGZdq{}}\PYG{l+s+s2}{Extended}\PYG{l+s+s2}{\PYGZdq{}}\PYG{p}{)}
\end{sphinxVerbatim}

\fvset{hllines={, ,}}%
\begin{sphinxVerbatim}[commandchars=\\\{\}]
        \PYG{n}{env} \PYG{o}{=} \PYG{n+nb+bp}{self}\PYG{o}{.}\PYG{n}{env}
        \PYG{p}{\PYGZob{}}\PYG{l+s+s1}{\PYGZsq{}}\PYG{l+s+s1}{name}\PYG{l+s+s1}{\PYGZsq{}}\PYG{p}{:} \PYG{l+s+s2}{\PYGZdq{}}\PYG{l+s+s2}{A}\PYG{l+s+s2}{\PYGZdq{}}\PYG{p}{,} \PYG{l+s+s1}{\PYGZsq{}}\PYG{l+s+s1}{description}\PYG{l+s+s1}{\PYGZsq{}}\PYG{p}{:} \PYG{l+s+s2}{\PYGZdq{}}\PYG{l+s+s2}{Extended}\PYG{l+s+s2}{\PYGZdq{}}\PYG{p}{\PYGZcb{}}
\end{sphinxVerbatim}

will yield:

\fvset{hllines={, ,}}%
\begin{sphinxVerbatim}[commandchars=\\\{\}]

\end{sphinxVerbatim}

\begin{sphinxadmonition}{note}{Note:}
it will also yield the various {\hyperref[\detokenize{reference/orm:reference-orm-model-automatic}]{\sphinxcrossref{\DUrole{std,std-ref}{automatic fields}}}} unless they’ve been disabled
\end{sphinxadmonition}


\subsubsection{Delegation}
\label{\detokenize{reference/orm:delegation}}
The third inheritance mechanism provides more flexibility (it can be altered
at runtime) but less power: using the {\hyperref[\detokenize{reference/orm:odoo.models.Model._inherits}]{\sphinxcrossref{\sphinxcode{\sphinxupquote{\_inherits}}}}}
a model \sphinxstyleemphasis{delegates} the lookup of any field not found on the current model
to “children” models. The delegation is performed via
{\hyperref[\detokenize{reference/orm:odoo.fields.Reference}]{\sphinxcrossref{\sphinxcode{\sphinxupquote{Reference}}}}} fields automatically set up on the parent
model:

\fvset{hllines={, ,}}%
\begin{sphinxVerbatim}[commandchars=\\\{\}]

\PYG{k}{class} \PYG{n+nc}{Child0}\PYG{p}{(}\PYG{n}{models}\PYG{o}{.}\PYG{n}{Model}\PYG{p}{)}\PYG{p}{:}
    \PYG{n}{\PYGZus{}name} \PYG{o}{=} \PYG{l+s+s1}{\PYGZsq{}}\PYG{l+s+s1}{delegation.child0}\PYG{l+s+s1}{\PYGZsq{}}

    \PYG{n}{field\PYGZus{}0} \PYG{o}{=} \PYG{n}{fields}\PYG{o}{.}\PYG{n}{Integer}\PYG{p}{(}\PYG{p}{)}

\PYG{k}{class} \PYG{n+nc}{Child1}\PYG{p}{(}\PYG{n}{models}\PYG{o}{.}\PYG{n}{Model}\PYG{p}{)}\PYG{p}{:}
    \PYG{n}{\PYGZus{}name} \PYG{o}{=} \PYG{l+s+s1}{\PYGZsq{}}\PYG{l+s+s1}{delegation.child1}\PYG{l+s+s1}{\PYGZsq{}}

    \PYG{n}{field\PYGZus{}1} \PYG{o}{=} \PYG{n}{fields}\PYG{o}{.}\PYG{n}{Integer}\PYG{p}{(}\PYG{p}{)}

\PYG{k}{class} \PYG{n+nc}{Delegating}\PYG{p}{(}\PYG{n}{models}\PYG{o}{.}\PYG{n}{Model}\PYG{p}{)}\PYG{p}{:}
    \PYG{n}{\PYGZus{}name} \PYG{o}{=} \PYG{l+s+s1}{\PYGZsq{}}\PYG{l+s+s1}{delegation.parent}\PYG{l+s+s1}{\PYGZsq{}}

    \PYG{n}{\PYGZus{}inherits} \PYG{o}{=} \PYG{p}{\PYGZob{}}
        \PYG{l+s+s1}{\PYGZsq{}}\PYG{l+s+s1}{delegation.child0}\PYG{l+s+s1}{\PYGZsq{}}\PYG{p}{:} \PYG{l+s+s1}{\PYGZsq{}}\PYG{l+s+s1}{child0\PYGZus{}id}\PYG{l+s+s1}{\PYGZsq{}}\PYG{p}{,}
        \PYG{l+s+s1}{\PYGZsq{}}\PYG{l+s+s1}{delegation.child1}\PYG{l+s+s1}{\PYGZsq{}}\PYG{p}{:} \PYG{l+s+s1}{\PYGZsq{}}\PYG{l+s+s1}{child1\PYGZus{}id}\PYG{l+s+s1}{\PYGZsq{}}\PYG{p}{,}
    \PYG{p}{\PYGZcb{}}

    \PYG{n}{child0\PYGZus{}id} \PYG{o}{=} \PYG{n}{fields}\PYG{o}{.}\PYG{n}{Many2one}\PYG{p}{(}\PYG{l+s+s1}{\PYGZsq{}}\PYG{l+s+s1}{delegation.child0}\PYG{l+s+s1}{\PYGZsq{}}\PYG{p}{,} \PYG{n}{required}\PYG{o}{=}\PYG{n+nb+bp}{True}\PYG{p}{,} \PYG{n}{ondelete}\PYG{o}{=}\PYG{l+s+s1}{\PYGZsq{}}\PYG{l+s+s1}{cascade}\PYG{l+s+s1}{\PYGZsq{}}\PYG{p}{)}
    \PYG{n}{child1\PYGZus{}id} \PYG{o}{=} \PYG{n}{fields}\PYG{o}{.}\PYG{n}{Many2one}\PYG{p}{(}\PYG{l+s+s1}{\PYGZsq{}}\PYG{l+s+s1}{delegation.child1}\PYG{l+s+s1}{\PYGZsq{}}\PYG{p}{,} \PYG{n}{required}\PYG{o}{=}\PYG{n+nb+bp}{True}\PYG{p}{,} \PYG{n}{ondelete}\PYG{o}{=}\PYG{l+s+s1}{\PYGZsq{}}\PYG{l+s+s1}{cascade}\PYG{l+s+s1}{\PYGZsq{}}\PYG{p}{)}
\end{sphinxVerbatim}

\fvset{hllines={, ,}}%
\begin{sphinxVerbatim}[commandchars=\\\{\}]
        \PYG{n}{record} \PYG{o}{=} \PYG{n}{env}\PYG{p}{[}\PYG{l+s+s1}{\PYGZsq{}}\PYG{l+s+s1}{delegation.parent}\PYG{l+s+s1}{\PYGZsq{}}\PYG{p}{]}\PYG{o}{.}\PYG{n}{create}\PYG{p}{(}\PYG{p}{\PYGZob{}}
            \PYG{l+s+s1}{\PYGZsq{}}\PYG{l+s+s1}{child0\PYGZus{}id}\PYG{l+s+s1}{\PYGZsq{}}\PYG{p}{:} \PYG{n}{env}\PYG{p}{[}\PYG{l+s+s1}{\PYGZsq{}}\PYG{l+s+s1}{delegation.child0}\PYG{l+s+s1}{\PYGZsq{}}\PYG{p}{]}\PYG{o}{.}\PYG{n}{create}\PYG{p}{(}\PYG{p}{\PYGZob{}}\PYG{l+s+s1}{\PYGZsq{}}\PYG{l+s+s1}{field\PYGZus{}0}\PYG{l+s+s1}{\PYGZsq{}}\PYG{p}{:} \PYG{l+m+mi}{0}\PYG{p}{\PYGZcb{}}\PYG{p}{)}\PYG{o}{.}\PYG{n}{id}\PYG{p}{,}
            \PYG{l+s+s1}{\PYGZsq{}}\PYG{l+s+s1}{child1\PYGZus{}id}\PYG{l+s+s1}{\PYGZsq{}}\PYG{p}{:} \PYG{n}{env}\PYG{p}{[}\PYG{l+s+s1}{\PYGZsq{}}\PYG{l+s+s1}{delegation.child1}\PYG{l+s+s1}{\PYGZsq{}}\PYG{p}{]}\PYG{o}{.}\PYG{n}{create}\PYG{p}{(}\PYG{p}{\PYGZob{}}\PYG{l+s+s1}{\PYGZsq{}}\PYG{l+s+s1}{field\PYGZus{}1}\PYG{l+s+s1}{\PYGZsq{}}\PYG{p}{:} \PYG{l+m+mi}{1}\PYG{p}{\PYGZcb{}}\PYG{p}{)}\PYG{o}{.}\PYG{n}{id}\PYG{p}{,}
        \PYG{p}{\PYGZcb{}}\PYG{p}{)}
            \PYG{n}{record}\PYG{o}{.}\PYG{n}{field\PYGZus{}0}
            \PYG{n}{record}\PYG{o}{.}\PYG{n}{field\PYGZus{}1}
\end{sphinxVerbatim}

will result in:

\fvset{hllines={, ,}}%
\begin{sphinxVerbatim}[commandchars=\\\{\}]
            0
            1
\end{sphinxVerbatim}

and it’s possible to write directly on the delegated field:

\fvset{hllines={, ,}}%
\begin{sphinxVerbatim}[commandchars=\\\{\}]
        \PYG{n}{record}\PYG{o}{.}\PYG{n}{write}\PYG{p}{(}\PYG{p}{\PYGZob{}}\PYG{l+s+s1}{\PYGZsq{}}\PYG{l+s+s1}{field\PYGZus{}1}\PYG{l+s+s1}{\PYGZsq{}}\PYG{p}{:} \PYG{l+m+mi}{4}\PYG{p}{\PYGZcb{}}\PYG{p}{)}
\end{sphinxVerbatim}

\begin{sphinxadmonition}{warning}{Warning:}
when using delegation inheritance, methods are \sphinxstyleemphasis{not} inherited,
only fields
\end{sphinxadmonition}


\subsection{Domains}
\label{\detokenize{reference/orm:domains}}\label{\detokenize{reference/orm:reference-orm-domains}}
A domain is a list of criteria, each criterion being a triple (either a
\sphinxcode{\sphinxupquote{list}} or a \sphinxcode{\sphinxupquote{tuple}}) of \sphinxcode{\sphinxupquote{(field\_name, operator, value)}} where:
\begin{description}
\item[{\sphinxcode{\sphinxupquote{field\_name}} (\sphinxcode{\sphinxupquote{str}})}] \leavevmode
a field name of the current model, or a relationship traversal through
a {\hyperref[\detokenize{reference/orm:odoo.fields.Many2one}]{\sphinxcrossref{\sphinxcode{\sphinxupquote{Many2one}}}}} using dot-notation e.g. \sphinxcode{\sphinxupquote{'street'}}
or \sphinxcode{\sphinxupquote{'partner\_id.country'}}

\item[{\sphinxcode{\sphinxupquote{operator}} (\sphinxcode{\sphinxupquote{str}})}] \leavevmode
an operator used to compare the \sphinxcode{\sphinxupquote{field\_name}} with the \sphinxcode{\sphinxupquote{value}}. Valid
operators are:
\begin{description}
\item[{\sphinxcode{\sphinxupquote{=}}}] \leavevmode
equals to

\item[{\sphinxcode{\sphinxupquote{!=}}}] \leavevmode
not equals to

\item[{\sphinxcode{\sphinxupquote{\textgreater{}}}}] \leavevmode
greater than

\item[{\sphinxcode{\sphinxupquote{\textgreater{}=}}}] \leavevmode
greater than or equal to

\item[{\sphinxcode{\sphinxupquote{\textless{}}}}] \leavevmode
less than

\item[{\sphinxcode{\sphinxupquote{\textless{}=}}}] \leavevmode
less than or equal to

\item[{\sphinxcode{\sphinxupquote{=?}}}] \leavevmode
unset or equals to (returns true if \sphinxcode{\sphinxupquote{value}} is either \sphinxcode{\sphinxupquote{None}} or
\sphinxcode{\sphinxupquote{False}}, otherwise behaves like \sphinxcode{\sphinxupquote{=}})

\item[{\sphinxcode{\sphinxupquote{=like}}}] \leavevmode
matches \sphinxcode{\sphinxupquote{field\_name}} against the \sphinxcode{\sphinxupquote{value}} pattern. An underscore
\sphinxcode{\sphinxupquote{\_}} in the pattern stands for (matches) any single character; a
percent sign \sphinxcode{\sphinxupquote{\%}} matches any string of zero or more characters.

\item[{\sphinxcode{\sphinxupquote{like}}}] \leavevmode
matches \sphinxcode{\sphinxupquote{field\_name}} against the \sphinxcode{\sphinxupquote{\%value\%}} pattern. Similar to
\sphinxcode{\sphinxupquote{=like}} but wraps \sphinxcode{\sphinxupquote{value}} with ‘\%’ before matching

\item[{\sphinxcode{\sphinxupquote{not like}}}] \leavevmode
doesn’t match against the \sphinxcode{\sphinxupquote{\%value\%}} pattern

\item[{\sphinxcode{\sphinxupquote{ilike}}}] \leavevmode
case insensitive \sphinxcode{\sphinxupquote{like}}

\item[{\sphinxcode{\sphinxupquote{not ilike}}}] \leavevmode
case insensitive \sphinxcode{\sphinxupquote{not like}}

\item[{\sphinxcode{\sphinxupquote{=ilike}}}] \leavevmode
case insensitive \sphinxcode{\sphinxupquote{=like}}

\item[{\sphinxcode{\sphinxupquote{in}}}] \leavevmode
is equal to any of the items from \sphinxcode{\sphinxupquote{value}}, \sphinxcode{\sphinxupquote{value}} should be a
list of items

\item[{\sphinxcode{\sphinxupquote{not in}}}] \leavevmode
is unequal to all of the items from \sphinxcode{\sphinxupquote{value}}

\item[{\sphinxcode{\sphinxupquote{child\_of}}}] \leavevmode
is a child (descendant) of a \sphinxcode{\sphinxupquote{value}} record.

Takes the semantics of the model into account (i.e following the
relationship field named by
\sphinxcode{\sphinxupquote{\_parent\_name}}).

\end{description}

\item[{\sphinxcode{\sphinxupquote{value}}}] \leavevmode
variable type, must be comparable (through \sphinxcode{\sphinxupquote{operator}}) to the named
field

\end{description}

Domain criteria can be combined using logical operators in \sphinxstyleemphasis{prefix} form:
\begin{description}
\item[{\sphinxcode{\sphinxupquote{'\&'}}}] \leavevmode
logical \sphinxstyleemphasis{AND}, default operation to combine criteria following one
another. Arity 2 (uses the next 2 criteria or combinations).

\item[{\sphinxcode{\sphinxupquote{'\textbar{}'}}}] \leavevmode
logical \sphinxstyleemphasis{OR}, arity 2.

\item[{\sphinxcode{\sphinxupquote{'!'}}}] \leavevmode
logical \sphinxstyleemphasis{NOT}, arity 1.

\begin{sphinxadmonition}{tip}{Tip:}
Mostly to negate combinations of criteria

Individual criterion generally have a negative form (e.g. \sphinxcode{\sphinxupquote{=}} -\textgreater{}
\sphinxcode{\sphinxupquote{!=}}, \sphinxcode{\sphinxupquote{\textless{}}} -\textgreater{} \sphinxcode{\sphinxupquote{\textgreater{}=}}) which is simpler than negating the positive.
\end{sphinxadmonition}

\end{description}

\begin{sphinxadmonition}{note}{Example}

To search for partners named \sphinxstyleemphasis{ABC}, from belgium or germany, whose language
is not english:

\fvset{hllines={, ,}}%
\begin{sphinxVerbatim}[commandchars=\\\{\}]
\PYG{p}{[}\PYG{p}{(}\PYG{l+s+s1}{\PYGZsq{}}\PYG{l+s+s1}{name}\PYG{l+s+s1}{\PYGZsq{}}\PYG{p}{,}\PYG{l+s+s1}{\PYGZsq{}}\PYG{l+s+s1}{=}\PYG{l+s+s1}{\PYGZsq{}}\PYG{p}{,}\PYG{l+s+s1}{\PYGZsq{}}\PYG{l+s+s1}{ABC}\PYG{l+s+s1}{\PYGZsq{}}\PYG{p}{)}\PYG{p}{,}
 \PYG{p}{(}\PYG{l+s+s1}{\PYGZsq{}}\PYG{l+s+s1}{language.code}\PYG{l+s+s1}{\PYGZsq{}}\PYG{p}{,}\PYG{l+s+s1}{\PYGZsq{}}\PYG{l+s+s1}{!=}\PYG{l+s+s1}{\PYGZsq{}}\PYG{p}{,}\PYG{l+s+s1}{\PYGZsq{}}\PYG{l+s+s1}{en\PYGZus{}US}\PYG{l+s+s1}{\PYGZsq{}}\PYG{p}{)}\PYG{p}{,}
 \PYG{l+s+s1}{\PYGZsq{}}\PYG{l+s+s1}{\textbar{}}\PYG{l+s+s1}{\PYGZsq{}}\PYG{p}{,}\PYG{p}{(}\PYG{l+s+s1}{\PYGZsq{}}\PYG{l+s+s1}{country\PYGZus{}id.code}\PYG{l+s+s1}{\PYGZsq{}}\PYG{p}{,}\PYG{l+s+s1}{\PYGZsq{}}\PYG{l+s+s1}{=}\PYG{l+s+s1}{\PYGZsq{}}\PYG{p}{,}\PYG{l+s+s1}{\PYGZsq{}}\PYG{l+s+s1}{be}\PYG{l+s+s1}{\PYGZsq{}}\PYG{p}{)}\PYG{p}{,}
     \PYG{p}{(}\PYG{l+s+s1}{\PYGZsq{}}\PYG{l+s+s1}{country\PYGZus{}id.code}\PYG{l+s+s1}{\PYGZsq{}}\PYG{p}{,}\PYG{l+s+s1}{\PYGZsq{}}\PYG{l+s+s1}{=}\PYG{l+s+s1}{\PYGZsq{}}\PYG{p}{,}\PYG{l+s+s1}{\PYGZsq{}}\PYG{l+s+s1}{de}\PYG{l+s+s1}{\PYGZsq{}}\PYG{p}{)}\PYG{p}{]}
\end{sphinxVerbatim}

This domain is interpreted as:

\fvset{hllines={, ,}}%
\begin{sphinxVerbatim}[commandchars=\\\{\}]
    (name is \PYGZsq{}ABC\PYGZsq{})
AND (language is NOT english)
AND (country is Belgium OR Germany)
\end{sphinxVerbatim}
\end{sphinxadmonition}


\subsection{Porting from the old API to the new API}
\label{\detokenize{reference/orm:porting-from-the-old-api-to-the-new-api}}\begin{itemize}
\item {} 
bare lists of ids are to be avoided in the new API, use recordsets instead

\item {} 
methods still written in the old API should be automatically bridged by the
ORM, no need to switch to the old API, just call them as if they were a new
API method. See {\hyperref[\detokenize{reference/orm:reference-orm-oldapi-bridging}]{\sphinxcrossref{\DUrole{std,std-ref}{Automatic bridging of old API methods}}}} for more details.

\item {} 
{\hyperref[\detokenize{reference/orm:odoo.models.Model.search}]{\sphinxcrossref{\sphinxcode{\sphinxupquote{search()}}}}} returns a recordset, no point in e.g.
browsing its result

\item {} 
\sphinxcode{\sphinxupquote{fields.related}} and \sphinxcode{\sphinxupquote{fields.function}} are replaced by using a normal
field type with either a \sphinxcode{\sphinxupquote{related=}} or a \sphinxcode{\sphinxupquote{compute=}} parameter

\item {} 
{\hyperref[\detokenize{reference/orm:odoo.api.depends}]{\sphinxcrossref{\sphinxcode{\sphinxupquote{depends()}}}}} on \sphinxcode{\sphinxupquote{compute=}} methods \sphinxstylestrong{must be complete},
it must list \sphinxstylestrong{all} the fields and sub-fields which the compute method
uses. It is better to have too many dependencies (will recompute the field
in cases where that is not needed) than not enough (will forget to recompute
the field and then values will be incorrect)

\item {} 
\sphinxstylestrong{remove} all \sphinxcode{\sphinxupquote{onchange}} methods on computed fields. Computed fields are
automatically re-computed when one of their dependencies is changed, and
that is used to auto-generate \sphinxcode{\sphinxupquote{onchange}} by the client

\item {} 
the decorators {\hyperref[\detokenize{reference/orm:odoo.api.model}]{\sphinxcrossref{\sphinxcode{\sphinxupquote{model()}}}}} and {\hyperref[\detokenize{reference/orm:odoo.api.multi}]{\sphinxcrossref{\sphinxcode{\sphinxupquote{multi()}}}}} are
for bridging \sphinxstyleemphasis{when calling from the old API context}, for internal or pure
new-api (e.g. compute) they are useless

\item {} 
remove \sphinxcode{\sphinxupquote{\_default}}, replace by \sphinxcode{\sphinxupquote{default=}}
parameter on corresponding fields

\item {} 
if a field’s \sphinxcode{\sphinxupquote{string=}} is the titlecased version of the field name:

\fvset{hllines={, ,}}%
\begin{sphinxVerbatim}[commandchars=\\\{\}]
\PYG{n}{name} \PYG{o}{=} \PYG{n}{fields}\PYG{o}{.}\PYG{n}{Char}\PYG{p}{(}\PYG{n}{string}\PYG{o}{=}\PYG{l+s+s2}{\PYGZdq{}}\PYG{l+s+s2}{Name}\PYG{l+s+s2}{\PYGZdq{}}\PYG{p}{)}
\end{sphinxVerbatim}

it is useless and should be removed

\item {} 
the \sphinxcode{\sphinxupquote{multi=}} parameter does not do anything on new API fields use the same
\sphinxcode{\sphinxupquote{compute=}} methods on all relevant fields for the same result

\item {} 
provide \sphinxcode{\sphinxupquote{compute=}}, \sphinxcode{\sphinxupquote{inverse=}} and \sphinxcode{\sphinxupquote{search=}} methods by name (as a
string), this makes them overridable (removes the need for an intermediate
“trampoline” function)

\item {} 
double check that all fields and methods have different names, there is no
warning in case of collision (because Python handles it before Odoo sees
anything)

\item {} 
the normal new-api import is \sphinxcode{\sphinxupquote{from odoo import fields, models}}. If
compatibility decorators are necessary, use \sphinxcode{\sphinxupquote{from odoo import api,
fields, models}}

\item {} 
avoid the {\hyperref[\detokenize{reference/orm:odoo.api.one}]{\sphinxcrossref{\sphinxcode{\sphinxupquote{one()}}}}} decorator, it probably does not do what
you expect

\item {} 
remove explicit definition of {\hyperref[\detokenize{reference/orm:odoo.models.Model.create_uid}]{\sphinxcrossref{\sphinxcode{\sphinxupquote{create\_uid}}}}},
{\hyperref[\detokenize{reference/orm:odoo.models.Model.create_date}]{\sphinxcrossref{\sphinxcode{\sphinxupquote{create\_date}}}}},
{\hyperref[\detokenize{reference/orm:odoo.models.Model.write_uid}]{\sphinxcrossref{\sphinxcode{\sphinxupquote{write\_uid}}}}} and
{\hyperref[\detokenize{reference/orm:odoo.models.Model.write_date}]{\sphinxcrossref{\sphinxcode{\sphinxupquote{write\_date}}}}} fields: they are now created as
regular “legitimate” fields, and can be read and written like any other
field out-of-the-box

\item {} 
when straight conversion is impossible (semantics can not be bridged) or the
“old API” version is not desirable and could be improved for the new API, it
is possible to use completely different “old API” and “new API”
implementations for the same method name using {\hyperref[\detokenize{reference/orm:odoo.api.v7}]{\sphinxcrossref{\sphinxcode{\sphinxupquote{v7()}}}}} and
{\hyperref[\detokenize{reference/orm:odoo.api.v8}]{\sphinxcrossref{\sphinxcode{\sphinxupquote{v8()}}}}}. The method should first be defined using the
old-API style and decorated with {\hyperref[\detokenize{reference/orm:odoo.api.v7}]{\sphinxcrossref{\sphinxcode{\sphinxupquote{v7()}}}}}, it should then be
re-defined using the exact same name but the new-API style and decorated
with {\hyperref[\detokenize{reference/orm:odoo.api.v8}]{\sphinxcrossref{\sphinxcode{\sphinxupquote{v8()}}}}}. Calls from an old-API context will be
dispatched to the first implementation and calls from a new-API context will
be dispatched to the second implementation. One implementation can call (and
frequently does) call the other by switching context.

\begin{sphinxadmonition}{danger}{Danger:}
using these decorators makes methods extremely difficult to
override and harder to understand and document
\end{sphinxadmonition}

\item {} 
uses of \sphinxcode{\sphinxupquote{\_columns}} or
\sphinxcode{\sphinxupquote{\_all\_columns}} should be replaced by
\sphinxcode{\sphinxupquote{\_fields}}, which provides access to instances of
new-style {\hyperref[\detokenize{reference/orm:odoo.fields.Field}]{\sphinxcrossref{\sphinxcode{\sphinxupquote{odoo.fields.Field}}}}} instances (rather than old-style
\sphinxcode{\sphinxupquote{odoo.osv.fields.\_column}}).

Non-stored computed fields created using the new API style are \sphinxstyleemphasis{not}
available in \sphinxcode{\sphinxupquote{\_columns}} and can only be
inspected through \sphinxcode{\sphinxupquote{\_fields}}

\item {} 
reassigning \sphinxcode{\sphinxupquote{self}} in a method is probably unnecessary and may break
translation introspection

\item {} 
\sphinxcode{\sphinxupquote{Environment}} objects rely on some threadlocal state,
which has to be set up before using them. It is necessary to do so using the
\sphinxcode{\sphinxupquote{odoo.api.Environment.manage()}} context manager when trying to use
the new API in contexts where it hasn’t been set up yet, such as new threads
or a Python interactive environment:

\fvset{hllines={, ,}}%
\begin{sphinxVerbatim}[commandchars=\\\{\}]
\PYG{g+gp}{\PYGZgt{}\PYGZgt{}\PYGZgt{} }\PYG{k+kn}{from} \PYG{n+nn}{odoo} \PYG{k}{import} \PYG{n}{api}\PYG{p}{,} \PYG{n}{modules}
\PYG{g+gp}{\PYGZgt{}\PYGZgt{}\PYGZgt{} }\PYG{n}{r} \PYG{o}{=} \PYG{n}{modules}\PYG{o}{.}\PYG{n}{registry}\PYG{o}{.}\PYG{n}{RegistryManager}\PYG{o}{.}\PYG{n}{get}\PYG{p}{(}\PYG{l+s+s1}{\PYGZsq{}}\PYG{l+s+s1}{test}\PYG{l+s+s1}{\PYGZsq{}}\PYG{p}{)}
\PYG{g+gp}{\PYGZgt{}\PYGZgt{}\PYGZgt{} }\PYG{n}{cr} \PYG{o}{=} \PYG{n}{r}\PYG{o}{.}\PYG{n}{cursor}\PYG{p}{(}\PYG{p}{)}
\PYG{g+gp}{\PYGZgt{}\PYGZgt{}\PYGZgt{} }\PYG{n}{env} \PYG{o}{=} \PYG{n}{api}\PYG{o}{.}\PYG{n}{Environment}\PYG{p}{(}\PYG{n}{cr}\PYG{p}{,} \PYG{l+m+mi}{1}\PYG{p}{,} \PYG{p}{\PYGZob{}}\PYG{p}{\PYGZcb{}}\PYG{p}{)}
\PYG{g+gt}{Traceback (most recent call last):}
  \PYG{c}{...}
\PYG{g+gr}{AttributeError}: \PYG{n}{environments}
\PYG{g+gp}{\PYGZgt{}\PYGZgt{}\PYGZgt{} }\PYG{k}{with} \PYG{n}{api}\PYG{o}{.}\PYG{n}{Environment}\PYG{o}{.}\PYG{n}{manage}\PYG{p}{(}\PYG{p}{)}\PYG{p}{:}
\PYG{g+gp}{... }    \PYG{n}{env} \PYG{o}{=} \PYG{n}{api}\PYG{o}{.}\PYG{n}{Environment}\PYG{p}{(}\PYG{n}{cr}\PYG{p}{,} \PYG{l+m+mi}{1}\PYG{p}{,} \PYG{p}{\PYGZob{}}\PYG{p}{\PYGZcb{}}\PYG{p}{)}
\PYG{g+gp}{... }    \PYG{n+nb}{print} \PYG{n}{env}\PYG{p}{[}\PYG{l+s+s1}{\PYGZsq{}}\PYG{l+s+s1}{res.partner}\PYG{l+s+s1}{\PYGZsq{}}\PYG{p}{]}\PYG{o}{.}\PYG{n}{browse}\PYG{p}{(}\PYG{l+m+mi}{1}\PYG{p}{)}
\PYG{g+gp}{...}
\PYG{g+go}{res.partner(1,)}
\end{sphinxVerbatim}

\end{itemize}


\subsubsection{Automatic bridging of old API methods}
\label{\detokenize{reference/orm:automatic-bridging-of-old-api-methods}}\label{\detokenize{reference/orm:reference-orm-oldapi-bridging}}
When models are initialized, all methods are automatically scanned and bridged
if they look like models declared in the old API style. This bridging makes
them transparently callable from new-API-style methods.

Methods are matched as “old-API style” if their second positional parameter
(after \sphinxcode{\sphinxupquote{self}}) is called either \sphinxcode{\sphinxupquote{cr}} or \sphinxcode{\sphinxupquote{cursor}}. The system also
recognizes the third positional parameter being called \sphinxcode{\sphinxupquote{uid}} or \sphinxcode{\sphinxupquote{user}} and
the fourth being called \sphinxcode{\sphinxupquote{id}} or \sphinxcode{\sphinxupquote{ids}}. It also recognizes the presence of
any parameter called \sphinxcode{\sphinxupquote{context}}.

When calling such methods from a new API context, the system will
automatically fill matched parameters from the current
\sphinxcode{\sphinxupquote{Environment}} (for \sphinxcode{\sphinxupquote{cr}},
\sphinxcode{\sphinxupquote{user}} and
\sphinxcode{\sphinxupquote{context}}) or the current recordset (for \sphinxcode{\sphinxupquote{id}}
and \sphinxcode{\sphinxupquote{ids}}).

In the rare cases where it is necessary, the bridging can be customized by
decorating the old-style method:
\begin{itemize}
\item {} 
disabling it entirely, by decorating a method with
\sphinxcode{\sphinxupquote{noguess()}} there will be no bridging and methods will be
called the exact same way from the new and old API styles

\item {} 
defining the bridge explicitly, this is mostly for methods which are matched
incorrectly (because parameters are named in unexpected ways):
\begin{description}
\item[{\sphinxcode{\sphinxupquote{cr()}}}] \leavevmode
will automatically prepend the current cursor to explicitly provided
parameters, positionally

\item[{\sphinxcode{\sphinxupquote{cr\_uid()}}}] \leavevmode
will automatically prepend the current cursor and user’s id to explictly
provided parameters

\item[{\sphinxcode{\sphinxupquote{cr\_uid\_ids()}}}] \leavevmode
will automatically prepend the current cursor, user’s id and recordset’s
ids to explicitly provided parameters

\item[{\sphinxcode{\sphinxupquote{cr\_uid\_id()}}}] \leavevmode
will loop over the current recordset and call the method once for each
record, prepending the current cursor, user’s id and record’s id to
explicitly provided parameters.

\begin{sphinxadmonition}{danger}{Danger:}
the result of this wrapper is \sphinxstyleemphasis{always a list} when calling
from a new-API context
\end{sphinxadmonition}

\end{description}

All of these methods have a \sphinxcode{\sphinxupquote{\_context}}-suffixed version
(e.g. \sphinxcode{\sphinxupquote{cr\_uid\_context()}}) which also passes the current
context \sphinxstyleemphasis{by keyword}.

\item {} 
dual implementations using {\hyperref[\detokenize{reference/orm:odoo.api.v7}]{\sphinxcrossref{\sphinxcode{\sphinxupquote{v7()}}}}} and
{\hyperref[\detokenize{reference/orm:odoo.api.v8}]{\sphinxcrossref{\sphinxcode{\sphinxupquote{v8()}}}}} will be ignored as they provide their own “bridging”

\end{itemize}


\section{Data Files}
\label{\detokenize{reference/data:data-files}}\label{\detokenize{reference/data::doc}}\label{\detokenize{reference/data:reference-data}}
Odoo is greatly data-driven, and a big part of modules definition is thus
the definition of the various records it manages: UI (menus and views),
security (access rights and access rules), reports and plain data are all
defined via records.


\subsection{Structure}
\label{\detokenize{reference/data:structure}}
The main way to define data in Odoo is via XML data files: The broad structure
of an XML data file is the following:
\begin{itemize}
\item {} 
Any number of operation elements within the root element \sphinxcode{\sphinxupquote{odoo}}

\end{itemize}

\fvset{hllines={, ,}}%
\begin{sphinxVerbatim}[commandchars=\\\{\}]
\PYG{c}{\PYGZlt{}!\PYGZhy{}\PYGZhy{}}\PYG{c}{ the root elements of the data file }\PYG{c}{\PYGZhy{}\PYGZhy{}\PYGZgt{}}
\PYG{n+nt}{\PYGZlt{}odoo}\PYG{n+nt}{\PYGZgt{}}
  \PYG{n+nt}{\PYGZlt{}operation}\PYG{n+nt}{/\PYGZgt{}}
  ...
\PYG{n+nt}{\PYGZlt{}/odoo\PYGZgt{}}
\end{sphinxVerbatim}

Data files are executed sequentially, operations can only refer to the result
of operations defined previously


\subsection{Core operations}
\label{\detokenize{reference/data:core-operations}}

\subsubsection{\sphinxstyleliteralintitle{\sphinxupquote{record}}}
\label{\detokenize{reference/data:record}}\label{\detokenize{reference/data:reference-data-record}}
\sphinxcode{\sphinxupquote{record}} appropriately defines or updates a database record, it has the
following attributes:
\begin{description}
\item[{\sphinxcode{\sphinxupquote{model}} (required)}] \leavevmode
name of the model to create (or update)

\item[{\sphinxcode{\sphinxupquote{id}}}] \leavevmode
the \DUrole{xref,std,std-term}{external identifier} for this record. It is strongly
recommended to provide one
\begin{itemize}
\item {} 
for record creation, allows subsequent definitions to either modify or
refer to this record

\item {} 
for record modification, the record to modify

\end{itemize}

\item[{\sphinxcode{\sphinxupquote{context}}}] \leavevmode
context to use when creating the record

\item[{\sphinxcode{\sphinxupquote{forcecreate}}}] \leavevmode
in update mode whether the record should be created if it doesn’t exist

Requires an \DUrole{xref,std,std-term}{external id}, defaults to \sphinxcode{\sphinxupquote{True}}.

\end{description}


\subsubsection{\sphinxstyleliteralintitle{\sphinxupquote{field}}}
\label{\detokenize{reference/data:field}}
Each record can be composed of \sphinxcode{\sphinxupquote{field}} tags, defining values to set when
creating the record. A \sphinxcode{\sphinxupquote{record}} with no \sphinxcode{\sphinxupquote{field}} will use all default
values (creation) or do nothing (update).

A \sphinxcode{\sphinxupquote{field}} has a mandatory \sphinxcode{\sphinxupquote{name}} attribute, the name of the field to set,
and various methods to define the value itself:
\begin{description}
\item[{Nothing}] \leavevmode
if no value is provided for the field, an implicit \sphinxcode{\sphinxupquote{False}} will be set
on the field. Can be used to clear a field, or avoid using a default value
for the field.

\item[{\sphinxcode{\sphinxupquote{search}}}] \leavevmode
for {\hyperref[\detokenize{reference/orm:reference-orm-fields-relational}]{\sphinxcrossref{\DUrole{std,std-ref}{relational fields}}}}, should be
a {\hyperref[\detokenize{reference/orm:reference-orm-domains}]{\sphinxcrossref{\DUrole{std,std-ref}{domain}}}} on the field’s model.

Will evaluate the domain, search the field’s model using it and set the
search’s result as the field’s value. Will only use the first result if
the field is a {\hyperref[\detokenize{reference/orm:odoo.fields.Many2one}]{\sphinxcrossref{\sphinxcode{\sphinxupquote{Many2one}}}}}

\item[{\sphinxcode{\sphinxupquote{ref}}}] \leavevmode
if a \sphinxcode{\sphinxupquote{ref}} attribute is provided, its value must be a valid
\DUrole{xref,std,std-term}{external id}, which will be looked up and set as the field’s value.

Mostly for {\hyperref[\detokenize{reference/orm:odoo.fields.Many2one}]{\sphinxcrossref{\sphinxcode{\sphinxupquote{Many2one}}}}} and
{\hyperref[\detokenize{reference/orm:odoo.fields.Reference}]{\sphinxcrossref{\sphinxcode{\sphinxupquote{Reference}}}}} fields

\item[{\sphinxcode{\sphinxupquote{type}}}] \leavevmode
if a \sphinxcode{\sphinxupquote{type}} attribute is provided, it is used to interpret and convert
the field’s content. The field’s content can be provided through an
external file using the \sphinxcode{\sphinxupquote{file}} attribute, or through the node’s body.

Available types are:
\begin{description}
\item[{\sphinxcode{\sphinxupquote{xml}}, \sphinxcode{\sphinxupquote{html}}}] \leavevmode
extracts the \sphinxcode{\sphinxupquote{field}}’s children as a single document, evaluates
any \DUrole{xref,std,std-term}{external id} specified with the form \sphinxcode{\sphinxupquote{\%(external\_id)s}}.
\sphinxcode{\sphinxupquote{\%\%}} can be used to output actual \sphinxstyleemphasis{\%} signs.

\item[{\sphinxcode{\sphinxupquote{file}}}] \leavevmode
ensures that the field content is a valid file path in the current
model, saves the pair \sphinxcode{\sphinxupquote{\sphinxstyleemphasis{module},\sphinxstyleemphasis{path}}} as the field value

\item[{\sphinxcode{\sphinxupquote{char}}}] \leavevmode
sets the field content directly as the field’s value without
alterations

\item[{\sphinxcode{\sphinxupquote{base64}}}] \leavevmode
\sphinxhref{http://tools.ietf.org/html/rfc3548.html\#section-3}{base64}-encodes the field’s content, useful combined with the \sphinxcode{\sphinxupquote{file}}
\sphinxstyleemphasis{attribute} to load e.g. image data into attachments

\item[{\sphinxcode{\sphinxupquote{int}}}] \leavevmode
converts the field’s content to an integer and sets it as the field’s
value

\item[{\sphinxcode{\sphinxupquote{float}}}] \leavevmode
converts the field’s content to a float and sets it as the field’s
value

\item[{\sphinxcode{\sphinxupquote{list}}, \sphinxcode{\sphinxupquote{tuple}}}] \leavevmode
should contain any number of \sphinxcode{\sphinxupquote{value}} elements with the same
properties as \sphinxcode{\sphinxupquote{field}}, each element resolves to an item of a
generated tuple or list, and the generated collection is set as the
field’s value

\end{description}

\item[{\sphinxcode{\sphinxupquote{eval}}}] \leavevmode
for cases where the previous methods are unsuitable, the \sphinxcode{\sphinxupquote{eval}}
attributes simply evaluates whatever Python expression it is provided and
sets the result as the field’s value.

The evaluation context contains various modules (\sphinxcode{\sphinxupquote{time}}, \sphinxcode{\sphinxupquote{datetime}},
\sphinxcode{\sphinxupquote{timedelta}}, \sphinxcode{\sphinxupquote{relativedelta}}), a function to resolve \DUrole{xref,std,std-term}{external
identifiers} (\sphinxcode{\sphinxupquote{ref}}) and the model object for the current field if
applicable (\sphinxcode{\sphinxupquote{obj}})

\end{description}


\subsubsection{\sphinxstyleliteralintitle{\sphinxupquote{delete}}}
\label{\detokenize{reference/data:delete}}
The \sphinxcode{\sphinxupquote{delete}} tag can remove any number of records previously defined. It
has the following attributes:
\begin{description}
\item[{\sphinxcode{\sphinxupquote{model}} (required)}] \leavevmode
the model in which a specified record should be deleted

\item[{\sphinxcode{\sphinxupquote{id}}}] \leavevmode
the \DUrole{xref,std,std-term}{external id} of a record to remove

\item[{\sphinxcode{\sphinxupquote{search}}}] \leavevmode
a {\hyperref[\detokenize{reference/orm:reference-orm-domains}]{\sphinxcrossref{\DUrole{std,std-ref}{domain}}}} to find records of the model to
remove

\end{description}

\sphinxcode{\sphinxupquote{id}} and \sphinxcode{\sphinxupquote{search}} are exclusive


\subsubsection{\sphinxstyleliteralintitle{\sphinxupquote{function}}}
\label{\detokenize{reference/data:function}}
The \sphinxcode{\sphinxupquote{function}} tag calls a method on a model, with provided parameters.
It has two mandatory parameters \sphinxcode{\sphinxupquote{model}} and \sphinxcode{\sphinxupquote{name}} specifying respectively
the model and the name of the method to call.

Parameters can be provided using \sphinxcode{\sphinxupquote{eval}} (should evaluate to a sequence of
parameters to call the method with) or \sphinxcode{\sphinxupquote{value}} elements (see \sphinxcode{\sphinxupquote{list}}
values).


\subsection{Shortcuts}
\label{\detokenize{reference/data:shortcuts}}
Because some important structural models of Odoo are complex and involved,
data files provide shorter alternatives to defining them using
{\hyperref[\detokenize{reference/data:reference-data-record}]{\sphinxcrossref{\DUrole{std,std-ref}{record tags}}}}:


\subsubsection{\sphinxstyleliteralintitle{\sphinxupquote{menuitem}}}
\label{\detokenize{reference/data:menuitem}}
Defines an \sphinxcode{\sphinxupquote{ir.ui.menu}} record with a number of defaults and fallbacks:
\begin{description}
\item[{Parent menu}] \leavevmode\begin{itemize}
\item {} 
If a \sphinxcode{\sphinxupquote{parent}} attribute is set, it should be the \DUrole{xref,std,std-term}{external id}
of an other menu item, used as the new item’s parent

\item {} 
If no \sphinxcode{\sphinxupquote{parent}} is provided, tries to interpret the \sphinxcode{\sphinxupquote{name}} attribute
as a \sphinxcode{\sphinxupquote{/}}-separated sequence of menu names and find a place in the menu
hierarchy. In that interpretation, intermediate menus are automatically
created

\item {} 
Otherwise the menu is defined as a “top-level” menu item (\sphinxstyleemphasis{not} a menu
with no parent)

\end{itemize}

\item[{Menu name}] \leavevmode
If no \sphinxcode{\sphinxupquote{name}} attribute is specified, tries to get the menu name from
a linked action if any. Otherwise uses the record’s \sphinxcode{\sphinxupquote{id}}

\item[{Groups}] \leavevmode
A \sphinxcode{\sphinxupquote{groups}} attribute is interpreted as a comma-separated sequence of
\DUrole{xref,std,std-term}{external identifiers} for \sphinxcode{\sphinxupquote{res.groups}} models. If an
\DUrole{xref,std,std-term}{external identifier} is prefixed with a minus (\sphinxcode{\sphinxupquote{-}}), the group
is \sphinxstyleemphasis{removed} from the menu’s groups

\item[{\sphinxcode{\sphinxupquote{action}}}] \leavevmode
if specified, the \sphinxcode{\sphinxupquote{action}} attribute should be the \DUrole{xref,std,std-term}{external id}
of an action to execute when the menu is open

\item[{\sphinxcode{\sphinxupquote{id}}}] \leavevmode
the menu item’s \DUrole{xref,std,std-term}{external id}

\end{description}


\subsubsection{\sphinxstyleliteralintitle{\sphinxupquote{template}}}
\label{\detokenize{reference/data:reference-data-template}}\label{\detokenize{reference/data:template}}
Creates a {\hyperref[\detokenize{reference/views:reference-views-qweb}]{\sphinxcrossref{\DUrole{std,std-ref}{QWeb view}}}} requiring only the \sphinxcode{\sphinxupquote{arch}}
section of the view, and allowing a few \sphinxstyleemphasis{optional} attributes:
\begin{description}
\item[{\sphinxcode{\sphinxupquote{id}}}] \leavevmode
the view’s \DUrole{xref,std,std-term}{external identifier}

\item[{\sphinxcode{\sphinxupquote{name}}, \sphinxcode{\sphinxupquote{inherit\_id}}, \sphinxcode{\sphinxupquote{priority}}}] \leavevmode
same as the corresponding field on \sphinxcode{\sphinxupquote{ir.ui.view}} (nb: \sphinxcode{\sphinxupquote{inherit\_id}}
should be an \DUrole{xref,std,std-term}{external identifier})

\item[{\sphinxcode{\sphinxupquote{primary}}}] \leavevmode
if set to \sphinxcode{\sphinxupquote{True}} and combined with a \sphinxcode{\sphinxupquote{inherit\_id}}, defines the view
as a primary

\item[{\sphinxcode{\sphinxupquote{groups}}}] \leavevmode
comma-separated list of group \DUrole{xref,std,std-term}{external identifiers}

\item[{\sphinxcode{\sphinxupquote{page}}}] \leavevmode
if set to \sphinxcode{\sphinxupquote{"True"}}, the template is a website page (linkable to,
deletable)

\item[{\sphinxcode{\sphinxupquote{optional}}}] \leavevmode
\sphinxcode{\sphinxupquote{enabled}} or \sphinxcode{\sphinxupquote{disabled}}, whether the view can be disabled (in the
website interface) and its default status. If unset, the view is always
enabled.

\end{description}


\subsubsection{\sphinxstyleliteralintitle{\sphinxupquote{report}}}
\label{\detokenize{reference/data:report}}
Creates a \sphinxcode{\sphinxupquote{ir.actions.report}} record with a few default values.

Mostly just proxies attributes to the corresponding fields on
\sphinxcode{\sphinxupquote{ir.actions.report}}, but also automatically creates the item in the
\sphinxmenuselection{More} menu of the report’s \sphinxcode{\sphinxupquote{model}}.


\subsection{CSV data files}
\label{\detokenize{reference/data:csv-data-files}}
XML data files are flexible and self-descriptive, but very verbose when
creating a number of simple records of the same model in bulk.

For this case, data files can also use \sphinxhref{http://en.wikipedia.org/wiki/Comma-separated\_values}{csv}, this is often the case for
{\hyperref[\detokenize{reference/security:reference-security-acl}]{\sphinxcrossref{\DUrole{std,std-ref}{access rights}}}}:
\begin{itemize}
\item {} 
the file name is \sphinxcode{\sphinxupquote{\sphinxstyleemphasis{model\_name}.csv}}

\item {} 
the first row lists the fields to write, with the special field \sphinxcode{\sphinxupquote{id}}
for \DUrole{xref,std,std-term}{external identifiers} (used for creation or update)

\item {} 
each row thereafter creates a new record

\end{itemize}

Here’s the first lines of the data file defining US states
\sphinxcode{\sphinxupquote{res.country.state.csv}}

\fvset{hllines={, ,}}%
\begin{sphinxVerbatim}[commandchars=\\\{\}]
\PYGZdq{}id\PYGZdq{},\PYGZdq{}country\PYGZus{}id:id\PYGZdq{},\PYGZdq{}name\PYGZdq{},\PYGZdq{}code\PYGZdq{}
state\PYGZus{}au\PYGZus{}1,au,\PYGZdq{}Australian Capital Territory\PYGZdq{},\PYGZdq{}ACT\PYGZdq{}
state\PYGZus{}au\PYGZus{}2,au,\PYGZdq{}New South Wales\PYGZdq{},\PYGZdq{}NSW\PYGZdq{}
state\PYGZus{}au\PYGZus{}3,au,\PYGZdq{}Northern Territory\PYGZdq{},\PYGZdq{}NT\PYGZdq{}
state\PYGZus{}au\PYGZus{}4,au,\PYGZdq{}Queensland\PYGZdq{},\PYGZdq{}QLD\PYGZdq{}
state\PYGZus{}au\PYGZus{}5,au,\PYGZdq{}South Australia\PYGZdq{},\PYGZdq{}SA\PYGZdq{}
state\PYGZus{}au\PYGZus{}6,au,\PYGZdq{}Tasmania\PYGZdq{},\PYGZdq{}TAS\PYGZdq{}
state\PYGZus{}au\PYGZus{}7,au,\PYGZdq{}Victoria\PYGZdq{},\PYGZdq{}VIC\PYGZdq{}
state\PYGZus{}au\PYGZus{}8,au,\PYGZdq{}Western Australia\PYGZdq{},\PYGZdq{}WA\PYGZdq{}
state\PYGZus{}us\PYGZus{}1,us,\PYGZdq{}Alabama\PYGZdq{},\PYGZdq{}AL\PYGZdq{}
state\PYGZus{}us\PYGZus{}2,us,\PYGZdq{}Alaska\PYGZdq{},\PYGZdq{}AK\PYGZdq{}
state\PYGZus{}us\PYGZus{}3,us,\PYGZdq{}Arizona\PYGZdq{},\PYGZdq{}AZ\PYGZdq{}
state\PYGZus{}us\PYGZus{}4,us,\PYGZdq{}Arkansas\PYGZdq{},\PYGZdq{}AR\PYGZdq{}
state\PYGZus{}us\PYGZus{}5,us,\PYGZdq{}California\PYGZdq{},\PYGZdq{}CA\PYGZdq{}
state\PYGZus{}us\PYGZus{}6,us,\PYGZdq{}Colorado\PYGZdq{},\PYGZdq{}CO\PYGZdq{}
\end{sphinxVerbatim}

rendered in a more readable format:


\begin{savenotes}\sphinxatlongtablestart\begin{longtable}{|l|l|l|l|}
\hline
\sphinxstyletheadfamily 
id
&\sphinxstyletheadfamily 
country\_id:id
&\sphinxstyletheadfamily 
name
&\sphinxstyletheadfamily 
code
\\
\hline
\endfirsthead

\multicolumn{4}{c}%
{\makebox[0pt]{\sphinxtablecontinued{\tablename\ \thetable{} -- continued from previous page}}}\\
\hline
\sphinxstyletheadfamily 
id
&\sphinxstyletheadfamily 
country\_id:id
&\sphinxstyletheadfamily 
name
&\sphinxstyletheadfamily 
code
\\
\hline
\endhead

\hline
\multicolumn{4}{r}{\makebox[0pt][r]{\sphinxtablecontinued{Continued on next page}}}\\
\endfoot

\endlastfoot

state\_au\_1
&
au
&
Australian Capital Territory
&
ACT
\\
\hline
state\_au\_2
&
au
&
New South Wales
&
NSW
\\
\hline
state\_au\_3
&
au
&
Northern Territory
&
NT
\\
\hline
state\_au\_4
&
au
&
Queensland
&
QLD
\\
\hline
state\_au\_5
&
au
&
South Australia
&
SA
\\
\hline
state\_au\_6
&
au
&
Tasmania
&
TAS
\\
\hline
state\_au\_7
&
au
&
Victoria
&
VIC
\\
\hline
state\_au\_8
&
au
&
Western Australia
&
WA
\\
\hline
state\_us\_1
&
us
&
Alabama
&
AL
\\
\hline
state\_us\_2
&
us
&
Alaska
&
AK
\\
\hline
state\_us\_3
&
us
&
Arizona
&
AZ
\\
\hline
state\_us\_4
&
us
&
Arkansas
&
AR
\\
\hline
state\_us\_5
&
us
&
California
&
CA
\\
\hline
state\_us\_6
&
us
&
Colorado
&
CO
\\
\hline
state\_us\_7
&
us
&
Connecticut
&
CT
\\
\hline
state\_us\_8
&
us
&
Delaware
&
DE
\\
\hline
state\_us\_9
&
us
&
District of Columbia
&
DC
\\
\hline
state\_us\_10
&
us
&
Florida
&
FL
\\
\hline
state\_us\_11
&
us
&
Georgia
&
GA
\\
\hline
state\_us\_12
&
us
&
Hawaii
&
HI
\\
\hline
state\_us\_13
&
us
&
Idaho
&
ID
\\
\hline
state\_us\_14
&
us
&
Illinois
&
IL
\\
\hline
state\_us\_15
&
us
&
Indiana
&
IN
\\
\hline
state\_us\_16
&
us
&
Iowa
&
IA
\\
\hline
state\_us\_17
&
us
&
Kansas
&
KS
\\
\hline
state\_us\_18
&
us
&
Kentucky
&
KY
\\
\hline
state\_us\_19
&
us
&
Louisiana
&
LA
\\
\hline
state\_us\_20
&
us
&
Maine
&
ME
\\
\hline
state\_us\_21
&
us
&
Montana
&
MT
\\
\hline
state\_us\_22
&
us
&
Nebraska
&
NE
\\
\hline
state\_us\_23
&
us
&
Nevada
&
NV
\\
\hline
state\_us\_24
&
us
&
New Hampshire
&
NH
\\
\hline
state\_us\_25
&
us
&
New Jersey
&
NJ
\\
\hline
state\_us\_26
&
us
&
New Mexico
&
NM
\\
\hline
state\_us\_27
&
us
&
New York
&
NY
\\
\hline
state\_us\_28
&
us
&
North Carolina
&
NC
\\
\hline
state\_us\_29
&
us
&
North Dakota
&
ND
\\
\hline
state\_us\_30
&
us
&
Ohio
&
OH
\\
\hline
state\_us\_31
&
us
&
Oklahoma
&
OK
\\
\hline
state\_us\_32
&
us
&
Oregon
&
OR
\\
\hline
state\_us\_33
&
us
&
Maryland
&
MD
\\
\hline
state\_us\_34
&
us
&
Massachusetts
&
MA
\\
\hline
state\_us\_35
&
us
&
Michigan
&
MI
\\
\hline
state\_us\_36
&
us
&
Minnesota
&
MN
\\
\hline
state\_us\_37
&
us
&
Mississippi
&
MS
\\
\hline
state\_us\_38
&
us
&
Missouri
&
MO
\\
\hline
state\_us\_39
&
us
&
Pennsylvania
&
PA
\\
\hline
state\_us\_40
&
us
&
Rhode Island
&
RI
\\
\hline
state\_us\_41
&
us
&
South Carolina
&
SC
\\
\hline
state\_us\_42
&
us
&
South Dakota
&
SD
\\
\hline
state\_us\_43
&
us
&
Tennessee
&
TN
\\
\hline
state\_us\_44
&
us
&
Texas
&
TX
\\
\hline
state\_us\_45
&
us
&
Utah
&
UT
\\
\hline
state\_us\_46
&
us
&
Vermont
&
VT
\\
\hline
state\_us\_47
&
us
&
Virginia
&
VA
\\
\hline
state\_us\_48
&
us
&
Washington
&
WA
\\
\hline
state\_us\_49
&
us
&
West Virginia
&
WV
\\
\hline
state\_us\_50
&
us
&
Wisconsin
&
WI
\\
\hline
state\_us\_51
&
us
&
Wyoming
&
WY
\\
\hline
state\_us\_as
&
us
&
American Samoa
&
AS
\\
\hline
state\_us\_fm
&
us
&
Federated States of Micronesia
&
FM
\\
\hline
state\_us\_gu
&
us
&
Guam
&
GU
\\
\hline
state\_us\_mh
&
us
&
Marshall Islands
&
MH
\\
\hline
state\_us\_mp
&
us
&
Northern Mariana Islands
&
MP
\\
\hline
state\_us\_pw
&
us
&
Palau
&
PW
\\
\hline
state\_us\_pr
&
us
&
Puerto Rico
&
PR
\\
\hline
state\_us\_vi
&
us
&
Virgin Islands
&
VI
\\
\hline
state\_us\_aa
&
us
&
Armed Forces Americas
&
AA
\\
\hline
state\_us\_ae
&
us
&
Armed Forces Europe
&
AE
\\
\hline
state\_us\_ap
&
us
&
Armed Forces Pacific
&
AP
\\
\hline
state\_br\_ac
&
br
&
Acre
&
AC
\\
\hline
state\_br\_al
&
br
&
Alagoas
&
AL
\\
\hline
state\_br\_ap
&
br
&
Amapá
&
AP
\\
\hline
state\_br\_am
&
br
&
Amazonas
&
AM
\\
\hline
state\_br\_ba
&
br
&
Bahia
&
BA
\\
\hline
state\_br\_ce
&
br
&
Ceará
&
CE
\\
\hline
state\_br\_df
&
br
&
Distrito Federal
&
DF
\\
\hline
state\_br\_es
&
br
&
Espírito Santo
&
ES
\\
\hline
state\_br\_go
&
br
&
Goiás
&
GO
\\
\hline
state\_br\_ma
&
br
&
Maranhão
&
MA
\\
\hline
state\_br\_mt
&
br
&
Mato Grosso
&
MT
\\
\hline
state\_br\_ms
&
br
&
Mato Grosso do Sul
&
MS
\\
\hline
state\_br\_mg
&
br
&
Minas Gerais
&
MG
\\
\hline
state\_br\_pa
&
br
&
Pará
&
PA
\\
\hline
state\_br\_pb
&
br
&
Paraíba
&
PB
\\
\hline
state\_br\_pr
&
br
&
Paraná
&
PR
\\
\hline
state\_br\_pe
&
br
&
Pernambuco
&
PE
\\
\hline
state\_br\_pi
&
br
&
Piauí
&
PI
\\
\hline
state\_br\_rj
&
br
&
Rio de Janeiro
&
RJ
\\
\hline
state\_br\_rn
&
br
&
Rio Grande do Norte
&
RN
\\
\hline
state\_br\_rs
&
br
&
Rio Grande do Sul
&
RS
\\
\hline
state\_br\_ro
&
br
&
Rondônia
&
RO
\\
\hline
state\_br\_rr
&
br
&
Roraima
&
RR
\\
\hline
state\_br\_sc
&
br
&
Santa Catarina
&
SC
\\
\hline
state\_br\_sp
&
br
&
São Paulo
&
SP
\\
\hline
state\_br\_se
&
br
&
Sergipe
&
SE
\\
\hline
state\_br\_to
&
br
&
Tocantins
&
TO
\\
\hline
state\_ru\_ad
&
ru
&
Republic of Adygeya
&
AD
\\
\hline
state\_ru\_al
&
ru
&
Altai Republic
&
AL
\\
\hline
state\_ru\_alt
&
ru
&
Altai Krai
&
ALT
\\
\hline
state\_ru\_amu
&
ru
&
Amur Oblast
&
AMU
\\
\hline
state\_ru\_ark
&
ru
&
Arkhangelsk Oblast
&
ARK
\\
\hline
state\_ru\_ast
&
ru
&
Astrakhan Oblast
&
AST
\\
\hline
state\_ru\_ba
&
ru
&
Republic of Bashkortostan
&
BA
\\
\hline
state\_ru\_bel
&
ru
&
Belgorod Oblast
&
BEL
\\
\hline
state\_ru\_bry
&
ru
&
Bryansk Oblast
&
BRY
\\
\hline
state\_ru\_bu
&
ru
&
Republic of Buryatia
&
BU
\\
\hline
state\_ru\_ce
&
ru
&
Chechen Republic
&
CE
\\
\hline
state\_ru\_che
&
ru
&
Chelyabinsk Oblast
&
CHE
\\
\hline
state\_ru\_chu
&
ru
&
Chukotka Autonomous Okrug
&
CHU
\\
\hline
state\_ru\_cu
&
ru
&
Chuvash Republic
&
CU
\\
\hline
state\_ru\_da
&
ru
&
Republic of Dagestan
&
DA
\\
\hline
state\_ru\_in
&
ru
&
Republic of Ingushetia
&
IN
\\
\hline
state\_ru\_irk
&
ru
&
Irkutsk Oblast
&
IRK
\\
\hline
state\_ru\_iva
&
ru
&
Ivanovo Oblast
&
IVA
\\
\hline
state\_ru\_kam
&
ru
&
Kamchatka Krai
&
KAM
\\
\hline
state\_ru\_kb
&
ru
&
Kabardino-Balkarian Republic
&
KB
\\
\hline
state\_ru\_kgd
&
ru
&
Kaliningrad Oblast
&
KGD
\\
\hline
state\_ru\_kl
&
ru
&
Republic of Kalmykia
&
KL
\\
\hline
state\_ru\_klu
&
ru
&
Kaluga Oblast
&
KLU
\\
\hline
state\_ru\_kc
&
ru
&
Karachay\textendash{}Cherkess Republic
&
KC
\\
\hline
state\_ru\_kr
&
ru
&
Republic of Karelia
&
KR
\\
\hline
state\_ru\_kem
&
ru
&
Kemerovo Oblast
&
KEM
\\
\hline
state\_ru\_kha
&
ru
&
Khabarovsk Krai
&
KHA
\\
\hline
state\_ru\_kk
&
ru
&
Republic of Khakassia
&
KK
\\
\hline
state\_ru\_khm
&
ru
&
Khanty-Mansi Autonomous Okrug
&
KHM
\\
\hline
state\_ru\_kir
&
ru
&
Kirov Oblast
&
KIR
\\
\hline
state\_ru\_ko
&
ru
&
Komi Republic
&
KO
\\
\hline
state\_ru\_kos
&
ru
&
Kostroma Oblast
&
KOS
\\
\hline
state\_ru\_kda
&
ru
&
Krasnodar Krai
&
KDA
\\
\hline
state\_ru\_kya
&
ru
&
Krasnoyarsk Krai
&
KYA
\\
\hline
state\_ru\_kgn
&
ru
&
Kurgan Oblast
&
KGN
\\
\hline
state\_ru\_krs
&
ru
&
Kursk Oblast
&
KRS
\\
\hline
state\_ru\_len
&
ru
&
Leningrad Oblast
&
LEN
\\
\hline
state\_ru\_lip
&
ru
&
Lipetsk Oblast
&
LIP
\\
\hline
state\_ru\_mag
&
ru
&
Magadan Oblast
&
MAG
\\
\hline
state\_ru\_me
&
ru
&
Mari El Republic
&
ME
\\
\hline
state\_ru\_mo
&
ru
&
Republic of Mordovia
&
MO
\\
\hline
state\_ru\_mos
&
ru
&
Moscow Oblast
&
MOS
\\
\hline
state\_ru\_mow
&
ru
&
Moscow
&
MOW
\\
\hline
state\_ru\_mur
&
ru
&
Murmansk Oblast
&
MUR
\\
\hline
state\_ru\_niz
&
ru
&
Nizhny Novgorod Oblast
&
NIZ
\\
\hline
state\_ru\_ngr
&
ru
&
Novgorod Oblast
&
NGR
\\
\hline
state\_ru\_nvs
&
ru
&
Novosibirsk Oblast
&
NVS
\\
\hline
state\_ru\_oms
&
ru
&
Omsk Oblast
&
OMS
\\
\hline
state\_ru\_ore
&
ru
&
Orenburg Oblast
&
ORE
\\
\hline
state\_ru\_orl
&
ru
&
Oryol Oblast
&
ORL
\\
\hline
state\_ru\_pnz
&
ru
&
Penza Oblast
&
PNZ
\\
\hline
state\_ru\_per
&
ru
&
Perm Krai
&
PER
\\
\hline
state\_ru\_pri
&
ru
&
Primorsky Krai
&
PRI
\\
\hline
state\_ru\_psk
&
ru
&
Pskov Oblast
&
PSK
\\
\hline
state\_ru\_ros
&
ru
&
Rostov Oblast
&
ROS
\\
\hline
state\_ru\_rya
&
ru
&
Ryazan Oblast
&
RYA
\\
\hline
state\_ru\_sa
&
ru
&
Sakha Republic (Yakutia)
&
SA
\\
\hline
state\_ru\_sak
&
ru
&
Sakhalin Oblast
&
SAK
\\
\hline
state\_ru\_sam
&
ru
&
Samara Oblast
&
SAM
\\
\hline
state\_ru\_spe
&
ru
&
Saint Petersburg
&
SPE
\\
\hline
state\_ru\_sar
&
ru
&
Saratov Oblast
&
SAR
\\
\hline
state\_ru\_se
&
ru
&
Republic of North Ossetia\textendash{}Alania
&
SE
\\
\hline
state\_ru\_smo
&
ru
&
Smolensk Oblast
&
SMO
\\
\hline
state\_ru\_sta
&
ru
&
Stavropol Krai
&
STA
\\
\hline
state\_ru\_sve
&
ru
&
Sverdlovsk Oblast
&
SVE
\\
\hline
state\_ru\_tam
&
ru
&
Tambov Oblast
&
TAM
\\
\hline
state\_ru\_ta
&
ru
&
Republic of Tatarstan
&
TA
\\
\hline
state\_ru\_tom
&
ru
&
Tomsk Oblast
&
TOM
\\
\hline
state\_ru\_tul
&
ru
&
Tula Oblast
&
TUL
\\
\hline
state\_ru\_tve
&
ru
&
Tver Oblast
&
TVE
\\
\hline
state\_ru\_tyu
&
ru
&
Tyumen Oblast
&
TYU
\\
\hline
state\_ru\_ty
&
ru
&
Tyva Republic
&
TY
\\
\hline
state\_ru\_ud
&
ru
&
Udmurtia
&
UD
\\
\hline
state\_ru\_uly
&
ru
&
Ulyanovsk Oblast
&
ULY
\\
\hline
state\_ru\_vla
&
ru
&
Vladimir Oblast
&
VLA
\\
\hline
state\_ru\_vgg
&
ru
&
Volgograd Oblast
&
VGG
\\
\hline
state\_ru\_vlg
&
ru
&
Vologda Oblast
&
VLG
\\
\hline
state\_ru\_vor
&
ru
&
Voronezh Oblast
&
VOR
\\
\hline
state\_ru\_yan
&
ru
&
Yamalo-Nenets Autonomous Okrug
&
YAN
\\
\hline
state\_ru\_yar
&
ru
&
Yaroslavl Oblast
&
YAR
\\
\hline
state\_ru\_yev
&
ru
&
Jewish Autonomous Oblast
&
YEV
\\
\hline
state\_gt\_ave
&
gt
&
Alta Verapaz
&
AVE
\\
\hline
state\_gt\_bve
&
gt
&
Baja Verapaz
&
BVE
\\
\hline
state\_gt\_cmt
&
gt
&
Chimaltenango
&
CMT
\\
\hline
state\_gt\_cqm
&
gt
&
Chiquimula
&
CQM
\\
\hline
state\_gt\_epr
&
gt
&
El Progreso
&
EPR
\\
\hline
state\_gt\_esc
&
gt
&
Escuintla
&
ESC
\\
\hline
state\_gt\_gua
&
gt
&
Guatemala
&
GUA
\\
\hline
state\_gt\_hue
&
gt
&
Huehuetenango
&
HUE
\\
\hline
state\_gt\_iza
&
gt
&
Izabal
&
IZA
\\
\hline
state\_gt\_jal
&
gt
&
Jalapa
&
JAL
\\
\hline
state\_gt\_jut
&
gt
&
Jutiapa
&
JUT
\\
\hline
state\_gt\_pet
&
gt
&
Petén
&
PET
\\
\hline
state\_gt\_que
&
gt
&
Quetzaltenango
&
QUE
\\
\hline
state\_gt\_qui
&
gt
&
Quiché
&
QUI
\\
\hline
state\_gt\_ret
&
gt
&
Retalhuleu
&
RET
\\
\hline
state\_gt\_sac
&
gt
&
Sacatepéquez
&
SAC
\\
\hline
state\_gt\_sma
&
gt
&
San Marcos
&
SMA
\\
\hline
state\_gt\_sro
&
gt
&
Santa Rosa
&
SRO
\\
\hline
state\_gt\_sol
&
gt
&
Sololá
&
SOL
\\
\hline
state\_gt\_suc
&
gt
&
Suchitepéquez
&
SUC
\\
\hline
state\_gt\_tot
&
gt
&
Totonicapán
&
TOT
\\
\hline
state\_gt\_zac
&
gt
&
Zacapa
&
ZAC
\\
\hline
state\_jp\_jp-23
&
jp
&
Aichi
&
23
\\
\hline
state\_jp\_jp-05
&
jp
&
Akita
&
05
\\
\hline
state\_jp\_jp-02
&
jp
&
Aomori
&
02
\\
\hline
state\_jp\_jp-12
&
jp
&
Chiba
&
12
\\
\hline
state\_jp\_jp-38
&
jp
&
Ehime
&
38
\\
\hline
state\_jp\_jp-18
&
jp
&
Fukui
&
18
\\
\hline
state\_jp\_jp-40
&
jp
&
Fukuoka
&
40
\\
\hline
state\_jp\_jp-07
&
jp
&
Fukushima
&
07
\\
\hline
state\_jp\_jp-21
&
jp
&
Gifu
&
21
\\
\hline
state\_jp\_jp-10
&
jp
&
Gunma
&
10
\\
\hline
state\_jp\_jp-34
&
jp
&
Hiroshima
&
34
\\
\hline
state\_jp\_jp-01
&
jp
&
Hokkaidō
&
01
\\
\hline
state\_jp\_jp-28
&
jp
&
Hyōgo
&
28
\\
\hline
state\_jp\_jp-08
&
jp
&
Ibaraki
&
08
\\
\hline
state\_jp\_jp-17
&
jp
&
Ishikawa
&
17
\\
\hline
state\_jp\_jp-03
&
jp
&
Iwate
&
03
\\
\hline
state\_jp\_jp-37
&
jp
&
Kagawa
&
37
\\
\hline
state\_jp\_jp-46
&
jp
&
Kagoshima
&
46
\\
\hline
state\_jp\_jp-14
&
jp
&
Kanagawa
&
14
\\
\hline
state\_jp\_jp-39
&
jp
&
Kōchi
&
39
\\
\hline
state\_jp\_jp-43
&
jp
&
Kumamoto
&
43
\\
\hline
state\_jp\_jp-26
&
jp
&
Kyōto
&
26
\\
\hline
state\_jp\_jp-24
&
jp
&
Mie
&
24
\\
\hline
state\_jp\_jp-04
&
jp
&
Miyagi
&
04
\\
\hline
state\_jp\_jp-45
&
jp
&
Miyazaki
&
45
\\
\hline
state\_jp\_jp-20
&
jp
&
Nagano
&
20
\\
\hline
state\_jp\_jp-42
&
jp
&
Nagasaki
&
42
\\
\hline
state\_jp\_jp-29
&
jp
&
Nara
&
29
\\
\hline
state\_jp\_jp-15
&
jp
&
Niigata
&
15
\\
\hline
state\_jp\_jp-44
&
jp
&
Ōita
&
44
\\
\hline
state\_jp\_jp-33
&
jp
&
Okayama
&
33
\\
\hline
state\_jp\_jp-47
&
jp
&
Okinawa
&
47
\\
\hline
state\_jp\_jp-27
&
jp
&
Ōsaka
&
27
\\
\hline
state\_jp\_jp-41
&
jp
&
Saga
&
41
\\
\hline
state\_jp\_jp-11
&
jp
&
Saitama
&
11
\\
\hline
state\_jp\_jp-25
&
jp
&
Shiga
&
25
\\
\hline
state\_jp\_jp-32
&
jp
&
Shimane
&
32
\\
\hline
state\_jp\_jp-22
&
jp
&
Shizuoka
&
22
\\
\hline
state\_jp\_jp-09
&
jp
&
Tochigi
&
09
\\
\hline
state\_jp\_jp-36
&
jp
&
Tokushima
&
36
\\
\hline
state\_jp\_jp-31
&
jp
&
Tottori
&
31
\\
\hline
state\_jp\_jp-16
&
jp
&
Toyama
&
16
\\
\hline
state\_jp\_jp-13
&
jp
&
Tōkyō
&
13
\\
\hline
state\_jp\_jp-30
&
jp
&
Wakayama
&
30
\\
\hline
state\_jp\_jp-06
&
jp
&
Yamagata
&
06
\\
\hline
state\_jp\_jp-35
&
jp
&
Yamaguchi
&
35
\\
\hline
state\_jp\_jp-19
&
jp
&
Yamanashi
&
19
\\
\hline
state\_pt\_pt-01
&
pt
&
Aveiro
&
01
\\
\hline
state\_pt\_pt-02
&
pt
&
Beja
&
02
\\
\hline
state\_pt\_pt-03
&
pt
&
Braga
&
03
\\
\hline
state\_pt\_pt-04
&
pt
&
Bragança
&
04
\\
\hline
state\_pt\_pt-05
&
pt
&
Castelo Branco
&
05
\\
\hline
state\_pt\_pt-06
&
pt
&
Coimbra
&
06
\\
\hline
state\_pt\_pt-07
&
pt
&
Évora
&
07
\\
\hline
state\_pt\_pt-08
&
pt
&
Faro
&
08
\\
\hline
state\_pt\_pt-09
&
pt
&
Guarda
&
09
\\
\hline
state\_pt\_pt-10
&
pt
&
Leiria
&
10
\\
\hline
state\_pt\_pt-11
&
pt
&
Lisboa
&
11
\\
\hline
state\_pt\_pt-12
&
pt
&
Portalegre
&
12
\\
\hline
state\_pt\_pt-13
&
pt
&
Porto
&
13
\\
\hline
state\_pt\_pt-14
&
pt
&
Santarém
&
14
\\
\hline
state\_pt\_pt-15
&
pt
&
Setúbal
&
15
\\
\hline
state\_pt\_pt-16
&
pt
&
Viana do Castelo
&
16
\\
\hline
state\_pt\_pt-17
&
pt
&
Vila Real
&
17
\\
\hline
state\_pt\_pt-18
&
pt
&
Viseu
&
18
\\
\hline
state\_pt\_pt-20
&
pt
&
Açores
&
20
\\
\hline
state\_pt\_pt-30
&
pt
&
Madeira
&
30
\\
\hline
state\_eg\_dk
&
eg
&
Dakahlia
&
DK
\\
\hline
state\_eg\_ba
&
eg
&
Red Sea
&
BA
\\
\hline
state\_eg\_bh
&
eg
&
Beheira
&
BH
\\
\hline
state\_eg\_fym
&
eg
&
Faiyum
&
FYM
\\
\hline
state\_eg\_gh
&
eg
&
Gharbia
&
GH
\\
\hline
state\_eg\_alx
&
eg
&
Alexandria
&
ALX
\\
\hline
state\_eg\_is
&
eg
&
Ismailia
&
IS
\\
\hline
state\_eg\_gz
&
eg
&
Giza
&
GZ
\\
\hline
state\_eg\_mnf
&
eg
&
Monufia
&
MNF
\\
\hline
state\_eg\_mn
&
eg
&
Minya
&
MN
\\
\hline
state\_eg\_c
&
eg
&
Cairo
&
C
\\
\hline
state\_eg\_kb
&
eg
&
Qalyubia
&
KB
\\
\hline
state\_eg\_lx
&
eg
&
Luxor
&
LX
\\
\hline
state\_eg\_wad
&
eg
&
New Valley
&
WAD
\\
\hline
state\_eg\_shr
&
eg
&
Al Sharqia
&
SHR
\\
\hline
state\_eg\_su
&
eg
&
6th of October
&
SU
\\
\hline
state\_eg\_suz
&
eg
&
Suez
&
SUZ
\\
\hline
state\_eg\_asn
&
eg
&
Aswan
&
ASN
\\
\hline
state\_eg\_ast
&
eg
&
Asyut
&
AST
\\
\hline
state\_eg\_bns
&
eg
&
Beni Suef
&
BNS
\\
\hline
state\_eg\_pts
&
eg
&
Port Said
&
PTS
\\
\hline
state\_eg\_dt
&
eg
&
Damietta
&
DT
\\
\hline
state\_eg\_hu
&
eg
&
Helwan
&
HU
\\
\hline
state\_eg\_js
&
eg
&
South Sinai
&
JS
\\
\hline
state\_eg\_kfs
&
eg
&
Kafr el-Sheikh
&
KFS
\\
\hline
state\_eg\_mt
&
eg
&
Matrouh
&
MT
\\
\hline
state\_eg\_kn
&
eg
&
Qena
&
KN
\\
\hline
state\_eg\_sin
&
eg
&
North Sinai
&
SIN
\\
\hline
state\_eg\_shg
&
eg
&
Sohag
&
SHG
\\
\hline
state\_za\_ec
&
za
&
Eastern Cape
&
EC
\\
\hline
state\_za\_fs
&
za
&
Free State
&
FS
\\
\hline
state\_za\_gt
&
za
&
Gauteng
&
GT
\\
\hline
state\_za\_nl
&
za
&
KwaZulu-Natal
&
NL
\\
\hline
state\_za\_lp
&
za
&
Limpopo
&
LP
\\
\hline
state\_za\_mp
&
za
&
Mpumalanga
&
MP
\\
\hline
state\_za\_nc
&
za
&
Northern Cape
&
NC
\\
\hline
state\_za\_nw
&
za
&
North West
&
NW
\\
\hline
state\_za\_wc
&
za
&
Western Cape
&
WC
\\
\hline
state\_it\_ag
&
it
&
Agrigento
&
AG
\\
\hline
state\_it\_al
&
it
&
Alessandria
&
AL
\\
\hline
state\_it\_an
&
it
&
Ancona
&
AN
\\
\hline
state\_it\_ao
&
it
&
Aosta
&
AO
\\
\hline
state\_it\_ar
&
it
&
Arezzo
&
AR
\\
\hline
state\_it\_ap
&
it
&
Ascoli Piceno
&
AP
\\
\hline
state\_it\_at
&
it
&
Asti
&
AT
\\
\hline
state\_it\_av
&
it
&
Avellino
&
AV
\\
\hline
state\_it\_ba
&
it
&
Bari
&
BA
\\
\hline
state\_it\_bt
&
it
&
Barletta-Andria-Trani
&
BT
\\
\hline
state\_it\_bl
&
it
&
Belluno
&
BL
\\
\hline
state\_it\_bn
&
it
&
Benevento
&
BN
\\
\hline
state\_it\_bg
&
it
&
Bergamo
&
BG
\\
\hline
state\_it\_bi
&
it
&
Biella
&
BI
\\
\hline
state\_it\_bo
&
it
&
Bologna
&
BO
\\
\hline
state\_it\_bz
&
it
&
Bolzano
&
BZ
\\
\hline
state\_it\_bs
&
it
&
Brescia
&
BS
\\
\hline
state\_it\_br
&
it
&
Brindisi
&
BR
\\
\hline
state\_it\_ca
&
it
&
Cagliari
&
CA
\\
\hline
state\_it\_cl
&
it
&
Caltanissetta
&
CL
\\
\hline
state\_it\_cb
&
it
&
Campobasso
&
CB
\\
\hline
state\_it\_ci
&
it
&
Carbonia-Iglesias
&
CI
\\
\hline
state\_it\_ce
&
it
&
Caserta
&
CE
\\
\hline
state\_it\_ct
&
it
&
Catania
&
CT
\\
\hline
state\_it\_cz
&
it
&
Catanzaro
&
CZ
\\
\hline
state\_it\_ch
&
it
&
Chieti
&
CH
\\
\hline
state\_it\_co
&
it
&
Como
&
CO
\\
\hline
state\_it\_cs
&
it
&
Cosenza
&
CS
\\
\hline
state\_it\_cr
&
it
&
Cremona
&
CR
\\
\hline
state\_it\_kr
&
it
&
Crotone
&
KR
\\
\hline
state\_it\_cn
&
it
&
Cuneo
&
CN
\\
\hline
state\_it\_en
&
it
&
Enna
&
EN
\\
\hline
state\_it\_fm
&
it
&
Fermo
&
FM
\\
\hline
state\_it\_fe
&
it
&
Ferrara
&
FE
\\
\hline
state\_it\_fi
&
it
&
Firenze
&
FI
\\
\hline
state\_it\_fg
&
it
&
Foggia
&
FG
\\
\hline
state\_it\_fc
&
it
&
Forlì-Cesena
&
FC
\\
\hline
state\_it\_fr
&
it
&
Frosinone
&
FR
\\
\hline
state\_it\_ge
&
it
&
Genova
&
GE
\\
\hline
state\_it\_go
&
it
&
Gorizia
&
GO
\\
\hline
state\_it\_gr
&
it
&
Grosseto
&
GR
\\
\hline
state\_it\_im
&
it
&
Imperia
&
IM
\\
\hline
state\_it\_is
&
it
&
Isernia
&
IS
\\
\hline
state\_it\_sp
&
it
&
La Spezia
&
SP
\\
\hline
state\_it\_aq
&
it
&
L’Aquila
&
AQ
\\
\hline
state\_it\_lt
&
it
&
Latina
&
LT
\\
\hline
state\_it\_le
&
it
&
Lecce
&
LE
\\
\hline
state\_it\_lc
&
it
&
Lecco
&
LC
\\
\hline
state\_it\_li
&
it
&
Livorno
&
LI
\\
\hline
state\_it\_lo
&
it
&
Lodi
&
LO
\\
\hline
state\_it\_lu
&
it
&
Lucca
&
LU
\\
\hline
state\_it\_mc
&
it
&
Macerata
&
MC
\\
\hline
state\_it\_mn
&
it
&
Mantova
&
MN
\\
\hline
state\_it\_ms
&
it
&
Massa-Carrara
&
MS
\\
\hline
state\_it\_mt
&
it
&
Matera
&
MT
\\
\hline
state\_it\_vs
&
it
&
Medio Campidano
&
VS
\\
\hline
state\_it\_me
&
it
&
Messina
&
ME
\\
\hline
state\_it\_mi
&
it
&
Milano
&
MI
\\
\hline
state\_it\_mo
&
it
&
Modena
&
MO
\\
\hline
state\_it\_mb
&
it
&
Monza e Brianza
&
MB
\\
\hline
state\_it\_na
&
it
&
Napoli
&
NA
\\
\hline
state\_it\_no
&
it
&
Novara
&
NO
\\
\hline
state\_it\_nu
&
it
&
Nuoro
&
NU
\\
\hline
state\_it\_og
&
it
&
Ogliastra
&
OG
\\
\hline
state\_it\_ot
&
it
&
Olbia-Tempio
&
OT
\\
\hline
state\_it\_or
&
it
&
Oristano
&
OR
\\
\hline
state\_it\_pd
&
it
&
Padova
&
PD
\\
\hline
state\_it\_pa
&
it
&
Palermo
&
PA
\\
\hline
state\_it\_pr
&
it
&
Parma
&
PR
\\
\hline
state\_it\_pv
&
it
&
Pavia
&
PV
\\
\hline
state\_it\_pg
&
it
&
Perugia
&
PG
\\
\hline
state\_it\_pu
&
it
&
Pesaro e Urbino
&
PU
\\
\hline
state\_it\_pe
&
it
&
Pescara
&
PE
\\
\hline
state\_it\_pc
&
it
&
Piacenza
&
PC
\\
\hline
state\_it\_pi
&
it
&
Pisa
&
PI
\\
\hline
state\_it\_pt
&
it
&
Pistoia
&
PT
\\
\hline
state\_it\_pn
&
it
&
Pordenone
&
PN
\\
\hline
state\_it\_pz
&
it
&
Potenza
&
PZ
\\
\hline
state\_it\_po
&
it
&
Prato
&
PO
\\
\hline
state\_it\_rg
&
it
&
Ragusa
&
RG
\\
\hline
state\_it\_ra
&
it
&
Ravenna
&
RA
\\
\hline
state\_it\_rc
&
it
&
Reggio Calabria
&
RC
\\
\hline
state\_it\_re
&
it
&
Reggio Emilia
&
RE
\\
\hline
state\_it\_ri
&
it
&
Rieti
&
RI
\\
\hline
state\_it\_rn
&
it
&
Rimini
&
RN
\\
\hline
state\_it\_rm
&
it
&
Roma
&
RM
\\
\hline
state\_it\_ro
&
it
&
Rovigo
&
RO
\\
\hline
state\_it\_sa
&
it
&
Salerno
&
SA
\\
\hline
state\_it\_ss
&
it
&
Sassari
&
SS
\\
\hline
state\_it\_sv
&
it
&
Savona
&
SV
\\
\hline
state\_it\_si
&
it
&
Siena
&
SI
\\
\hline
state\_it\_sr
&
it
&
Siracusa
&
SR
\\
\hline
state\_it\_so
&
it
&
Sondrio
&
SO
\\
\hline
state\_it\_ta
&
it
&
Taranto
&
TA
\\
\hline
state\_it\_te
&
it
&
Teramo
&
TE
\\
\hline
state\_it\_tr
&
it
&
Terni
&
TR
\\
\hline
state\_it\_to
&
it
&
Torino
&
TO
\\
\hline
state\_it\_tp
&
it
&
Trapani
&
TP
\\
\hline
state\_it\_tn
&
it
&
Trento
&
TN
\\
\hline
state\_it\_tv
&
it
&
Treviso
&
TV
\\
\hline
state\_it\_ts
&
it
&
Trieste
&
TS
\\
\hline
state\_it\_ud
&
it
&
Udine
&
UD
\\
\hline
state\_it\_va
&
it
&
Varese
&
VA
\\
\hline
state\_it\_ve
&
it
&
Venezia
&
VE
\\
\hline
state\_it\_vb
&
it
&
Verbano-Cusio-Ossola
&
VB
\\
\hline
state\_it\_vc
&
it
&
Vercelli
&
VC
\\
\hline
state\_it\_vr
&
it
&
Verona
&
VR
\\
\hline
state\_it\_vv
&
it
&
Vibo Valentia
&
VV
\\
\hline
state\_it\_vi
&
it
&
Vicenza
&
VI
\\
\hline
state\_it\_vt
&
it
&
Viterbo
&
VT
\\
\hline
state\_es\_c
&
es
&
A Coruña (La Coruña)
&
C
\\
\hline
state\_es\_vi
&
es
&
Araba/Álava
&
VI
\\
\hline
state\_es\_ab
&
es
&
Albacete
&
AB
\\
\hline
state\_es\_a
&
es
&
Alacant (Alicante)
&
A
\\
\hline
state\_es\_al
&
es
&
Almería
&
AL
\\
\hline
state\_es\_o
&
es
&
Asturias
&
O
\\
\hline
state\_es\_av
&
es
&
Ávila
&
AV
\\
\hline
state\_es\_ba
&
es
&
Badajoz
&
BA
\\
\hline
state\_es\_pm
&
es
&
Illes Balears (Islas Baleares)
&
PM
\\
\hline
state\_es\_b
&
es
&
Barcelona
&
B
\\
\hline
state\_es\_bu
&
es
&
Burgos
&
BU
\\
\hline
state\_es\_cc
&
es
&
Cáceres
&
CC
\\
\hline
state\_es\_ca
&
es
&
Cádiz
&
CA
\\
\hline
state\_es\_s
&
es
&
Cantabria
&
S
\\
\hline
state\_es\_cs
&
es
&
Castelló (Castellón)
&
CS
\\
\hline
state\_es\_ce
&
es
&
Ceuta
&
CE
\\
\hline
state\_es\_cr
&
es
&
Ciudad Real
&
CR
\\
\hline
state\_es\_co
&
es
&
Córdoba
&
CO
\\
\hline
state\_es\_cu
&
es
&
Cuenca
&
CU
\\
\hline
state\_es\_gi
&
es
&
Girona (Gerona)
&
GI
\\
\hline
state\_es\_gr
&
es
&
Granada
&
GR
\\
\hline
state\_es\_gu
&
es
&
Guadalajara
&
GU
\\
\hline
state\_es\_ss
&
es
&
Gipuzkoa (Guipúzcoa)
&
SS
\\
\hline
state\_es\_h
&
es
&
Huelva
&
H
\\
\hline
state\_es\_hu
&
es
&
Huesca
&
HU
\\
\hline
state\_es\_j
&
es
&
Jaén
&
J
\\
\hline
state\_es\_lo
&
es
&
La Rioja
&
LO
\\
\hline
state\_es\_gc
&
es
&
Las Palmas
&
GC
\\
\hline
state\_es\_le
&
es
&
León
&
LE
\\
\hline
state\_es\_l
&
es
&
Lleida (Lérida)
&
L
\\
\hline
state\_es\_lu
&
es
&
Lugo
&
LU
\\
\hline
state\_es\_m
&
es
&
Madrid
&
M
\\
\hline
state\_es\_ma
&
es
&
Málaga
&
MA
\\
\hline
state\_es\_ml
&
es
&
Melilla
&
ME
\\
\hline
state\_es\_mu
&
es
&
Murcia
&
MU
\\
\hline
state\_es\_na
&
es
&
Nafarroa (Navarra)
&
NA
\\
\hline
state\_es\_or
&
es
&
Ourense (Orense)
&
OR
\\
\hline
state\_es\_p
&
es
&
Palencia
&
P
\\
\hline
state\_es\_po
&
es
&
Pontevedra
&
PO
\\
\hline
state\_es\_sa
&
es
&
Salamanca
&
SA
\\
\hline
state\_es\_tf
&
es
&
Santa Cruz de Tenerife
&
TF
\\
\hline
state\_es\_sg
&
es
&
Segovia
&
SG
\\
\hline
state\_es\_se
&
es
&
Sevilla
&
SE
\\
\hline
state\_es\_so
&
es
&
Soria
&
SO
\\
\hline
state\_es\_t
&
es
&
Tarragona
&
T
\\
\hline
state\_es\_te
&
es
&
Teruel
&
TE
\\
\hline
state\_es\_to
&
es
&
Toledo
&
TO
\\
\hline
state\_es\_v
&
es
&
València (Valencia)
&
V
\\
\hline
state\_es\_va
&
es
&
Valladolid
&
VA
\\
\hline
state\_es\_bi
&
es
&
Bizkaia (Vizcaya)
&
BI
\\
\hline
state\_es\_za
&
es
&
Zamora
&
ZA
\\
\hline
state\_es\_z
&
es
&
Zaragoza
&
Z
\\
\hline
state\_my\_jhr
&
my
&
Johor
&
JHR
\\
\hline
state\_my\_kdh
&
my
&
Kedah
&
KDH
\\
\hline
state\_my\_ktn
&
my
&
Kelantan
&
KTN
\\
\hline
state\_my\_kul
&
my
&
Kuala Lumpur
&
KUL
\\
\hline
state\_my\_lbn
&
my
&
Labuan
&
LBN
\\
\hline
state\_my\_mlk
&
my
&
Melaka
&
MLK
\\
\hline
state\_my\_nsn
&
my
&
Negeri Sembilan
&
NSN
\\
\hline
state\_my\_phg
&
my
&
Pahang
&
PHG
\\
\hline
state\_my\_prk
&
my
&
Perak
&
PRK
\\
\hline
state\_my\_pls
&
my
&
Perlis
&
PLS
\\
\hline
state\_my\_png
&
my
&
Pulau Pinang
&
PNG
\\
\hline
state\_my\_pjy
&
my
&
Putrajaya
&
PJY
\\
\hline
state\_my\_sbh
&
my
&
Sabah
&
SBH
\\
\hline
state\_my\_swk
&
my
&
Sarawak
&
SWK
\\
\hline
state\_my\_sgr
&
my
&
Selangor
&
SGR
\\
\hline
state\_my\_trg
&
my
&
Terengganu
&
TRG
\\
\hline
state\_mx\_ags
&
mx
&
Aguascalientes
&
AGU
\\
\hline
state\_mx\_bc
&
mx
&
Baja California
&
BCN
\\
\hline
state\_mx\_bcs
&
mx
&
Baja California Sur
&
BCS
\\
\hline
state\_mx\_chih
&
mx
&
Chihuahua
&
CHH
\\
\hline
state\_mx\_col
&
mx
&
Colima
&
COL
\\
\hline
state\_mx\_camp
&
mx
&
Campeche
&
CAM
\\
\hline
state\_mx\_coah
&
mx
&
Coahuila
&
COA
\\
\hline
state\_mx\_chis
&
mx
&
Chiapas
&
CHP
\\
\hline
state\_mx\_df
&
mx
&
Ciudad de México
&
DIF
\\
\hline
state\_mx\_dgo
&
mx
&
Durango
&
DUR
\\
\hline
state\_mx\_gro
&
mx
&
Guerrero
&
GRO
\\
\hline
state\_mx\_gto
&
mx
&
Guanajuato
&
GUA
\\
\hline
state\_mx\_hgo
&
mx
&
Hidalgo
&
HID
\\
\hline
state\_mx\_jal
&
mx
&
Jalisco
&
JAL
\\
\hline
state\_mx\_mich
&
mx
&
Michoacán
&
MIC
\\
\hline
state\_mx\_mor
&
mx
&
Morelos
&
MOR
\\
\hline
state\_mx\_mex
&
mx
&
México
&
MEX
\\
\hline
state\_mx\_nay
&
mx
&
Nayarit
&
NAY
\\
\hline
state\_mx\_nl
&
mx
&
Nuevo León
&
NLE
\\
\hline
state\_mx\_oax
&
mx
&
Oaxaca
&
OAX
\\
\hline
state\_mx\_pue
&
mx
&
Puebla
&
PUE
\\
\hline
state\_mx\_q roo
&
mx
&
Quintana Roo
&
ROO
\\
\hline
state\_mx\_qro
&
mx
&
Querétaro
&
QUE
\\
\hline
state\_mx\_sin
&
mx
&
Sinaloa
&
SIN
\\
\hline
state\_mx\_slp
&
mx
&
San Luis Potosí
&
SLP
\\
\hline
state\_mx\_son
&
mx
&
Sonora
&
SON
\\
\hline
state\_mx\_tab
&
mx
&
Tabasco
&
TAB
\\
\hline
state\_mx\_tlax
&
mx
&
Tlaxcala
&
TLA
\\
\hline
state\_mx\_tamps
&
mx
&
Tamaulipas
&
TAM
\\
\hline
state\_mx\_ver
&
mx
&
Veracruz
&
VER
\\
\hline
state\_mx\_yuc
&
mx
&
Yucatán
&
YUC
\\
\hline
state\_mx\_zac
&
mx
&
Zacatecas
&
ZAC
\\
\hline
state\_nz\_auk
&
nz
&
Auckland
&
AUK
\\
\hline
state\_nz\_bop
&
nz
&
Bay of Plenty
&
BOP
\\
\hline
state\_nz\_can
&
nz
&
Canterbury
&
CAN
\\
\hline
state\_nz\_gis
&
nz
&
Gisborne
&
GIS
\\
\hline
state\_nz\_hkb
&
nz
&
Hawke’s Bay
&
HKB
\\
\hline
state\_nz\_mwt
&
nz
&
Manawatu-Wanganui
&
MWT
\\
\hline
state\_nz\_mbh
&
nz
&
Marlborough
&
MBH
\\
\hline
state\_nz\_nsn
&
nz
&
Nelson
&
NSN
\\
\hline
state\_nz\_ntl
&
nz
&
Northland
&
NTL
\\
\hline
state\_nz\_ota
&
nz
&
Otago
&
OTA
\\
\hline
state\_nz\_stl
&
nz
&
Southland
&
STL
\\
\hline
state\_nz\_tki
&
nz
&
Taranaki
&
TKI
\\
\hline
state\_nz\_tas
&
nz
&
Tasman
&
TAS
\\
\hline
state\_nz\_wko
&
nz
&
Waikato
&
WKO
\\
\hline
state\_nz\_wgn
&
nz
&
Wellington
&
WGN
\\
\hline
state\_nz\_wtc
&
nz
&
West Coast
&
WTC
\\
\hline
state\_ca\_ab
&
ca
&
Alberta
&
AB
\\
\hline
state\_ca\_bc
&
ca
&
British Columbia
&
BC
\\
\hline
state\_ca\_mb
&
ca
&
Manitoba
&
MB
\\
\hline
state\_ca\_nb
&
ca
&
New Brunswick
&
NB
\\
\hline
state\_ca\_nl
&
ca
&
Newfoundland and Labrador
&
NL
\\
\hline
state\_ca\_nt
&
ca
&
Northwest Territories
&
NT
\\
\hline
state\_ca\_ns
&
ca
&
Nova Scotia
&
NS
\\
\hline
state\_ca\_nu
&
ca
&
Nunavut
&
NU
\\
\hline
state\_ca\_on
&
ca
&
Ontario
&
ON
\\
\hline
state\_ca\_pe
&
ca
&
Prince Edward Island
&
PE
\\
\hline
state\_ca\_qc
&
ca
&
Quebec
&
QC
\\
\hline
state\_ca\_sk
&
ca
&
Saskatchewan
&
SK
\\
\hline
state\_ca\_yt
&
ca
&
Yukon
&
YT
\\
\hline
state\_ae\_az
&
ae
&
Abu Dhabi
&
AZ
\\
\hline
state\_ae\_aj
&
ae
&
Ajman
&
AJ
\\
\hline
state\_ae\_du
&
ae
&
Dubai
&
DU
\\
\hline
state\_ae\_fu
&
ae
&
Fujairah
&
FU
\\
\hline
state\_ae\_rk
&
ae
&
Ras al-Khaimah
&
RK
\\
\hline
state\_ae\_sh
&
ae
&
Sharjah
&
SH
\\
\hline
state\_ae\_uq
&
ae
&
Umm al-Quwain
&
UQ
\\
\hline
state\_ar\_c
&
ar
&
Buenos Aires City
&
C
\\
\hline
state\_ar\_b
&
ar
&
Buenos Aires
&
B
\\
\hline
state\_ar\_k
&
ar
&
Catamarca
&
K
\\
\hline
state\_ar\_h
&
ar
&
Chaco
&
H
\\
\hline
state\_ar\_u
&
ar
&
Chobut
&
U
\\
\hline
state\_ar\_x
&
ar
&
Córdoba
&
X
\\
\hline
state\_ar\_w
&
ar
&
Corrientes
&
W
\\
\hline
state\_ar\_e
&
ar
&
Ente Ríos
&
E
\\
\hline
state\_ar\_p
&
ar
&
Formosa
&
P
\\
\hline
state\_ar\_y
&
ar
&
Jujuy
&
Y
\\
\hline
state\_ar\_l
&
ar
&
La Pampa
&
L
\\
\hline
state\_ar\_f
&
ar
&
La Rioja
&
F
\\
\hline
state\_ar\_m
&
ar
&
Mendoza
&
M
\\
\hline
state\_ar\_n
&
ar
&
Misiones
&
N
\\
\hline
state\_ar\_q
&
ar
&
Neuquén
&
Q
\\
\hline
state\_ar\_r
&
ar
&
Río Negro
&
R
\\
\hline
state\_ar\_a
&
ar
&
Salta
&
A
\\
\hline
state\_ar\_j
&
ar
&
San Juan
&
J
\\
\hline
state\_ar\_d
&
ar
&
San Luis
&
D
\\
\hline
state\_ar\_z
&
ar
&
Santa Cruz
&
Z
\\
\hline
state\_ar\_s
&
ar
&
Santa Fe
&
S
\\
\hline
state\_ar\_g
&
ar
&
Santiago Del Estero
&
G
\\
\hline
state\_ar\_v
&
ar
&
Tierra del Fuego
&
V
\\
\hline
state\_ar\_t
&
ar
&
Tucumán
&
T
\\
\hline
state\_in\_an
&
in
&
Andaman and Nicobar
&
AN
\\
\hline
state\_in\_ap
&
in
&
Andhra Pradesh
&
AP
\\
\hline
state\_in\_ar
&
in
&
Arunachal Pradesh
&
AR
\\
\hline
state\_in\_as
&
in
&
Assam
&
AS
\\
\hline
state\_in\_br
&
in
&
Bihar
&
BR
\\
\hline
state\_in\_ch
&
in
&
Chandigarh
&
CH
\\
\hline
state\_in\_cg
&
in
&
Chattisgarh
&
CG
\\
\hline
state\_in\_dn
&
in
&
Dadra and Nagar Haveli
&
DN
\\
\hline
state\_in\_dd
&
in
&
Daman and Diu
&
DD
\\
\hline
state\_in\_dl
&
in
&
Delhi
&
DL
\\
\hline
state\_in\_ga
&
in
&
Goa
&
GA
\\
\hline
state\_in\_gj
&
in
&
Gujarat
&
GJ
\\
\hline
state\_in\_hr
&
in
&
Haryana
&
HR
\\
\hline
state\_in\_hp
&
in
&
Himachal Pradesh
&
HP
\\
\hline
state\_in\_jk
&
in
&
Jammu and Kashmir
&
JK
\\
\hline
state\_in\_jh
&
in
&
Jharkhand
&
JH
\\
\hline
state\_in\_ka
&
in
&
Karnataka
&
KA
\\
\hline
state\_in\_kl
&
in
&
Kerala
&
KL
\\
\hline
state\_in\_ld
&
in
&
Lakshadweep
&
LD
\\
\hline
state\_in\_mp
&
in
&
Madhya Pradesh
&
MP
\\
\hline
state\_in\_mh
&
in
&
Maharashtra
&
MH
\\
\hline
state\_in\_mn
&
in
&
Manipur
&
MN
\\
\hline
state\_in\_ml
&
in
&
Meghalaya
&
ML
\\
\hline
state\_in\_mz
&
in
&
Mizoram
&
MZ
\\
\hline
state\_in\_nl
&
in
&
Nagaland
&
NL
\\
\hline
state\_in\_or
&
in
&
Orissa
&
OR
\\
\hline
state\_in\_py
&
in
&
Puducherry
&
PY
\\
\hline
state\_in\_pb
&
in
&
Punjab
&
PB
\\
\hline
state\_in\_rj
&
in
&
Rajasthan
&
RJ
\\
\hline
state\_in\_sk
&
in
&
Sikkim
&
SK
\\
\hline
state\_in\_tn
&
in
&
Tamil Nadu
&
TN
\\
\hline
state\_in\_ts
&
in
&
Telangana
&
TS
\\
\hline
state\_in\_tr
&
in
&
Tripura
&
TR
\\
\hline
state\_in\_up
&
in
&
Uttar Pradesh
&
UP
\\
\hline
state\_in\_uk
&
in
&
Uttarakhand
&
UK
\\
\hline
state\_in\_wb
&
in
&
West Bengal
&
WB
\\
\hline
state\_id\_ac
&
id
&
Aceh
&
AC
\\
\hline
state\_id\_ba
&
id
&
Bali
&
BA
\\
\hline
state\_id\_bb
&
id
&
Bangka Belitung
&
BB
\\
\hline
state\_id\_bt
&
id
&
Banten
&
BT
\\
\hline
state\_id\_be
&
id
&
Bengkulu
&
BE
\\
\hline
state\_id\_go
&
id
&
Gorontalo
&
GO
\\
\hline
state\_id\_jk
&
id
&
Jakarta
&
JK
\\
\hline
state\_id\_ja
&
id
&
Jambi
&
JA
\\
\hline
state\_id\_jb
&
id
&
Jawa Barat
&
JB
\\
\hline
state\_id\_jt
&
id
&
Jawa Tengah
&
JT
\\
\hline
state\_id\_ji
&
id
&
Jawa Timur
&
JI
\\
\hline
state\_id\_kb
&
id
&
Kalimantan Barat
&
KB
\\
\hline
state\_id\_ks
&
id
&
Kalimantan Selatan
&
KS
\\
\hline
state\_id\_kt
&
id
&
Kalimantan Tengah
&
KT
\\
\hline
state\_id\_ki
&
id
&
Kalimantan Timur
&
KI
\\
\hline
state\_id\_ku
&
id
&
Kalimantan Utara
&
KU
\\
\hline
state\_id\_kr
&
id
&
Kepulauan Riau
&
KR
\\
\hline
state\_id\_la
&
id
&
Lampung
&
LA
\\
\hline
state\_id\_ma
&
id
&
Maluku
&
MA
\\
\hline
state\_id\_mu
&
id
&
Maluku Utara
&
MU
\\
\hline
state\_id\_nb
&
id
&
Nusa Tenggara Barat
&
NB
\\
\hline
state\_id\_nt
&
id
&
Nusa Tenggara Timur
&
NT
\\
\hline
state\_id\_pa
&
id
&
Papua
&
PA
\\
\hline
state\_id\_pb
&
id
&
Papua Barat
&
PB
\\
\hline
state\_id\_ri
&
id
&
Riau
&
RI
\\
\hline
state\_id\_sr
&
id
&
Sulawesi Barat
&
SR
\\
\hline
state\_id\_sn
&
id
&
Sulawesi Selatan
&
SN
\\
\hline
state\_id\_st
&
id
&
Sulawesi Tengah
&
ST
\\
\hline
state\_id\_sg
&
id
&
Sulawesi Tenggara
&
SG
\\
\hline
state\_id\_sa
&
id
&
Sulawesi Utara
&
SA
\\
\hline
state\_id\_sb
&
id
&
Sumatra Barat
&
SB
\\
\hline
state\_id\_ss
&
id
&
Sumatra Selatan
&
SS
\\
\hline
state\_id\_su
&
id
&
Sumatra Utara
&
SU
\\
\hline
state\_id\_yo
&
id
&
Yogyakarta
&
YO
\\
\hline
state\_co\_01
&
co
&
Antioquia
&
ANT
\\
\hline
state\_co\_02
&
co
&
Atlántico
&
ATL
\\
\hline
state\_co\_03
&
co
&
D.C.
&
DC
\\
\hline
state\_co\_04
&
co
&
Bolívar
&
BOL
\\
\hline
state\_co\_05
&
co
&
Boyacá
&
BOY
\\
\hline
state\_co\_06
&
co
&
Caldas
&
CAL
\\
\hline
state\_co\_07
&
co
&
Caquetá
&
CAQ
\\
\hline
state\_co\_08
&
co
&
Cauca
&
CAU
\\
\hline
state\_co\_09
&
co
&
Cesar
&
CES
\\
\hline
state\_co\_10
&
co
&
Córdoba
&
COR
\\
\hline
state\_co\_11
&
co
&
Cundinamarca
&
CUN
\\
\hline
state\_co\_12
&
co
&
Chocó
&
CHO
\\
\hline
state\_co\_13
&
co
&
Huila
&
HUI
\\
\hline
state\_co\_14
&
co
&
La Guajira
&
LAG
\\
\hline
state\_co\_15
&
co
&
Magdalena
&
MAG
\\
\hline
state\_co\_16
&
co
&
Meta
&
MET
\\
\hline
state\_co\_17
&
co
&
Nariño
&
NAR
\\
\hline
state\_co\_18
&
co
&
Norte de Santander
&
NSA
\\
\hline
state\_co\_19
&
co
&
Quindio
&
QUI
\\
\hline
state\_co\_20
&
co
&
Risaralda
&
RIS
\\
\hline
state\_co\_21
&
co
&
Santander
&
SAN
\\
\hline
state\_co\_22
&
co
&
Sucre
&
SUC
\\
\hline
state\_co\_23
&
co
&
Tolima
&
TOL
\\
\hline
state\_co\_24
&
co
&
Valle del Cauca
&
VAC
\\
\hline
state\_co\_25
&
co
&
Arauca
&
ARA
\\
\hline
state\_co\_26
&
co
&
Casanare
&
CAS
\\
\hline
state\_co\_27
&
co
&
Putumayo
&
PUT
\\
\hline
state\_co\_28
&
co
&
Archipiélago de San Andrés, Providencia y Santa Catalina
&
SAP
\\
\hline
state\_co\_29
&
co
&
Amazonas
&
AMA
\\
\hline
state\_co\_30
&
co
&
Guainía
&
GUA
\\
\hline
state\_co\_31
&
co
&
Guaviare
&
GUV
\\
\hline
state\_co\_32
&
co
&
Vaupés
&
VAU
\\
\hline
state\_co\_33
&
co
&
Vichada
&
VID
\\
\hline
\end{longtable}\sphinxatlongtableend\end{savenotes}

For each row (record):
\begin{itemize}
\item {} 
the first column is the \DUrole{xref,std,std-term}{external id} of the record to create or
update

\item {} 
the second column is the \DUrole{xref,std,std-term}{external id} of the country object to link
to (country objects must have been defined beforehand)

\item {} 
the third column is the \sphinxcode{\sphinxupquote{name}} field for \sphinxcode{\sphinxupquote{res.country.state}}

\item {} 
the fourth column is the \sphinxcode{\sphinxupquote{code}} field for \sphinxcode{\sphinxupquote{res.country.state}}

\end{itemize}


\section{Actions}
\label{\detokenize{reference/actions:reference-actions}}\label{\detokenize{reference/actions:csv}}\label{\detokenize{reference/actions::doc}}\label{\detokenize{reference/actions:actions}}
Actions define the behavior of the system in response to user actions: login,
action button, selection of an invoice, …

Actions can be stored in the database or returned directly as dictionaries in
e.g. button methods. All actions share two mandatory attributes:
\begin{description}
\item[{\sphinxcode{\sphinxupquote{type}}}] \leavevmode
the category of the current action, determines which fields may be
used and how the action is interpreted

\item[{\sphinxcode{\sphinxupquote{name}}}] \leavevmode
short user-readable description of the action, may be displayed in the
client’s interface

\end{description}

A client can get actions in 4 forms:
\begin{description}
\item[{\sphinxcode{\sphinxupquote{False}}}] \leavevmode
if any action dialog is currently open, close it

\item[{A string}] \leavevmode
if a {\hyperref[\detokenize{reference/actions:reference-actions-client}]{\sphinxcrossref{\DUrole{std,std-ref}{client action}}}} matches, interpret as
a client action’s tag, otherwise treat as a number

\item[{A number}] \leavevmode
read the corresponding action record from the database, may be a database
identifier or an \DUrole{xref,std,std-term}{external id}

\item[{A dictionary}] \leavevmode
treat as a client action descriptor and execute

\end{description}


\subsection{Window Actions (\sphinxstyleliteralintitle{\sphinxupquote{ir.actions.act\_window}})}
\label{\detokenize{reference/actions:window-actions-ir-actions-act-window}}\label{\detokenize{reference/actions:reference-actions-window}}
The most common action type, used to present visualisations of a model through
{\hyperref[\detokenize{reference/views:reference-views}]{\sphinxcrossref{\DUrole{std,std-ref}{views}}}}: a window action defines a set of view types
(and possibly specific views) for a model (and possibly specific record of the
model).

Its fields are:
\begin{description}
\item[{\sphinxcode{\sphinxupquote{res\_model}}}] \leavevmode
model to present views for

\item[{\sphinxcode{\sphinxupquote{views}}}] \leavevmode
a list of \sphinxcode{\sphinxupquote{(view\_id, view\_type)}} pairs. The second element of each pair
is the category of the view (tree, form, graph, …) and the first is
an optional database id (or \sphinxcode{\sphinxupquote{False}}). If no id is provided, the client
should fetch the default view of the specified type for the requested
model (this is automatically done by
{\hyperref[\detokenize{reference/orm:odoo.models.Model.fields_view_get}]{\sphinxcrossref{\sphinxcode{\sphinxupquote{fields\_view\_get()}}}}}). The first type of the
list is the default view type and will be open by default when the action
is executed. Each view type should be present at most once in the list

\item[{\sphinxcode{\sphinxupquote{res\_id}} (optional)}] \leavevmode
if the default view is \sphinxcode{\sphinxupquote{form}}, specifies the record to load (otherwise
a new record should be created)

\item[{\sphinxcode{\sphinxupquote{search\_view\_id}} (optional)}] \leavevmode
\sphinxcode{\sphinxupquote{(id, name)}} pair, \sphinxcode{\sphinxupquote{id}} is the database identifier of a specific
search view to load for the action. Defaults to fetching the default
search view for the model

\item[{\sphinxcode{\sphinxupquote{target}} (optional)}] \leavevmode
whether the views should be open in the main content area (\sphinxcode{\sphinxupquote{current}}),
in full screen mode (\sphinxcode{\sphinxupquote{fullscreen}}) or in a dialog/popup (\sphinxcode{\sphinxupquote{new}}). Use
\sphinxcode{\sphinxupquote{main}} instead of \sphinxcode{\sphinxupquote{current}} to clear the breadcrumbs. Defaults to
\sphinxcode{\sphinxupquote{current}}.

\item[{\sphinxcode{\sphinxupquote{context}} (optional)}] \leavevmode
additional context data to pass to the views

\item[{\sphinxcode{\sphinxupquote{domain}} (optional)}] \leavevmode
filtering domain to implicitly add to all view search queries

\item[{\sphinxcode{\sphinxupquote{limit}} (optional)}] \leavevmode
number of records to display in lists by default. Defaults to 80 in the
web client

\item[{\sphinxcode{\sphinxupquote{auto\_search}} (optional)}] \leavevmode
whether a search should be performed immediately after loading the default
view. Defaults to \sphinxcode{\sphinxupquote{True}}

\end{description}

For instance, to open customers (partner with the \sphinxcode{\sphinxupquote{customer}} flag set) with
list and form views:

\fvset{hllines={, ,}}%
\begin{sphinxVerbatim}[commandchars=\\\{\}]
\PYG{p}{\PYGZob{}}
    \PYG{l+s+s2}{\PYGZdq{}}\PYG{l+s+s2}{type}\PYG{l+s+s2}{\PYGZdq{}}\PYG{p}{:} \PYG{l+s+s2}{\PYGZdq{}}\PYG{l+s+s2}{ir.actions.act\PYGZus{}window}\PYG{l+s+s2}{\PYGZdq{}}\PYG{p}{,}
    \PYG{l+s+s2}{\PYGZdq{}}\PYG{l+s+s2}{res\PYGZus{}model}\PYG{l+s+s2}{\PYGZdq{}}\PYG{p}{:} \PYG{l+s+s2}{\PYGZdq{}}\PYG{l+s+s2}{res.partner}\PYG{l+s+s2}{\PYGZdq{}}\PYG{p}{,}
    \PYG{l+s+s2}{\PYGZdq{}}\PYG{l+s+s2}{views}\PYG{l+s+s2}{\PYGZdq{}}\PYG{p}{:} \PYG{p}{[}\PYG{p}{[}\PYG{k+kc}{False}\PYG{p}{,} \PYG{l+s+s2}{\PYGZdq{}}\PYG{l+s+s2}{tree}\PYG{l+s+s2}{\PYGZdq{}}\PYG{p}{]}\PYG{p}{,} \PYG{p}{[}\PYG{k+kc}{False}\PYG{p}{,} \PYG{l+s+s2}{\PYGZdq{}}\PYG{l+s+s2}{form}\PYG{l+s+s2}{\PYGZdq{}}\PYG{p}{]}\PYG{p}{]}\PYG{p}{,}
    \PYG{l+s+s2}{\PYGZdq{}}\PYG{l+s+s2}{domain}\PYG{l+s+s2}{\PYGZdq{}}\PYG{p}{:} \PYG{p}{[}\PYG{p}{[}\PYG{l+s+s2}{\PYGZdq{}}\PYG{l+s+s2}{customer}\PYG{l+s+s2}{\PYGZdq{}}\PYG{p}{,} \PYG{l+s+s2}{\PYGZdq{}}\PYG{l+s+s2}{=}\PYG{l+s+s2}{\PYGZdq{}}\PYG{p}{,} \PYG{n}{true}\PYG{p}{]}\PYG{p}{]}\PYG{p}{,}
\PYG{p}{\PYGZcb{}}
\end{sphinxVerbatim}

Or to open the form view of a specific product (obtained separately) in a new
dialog:

\fvset{hllines={, ,}}%
\begin{sphinxVerbatim}[commandchars=\\\{\}]
\PYG{p}{\PYGZob{}}
    \PYG{l+s+s2}{\PYGZdq{}}\PYG{l+s+s2}{type}\PYG{l+s+s2}{\PYGZdq{}}\PYG{p}{:} \PYG{l+s+s2}{\PYGZdq{}}\PYG{l+s+s2}{ir.actions.act\PYGZus{}window}\PYG{l+s+s2}{\PYGZdq{}}\PYG{p}{,}
    \PYG{l+s+s2}{\PYGZdq{}}\PYG{l+s+s2}{res\PYGZus{}model}\PYG{l+s+s2}{\PYGZdq{}}\PYG{p}{:} \PYG{l+s+s2}{\PYGZdq{}}\PYG{l+s+s2}{product.product}\PYG{l+s+s2}{\PYGZdq{}}\PYG{p}{,}
    \PYG{l+s+s2}{\PYGZdq{}}\PYG{l+s+s2}{views}\PYG{l+s+s2}{\PYGZdq{}}\PYG{p}{:} \PYG{p}{[}\PYG{p}{[}\PYG{k+kc}{False}\PYG{p}{,} \PYG{l+s+s2}{\PYGZdq{}}\PYG{l+s+s2}{form}\PYG{l+s+s2}{\PYGZdq{}}\PYG{p}{]}\PYG{p}{]}\PYG{p}{,}
    \PYG{l+s+s2}{\PYGZdq{}}\PYG{l+s+s2}{res\PYGZus{}id}\PYG{l+s+s2}{\PYGZdq{}}\PYG{p}{:} \PYG{n}{a\PYGZus{}product\PYGZus{}id}\PYG{p}{,}
    \PYG{l+s+s2}{\PYGZdq{}}\PYG{l+s+s2}{target}\PYG{l+s+s2}{\PYGZdq{}}\PYG{p}{:} \PYG{l+s+s2}{\PYGZdq{}}\PYG{l+s+s2}{new}\PYG{l+s+s2}{\PYGZdq{}}\PYG{p}{,}
\PYG{p}{\PYGZcb{}}
\end{sphinxVerbatim}

In-database window actions have a few different fields which should be ignored
by clients, mostly to use in composing the \sphinxcode{\sphinxupquote{views}} list:
\begin{description}
\item[{\sphinxcode{\sphinxupquote{view\_mode}}}] \leavevmode
comma-separated list of view types as a string. All of these types will be
present in the generated \sphinxcode{\sphinxupquote{views}} list (with at least a \sphinxcode{\sphinxupquote{False}} view\_id)

\item[{\sphinxcode{\sphinxupquote{view\_ids}}}] \leavevmode
M2M%
\begin{footnote}[1]\sphinxAtStartFootnote
technically not an M2M: adds a sequence field and may be
composed of just a view type, without a view id.
%
\end{footnote} to view objects, defines the initial content of
\sphinxcode{\sphinxupquote{views}}

\item[{\sphinxcode{\sphinxupquote{view\_id}}}] \leavevmode
specific view added to the \sphinxcode{\sphinxupquote{views}} list in case its type is part of the
\sphinxcode{\sphinxupquote{view\_mode}} list and not already filled by one of the views in
\sphinxcode{\sphinxupquote{view\_ids}}

\end{description}

These are mostly used when defining actions from {\hyperref[\detokenize{reference/data:reference-data}]{\sphinxcrossref{\DUrole{std,std-ref}{Data Files}}}}:

\fvset{hllines={, ,}}%
\begin{sphinxVerbatim}[commandchars=\\\{\}]
\PYG{n+nt}{\PYGZlt{}record} \PYG{n+na}{model=}\PYG{l+s}{\PYGZdq{}ir.actions.act\PYGZus{}window\PYGZdq{}} \PYG{n+na}{id=}\PYG{l+s}{\PYGZdq{}test\PYGZus{}action\PYGZdq{}}\PYG{n+nt}{\PYGZgt{}}
    \PYG{n+nt}{\PYGZlt{}field} \PYG{n+na}{name=}\PYG{l+s}{\PYGZdq{}name\PYGZdq{}}\PYG{n+nt}{\PYGZgt{}}A Test Action\PYG{n+nt}{\PYGZlt{}/field\PYGZgt{}}
    \PYG{n+nt}{\PYGZlt{}field} \PYG{n+na}{name=}\PYG{l+s}{\PYGZdq{}res\PYGZus{}model\PYGZdq{}}\PYG{n+nt}{\PYGZgt{}}some.model\PYG{n+nt}{\PYGZlt{}/field\PYGZgt{}}
    \PYG{n+nt}{\PYGZlt{}field} \PYG{n+na}{name=}\PYG{l+s}{\PYGZdq{}view\PYGZus{}mode\PYGZdq{}}\PYG{n+nt}{\PYGZgt{}}graph\PYG{n+nt}{\PYGZlt{}/field\PYGZgt{}}
    \PYG{n+nt}{\PYGZlt{}field} \PYG{n+na}{name=}\PYG{l+s}{\PYGZdq{}view\PYGZus{}id\PYGZdq{}} \PYG{n+na}{ref=}\PYG{l+s}{\PYGZdq{}my\PYGZus{}specific\PYGZus{}view\PYGZdq{}}\PYG{n+nt}{/\PYGZgt{}}
\PYG{n+nt}{\PYGZlt{}/record\PYGZgt{}}
\end{sphinxVerbatim}

will use the “my\_specific\_view” view even if that’s not the default view for
the model.

The server-side composition of the \sphinxcode{\sphinxupquote{views}} sequence is the following:
\begin{itemize}
\item {} 
get each \sphinxcode{\sphinxupquote{(id, type)}} from \sphinxcode{\sphinxupquote{view\_ids}} (ordered by \sphinxcode{\sphinxupquote{sequence}})

\item {} 
if \sphinxcode{\sphinxupquote{view\_id}} is defined and its type isn’t already filled, append its
\sphinxcode{\sphinxupquote{(id, type)}}

\item {} 
for each unfilled type in \sphinxcode{\sphinxupquote{view\_mode}}, append \sphinxcode{\sphinxupquote{(False, type)}}

\end{itemize}


\subsection{URL Actions (\sphinxstyleliteralintitle{\sphinxupquote{ir.actions.act\_url}})}
\label{\detokenize{reference/actions:url-actions-ir-actions-act-url}}\label{\detokenize{reference/actions:reference-actions-url}}
Allow opening a URL (website/web page) via an Odoo action. Can be customized
via two fields:
\begin{description}
\item[{\sphinxcode{\sphinxupquote{url}}}] \leavevmode
the address to open when activating the action

\item[{\sphinxcode{\sphinxupquote{target}}}] \leavevmode
opens the address in a new window/page if \sphinxcode{\sphinxupquote{new}}, replaces
the current content with the page if \sphinxcode{\sphinxupquote{self}}. Defaults to \sphinxcode{\sphinxupquote{new}}

\end{description}

\fvset{hllines={, ,}}%
\begin{sphinxVerbatim}[commandchars=\\\{\}]
\PYG{p}{\PYGZob{}}
    \PYG{l+s+s2}{\PYGZdq{}}\PYG{l+s+s2}{type}\PYG{l+s+s2}{\PYGZdq{}}\PYG{p}{:} \PYG{l+s+s2}{\PYGZdq{}}\PYG{l+s+s2}{ir.actions.act\PYGZus{}url}\PYG{l+s+s2}{\PYGZdq{}}\PYG{p}{,}
    \PYG{l+s+s2}{\PYGZdq{}}\PYG{l+s+s2}{url}\PYG{l+s+s2}{\PYGZdq{}}\PYG{p}{:} \PYG{l+s+s2}{\PYGZdq{}}\PYG{l+s+s2}{http://odoo.com}\PYG{l+s+s2}{\PYGZdq{}}\PYG{p}{,}
    \PYG{l+s+s2}{\PYGZdq{}}\PYG{l+s+s2}{target}\PYG{l+s+s2}{\PYGZdq{}}\PYG{p}{:} \PYG{l+s+s2}{\PYGZdq{}}\PYG{l+s+s2}{self}\PYG{l+s+s2}{\PYGZdq{}}\PYG{p}{,}
\PYG{p}{\PYGZcb{}}
\end{sphinxVerbatim}

will replace the current content section by the Odoo home page.


\subsection{Server Actions (\sphinxstyleliteralintitle{\sphinxupquote{ir.actions.server}})}
\label{\detokenize{reference/actions:server-actions-ir-actions-server}}\label{\detokenize{reference/actions:reference-actions-server}}
Allow triggering complex server code from any valid action location. Only
two fields are relevant to clients:
\begin{description}
\item[{\sphinxcode{\sphinxupquote{id}}}] \leavevmode
the in-database identifier of the server action to run

\item[{\sphinxcode{\sphinxupquote{context}} (optional)}] \leavevmode
context data to use when running the server action

\end{description}

In-database records are significantly richer and can perform a number of
specific or generic actions based on their \sphinxcode{\sphinxupquote{state}}. Some fields (and
corresponding behaviors) are shared between states:
\begin{description}
\item[{\sphinxcode{\sphinxupquote{model\_id}}}] \leavevmode
Odoo model linked to the action, made available in
{\hyperref[\detokenize{reference/actions:reference-actions-server-context}]{\sphinxcrossref{\DUrole{std,std-ref}{evaluation contexts}}}}

\item[{\sphinxcode{\sphinxupquote{condition}} (optional)}] \leavevmode
evaluated as Python code using the server action’s
{\hyperref[\detokenize{reference/actions:reference-actions-server-context}]{\sphinxcrossref{\DUrole{std,std-ref}{evaluation context}}}}. If
\sphinxcode{\sphinxupquote{False}}, prevents the action from running. Default: \sphinxcode{\sphinxupquote{True}}

\end{description}

Valid action types (\sphinxcode{\sphinxupquote{state}} field) are extensible, the default types are:


\subsubsection{\sphinxstyleliteralintitle{\sphinxupquote{code}}}
\label{\detokenize{reference/actions:code}}
The default and most flexible server action type, executes arbitrary Python
code with the action’s {\hyperref[\detokenize{reference/actions:reference-actions-server-context}]{\sphinxcrossref{\DUrole{std,std-ref}{evaluation context}}}}. Only uses one specific type-specific
field:
\begin{description}
\item[{\sphinxcode{\sphinxupquote{code}}}] \leavevmode
a piece of Python code to execute when the action is called

\end{description}

\fvset{hllines={, ,}}%
\begin{sphinxVerbatim}[commandchars=\\\{\}]
\PYG{n+nt}{\PYGZlt{}record} \PYG{n+na}{model=}\PYG{l+s}{\PYGZdq{}ir.actions.server\PYGZdq{}} \PYG{n+na}{id=}\PYG{l+s}{\PYGZdq{}print\PYGZus{}instance\PYGZdq{}}\PYG{n+nt}{\PYGZgt{}}
    \PYG{n+nt}{\PYGZlt{}field} \PYG{n+na}{name=}\PYG{l+s}{\PYGZdq{}name\PYGZdq{}}\PYG{n+nt}{\PYGZgt{}}Res Partner Server Action\PYG{n+nt}{\PYGZlt{}/field\PYGZgt{}}
    \PYG{n+nt}{\PYGZlt{}field} \PYG{n+na}{name=}\PYG{l+s}{\PYGZdq{}model\PYGZus{}id\PYGZdq{}} \PYG{n+na}{ref=}\PYG{l+s}{\PYGZdq{}model\PYGZus{}res\PYGZus{}partner\PYGZdq{}}\PYG{n+nt}{/\PYGZgt{}}
    \PYG{n+nt}{\PYGZlt{}field} \PYG{n+na}{name=}\PYG{l+s}{\PYGZdq{}code\PYGZdq{}}\PYG{n+nt}{\PYGZgt{}}
        raise Warning(object.name)
    \PYG{n+nt}{\PYGZlt{}/field\PYGZgt{}}
\PYG{n+nt}{\PYGZlt{}/record\PYGZgt{}}
\end{sphinxVerbatim}

\begin{sphinxadmonition}{note}{Note:}
The code segment can define a variable called \sphinxcode{\sphinxupquote{action}}, which will be
returned to the client as the next action to execute:

\fvset{hllines={, ,}}%
\begin{sphinxVerbatim}[commandchars=\\\{\}]
\PYG{n+nt}{\PYGZlt{}record} \PYG{n+na}{model=}\PYG{l+s}{\PYGZdq{}ir.actions.server\PYGZdq{}} \PYG{n+na}{id=}\PYG{l+s}{\PYGZdq{}print\PYGZus{}instance\PYGZdq{}}\PYG{n+nt}{\PYGZgt{}}
    \PYG{n+nt}{\PYGZlt{}field} \PYG{n+na}{name=}\PYG{l+s}{\PYGZdq{}name\PYGZdq{}}\PYG{n+nt}{\PYGZgt{}}Res Partner Server Action\PYG{n+nt}{\PYGZlt{}/field\PYGZgt{}}
    \PYG{n+nt}{\PYGZlt{}field} \PYG{n+na}{name=}\PYG{l+s}{\PYGZdq{}model\PYGZus{}id\PYGZdq{}} \PYG{n+na}{ref=}\PYG{l+s}{\PYGZdq{}model\PYGZus{}res\PYGZus{}partner\PYGZdq{}}\PYG{n+nt}{/\PYGZgt{}}
    \PYG{n+nt}{\PYGZlt{}field} \PYG{n+na}{name=}\PYG{l+s}{\PYGZdq{}code\PYGZdq{}}\PYG{n+nt}{\PYGZgt{}}
        if object.some\PYGZus{}condition():
            action = \PYGZob{}
                \PYGZdq{}type\PYGZdq{}: \PYGZdq{}ir.actions.act\PYGZus{}window\PYGZdq{},
                \PYGZdq{}view\PYGZus{}mode\PYGZdq{}: \PYGZdq{}form\PYGZdq{},
                \PYGZdq{}res\PYGZus{}model\PYGZdq{}: object.\PYGZus{}name,
                \PYGZdq{}res\PYGZus{}id\PYGZdq{}: object.id,
            \PYGZcb{}
    \PYG{n+nt}{\PYGZlt{}/field\PYGZgt{}}
\PYG{n+nt}{\PYGZlt{}/record\PYGZgt{}}
\end{sphinxVerbatim}

will ask the client to open a form for the record if it fulfills some
condition
\end{sphinxadmonition}

This tends to be the only action type created from {\hyperref[\detokenize{reference/data:reference-data}]{\sphinxcrossref{\DUrole{std,std-ref}{data files}}}}, other types aside from
{\hyperref[\detokenize{reference/actions:reference-actions-server-multi}]{\sphinxcrossref{\DUrole{std,std-ref}{multi}}}} are simpler than Python code to define
from the UI, but not from {\hyperref[\detokenize{reference/data:reference-data}]{\sphinxcrossref{\DUrole{std,std-ref}{data files}}}}.


\subsubsection{\sphinxstyleliteralintitle{\sphinxupquote{object\_create}}}
\label{\detokenize{reference/actions:reference-actions-server-object-create}}\label{\detokenize{reference/actions:object-create}}
Creates a new record, from scratch (via {\hyperref[\detokenize{reference/orm:odoo.models.Model.create}]{\sphinxcrossref{\sphinxcode{\sphinxupquote{create()}}}}})
or by copying an existing record (via {\hyperref[\detokenize{reference/orm:odoo.models.Model.copy}]{\sphinxcrossref{\sphinxcode{\sphinxupquote{copy()}}}}})
\begin{description}
\item[{\sphinxcode{\sphinxupquote{use\_create}}}] \leavevmode
the creation policy, one of:
\begin{description}
\item[{\sphinxcode{\sphinxupquote{new}}}] \leavevmode
creates a record in the model specified by \sphinxcode{\sphinxupquote{model\_id}}

\item[{\sphinxcode{\sphinxupquote{new\_other}}}] \leavevmode
creates a record in the model specified by \sphinxcode{\sphinxupquote{crud\_model\_id}}

\item[{\sphinxcode{\sphinxupquote{copy\_current}}}] \leavevmode
copies the record on which the action was invoked

\item[{\sphinxcode{\sphinxupquote{copy\_other}}}] \leavevmode
copies an other record, obtained via \sphinxcode{\sphinxupquote{ref\_object}}

\end{description}

\item[{\sphinxcode{\sphinxupquote{fields\_lines}}}] \leavevmode
fields to override when creating or copying the record.
{\hyperref[\detokenize{reference/orm:odoo.fields.One2many}]{\sphinxcrossref{\sphinxcode{\sphinxupquote{One2many}}}}} with the fields:
\begin{description}
\item[{\sphinxcode{\sphinxupquote{col1}}}] \leavevmode
\sphinxcode{\sphinxupquote{ir.model.fields}} to set in the model implied by \sphinxcode{\sphinxupquote{use\_create}}

\item[{\sphinxcode{\sphinxupquote{value}}}] \leavevmode
value for the field, interpreted via \sphinxcode{\sphinxupquote{type}}

\item[{\sphinxcode{\sphinxupquote{type}}}] \leavevmode
If \sphinxcode{\sphinxupquote{value}}, the \sphinxcode{\sphinxupquote{value}} field is interpreted as a literal value
(possibly converted), if \sphinxcode{\sphinxupquote{equation}} the \sphinxcode{\sphinxupquote{value}} field is
interpreted as a Python expression and evaluated

\end{description}

\item[{\sphinxcode{\sphinxupquote{crud\_model\_id}}}] \leavevmode
model in which to create a new record, if \sphinxcode{\sphinxupquote{use\_create}} is set to
\sphinxcode{\sphinxupquote{new\_other}}

\item[{\sphinxcode{\sphinxupquote{ref\_object}}}] \leavevmode
{\hyperref[\detokenize{reference/orm:odoo.fields.Reference}]{\sphinxcrossref{\sphinxcode{\sphinxupquote{Reference}}}}} to an arbitrary record to copy, used if
\sphinxcode{\sphinxupquote{use\_create}} is set to \sphinxcode{\sphinxupquote{copy\_other}}

\item[{\sphinxcode{\sphinxupquote{link\_new\_record}}}] \leavevmode
boolean flag linking the newly created record to the current one via a
many2one field specified through \sphinxcode{\sphinxupquote{link\_field\_id}}, defaults to \sphinxcode{\sphinxupquote{False}}

\item[{\sphinxcode{\sphinxupquote{link\_field\_id}}}] \leavevmode
many2one to \sphinxcode{\sphinxupquote{ir.model.fields}}, specifies the current record’s m2o field
on which the newly created record should be set (models should match)

\end{description}


\subsubsection{\sphinxstyleliteralintitle{\sphinxupquote{object\_write}}}
\label{\detokenize{reference/actions:object-write}}
Similar to {\hyperref[\detokenize{reference/actions:reference-actions-server-object-create}]{\sphinxcrossref{\DUrole{std,std-ref}{object\_create}}}} but alters an
existing records instead of creating one
\begin{description}
\item[{\sphinxcode{\sphinxupquote{use\_write}}}] \leavevmode
write policy, one of:
\begin{description}
\item[{\sphinxcode{\sphinxupquote{current}}}] \leavevmode
write to the current record

\item[{\sphinxcode{\sphinxupquote{other}}}] \leavevmode
write to an other record selected via \sphinxcode{\sphinxupquote{crud\_model\_id}} and
\sphinxcode{\sphinxupquote{ref\_object}}

\item[{\sphinxcode{\sphinxupquote{expression}}}] \leavevmode
write to an other record whose model is selected via \sphinxcode{\sphinxupquote{crud\_model\_id}}
and whose id is selected by evaluating \sphinxcode{\sphinxupquote{write\_expression}}

\end{description}

\item[{\sphinxcode{\sphinxupquote{write\_expression}}}] \leavevmode
Python expression returning a record or an object id, used when
\sphinxcode{\sphinxupquote{use\_write}} is set to \sphinxcode{\sphinxupquote{expression}} in order to decide which record
should be modified

\item[{\sphinxcode{\sphinxupquote{fields\_lines}}}] \leavevmode
see {\hyperref[\detokenize{reference/actions:reference-actions-server-object-create}]{\sphinxcrossref{\DUrole{std,std-ref}{object\_create}}}}

\item[{\sphinxcode{\sphinxupquote{crud\_model\_id}}}] \leavevmode
see {\hyperref[\detokenize{reference/actions:reference-actions-server-object-create}]{\sphinxcrossref{\DUrole{std,std-ref}{object\_create}}}}

\item[{\sphinxcode{\sphinxupquote{ref\_object}}}] \leavevmode
see {\hyperref[\detokenize{reference/actions:reference-actions-server-object-create}]{\sphinxcrossref{\DUrole{std,std-ref}{object\_create}}}}

\end{description}


\subsubsection{\sphinxstyleliteralintitle{\sphinxupquote{multi}}}
\label{\detokenize{reference/actions:reference-actions-server-multi}}\label{\detokenize{reference/actions:multi}}
Executes multiple actions one after the other. Actions to execute are defined
via the \sphinxcode{\sphinxupquote{child\_ids}} m2m. If sub-actions themselves return actions, the last
one will be returned to the client as the multi’s own next action


\subsubsection{\sphinxstyleliteralintitle{\sphinxupquote{client\_action}}}
\label{\detokenize{reference/actions:client-action}}
Indirection for directly returning an other action defined using
\sphinxcode{\sphinxupquote{action\_id}}. Simply returns that action to the client for execution.


\subsubsection{Evaluation context}
\label{\detokenize{reference/actions:reference-actions-server-context}}\label{\detokenize{reference/actions:evaluation-context}}
A number of keys are available in the evaluation context of or surrounding
server actions:
\begin{description}
\item[{\sphinxcode{\sphinxupquote{model}}}] \leavevmode
the model object linked to the action via \sphinxcode{\sphinxupquote{model\_id}}

\item[{\sphinxcode{\sphinxupquote{object}}, \sphinxcode{\sphinxupquote{obj}}}] \leavevmode
only available if \sphinxcode{\sphinxupquote{active\_model}} and \sphinxcode{\sphinxupquote{active\_id}} are provided (via
context) otherwise \sphinxcode{\sphinxupquote{None}}. The actual record selected by \sphinxcode{\sphinxupquote{active\_id}}

\item[{\sphinxcode{\sphinxupquote{pool}}}] \leavevmode
the current database registry

\item[{\sphinxcode{\sphinxupquote{datetime}}, \sphinxcode{\sphinxupquote{dateutil}}, \sphinxcode{\sphinxupquote{time}}}] \leavevmode
corresponding Python modules

\item[{\sphinxcode{\sphinxupquote{cr}}}] \leavevmode
the current cursor

\item[{\sphinxcode{\sphinxupquote{user}}}] \leavevmode
the current user record

\item[{\sphinxcode{\sphinxupquote{context}}}] \leavevmode
execution context

\item[{\sphinxcode{\sphinxupquote{Warning}}}] \leavevmode
constructor for the \sphinxcode{\sphinxupquote{Warning}} exception

\end{description}


\subsection{Report Actions (\sphinxstyleliteralintitle{\sphinxupquote{ir.actions.report}})}
\label{\detokenize{reference/actions:reference-actions-report}}\label{\detokenize{reference/actions:report-actions-ir-actions-report}}
Triggers the printing of a report
\begin{description}
\item[{\sphinxcode{\sphinxupquote{name}} (mandatory)}] \leavevmode
only useful as a mnemonic/description of the report when looking for one
in a list of some sort

\item[{\sphinxcode{\sphinxupquote{model}} (mandatory)}] \leavevmode
the model your report will be about

\item[{\sphinxcode{\sphinxupquote{report\_type}} (mandatory)}] \leavevmode
either \sphinxcode{\sphinxupquote{qweb-pdf}} for PDF reports or \sphinxcode{\sphinxupquote{qweb-html}} for HTML

\item[{\sphinxcode{\sphinxupquote{report\_name}}}] \leavevmode
the name of your report (which will be the name of the PDF output)

\item[{\sphinxcode{\sphinxupquote{groups\_id}}}] \leavevmode
{\hyperref[\detokenize{reference/orm:odoo.fields.Many2many}]{\sphinxcrossref{\sphinxcode{\sphinxupquote{Many2many}}}}} field to the groups allowed to view/use
the current report

\item[{\sphinxcode{\sphinxupquote{paperformat\_id}}}] \leavevmode
{\hyperref[\detokenize{reference/orm:odoo.fields.Many2one}]{\sphinxcrossref{\sphinxcode{\sphinxupquote{Many2one}}}}} field to the paper format you wish to
use for this report (if not specified, the company format will be used)

\item[{\sphinxcode{\sphinxupquote{attachment\_use}}}] \leavevmode
if set to \sphinxcode{\sphinxupquote{True}}, the report is only generated once the first time it is
requested, and re-printed from the stored report afterwards instead of
being re-generated every time.

Can be used for reports which must only be generated once (e.g. for legal
reasons)

\item[{\sphinxcode{\sphinxupquote{attachment}}}] \leavevmode
python expression that defines the name of the report; the record is
accessible as the variable \sphinxcode{\sphinxupquote{object}}

\end{description}


\subsection{Client Actions (\sphinxstyleliteralintitle{\sphinxupquote{ir.actions.client}})}
\label{\detokenize{reference/actions:client-actions-ir-actions-client}}\label{\detokenize{reference/actions:reference-actions-client}}
Triggers an action implemented entirely in the client.
\begin{description}
\item[{\sphinxcode{\sphinxupquote{tag}}}] \leavevmode
the client-side identifier of the action, an arbitrary string which
the client should know how to react to

\item[{\sphinxcode{\sphinxupquote{params}} (optional)}] \leavevmode
a Python dictionary of additional data to send to the client, alongside
the client action tag

\item[{\sphinxcode{\sphinxupquote{target}} (optional)}] \leavevmode
whether the client action should be open in the main content area
(\sphinxcode{\sphinxupquote{current}}), in full screen mode (\sphinxcode{\sphinxupquote{fullscreen}}) or in a dialog/popup
(\sphinxcode{\sphinxupquote{new}}). Use \sphinxcode{\sphinxupquote{main}} instead of \sphinxcode{\sphinxupquote{current}} to clear the breadcrumbs.
Defaults to \sphinxcode{\sphinxupquote{current}}.

\end{description}

\fvset{hllines={, ,}}%
\begin{sphinxVerbatim}[commandchars=\\\{\}]
\PYG{p}{\PYGZob{}}
    \PYG{l+s+s2}{\PYGZdq{}}\PYG{l+s+s2}{type}\PYG{l+s+s2}{\PYGZdq{}}\PYG{p}{:} \PYG{l+s+s2}{\PYGZdq{}}\PYG{l+s+s2}{ir.actions.client}\PYG{l+s+s2}{\PYGZdq{}}\PYG{p}{,}
    \PYG{l+s+s2}{\PYGZdq{}}\PYG{l+s+s2}{tag}\PYG{l+s+s2}{\PYGZdq{}}\PYG{p}{:} \PYG{l+s+s2}{\PYGZdq{}}\PYG{l+s+s2}{pos.ui}\PYG{l+s+s2}{\PYGZdq{}}
\PYG{p}{\PYGZcb{}}
\end{sphinxVerbatim}

tells the client to start the Point of Sale interface, the server has no idea
how the POS interface works.


\section{Views}
\label{\detokenize{reference/views::doc}}\label{\detokenize{reference/views:reference-views}}\label{\detokenize{reference/views:views}}

\subsection{Common Structure}
\label{\detokenize{reference/views:common-structure}}\label{\detokenize{reference/views:reference-views-structure}}
View objects expose a number of fields, they are optional unless specified
otherwise.
\begin{description}
\item[{\sphinxcode{\sphinxupquote{name}} (mandatory)}] \leavevmode
only useful as a mnemonic/description of the view when looking for one in
a list of some sort

\item[{\sphinxcode{\sphinxupquote{model}}}] \leavevmode
the model linked to the view, if applicable (it doesn’t for QWeb views)

\item[{\sphinxcode{\sphinxupquote{priority}}}] \leavevmode
client programs can request views by \sphinxcode{\sphinxupquote{id}}, or by \sphinxcode{\sphinxupquote{(model, type)}}. For
the latter, all the views for the right type and model will be searched,
and the one with the lowest \sphinxcode{\sphinxupquote{priority}} number will be returned (it is
the “default view”).

\sphinxcode{\sphinxupquote{priority}} also defines the order of application during {\hyperref[\detokenize{reference/views:reference-views-inheritance}]{\sphinxcrossref{\DUrole{std,std-ref}{view
inheritance}}}}

\item[{\sphinxcode{\sphinxupquote{arch}}}] \leavevmode
the description of the view’s layout

\item[{\sphinxcode{\sphinxupquote{groups\_id}}}] \leavevmode
{\hyperref[\detokenize{reference/orm:odoo.fields.Many2many}]{\sphinxcrossref{\sphinxcode{\sphinxupquote{Many2many}}}}} field to the groups allowed to view/use
the current view

\item[{\sphinxcode{\sphinxupquote{inherit\_id}}}] \leavevmode
the current view’s parent view, see {\hyperref[\detokenize{reference/views:reference-views-inheritance}]{\sphinxcrossref{\DUrole{std,std-ref}{Inheritance}}}},
unset by default

\item[{\sphinxcode{\sphinxupquote{mode}}}] \leavevmode
inheritance mode, see {\hyperref[\detokenize{reference/views:reference-views-inheritance}]{\sphinxcrossref{\DUrole{std,std-ref}{Inheritance}}}}. If
\sphinxcode{\sphinxupquote{inherit\_id}} is unset the \sphinxcode{\sphinxupquote{mode}} can only be \sphinxcode{\sphinxupquote{primary}}. If
\sphinxcode{\sphinxupquote{inherit\_id}} is set, \sphinxcode{\sphinxupquote{extension}} by default but can be explicitly set
to \sphinxcode{\sphinxupquote{primary}}

\item[{\sphinxcode{\sphinxupquote{application}}}] \leavevmode
website feature defining togglable views. By default, views are always
applied

\end{description}


\subsection{Inheritance}
\label{\detokenize{reference/views:reference-views-inheritance}}\label{\detokenize{reference/views:inheritance}}

\subsubsection{View matching}
\label{\detokenize{reference/views:view-matching}}\begin{itemize}
\item {} 
if a view is requested by \sphinxcode{\sphinxupquote{(model, type)}}, the view with the right model
and type, \sphinxcode{\sphinxupquote{mode=primary}} and the lowest priority is matched

\item {} 
when a view is requested by \sphinxcode{\sphinxupquote{id}}, if its mode is not \sphinxcode{\sphinxupquote{primary}} its
\sphinxstyleemphasis{closest} parent with mode \sphinxcode{\sphinxupquote{primary}} is matched

\end{itemize}


\subsubsection{View resolution}
\label{\detokenize{reference/views:view-resolution}}
Resolution generates the final \sphinxcode{\sphinxupquote{arch}} for a requested/matched \sphinxcode{\sphinxupquote{primary}}
view:
\begin{enumerate}
\item {} 
if the view has a parent, the parent is fully resolved then the current
view’s inheritance specs are applied

\item {} 
if the view has no parent, its \sphinxcode{\sphinxupquote{arch}} is used as-is

\item {} 
the current view’s children with mode \sphinxcode{\sphinxupquote{extension}} are looked up  and their
inheritance specs are applied depth-first (a child view is applied, then
its children, then its siblings)

\end{enumerate}

The result of applying children views yields the final \sphinxcode{\sphinxupquote{arch}}


\subsubsection{Inheritance specs}
\label{\detokenize{reference/views:inheritance-specs}}
Inheritance specs are comprised of an element locator, to match
the inherited element in the parent view, and children element that
will be used to modify the inherited element.

There are three types of element locators for matching a target element:
\begin{itemize}
\item {} 
An \sphinxcode{\sphinxupquote{xpath}} element with an \sphinxcode{\sphinxupquote{expr}} attribute. \sphinxcode{\sphinxupquote{expr}} is an \sphinxhref{http://en.wikipedia.org/wiki/XPath}{XPath}
expression%
\begin{footnote}[2]\sphinxAtStartFootnote
an extension function is added for simpler matching in QWeb
views: \sphinxcode{\sphinxupquote{hasclass(*classes)}} matches if the context node has
all the specified classes
%
\end{footnote} applied to the current \sphinxcode{\sphinxupquote{arch}}, the first node
it finds is the match

\item {} 
a \sphinxcode{\sphinxupquote{field}} element with a \sphinxcode{\sphinxupquote{name}} attribute, matches the first \sphinxcode{\sphinxupquote{field}}
with the same \sphinxcode{\sphinxupquote{name}}. All other attributes are ignored during matching

\item {} 
any other element: the first element with the same name and identical
attributes (ignoring \sphinxcode{\sphinxupquote{position}} and \sphinxcode{\sphinxupquote{version}} attributes) is matched

\end{itemize}

The inheritance spec may have an optional \sphinxcode{\sphinxupquote{position}} attribute specifying
how the matched node should be altered:
\begin{description}
\item[{\sphinxcode{\sphinxupquote{inside}} (default)}] \leavevmode
the content of the inheritance spec is appended to the matched node

\item[{\sphinxcode{\sphinxupquote{replace}}}] \leavevmode
the content of the inheritance spec replaces the matched node.
Any text node containing only \sphinxcode{\sphinxupquote{\$0}} within the contents of the spec will
be replaced  by a complete copy of the matched node, effectively wrapping
the matched node.

\item[{\sphinxcode{\sphinxupquote{after}}}] \leavevmode
the content of the inheritance spec is added to the matched node’s
parent, after the matched node

\item[{\sphinxcode{\sphinxupquote{before}}}] \leavevmode
the content of the inheritance spec is added to the matched node’s
parent, before the matched node

\item[{\sphinxcode{\sphinxupquote{attributes}}}] \leavevmode
the content of the inheritance spec should be \sphinxcode{\sphinxupquote{attribute}} elements
with a \sphinxcode{\sphinxupquote{name}} attribute and an optional body:
\begin{itemize}
\item {} 
if the \sphinxcode{\sphinxupquote{attribute}} element has a body, a new attributed named
after its \sphinxcode{\sphinxupquote{name}} is created on the matched node with the
\sphinxcode{\sphinxupquote{attribute}} element’s text as value

\item {} 
if the \sphinxcode{\sphinxupquote{attribute}} element has no body, the attribute named after
its \sphinxcode{\sphinxupquote{name}} is removed from the matched node. If no such attribute
exists, an error is raised

\end{itemize}

\end{description}

A view’s specs are applied sequentially.


\subsection{Lists}
\label{\detokenize{reference/views:reference-views-list}}\label{\detokenize{reference/views:lists}}
The root element of list views is \sphinxcode{\sphinxupquote{\textless{}tree\textgreater{}}}%
\begin{footnote}[3]\sphinxAtStartFootnote
for historical reasons, it has its origin in tree-type views
later repurposed to a more table/list-type display
%
\end{footnote}. The list view’s
root can have the following attributes:
\begin{description}
\item[{\sphinxcode{\sphinxupquote{editable}}}] \leavevmode
by default, selecting a list view’s row opens the corresponding
{\hyperref[\detokenize{reference/views:reference-views-form}]{\sphinxcrossref{\DUrole{std,std-ref}{form view}}}}. The \sphinxcode{\sphinxupquote{editable}} attributes makes
the list view itself editable in-place.

Valid values are \sphinxcode{\sphinxupquote{top}} and \sphinxcode{\sphinxupquote{bottom}}, making \sphinxstyleemphasis{new} records appear
respectively at the top or bottom of the list.

The architecture for the inline {\hyperref[\detokenize{reference/views:reference-views-form}]{\sphinxcrossref{\DUrole{std,std-ref}{form view}}}} is
derived from the list view. Most attributes valid on a {\hyperref[\detokenize{reference/views:reference-views-form}]{\sphinxcrossref{\DUrole{std,std-ref}{form view}}}}’s fields and buttons are thus accepted by list
views although they may not have any meaning if the list view is
non-editable

\item[{\sphinxcode{\sphinxupquote{default\_order}}}] \leavevmode
overrides the ordering of the view, replacing the model’s default order.
The value is a comma-separated list of fields, postfixed by \sphinxcode{\sphinxupquote{desc}} to
sort in reverse order:

\fvset{hllines={, ,}}%
\begin{sphinxVerbatim}[commandchars=\\\{\}]
\PYG{n+nt}{\PYGZlt{}tree} \PYG{n+na}{default\PYGZus{}order=}\PYG{l+s}{\PYGZdq{}sequence,name desc\PYGZdq{}}\PYG{n+nt}{\PYGZgt{}}
\end{sphinxVerbatim}

\item[{\sphinxcode{\sphinxupquote{colors}}}] \leavevmode
\DUrole{versionmodified}{Deprecated since version 9.0: }replaced by \sphinxcode{\sphinxupquote{decoration-\{\$name\}}}

\item[{\sphinxcode{\sphinxupquote{fonts}}}] \leavevmode
\DUrole{versionmodified}{Deprecated since version 9.0: }replaced by \sphinxcode{\sphinxupquote{decoration-\{\$name\}}}

\item[{\sphinxcode{\sphinxupquote{decoration-\{\$name\}}}}] \leavevmode
allow changing the style of a row’s text based on the corresponding
record’s attributes.

Values are Python expressions. For each record, the expression is evaluated
with the record’s attributes as context values and if \sphinxcode{\sphinxupquote{true}}, the
corresponding style is applied to the row. Other context values are
\sphinxcode{\sphinxupquote{uid}} (the id of the current user) and \sphinxcode{\sphinxupquote{current\_date}} (the current date
as a string of the form \sphinxcode{\sphinxupquote{yyyy-MM-dd}}).

\sphinxcode{\sphinxupquote{\{\$name\}}} can be \sphinxcode{\sphinxupquote{bf}} (\sphinxcode{\sphinxupquote{font-weight: bold}}), \sphinxcode{\sphinxupquote{it}}
(\sphinxcode{\sphinxupquote{font-style: italic}}), or any \sphinxhref{https://getbootstrap.com/docs/3.3/components/\#available-variations}{bootstrap contextual color} (\sphinxcode{\sphinxupquote{danger}},
\sphinxcode{\sphinxupquote{info}}, \sphinxcode{\sphinxupquote{muted}}, \sphinxcode{\sphinxupquote{primary}}, \sphinxcode{\sphinxupquote{success}} or \sphinxcode{\sphinxupquote{warning}}).

\item[{\sphinxcode{\sphinxupquote{create}}, \sphinxcode{\sphinxupquote{edit}}, \sphinxcode{\sphinxupquote{delete}}}] \leavevmode
allows \sphinxstyleemphasis{dis}abling the corresponding action in the view by setting the
corresponding attribute to \sphinxcode{\sphinxupquote{false}}

\item[{\sphinxcode{\sphinxupquote{limit}}}] \leavevmode
the default size of a page. It should be a positive integer

\item[{\sphinxcode{\sphinxupquote{on\_write}}}] \leavevmode
only makes sense on an \sphinxcode{\sphinxupquote{editable}} list. Should be the name of a method
on the list’s model. The method will be called with the \sphinxcode{\sphinxupquote{id}} of a record
after having created or edited that record (in database).

The method should return a list of ids of other records to load or update.

\item[{\sphinxcode{\sphinxupquote{string}}}] \leavevmode
alternative translatable label for the view

\DUrole{versionmodified}{Deprecated since version 8.0: }not displayed anymore

\end{description}

Possible children elements of the list view are:

\phantomsection\label{\detokenize{reference/views:reference-views-list-button}}\begin{description}
\item[{\sphinxcode{\sphinxupquote{button}}}] \leavevmode
displays a button in a list cell
\begin{description}
\item[{\sphinxcode{\sphinxupquote{icon}}}] \leavevmode
icon to use to display the button

\item[{\sphinxcode{\sphinxupquote{string}}}] \leavevmode\begin{itemize}
\item {} 
if there is no \sphinxcode{\sphinxupquote{icon}}, the button’s text

\item {} 
if there is an \sphinxcode{\sphinxupquote{icon}}, \sphinxcode{\sphinxupquote{alt}} text for the icon

\end{itemize}

\item[{\sphinxcode{\sphinxupquote{type}}}] \leavevmode
type of button, indicates how it clicking it affects Odoo:
\begin{description}
\item[{\sphinxcode{\sphinxupquote{object}}}] \leavevmode
call a method on the list’s model. The button’s \sphinxcode{\sphinxupquote{name}} is the
method, which is called with the current row’s record id and the
current context.

\item[{\sphinxcode{\sphinxupquote{action}}}] \leavevmode
load an execute an \sphinxcode{\sphinxupquote{ir.actions}}, the button’s \sphinxcode{\sphinxupquote{name}} is the
database id of the action. The context is expanded with the list’s
model (as \sphinxcode{\sphinxupquote{active\_model}}), the current row’s record
(\sphinxcode{\sphinxupquote{active\_id}}) and all the records currently loaded in the list
(\sphinxcode{\sphinxupquote{active\_ids}}, may be just a subset of the database records
matching the current search)

\end{description}

\item[{\sphinxcode{\sphinxupquote{name}}}] \leavevmode
see \sphinxcode{\sphinxupquote{type}}

\item[{\sphinxcode{\sphinxupquote{args}}}] \leavevmode
see \sphinxcode{\sphinxupquote{type}}

\item[{\sphinxcode{\sphinxupquote{attrs}}}] \leavevmode
dynamic attributes based on record values.

A mapping of attributes to domains, domains are evaluated in the
context of the current row’s record, if \sphinxcode{\sphinxupquote{True}} the corresponding
attribute is set on the cell.

Possible attribute is \sphinxcode{\sphinxupquote{invisible}} (hides the button).

\item[{\sphinxcode{\sphinxupquote{states}}}] \leavevmode
shorthand for \sphinxcode{\sphinxupquote{invisible}} \sphinxcode{\sphinxupquote{attrs}}: a list of states, comma separated,
requires that the model has a \sphinxcode{\sphinxupquote{state}} field and that it is
used in the view.

Makes the button \sphinxcode{\sphinxupquote{invisible}} if the record is \sphinxstyleemphasis{not} in one of the
listed states

\begin{sphinxadmonition}{danger}{Danger:}
Using \sphinxcode{\sphinxupquote{states}} in combination with \sphinxcode{\sphinxupquote{attrs}} may lead to
unexpected results as domains are combined with a logical AND.
\end{sphinxadmonition}

\item[{\sphinxcode{\sphinxupquote{context}}}] \leavevmode
merged into the view’s context when performing the button’s Odoo call

\item[{\sphinxcode{\sphinxupquote{confirm}}}] \leavevmode
confirmation message to display (and for the user to accept) before
performing the button’s Odoo call

\end{description}

\item[{\sphinxcode{\sphinxupquote{field}}}] \leavevmode
defines a column where the corresponding field should be displayed for
each record. Can use the following attributes:
\begin{description}
\item[{\sphinxcode{\sphinxupquote{name}}}] \leavevmode
the name of the field to display in the current model. A given name
can only be used once per view

\item[{\sphinxcode{\sphinxupquote{string}}}] \leavevmode
the title of the field’s column (by default, uses the \sphinxcode{\sphinxupquote{string}} of
the model’s field)

\item[{\sphinxcode{\sphinxupquote{invisible}}}] \leavevmode
fetches and stores the field, but doesn’t display the column in the
table. Necessary for fields which shouldn’t be displayed but are
used by e.g. \sphinxcode{\sphinxupquote{@colors}}

\item[{\sphinxcode{\sphinxupquote{groups}}}] \leavevmode
lists the groups which should be able to see the field

\item[{\sphinxcode{\sphinxupquote{widget}}}] \leavevmode
alternate representations for a field’s display. Possible list view
values are:
\begin{description}
\item[{\sphinxcode{\sphinxupquote{progressbar}}}] \leavevmode
displays \sphinxcode{\sphinxupquote{float}} fields as a progress bar.

\item[{\sphinxcode{\sphinxupquote{many2onebutton}}}] \leavevmode
replaces the m2o field’s value by a checkmark if the field is
filled, and a cross if it is not

\item[{\sphinxcode{\sphinxupquote{handle}}}] \leavevmode
for \sphinxcode{\sphinxupquote{sequence}} fields, instead of displaying the field’s value
just displays a drag\&drop icon

\end{description}

\item[{\sphinxcode{\sphinxupquote{sum}}, \sphinxcode{\sphinxupquote{avg}}}] \leavevmode
displays the corresponding aggregate at the bottom of the column. The
aggregation is only computed on \sphinxstyleemphasis{currently displayed} records. The
aggregation operation must match the corresponding field’s
\sphinxcode{\sphinxupquote{group\_operator}}

\item[{\sphinxcode{\sphinxupquote{attrs}}}] \leavevmode
dynamic attributes based on record values. Only effects the current
field, so e.g. \sphinxcode{\sphinxupquote{invisible}} will hide the field but leave the same
field of other records visible, it will not hide the column itself

\end{description}

\begin{sphinxadmonition}{note}{Note:}
if the list view is \sphinxcode{\sphinxupquote{editable}}, any field attribute from the
{\hyperref[\detokenize{reference/views:reference-views-form}]{\sphinxcrossref{\DUrole{std,std-ref}{form view}}}} is also valid and will
be used when setting up the inline form view
\end{sphinxadmonition}

\end{description}


\subsection{Forms}
\label{\detokenize{reference/views:forms}}\label{\detokenize{reference/views:reference-views-form}}
Form views are used to display the data from a single record. Their root
element is \sphinxcode{\sphinxupquote{\textless{}form\textgreater{}}}. They are composed of regular \sphinxhref{http://en.wikipedia.org/wiki/HTML}{HTML} with additional
structural and semantic components.


\subsubsection{Structural components}
\label{\detokenize{reference/views:structural-components}}
Structural components provide structure or “visual” features with little
logic. They are used as elements or sets of elements in form views.
\begin{description}
\item[{\sphinxcode{\sphinxupquote{notebook}}}] \leavevmode
defines a tabbed section. Each tab is defined through a \sphinxcode{\sphinxupquote{page}} child
element. Pages can have the following attributes:
\begin{description}
\item[{\sphinxcode{\sphinxupquote{string}} (required)}] \leavevmode
the title of the tab

\item[{\sphinxcode{\sphinxupquote{accesskey}}}] \leavevmode
an HTML \sphinxhref{http://www.w3.org/TR/html5/editing.html\#the-accesskey-attribute}{accesskey}

\item[{\sphinxcode{\sphinxupquote{attrs}}}] \leavevmode
standard dynamic attributes based on record values

\end{description}

\item[{\sphinxcode{\sphinxupquote{group}}}] \leavevmode
used to define column layouts in forms. By default, groups define 2 columns
and most direct children of groups take a single column. \sphinxcode{\sphinxupquote{field}} direct
children of groups display a label by default, and the label and the field
itself have a colspan of 1 each.

The number of columns in a \sphinxcode{\sphinxupquote{group}} can be customized using the \sphinxcode{\sphinxupquote{col}}
attribute, the number of columns taken by an element can be customized using
\sphinxcode{\sphinxupquote{colspan}}.

Children are laid out horizontally (tries to fill the next column before
changing row).

Groups can have a \sphinxcode{\sphinxupquote{string}} attribute, which is displayed as the group’s
title

\item[{\sphinxcode{\sphinxupquote{newline}}}] \leavevmode
only useful within \sphinxcode{\sphinxupquote{group}} elements, ends the current row early and
immediately switches to a new row (without filling any remaining column
beforehand)

\item[{\sphinxcode{\sphinxupquote{separator}}}] \leavevmode
small horizontal spacing, with a \sphinxcode{\sphinxupquote{string}} attribute behaves as a section
title

\item[{\sphinxcode{\sphinxupquote{sheet}}}] \leavevmode
can be used as a direct child to \sphinxcode{\sphinxupquote{form}} for a narrower and more responsive
form layout

\item[{\sphinxcode{\sphinxupquote{header}}}] \leavevmode
combined with \sphinxcode{\sphinxupquote{sheet}}, provides a full-width location above the sheet
itself, generally used to display workflow buttons and status widgets

\end{description}


\subsubsection{Semantic components}
\label{\detokenize{reference/views:semantic-components}}
Semantic components tie into and allow interaction with the Odoo
system. Available semantic components are:
\begin{description}
\item[{\sphinxcode{\sphinxupquote{button}}}] \leavevmode
call into the Odoo system, similar to {\hyperref[\detokenize{reference/views:reference-views-list-button}]{\sphinxcrossref{\DUrole{std,std-ref}{list view buttons}}}}. In addition, the following attribute can be
specified:
\begin{description}
\item[{\sphinxcode{\sphinxupquote{special}}}] \leavevmode
for form views opened in dialogs: \sphinxcode{\sphinxupquote{save}} to save the record and close the
dialog, \sphinxcode{\sphinxupquote{cancel}} to close the dialog without saving.

\end{description}

\item[{\sphinxcode{\sphinxupquote{field}}}] \leavevmode
renders (and allow edition of, possibly) a single field of the current
record. Possible attributes are:
\begin{description}
\item[{\sphinxcode{\sphinxupquote{name}} (mandatory)}] \leavevmode
the name of the field to render

\item[{\sphinxcode{\sphinxupquote{widget}}}] \leavevmode
fields have a default rendering based on their type
(e.g. {\hyperref[\detokenize{reference/orm:odoo.fields.Char}]{\sphinxcrossref{\sphinxcode{\sphinxupquote{Char}}}}},
{\hyperref[\detokenize{reference/orm:odoo.fields.Many2one}]{\sphinxcrossref{\sphinxcode{\sphinxupquote{Many2one}}}}}). The \sphinxcode{\sphinxupquote{widget}} attributes allows using
a different rendering method and context.

\item[{\sphinxcode{\sphinxupquote{options}}}] \leavevmode
JSON object specifying configuration option for the field’s widget
(including default widgets)

\item[{\sphinxcode{\sphinxupquote{class}}}] \leavevmode
HTML class to set on the generated element, common field classes are:
\begin{description}
\item[{\sphinxcode{\sphinxupquote{oe\_inline}}}] \leavevmode
prevent the usual line break following fields

\item[{\sphinxcode{\sphinxupquote{oe\_left}}, \sphinxcode{\sphinxupquote{oe\_right}}}] \leavevmode
\sphinxhref{https://developer.mozilla.org/en-US/docs/Web/CSS/float}{floats} the field to the corresponding direction

\item[{\sphinxcode{\sphinxupquote{oe\_read\_only}}, \sphinxcode{\sphinxupquote{oe\_edit\_only}}}] \leavevmode
only displays the field in the corresponding form mode

\item[{\sphinxcode{\sphinxupquote{oe\_no\_button}}}] \leavevmode
avoids displaying the navigation button in a
{\hyperref[\detokenize{reference/orm:odoo.fields.Many2one}]{\sphinxcrossref{\sphinxcode{\sphinxupquote{Many2one}}}}}

\item[{\sphinxcode{\sphinxupquote{oe\_avatar}}}] \leavevmode
for image fields, displays images as “avatar” (square, 90x90 maximum
size, some image decorations)

\end{description}

\item[{\sphinxcode{\sphinxupquote{groups}}}] \leavevmode
only displays the field for specific users

\item[{\sphinxcode{\sphinxupquote{on\_change}}}] \leavevmode
calls the specified method when this field’s value is edited, can generate
update other fields or display warnings for the user

\DUrole{versionmodified}{Deprecated since version 8.0: }Use {\hyperref[\detokenize{reference/orm:odoo.api.onchange}]{\sphinxcrossref{\sphinxcode{\sphinxupquote{odoo.api.onchange()}}}}} on the model

\item[{\sphinxcode{\sphinxupquote{attrs}}}] \leavevmode
dynamic meta-parameters based on record values

\item[{\sphinxcode{\sphinxupquote{domain}}}] \leavevmode
for relational fields only, filters to apply when displaying existing
records for selection

\item[{\sphinxcode{\sphinxupquote{context}}}] \leavevmode
for relational fields only, context to pass when fetching possible values

\item[{\sphinxcode{\sphinxupquote{readonly}}}] \leavevmode
display the field in both readonly and edition mode, but never make it
editable

\item[{\sphinxcode{\sphinxupquote{required}}}] \leavevmode
generates an error and prevents saving the record if the field doesn’t
have a value

\item[{\sphinxcode{\sphinxupquote{nolabel}}}] \leavevmode
don’t automatically display the field’s label, only makes sense if the
field is a direct child of a \sphinxcode{\sphinxupquote{group}} element

\item[{\sphinxcode{\sphinxupquote{placeholder}}}] \leavevmode
help message to display in \sphinxstyleemphasis{empty} fields. Can replace field labels in
complex forms. \sphinxstyleemphasis{Should not} be an example of data as users are liable to
confuse placeholder text with filled fields

\item[{\sphinxcode{\sphinxupquote{mode}}}] \leavevmode
for {\hyperref[\detokenize{reference/orm:odoo.fields.One2many}]{\sphinxcrossref{\sphinxcode{\sphinxupquote{One2many}}}}}, display mode (view type) to use for
the field’s linked records. One of \sphinxcode{\sphinxupquote{tree}}, \sphinxcode{\sphinxupquote{form}}, \sphinxcode{\sphinxupquote{kanban}} or
\sphinxcode{\sphinxupquote{graph}}. The default is \sphinxcode{\sphinxupquote{tree}} (a list display)

\item[{\sphinxcode{\sphinxupquote{help}}}] \leavevmode
tooltip displayed for users when hovering the field or its label

\item[{\sphinxcode{\sphinxupquote{filename}}}] \leavevmode
for binary fields, name of the related field providing the name of the
file

\item[{\sphinxcode{\sphinxupquote{password}}}] \leavevmode
indicates that a {\hyperref[\detokenize{reference/orm:odoo.fields.Char}]{\sphinxcrossref{\sphinxcode{\sphinxupquote{Char}}}}} field stores a password and
that its data shouldn’t be displayed

\end{description}

\end{description}


\subsubsection{Business Views guidelines}
\label{\detokenize{reference/views:business-views-guidelines}}\label{\detokenize{reference/views:index-2}}
Business views are targeted at regular users, not advanced users.  Examples
are: Opportunities, Products, Partners, Tasks, Projects, etc.

\noindent\sphinxincludegraphics{{oppreadonly}.png}

In general, a business view is composed of
\begin{enumerate}
\item {} 
a status bar on top (with technical or business flow),

\item {} 
a sheet in the middle (the form itself),

\item {} 
a bottom part with History and Comments.

\end{enumerate}

Technically, the new form views are structured as follows in XML:

\fvset{hllines={, ,}}%
\begin{sphinxVerbatim}[commandchars=\\\{\}]
\PYG{n+nt}{\PYGZlt{}form}\PYG{n+nt}{\PYGZgt{}}
    \PYG{n+nt}{\PYGZlt{}header}\PYG{n+nt}{\PYGZgt{}} ... content of the status bar  ... \PYG{n+nt}{\PYGZlt{}/header\PYGZgt{}}
    \PYG{n+nt}{\PYGZlt{}sheet}\PYG{n+nt}{\PYGZgt{}}  ... content of the sheet       ... \PYG{n+nt}{\PYGZlt{}/sheet\PYGZgt{}}
    \PYG{n+nt}{\PYGZlt{}div} \PYG{n+na}{class=}\PYG{l+s}{\PYGZdq{}oe\PYGZus{}chatter\PYGZdq{}}\PYG{n+nt}{\PYGZgt{}} ... content of the bottom part ... \PYG{n+nt}{\PYGZlt{}/div\PYGZgt{}}
\PYG{n+nt}{\PYGZlt{}/form\PYGZgt{}}
\end{sphinxVerbatim}


\paragraph{The Status Bar}
\label{\detokenize{reference/views:the-status-bar}}
The purpose of the status bar is to show the status of the current record and
the action buttons.

\noindent\sphinxincludegraphics{{status}.png}


\subparagraph{The Buttons}
\label{\detokenize{reference/views:the-buttons}}
The order of buttons follows the business flow. For instance, in a sale order,
the logical steps are:
\begin{enumerate}
\item {} 
Send the quotation

\item {} 
Confirm the quotation

\item {} 
Create the final invoice

\item {} 
Send the goods

\end{enumerate}

Highlighted buttons (in red by default) emphasize the logical next step, to
help the user. It is usually the first active button. On the other hand,
\sphinxmenuselection{cancel} buttons \sphinxstyleemphasis{must} remain grey (normal).  For instance, in
Invoice the button \sphinxmenuselection{Refund} must never be red.

Technically, buttons are highlighted by adding the class “oe\_highlight”:

\fvset{hllines={, ,}}%
\begin{sphinxVerbatim}[commandchars=\\\{\}]
\PYG{n+nt}{\PYGZlt{}button} \PYG{n+na}{class=}\PYG{l+s}{\PYGZdq{}oe\PYGZus{}highlight\PYGZdq{}} \PYG{n+na}{name=}\PYG{l+s}{\PYGZdq{}...\PYGZdq{}} \PYG{n+na}{type=}\PYG{l+s}{\PYGZdq{}...\PYGZdq{}} \PYG{n+na}{states=}\PYG{l+s}{\PYGZdq{}...\PYGZdq{}}\PYG{n+nt}{/\PYGZgt{}}
\end{sphinxVerbatim}


\subparagraph{The Status}
\label{\detokenize{reference/views:the-status}}
Uses the \sphinxcode{\sphinxupquote{statusbar}} widget, and shows the current state in red. States
common to all flows (for instance, a sale order begins as a quotation, then we
send it, then it becomes a full sale order, and finally it is done) should be
visible at all times but exceptions or states depending on particular sub-flow
should only be visible when current.

\noindent\sphinxincludegraphics{{status1}.png}

\noindent\sphinxincludegraphics{{status2}.png}

The states are shown following the order used in the field (the list in a
selection field, etc). States that are always visible are specified with the
attribute \sphinxcode{\sphinxupquote{statusbar\_visible}}.

\fvset{hllines={, ,}}%
\begin{sphinxVerbatim}[commandchars=\\\{\}]
\PYG{n+nt}{\PYGZlt{}field} \PYG{n+na}{name=}\PYG{l+s}{\PYGZdq{}state\PYGZdq{}} \PYG{n+na}{widget=}\PYG{l+s}{\PYGZdq{}statusbar\PYGZdq{}}
    \PYG{n+na}{statusbar\PYGZus{}visible=}\PYG{l+s}{\PYGZdq{}draft,sent,progress,invoiced,done\PYGZdq{}} \PYG{n+nt}{/\PYGZgt{}}
\end{sphinxVerbatim}


\paragraph{The Sheet}
\label{\detokenize{reference/views:the-sheet}}
All business views should look like a printed sheet:

\noindent\sphinxincludegraphics{{sheet}.png}
\begin{enumerate}
\item {} 
Elements inside a \sphinxcode{\sphinxupquote{\textless{}form\textgreater{}}} or \sphinxcode{\sphinxupquote{\textless{}page\textgreater{}}} do not define groups, elements
inside them are laid out according to normal HTML rules. They content can
be explicitly grouped using \sphinxcode{\sphinxupquote{\textless{}group\textgreater{}}} or regular \sphinxcode{\sphinxupquote{\textless{}div\textgreater{}}} elements.

\item {} 
By default, the element \sphinxcode{\sphinxupquote{\textless{}group\textgreater{}}} defines two columns inside, unless an
attribute \sphinxcode{\sphinxupquote{col="n"}} is used.  The columns have the same width (1/n th of
the group’s width). Use a \sphinxcode{\sphinxupquote{\textless{}group\textgreater{}}} element to produce a column of fields.

\item {} 
To give a title to a section, add a \sphinxcode{\sphinxupquote{string}} attribute to a \sphinxcode{\sphinxupquote{\textless{}group\textgreater{}}} element:

\fvset{hllines={, ,}}%
\begin{sphinxVerbatim}[commandchars=\\\{\}]
\PYG{n+nt}{\PYGZlt{}group} \PYG{n+na}{string=}\PYG{l+s}{\PYGZdq{}Time\PYGZhy{}sensitive operations\PYGZdq{}}\PYG{n+nt}{\PYGZgt{}}
\end{sphinxVerbatim}

this replaces the former use of \sphinxcode{\sphinxupquote{\textless{}separator string="XXX"/\textgreater{}}}.

\item {} 
The \sphinxcode{\sphinxupquote{\textless{}field\textgreater{}}} element does not produce a label, except as direct children
of a \sphinxcode{\sphinxupquote{\textless{}group\textgreater{}}} element%
\begin{footnote}[1]\sphinxAtStartFootnote
for backwards compatibility reasons
%
\end{footnote}.  Use \sphinxcode{\sphinxupquote{\textless{}label
for="\sphinxstyleemphasis{field\_name}\textgreater{}}} to produce a label of a field.

\end{enumerate}


\subparagraph{Sheet Headers}
\label{\detokenize{reference/views:sheet-headers}}
Some sheets have headers with one or more fields, and the labels of those
fields are only shown in edit mode.


\begin{savenotes}\sphinxattablestart
\centering
\begin{tabulary}{\linewidth}[t]{|T|T|}
\hline
\sphinxstyletheadfamily 
View mode
&\sphinxstyletheadfamily 
Edit mode
\\
\hline
\noindent\sphinxincludegraphics{{header}.png}
&
\noindent\sphinxincludegraphics{{header2}.png}
\\
\hline
\end{tabulary}
\par
\sphinxattableend\end{savenotes}

Use HTML text, \sphinxcode{\sphinxupquote{\textless{}div\textgreater{}}}, \sphinxcode{\sphinxupquote{\textless{}h1\textgreater{}}}, \sphinxcode{\sphinxupquote{\textless{}h2\textgreater{}}}… to produce nice headers, and
\sphinxcode{\sphinxupquote{\textless{}label\textgreater{}}} with the class \sphinxcode{\sphinxupquote{oe\_edit\_only}} to only display the field’s label
in edit mode. The class \sphinxcode{\sphinxupquote{oe\_inline}} will make fields inline (instead of
blocks): content following the field will be displayed on the same line rather
than on the line below it. The form above is produced by the following XML:

\fvset{hllines={, ,}}%
\begin{sphinxVerbatim}[commandchars=\\\{\}]
\PYG{n+nt}{\PYGZlt{}label} \PYG{n+na}{for=}\PYG{l+s}{\PYGZdq{}name\PYGZdq{}} \PYG{n+na}{class=}\PYG{l+s}{\PYGZdq{}oe\PYGZus{}edit\PYGZus{}only\PYGZdq{}}\PYG{n+nt}{/\PYGZgt{}}
\PYG{n+nt}{\PYGZlt{}h1}\PYG{n+nt}{\PYGZgt{}}\PYG{n+nt}{\PYGZlt{}field} \PYG{n+na}{name=}\PYG{l+s}{\PYGZdq{}name\PYGZdq{}}\PYG{n+nt}{/\PYGZgt{}}\PYG{n+nt}{\PYGZlt{}/h1\PYGZgt{}}

\PYG{n+nt}{\PYGZlt{}label} \PYG{n+na}{for=}\PYG{l+s}{\PYGZdq{}planned\PYGZus{}revenue\PYGZdq{}} \PYG{n+na}{class=}\PYG{l+s}{\PYGZdq{}oe\PYGZus{}edit\PYGZus{}only\PYGZdq{}}\PYG{n+nt}{/\PYGZgt{}}
\PYG{n+nt}{\PYGZlt{}h2}\PYG{n+nt}{\PYGZgt{}}
    \PYG{n+nt}{\PYGZlt{}field} \PYG{n+na}{name=}\PYG{l+s}{\PYGZdq{}planned\PYGZus{}revenue\PYGZdq{}} \PYG{n+na}{class=}\PYG{l+s}{\PYGZdq{}oe\PYGZus{}inline\PYGZdq{}}\PYG{n+nt}{/\PYGZgt{}}
    \PYG{n+nt}{\PYGZlt{}field} \PYG{n+na}{name=}\PYG{l+s}{\PYGZdq{}company\PYGZus{}currency\PYGZdq{}} \PYG{n+na}{class=}\PYG{l+s}{\PYGZdq{}oe\PYGZus{}inline oe\PYGZus{}edit\PYGZus{}only\PYGZdq{}}\PYG{n+nt}{/\PYGZgt{}} at
    \PYG{n+nt}{\PYGZlt{}field} \PYG{n+na}{name=}\PYG{l+s}{\PYGZdq{}probability\PYGZdq{}} \PYG{n+na}{class=}\PYG{l+s}{\PYGZdq{}oe\PYGZus{}inline\PYGZdq{}}\PYG{n+nt}{/\PYGZgt{}} \PYGZpc{} success rate
\PYG{n+nt}{\PYGZlt{}/h2\PYGZgt{}}
\end{sphinxVerbatim}


\subparagraph{Button Box}
\label{\detokenize{reference/views:button-box}}
Many relevant actions or links can be displayed in the form. For example, in
Opportunity form, the actions “Schedule a Call” and “Schedule a Meeting” have
an important place in the use of the CRM. Instead of placing them in the
“More” menu, put them directly in the sheet as buttons (on the top) to make
them more visible and more easily accessible.

\noindent\sphinxincludegraphics{{header3}.png}

Technically, the buttons are placed inside a \sphinxcode{\sphinxupquote{\textless{}div\textgreater{}}} to group them as a
block on the top of the sheet.

\fvset{hllines={, ,}}%
\begin{sphinxVerbatim}[commandchars=\\\{\}]
\PYG{n+nt}{\PYGZlt{}div} \PYG{n+na}{class=}\PYG{l+s}{\PYGZdq{}oe\PYGZus{}button\PYGZus{}box\PYGZdq{}} \PYG{n+na}{name=}\PYG{l+s}{\PYGZdq{}button\PYGZus{}box\PYGZdq{}}\PYG{n+nt}{\PYGZgt{}}
    \PYG{n+nt}{\PYGZlt{}button} \PYG{n+na}{string=}\PYG{l+s}{\PYGZdq{}Schedule/Log Call\PYGZdq{}} \PYG{n+na}{name=}\PYG{l+s}{\PYGZdq{}...\PYGZdq{}} \PYG{n+na}{type=}\PYG{l+s}{\PYGZdq{}action\PYGZdq{}}\PYG{n+nt}{/\PYGZgt{}}
    \PYG{n+nt}{\PYGZlt{}button} \PYG{n+na}{string=}\PYG{l+s}{\PYGZdq{}Schedule Meeting\PYGZdq{}} \PYG{n+na}{name=}\PYG{l+s}{\PYGZdq{}action\PYGZus{}makeMeeting\PYGZdq{}} \PYG{n+na}{type=}\PYG{l+s}{\PYGZdq{}object\PYGZdq{}}\PYG{n+nt}{/\PYGZgt{}}
\PYG{n+nt}{\PYGZlt{}/div\PYGZgt{}}
\end{sphinxVerbatim}


\subparagraph{Groups and Titles}
\label{\detokenize{reference/views:groups-and-titles}}
A column of fields is now produced with a \sphinxcode{\sphinxupquote{\textless{}group\textgreater{}}} element, with an
optional title.

\noindent\sphinxincludegraphics{{screenshot-03}.png}

\fvset{hllines={, ,}}%
\begin{sphinxVerbatim}[commandchars=\\\{\}]
\PYG{n+nt}{\PYGZlt{}group} \PYG{n+na}{string=}\PYG{l+s}{\PYGZdq{}Payment Options\PYGZdq{}}\PYG{n+nt}{\PYGZgt{}}
    \PYG{n+nt}{\PYGZlt{}field} \PYG{n+na}{name=}\PYG{l+s}{\PYGZdq{}writeoff\PYGZus{}amount\PYGZdq{}}\PYG{n+nt}{/\PYGZgt{}}
    \PYG{n+nt}{\PYGZlt{}field} \PYG{n+na}{name=}\PYG{l+s}{\PYGZdq{}payment\PYGZus{}option\PYGZdq{}}\PYG{n+nt}{/\PYGZgt{}}
\PYG{n+nt}{\PYGZlt{}/group\PYGZgt{}}
\end{sphinxVerbatim}

It is recommended to have two columns of fields on the form. For this, simply
put the \sphinxcode{\sphinxupquote{\textless{}group\textgreater{}}} elements that contain the fields inside a top-level
\sphinxcode{\sphinxupquote{\textless{}group\textgreater{}}} element.

To make {\hyperref[\detokenize{reference/views:reference-views-inheritance}]{\sphinxcrossref{\DUrole{std,std-ref}{view extension}}}} simpler, it is
recommended to put a \sphinxcode{\sphinxupquote{name}} attribute on \sphinxcode{\sphinxupquote{\textless{}group\textgreater{}}} elements, so new fields
can easily be added at the right place.


\subparagraph{Special Case: Subtotals}
\label{\detokenize{reference/views:special-case-subtotals}}
Some classes are defined to render subtotals like in invoice forms:

\noindent\sphinxincludegraphics{{screenshot-00}.png}

\fvset{hllines={, ,}}%
\begin{sphinxVerbatim}[commandchars=\\\{\}]
\PYG{n+nt}{\PYGZlt{}group} \PYG{n+na}{class=}\PYG{l+s}{\PYGZdq{}oe\PYGZus{}subtotal\PYGZus{}footer\PYGZdq{}}\PYG{n+nt}{\PYGZgt{}}
    \PYG{n+nt}{\PYGZlt{}field} \PYG{n+na}{name=}\PYG{l+s}{\PYGZdq{}amount\PYGZus{}untaxed\PYGZdq{}}\PYG{n+nt}{/\PYGZgt{}}
    \PYG{n+nt}{\PYGZlt{}field} \PYG{n+na}{name=}\PYG{l+s}{\PYGZdq{}amount\PYGZus{}tax\PYGZdq{}}\PYG{n+nt}{/\PYGZgt{}}
    \PYG{n+nt}{\PYGZlt{}field} \PYG{n+na}{name=}\PYG{l+s}{\PYGZdq{}amount\PYGZus{}total\PYGZdq{}} \PYG{n+na}{class=}\PYG{l+s}{\PYGZdq{}oe\PYGZus{}subtotal\PYGZus{}footer\PYGZus{}separator\PYGZdq{}}\PYG{n+nt}{/\PYGZgt{}}
    \PYG{n+nt}{\PYGZlt{}field} \PYG{n+na}{name=}\PYG{l+s}{\PYGZdq{}residual\PYGZdq{}} \PYG{n+na}{style=}\PYG{l+s}{\PYGZdq{}margin\PYGZhy{}top: 10px\PYGZdq{}}\PYG{n+nt}{/\PYGZgt{}}
\PYG{n+nt}{\PYGZlt{}/group\PYGZgt{}}
\end{sphinxVerbatim}


\subparagraph{Placeholders and Inline Fields}
\label{\detokenize{reference/views:placeholders-and-inline-fields}}
Sometimes field labels make the form too complex. One can omit field labels,
and instead put a placeholder inside the field. The placeholder text is
visible only when the field is empty. The placeholder should tell what to
place inside the field, it \sphinxstyleemphasis{must not} be an example as they are often confused
with filled data.

One can also group fields together by rendering them “inline” inside an
explicit block element like \sphinxcode{\sphinxupquote{\textless{}div\textgreater{}}}. This allows grouping semantically
related fields as if they were a single (composite) fields.

The following example, taken from the \sphinxstyleemphasis{Leads} form, shows both placeholders and
inline fields (zip and city).


\begin{savenotes}\sphinxattablestart
\centering
\begin{tabulary}{\linewidth}[t]{|T|T|}
\hline
\sphinxstyletheadfamily 
Edit mode
&\sphinxstyletheadfamily 
View mode
\\
\hline
\noindent\sphinxincludegraphics{{placeholder}.png}
&
\noindent\sphinxincludegraphics{{screenshot-01}.png}
\\
\hline
\end{tabulary}
\par
\sphinxattableend\end{savenotes}

\fvset{hllines={, ,}}%
\begin{sphinxVerbatim}[commandchars=\\\{\}]
\PYG{n+nt}{\PYGZlt{}group}\PYG{n+nt}{\PYGZgt{}}
    \PYG{n+nt}{\PYGZlt{}label} \PYG{n+na}{for=}\PYG{l+s}{\PYGZdq{}street\PYGZdq{}} \PYG{n+na}{string=}\PYG{l+s}{\PYGZdq{}Address\PYGZdq{}}\PYG{n+nt}{/\PYGZgt{}}
    \PYG{n+nt}{\PYGZlt{}div}\PYG{n+nt}{\PYGZgt{}}
        \PYG{n+nt}{\PYGZlt{}field} \PYG{n+na}{name=}\PYG{l+s}{\PYGZdq{}street\PYGZdq{}} \PYG{n+na}{placeholder=}\PYG{l+s}{\PYGZdq{}Street...\PYGZdq{}}\PYG{n+nt}{/\PYGZgt{}}
        \PYG{n+nt}{\PYGZlt{}field} \PYG{n+na}{name=}\PYG{l+s}{\PYGZdq{}street2\PYGZdq{}}\PYG{n+nt}{/\PYGZgt{}}
        \PYG{n+nt}{\PYGZlt{}div}\PYG{n+nt}{\PYGZgt{}}
            \PYG{n+nt}{\PYGZlt{}field} \PYG{n+na}{name=}\PYG{l+s}{\PYGZdq{}zip\PYGZdq{}} \PYG{n+na}{class=}\PYG{l+s}{\PYGZdq{}oe\PYGZus{}inline\PYGZdq{}} \PYG{n+na}{placeholder=}\PYG{l+s}{\PYGZdq{}ZIP\PYGZdq{}}\PYG{n+nt}{/\PYGZgt{}}
            \PYG{n+nt}{\PYGZlt{}field} \PYG{n+na}{name=}\PYG{l+s}{\PYGZdq{}city\PYGZdq{}} \PYG{n+na}{class=}\PYG{l+s}{\PYGZdq{}oe\PYGZus{}inline\PYGZdq{}} \PYG{n+na}{placeholder=}\PYG{l+s}{\PYGZdq{}City\PYGZdq{}}\PYG{n+nt}{/\PYGZgt{}}
        \PYG{n+nt}{\PYGZlt{}/div\PYGZgt{}}
        \PYG{n+nt}{\PYGZlt{}field} \PYG{n+na}{name=}\PYG{l+s}{\PYGZdq{}state\PYGZus{}id\PYGZdq{}} \PYG{n+na}{placeholder=}\PYG{l+s}{\PYGZdq{}State\PYGZdq{}}\PYG{n+nt}{/\PYGZgt{}}
        \PYG{n+nt}{\PYGZlt{}field} \PYG{n+na}{name=}\PYG{l+s}{\PYGZdq{}country\PYGZus{}id\PYGZdq{}} \PYG{n+na}{placeholder=}\PYG{l+s}{\PYGZdq{}Country\PYGZdq{}}\PYG{n+nt}{/\PYGZgt{}}
    \PYG{n+nt}{\PYGZlt{}/div\PYGZgt{}}
\PYG{n+nt}{\PYGZlt{}/group\PYGZgt{}}
\end{sphinxVerbatim}


\subparagraph{Images}
\label{\detokenize{reference/views:images}}
Images, like avatars, should be displayed on the right of the sheet.  The
product form looks like:

\noindent\sphinxincludegraphics{{screenshot-02}.png}

The form above contains a \textless{}sheet\textgreater{} element that starts with:

\fvset{hllines={, ,}}%
\begin{sphinxVerbatim}[commandchars=\\\{\}]
\PYG{n+nt}{\PYGZlt{}field} \PYG{n+na}{name=}\PYG{l+s}{\PYGZdq{}product\PYGZus{}image\PYGZdq{}} \PYG{n+na}{widget=}\PYG{l+s}{\PYGZdq{}image\PYGZdq{}} \PYG{n+na}{class=}\PYG{l+s}{\PYGZdq{}oe\PYGZus{}avatar oe\PYGZus{}right\PYGZdq{}}\PYG{n+nt}{/\PYGZgt{}}
\end{sphinxVerbatim}


\subparagraph{Tags}
\label{\detokenize{reference/views:tags}}
Most {\hyperref[\detokenize{reference/orm:odoo.fields.Many2many}]{\sphinxcrossref{\sphinxcode{\sphinxupquote{Many2many}}}}} fields, like categories, are better
rendered as a list of tags. Use the widget \sphinxcode{\sphinxupquote{many2many\_tags}} for this:

\noindent\sphinxincludegraphics{{screenshot-04}.png}

\fvset{hllines={, ,}}%
\begin{sphinxVerbatim}[commandchars=\\\{\}]
\PYG{n+nt}{\PYGZlt{}field} \PYG{n+na}{name=}\PYG{l+s}{\PYGZdq{}category\PYGZus{}id\PYGZdq{}} \PYG{n+na}{widget=}\PYG{l+s}{\PYGZdq{}many2many\PYGZus{}tags\PYGZdq{}}\PYG{n+nt}{/\PYGZgt{}}
\end{sphinxVerbatim}


\subsubsection{Configuration forms guidelines}
\label{\detokenize{reference/views:configuration-forms-guidelines}}
Examples of configuration forms: Stages, Leave Type, etc.  This concerns all
menu items under Configuration of each application (like Sales/Configuration).

\noindent\sphinxincludegraphics{{nosheet}.png}
\begin{enumerate}
\item {} 
no header (because no state, no workflow, no button)

\item {} 
no sheet

\end{enumerate}


\subsubsection{Dialog forms guidelines}
\label{\detokenize{reference/views:dialog-forms-guidelines}}
Example: “Schedule a Call” from an opportunity.

\noindent\sphinxincludegraphics{{wizard-popup}.png}
\begin{enumerate}
\item {} 
avoid separators (the title is already in the popup title bar, so another
separator is not relevant)

\item {} 
avoid cancel buttons (user generally close the popup window to get the same
effect)

\item {} 
action buttons must be highlighted (red)

\item {} 
when there is a text area, use a placeholder instead of a label or a
separator

\item {} 
like in regular form views, put buttons in the \textless{}header\textgreater{} element

\end{enumerate}


\subsubsection{Configuration Wizards guidelines}
\label{\detokenize{reference/views:configuration-wizards-guidelines}}
Example: Settings / Configuration / Sales.
\begin{enumerate}
\item {} 
always in line (no popup)

\item {} 
no sheet

\item {} 
keep the cancel button (users cannot close the window)

\item {} 
the button “Apply” must be red

\end{enumerate}


\subsection{Graphs}
\label{\detokenize{reference/views:graphs}}\label{\detokenize{reference/views:reference-views-graph}}
The graph view is used to visualize aggregations over a number of records or
record groups. Its root element is \sphinxcode{\sphinxupquote{\textless{}graph\textgreater{}}} which can take the following
attributes:
\begin{description}
\item[{\sphinxcode{\sphinxupquote{type}}}] \leavevmode
one of \sphinxcode{\sphinxupquote{bar}} (default), \sphinxcode{\sphinxupquote{pie}} and \sphinxcode{\sphinxupquote{line}}, the type of graph to use

\item[{\sphinxcode{\sphinxupquote{stacked}}}] \leavevmode
only used for \sphinxcode{\sphinxupquote{bar}} charts. If present and set to \sphinxcode{\sphinxupquote{True}}, stacks bars
within a group

\end{description}

The only allowed element within a graph view is \sphinxcode{\sphinxupquote{field}} which can have the
following attributes:
\begin{description}
\item[{\sphinxcode{\sphinxupquote{name}} (required)}] \leavevmode
the name of a field to use in a graph view. If used for grouping (rather
than aggregating)

\item[{\sphinxcode{\sphinxupquote{type}}}] \leavevmode
indicates whether the field should be used as a grouping criteria or as an
aggregated value within a group. Possible values are:
\begin{description}
\item[{\sphinxcode{\sphinxupquote{row}} (default)}] \leavevmode
groups by the specified field. All graph types support at least one level
of grouping, some may support more. For pivot views, each group gets its
own row.

\item[{\sphinxcode{\sphinxupquote{col}}}] \leavevmode
only used by pivot tables, creates column-wise groups

\item[{\sphinxcode{\sphinxupquote{measure}}}] \leavevmode
field to aggregate within a group

\end{description}

\item[{\sphinxcode{\sphinxupquote{interval}}}] \leavevmode
on date and datetime fields, groups by the specified interval (\sphinxcode{\sphinxupquote{day}},
\sphinxcode{\sphinxupquote{week}}, \sphinxcode{\sphinxupquote{month}}, \sphinxcode{\sphinxupquote{quarter}} or \sphinxcode{\sphinxupquote{year}}) instead of grouping on the
specific datetime (fixed second resolution) or date (fixed day resolution).

\end{description}

\begin{sphinxadmonition}{warning}{Warning:}
graph view aggregations are performed on database content, non-stored
function fields can not be used in graph views
\end{sphinxadmonition}


\subsubsection{Pivots}
\label{\detokenize{reference/views:pivots}}
The pivot view is used to visualize aggregations as a \sphinxhref{http://en.wikipedia.org/wiki/Pivot\_table}{pivot table}. Its root
element is \sphinxcode{\sphinxupquote{\textless{}pivot\textgreater{}}} which can take the following attributes:
\begin{description}
\item[{\sphinxcode{\sphinxupquote{disable\_linking}}}] \leavevmode
Set to \sphinxcode{\sphinxupquote{True}} to remove table cell’s links to list view.

\item[{\sphinxcode{\sphinxupquote{display\_quantity}}}] \leavevmode
Set to \sphinxcode{\sphinxupquote{true}} to display the Quantity column by default.

\end{description}

The elements allowed within a pivot view are the same as for the graph view.


\subsection{Kanban}
\label{\detokenize{reference/views:reference-views-kanban}}\label{\detokenize{reference/views:kanban}}
The kanban view is a \sphinxhref{http://en.wikipedia.org/wiki/Kanban\_board}{kanban board} visualisation: it displays records as
“cards”, halfway between a {\hyperref[\detokenize{reference/views:reference-views-list}]{\sphinxcrossref{\DUrole{std,std-ref}{list view}}}} and a
non-editable {\hyperref[\detokenize{reference/views:reference-views-form}]{\sphinxcrossref{\DUrole{std,std-ref}{form view}}}}. Records may be grouped
in columns for use in workflow visualisation or manipulation (e.g. tasks or
work-progress management), or ungrouped (used simply to visualize records).

The root element of the Kanban view is \sphinxcode{\sphinxupquote{\textless{}kanban\textgreater{}}}, it can use the following
attributes:
\begin{description}
\item[{\sphinxcode{\sphinxupquote{default\_group\_by}}}] \leavevmode
whether the kanban view should be grouped if no grouping is specified via
the action or the current search. Should be the name of the field to group
by when no grouping is otherwise specified

\item[{\sphinxcode{\sphinxupquote{default\_order}}}] \leavevmode
cards sorting order used if the user has not already sorted the records (via
the list view)

\item[{\sphinxcode{\sphinxupquote{class}}}] \leavevmode
adds HTML classes to the root HTML element of the Kanban view

\item[{\sphinxcode{\sphinxupquote{group\_create}}}] \leavevmode
whether the “Add a new column” bar is visible or not. Default: true.

\item[{\sphinxcode{\sphinxupquote{group\_delete}}}] \leavevmode
whether groups can be deleted via the context menu. Default: true.

\item[{\sphinxcode{\sphinxupquote{group\_edit}}}] \leavevmode
whether groups can be edited via the context menu. Default: true.

\item[{\sphinxcode{\sphinxupquote{quick\_create}}}] \leavevmode
whether it should be possible to create records without switching to the
form view. By default, \sphinxcode{\sphinxupquote{quick\_create}} is enabled when the Kanban view is
grouped, and disabled when not.

Set to \sphinxcode{\sphinxupquote{true}} to always enable it, and to \sphinxcode{\sphinxupquote{false}} to always disable it.

\end{description}

Possible children of the view element are:
\begin{description}
\item[{\sphinxcode{\sphinxupquote{field}}}] \leavevmode
declares fields to use in kanban \sphinxstyleemphasis{logic}. If the field is simply displayed in
the kanban view, it does not need to be pre-declared.

Possible attributes are:
\begin{description}
\item[{\sphinxcode{\sphinxupquote{name}} (required)}] \leavevmode
the name of the field to fetch

\end{description}

\item[{\sphinxcode{\sphinxupquote{progressbar}}}] \leavevmode
declares a progressbar element to put on top of kanban columns.

Possible attributes are:
\begin{description}
\item[{\sphinxcode{\sphinxupquote{field}} (required)}] \leavevmode
the name of the field whose values are used to subgroup column’s records in
the progressbar

\item[{\sphinxcode{\sphinxupquote{colors}} (required)}] \leavevmode
JSON mapping the above field values to either “danger”, “warning” or
“success” colors

\item[{\sphinxcode{\sphinxupquote{sum\_field}} (optional)}] \leavevmode
the name of the field whose column’s records’ values will be summed and
displayed next to the progressbar (if omitted, displays the total number of
records)

\end{description}

\item[{\sphinxcode{\sphinxupquote{templates}}}] \leavevmode
defines a list of {\hyperref[\detokenize{reference/qweb:reference-qweb}]{\sphinxcrossref{\DUrole{std,std-ref}{QWeb}}}} templates. Cards definition may be
split into multiple templates for clarity, but kanban views \sphinxstyleemphasis{must} define at
least one root template \sphinxcode{\sphinxupquote{kanban-box}}, which will be rendered once for each
record.

The kanban view uses mostly-standard {\hyperref[\detokenize{reference/qweb:reference-qweb-javascript}]{\sphinxcrossref{\DUrole{std,std-ref}{javascript qweb}}}} and provides the following context variables:
\begin{description}
\item[{\sphinxcode{\sphinxupquote{widget}}}] \leavevmode
the current {\hyperref[\detokenize{reference/javascript_api:KanbanRecord}]{\sphinxcrossref{\sphinxcode{\sphinxupquote{KanbanRecord()}}}}}, can be used to fetch some
meta-information. These methods are also available directly in the
template context and don’t need to be accessed via \sphinxcode{\sphinxupquote{widget}}

\item[{\sphinxcode{\sphinxupquote{record}}}] \leavevmode
an object with all the requested fields as its attributes. Each field has
two attributes \sphinxcode{\sphinxupquote{value}} and \sphinxcode{\sphinxupquote{raw\_value}}, the former is formatted
according to current user parameters, the latter is the direct value from
a {\hyperref[\detokenize{reference/orm:odoo.models.Model.read}]{\sphinxcrossref{\sphinxcode{\sphinxupquote{read()}}}}} (except for date and datetime fields
that are \sphinxhref{https://github.com/odoo/odoo/blob/a678bd4e/addons/web\_kanban/static/src/js/kanban\_record.js\#L102}{formatted according to user’s locale})

\item[{\sphinxcode{\sphinxupquote{read\_only\_mode}}}] \leavevmode
self-explanatory
\paragraph{buttons and fields}

While most of the Kanban templates are standard {\hyperref[\detokenize{reference/qweb:reference-qweb}]{\sphinxcrossref{\DUrole{std,std-ref}{QWeb}}}}, the
Kanban view processes \sphinxcode{\sphinxupquote{field}}, \sphinxcode{\sphinxupquote{button}} and \sphinxcode{\sphinxupquote{a}} elements specially:
\begin{itemize}
\item {} 
by default fields are replaced by their formatted value, unless they
match specific kanban view widgets

\item {} 
buttons and links with a \sphinxcode{\sphinxupquote{type}} attribute become perform Odoo-related
operations rather than their standard HTML function. Possible types are:
\begin{description}
\item[{\sphinxcode{\sphinxupquote{action}}, \sphinxcode{\sphinxupquote{object}}}] \leavevmode
standard behavior for {\hyperref[\detokenize{reference/views:reference-views-list-button}]{\sphinxcrossref{\DUrole{std,std-ref}{Odoo buttons}}}}, most attributes relevant to standard
Odoo buttons can be used.

\item[{\sphinxcode{\sphinxupquote{open}}}] \leavevmode
opens the card’s record in the form view in read-only mode

\item[{\sphinxcode{\sphinxupquote{edit}}}] \leavevmode
opens the card’s record in the form view in editable mode

\item[{\sphinxcode{\sphinxupquote{delete}}}] \leavevmode
deletes the card’s record and removes the card

\end{description}

\end{itemize}

\end{description}

\end{description}

If you need to extend the Kanban view, see :js:class::\sphinxcode{\sphinxupquote{the JS API \textless{}KanbanRecord\textgreater{}}}.


\subsection{Calendar}
\label{\detokenize{reference/views:reference-views-calendar}}\label{\detokenize{reference/views:calendar}}
Calendar views display records as events in a daily, weekly or monthly
calendar. Their root element is \sphinxcode{\sphinxupquote{\textless{}calendar\textgreater{}}}. Available attributes on the
calendar view are:
\begin{description}
\item[{\sphinxcode{\sphinxupquote{date\_start}} (required)}] \leavevmode
name of the record’s field holding the start date for the event

\item[{\sphinxcode{\sphinxupquote{date\_stop}}}] \leavevmode
name of the record’s field holding the end date for the event, if
\sphinxcode{\sphinxupquote{date\_stop}} is provided records become movable (via drag and drop)
directly in the calendar

\item[{\sphinxcode{\sphinxupquote{date\_delay}}}] \leavevmode
alternative to \sphinxcode{\sphinxupquote{date\_stop}}, provides the duration of the event instead of
its end date (unit: day)

\item[{\sphinxcode{\sphinxupquote{color}}}] \leavevmode
name of a record field to use for \sphinxstyleemphasis{color segmentation}. Records in the
same color segment are allocated the same highlight color in the calendar,
colors are allocated semi-randomly.
Displayed the display\_name/avatar of the visible record in the sidebar

\item[{\sphinxcode{\sphinxupquote{readonly\_form\_view\_id}}}] \leavevmode
view to open in readonly mode

\item[{\sphinxcode{\sphinxupquote{form\_view\_id}}}] \leavevmode
view to open when the user create or edit an event. Note that if this attribute
is not set, the calendar view will fall back to the id of the form view in the
current action, if any.

\item[{\sphinxcode{\sphinxupquote{event\_open\_popup}}}] \leavevmode
If the option ‘event\_open\_popup’ is set to true, then the calendar view will
open events (or records) in a FormViewDialog. Otherwise, it will open events
in a new form view (with a do\_action)

\item[{\sphinxcode{\sphinxupquote{quick\_add}}}] \leavevmode
enables quick-event creation on click: only asks the user for a \sphinxcode{\sphinxupquote{name}}
and tries to create a new event with just that and the clicked event
time. Falls back to a full form dialog if the quick creation fails

\item[{\sphinxcode{\sphinxupquote{all\_day}}}] \leavevmode
name of a boolean field on the record indicating whether the corresponding
event is flagged as day-long (and duration is irrelevant)

\item[{\sphinxcode{\sphinxupquote{mode}}}] \leavevmode
Default display mode when loading the calendar.
Possible attributes are: \sphinxcode{\sphinxupquote{day}}, \sphinxcode{\sphinxupquote{week}}, \sphinxcode{\sphinxupquote{month}}

\item[{\sphinxcode{\sphinxupquote{\textless{}field\textgreater{}}}}] \leavevmode
declares fields to aggregate or to use in kanban \sphinxstyleemphasis{logic}. If the field is
simply displayed in the calendar cards.

Fields can have additional attributes:
\begin{quote}
\begin{description}
\item[{\sphinxcode{\sphinxupquote{invisible}}}] \leavevmode
use “True” to hide the value in the cards

\item[{\sphinxcode{\sphinxupquote{avatar\_field}}}] \leavevmode
only for x2many field, to display the avatar instead the display\_name
in the cards

\item[{\sphinxcode{\sphinxupquote{write\_model}} and \sphinxcode{\sphinxupquote{write\_field}}}] \leavevmode
you can add a filter and save the result in the defined model, the
filter is added in the sidebar

\end{description}
\end{quote}

\item[{\sphinxcode{\sphinxupquote{templates}}}] \leavevmode
defines the {\hyperref[\detokenize{reference/qweb:reference-qweb}]{\sphinxcrossref{\DUrole{std,std-ref}{QWeb}}}} template \sphinxcode{\sphinxupquote{calendar-box}}. Cards definition
may be split into multiple templates for clarity which will be rendered once
for each record.

The kanban view uses mostly-standard {\hyperref[\detokenize{reference/qweb:reference-qweb-javascript}]{\sphinxcrossref{\DUrole{std,std-ref}{javascript qweb}}}} and provides the following context variables:
\begin{description}
\item[{\sphinxcode{\sphinxupquote{widget}}}] \leavevmode
the current {\hyperref[\detokenize{reference/javascript_api:KanbanRecord}]{\sphinxcrossref{\sphinxcode{\sphinxupquote{KanbanRecord()}}}}}, can be used to fetch some
meta-information. These methods are also available directly in the
template context and don’t need to be accessed via \sphinxcode{\sphinxupquote{widget}}
\sphinxcode{\sphinxupquote{getColor}} to convert in a color integer
\sphinxcode{\sphinxupquote{getAvatars}} to convert in an avatar image
\sphinxcode{\sphinxupquote{displayFields}} list of not invisible fields

\item[{\sphinxcode{\sphinxupquote{record}}}] \leavevmode
an object with all the requested fields as its attributes. Each field has
two attributes \sphinxcode{\sphinxupquote{value}} and \sphinxcode{\sphinxupquote{raw\_value}}

\item[{\sphinxcode{\sphinxupquote{event}}}] \leavevmode
the calendar event object

\item[{\sphinxcode{\sphinxupquote{format}}}] \leavevmode
format method to convert values into a readable string with the user
parameters

\item[{\sphinxcode{\sphinxupquote{fields}}}] \leavevmode
definition of all model fields
parameters

\item[{\sphinxcode{\sphinxupquote{user\_context}}}] \leavevmode
self-explanatory

\item[{\sphinxcode{\sphinxupquote{read\_only\_mode}}}] \leavevmode
self-explanatory

\end{description}

\end{description}


\subsection{Gantt}
\label{\detokenize{reference/views:reference-views-gantt}}\label{\detokenize{reference/views:gantt}}
Gantt views appropriately display Gantt charts (for scheduling).

The root element of gantt views is \sphinxcode{\sphinxupquote{\textless{}gantt/\textgreater{}}}, it has no children but can
take the following attributes:
\begin{description}
\item[{\sphinxcode{\sphinxupquote{date\_start}} (required)}] \leavevmode
name of the field providing the start datetime of the event for each
record.

\item[{\sphinxcode{\sphinxupquote{date\_stop}}}] \leavevmode
name of the field providing the end duration of the event for each
record. Can be replaced by \sphinxcode{\sphinxupquote{date\_delay}}. One (and only one) of
\sphinxcode{\sphinxupquote{date\_stop}} and \sphinxcode{\sphinxupquote{date\_delay}} must be provided.

If the field is \sphinxcode{\sphinxupquote{False}} for a record, it’s assumed to be a “point event”
and the end date will be set to the start date

\item[{\sphinxcode{\sphinxupquote{date\_delay}}}] \leavevmode
name of the field providing the duration of the event

\item[{\sphinxcode{\sphinxupquote{duration\_unit}}}] \leavevmode
one of \sphinxcode{\sphinxupquote{minute}}, \sphinxcode{\sphinxupquote{hour}} (default), \sphinxcode{\sphinxupquote{day}}, \sphinxcode{\sphinxupquote{week}}, \sphinxcode{\sphinxupquote{month}}, \sphinxcode{\sphinxupquote{year}}

\item[{\sphinxcode{\sphinxupquote{default\_group\_by}}}] \leavevmode
name of a field to group tasks by

\item[{\sphinxcode{\sphinxupquote{type}}}] \leavevmode
\sphinxcode{\sphinxupquote{gantt}} classic gantt view (default)

\sphinxcode{\sphinxupquote{consolidate}} values of the first children are consolidated in the gantt’s task

\sphinxcode{\sphinxupquote{planning}} children are displayed in the gantt’s task

\item[{\sphinxcode{\sphinxupquote{consolidation}}}] \leavevmode
field name to display consolidation value in record cell

\item[{\sphinxcode{\sphinxupquote{consolidation\_max}}}] \leavevmode
dictionary with the “group by” field as key and the maximum consolidation
value that can be reached before displaying the cell in red
(e.g. \sphinxcode{\sphinxupquote{\{"user\_id": 100\}}})

\item[{\sphinxcode{\sphinxupquote{consolidation\_exclude}}}] \leavevmode
name of the field that describe if the task has to be excluded
from the consolidation
if set to true it displays a striped zone in the consolidation line

\begin{sphinxadmonition}{warning}{Warning:}
The dictionnary definition must use double-quotes, \sphinxcode{\sphinxupquote{\{'user\_id': 100\}}} is
not a valid value
\end{sphinxadmonition}

\item[{\sphinxcode{\sphinxupquote{string}}}] \leavevmode
string to display next to the consolidation value, if not specified, the label
of the consolidation field will be used

\item[{\sphinxcode{\sphinxupquote{fold\_last\_level}}}] \leavevmode
If a value is set, the last grouping level is folded

\item[{\sphinxcode{\sphinxupquote{round\_dnd\_dates}}}] \leavevmode
enables rounding the task’s start and end dates to the nearest scale marks

\item[{\sphinxcode{\sphinxupquote{drag\_resize}}}] \leavevmode
resizing of the tasks, default is \sphinxcode{\sphinxupquote{true}}

\item[{\sphinxcode{\sphinxupquote{progress}}}] \leavevmode
name of a field providing the completion percentage for the record’s event,
between 0 and 100

\end{description}


\subsection{Diagram}
\label{\detokenize{reference/views:diagram}}\label{\detokenize{reference/views:reference-views-diagram}}
The diagram view can be used to display directed graphs of records. The root
element is \sphinxcode{\sphinxupquote{\textless{}diagram\textgreater{}}} and takes no attributes.

Possible children of the diagram view are:
\begin{description}
\item[{\sphinxcode{\sphinxupquote{node}} (required, 1)}] \leavevmode
Defines the nodes of the graph. Its attributes are:
\begin{description}
\item[{\sphinxcode{\sphinxupquote{object}}}] \leavevmode
the node’s Odoo model

\item[{\sphinxcode{\sphinxupquote{shape}}}] \leavevmode
conditional shape mapping similar to colors and fonts in {\hyperref[\detokenize{reference/views:reference-views-list}]{\sphinxcrossref{\DUrole{std,std-ref}{the list
view}}}}. The only valid shape is \sphinxcode{\sphinxupquote{rectangle}} (the
default shape is an ellipsis)

\item[{\sphinxcode{\sphinxupquote{bgcolor}}}] \leavevmode
same as \sphinxcode{\sphinxupquote{shape}}, but conditionally maps a background color for
nodes. The default background color is white, the only valid alternative
is \sphinxcode{\sphinxupquote{grey}}.

\end{description}

\item[{\sphinxcode{\sphinxupquote{arrow}} (required, 1)}] \leavevmode
Defines the directed edges of the graph. Its attributes are:
\begin{description}
\item[{\sphinxcode{\sphinxupquote{object}} (required)}] \leavevmode
the edge’s Odoo model

\item[{\sphinxcode{\sphinxupquote{source}} (required)}] \leavevmode
{\hyperref[\detokenize{reference/orm:odoo.fields.Many2one}]{\sphinxcrossref{\sphinxcode{\sphinxupquote{Many2one}}}}} field of the edge’s model pointing to
the edge’s source node record

\item[{\sphinxcode{\sphinxupquote{destination}} (required)}] \leavevmode
{\hyperref[\detokenize{reference/orm:odoo.fields.Many2one}]{\sphinxcrossref{\sphinxcode{\sphinxupquote{Many2one}}}}} field of the edge’s model pointing to
the edge’s destination node record

\item[{\sphinxcode{\sphinxupquote{label}}}] \leavevmode
Python list of attributes (as quoted strings). The corresponding
attributes’s values will be concatenated and displayed as the edge’s
label

\end{description}

\item[{\sphinxcode{\sphinxupquote{label}}}] \leavevmode
Explanatory note for the diagram, the \sphinxcode{\sphinxupquote{string}} attribute defines the
note’s content. Each \sphinxcode{\sphinxupquote{label}} is output as a paragraph in the diagram
header, easily visible but without any special emphasis.

\end{description}


\subsection{Search}
\label{\detokenize{reference/views:reference-views-search}}\label{\detokenize{reference/views:search}}
Search views are a break from previous view types in that they don’t display
\sphinxstyleemphasis{content}: although they apply to a specific model, they are used to filter
other view’s content (generally aggregated views
e.g. {\hyperref[\detokenize{reference/views:reference-views-list}]{\sphinxcrossref{\DUrole{std,std-ref}{Lists}}}} or {\hyperref[\detokenize{reference/views:reference-views-graph}]{\sphinxcrossref{\DUrole{std,std-ref}{Graphs}}}}). Beyond that
difference in use case, they are defined the same way.

The root element of search views is \sphinxcode{\sphinxupquote{\textless{}search\textgreater{}}}. It takes no attributes.

Possible children elements of the search view are:
\begin{description}
\item[{\sphinxcode{\sphinxupquote{field}}}] \leavevmode
fields define domains or contexts with user-provided values. When search
domains are generated, field domains are composed with one another and
with filters using \sphinxstylestrong{AND}.

Fields can have the following attributes:
\begin{description}
\item[{\sphinxcode{\sphinxupquote{name}}}] \leavevmode
the name of the field to filter on

\item[{\sphinxcode{\sphinxupquote{string}}}] \leavevmode
the field’s label

\item[{\sphinxcode{\sphinxupquote{operator}}}] \leavevmode
by default, fields generate domains of the form \sphinxcode{\sphinxupquote{{[}(\sphinxstyleemphasis{name},
\sphinxstyleemphasis{operator}, \sphinxstyleemphasis{provided\_value}){]}}} where \sphinxcode{\sphinxupquote{name}} is the field’s name and
\sphinxcode{\sphinxupquote{provided\_value}} is the value provided by the user, possibly
filtered or transformed (e.g. a user is expected to provide the
\sphinxstyleemphasis{label} of a selection field’s value, not the value itself).

The \sphinxcode{\sphinxupquote{operator}} attribute allows overriding the default operator,
which depends on the field’s type (e.g. \sphinxcode{\sphinxupquote{=}} for float fields but
\sphinxcode{\sphinxupquote{ilike}} for char fields)

\item[{\sphinxcode{\sphinxupquote{filter\_domain}}}] \leavevmode
complete domain to use as the field’s search domain, can use a
\sphinxcode{\sphinxupquote{self}} variable to inject the provided value in the custom
domain. Can be used to generate significantly more flexible domains
than \sphinxcode{\sphinxupquote{operator}} alone (e.g. searches on multiple fields at once)

If both \sphinxcode{\sphinxupquote{operator}} and \sphinxcode{\sphinxupquote{filter\_domain}} are provided,
\sphinxcode{\sphinxupquote{filter\_domain}} takes precedence.

\item[{\sphinxcode{\sphinxupquote{context}}}] \leavevmode
allows adding context keys, including the user-provided value (which
as for \sphinxcode{\sphinxupquote{domain}} is available as a \sphinxcode{\sphinxupquote{self}} variable). By default,
fields don’t generate domains.

\begin{sphinxadmonition}{note}{Note:}
the domain and context are inclusive and both are generated
if a \sphinxcode{\sphinxupquote{context}} is specified. To only generate context
values, set \sphinxcode{\sphinxupquote{filter\_domain}} to an empty list:
\sphinxcode{\sphinxupquote{filter\_domain="{[}{]}"}}
\end{sphinxadmonition}

\item[{\sphinxcode{\sphinxupquote{groups}}}] \leavevmode
make the field only available to specific users

\item[{\sphinxcode{\sphinxupquote{widget}}}] \leavevmode
use specific search widget for the field (the only use case in
standard Odoo 8.0 is a \sphinxcode{\sphinxupquote{selection}} widget for
{\hyperref[\detokenize{reference/orm:odoo.fields.Many2one}]{\sphinxcrossref{\sphinxcode{\sphinxupquote{Many2one}}}}} fields)

\item[{\sphinxcode{\sphinxupquote{domain}}}] \leavevmode
if the field can provide an auto-completion
(e.g. {\hyperref[\detokenize{reference/orm:odoo.fields.Many2one}]{\sphinxcrossref{\sphinxcode{\sphinxupquote{Many2one}}}}}), filters the possible
completion results.

\end{description}

\item[{\sphinxcode{\sphinxupquote{filter}}}] \leavevmode
a filter is a predefined toggle in the search view, it can only be enabled
or disabled. Its main purposes are to add data to the search context (the
context passed to the data view for searching/filtering), or to append new
sections to the search filter.

Filters can have the following attributes:
\begin{description}
\item[{\sphinxcode{\sphinxupquote{string}} (required)}] \leavevmode
the label of the filter

\item[{\sphinxcode{\sphinxupquote{domain}}}] \leavevmode
an Odoo {\hyperref[\detokenize{reference/orm:reference-orm-domains}]{\sphinxcrossref{\DUrole{std,std-ref}{domain}}}}, will be appended to the
action’s domain as part of the search domain

\item[{\sphinxcode{\sphinxupquote{context}}}] \leavevmode
a Python dictionary, merged into the action’s domain to generate the
search domain

\item[{\sphinxcode{\sphinxupquote{name}}}] \leavevmode
logical name for the filter, can be used to {\hyperref[\detokenize{reference/views:reference-views-search-defaults}]{\sphinxcrossref{\DUrole{std,std-ref}{enable it by default}}}}, can also be used as
{\hyperref[\detokenize{reference/views:reference-views-inheritance}]{\sphinxcrossref{\DUrole{std,std-ref}{inheritance hook}}}}

\item[{\sphinxcode{\sphinxupquote{help}}}] \leavevmode
a longer explanatory text for the filter, may be displayed as a
tooltip

\item[{\sphinxcode{\sphinxupquote{groups}}}] \leavevmode
makes a filter only available to specific users

\end{description}

\begin{sphinxadmonition}{tip}{Tip:}
\DUrole{versionmodified}{New in version 7.0.}

Sequences of filters (without non-filters separating them) are treated
as inclusively composited: they will be composed with \sphinxcode{\sphinxupquote{OR}} rather
than the usual \sphinxcode{\sphinxupquote{AND}}, e.g.

\fvset{hllines={, ,}}%
\begin{sphinxVerbatim}[commandchars=\\\{\}]
\PYG{n+nt}{\PYGZlt{}filter} \PYG{n+na}{domain=}\PYG{l+s}{\PYGZdq{}[(\PYGZsq{}state\PYGZsq{}, \PYGZsq{}=\PYGZsq{}, \PYGZsq{}draft\PYGZsq{})]\PYGZdq{}}\PYG{n+nt}{/\PYGZgt{}}
\PYG{n+nt}{\PYGZlt{}filter} \PYG{n+na}{domain=}\PYG{l+s}{\PYGZdq{}[(\PYGZsq{}state\PYGZsq{}, \PYGZsq{}=\PYGZsq{}, \PYGZsq{}done\PYGZsq{})]\PYGZdq{}}\PYG{n+nt}{/\PYGZgt{}}
\end{sphinxVerbatim}

if both filters are selected, will select the records whose \sphinxcode{\sphinxupquote{state}}
is \sphinxcode{\sphinxupquote{draft}} or \sphinxcode{\sphinxupquote{done}}, but

\fvset{hllines={, ,}}%
\begin{sphinxVerbatim}[commandchars=\\\{\}]
\PYG{n+nt}{\PYGZlt{}filter} \PYG{n+na}{domain=}\PYG{l+s}{\PYGZdq{}[(\PYGZsq{}state\PYGZsq{}, \PYGZsq{}=\PYGZsq{}, \PYGZsq{}draft\PYGZsq{})]\PYGZdq{}}\PYG{n+nt}{/\PYGZgt{}}
\PYG{n+nt}{\PYGZlt{}separator}\PYG{n+nt}{/\PYGZgt{}}
\PYG{n+nt}{\PYGZlt{}filter} \PYG{n+na}{domain=}\PYG{l+s}{\PYGZdq{}[(\PYGZsq{}delay\PYGZsq{}, \PYGZsq{}\PYGZlt{}\PYGZsq{}, 15)]\PYGZdq{}}\PYG{n+nt}{/\PYGZgt{}}
\end{sphinxVerbatim}

if both filters are selected, will select the records whose \sphinxcode{\sphinxupquote{state}}
is \sphinxcode{\sphinxupquote{draft}} \sphinxstylestrong{and} \sphinxcode{\sphinxupquote{delay}} is below 15.
\end{sphinxadmonition}

\item[{\sphinxcode{\sphinxupquote{separator}}}] \leavevmode
can be used to separates groups of filters in simple search views

\item[{\sphinxcode{\sphinxupquote{group}}}] \leavevmode
can be used to separate groups of filters, more readable than
\sphinxcode{\sphinxupquote{separator}} in complex search views

\end{description}


\subsubsection{Search defaults}
\label{\detokenize{reference/views:search-defaults}}\label{\detokenize{reference/views:reference-views-search-defaults}}
Search fields and filters can be configured through the action’s \sphinxcode{\sphinxupquote{context}}
using \sphinxcode{\sphinxupquote{search\_default\_\sphinxstyleemphasis{name}}} keys. For fields, the value should be the
value to set in the field, for filters it’s a boolean value. For instance,
assuming \sphinxcode{\sphinxupquote{foo}} is a field and \sphinxcode{\sphinxupquote{bar}} is a filter an action context of:

\fvset{hllines={, ,}}%
\begin{sphinxVerbatim}[commandchars=\\\{\}]
\PYG{p}{\PYGZob{}}
  \PYG{l+s+s1}{\PYGZsq{}}\PYG{l+s+s1}{search\PYGZus{}default\PYGZus{}foo}\PYG{l+s+s1}{\PYGZsq{}}\PYG{p}{:} \PYG{l+s+s1}{\PYGZsq{}}\PYG{l+s+s1}{acro}\PYG{l+s+s1}{\PYGZsq{}}\PYG{p}{,}
  \PYG{l+s+s1}{\PYGZsq{}}\PYG{l+s+s1}{search\PYGZus{}default\PYGZus{}bar}\PYG{l+s+s1}{\PYGZsq{}}\PYG{p}{:} \PYG{l+m+mi}{1}
\PYG{p}{\PYGZcb{}}
\end{sphinxVerbatim}

will automatically enable the \sphinxcode{\sphinxupquote{bar}} filter and search the \sphinxcode{\sphinxupquote{foo}} field for
\sphinxstyleemphasis{acro}.


\subsection{QWeb}
\label{\detokenize{reference/views:qweb}}\label{\detokenize{reference/views:reference-views-qweb}}
QWeb views are standard {\hyperref[\detokenize{reference/qweb:reference-qweb}]{\sphinxcrossref{\DUrole{std,std-ref}{QWeb}}}} templates inside a view’s
\sphinxcode{\sphinxupquote{arch}}. They don’t have a specific root element.

A QWeb view can only contain a single template%
\begin{footnote}[4]\sphinxAtStartFootnote
or no template if it’s an inherited view, then {\hyperref[\detokenize{reference/views:reference-views-inheritance}]{\sphinxcrossref{\DUrole{std,std-ref}{it
should only contain xpath elements}}}}
%
\end{footnote}, and the
template’s name \sphinxstyleemphasis{must} match the view’s complete (including module name)
\DUrole{xref,std,std-term}{external id}.

{\hyperref[\detokenize{reference/data:reference-data-template}]{\sphinxcrossref{\DUrole{std,std-ref}{template}}}} should be used as a shortcut to define QWeb
views.


\section{Module Manifests}
\label{\detokenize{reference/module:xpath}}\label{\detokenize{reference/module:module-manifests}}\label{\detokenize{reference/module::doc}}

\subsection{Manifest}
\label{\detokenize{reference/module:manifest}}\label{\detokenize{reference/module:reference-module-manifest}}
The manifest file serves to declare a python package as an Odoo module
and to specify module metadata.

It is a file called \sphinxcode{\sphinxupquote{\_\_manifest\_\_.py}} and contains a single Python
dictionary, where each key specifies module metadatum.

\fvset{hllines={, ,}}%
\begin{sphinxVerbatim}[commandchars=\\\{\}]
\PYG{p}{\PYGZob{}}
    \PYG{l+s+s1}{\PYGZsq{}}\PYG{l+s+s1}{name}\PYG{l+s+s1}{\PYGZsq{}}\PYG{p}{:} \PYG{l+s+s2}{\PYGZdq{}}\PYG{l+s+s2}{A Module}\PYG{l+s+s2}{\PYGZdq{}}\PYG{p}{,}
    \PYG{l+s+s1}{\PYGZsq{}}\PYG{l+s+s1}{version}\PYG{l+s+s1}{\PYGZsq{}}\PYG{p}{:} \PYG{l+s+s1}{\PYGZsq{}}\PYG{l+s+s1}{1.0}\PYG{l+s+s1}{\PYGZsq{}}\PYG{p}{,}
    \PYG{l+s+s1}{\PYGZsq{}}\PYG{l+s+s1}{depends}\PYG{l+s+s1}{\PYGZsq{}}\PYG{p}{:} \PYG{p}{[}\PYG{l+s+s1}{\PYGZsq{}}\PYG{l+s+s1}{base}\PYG{l+s+s1}{\PYGZsq{}}\PYG{p}{]}\PYG{p}{,}
    \PYG{l+s+s1}{\PYGZsq{}}\PYG{l+s+s1}{author}\PYG{l+s+s1}{\PYGZsq{}}\PYG{p}{:} \PYG{l+s+s2}{\PYGZdq{}}\PYG{l+s+s2}{Author Name}\PYG{l+s+s2}{\PYGZdq{}}\PYG{p}{,}
    \PYG{l+s+s1}{\PYGZsq{}}\PYG{l+s+s1}{category}\PYG{l+s+s1}{\PYGZsq{}}\PYG{p}{:} \PYG{l+s+s1}{\PYGZsq{}}\PYG{l+s+s1}{Category}\PYG{l+s+s1}{\PYGZsq{}}\PYG{p}{,}
    \PYG{l+s+s1}{\PYGZsq{}}\PYG{l+s+s1}{description}\PYG{l+s+s1}{\PYGZsq{}}\PYG{p}{:} \PYG{l+s+s2}{\PYGZdq{}\PYGZdq{}\PYGZdq{}}
\PYG{l+s+s2}{    Description text}
\PYG{l+s+s2}{    }\PYG{l+s+s2}{\PYGZdq{}\PYGZdq{}\PYGZdq{}}\PYG{p}{,}
    \PYG{c+c1}{\PYGZsh{} data files always loaded at installation}
    \PYG{l+s+s1}{\PYGZsq{}}\PYG{l+s+s1}{data}\PYG{l+s+s1}{\PYGZsq{}}\PYG{p}{:} \PYG{p}{[}
        \PYG{l+s+s1}{\PYGZsq{}}\PYG{l+s+s1}{views/mymodule\PYGZus{}view.xml}\PYG{l+s+s1}{\PYGZsq{}}\PYG{p}{,}
    \PYG{p}{]}\PYG{p}{,}
    \PYG{c+c1}{\PYGZsh{} data files containing optionally loaded demonstration data}
    \PYG{l+s+s1}{\PYGZsq{}}\PYG{l+s+s1}{demo}\PYG{l+s+s1}{\PYGZsq{}}\PYG{p}{:} \PYG{p}{[}
        \PYG{l+s+s1}{\PYGZsq{}}\PYG{l+s+s1}{demo/demo\PYGZus{}data.xml}\PYG{l+s+s1}{\PYGZsq{}}\PYG{p}{,}
    \PYG{p}{]}\PYG{p}{,}
\PYG{p}{\PYGZcb{}}
\end{sphinxVerbatim}

Available manifest fields are:
\begin{description}
\item[{\sphinxcode{\sphinxupquote{name}} (\sphinxcode{\sphinxupquote{str}}, required)}] \leavevmode
the human-readable name of the module

\item[{\sphinxcode{\sphinxupquote{version}} (\sphinxcode{\sphinxupquote{str}})}] \leavevmode
this module’s version, should follow \sphinxhref{http://semver.org}{semantic versioning} rules

\item[{\sphinxcode{\sphinxupquote{description}} (\sphinxcode{\sphinxupquote{str}})}] \leavevmode
extended description for the module, in reStructuredText

\item[{\sphinxcode{\sphinxupquote{author}} (\sphinxcode{\sphinxupquote{str}})}] \leavevmode
name of the module author

\item[{\sphinxcode{\sphinxupquote{website}} (\sphinxcode{\sphinxupquote{str}})}] \leavevmode
website URL for the module author

\item[{\sphinxcode{\sphinxupquote{license}} (\sphinxcode{\sphinxupquote{str}}, defaults: \sphinxcode{\sphinxupquote{LGPL-3}})}] \leavevmode
distribution license for the module

\item[{\sphinxcode{\sphinxupquote{category}} (\sphinxcode{\sphinxupquote{str}}, default: \sphinxcode{\sphinxupquote{Uncategorized}})}] \leavevmode
classification category within Odoo, rough business domain for the module.

Although using \sphinxhref{https://github.com/odoo/odoo/blob/master/odoo/addons/base/module/module\_data.xml}{existing categories} is recommended, the field is
freeform and unknown categories are created on-the-fly. Category
hierarchies can be created using the separator \sphinxcode{\sphinxupquote{/}} e.g. \sphinxcode{\sphinxupquote{Foo / Bar}}
will create a category \sphinxcode{\sphinxupquote{Foo}}, a category \sphinxcode{\sphinxupquote{Bar}} as child category of
\sphinxcode{\sphinxupquote{Foo}}, and will set \sphinxcode{\sphinxupquote{Bar}} as the module’s category.

\item[{\sphinxcode{\sphinxupquote{depends}} (\sphinxcode{\sphinxupquote{list(str)}})}] \leavevmode
Odoo modules which must be loaded before this one, either because this
module uses features they create or because it alters resources they
define.

When a module is installed, all of its dependencies are installed before
it. Likewise dependencies are loaded before a module is loaded.

\item[{\sphinxcode{\sphinxupquote{data}} (\sphinxcode{\sphinxupquote{list(str)}})}] \leavevmode
List of data files which must always be installed or updated with the
module. A list of paths from the module root directory

\item[{\sphinxcode{\sphinxupquote{demo}} (\sphinxcode{\sphinxupquote{list(str)}})}] \leavevmode
List of data files which are only installed or updated in \sphinxstyleemphasis{demonstration
mode}

\item[{\sphinxcode{\sphinxupquote{auto\_install}} (\sphinxcode{\sphinxupquote{bool}}, default: \sphinxcode{\sphinxupquote{False}})}] \leavevmode
If \sphinxcode{\sphinxupquote{True}}, this module will automatically be installed if all of its
dependencies are installed.

It is generally used for “link modules” implementing synergic integration
between two otherwise independent modules.

For instance \sphinxcode{\sphinxupquote{sale\_crm}} depends on both \sphinxcode{\sphinxupquote{sale}} and \sphinxcode{\sphinxupquote{crm}} and is set
to \sphinxcode{\sphinxupquote{auto\_install}}. When both \sphinxcode{\sphinxupquote{sale}} and \sphinxcode{\sphinxupquote{crm}} are installed, it
automatically adds CRM campaigns tracking to sale orders without either
\sphinxcode{\sphinxupquote{sale}} or \sphinxcode{\sphinxupquote{crm}} being aware of one another

\item[{\sphinxcode{\sphinxupquote{external\_dependencies}} (\sphinxcode{\sphinxupquote{dict(key=list(str))}})}] \leavevmode
A dictionary containing python and/or binary dependencies.

For python dependencies, the \sphinxcode{\sphinxupquote{python}} key must be defined for this
dictionary and a list of python modules to be imported should be assigned
to it.

For binary dependencies, the \sphinxcode{\sphinxupquote{bin}} key must be defined for this
dictionary and a list of binary executable names should be assigned to it.

The module won’t be installed if either the python module is not installed
in the host machine or the binary executable is not found within the
host machine’s PATH environment variable.

\item[{\sphinxcode{\sphinxupquote{application}} (\sphinxcode{\sphinxupquote{bool}}, default: \sphinxcode{\sphinxupquote{False}})}] \leavevmode
Whether the module should be considered as a fully-fledged application
(\sphinxcode{\sphinxupquote{True}}) or is just a technical module (\sphinxcode{\sphinxupquote{False}}) that provides some
extra functionality to an existing application module.

\item[{\sphinxcode{\sphinxupquote{css}} (\sphinxcode{\sphinxupquote{list(str)}})}] \leavevmode
Specify css files with custom rules to be imported, these files should be
located in \sphinxcode{\sphinxupquote{static/src/css}} inside the module.

\item[{\sphinxcode{\sphinxupquote{images}} (\sphinxcode{\sphinxupquote{list(str)}})}] \leavevmode
Specify image files to be used by the module.

\item[{\sphinxcode{\sphinxupquote{installable}} (\sphinxcode{\sphinxupquote{bool}} default: \sphinxcode{\sphinxupquote{False}})}] \leavevmode
Whether a user should be able to install the module from the Web UI or not.

\item[{\sphinxcode{\sphinxupquote{maintainer}} (\sphinxcode{\sphinxupquote{str}})}] \leavevmode
Person or entity in charge of the maintenance of this module, by default
it is assumed that the author is the maintainer.

\item[{\sphinxcode{\sphinxupquote{\{pre\_init, post\_init, uninstall\}\_hook}} (\sphinxcode{\sphinxupquote{str}})}] \leavevmode
Hooks for module installation/uninstallation, their value should be a
string representing the name of a function defined inside the module’s
\sphinxcode{\sphinxupquote{\_\_init\_\_.py}}.

\sphinxcode{\sphinxupquote{pre\_init\_hook}} takes a cursor as its only argument, this function is
executed prior to the module’s installation.

\sphinxcode{\sphinxupquote{post\_init\_hook}} takes a cursor and a registry as its arguments, this
function is executed right after the module’s installation.

\sphinxcode{\sphinxupquote{uninstall\_hook}} takes a cursor and a registry as its arguments, this
function is executed after the module’s uninstallation.

These hooks should only be used when setup/cleanup required for this module
is either extremely difficult or impossible through the api.

\end{description}


\section{Command-line interface: odoo-bin}
\label{\detokenize{reference/cmdline:command-line-interface-odoo-bin}}\label{\detokenize{reference/cmdline:reference-cmdline}}\label{\detokenize{reference/cmdline::doc}}\label{\detokenize{reference/cmdline:existing-categories}}

\subsection{Running the server}
\label{\detokenize{reference/cmdline:running-the-server}}\label{\detokenize{reference/cmdline:reference-cmdline-server}}\index{odoo-bin command line option!-d \textless{}database\textgreater{}, --database \textless{}database\textgreater{}}\index{-d \textless{}database\textgreater{}, --database \textless{}database\textgreater{}!odoo-bin command line option}

\begin{fulllineitems}
\phantomsection\label{\detokenize{reference/cmdline:cmdoption-odoo-bin-d}}\pysigline{\sphinxbfcode{\sphinxupquote{-d}}\sphinxcode{\sphinxupquote{~\textless{}database\textgreater{}}}\sphinxcode{\sphinxupquote{,~}}\sphinxbfcode{\sphinxupquote{-{-}database}}\sphinxcode{\sphinxupquote{~\textless{}database\textgreater{}}}}
database(s) used when installing or updating modules.
Providing a comma-separated list restrict access to databases provided in
list.

\end{fulllineitems}

\index{odoo-bin command line option!-i \textless{}modules\textgreater{}, --init \textless{}modules\textgreater{}}\index{-i \textless{}modules\textgreater{}, --init \textless{}modules\textgreater{}!odoo-bin command line option}

\begin{fulllineitems}
\phantomsection\label{\detokenize{reference/cmdline:cmdoption-odoo-bin-i}}\pysigline{\sphinxbfcode{\sphinxupquote{-i}}\sphinxcode{\sphinxupquote{~\textless{}modules\textgreater{}}}\sphinxcode{\sphinxupquote{,~}}\sphinxbfcode{\sphinxupquote{-{-}init}}\sphinxcode{\sphinxupquote{~\textless{}modules\textgreater{}}}}
comma-separated list of modules to install before running the server
(requires {\hyperref[\detokenize{reference/cmdline:cmdoption-odoo-bin-d}]{\sphinxcrossref{\sphinxcode{\sphinxupquote{-d}}}}}).

\end{fulllineitems}

\index{odoo-bin command line option!-u \textless{}modules\textgreater{}, --update \textless{}modules\textgreater{}}\index{-u \textless{}modules\textgreater{}, --update \textless{}modules\textgreater{}!odoo-bin command line option}

\begin{fulllineitems}
\phantomsection\label{\detokenize{reference/cmdline:cmdoption-odoo-bin-u}}\pysigline{\sphinxbfcode{\sphinxupquote{-u}}\sphinxcode{\sphinxupquote{~\textless{}modules\textgreater{}}}\sphinxcode{\sphinxupquote{,~}}\sphinxbfcode{\sphinxupquote{-{-}update}}\sphinxcode{\sphinxupquote{~\textless{}modules\textgreater{}}}}
comma-separated list of modules to update before running the server
(requires {\hyperref[\detokenize{reference/cmdline:cmdoption-odoo-bin-d}]{\sphinxcrossref{\sphinxcode{\sphinxupquote{-d}}}}}).

\end{fulllineitems}

\index{odoo-bin command line option!--addons-path \textless{}directories\textgreater{}}\index{--addons-path \textless{}directories\textgreater{}!odoo-bin command line option}

\begin{fulllineitems}
\phantomsection\label{\detokenize{reference/cmdline:cmdoption-odoo-bin-addons-path}}\pysigline{\sphinxbfcode{\sphinxupquote{-{-}addons-path}}\sphinxcode{\sphinxupquote{~\textless{}directories\textgreater{}}}}
comma-separated list of directories in which modules are stored. These
directories are scanned for modules (nb: when and why?)

\end{fulllineitems}

\index{odoo-bin command line option!--workers \textless{}count\textgreater{}}\index{--workers \textless{}count\textgreater{}!odoo-bin command line option}

\begin{fulllineitems}
\phantomsection\label{\detokenize{reference/cmdline:cmdoption-odoo-bin-workers}}\pysigline{\sphinxbfcode{\sphinxupquote{-{-}workers}}\sphinxcode{\sphinxupquote{~\textless{}count\textgreater{}}}}
if \sphinxcode{\sphinxupquote{count}} is not 0 (the default), enables multiprocessing and sets up
the specified number of HTTP workers (sub-processes processing HTTP
and RPC requests).

\begin{sphinxadmonition}{note}{Note:}
multiprocessing mode is only available on Unix-based systems
\end{sphinxadmonition}

A number of options allow limiting and recycling workers:
\index{odoo-bin command line option!--limit-request \textless{}limit\textgreater{}}\index{--limit-request \textless{}limit\textgreater{}!odoo-bin command line option}

\begin{fulllineitems}
\phantomsection\label{\detokenize{reference/cmdline:cmdoption-odoo-bin-limit-request}}\pysigline{\sphinxbfcode{\sphinxupquote{-{-}limit-request}}\sphinxcode{\sphinxupquote{~\textless{}limit\textgreater{}}}}
Number of requests a worker will process before being recycled and
restarted.

Defaults to 8196.

\end{fulllineitems}

\index{odoo-bin command line option!--limit-memory-soft \textless{}limit\textgreater{}}\index{--limit-memory-soft \textless{}limit\textgreater{}!odoo-bin command line option}

\begin{fulllineitems}
\phantomsection\label{\detokenize{reference/cmdline:cmdoption-odoo-bin-limit-memory-soft}}\pysigline{\sphinxbfcode{\sphinxupquote{-{-}limit-memory-soft}}\sphinxcode{\sphinxupquote{~\textless{}limit\textgreater{}}}}
Maximum allowed virtual memory per worker. If the limit is exceeded,
the worker is killed and recycled at the end of the current request.

Defaults to 2048MB.

\end{fulllineitems}

\index{odoo-bin command line option!--limit-memory-hard \textless{}limit\textgreater{}}\index{--limit-memory-hard \textless{}limit\textgreater{}!odoo-bin command line option}

\begin{fulllineitems}
\phantomsection\label{\detokenize{reference/cmdline:cmdoption-odoo-bin-limit-memory-hard}}\pysigline{\sphinxbfcode{\sphinxupquote{-{-}limit-memory-hard}}\sphinxcode{\sphinxupquote{~\textless{}limit\textgreater{}}}}
Hard limit on virtual memory, any worker exceeding the limit will be
immediately killed without waiting for the end of the current request
processing.

Defaults to 2560MB.

\end{fulllineitems}

\index{odoo-bin command line option!--limit-time-cpu \textless{}limit\textgreater{}}\index{--limit-time-cpu \textless{}limit\textgreater{}!odoo-bin command line option}

\begin{fulllineitems}
\phantomsection\label{\detokenize{reference/cmdline:cmdoption-odoo-bin-limit-time-cpu}}\pysigline{\sphinxbfcode{\sphinxupquote{-{-}limit-time-cpu}}\sphinxcode{\sphinxupquote{~\textless{}limit\textgreater{}}}}
Prevents the worker from using more than \textless{}limit\textgreater{} CPU seconds for each
request. If the limit is exceeded, the worker is killed.

Defaults to 60.

\end{fulllineitems}

\index{odoo-bin command line option!--limit-time-real \textless{}limit\textgreater{}}\index{--limit-time-real \textless{}limit\textgreater{}!odoo-bin command line option}

\begin{fulllineitems}
\phantomsection\label{\detokenize{reference/cmdline:cmdoption-odoo-bin-limit-time-real}}\pysigline{\sphinxbfcode{\sphinxupquote{-{-}limit-time-real}}\sphinxcode{\sphinxupquote{~\textless{}limit\textgreater{}}}}
Prevents the worker from taking longer than \textless{}limit\textgreater{} seconds to process
a request. If the limit is exceeded, the worker is killed.

Differs from {\hyperref[\detokenize{reference/cmdline:cmdoption-odoo-bin-limit-time-cpu}]{\sphinxcrossref{\sphinxcode{\sphinxupquote{-{-}limit-time-cpu}}}}} in that this is a “wall time”
limit including e.g. SQL queries.

Defaults to 120.

\end{fulllineitems}


\end{fulllineitems}

\index{odoo-bin command line option!--max-cron-threads \textless{}count\textgreater{}}\index{--max-cron-threads \textless{}count\textgreater{}!odoo-bin command line option}

\begin{fulllineitems}
\phantomsection\label{\detokenize{reference/cmdline:cmdoption-odoo-bin-max-cron-threads}}\pysigline{\sphinxbfcode{\sphinxupquote{-{-}max-cron-threads}}\sphinxcode{\sphinxupquote{~\textless{}count\textgreater{}}}}
number of workers dedicated to cron jobs. Defaults to 2. The workers are
threads in multi-threading mode and processes in multi-processing mode.

For multi-processing mode, this is in addition to the HTTP worker
processes.

\end{fulllineitems}

\index{odoo-bin command line option!-c \textless{}config\textgreater{}, --config \textless{}config\textgreater{}}\index{-c \textless{}config\textgreater{}, --config \textless{}config\textgreater{}!odoo-bin command line option}

\begin{fulllineitems}
\phantomsection\label{\detokenize{reference/cmdline:cmdoption-odoo-bin-c}}\pysigline{\sphinxbfcode{\sphinxupquote{-c}}\sphinxcode{\sphinxupquote{~\textless{}config\textgreater{}}}\sphinxcode{\sphinxupquote{,~}}\sphinxbfcode{\sphinxupquote{-{-}config}}\sphinxcode{\sphinxupquote{~\textless{}config\textgreater{}}}}
provide an alternate configuration file

\end{fulllineitems}

\index{odoo-bin command line option!-s, --save}\index{-s, --save!odoo-bin command line option}

\begin{fulllineitems}
\phantomsection\label{\detokenize{reference/cmdline:cmdoption-odoo-bin-s}}\pysigline{\sphinxbfcode{\sphinxupquote{-s}}\sphinxcode{\sphinxupquote{}}\sphinxcode{\sphinxupquote{,~}}\sphinxbfcode{\sphinxupquote{-{-}save}}\sphinxcode{\sphinxupquote{}}}
saves the server configuration to the current configuration file
(\sphinxcode{\sphinxupquote{\sphinxstyleemphasis{\$HOME}/.odoorc}} by default, and can be overridden using
{\hyperref[\detokenize{reference/cmdline:cmdoption-odoo-bin-c}]{\sphinxcrossref{\sphinxcode{\sphinxupquote{-c}}}}})

\end{fulllineitems}

\index{odoo-bin command line option!--proxy-mode}\index{--proxy-mode!odoo-bin command line option}

\begin{fulllineitems}
\phantomsection\label{\detokenize{reference/cmdline:cmdoption-odoo-bin-proxy-mode}}\pysigline{\sphinxbfcode{\sphinxupquote{-{-}proxy-mode}}\sphinxcode{\sphinxupquote{}}}
enables the use of \sphinxcode{\sphinxupquote{X-Forwarded-*}} headers through \sphinxhref{http://werkzeug.pocoo.org/docs/contrib/fixers/\#werkzeug.contrib.fixers.ProxyFix}{Werkzeug’s proxy
support}.

\begin{sphinxadmonition}{warning}{Warning:}
proxy mode \sphinxstyleemphasis{must not} be enabled outside of a reverse proxy
scenario
\end{sphinxadmonition}

\end{fulllineitems}

\index{odoo-bin command line option!--test-enable}\index{--test-enable!odoo-bin command line option}

\begin{fulllineitems}
\phantomsection\label{\detokenize{reference/cmdline:cmdoption-odoo-bin-test-enable}}\pysigline{\sphinxbfcode{\sphinxupquote{-{-}test-enable}}\sphinxcode{\sphinxupquote{}}}
runs tests after installing modules

\end{fulllineitems}

\index{odoo-bin command line option!--dev \textless{}feature,feature,...,feature\textgreater{}}\index{--dev \textless{}feature,feature,...,feature\textgreater{}!odoo-bin command line option}

\begin{fulllineitems}
\phantomsection\label{\detokenize{reference/cmdline:cmdoption-odoo-bin-dev}}\pysigline{\sphinxbfcode{\sphinxupquote{-{-}dev}}\sphinxcode{\sphinxupquote{~\textless{}feature,feature,...,feature\textgreater{}}}}~\begin{itemize}
\item {} 
\sphinxcode{\sphinxupquote{all}}: all the features below are activated

\item {} 
\sphinxcode{\sphinxupquote{xml}}: read template qweb from xml file directly instead of database.
Once a template has been modified in database, it will be not be read from
the xml file until the next update/init.

\item {} 
\sphinxcode{\sphinxupquote{reload}}: restart server when python file are updated (may not be detected
depending on the text editor used)

\item {} 
\sphinxcode{\sphinxupquote{qweb}}: break in the evaluation of qweb template when a node contains \sphinxcode{\sphinxupquote{t-debug='debugger'}}

\item {} 
\sphinxcode{\sphinxupquote{(i)p(u)db}}: start the chosen python debugger in the code when an
unexpected error is raised before logging and returning the error.

\end{itemize}

\end{fulllineitems}



\subsubsection{database}
\label{\detokenize{reference/cmdline:reference-cmdline-server-database}}\label{\detokenize{reference/cmdline:database}}\index{odoo-bin command line option!-r \textless{}user\textgreater{}, --db\_user \textless{}user\textgreater{}}\index{-r \textless{}user\textgreater{}, --db\_user \textless{}user\textgreater{}!odoo-bin command line option}

\begin{fulllineitems}
\phantomsection\label{\detokenize{reference/cmdline:cmdoption-odoo-bin-r}}\pysigline{\sphinxbfcode{\sphinxupquote{-r}}\sphinxcode{\sphinxupquote{~\textless{}user\textgreater{}}}\sphinxcode{\sphinxupquote{,~}}\sphinxbfcode{\sphinxupquote{-{-}db\_user}}\sphinxcode{\sphinxupquote{~\textless{}user\textgreater{}}}}
database username, used to connect to PostgreSQL.

\end{fulllineitems}

\index{odoo-bin command line option!-w \textless{}password\textgreater{}, --db\_password \textless{}password\textgreater{}}\index{-w \textless{}password\textgreater{}, --db\_password \textless{}password\textgreater{}!odoo-bin command line option}

\begin{fulllineitems}
\phantomsection\label{\detokenize{reference/cmdline:cmdoption-odoo-bin-w}}\pysigline{\sphinxbfcode{\sphinxupquote{-w}}\sphinxcode{\sphinxupquote{~\textless{}password\textgreater{}}}\sphinxcode{\sphinxupquote{,~}}\sphinxbfcode{\sphinxupquote{-{-}db\_password}}\sphinxcode{\sphinxupquote{~\textless{}password\textgreater{}}}}
database password, if using \sphinxhref{http://www.postgresql.org/docs/9.3/static/auth-methods.html\#AUTH-PASSWORD}{password authentication}.

\end{fulllineitems}

\index{odoo-bin command line option!--db\_host \textless{}hostname\textgreater{}}\index{--db\_host \textless{}hostname\textgreater{}!odoo-bin command line option}

\begin{fulllineitems}
\phantomsection\label{\detokenize{reference/cmdline:cmdoption-odoo-bin-db-host}}\pysigline{\sphinxbfcode{\sphinxupquote{-{-}db\_host}}\sphinxcode{\sphinxupquote{~\textless{}hostname\textgreater{}}}}
host for the database server
\begin{itemize}
\item {} 
\sphinxcode{\sphinxupquote{localhost}} on Windows

\item {} 
UNIX socket otherwise

\end{itemize}

\end{fulllineitems}

\index{odoo-bin command line option!--db\_port \textless{}port\textgreater{}}\index{--db\_port \textless{}port\textgreater{}!odoo-bin command line option}

\begin{fulllineitems}
\phantomsection\label{\detokenize{reference/cmdline:cmdoption-odoo-bin-db-port}}\pysigline{\sphinxbfcode{\sphinxupquote{-{-}db\_port}}\sphinxcode{\sphinxupquote{~\textless{}port\textgreater{}}}}
port the database listens on, defaults to 5432

\end{fulllineitems}

\index{odoo-bin command line option!--db-filter \textless{}filter\textgreater{}}\index{--db-filter \textless{}filter\textgreater{}!odoo-bin command line option}

\begin{fulllineitems}
\phantomsection\label{\detokenize{reference/cmdline:cmdoption-odoo-bin-db-filter}}\pysigline{\sphinxbfcode{\sphinxupquote{-{-}db-filter}}\sphinxcode{\sphinxupquote{~\textless{}filter\textgreater{}}}}
hides databases that do not match \sphinxcode{\sphinxupquote{\textless{}filter\textgreater{}}}. The filter is a
\sphinxhref{https://docs.python.org/2/library/re.html}{regular expression}, with the additions that:
\begin{itemize}
\item {} 
\sphinxcode{\sphinxupquote{\%h}} is replaced by the whole hostname the request is made on.

\item {} 
\sphinxcode{\sphinxupquote{\%d}} is replaced by the subdomain the request is made on, with the
exception of \sphinxcode{\sphinxupquote{www}} (so domain \sphinxcode{\sphinxupquote{odoo.com}} and \sphinxcode{\sphinxupquote{www.odoo.com}} both
match the database \sphinxcode{\sphinxupquote{odoo}}).

These operations are case sensitive. Add option \sphinxcode{\sphinxupquote{(?i)}} to match all
databases (so domain \sphinxcode{\sphinxupquote{odoo.com}} using \sphinxcode{\sphinxupquote{(?i)\%d}} matches the database
\sphinxcode{\sphinxupquote{Odoo}}).

\end{itemize}

Since version 11, it’s also possible to restrict access to a given database
listen by using the \textendash{}database parameter and specifying a comma-separated
list of databases

When combining the two parameters, db-filter superseed the comma-separated
database list for restricting database list, while the comma-separated list
is used for performing requested operations like upgrade of modules.

\fvset{hllines={, ,}}%
\begin{sphinxVerbatim}[commandchars=\\\{\}]
odoo\PYGZhy{}bin \PYGZhy{}\PYGZhy{}db\PYGZhy{}filter \PYGZca{}11.*\PYGZdl{}
\end{sphinxVerbatim}

Restrict access to databases whose name starts with 11

\fvset{hllines={, ,}}%
\begin{sphinxVerbatim}[commandchars=\\\{\}]
odoo\PYGZhy{}bin \PYGZhy{}\PYGZhy{}database 11firstdatabase,11seconddatabase
\end{sphinxVerbatim}

Restrict access to only two databases, 11firstdatabase and 11seconddatabase

\fvset{hllines={, ,}}%
\begin{sphinxVerbatim}[commandchars=\\\{\}]
odoo\PYGZhy{}bin \PYGZhy{}\PYGZhy{}database 11firstdatabase,11seconddatabase \PYGZhy{}u base
\end{sphinxVerbatim}

Restrict access to only two databases, 11firstdatabase and 11seconddatabase,
and update base module on one database: 11firstdatabase
If database 11seconddatabase doesn’t exist, the database is created and base modules
is installed

\fvset{hllines={, ,}}%
\begin{sphinxVerbatim}[commandchars=\\\{\}]
odoo\PYGZhy{}bin \PYGZhy{}\PYGZhy{}db\PYGZhy{}filter \PYGZca{}11.*\PYGZdl{} \PYGZhy{}\PYGZhy{}database 11firstdatabase,11seconddatabase \PYGZhy{}u base
\end{sphinxVerbatim}

Restrict access to databases whose name starts with 11,
and update base module on one database: 11firstdatabase
If database 11seconddatabase doesn’t exist, the database is created and base modules
is installed

\end{fulllineitems}

\index{odoo-bin command line option!--db-template \textless{}template\textgreater{}}\index{--db-template \textless{}template\textgreater{}!odoo-bin command line option}

\begin{fulllineitems}
\phantomsection\label{\detokenize{reference/cmdline:cmdoption-odoo-bin-db-template}}\pysigline{\sphinxbfcode{\sphinxupquote{-{-}db-template}}\sphinxcode{\sphinxupquote{~\textless{}template\textgreater{}}}}
when creating new databases from the database-management screens, use the
specified \sphinxhref{http://www.postgresql.org/docs/9.3/static/manage-ag-templatedbs.html}{template database}. Defaults to \sphinxcode{\sphinxupquote{template1}}.

\end{fulllineitems}

\index{odoo-bin command line option!--no-database-list}\index{--no-database-list!odoo-bin command line option}

\begin{fulllineitems}
\phantomsection\label{\detokenize{reference/cmdline:cmdoption-odoo-bin-no-database-list}}\pysigline{\sphinxbfcode{\sphinxupquote{-{-}no-database-list}}\sphinxcode{\sphinxupquote{}}}
Suppresses the ability to list databases available on the system

\end{fulllineitems}

\index{odoo-bin command line option!--db\_sslmode}\index{--db\_sslmode!odoo-bin command line option}

\begin{fulllineitems}
\phantomsection\label{\detokenize{reference/cmdline:cmdoption-odoo-bin-db-sslmode}}\pysigline{\sphinxbfcode{\sphinxupquote{-{-}db\_sslmode}}\sphinxcode{\sphinxupquote{}}}
Control the SSL security of the connection between Odoo and PostgreSQL.
Value should bve one of ‘disable’, ‘allow’, ‘prefer’, ‘require’,
‘verify-ca’ or ‘verify-full’
Default value is ‘prefer’

\end{fulllineitems}



\subsubsection{built-in HTTP}
\label{\detokenize{reference/cmdline:built-in-http}}\index{odoo-bin command line option!--no-http}\index{--no-http!odoo-bin command line option}

\begin{fulllineitems}
\phantomsection\label{\detokenize{reference/cmdline:cmdoption-odoo-bin-no-http}}\pysigline{\sphinxbfcode{\sphinxupquote{-{-}no-http}}\sphinxcode{\sphinxupquote{}}}
do not start the HTTP or long-polling workers (may still start cron
workers)

\begin{sphinxadmonition}{warning}{Warning:}
has no effect if {\hyperref[\detokenize{reference/cmdline:cmdoption-odoo-bin-test-enable}]{\sphinxcrossref{\sphinxcode{\sphinxupquote{-{-}test-enable}}}}} is set, as tests
require an accessible HTTP server
\end{sphinxadmonition}

\end{fulllineitems}

\index{odoo-bin command line option!--http-interface \textless{}interface\textgreater{}}\index{--http-interface \textless{}interface\textgreater{}!odoo-bin command line option}

\begin{fulllineitems}
\phantomsection\label{\detokenize{reference/cmdline:cmdoption-odoo-bin-http-interface}}\pysigline{\sphinxbfcode{\sphinxupquote{-{-}http-interface}}\sphinxcode{\sphinxupquote{~\textless{}interface\textgreater{}}}}
TCP/IP address on which the HTTP server listens, defaults to \sphinxcode{\sphinxupquote{0.0.0.0}}
(all addresses)

\end{fulllineitems}

\index{odoo-bin command line option!--http-port \textless{}port\textgreater{}}\index{--http-port \textless{}port\textgreater{}!odoo-bin command line option}

\begin{fulllineitems}
\phantomsection\label{\detokenize{reference/cmdline:cmdoption-odoo-bin-http-port}}\pysigline{\sphinxbfcode{\sphinxupquote{-{-}http-port}}\sphinxcode{\sphinxupquote{~\textless{}port\textgreater{}}}}
Port on which the HTTP server listens, defaults to 8069.

\end{fulllineitems}

\index{odoo-bin command line option!--longpolling-port \textless{}port\textgreater{}}\index{--longpolling-port \textless{}port\textgreater{}!odoo-bin command line option}

\begin{fulllineitems}
\phantomsection\label{\detokenize{reference/cmdline:cmdoption-odoo-bin-longpolling-port}}\pysigline{\sphinxbfcode{\sphinxupquote{-{-}longpolling-port}}\sphinxcode{\sphinxupquote{~\textless{}port\textgreater{}}}}
TCP port for long-polling connections in multiprocessing or gevent mode,
defaults to 8072. Not used in default (threaded) mode.

\end{fulllineitems}



\subsubsection{logging}
\label{\detokenize{reference/cmdline:logging}}
By default, Odoo displays all logging of \sphinxhref{https://docs.python.org/2/library/logging.html\#logging.Logger.setLevel}{level} \sphinxcode{\sphinxupquote{info}} except for workflow
logging (\sphinxcode{\sphinxupquote{warning}} only), and log output is sent to \sphinxcode{\sphinxupquote{stdout}}. Various
options are available to redirect logging to other destinations and to
customize the amount of logging output
\index{odoo-bin command line option!--logfile \textless{}file\textgreater{}}\index{--logfile \textless{}file\textgreater{}!odoo-bin command line option}

\begin{fulllineitems}
\phantomsection\label{\detokenize{reference/cmdline:cmdoption-odoo-bin-logfile}}\pysigline{\sphinxbfcode{\sphinxupquote{-{-}logfile}}\sphinxcode{\sphinxupquote{~\textless{}file\textgreater{}}}}
sends logging output to the specified file instead of stdout. On Unix, the
file \sphinxhref{https://docs.python.org/2/library/logging.handlers.html\#watchedfilehandler}{can be managed by external log rotation programs}
and will automatically be reopened when replaced

\end{fulllineitems}

\index{odoo-bin command line option!--logrotate}\index{--logrotate!odoo-bin command line option}

\begin{fulllineitems}
\phantomsection\label{\detokenize{reference/cmdline:cmdoption-odoo-bin-logrotate}}\pysigline{\sphinxbfcode{\sphinxupquote{-{-}logrotate}}\sphinxcode{\sphinxupquote{}}}
enables \sphinxhref{https://docs.python.org/2/library/logging.handlers.html\#timedrotatingfilehandler}{log rotation}
daily, keeping 30 backups. Log rotation frequency and number of backups is
not configurable.

\end{fulllineitems}

\index{odoo-bin command line option!--syslog}\index{--syslog!odoo-bin command line option}

\begin{fulllineitems}
\phantomsection\label{\detokenize{reference/cmdline:cmdoption-odoo-bin-syslog}}\pysigline{\sphinxbfcode{\sphinxupquote{-{-}syslog}}\sphinxcode{\sphinxupquote{}}}
logs to the system’s event logger: \sphinxhref{https://docs.python.org/2/library/logging.handlers.html\#sysloghandler}{syslog on unices}
and \sphinxhref{https://docs.python.org/2/library/logging.handlers.html\#nteventloghandler}{the Event Log on Windows}.

Neither is configurable

\end{fulllineitems}

\index{odoo-bin command line option!--log-db \textless{}dbname\textgreater{}}\index{--log-db \textless{}dbname\textgreater{}!odoo-bin command line option}

\begin{fulllineitems}
\phantomsection\label{\detokenize{reference/cmdline:cmdoption-odoo-bin-log-db}}\pysigline{\sphinxbfcode{\sphinxupquote{-{-}log-db}}\sphinxcode{\sphinxupquote{~\textless{}dbname\textgreater{}}}}
logs to the \sphinxcode{\sphinxupquote{ir.logging}} model (\sphinxcode{\sphinxupquote{ir\_logging}} table) of the specified
database. The database can be the name of a database in the “current”
PostgreSQL, or \sphinxhref{http://www.postgresql.org/docs/9.2/static/libpq-connect.html\#AEN38208}{a PostgreSQL URI} for e.g. log aggregation

\end{fulllineitems}

\index{odoo-bin command line option!--log-handler \textless{}handler-spec\textgreater{}}\index{--log-handler \textless{}handler-spec\textgreater{}!odoo-bin command line option}

\begin{fulllineitems}
\phantomsection\label{\detokenize{reference/cmdline:cmdoption-odoo-bin-log-handler}}\pysigline{\sphinxbfcode{\sphinxupquote{-{-}log-handler}}\sphinxcode{\sphinxupquote{~\textless{}handler-spec\textgreater{}}}}
\sphinxcode{\sphinxupquote{\sphinxstyleemphasis{LOGGER}:\sphinxstyleemphasis{LEVEL}}}, enables \sphinxcode{\sphinxupquote{LOGGER}} at the provided \sphinxcode{\sphinxupquote{LEVEL}}
e.g. \sphinxcode{\sphinxupquote{odoo.models:DEBUG}} will enable all logging messages at or above
\sphinxcode{\sphinxupquote{DEBUG}} level in the models.
\begin{itemize}
\item {} 
The colon \sphinxcode{\sphinxupquote{:}} is mandatory

\item {} 
The logger can be omitted to configure the root (default) handler

\item {} 
If the level is omitted, the logger is set to \sphinxcode{\sphinxupquote{INFO}}

\end{itemize}

The option can be repeated to configure multiple loggers e.g.

\fvset{hllines={, ,}}%
\begin{sphinxVerbatim}[commandchars=\\\{\}]
\PYG{g+gp}{\PYGZdl{}} odoo\PYGZhy{}bin \PYGZhy{}\PYGZhy{}log\PYGZhy{}handler :DEBUG \PYGZhy{}\PYGZhy{}log\PYGZhy{}handler werkzeug:CRITICAL \PYGZhy{}\PYGZhy{}log\PYGZhy{}handler odoo.fields:WARNING
\end{sphinxVerbatim}

\end{fulllineitems}

\index{odoo-bin command line option!--log-request}\index{--log-request!odoo-bin command line option}

\begin{fulllineitems}
\phantomsection\label{\detokenize{reference/cmdline:cmdoption-odoo-bin-log-request}}\pysigline{\sphinxbfcode{\sphinxupquote{-{-}log-request}}\sphinxcode{\sphinxupquote{}}}
enable DEBUG logging for RPC requests, equivalent to
\sphinxcode{\sphinxupquote{-{-}log-handler=odoo.http.rpc.request:DEBUG}}

\end{fulllineitems}

\index{odoo-bin command line option!--log-response}\index{--log-response!odoo-bin command line option}

\begin{fulllineitems}
\phantomsection\label{\detokenize{reference/cmdline:cmdoption-odoo-bin-log-response}}\pysigline{\sphinxbfcode{\sphinxupquote{-{-}log-response}}\sphinxcode{\sphinxupquote{}}}
enable DEBUG logging for RPC responses, equivalent to
\sphinxcode{\sphinxupquote{-{-}log-handler=odoo.http.rpc.response:DEBUG}}

\end{fulllineitems}

\index{odoo-bin command line option!--log-web}\index{--log-web!odoo-bin command line option}

\begin{fulllineitems}
\phantomsection\label{\detokenize{reference/cmdline:cmdoption-odoo-bin-log-web}}\pysigline{\sphinxbfcode{\sphinxupquote{-{-}log-web}}\sphinxcode{\sphinxupquote{}}}
enables DEBUG logging of HTTP requests and responses, equivalent to
\sphinxcode{\sphinxupquote{-{-}log-handler=odoo.http:DEBUG}}

\end{fulllineitems}

\index{odoo-bin command line option!--log-sql}\index{--log-sql!odoo-bin command line option}

\begin{fulllineitems}
\phantomsection\label{\detokenize{reference/cmdline:cmdoption-odoo-bin-log-sql}}\pysigline{\sphinxbfcode{\sphinxupquote{-{-}log-sql}}\sphinxcode{\sphinxupquote{}}}
enables DEBUG logging of SQL querying, equivalent to
\sphinxcode{\sphinxupquote{-{-}log-handler=odoo.sql\_db:DEBUG}}

\end{fulllineitems}

\index{odoo-bin command line option!--log-level \textless{}level\textgreater{}}\index{--log-level \textless{}level\textgreater{}!odoo-bin command line option}

\begin{fulllineitems}
\phantomsection\label{\detokenize{reference/cmdline:cmdoption-odoo-bin-log-level}}\pysigline{\sphinxbfcode{\sphinxupquote{-{-}log-level}}\sphinxcode{\sphinxupquote{~\textless{}level\textgreater{}}}}
Shortcut to more easily set predefined levels on specific loggers. “real”
levels (\sphinxcode{\sphinxupquote{critical}}, \sphinxcode{\sphinxupquote{error}}, \sphinxcode{\sphinxupquote{warn}}, \sphinxcode{\sphinxupquote{debug}}) are set on the
\sphinxcode{\sphinxupquote{odoo}} and \sphinxcode{\sphinxupquote{werkzeug}} loggers (except for \sphinxcode{\sphinxupquote{debug}} which is only
set on \sphinxcode{\sphinxupquote{odoo}}).

Odoo also provides debugging pseudo-levels which apply to different sets
of loggers:
\begin{description}
\item[{\sphinxcode{\sphinxupquote{debug\_sql}}}] \leavevmode
sets the SQL logger to \sphinxcode{\sphinxupquote{debug}}

equivalent to \sphinxcode{\sphinxupquote{-{-}log-sql}}

\item[{\sphinxcode{\sphinxupquote{debug\_rpc}}}] \leavevmode
sets the \sphinxcode{\sphinxupquote{odoo}} and HTTP request loggers to \sphinxcode{\sphinxupquote{debug}}

equivalent to \sphinxcode{\sphinxupquote{-{-}log-level debug -{-}log-request}}

\item[{\sphinxcode{\sphinxupquote{debug\_rpc\_answer}}}] \leavevmode
sets the \sphinxcode{\sphinxupquote{odoo}} and HTTP request and response loggers to
\sphinxcode{\sphinxupquote{debug}}

equivalent to \sphinxcode{\sphinxupquote{-{-}log-level debug -{-}log-request -{-}log-response}}

\end{description}

\begin{sphinxadmonition}{note}{Note:}
In case of conflict between {\hyperref[\detokenize{reference/cmdline:cmdoption-odoo-bin-log-level}]{\sphinxcrossref{\sphinxcode{\sphinxupquote{-{-}log-level}}}}} and
{\hyperref[\detokenize{reference/cmdline:cmdoption-odoo-bin-log-handler}]{\sphinxcrossref{\sphinxcode{\sphinxupquote{-{-}log-handler}}}}}, the latter is used
\end{sphinxadmonition}

\end{fulllineitems}



\subsubsection{translations}
\label{\detokenize{reference/cmdline:translations}}\index{odoo-bin command line option!--i18n-import}\index{--i18n-import!odoo-bin command line option}

\begin{fulllineitems}
\phantomsection\label{\detokenize{reference/cmdline:cmdoption-odoo-bin-i18n-import}}\pysigline{\sphinxbfcode{\sphinxupquote{-{-}i18n-import}}\sphinxcode{\sphinxupquote{}}}
\end{fulllineitems}

\index{odoo-bin command line option!--i18n-export}\index{--i18n-export!odoo-bin command line option}

\begin{fulllineitems}
\phantomsection\label{\detokenize{reference/cmdline:cmdoption-odoo-bin-i18n-export}}\pysigline{\sphinxbfcode{\sphinxupquote{-{-}i18n-export}}\sphinxcode{\sphinxupquote{}}}
\end{fulllineitems}



\subsubsection{emails}
\label{\detokenize{reference/cmdline:emails}}\index{odoo-bin command line option!--email-from \textless{}address\textgreater{}}\index{--email-from \textless{}address\textgreater{}!odoo-bin command line option}

\begin{fulllineitems}
\phantomsection\label{\detokenize{reference/cmdline:cmdoption-odoo-bin-email-from}}\pysigline{\sphinxbfcode{\sphinxupquote{-{-}email-from}}\sphinxcode{\sphinxupquote{~\textless{}address\textgreater{}}}}
Email address used as \textless{}FROM\textgreater{} when Odoo needs to send mails

\end{fulllineitems}

\index{odoo-bin command line option!--smtp \textless{}server\textgreater{}}\index{--smtp \textless{}server\textgreater{}!odoo-bin command line option}

\begin{fulllineitems}
\phantomsection\label{\detokenize{reference/cmdline:cmdoption-odoo-bin-smtp}}\pysigline{\sphinxbfcode{\sphinxupquote{-{-}smtp}}\sphinxcode{\sphinxupquote{~\textless{}server\textgreater{}}}}
Address of the SMTP server to connect to in order to send mails

\end{fulllineitems}

\index{odoo-bin command line option!--smtp-port \textless{}port\textgreater{}}\index{--smtp-port \textless{}port\textgreater{}!odoo-bin command line option}

\begin{fulllineitems}
\phantomsection\label{\detokenize{reference/cmdline:cmdoption-odoo-bin-smtp-port}}\pysigline{\sphinxbfcode{\sphinxupquote{-{-}smtp-port}}\sphinxcode{\sphinxupquote{~\textless{}port\textgreater{}}}}
\end{fulllineitems}

\index{odoo-bin command line option!--smtp-ssl}\index{--smtp-ssl!odoo-bin command line option}

\begin{fulllineitems}
\phantomsection\label{\detokenize{reference/cmdline:cmdoption-odoo-bin-smtp-ssl}}\pysigline{\sphinxbfcode{\sphinxupquote{-{-}smtp-ssl}}\sphinxcode{\sphinxupquote{}}}
If set, odoo should use SSL/STARTSSL SMTP connections

\end{fulllineitems}

\index{odoo-bin command line option!--smtp-user \textless{}name\textgreater{}}\index{--smtp-user \textless{}name\textgreater{}!odoo-bin command line option}

\begin{fulllineitems}
\phantomsection\label{\detokenize{reference/cmdline:cmdoption-odoo-bin-smtp-user}}\pysigline{\sphinxbfcode{\sphinxupquote{-{-}smtp-user}}\sphinxcode{\sphinxupquote{~\textless{}name\textgreater{}}}}
Username to connect to the SMTP server

\end{fulllineitems}

\index{odoo-bin command line option!--smtp-password \textless{}password\textgreater{}}\index{--smtp-password \textless{}password\textgreater{}!odoo-bin command line option}

\begin{fulllineitems}
\phantomsection\label{\detokenize{reference/cmdline:cmdoption-odoo-bin-smtp-password}}\pysigline{\sphinxbfcode{\sphinxupquote{-{-}smtp-password}}\sphinxcode{\sphinxupquote{~\textless{}password\textgreater{}}}}
Password to connect to the SMTP server

\end{fulllineitems}



\subsection{Scaffolding}
\label{\detokenize{reference/cmdline:scaffolding}}\label{\detokenize{reference/cmdline:reference-cmdline-scaffold}}
Scaffolding is the automated creation of a skeleton structure to simplify
bootstrapping (of new modules, in the case of Odoo). While not necessary it
avoids the tedium of setting up basic structures and looking up what all
starting requirements are.

Scaffolding is available via the \sphinxstyleliteralstrong{\sphinxupquote{odoo-bin scaffold}} subcommand.
\index{odoo-bin-scaffold command line option!-t \textless{}template\textgreater{}}\index{-t \textless{}template\textgreater{}!odoo-bin-scaffold command line option}

\begin{fulllineitems}
\phantomsection\label{\detokenize{reference/cmdline:cmdoption-odoo-bin-scaffold-t}}\pysigline{\sphinxbfcode{\sphinxupquote{-t}}\sphinxcode{\sphinxupquote{~\textless{}template\textgreater{}}}}
a template directory, files are passed through \sphinxhref{http://jinja.pocoo.org}{jinja2} then copied to
the \sphinxcode{\sphinxupquote{destination}} directory

\end{fulllineitems}

\index{odoo-bin-scaffold command line option!name}\index{name!odoo-bin-scaffold command line option}

\begin{fulllineitems}
\phantomsection\label{\detokenize{reference/cmdline:cmdoption-odoo-bin-scaffold-arg-name}}\pysigline{\sphinxbfcode{\sphinxupquote{name}}\sphinxcode{\sphinxupquote{}}}
the name of the module to create, may munged in various manners to
generate programmatic names (e.g. module directory name, model names, …)

\end{fulllineitems}

\index{odoo-bin-scaffold command line option!destination}\index{destination!odoo-bin-scaffold command line option}

\begin{fulllineitems}
\phantomsection\label{\detokenize{reference/cmdline:cmdoption-odoo-bin-scaffold-arg-destination}}\pysigline{\sphinxbfcode{\sphinxupquote{destination}}\sphinxcode{\sphinxupquote{}}}
directory in which to create the new module, defaults to the current
directory

\end{fulllineitems}



\subsection{Configuration file}
\label{\detokenize{reference/cmdline:configuration-file}}\label{\detokenize{reference/cmdline:reference-cmdline-config}}
Most of the command-line options can also be specified via a configuration
file. Most of the time, they use similar names with the prefix \sphinxcode{\sphinxupquote{-}} removed
and other \sphinxcode{\sphinxupquote{-}} are replaced by \sphinxcode{\sphinxupquote{\_}} e.g. {\hyperref[\detokenize{reference/cmdline:cmdoption-odoo-bin-db-template}]{\sphinxcrossref{\sphinxcode{\sphinxupquote{-{-}db-template}}}}} becomes
\sphinxcode{\sphinxupquote{db\_template}}.

Some conversions don’t match the pattern:
\begin{itemize}
\item {} 
{\hyperref[\detokenize{reference/cmdline:cmdoption-odoo-bin-db-filter}]{\sphinxcrossref{\sphinxcode{\sphinxupquote{-{-}db-filter}}}}} becomes \sphinxcode{\sphinxupquote{dbfilter}}

\item {} 
{\hyperref[\detokenize{reference/cmdline:cmdoption-odoo-bin-no-http}]{\sphinxcrossref{\sphinxcode{\sphinxupquote{-{-}no-http}}}}} corresponds to the \sphinxcode{\sphinxupquote{http\_enable}} boolean

\item {} 
logging presets (all options starting with \sphinxcode{\sphinxupquote{-{-}log-}} except for
{\hyperref[\detokenize{reference/cmdline:cmdoption-odoo-bin-log-handler}]{\sphinxcrossref{\sphinxcode{\sphinxupquote{-{-}log-handler}}}}} and {\hyperref[\detokenize{reference/cmdline:cmdoption-odoo-bin-log-db}]{\sphinxcrossref{\sphinxcode{\sphinxupquote{-{-}log-db}}}}}) just add content to
\sphinxcode{\sphinxupquote{log\_handler}}, use that directly in the configuration file

\item {} 
{\hyperref[\detokenize{reference/cmdline:cmdoption-odoo-bin-smtp}]{\sphinxcrossref{\sphinxcode{\sphinxupquote{-{-}smtp}}}}} is stored as \sphinxcode{\sphinxupquote{smtp\_server}}

\item {} 
{\hyperref[\detokenize{reference/cmdline:cmdoption-odoo-bin-d}]{\sphinxcrossref{\sphinxcode{\sphinxupquote{-{-}database}}}}} is stored as \sphinxcode{\sphinxupquote{db\_name}}

\item {} 
{\hyperref[\detokenize{reference/cmdline:cmdoption-odoo-bin-i18n-import}]{\sphinxcrossref{\sphinxcode{\sphinxupquote{-{-}i18n-import}}}}} and {\hyperref[\detokenize{reference/cmdline:cmdoption-odoo-bin-i18n-export}]{\sphinxcrossref{\sphinxcode{\sphinxupquote{-{-}i18n-export}}}}} aren’t available at all
from configuration files

\end{itemize}

The default configuration file is \sphinxcode{\sphinxupquote{\sphinxstyleemphasis{\$HOME}/.odoorc}} which
can be overridden using {\hyperref[\detokenize{reference/cmdline:cmdoption-odoo-bin-c}]{\sphinxcrossref{\sphinxcode{\sphinxupquote{-{-}config}}}}}. Specifying
{\hyperref[\detokenize{reference/cmdline:cmdoption-odoo-bin-s}]{\sphinxcrossref{\sphinxcode{\sphinxupquote{-{-}save}}}}} will save the current configuration state back
to that file.


\section{Security in Odoo}
\label{\detokenize{reference/security:pyinotify}}\label{\detokenize{reference/security:reference-security}}\label{\detokenize{reference/security:security-in-odoo}}\label{\detokenize{reference/security::doc}}
Aside from manually managing access using custom code, Odoo provides two main
data-driven mechanisms to manage or restrict access to data.

Both mechanisms are linked to specific users through \sphinxstyleemphasis{groups}: a user belongs
to any number of groups, and security mechanisms are associated to groups,
thus applying security mechamisms to users.


\subsection{Access Control}
\label{\detokenize{reference/security:reference-security-acl}}\label{\detokenize{reference/security:access-control}}
Managed by the \sphinxcode{\sphinxupquote{ir.model.access}} records, defines access to a whole model.

Each access control has a model to which it grants permissions, the
permissions it grants and optionally a group.

Access controls are additive, for a given model a user has access all
permissions granted to any of its groups: if the user belongs to one group
which allows writing and another which allows deleting, they can both write
and delete.

If no group is specified, the access control applies to all users, otherwise
it only applies to the members of the given group.

Available permissions are creation (\sphinxcode{\sphinxupquote{perm\_create}}), searching and reading
(\sphinxcode{\sphinxupquote{perm\_read}}), updating existing records (\sphinxcode{\sphinxupquote{perm\_write}}) and deleting
existing records (\sphinxcode{\sphinxupquote{perm\_unlink}})


\subsection{Record Rules}
\label{\detokenize{reference/security:record-rules}}\label{\detokenize{reference/security:reference-security-rules}}
Record rules are conditions that records must satisfy for an operation
(create, read, update or delete) to be allowed. It is applied record-by-record
after access control has been applied.

A record rule has:
\begin{itemize}
\item {} 
a model on which it applies

\item {} 
a set of permissions to which it applies (e.g. if \sphinxcode{\sphinxupquote{perm\_read}} is set, the
rule will only be checked when reading a record)

\item {} 
a set of user groups to which the rule applies, if no group is specified
the rule is \sphinxstyleemphasis{global}

\item {} 
a {\hyperref[\detokenize{reference/orm:reference-orm-domains}]{\sphinxcrossref{\DUrole{std,std-ref}{domain}}}} used to check whether a given record
matches the rule (and is accessible) or does not (and is not accessible).
The domain is evaluated with two variables in context: \sphinxcode{\sphinxupquote{user}} is the
current user’s record and \sphinxcode{\sphinxupquote{time}} is the \sphinxhref{https://docs.python.org/2/library/time.html}{time module}

\end{itemize}

Global rules and group rules (rules restricted to specific groups versus
groups applying to all users) are used quite differently:
\begin{itemize}
\item {} 
Global rules are subtractive, they \sphinxstyleemphasis{must all} be matched for a record to be
accessible

\item {} 
Group rules are additive, if \sphinxstyleemphasis{any} of them matches (and all global rules
match) then the record is accessible

\end{itemize}

This means the first \sphinxstyleemphasis{group rule} restricts access, but any further
\sphinxstyleemphasis{group rule} expands it, while \sphinxstyleemphasis{global rules} can only ever restrict access
(or have no effect).

\begin{sphinxadmonition}{warning}{Warning:}
record rules do not apply to the Administrator user

although access rules do
\end{sphinxadmonition}


\subsection{Field Access}
\label{\detokenize{reference/security:reference-security-fields}}\label{\detokenize{reference/security:field-access}}
\DUrole{versionmodified}{New in version 7.0.}

An ORM {\hyperref[\detokenize{reference/orm:odoo.fields.Field}]{\sphinxcrossref{\sphinxcode{\sphinxupquote{Field}}}}} can have a \sphinxcode{\sphinxupquote{groups}} attribute
providing a list of groups (as a comma-separated string of
\DUrole{xref,std,std-term}{external identifiers}).

If the current user is not in one of the listed groups, he will not have
access to the field:
\begin{itemize}
\item {} 
restricted fields are automatically removed from requested views

\item {} 
restricted fields are removed from {\hyperref[\detokenize{reference/orm:odoo.models.Model.fields_get}]{\sphinxcrossref{\sphinxcode{\sphinxupquote{fields\_get()}}}}}
responses

\item {} 
attempts to (explicitly) read from or write to restricted fields results in
an access error

\end{itemize}


\section{Testing Odoo}
\label{\detokenize{reference/testing::doc}}\label{\detokenize{reference/testing:reference-testing}}\label{\detokenize{reference/testing:testing-odoo}}\label{\detokenize{reference/testing:time-module}}
There are many ways to test an application.  In Odoo, we have three kinds of
tests
\begin{itemize}
\item {} 
python unit tests: useful for testing model business logic

\item {} 
js unit tests: this is necessary to test the javascript code in isolation

\item {} 
tours: this is a form of integration testing.  The tours ensure that the
python and the javascript parts properly talk to each other.

\end{itemize}


\subsection{Testing Python code}
\label{\detokenize{reference/testing:testing-python-code}}
Odoo provides support for testing modules using unittest.

To write tests, simply define a \sphinxcode{\sphinxupquote{tests}} sub-package in your module, it will
be automatically inspected for test modules. Test modules should have a name
starting with \sphinxcode{\sphinxupquote{test\_}} and should be imported from \sphinxcode{\sphinxupquote{tests/\_\_init\_\_.py}},
e.g.

\fvset{hllines={, ,}}%
\begin{sphinxVerbatim}[commandchars=\\\{\}]
your\PYGZus{}module
\textbar{}\PYGZhy{}\PYGZhy{} ...
{}`\PYGZhy{}\PYGZhy{} tests
    \textbar{}\PYGZhy{}\PYGZhy{} \PYGZus{}\PYGZus{}init\PYGZus{}\PYGZus{}.py
    \textbar{}\PYGZhy{}\PYGZhy{} test\PYGZus{}bar.py
    {}`\PYGZhy{}\PYGZhy{} test\PYGZus{}foo.py
\end{sphinxVerbatim}

and \sphinxcode{\sphinxupquote{\_\_init\_\_.py}} contains:

\fvset{hllines={, ,}}%
\begin{sphinxVerbatim}[commandchars=\\\{\}]
\PYG{k+kn}{from} \PYG{n+nn}{.} \PYG{k}{import} \PYG{n}{test\PYGZus{}foo}\PYG{p}{,} \PYG{n}{test\PYGZus{}bar}
\end{sphinxVerbatim}

\begin{sphinxadmonition}{warning}{Warning:}
test modules which are not imported from \sphinxcode{\sphinxupquote{tests/\_\_init\_\_.py}} will not be
run
\end{sphinxadmonition}

\DUrole{versionmodified}{Changed in version 8.0: }previously, the test runner would only run modules added to two lists
\sphinxcode{\sphinxupquote{fast\_suite}} and \sphinxcode{\sphinxupquote{checks}} in \sphinxcode{\sphinxupquote{tests/\_\_init\_\_.py}}. In 8.0 it will
run all imported modules

The test runner will simply run any test case, as described in the official
\sphinxhref{https://docs.python.org/2/library/unittest.html}{unittest documentation}, but Odoo provides a number of utilities and helpers
related to testing Odoo content (modules, mainly):
\index{TransactionCase (class in odoo.tests.common)}

\begin{fulllineitems}
\phantomsection\label{\detokenize{reference/testing:odoo.tests.common.TransactionCase}}\pysiglinewithargsret{\sphinxbfcode{\sphinxupquote{class }}\sphinxcode{\sphinxupquote{odoo.tests.common.}}\sphinxbfcode{\sphinxupquote{TransactionCase}}}{\emph{methodName='runTest'}}{}
TestCase in which each test method is run in its own transaction,
and with its own cursor. The transaction is rolled back and the cursor
is closed after each test.
\index{browse\_ref() (odoo.tests.common.TransactionCase method)}

\begin{fulllineitems}
\phantomsection\label{\detokenize{reference/testing:odoo.tests.common.TransactionCase.browse_ref}}\pysiglinewithargsret{\sphinxbfcode{\sphinxupquote{browse\_ref}}}{\emph{xid}}{}
Returns a record object for the provided
\DUrole{xref,std,std-term}{external identifier}
\begin{quote}\begin{description}
\item[{Parameters}] \leavevmode
\sphinxstyleliteralstrong{\sphinxupquote{xid}} \textendash{} fully-qualified \DUrole{xref,std,std-term}{external identifier}, in the form
\sphinxcode{\sphinxupquote{\sphinxstyleemphasis{module}.\sphinxstyleemphasis{identifier}}}

\item[{Raise}] \leavevmode
ValueError if not found

\item[{Returns}] \leavevmode
\sphinxcode{\sphinxupquote{BaseModel}}

\end{description}\end{quote}

\end{fulllineitems}

\index{ref() (odoo.tests.common.TransactionCase method)}

\begin{fulllineitems}
\phantomsection\label{\detokenize{reference/testing:odoo.tests.common.TransactionCase.ref}}\pysiglinewithargsret{\sphinxbfcode{\sphinxupquote{ref}}}{\emph{xid}}{}
Returns database ID for the provided \DUrole{xref,std,std-term}{external identifier},
shortcut for \sphinxcode{\sphinxupquote{get\_object\_reference}}
\begin{quote}\begin{description}
\item[{Parameters}] \leavevmode
\sphinxstyleliteralstrong{\sphinxupquote{xid}} \textendash{} fully-qualified \DUrole{xref,std,std-term}{external identifier}, in the form
\sphinxcode{\sphinxupquote{\sphinxstyleemphasis{module}.\sphinxstyleemphasis{identifier}}}

\item[{Raise}] \leavevmode
ValueError if not found

\item[{Returns}] \leavevmode
registered id

\end{description}\end{quote}

\end{fulllineitems}


\end{fulllineitems}

\index{SingleTransactionCase (class in odoo.tests.common)}

\begin{fulllineitems}
\phantomsection\label{\detokenize{reference/testing:odoo.tests.common.SingleTransactionCase}}\pysiglinewithargsret{\sphinxbfcode{\sphinxupquote{class }}\sphinxcode{\sphinxupquote{odoo.tests.common.}}\sphinxbfcode{\sphinxupquote{SingleTransactionCase}}}{\emph{methodName='runTest'}}{}
TestCase in which all test methods are run in the same transaction,
the transaction is started with the first test method and rolled back at
the end of the last.
\index{browse\_ref() (odoo.tests.common.SingleTransactionCase method)}

\begin{fulllineitems}
\phantomsection\label{\detokenize{reference/testing:odoo.tests.common.SingleTransactionCase.browse_ref}}\pysiglinewithargsret{\sphinxbfcode{\sphinxupquote{browse\_ref}}}{\emph{xid}}{}
Returns a record object for the provided
\DUrole{xref,std,std-term}{external identifier}
\begin{quote}\begin{description}
\item[{Parameters}] \leavevmode
\sphinxstyleliteralstrong{\sphinxupquote{xid}} \textendash{} fully-qualified \DUrole{xref,std,std-term}{external identifier}, in the form
\sphinxcode{\sphinxupquote{\sphinxstyleemphasis{module}.\sphinxstyleemphasis{identifier}}}

\item[{Raise}] \leavevmode
ValueError if not found

\item[{Returns}] \leavevmode
\sphinxcode{\sphinxupquote{BaseModel}}

\end{description}\end{quote}

\end{fulllineitems}

\index{ref() (odoo.tests.common.SingleTransactionCase method)}

\begin{fulllineitems}
\phantomsection\label{\detokenize{reference/testing:odoo.tests.common.SingleTransactionCase.ref}}\pysiglinewithargsret{\sphinxbfcode{\sphinxupquote{ref}}}{\emph{xid}}{}
Returns database ID for the provided \DUrole{xref,std,std-term}{external identifier},
shortcut for \sphinxcode{\sphinxupquote{get\_object\_reference}}
\begin{quote}\begin{description}
\item[{Parameters}] \leavevmode
\sphinxstyleliteralstrong{\sphinxupquote{xid}} \textendash{} fully-qualified \DUrole{xref,std,std-term}{external identifier}, in the form
\sphinxcode{\sphinxupquote{\sphinxstyleemphasis{module}.\sphinxstyleemphasis{identifier}}}

\item[{Raise}] \leavevmode
ValueError if not found

\item[{Returns}] \leavevmode
registered id

\end{description}\end{quote}

\end{fulllineitems}


\end{fulllineitems}

\index{SavepointCase (class in odoo.tests.common)}

\begin{fulllineitems}
\phantomsection\label{\detokenize{reference/testing:odoo.tests.common.SavepointCase}}\pysiglinewithargsret{\sphinxbfcode{\sphinxupquote{class }}\sphinxcode{\sphinxupquote{odoo.tests.common.}}\sphinxbfcode{\sphinxupquote{SavepointCase}}}{\emph{methodName='runTest'}}{}
Similar to {\hyperref[\detokenize{reference/testing:odoo.tests.common.SingleTransactionCase}]{\sphinxcrossref{\sphinxcode{\sphinxupquote{SingleTransactionCase}}}}} in that all test methods
are run in a single transaction \sphinxstyleemphasis{but} each test case is run inside a
rollbacked savepoint (sub-transaction).

Useful for test cases containing fast tests but with significant database
setup common to all cases (complex in-db test data): \sphinxcode{\sphinxupquote{setUpClass()}}
can be used to generate db test data once, then all test cases use the
same data without influencing one another but without having to recreate
the test data either.

\end{fulllineitems}

\index{HttpCase (class in odoo.tests.common)}

\begin{fulllineitems}
\phantomsection\label{\detokenize{reference/testing:odoo.tests.common.HttpCase}}\pysiglinewithargsret{\sphinxbfcode{\sphinxupquote{class }}\sphinxcode{\sphinxupquote{odoo.tests.common.}}\sphinxbfcode{\sphinxupquote{HttpCase}}}{\emph{methodName='runTest'}}{}
Transactional HTTP TestCase with url\_open and phantomjs helpers.
\index{browse\_ref() (odoo.tests.common.HttpCase method)}

\begin{fulllineitems}
\phantomsection\label{\detokenize{reference/testing:odoo.tests.common.HttpCase.browse_ref}}\pysiglinewithargsret{\sphinxbfcode{\sphinxupquote{browse\_ref}}}{\emph{xid}}{}
Returns a record object for the provided
\DUrole{xref,std,std-term}{external identifier}
\begin{quote}\begin{description}
\item[{Parameters}] \leavevmode
\sphinxstyleliteralstrong{\sphinxupquote{xid}} \textendash{} fully-qualified \DUrole{xref,std,std-term}{external identifier}, in the form
\sphinxcode{\sphinxupquote{\sphinxstyleemphasis{module}.\sphinxstyleemphasis{identifier}}}

\item[{Raise}] \leavevmode
ValueError if not found

\item[{Returns}] \leavevmode
\sphinxcode{\sphinxupquote{BaseModel}}

\end{description}\end{quote}

\end{fulllineitems}

\index{phantom\_js() (odoo.tests.common.HttpCase method)}

\begin{fulllineitems}
\phantomsection\label{\detokenize{reference/testing:odoo.tests.common.HttpCase.phantom_js}}\pysiglinewithargsret{\sphinxbfcode{\sphinxupquote{phantom\_js}}}{\emph{url\_path}, \emph{code}, \emph{ready='window'}, \emph{login=None}, \emph{timeout=60}, \emph{**kw}}{}
Test js code running in the browser
- optionnally log as ‘login’
- load page given by url\_path
- wait for ready object to be available
- eval(code) inside the page

To signal success test do:
console.log(‘ok’)

To signal failure do:
console.log(‘error’)

If neither are done before timeout test fails.

\end{fulllineitems}

\index{ref() (odoo.tests.common.HttpCase method)}

\begin{fulllineitems}
\phantomsection\label{\detokenize{reference/testing:odoo.tests.common.HttpCase.ref}}\pysiglinewithargsret{\sphinxbfcode{\sphinxupquote{ref}}}{\emph{xid}}{}
Returns database ID for the provided \DUrole{xref,std,std-term}{external identifier},
shortcut for \sphinxcode{\sphinxupquote{get\_object\_reference}}
\begin{quote}\begin{description}
\item[{Parameters}] \leavevmode
\sphinxstyleliteralstrong{\sphinxupquote{xid}} \textendash{} fully-qualified \DUrole{xref,std,std-term}{external identifier}, in the form
\sphinxcode{\sphinxupquote{\sphinxstyleemphasis{module}.\sphinxstyleemphasis{identifier}}}

\item[{Raise}] \leavevmode
ValueError if not found

\item[{Returns}] \leavevmode
registered id

\end{description}\end{quote}

\end{fulllineitems}


\end{fulllineitems}


By default, tests are run once right after the corresponding module has been
installed. Test cases can also be configured to run after all modules have
been installed, and not run right after the module installation:
\index{at\_install() (in module odoo.tests.common)}

\begin{fulllineitems}
\phantomsection\label{\detokenize{reference/testing:odoo.tests.common.at_install}}\pysiglinewithargsret{\sphinxcode{\sphinxupquote{odoo.tests.common.}}\sphinxbfcode{\sphinxupquote{at\_install}}}{\emph{flag}}{}
Sets the at-install state of a test, the flag is a boolean specifying
whether the test should (\sphinxcode{\sphinxupquote{True}}) or should not (\sphinxcode{\sphinxupquote{False}}) run during
module installation.

By default, tests are run right after installing the module, before
starting the installation of the next module.

\end{fulllineitems}

\index{post\_install() (in module odoo.tests.common)}

\begin{fulllineitems}
\phantomsection\label{\detokenize{reference/testing:odoo.tests.common.post_install}}\pysiglinewithargsret{\sphinxcode{\sphinxupquote{odoo.tests.common.}}\sphinxbfcode{\sphinxupquote{post\_install}}}{\emph{flag}}{}
Sets the post-install state of a test. The flag is a boolean
specifying whether the test should or should not run after a set of
module installations.

By default, tests are \sphinxstyleemphasis{not} run after installation of all modules in the
current installation set.

\end{fulllineitems}


The most common situation is to use
{\hyperref[\detokenize{reference/testing:odoo.tests.common.TransactionCase}]{\sphinxcrossref{\sphinxcode{\sphinxupquote{TransactionCase}}}}} and test a property of a model
in each method:

\fvset{hllines={, ,}}%
\begin{sphinxVerbatim}[commandchars=\\\{\}]
\PYG{k}{class} \PYG{n+nc}{TestModelA}\PYG{p}{(}\PYG{n}{common}\PYG{o}{.}\PYG{n}{TransactionCase}\PYG{p}{)}\PYG{p}{:}
    \PYG{k}{def} \PYG{n+nf}{test\PYGZus{}some\PYGZus{}action}\PYG{p}{(}\PYG{n+nb+bp}{self}\PYG{p}{)}\PYG{p}{:}
        \PYG{n}{record} \PYG{o}{=} \PYG{n+nb+bp}{self}\PYG{o}{.}\PYG{n}{env}\PYG{p}{[}\PYG{l+s+s1}{\PYGZsq{}}\PYG{l+s+s1}{model.a}\PYG{l+s+s1}{\PYGZsq{}}\PYG{p}{]}\PYG{o}{.}\PYG{n}{create}\PYG{p}{(}\PYG{p}{\PYGZob{}}\PYG{l+s+s1}{\PYGZsq{}}\PYG{l+s+s1}{field}\PYG{l+s+s1}{\PYGZsq{}}\PYG{p}{:} \PYG{l+s+s1}{\PYGZsq{}}\PYG{l+s+s1}{value}\PYG{l+s+s1}{\PYGZsq{}}\PYG{p}{\PYGZcb{}}\PYG{p}{)}
        \PYG{n}{record}\PYG{o}{.}\PYG{n}{some\PYGZus{}action}\PYG{p}{(}\PYG{p}{)}
        \PYG{n+nb+bp}{self}\PYG{o}{.}\PYG{n}{assertEqual}\PYG{p}{(}
            \PYG{n}{record}\PYG{o}{.}\PYG{n}{field}\PYG{p}{,}
            \PYG{n}{expected\PYGZus{}field\PYGZus{}value}\PYG{p}{)}

    \PYG{c+c1}{\PYGZsh{} other tests...}
\end{sphinxVerbatim}

\begin{sphinxadmonition}{note}{Note:}
Test methods must start with \sphinxcode{\sphinxupquote{test\_}}
\end{sphinxadmonition}


\subsubsection{Running tests}
\label{\detokenize{reference/testing:running-tests}}
Tests are automatically run when installing or updating modules if
{\hyperref[\detokenize{reference/cmdline:cmdoption-odoo-bin-test-enable}]{\sphinxcrossref{\sphinxcode{\sphinxupquote{-{-}test-enable}}}}} was enabled when starting the
Odoo server.

As of Odoo 8, running tests outside of the install/update cycle is not
supported.


\subsection{Testing JS code}
\label{\detokenize{reference/testing:unittest-documentation}}\label{\detokenize{reference/testing:testing-js-code}}

\subsubsection{Qunit test suite}
\label{\detokenize{reference/testing:qunit-test-suite}}
Odoo Web includes means to unit-test both the core code of
Odoo Web and your own javascript modules. On the javascript side,
unit-testing is based on \sphinxhref{http://qunitjs.com/}{QUnit} with a number of helpers and
extensions for better integration with Odoo.

To see what the runner looks like, find (or start) an Odoo server
with the web client enabled, and navigate to \sphinxcode{\sphinxupquote{/web/tests}}
This will show the runner selector, which lists all modules with javascript
unit tests, and allows starting any of them (or all javascript tests in all
modules at once).

\noindent{\hspace*{\fill}\sphinxincludegraphics{{runner}.png}\hspace*{\fill}}

Clicking any runner button will launch the corresponding tests in the
bundled \sphinxhref{http://qunitjs.com/}{QUnit} runner:

\noindent{\hspace*{\fill}\sphinxincludegraphics{{tests}.png}\hspace*{\fill}}


\subsubsection{Writing a test case}
\label{\detokenize{reference/testing:writing-a-test-case}}
This section will be updated as soon as possible.


\subsection{Integration Testing}
\label{\detokenize{reference/testing:integration-testing}}\label{\detokenize{reference/testing:qunit}}
Testing Python code and JS code separately is very useful, but it does not prove
that the web client and the server work together.  In order to do that, we can
write another kind of test: tours.  A tour is a mini scenario of some interesting
business flow.  It explains a sequence of steps that should be followed.  The
test runner will then create a phantom\_js browser, point it to the proper url
and simulate the click and inputs, according to the scenario.


\section{Web Controllers}
\label{\detokenize{reference/http:web-controllers}}\label{\detokenize{reference/http::doc}}

\subsection{Routing}
\label{\detokenize{reference/http:reference-http-routing}}\label{\detokenize{reference/http:routing}}\index{route() (in module odoo.http)}

\begin{fulllineitems}
\phantomsection\label{\detokenize{reference/http:odoo.http.route}}\pysiglinewithargsret{\sphinxcode{\sphinxupquote{odoo.http.}}\sphinxbfcode{\sphinxupquote{route}}}{\emph{route=None}, \emph{**kw}}{}
Decorator marking the decorated method as being a handler for
requests. The method must be part of a subclass of \sphinxcode{\sphinxupquote{Controller}}.
\begin{quote}\begin{description}
\item[{Parameters}] \leavevmode\begin{itemize}
\item {} 
\sphinxstyleliteralstrong{\sphinxupquote{route}} \textendash{} string or array. The route part that will determine which
http requests will match the decorated method. Can be a
single string or an array of strings. See werkzeug’s routing
documentation for the format of route expression (
\sphinxurl{http://werkzeug.pocoo.org/docs/routing/} ).

\item {} 
\sphinxstyleliteralstrong{\sphinxupquote{type}} \textendash{} The type of request, can be \sphinxcode{\sphinxupquote{'http'}} or \sphinxcode{\sphinxupquote{'json'}}.

\item {} 
\sphinxstyleliteralstrong{\sphinxupquote{auth}} \textendash{} 
The type of authentication method, can on of the following:
\begin{itemize}
\item {} 
\sphinxcode{\sphinxupquote{user}}: The user must be authenticated and the current request
will perform using the rights of the user.

\item {} 
\sphinxcode{\sphinxupquote{public}}: The user may or may not be authenticated. If she isn’t,
the current request will perform using the shared Public user.

\item {} 
\sphinxcode{\sphinxupquote{none}}: The method is always active, even if there is no
database. Mainly used by the framework and authentication
modules. There request code will not have any facilities to access
the database nor have any configuration indicating the current
database nor the current user.

\end{itemize}


\item {} 
\sphinxstyleliteralstrong{\sphinxupquote{methods}} \textendash{} A sequence of http methods this route applies to. If not
specified, all methods are allowed.

\item {} 
\sphinxstyleliteralstrong{\sphinxupquote{cors}} \textendash{} The Access-Control-Allow-Origin cors directive value.

\item {} 
\sphinxstyleliteralstrong{\sphinxupquote{csrf}} (\sphinxhref{https://docs.python.org/3/library/functions.html\#bool}{\sphinxstyleliteralemphasis{\sphinxupquote{bool}}}) \textendash{} 
Whether CSRF protection should be enabled for the route.

Defaults to \sphinxcode{\sphinxupquote{True}}. See {\hyperref[\detokenize{reference/http:csrf}]{\sphinxcrossref{\DUrole{std,std-ref}{CSRF Protection}}}} for more.


\end{itemize}

\end{description}\end{quote}
\phantomsection\label{\detokenize{reference/http:csrf}}
\begin{sphinxadmonition}{note}{CSRF Protection}

\DUrole{versionmodified}{New in version 9.0.}

Odoo implements token-based \sphinxhref{https://en.wikipedia.org/wiki/CSRF}{CSRF protection}.

CSRF protection is enabled by default and applies to \sphinxstyleemphasis{UNSAFE}
HTTP methods as defined by \index{RFC!RFC 7231}\sphinxhref{https://tools.ietf.org/html/rfc7231.html}{\sphinxstylestrong{RFC 7231}} (all methods other than
\sphinxcode{\sphinxupquote{GET}}, \sphinxcode{\sphinxupquote{HEAD}}, \sphinxcode{\sphinxupquote{TRACE}} and \sphinxcode{\sphinxupquote{OPTIONS}}).

CSRF protection is implemented by checking requests using
unsafe methods for a value called \sphinxcode{\sphinxupquote{csrf\_token}} as part of
the request’s form data. That value is removed from the form
as part of the validation and does not have to be taken in
account by your own form processing.

When adding a new controller for an unsafe method (mostly POST
for e.g. forms):
\begin{itemize}
\item {} 
if the form is generated in Python, a csrf token is
available via \sphinxcode{\sphinxupquote{request.csrf\_token()
\textless{}odoo.http.WebRequest.csrf\_token()}}, the
\sphinxcode{\sphinxupquote{request}} object is available by default
in QWeb (python) templates, it may have to be added
explicitly if you are not using QWeb.

\item {} 
if the form is generated in Javascript, the CSRF token is
added by default to the QWeb (js) rendering context as
\sphinxcode{\sphinxupquote{csrf\_token}} and is otherwise available as \sphinxcode{\sphinxupquote{csrf\_token}}
on the \sphinxcode{\sphinxupquote{web.core}} module:

\fvset{hllines={, ,}}%
\begin{sphinxVerbatim}[commandchars=\\\{\}]
\PYG{n+nx}{require}\PYG{p}{(}\PYG{l+s+s1}{\PYGZsq{}web.core\PYGZsq{}}\PYG{p}{)}\PYG{p}{.}\PYG{n+nx}{csrf\PYGZus{}token}
\end{sphinxVerbatim}

\item {} 
if the endpoint can be called by external parties (not from
Odoo) as e.g. it is a REST API or a \sphinxhref{https://en.wikipedia.org/wiki/Webhook}{webhook}, CSRF protection
must be disabled on the endpoint. If possible, you may want
to implement other methods of request validation (to ensure
it is not called by an unrelated third-party).

\end{itemize}
\end{sphinxadmonition}

\end{fulllineitems}



\subsection{Request}
\label{\detokenize{reference/http:request}}\label{\detokenize{reference/http:reference-http-request}}
The request object is automatically set on \sphinxcode{\sphinxupquote{odoo.http.request}} at
the start of the request
\index{WebRequest (class in odoo.http)}

\begin{fulllineitems}
\phantomsection\label{\detokenize{reference/http:odoo.http.WebRequest}}\pysiglinewithargsret{\sphinxbfcode{\sphinxupquote{class }}\sphinxcode{\sphinxupquote{odoo.http.}}\sphinxbfcode{\sphinxupquote{WebRequest}}}{\emph{httprequest}}{}
Parent class for all Odoo Web request types, mostly deals with
initialization and setup of the request object (the dispatching itself has
to be handled by the subclasses)
\begin{quote}\begin{description}
\item[{Parameters}] \leavevmode
\sphinxstyleliteralstrong{\sphinxupquote{httprequest}} (\sphinxhref{http://werkzeug.pocoo.org/docs/wrappers/\#werkzeug.wrappers.BaseRequest}{\sphinxcode{\sphinxupquote{werkzeug.wrappers.BaseRequest}}}) \textendash{} a wrapped werkzeug Request object

\end{description}\end{quote}
\index{httprequest (odoo.http.WebRequest attribute)}

\begin{fulllineitems}
\phantomsection\label{\detokenize{reference/http:odoo.http.WebRequest.httprequest}}\pysigline{\sphinxbfcode{\sphinxupquote{httprequest}}}
the original \sphinxhref{http://werkzeug.pocoo.org/docs/wrappers/\#werkzeug.wrappers.Request}{\sphinxcode{\sphinxupquote{werkzeug.wrappers.Request}}} object provided to the
request

\end{fulllineitems}

\index{params (odoo.http.WebRequest attribute)}

\begin{fulllineitems}
\phantomsection\label{\detokenize{reference/http:odoo.http.WebRequest.params}}\pysigline{\sphinxbfcode{\sphinxupquote{params}}}
\sphinxcode{\sphinxupquote{Mapping}} of request parameters, not generally
useful as they’re provided directly to the handler method as keyword
arguments

\end{fulllineitems}

\index{cr (odoo.http.WebRequest attribute)}

\begin{fulllineitems}
\phantomsection\label{\detokenize{reference/http:odoo.http.WebRequest.cr}}\pysigline{\sphinxbfcode{\sphinxupquote{cr}}}
\sphinxcode{\sphinxupquote{Cursor}} initialized for the current method call.

Accessing the cursor when the current request uses the \sphinxcode{\sphinxupquote{none}}
authentication will raise an exception.

\end{fulllineitems}

\index{context (odoo.http.WebRequest attribute)}

\begin{fulllineitems}
\phantomsection\label{\detokenize{reference/http:odoo.http.WebRequest.context}}\pysigline{\sphinxbfcode{\sphinxupquote{context}}}
\sphinxcode{\sphinxupquote{Mapping}} of context values for the current request

\end{fulllineitems}

\index{env (odoo.http.WebRequest attribute)}

\begin{fulllineitems}
\phantomsection\label{\detokenize{reference/http:odoo.http.WebRequest.env}}\pysigline{\sphinxbfcode{\sphinxupquote{env}}}
The \sphinxcode{\sphinxupquote{Environment}} bound to current request.

\end{fulllineitems}

\index{session (odoo.http.WebRequest attribute)}

\begin{fulllineitems}
\phantomsection\label{\detokenize{reference/http:odoo.http.WebRequest.session}}\pysigline{\sphinxbfcode{\sphinxupquote{session}}}
\sphinxcode{\sphinxupquote{OpenERPSession}} holding the HTTP session data for the
current http session

\end{fulllineitems}

\index{debug (odoo.http.WebRequest attribute)}

\begin{fulllineitems}
\phantomsection\label{\detokenize{reference/http:odoo.http.WebRequest.debug}}\pysigline{\sphinxbfcode{\sphinxupquote{debug}}}
Indicates whether the current request is in “debug” mode

\end{fulllineitems}

\index{registry (odoo.http.WebRequest attribute)}

\begin{fulllineitems}
\phantomsection\label{\detokenize{reference/http:odoo.http.WebRequest.registry}}\pysigline{\sphinxbfcode{\sphinxupquote{registry}}}
The registry to the database linked to this request. Can be \sphinxcode{\sphinxupquote{None}}
if the current request uses the \sphinxcode{\sphinxupquote{none}} authentication.

\DUrole{versionmodified}{Deprecated since version 8.0: }use {\hyperref[\detokenize{reference/http:odoo.http.WebRequest.env}]{\sphinxcrossref{\sphinxcode{\sphinxupquote{env}}}}}

\end{fulllineitems}

\index{db (odoo.http.WebRequest attribute)}

\begin{fulllineitems}
\phantomsection\label{\detokenize{reference/http:odoo.http.WebRequest.db}}\pysigline{\sphinxbfcode{\sphinxupquote{db}}}
The database linked to this request. Can be \sphinxcode{\sphinxupquote{None}}
if the current request uses the \sphinxcode{\sphinxupquote{none}} authentication.

\end{fulllineitems}

\index{csrf\_token() (odoo.http.WebRequest method)}

\begin{fulllineitems}
\phantomsection\label{\detokenize{reference/http:odoo.http.WebRequest.csrf_token}}\pysiglinewithargsret{\sphinxbfcode{\sphinxupquote{csrf\_token}}}{\emph{time\_limit=3600}}{}
Generates and returns a CSRF token for the current session
\begin{quote}\begin{description}
\item[{Parameters}] \leavevmode
\sphinxstyleliteralstrong{\sphinxupquote{time\_limit}} (\sphinxstyleliteralemphasis{\sphinxupquote{int \textbar{} None}}) \textendash{} the CSRF token should only be valid for the
specified duration (in second), by default 1h,
\sphinxcode{\sphinxupquote{None}} for the token to be valid as long as the
current user’s session is.

\item[{Returns}] \leavevmode
ASCII token string

\end{description}\end{quote}

\end{fulllineitems}


\end{fulllineitems}

\index{HttpRequest (class in odoo.http)}

\begin{fulllineitems}
\phantomsection\label{\detokenize{reference/http:odoo.http.HttpRequest}}\pysiglinewithargsret{\sphinxbfcode{\sphinxupquote{class }}\sphinxcode{\sphinxupquote{odoo.http.}}\sphinxbfcode{\sphinxupquote{HttpRequest}}}{\emph{*args}}{}
Handler for the \sphinxcode{\sphinxupquote{http}} request type.

matched routing parameters, query string parameters, \sphinxhref{http://www.w3.org/TR/html401/interact/forms.html\#h-17.13.4.2}{form} parameters
and files are passed to the handler method as keyword arguments.

In case of name conflict, routing parameters have priority.

The handler method’s result can be:
\begin{itemize}
\item {} 
a falsy value, in which case the HTTP response will be an
\sphinxhref{http://tools.ietf.org/html/rfc7231\#section-6.3.5}{HTTP 204} (No Content)

\item {} 
a werkzeug Response object, which is returned as-is

\item {} 
a \sphinxcode{\sphinxupquote{str}} or \sphinxcode{\sphinxupquote{unicode}}, will be wrapped in a Response object and
interpreted as HTML

\end{itemize}
\index{make\_response() (odoo.http.HttpRequest method)}

\begin{fulllineitems}
\phantomsection\label{\detokenize{reference/http:odoo.http.HttpRequest.make_response}}\pysiglinewithargsret{\sphinxbfcode{\sphinxupquote{make\_response}}}{\emph{data}, \emph{headers=None}, \emph{cookies=None}}{}
Helper for non-HTML responses, or HTML responses with custom
response headers or cookies.

While handlers can just return the HTML markup of a page they want to
send as a string if non-HTML data is returned they need to create a
complete response object, or the returned data will not be correctly
interpreted by the clients.
\begin{quote}\begin{description}
\item[{Parameters}] \leavevmode\begin{itemize}
\item {} 
\sphinxstyleliteralstrong{\sphinxupquote{data}} (\sphinxstyleliteralemphasis{\sphinxupquote{basestring}}) \textendash{} response body

\item {} 
\sphinxstyleliteralstrong{\sphinxupquote{headers}} (\sphinxcode{\sphinxupquote{{[}(name, value){]}}}) \textendash{} HTTP headers to set on the response

\item {} 
\sphinxstyleliteralstrong{\sphinxupquote{cookies}} (\sphinxstyleliteralemphasis{\sphinxupquote{collections.Mapping}}) \textendash{} cookies to set on the client

\end{itemize}

\end{description}\end{quote}

\end{fulllineitems}

\index{not\_found() (odoo.http.HttpRequest method)}

\begin{fulllineitems}
\phantomsection\label{\detokenize{reference/http:odoo.http.HttpRequest.not_found}}\pysiglinewithargsret{\sphinxbfcode{\sphinxupquote{not\_found}}}{\emph{description=None}}{}
Shortcut for a \sphinxhref{http://tools.ietf.org/html/rfc7231\#section-6.5.4}{HTTP 404} (Not Found)
response

\end{fulllineitems}

\index{render() (odoo.http.HttpRequest method)}

\begin{fulllineitems}
\phantomsection\label{\detokenize{reference/http:odoo.http.HttpRequest.render}}\pysiglinewithargsret{\sphinxbfcode{\sphinxupquote{render}}}{\emph{template}, \emph{qcontext=None}, \emph{lazy=True}, \emph{**kw}}{}
Lazy render of a QWeb template.

The actual rendering of the given template will occur at then end of
the dispatching. Meanwhile, the template and/or qcontext can be
altered or even replaced by a static response.
\begin{quote}\begin{description}
\item[{Parameters}] \leavevmode\begin{itemize}
\item {} 
\sphinxstyleliteralstrong{\sphinxupquote{template}} (\sphinxstyleliteralemphasis{\sphinxupquote{basestring}}) \textendash{} template to render

\item {} 
\sphinxstyleliteralstrong{\sphinxupquote{qcontext}} (\sphinxhref{https://docs.python.org/3/library/stdtypes.html\#dict}{\sphinxstyleliteralemphasis{\sphinxupquote{dict}}}) \textendash{} Rendering context to use

\item {} 
\sphinxstyleliteralstrong{\sphinxupquote{lazy}} (\sphinxhref{https://docs.python.org/3/library/functions.html\#bool}{\sphinxstyleliteralemphasis{\sphinxupquote{bool}}}) \textendash{} whether the template rendering should be deferred
until the last possible moment

\item {} 
\sphinxstyleliteralstrong{\sphinxupquote{kw}} \textendash{} forwarded to werkzeug’s Response object

\end{itemize}

\end{description}\end{quote}

\end{fulllineitems}


\end{fulllineitems}

\index{JsonRequest (class in odoo.http)}

\begin{fulllineitems}
\phantomsection\label{\detokenize{reference/http:odoo.http.JsonRequest}}\pysiglinewithargsret{\sphinxbfcode{\sphinxupquote{class }}\sphinxcode{\sphinxupquote{odoo.http.}}\sphinxbfcode{\sphinxupquote{JsonRequest}}}{\emph{*args}}{}
Request handler for \sphinxhref{http://www.jsonrpc.org/specification}{JSON-RPC 2} over HTTP
\begin{itemize}
\item {} 
\sphinxcode{\sphinxupquote{method}} is ignored

\item {} 
\sphinxcode{\sphinxupquote{params}} must be a JSON object (not an array) and is passed as keyword
arguments to the handler method

\item {} 
the handler method’s result is returned as JSON-RPC \sphinxcode{\sphinxupquote{result}} and
wrapped in the \sphinxhref{http://www.jsonrpc.org/specification\#response\_object}{JSON-RPC Response}

\end{itemize}

Sucessful request:

\fvset{hllines={, ,}}%
\begin{sphinxVerbatim}[commandchars=\\\{\}]
\PYG{o}{\PYGZhy{}}\PYG{o}{\PYGZhy{}}\PYG{o}{\PYGZgt{}} \PYG{p}{\PYGZob{}}\PYG{l+s+s2}{\PYGZdq{}}\PYG{l+s+s2}{jsonrpc}\PYG{l+s+s2}{\PYGZdq{}}\PYG{p}{:} \PYG{l+s+s2}{\PYGZdq{}}\PYG{l+s+s2}{2.0}\PYG{l+s+s2}{\PYGZdq{}}\PYG{p}{,}
     \PYG{l+s+s2}{\PYGZdq{}}\PYG{l+s+s2}{method}\PYG{l+s+s2}{\PYGZdq{}}\PYG{p}{:} \PYG{l+s+s2}{\PYGZdq{}}\PYG{l+s+s2}{call}\PYG{l+s+s2}{\PYGZdq{}}\PYG{p}{,}
     \PYG{l+s+s2}{\PYGZdq{}}\PYG{l+s+s2}{params}\PYG{l+s+s2}{\PYGZdq{}}\PYG{p}{:} \PYG{p}{\PYGZob{}}\PYG{l+s+s2}{\PYGZdq{}}\PYG{l+s+s2}{context}\PYG{l+s+s2}{\PYGZdq{}}\PYG{p}{:} \PYG{p}{\PYGZob{}}\PYG{p}{\PYGZcb{}}\PYG{p}{,}
                \PYG{l+s+s2}{\PYGZdq{}}\PYG{l+s+s2}{arg1}\PYG{l+s+s2}{\PYGZdq{}}\PYG{p}{:} \PYG{l+s+s2}{\PYGZdq{}}\PYG{l+s+s2}{val1}\PYG{l+s+s2}{\PYGZdq{}} \PYG{p}{\PYGZcb{}}\PYG{p}{,}
     \PYG{l+s+s2}{\PYGZdq{}}\PYG{l+s+s2}{id}\PYG{l+s+s2}{\PYGZdq{}}\PYG{p}{:} \PYG{n}{null}\PYG{p}{\PYGZcb{}}

\PYG{o}{\PYGZlt{}}\PYG{o}{\PYGZhy{}}\PYG{o}{\PYGZhy{}} \PYG{p}{\PYGZob{}}\PYG{l+s+s2}{\PYGZdq{}}\PYG{l+s+s2}{jsonrpc}\PYG{l+s+s2}{\PYGZdq{}}\PYG{p}{:} \PYG{l+s+s2}{\PYGZdq{}}\PYG{l+s+s2}{2.0}\PYG{l+s+s2}{\PYGZdq{}}\PYG{p}{,}
     \PYG{l+s+s2}{\PYGZdq{}}\PYG{l+s+s2}{result}\PYG{l+s+s2}{\PYGZdq{}}\PYG{p}{:} \PYG{p}{\PYGZob{}} \PYG{l+s+s2}{\PYGZdq{}}\PYG{l+s+s2}{res1}\PYG{l+s+s2}{\PYGZdq{}}\PYG{p}{:} \PYG{l+s+s2}{\PYGZdq{}}\PYG{l+s+s2}{val1}\PYG{l+s+s2}{\PYGZdq{}} \PYG{p}{\PYGZcb{}}\PYG{p}{,}
     \PYG{l+s+s2}{\PYGZdq{}}\PYG{l+s+s2}{id}\PYG{l+s+s2}{\PYGZdq{}}\PYG{p}{:} \PYG{n}{null}\PYG{p}{\PYGZcb{}}
\end{sphinxVerbatim}

Request producing a error:

\fvset{hllines={, ,}}%
\begin{sphinxVerbatim}[commandchars=\\\{\}]
\PYG{o}{\PYGZhy{}}\PYG{o}{\PYGZhy{}}\PYG{o}{\PYGZgt{}} \PYG{p}{\PYGZob{}}\PYG{l+s+s2}{\PYGZdq{}}\PYG{l+s+s2}{jsonrpc}\PYG{l+s+s2}{\PYGZdq{}}\PYG{p}{:} \PYG{l+s+s2}{\PYGZdq{}}\PYG{l+s+s2}{2.0}\PYG{l+s+s2}{\PYGZdq{}}\PYG{p}{,}
     \PYG{l+s+s2}{\PYGZdq{}}\PYG{l+s+s2}{method}\PYG{l+s+s2}{\PYGZdq{}}\PYG{p}{:} \PYG{l+s+s2}{\PYGZdq{}}\PYG{l+s+s2}{call}\PYG{l+s+s2}{\PYGZdq{}}\PYG{p}{,}
     \PYG{l+s+s2}{\PYGZdq{}}\PYG{l+s+s2}{params}\PYG{l+s+s2}{\PYGZdq{}}\PYG{p}{:} \PYG{p}{\PYGZob{}}\PYG{l+s+s2}{\PYGZdq{}}\PYG{l+s+s2}{context}\PYG{l+s+s2}{\PYGZdq{}}\PYG{p}{:} \PYG{p}{\PYGZob{}}\PYG{p}{\PYGZcb{}}\PYG{p}{,}
                \PYG{l+s+s2}{\PYGZdq{}}\PYG{l+s+s2}{arg1}\PYG{l+s+s2}{\PYGZdq{}}\PYG{p}{:} \PYG{l+s+s2}{\PYGZdq{}}\PYG{l+s+s2}{val1}\PYG{l+s+s2}{\PYGZdq{}} \PYG{p}{\PYGZcb{}}\PYG{p}{,}
     \PYG{l+s+s2}{\PYGZdq{}}\PYG{l+s+s2}{id}\PYG{l+s+s2}{\PYGZdq{}}\PYG{p}{:} \PYG{n}{null}\PYG{p}{\PYGZcb{}}

\PYG{o}{\PYGZlt{}}\PYG{o}{\PYGZhy{}}\PYG{o}{\PYGZhy{}} \PYG{p}{\PYGZob{}}\PYG{l+s+s2}{\PYGZdq{}}\PYG{l+s+s2}{jsonrpc}\PYG{l+s+s2}{\PYGZdq{}}\PYG{p}{:} \PYG{l+s+s2}{\PYGZdq{}}\PYG{l+s+s2}{2.0}\PYG{l+s+s2}{\PYGZdq{}}\PYG{p}{,}
     \PYG{l+s+s2}{\PYGZdq{}}\PYG{l+s+s2}{error}\PYG{l+s+s2}{\PYGZdq{}}\PYG{p}{:} \PYG{p}{\PYGZob{}}\PYG{l+s+s2}{\PYGZdq{}}\PYG{l+s+s2}{code}\PYG{l+s+s2}{\PYGZdq{}}\PYG{p}{:} \PYG{l+m+mi}{1}\PYG{p}{,}
               \PYG{l+s+s2}{\PYGZdq{}}\PYG{l+s+s2}{message}\PYG{l+s+s2}{\PYGZdq{}}\PYG{p}{:} \PYG{l+s+s2}{\PYGZdq{}}\PYG{l+s+s2}{End user error message.}\PYG{l+s+s2}{\PYGZdq{}}\PYG{p}{,}
               \PYG{l+s+s2}{\PYGZdq{}}\PYG{l+s+s2}{data}\PYG{l+s+s2}{\PYGZdq{}}\PYG{p}{:} \PYG{p}{\PYGZob{}}\PYG{l+s+s2}{\PYGZdq{}}\PYG{l+s+s2}{code}\PYG{l+s+s2}{\PYGZdq{}}\PYG{p}{:} \PYG{l+s+s2}{\PYGZdq{}}\PYG{l+s+s2}{codestring}\PYG{l+s+s2}{\PYGZdq{}}\PYG{p}{,}
                        \PYG{l+s+s2}{\PYGZdq{}}\PYG{l+s+s2}{debug}\PYG{l+s+s2}{\PYGZdq{}}\PYG{p}{:} \PYG{l+s+s2}{\PYGZdq{}}\PYG{l+s+s2}{traceback}\PYG{l+s+s2}{\PYGZdq{}} \PYG{p}{\PYGZcb{}} \PYG{p}{\PYGZcb{}}\PYG{p}{,}
     \PYG{l+s+s2}{\PYGZdq{}}\PYG{l+s+s2}{id}\PYG{l+s+s2}{\PYGZdq{}}\PYG{p}{:} \PYG{n}{null}\PYG{p}{\PYGZcb{}}
\end{sphinxVerbatim}

\end{fulllineitems}



\subsection{Response}
\label{\detokenize{reference/http:response}}\index{Response (class in odoo.http)}

\begin{fulllineitems}
\phantomsection\label{\detokenize{reference/http:odoo.http.Response}}\pysiglinewithargsret{\sphinxbfcode{\sphinxupquote{class }}\sphinxcode{\sphinxupquote{odoo.http.}}\sphinxbfcode{\sphinxupquote{Response}}}{\emph{*args}, \emph{**kw}}{}
Response object passed through controller route chain.

In addition to the \sphinxhref{http://werkzeug.pocoo.org/docs/wrappers/\#werkzeug.wrappers.Response}{\sphinxcode{\sphinxupquote{werkzeug.wrappers.Response}}} parameters, this
class’s constructor can take the following additional parameters
for QWeb Lazy Rendering.
\begin{quote}\begin{description}
\item[{Parameters}] \leavevmode\begin{itemize}
\item {} 
\sphinxstyleliteralstrong{\sphinxupquote{template}} (\sphinxstyleliteralemphasis{\sphinxupquote{basestring}}) \textendash{} template to render

\item {} 
\sphinxstyleliteralstrong{\sphinxupquote{qcontext}} (\sphinxhref{https://docs.python.org/3/library/stdtypes.html\#dict}{\sphinxstyleliteralemphasis{\sphinxupquote{dict}}}) \textendash{} Rendering context to use

\item {} 
\sphinxstyleliteralstrong{\sphinxupquote{uid}} (\sphinxhref{https://docs.python.org/3/library/functions.html\#int}{\sphinxstyleliteralemphasis{\sphinxupquote{int}}}) \textendash{} User id to use for the ir.ui.view render call,
\sphinxcode{\sphinxupquote{None}} to use the request’s user (the default)

\end{itemize}

\end{description}\end{quote}

these attributes are available as parameters on the Response object and
can be altered at any time before rendering

Also exposes all the attributes and methods of
\sphinxhref{http://werkzeug.pocoo.org/docs/wrappers/\#werkzeug.wrappers.Response}{\sphinxcode{\sphinxupquote{werkzeug.wrappers.Response}}}.
\index{render() (odoo.http.Response method)}

\begin{fulllineitems}
\phantomsection\label{\detokenize{reference/http:odoo.http.Response.render}}\pysiglinewithargsret{\sphinxbfcode{\sphinxupquote{render}}}{}{}
Renders the Response’s template, returns the result

\end{fulllineitems}

\index{flatten() (odoo.http.Response method)}

\begin{fulllineitems}
\phantomsection\label{\detokenize{reference/http:odoo.http.Response.flatten}}\pysiglinewithargsret{\sphinxbfcode{\sphinxupquote{flatten}}}{}{}
Forces the rendering of the response’s template, sets the result
as response body and unsets \sphinxcode{\sphinxupquote{template}}

\end{fulllineitems}


\end{fulllineitems}



\subsection{Controllers}
\label{\detokenize{reference/http:controllers}}\label{\detokenize{reference/http:reference-http-controllers}}
Controllers need to provide extensibility, much like
{\hyperref[\detokenize{reference/orm:odoo.models.Model}]{\sphinxcrossref{\sphinxcode{\sphinxupquote{Model}}}}}, but can’t use the same mechanism as the
pre-requisites (a database with loaded modules) may not be available yet (e.g.
no database created, or no database selected).

Controllers thus provide their own extension mechanism, separate from that of
models:

Controllers are created by \sphinxhref{https://docs.python.org/3/tutorial/classes.html\#tut-inheritance}{\DUrole{xref,std,std-ref}{inheriting}} from
\index{Controller (class in odoo.http)}

\begin{fulllineitems}
\phantomsection\label{\detokenize{reference/http:odoo.http.Controller}}\pysigline{\sphinxbfcode{\sphinxupquote{class }}\sphinxcode{\sphinxupquote{odoo.http.}}\sphinxbfcode{\sphinxupquote{Controller}}}
\end{fulllineitems}


and defining methods decorated with {\hyperref[\detokenize{reference/http:odoo.http.route}]{\sphinxcrossref{\sphinxcode{\sphinxupquote{route()}}}}}:

\fvset{hllines={, ,}}%
\begin{sphinxVerbatim}[commandchars=\\\{\}]
\PYG{k}{class} \PYG{n+nc}{MyController}\PYG{p}{(}\PYG{n}{odoo}\PYG{o}{.}\PYG{n}{http}\PYG{o}{.}\PYG{n}{Controller}\PYG{p}{)}\PYG{p}{:}
    \PYG{n+nd}{@route}\PYG{p}{(}\PYG{l+s+s1}{\PYGZsq{}}\PYG{l+s+s1}{/some\PYGZus{}url}\PYG{l+s+s1}{\PYGZsq{}}\PYG{p}{,} \PYG{n}{auth}\PYG{o}{=}\PYG{l+s+s1}{\PYGZsq{}}\PYG{l+s+s1}{public}\PYG{l+s+s1}{\PYGZsq{}}\PYG{p}{)}
    \PYG{k}{def} \PYG{n+nf}{handler}\PYG{p}{(}\PYG{n+nb+bp}{self}\PYG{p}{)}\PYG{p}{:}
        \PYG{k}{return} \PYG{n}{stuff}\PYG{p}{(}\PYG{p}{)}
\end{sphinxVerbatim}

To \sphinxstyleemphasis{override} a controller, \sphinxhref{https://docs.python.org/3/tutorial/classes.html\#tut-inheritance}{\DUrole{xref,std,std-ref}{inherit}} from its
class and override relevant methods, re-exposing them if necessary:

\fvset{hllines={, ,}}%
\begin{sphinxVerbatim}[commandchars=\\\{\}]
\PYG{k}{class} \PYG{n+nc}{Extension}\PYG{p}{(}\PYG{n}{MyController}\PYG{p}{)}\PYG{p}{:}
    \PYG{n+nd}{@route}\PYG{p}{(}\PYG{p}{)}
    \PYG{k}{def} \PYG{n+nf}{handler}\PYG{p}{(}\PYG{n+nb+bp}{self}\PYG{p}{)}\PYG{p}{:}
        \PYG{n}{do\PYGZus{}before}\PYG{p}{(}\PYG{p}{)}
        \PYG{k}{return} \PYG{n+nb}{super}\PYG{p}{(}\PYG{n}{Extension}\PYG{p}{,} \PYG{n+nb+bp}{self}\PYG{p}{)}\PYG{o}{.}\PYG{n}{handler}\PYG{p}{(}\PYG{p}{)}
\end{sphinxVerbatim}
\begin{itemize}
\item {} 
decorating with {\hyperref[\detokenize{reference/http:odoo.http.route}]{\sphinxcrossref{\sphinxcode{\sphinxupquote{route()}}}}} is necessary to keep the method
(and route) visible: if the method is redefined without decorating, it
will be “unpublished”

\item {} 
the decorators of all methods are combined, if the overriding method’s
decorator has no argument all previous ones will be kept, any provided
argument will override previously defined ones e.g.:

\fvset{hllines={, ,}}%
\begin{sphinxVerbatim}[commandchars=\\\{\}]
\PYG{k}{class} \PYG{n+nc}{Restrict}\PYG{p}{(}\PYG{n}{MyController}\PYG{p}{)}\PYG{p}{:}
    \PYG{n+nd}{@route}\PYG{p}{(}\PYG{n}{auth}\PYG{o}{=}\PYG{l+s+s1}{\PYGZsq{}}\PYG{l+s+s1}{user}\PYG{l+s+s1}{\PYGZsq{}}\PYG{p}{)}
    \PYG{k}{def} \PYG{n+nf}{handler}\PYG{p}{(}\PYG{n+nb+bp}{self}\PYG{p}{)}\PYG{p}{:}
        \PYG{k}{return} \PYG{n+nb}{super}\PYG{p}{(}\PYG{n}{Restrict}\PYG{p}{,} \PYG{n+nb+bp}{self}\PYG{p}{)}\PYG{o}{.}\PYG{n}{handler}\PYG{p}{(}\PYG{p}{)}
\end{sphinxVerbatim}

will change \sphinxcode{\sphinxupquote{/some\_url}} from public authentication to user (requiring a
log-in)

\end{itemize}


\section{QWeb}
\label{\detokenize{reference/qweb::doc}}\label{\detokenize{reference/qweb:qweb}}\label{\detokenize{reference/qweb:reference-qweb}}
QWeb is the primary \sphinxhref{http://en.wikipedia.org/wiki/Template\_processor}{templating} engine used by Odoo%
\begin{footnote}[2]\sphinxAtStartFootnote
although it uses a few others, either for historical
reasons or because they remain better fits for the
use case. Odoo 9.0 still depends on \sphinxhref{http://jinja.pocoo.org}{Jinja} and \sphinxhref{http://www.makotemplates.org}{Mako}.
%
\end{footnote}. It
is an XML templating engine%
\begin{footnote}[1]\sphinxAtStartFootnote
it is similar in that to \sphinxhref{http://genshi.edgewall.org}{Genshi}, although it does not use (and
has no support for) \sphinxhref{http://en.wikipedia.org/wiki/XML\_namespace}{XML namespaces}
%
\end{footnote} and used mostly to generate \sphinxhref{http://en.wikipedia.org/wiki/HTML}{HTML}
fragments and pages.

Template directives are specified as XML attributes prefixed with \sphinxcode{\sphinxupquote{t-}},
for instance \sphinxcode{\sphinxupquote{t-if}} for {\hyperref[\detokenize{reference/qweb:reference-qweb-conditionals}]{\sphinxcrossref{\DUrole{std,std-ref}{conditionals}}}}, with elements
and other attributes being rendered directly.

To avoid element rendering, a placeholder element \sphinxcode{\sphinxupquote{\textless{}t\textgreater{}}} is also available,
which executes its directive but doesn’t generate any output in and of
itself:

\fvset{hllines={, ,}}%
\begin{sphinxVerbatim}[commandchars=\\\{\}]
\PYG{n+nt}{\PYGZlt{}t} \PYG{n+na}{t\PYGZhy{}if=}\PYG{l+s}{\PYGZdq{}condition\PYGZdq{}}\PYG{n+nt}{\PYGZgt{}}
    \PYG{n+nt}{\PYGZlt{}p}\PYG{n+nt}{\PYGZgt{}}Test\PYG{n+nt}{\PYGZlt{}/p\PYGZgt{}}
\PYG{n+nt}{\PYGZlt{}/t\PYGZgt{}}
\end{sphinxVerbatim}

will result in:

\fvset{hllines={, ,}}%
\begin{sphinxVerbatim}[commandchars=\\\{\}]
\PYG{n+nt}{\PYGZlt{}p}\PYG{n+nt}{\PYGZgt{}}Test\PYG{n+nt}{\PYGZlt{}/p\PYGZgt{}}
\end{sphinxVerbatim}

if \sphinxcode{\sphinxupquote{condition}} is true, but:

\fvset{hllines={, ,}}%
\begin{sphinxVerbatim}[commandchars=\\\{\}]
\PYG{n+nt}{\PYGZlt{}div} \PYG{n+na}{t\PYGZhy{}if=}\PYG{l+s}{\PYGZdq{}condition\PYGZdq{}}\PYG{n+nt}{\PYGZgt{}}
    \PYG{n+nt}{\PYGZlt{}p}\PYG{n+nt}{\PYGZgt{}}Test\PYG{n+nt}{\PYGZlt{}/p\PYGZgt{}}
\PYG{n+nt}{\PYGZlt{}/div\PYGZgt{}}
\end{sphinxVerbatim}

will result in:

\fvset{hllines={, ,}}%
\begin{sphinxVerbatim}[commandchars=\\\{\}]
\PYG{n+nt}{\PYGZlt{}div}\PYG{n+nt}{\PYGZgt{}}
    \PYG{n+nt}{\PYGZlt{}p}\PYG{n+nt}{\PYGZgt{}}Test\PYG{n+nt}{\PYGZlt{}/p\PYGZgt{}}
\PYG{n+nt}{\PYGZlt{}/div\PYGZgt{}}
\end{sphinxVerbatim}


\subsection{data output}
\label{\detokenize{reference/qweb:reference-qweb-output}}\label{\detokenize{reference/qweb:data-output}}
QWeb has a primary output directive which automatically HTML-escape its
content limiting \sphinxhref{http://en.wikipedia.org/wiki/Cross-site\_scripting}{XSS} risks when displaying user-provided content: \sphinxcode{\sphinxupquote{esc}}.

\sphinxcode{\sphinxupquote{esc}} takes an expression, evaluates it and prints the content:

\fvset{hllines={, ,}}%
\begin{sphinxVerbatim}[commandchars=\\\{\}]
\PYG{n+nt}{\PYGZlt{}p}\PYG{n+nt}{\PYGZgt{}}\PYG{n+nt}{\PYGZlt{}t} \PYG{n+na}{t\PYGZhy{}esc=}\PYG{l+s}{\PYGZdq{}value\PYGZdq{}}\PYG{n+nt}{/\PYGZgt{}}\PYG{n+nt}{\PYGZlt{}/p\PYGZgt{}}
\end{sphinxVerbatim}

rendered with the value \sphinxcode{\sphinxupquote{value}} set to \sphinxcode{\sphinxupquote{42}} yields:

\fvset{hllines={, ,}}%
\begin{sphinxVerbatim}[commandchars=\\\{\}]
\PYG{n+nt}{\PYGZlt{}p}\PYG{n+nt}{\PYGZgt{}}42\PYG{n+nt}{\PYGZlt{}/p\PYGZgt{}}
\end{sphinxVerbatim}

There is one other output directive \sphinxcode{\sphinxupquote{raw}} which behaves the same as
respectively \sphinxcode{\sphinxupquote{esc}} but \sphinxstyleemphasis{does not HTML-escape its output}. It can be useful
to display separately constructed markup (e.g. from functions) or already
sanitized user-provided markup.


\subsection{conditionals}
\label{\detokenize{reference/qweb:reference-qweb-conditionals}}\label{\detokenize{reference/qweb:conditionals}}
QWeb has a conditional directive \sphinxcode{\sphinxupquote{if}}, which evaluates an expression given
as attribute value:

\fvset{hllines={, ,}}%
\begin{sphinxVerbatim}[commandchars=\\\{\}]
\PYG{n+nt}{\PYGZlt{}div}\PYG{n+nt}{\PYGZgt{}}
    \PYG{n+nt}{\PYGZlt{}t} \PYG{n+na}{t\PYGZhy{}if=}\PYG{l+s}{\PYGZdq{}condition\PYGZdq{}}\PYG{n+nt}{\PYGZgt{}}
        \PYG{n+nt}{\PYGZlt{}p}\PYG{n+nt}{\PYGZgt{}}ok\PYG{n+nt}{\PYGZlt{}/p\PYGZgt{}}
    \PYG{n+nt}{\PYGZlt{}/t\PYGZgt{}}
\PYG{n+nt}{\PYGZlt{}/div\PYGZgt{}}
\end{sphinxVerbatim}

The element is rendered if the condition is true:

\fvset{hllines={, ,}}%
\begin{sphinxVerbatim}[commandchars=\\\{\}]
\PYG{n+nt}{\PYGZlt{}div}\PYG{n+nt}{\PYGZgt{}}
    \PYG{n+nt}{\PYGZlt{}p}\PYG{n+nt}{\PYGZgt{}}ok\PYG{n+nt}{\PYGZlt{}/p\PYGZgt{}}
\PYG{n+nt}{\PYGZlt{}/div\PYGZgt{}}
\end{sphinxVerbatim}

but if the condition is false it is removed from the result:

\fvset{hllines={, ,}}%
\begin{sphinxVerbatim}[commandchars=\\\{\}]
\PYG{n+nt}{\PYGZlt{}div}\PYG{n+nt}{\PYGZgt{}}
\PYG{n+nt}{\PYGZlt{}/div\PYGZgt{}}
\end{sphinxVerbatim}

The conditional rendering applies to the bearer of the directive, which does
not have to be \sphinxcode{\sphinxupquote{\textless{}t\textgreater{}}}:

\fvset{hllines={, ,}}%
\begin{sphinxVerbatim}[commandchars=\\\{\}]
\PYG{n+nt}{\PYGZlt{}div}\PYG{n+nt}{\PYGZgt{}}
    \PYG{n+nt}{\PYGZlt{}p} \PYG{n+na}{t\PYGZhy{}if=}\PYG{l+s}{\PYGZdq{}condition\PYGZdq{}}\PYG{n+nt}{\PYGZgt{}}ok\PYG{n+nt}{\PYGZlt{}/p\PYGZgt{}}
\PYG{n+nt}{\PYGZlt{}/div\PYGZgt{}}
\end{sphinxVerbatim}

will give the same results as the previous example.

Extra conditional branching directives \sphinxcode{\sphinxupquote{t-elif}} and \sphinxcode{\sphinxupquote{t-else}} are also
available:

\fvset{hllines={, ,}}%
\begin{sphinxVerbatim}[commandchars=\\\{\}]
\PYG{n+nt}{\PYGZlt{}div}\PYG{n+nt}{\PYGZgt{}}
    \PYG{n+nt}{\PYGZlt{}p} \PYG{n+na}{t\PYGZhy{}if=}\PYG{l+s}{\PYGZdq{}user.birthday == today()\PYGZdq{}}\PYG{n+nt}{\PYGZgt{}}Happy bithday!\PYG{n+nt}{\PYGZlt{}/p\PYGZgt{}}
    \PYG{n+nt}{\PYGZlt{}p} \PYG{n+na}{t\PYGZhy{}elif=}\PYG{l+s}{\PYGZdq{}user.login == \PYGZsq{}root\PYGZsq{}\PYGZdq{}}\PYG{n+nt}{\PYGZgt{}}Welcome master!\PYG{n+nt}{\PYGZlt{}/p\PYGZgt{}}
    \PYG{n+nt}{\PYGZlt{}p} \PYG{n+na}{t\PYGZhy{}else=}\PYG{l+s}{\PYGZdq{}\PYGZdq{}}\PYG{n+nt}{\PYGZgt{}}Welcome!\PYG{n+nt}{\PYGZlt{}/p\PYGZgt{}}
\PYG{n+nt}{\PYGZlt{}/div\PYGZgt{}}
\end{sphinxVerbatim}


\subsection{loops}
\label{\detokenize{reference/qweb:loops}}\label{\detokenize{reference/qweb:reference-qweb-loops}}
QWeb has an iteration directive \sphinxcode{\sphinxupquote{foreach}} which take an expression returning
the collection to iterate on, and a second parameter \sphinxcode{\sphinxupquote{t-as}} providing the
name to use for the “current item” of the iteration:

\fvset{hllines={, ,}}%
\begin{sphinxVerbatim}[commandchars=\\\{\}]
\PYG{n+nt}{\PYGZlt{}t} \PYG{n+na}{t\PYGZhy{}foreach=}\PYG{l+s}{\PYGZdq{}[1, 2, 3]\PYGZdq{}} \PYG{n+na}{t\PYGZhy{}as=}\PYG{l+s}{\PYGZdq{}i\PYGZdq{}}\PYG{n+nt}{\PYGZgt{}}
    \PYG{n+nt}{\PYGZlt{}p}\PYG{n+nt}{\PYGZgt{}}\PYG{n+nt}{\PYGZlt{}t} \PYG{n+na}{t\PYGZhy{}esc=}\PYG{l+s}{\PYGZdq{}i\PYGZdq{}}\PYG{n+nt}{/\PYGZgt{}}\PYG{n+nt}{\PYGZlt{}/p\PYGZgt{}}
\PYG{n+nt}{\PYGZlt{}/t\PYGZgt{}}
\end{sphinxVerbatim}

will be rendered as:

\fvset{hllines={, ,}}%
\begin{sphinxVerbatim}[commandchars=\\\{\}]
\PYG{n+nt}{\PYGZlt{}p}\PYG{n+nt}{\PYGZgt{}}1\PYG{n+nt}{\PYGZlt{}/p\PYGZgt{}}
\PYG{n+nt}{\PYGZlt{}p}\PYG{n+nt}{\PYGZgt{}}2\PYG{n+nt}{\PYGZlt{}/p\PYGZgt{}}
\PYG{n+nt}{\PYGZlt{}p}\PYG{n+nt}{\PYGZgt{}}3\PYG{n+nt}{\PYGZlt{}/p\PYGZgt{}}
\end{sphinxVerbatim}

Like conditions, \sphinxcode{\sphinxupquote{foreach}} applies to the element bearing the directive’s
attribute, and

\fvset{hllines={, ,}}%
\begin{sphinxVerbatim}[commandchars=\\\{\}]
\PYG{n+nt}{\PYGZlt{}p} \PYG{n+na}{t\PYGZhy{}foreach=}\PYG{l+s}{\PYGZdq{}[1, 2, 3]\PYGZdq{}} \PYG{n+na}{t\PYGZhy{}as=}\PYG{l+s}{\PYGZdq{}i\PYGZdq{}}\PYG{n+nt}{\PYGZgt{}}
    \PYG{n+nt}{\PYGZlt{}t} \PYG{n+na}{t\PYGZhy{}esc=}\PYG{l+s}{\PYGZdq{}i\PYGZdq{}}\PYG{n+nt}{/\PYGZgt{}}
\PYG{n+nt}{\PYGZlt{}/p\PYGZgt{}}
\end{sphinxVerbatim}

is equivalent to the previous example.

\sphinxcode{\sphinxupquote{foreach}} can iterate on an array (the current item will be the current
value), a mapping (the current item will be the current key) or an integer
(equivalent to iterating on an array between 0 inclusive and the provided
integer exclusive).

In addition to the name passed via \sphinxcode{\sphinxupquote{t-as}}, \sphinxcode{\sphinxupquote{foreach}} provides a few other
variables for various data points:

\begin{sphinxadmonition}{warning}{Warning:}
\sphinxcode{\sphinxupquote{\$as}} will be replaced by the name passed to \sphinxcode{\sphinxupquote{t-as}}
\end{sphinxadmonition}
\begin{description}
\item[{\sphinxcode{\sphinxupquote{\sphinxstyleemphasis{\$as}\_all}}}] \leavevmode
the object being iterated over

\item[{\sphinxcode{\sphinxupquote{\sphinxstyleemphasis{\$as}\_value}}}] \leavevmode
the current iteration value, identical to \sphinxcode{\sphinxupquote{\$as}} for lists and integers,
but for mappings it provides the value (where \sphinxcode{\sphinxupquote{\$as}} provides the key)

\item[{\sphinxcode{\sphinxupquote{\sphinxstyleemphasis{\$as}\_index}}}] \leavevmode
the current iteration index (the first item of the iteration has index 0)

\item[{\sphinxcode{\sphinxupquote{\sphinxstyleemphasis{\$as}\_size}}}] \leavevmode
the size of the collection if it is available

\item[{\sphinxcode{\sphinxupquote{\sphinxstyleemphasis{\$as}\_first}}}] \leavevmode
whether the current item is the first of the iteration (equivalent to
\sphinxcode{\sphinxupquote{\sphinxstyleemphasis{\$as}\_index == 0}})

\item[{\sphinxcode{\sphinxupquote{\sphinxstyleemphasis{\$as}\_last}}}] \leavevmode
whether the current item is the last of the iteration (equivalent to
\sphinxcode{\sphinxupquote{\sphinxstyleemphasis{\$as}\_index + 1 == \sphinxstyleemphasis{\$as}\_size}}), requires the iteratee’s size be
available

\item[{\sphinxcode{\sphinxupquote{\sphinxstyleemphasis{\$as}\_parity}}}] \leavevmode
either \sphinxcode{\sphinxupquote{"even"}} or \sphinxcode{\sphinxupquote{"odd"}}, the parity of the current iteration round

\item[{\sphinxcode{\sphinxupquote{\sphinxstyleemphasis{\$as}\_even}}}] \leavevmode
a boolean flag indicating that the current iteration round is on an even
index

\item[{\sphinxcode{\sphinxupquote{\sphinxstyleemphasis{\$as}\_odd}}}] \leavevmode
a boolean flag indicating that the current iteration round is on an odd
index

\end{description}

These extra variables provided and all new variables created into the
\sphinxcode{\sphinxupquote{foreach}} are only available in the scope of the{}`{}`foreach{}`{}`. If the
variable exists outside the context of the \sphinxcode{\sphinxupquote{foreach}}, the value is copied
at the end of the foreach into the global context.

\fvset{hllines={, ,}}%
\begin{sphinxVerbatim}[commandchars=\\\{\}]
\PYG{n+nt}{\PYGZlt{}t} \PYG{n+na}{t\PYGZhy{}set=}\PYG{l+s}{\PYGZdq{}existing\PYGZus{}variable\PYGZdq{}} \PYG{n+na}{t\PYGZhy{}value=}\PYG{l+s}{\PYGZdq{}False\PYGZdq{}}\PYG{n+nt}{/\PYGZgt{}}
\PYG{c}{\PYGZlt{}!\PYGZhy{}\PYGZhy{}}\PYG{c}{ existing\PYGZus{}variable now False }\PYG{c}{\PYGZhy{}\PYGZhy{}\PYGZgt{}}

\PYG{n+nt}{\PYGZlt{}p} \PYG{n+na}{t\PYGZhy{}foreach=}\PYG{l+s}{\PYGZdq{}[1, 2, 3]\PYGZdq{}} \PYG{n+na}{t\PYGZhy{}as=}\PYG{l+s}{\PYGZdq{}i\PYGZdq{}}\PYG{n+nt}{\PYGZgt{}}
    \PYG{n+nt}{\PYGZlt{}t} \PYG{n+na}{t\PYGZhy{}set=}\PYG{l+s}{\PYGZdq{}existing\PYGZus{}variable\PYGZdq{}} \PYG{n+na}{t\PYGZhy{}value=}\PYG{l+s}{\PYGZdq{}True\PYGZdq{}}\PYG{n+nt}{/\PYGZgt{}}
    \PYG{n+nt}{\PYGZlt{}t} \PYG{n+na}{t\PYGZhy{}set=}\PYG{l+s}{\PYGZdq{}new\PYGZus{}variable\PYGZdq{}} \PYG{n+na}{t\PYGZhy{}value=}\PYG{l+s}{\PYGZdq{}True\PYGZdq{}}\PYG{n+nt}{/\PYGZgt{}}
    \PYG{c}{\PYGZlt{}!\PYGZhy{}\PYGZhy{}}\PYG{c}{ existing\PYGZus{}variable and new\PYGZus{}variable now True }\PYG{c}{\PYGZhy{}\PYGZhy{}\PYGZgt{}}
\PYG{n+nt}{\PYGZlt{}/p\PYGZgt{}}

\PYG{c}{\PYGZlt{}!\PYGZhy{}\PYGZhy{}}\PYG{c}{ existing\PYGZus{}variable always True }\PYG{c}{\PYGZhy{}\PYGZhy{}\PYGZgt{}}
\PYG{c}{\PYGZlt{}!\PYGZhy{}\PYGZhy{}}\PYG{c}{ new\PYGZus{}variable undefined }\PYG{c}{\PYGZhy{}\PYGZhy{}\PYGZgt{}}
\end{sphinxVerbatim}


\subsection{attributes}
\label{\detokenize{reference/qweb:attributes}}\label{\detokenize{reference/qweb:reference-qweb-attributes}}
QWeb can compute attributes on-the-fly and set the result of the computation
on the output node. This is done via the \sphinxcode{\sphinxupquote{t-att}} (attribute) directive which
exists in 3 different forms:
\begin{description}
\item[{\sphinxcode{\sphinxupquote{t-att-\sphinxstyleemphasis{\$name}}}}] \leavevmode
an attribute called \sphinxcode{\sphinxupquote{\$name}} is created, the attribute value is evaluated
and the result is set as the attribute’s value:

\fvset{hllines={, ,}}%
\begin{sphinxVerbatim}[commandchars=\\\{\}]
\PYG{n+nt}{\PYGZlt{}div} \PYG{n+na}{t\PYGZhy{}att\PYGZhy{}a=}\PYG{l+s}{\PYGZdq{}42\PYGZdq{}}\PYG{n+nt}{/\PYGZgt{}}
\end{sphinxVerbatim}

will be rendered as:

\fvset{hllines={, ,}}%
\begin{sphinxVerbatim}[commandchars=\\\{\}]
\PYG{n+nt}{\PYGZlt{}div} \PYG{n+na}{a=}\PYG{l+s}{\PYGZdq{}42\PYGZdq{}}\PYG{n+nt}{\PYGZgt{}}\PYG{n+nt}{\PYGZlt{}/div\PYGZgt{}}
\end{sphinxVerbatim}

\item[{\sphinxcode{\sphinxupquote{t-attf-\sphinxstyleemphasis{\$name}}}}] \leavevmode
same as previous, but the parameter is a \DUrole{xref,std,std-term}{format string}
instead of just an expression, often useful to mix literal and non-literal
string (e.g. classes):

\fvset{hllines={, ,}}%
\begin{sphinxVerbatim}[commandchars=\\\{\}]
\PYG{n+nt}{\PYGZlt{}t} \PYG{n+na}{t\PYGZhy{}foreach=}\PYG{l+s}{\PYGZdq{}[1, 2, 3]\PYGZdq{}} \PYG{n+na}{t\PYGZhy{}as=}\PYG{l+s}{\PYGZdq{}item\PYGZdq{}}\PYG{n+nt}{\PYGZgt{}}
    \PYG{n+nt}{\PYGZlt{}li} \PYG{n+na}{t\PYGZhy{}attf\PYGZhy{}class=}\PYG{l+s}{\PYGZdq{}row \PYGZob{}\PYGZob{} item\PYGZus{}parity \PYGZcb{}\PYGZcb{}\PYGZdq{}}\PYG{n+nt}{\PYGZgt{}}\PYG{n+nt}{\PYGZlt{}t} \PYG{n+na}{t\PYGZhy{}esc=}\PYG{l+s}{\PYGZdq{}item\PYGZdq{}}\PYG{n+nt}{/\PYGZgt{}}\PYG{n+nt}{\PYGZlt{}/li\PYGZgt{}}
\PYG{n+nt}{\PYGZlt{}/t\PYGZgt{}}
\end{sphinxVerbatim}

will be rendered as:

\fvset{hllines={, ,}}%
\begin{sphinxVerbatim}[commandchars=\\\{\}]
\PYG{n+nt}{\PYGZlt{}li} \PYG{n+na}{class=}\PYG{l+s}{\PYGZdq{}row even\PYGZdq{}}\PYG{n+nt}{\PYGZgt{}}1\PYG{n+nt}{\PYGZlt{}/li\PYGZgt{}}
\PYG{n+nt}{\PYGZlt{}li} \PYG{n+na}{class=}\PYG{l+s}{\PYGZdq{}row odd\PYGZdq{}}\PYG{n+nt}{\PYGZgt{}}2\PYG{n+nt}{\PYGZlt{}/li\PYGZgt{}}
\PYG{n+nt}{\PYGZlt{}li} \PYG{n+na}{class=}\PYG{l+s}{\PYGZdq{}row even\PYGZdq{}}\PYG{n+nt}{\PYGZgt{}}3\PYG{n+nt}{\PYGZlt{}/li\PYGZgt{}}
\end{sphinxVerbatim}

\item[{\sphinxcode{\sphinxupquote{t-att=mapping}}}] \leavevmode
if the parameter is a mapping, each (key, value) pair generates a new
attribute and its value:

\fvset{hllines={, ,}}%
\begin{sphinxVerbatim}[commandchars=\\\{\}]
\PYG{n+nt}{\PYGZlt{}div} \PYG{n+na}{t\PYGZhy{}att=}\PYG{l+s}{\PYGZdq{}\PYGZob{}\PYGZsq{}a\PYGZsq{}: 1, \PYGZsq{}b\PYGZsq{}: 2\PYGZcb{}\PYGZdq{}}\PYG{n+nt}{/\PYGZgt{}}
\end{sphinxVerbatim}

will be rendered as:

\fvset{hllines={, ,}}%
\begin{sphinxVerbatim}[commandchars=\\\{\}]
\PYG{n+nt}{\PYGZlt{}div} \PYG{n+na}{a=}\PYG{l+s}{\PYGZdq{}1\PYGZdq{}} \PYG{n+na}{b=}\PYG{l+s}{\PYGZdq{}2\PYGZdq{}}\PYG{n+nt}{\PYGZgt{}}\PYG{n+nt}{\PYGZlt{}/div\PYGZgt{}}
\end{sphinxVerbatim}

\item[{\sphinxcode{\sphinxupquote{t-att=pair}}}] \leavevmode
if the parameter is a pair (tuple or array of 2 element), the first
item of the pair is the name of the attribute and the second item is the
value:

\fvset{hllines={, ,}}%
\begin{sphinxVerbatim}[commandchars=\\\{\}]
\PYG{n+nt}{\PYGZlt{}div} \PYG{n+na}{t\PYGZhy{}att=}\PYG{l+s}{\PYGZdq{}[\PYGZsq{}a\PYGZsq{}, \PYGZsq{}b\PYGZsq{}]\PYGZdq{}}\PYG{n+nt}{/\PYGZgt{}}
\end{sphinxVerbatim}

will be rendered as:

\fvset{hllines={, ,}}%
\begin{sphinxVerbatim}[commandchars=\\\{\}]
\PYG{n+nt}{\PYGZlt{}div} \PYG{n+na}{a=}\PYG{l+s}{\PYGZdq{}b\PYGZdq{}}\PYG{n+nt}{\PYGZgt{}}\PYG{n+nt}{\PYGZlt{}/div\PYGZgt{}}
\end{sphinxVerbatim}

\end{description}


\subsection{setting variables}
\label{\detokenize{reference/qweb:setting-variables}}
QWeb allows creating variables from within the template, to memoize a
computation (to use it multiple times), give a piece of data a clearer name,
…

This is done via the \sphinxcode{\sphinxupquote{set}} directive, which takes the name of the variable
to create. The value to set can be provided in two ways:
\begin{itemize}
\item {} 
a \sphinxcode{\sphinxupquote{t-value}} attribute containing an expression, and the result of its
evaluation will be set:

\fvset{hllines={, ,}}%
\begin{sphinxVerbatim}[commandchars=\\\{\}]
\PYG{n+nt}{\PYGZlt{}t} \PYG{n+na}{t\PYGZhy{}set=}\PYG{l+s}{\PYGZdq{}foo\PYGZdq{}} \PYG{n+na}{t\PYGZhy{}value=}\PYG{l+s}{\PYGZdq{}2 + 1\PYGZdq{}}\PYG{n+nt}{/\PYGZgt{}}
\PYG{n+nt}{\PYGZlt{}t} \PYG{n+na}{t\PYGZhy{}esc=}\PYG{l+s}{\PYGZdq{}foo\PYGZdq{}}\PYG{n+nt}{/\PYGZgt{}}
\end{sphinxVerbatim}

will print \sphinxcode{\sphinxupquote{3}}

\item {} 
if there is no \sphinxcode{\sphinxupquote{t-value}} attribute, the node’s body is rendered and set
as the variable’s value:

\fvset{hllines={, ,}}%
\begin{sphinxVerbatim}[commandchars=\\\{\}]
\PYG{n+nt}{\PYGZlt{}t} \PYG{n+na}{t\PYGZhy{}set=}\PYG{l+s}{\PYGZdq{}foo\PYGZdq{}}\PYG{n+nt}{\PYGZgt{}}
    \PYG{n+nt}{\PYGZlt{}li}\PYG{n+nt}{\PYGZgt{}}ok\PYG{n+nt}{\PYGZlt{}/li\PYGZgt{}}
\PYG{n+nt}{\PYGZlt{}/t\PYGZgt{}}
\PYG{n+nt}{\PYGZlt{}t} \PYG{n+na}{t\PYGZhy{}esc=}\PYG{l+s}{\PYGZdq{}foo\PYGZdq{}}\PYG{n+nt}{/\PYGZgt{}}
\end{sphinxVerbatim}

will generate \sphinxcode{\sphinxupquote{\&lt;li\&gt;ok\&lt;/li\&gt;}} (the content is escaped as we
used the \sphinxcode{\sphinxupquote{esc}} directive)

\begin{sphinxadmonition}{note}{Note:}
using the result of this operation is a significant use-case for
the \sphinxcode{\sphinxupquote{raw}} directive.
\end{sphinxadmonition}

\end{itemize}


\subsection{calling sub-templates}
\label{\detokenize{reference/qweb:calling-sub-templates}}
QWeb templates can be used for top-level rendering, but they can also be used
from within another template (to avoid duplication or give names to parts of
templates) using the \sphinxcode{\sphinxupquote{t-call}} directive:

\fvset{hllines={, ,}}%
\begin{sphinxVerbatim}[commandchars=\\\{\}]
\PYG{n+nt}{\PYGZlt{}t} \PYG{n+na}{t\PYGZhy{}call=}\PYG{l+s}{\PYGZdq{}other\PYGZhy{}template\PYGZdq{}}\PYG{n+nt}{/\PYGZgt{}}
\end{sphinxVerbatim}

This calls the named template with the execution context of the parent, if
\sphinxcode{\sphinxupquote{other\_template}} is defined as:

\fvset{hllines={, ,}}%
\begin{sphinxVerbatim}[commandchars=\\\{\}]
\PYG{n+nt}{\PYGZlt{}p}\PYG{n+nt}{\PYGZgt{}}\PYG{n+nt}{\PYGZlt{}t} \PYG{n+na}{t\PYGZhy{}value=}\PYG{l+s}{\PYGZdq{}var\PYGZdq{}}\PYG{n+nt}{/\PYGZgt{}}\PYG{n+nt}{\PYGZlt{}/p\PYGZgt{}}
\end{sphinxVerbatim}

the call above will be rendered as \sphinxcode{\sphinxupquote{\textless{}p/\textgreater{}}} (no content), but:

\fvset{hllines={, ,}}%
\begin{sphinxVerbatim}[commandchars=\\\{\}]
\PYG{n+nt}{\PYGZlt{}t} \PYG{n+na}{t\PYGZhy{}set=}\PYG{l+s}{\PYGZdq{}var\PYGZdq{}} \PYG{n+na}{t\PYGZhy{}value=}\PYG{l+s}{\PYGZdq{}1\PYGZdq{}}\PYG{n+nt}{/\PYGZgt{}}
\PYG{n+nt}{\PYGZlt{}t} \PYG{n+na}{t\PYGZhy{}call=}\PYG{l+s}{\PYGZdq{}other\PYGZhy{}template\PYGZdq{}}\PYG{n+nt}{/\PYGZgt{}}
\end{sphinxVerbatim}

will be rendered as \sphinxcode{\sphinxupquote{\textless{}p\textgreater{}1\textless{}/p\textgreater{}}}.

However this has the problem of being visible from outside the \sphinxcode{\sphinxupquote{t-call}}.
Alternatively, content set in the body of the \sphinxcode{\sphinxupquote{call}} directive will be
evaluated \sphinxstyleemphasis{before} calling the sub-template, and can alter a local context:

\fvset{hllines={, ,}}%
\begin{sphinxVerbatim}[commandchars=\\\{\}]
\PYG{n+nt}{\PYGZlt{}t} \PYG{n+na}{t\PYGZhy{}call=}\PYG{l+s}{\PYGZdq{}other\PYGZhy{}template\PYGZdq{}}\PYG{n+nt}{\PYGZgt{}}
    \PYG{n+nt}{\PYGZlt{}t} \PYG{n+na}{t\PYGZhy{}set=}\PYG{l+s}{\PYGZdq{}var\PYGZdq{}} \PYG{n+na}{t\PYGZhy{}value=}\PYG{l+s}{\PYGZdq{}1\PYGZdq{}}\PYG{n+nt}{/\PYGZgt{}}
\PYG{n+nt}{\PYGZlt{}/t\PYGZgt{}}
\PYG{c}{\PYGZlt{}!\PYGZhy{}\PYGZhy{}}\PYG{c}{ \PYGZdq{}var\PYGZdq{} does not exist here }\PYG{c}{\PYGZhy{}\PYGZhy{}\PYGZgt{}}
\end{sphinxVerbatim}

The body of the \sphinxcode{\sphinxupquote{call}} directive can be arbitrarily complex (not just
\sphinxcode{\sphinxupquote{set}} directives), and its rendered form will be available within the called
template as a magical \sphinxcode{\sphinxupquote{0}} variable:

\fvset{hllines={, ,}}%
\begin{sphinxVerbatim}[commandchars=\\\{\}]
\PYG{n+nt}{\PYGZlt{}div}\PYG{n+nt}{\PYGZgt{}}
    This template was called with content:
    \PYG{n+nt}{\PYGZlt{}t} \PYG{n+na}{t\PYGZhy{}raw=}\PYG{l+s}{\PYGZdq{}0\PYGZdq{}}\PYG{n+nt}{/\PYGZgt{}}
\PYG{n+nt}{\PYGZlt{}/div\PYGZgt{}}
\end{sphinxVerbatim}

being called thus:

\fvset{hllines={, ,}}%
\begin{sphinxVerbatim}[commandchars=\\\{\}]
\PYG{n+nt}{\PYGZlt{}t} \PYG{n+na}{t\PYGZhy{}call=}\PYG{l+s}{\PYGZdq{}other\PYGZhy{}template\PYGZdq{}}\PYG{n+nt}{\PYGZgt{}}
    \PYG{n+nt}{\PYGZlt{}em}\PYG{n+nt}{\PYGZgt{}}content\PYG{n+nt}{\PYGZlt{}/em\PYGZgt{}}
\PYG{n+nt}{\PYGZlt{}/t\PYGZgt{}}
\end{sphinxVerbatim}

will result in:

\fvset{hllines={, ,}}%
\begin{sphinxVerbatim}[commandchars=\\\{\}]
\PYG{n+nt}{\PYGZlt{}div}\PYG{n+nt}{\PYGZgt{}}
    This template was called with content:
    \PYG{n+nt}{\PYGZlt{}em}\PYG{n+nt}{\PYGZgt{}}content\PYG{n+nt}{\PYGZlt{}/em\PYGZgt{}}
\PYG{n+nt}{\PYGZlt{}/div\PYGZgt{}}
\end{sphinxVerbatim}


\subsection{Python}
\label{\detokenize{reference/qweb:python}}

\subsubsection{Exclusive directives}
\label{\detokenize{reference/qweb:exclusive-directives}}

\paragraph{asset bundles}
\label{\detokenize{reference/qweb:asset-bundles}}

\paragraph{“smart records” fields formatting}
\label{\detokenize{reference/qweb:smart-records-fields-formatting}}\label{\detokenize{reference/qweb:index-0}}
The \sphinxcode{\sphinxupquote{t-field}} directive can only be used when performing field access
(\sphinxcode{\sphinxupquote{a.b}}) on a “smart” record (result of the \sphinxcode{\sphinxupquote{browse}} method). It is able
to automatically format based on field type, and is integrated in the
website’s rich text edition.

\sphinxcode{\sphinxupquote{t-options}} can be used to customize fields, the most common option
is \sphinxcode{\sphinxupquote{widget}}, other options are field- or widget-dependent.


\subsubsection{debugging}
\label{\detokenize{reference/qweb:debugging}}\begin{description}
\item[{\sphinxcode{\sphinxupquote{t-debug}}}] \leavevmode
invokes a debugger using PDB’s \sphinxcode{\sphinxupquote{set\_trace}} API. The parameter should
be the name of a module, on which a \sphinxcode{\sphinxupquote{set\_trace}} method is called:

\fvset{hllines={, ,}}%
\begin{sphinxVerbatim}[commandchars=\\\{\}]
\PYG{n+nt}{\PYGZlt{}t} \PYG{n+na}{t\PYGZhy{}debug=}\PYG{l+s}{\PYGZdq{}pdb\PYGZdq{}}\PYG{n+nt}{/\PYGZgt{}}
\end{sphinxVerbatim}

is equivalent to \sphinxcode{\sphinxupquote{importlib.import\_module("pdb").set\_trace()}}

\end{description}


\subsubsection{Helpers}
\label{\detokenize{reference/qweb:helpers}}

\paragraph{Request-based}
\label{\detokenize{reference/qweb:request-based}}
Most Python-side uses of QWeb are in controllers (and during HTTP requests),
in which case templates stored in the database (as
{\hyperref[\detokenize{reference/views:reference-views-qweb}]{\sphinxcrossref{\DUrole{std,std-ref}{views}}}}) can be trivially rendered by calling
{\hyperref[\detokenize{reference/http:odoo.http.HttpRequest.render}]{\sphinxcrossref{\sphinxcode{\sphinxupquote{odoo.http.HttpRequest.render()}}}}}:

\fvset{hllines={, ,}}%
\begin{sphinxVerbatim}[commandchars=\\\{\}]
\PYG{n}{response} \PYG{o}{=} \PYG{n}{http}\PYG{o}{.}\PYG{n}{request}\PYG{o}{.}\PYG{n}{render}\PYG{p}{(}\PYG{l+s+s1}{\PYGZsq{}}\PYG{l+s+s1}{my\PYGZhy{}template}\PYG{l+s+s1}{\PYGZsq{}}\PYG{p}{,} \PYG{p}{\PYGZob{}}
    \PYG{l+s+s1}{\PYGZsq{}}\PYG{l+s+s1}{context\PYGZus{}value}\PYG{l+s+s1}{\PYGZsq{}}\PYG{p}{:} \PYG{l+m+mi}{42}
\PYG{p}{\PYGZcb{}}\PYG{p}{)}
\end{sphinxVerbatim}

This automatically creates a {\hyperref[\detokenize{reference/http:odoo.http.Response}]{\sphinxcrossref{\sphinxcode{\sphinxupquote{Response}}}}} object which can
be returned from the controller (or further customized to suit).


\paragraph{View-based}
\label{\detokenize{reference/qweb:view-based}}
At a deeper level than the previous helper is the \sphinxcode{\sphinxupquote{render}} method on
\sphinxcode{\sphinxupquote{ir.ui.view}}:
\index{render()}

\begin{fulllineitems}
\phantomsection\label{\detokenize{reference/qweb:render}}\pysiglinewithargsret{\sphinxbfcode{\sphinxupquote{render}}}{\emph{cr, uid, id{[}, values{]}{[}, engine='ir.qweb{]}{[}, context{]}}}{}
Renders a QWeb view/template by database id or \DUrole{xref,std,std-term}{external id}.
Templates are automatically loaded from \sphinxcode{\sphinxupquote{ir.ui.view}} records.

Sets up a number of default values in the rendering context:
\begin{description}
\item[{\sphinxcode{\sphinxupquote{request}}}] \leavevmode
the current {\hyperref[\detokenize{reference/http:odoo.http.WebRequest}]{\sphinxcrossref{\sphinxcode{\sphinxupquote{WebRequest}}}}} object, if any

\item[{\sphinxcode{\sphinxupquote{debug}}}] \leavevmode
whether the current request (if any) is in \sphinxcode{\sphinxupquote{debug}} mode

\item[{\sphinxhref{http://werkzeug.pocoo.org/docs/urls/\#werkzeug.urls.url\_quote\_plus}{\sphinxcode{\sphinxupquote{quote\_plus}}}}] \leavevmode
url-encoding utility function

\item[{\sphinxhref{https://docs.python.org/3/library/json.html\#module-json}{\sphinxcode{\sphinxupquote{json}}}}] \leavevmode
the corresponding standard library module

\item[{\sphinxhref{https://docs.python.org/3/library/time.html\#module-time}{\sphinxcode{\sphinxupquote{time}}}}] \leavevmode
the corresponding standard library module

\item[{\sphinxhref{https://docs.python.org/3/library/datetime.html\#module-datetime}{\sphinxcode{\sphinxupquote{datetime}}}}] \leavevmode
the corresponding standard library module

\item[{\sphinxhref{https://labix.org/python-dateutil\#head-ba5ffd4df8111d1b83fc194b97ebecf837add454}{relativedelta}}] \leavevmode
see module

\item[{\sphinxcode{\sphinxupquote{keep\_query}}}] \leavevmode
the \sphinxcode{\sphinxupquote{keep\_query}} helper function

\end{description}
\begin{quote}\begin{description}
\item[{Parameters}] \leavevmode\begin{itemize}
\item {} 
\sphinxstyleliteralstrong{\sphinxupquote{values}} \textendash{} context values to pass to QWeb for rendering

\item {} 
\sphinxstyleliteralstrong{\sphinxupquote{engine}} (\sphinxhref{https://docs.python.org/3/library/stdtypes.html\#str}{\sphinxstyleliteralemphasis{\sphinxupquote{str}}}) \textendash{} name of the Odoo model to use for rendering, can be
used to expand or customize QWeb locally (by creating
a “new” qweb based on \sphinxcode{\sphinxupquote{ir.qweb}} with alterations)

\end{itemize}

\end{description}\end{quote}

\end{fulllineitems}

\phantomsection\label{\detokenize{reference/qweb:reference-qweb-javascript}}

\subsection{Javascript}
\label{\detokenize{reference/qweb:javascript}}

\subsubsection{Exclusive directives}
\label{\detokenize{reference/qweb:id3}}

\paragraph{defining templates}
\label{\detokenize{reference/qweb:defining-templates}}
The \sphinxcode{\sphinxupquote{t-name}} directive can only be placed at the top-level of a template
file (direct children to the document root):

\fvset{hllines={, ,}}%
\begin{sphinxVerbatim}[commandchars=\\\{\}]
\PYG{n+nt}{\PYGZlt{}templates}\PYG{n+nt}{\PYGZgt{}}
    \PYG{n+nt}{\PYGZlt{}t} \PYG{n+na}{t\PYGZhy{}name=}\PYG{l+s}{\PYGZdq{}template\PYGZhy{}name\PYGZdq{}}\PYG{n+nt}{\PYGZgt{}}
        \PYG{c}{\PYGZlt{}!\PYGZhy{}\PYGZhy{}}\PYG{c}{ template code }\PYG{c}{\PYGZhy{}\PYGZhy{}\PYGZgt{}}
    \PYG{n+nt}{\PYGZlt{}/t\PYGZgt{}}
\PYG{n+nt}{\PYGZlt{}/templates\PYGZgt{}}
\end{sphinxVerbatim}

It takes no other parameter, but can be used with a \sphinxcode{\sphinxupquote{\textless{}t\textgreater{}}} element or any
other. With a \sphinxcode{\sphinxupquote{\textless{}t\textgreater{}}} element, the \sphinxcode{\sphinxupquote{\textless{}t\textgreater{}}} should have a single child.

The template name is an arbitrary string, although when multiple templates
are related (e.g. called sub-templates) it is customary to use dot-separated
names to indicate hierarchical relationships.


\paragraph{template inheritance}
\label{\detokenize{reference/qweb:template-inheritance}}
Template inheritance is used to alter existing templates in-place, e.g. to
add information to templates created by an other modules.

Template inheritance is performed via the \sphinxcode{\sphinxupquote{t-extend}} directive which takes
the name of the template to alter as parameter.

The alteration is then performed with any number of \sphinxcode{\sphinxupquote{t-jquery}}
sub-directives:

\fvset{hllines={, ,}}%
\begin{sphinxVerbatim}[commandchars=\\\{\}]
\PYG{n+nt}{\PYGZlt{}t} \PYG{n+na}{t\PYGZhy{}extend=}\PYG{l+s}{\PYGZdq{}base.template\PYGZdq{}}\PYG{n+nt}{\PYGZgt{}}
    \PYG{n+nt}{\PYGZlt{}t} \PYG{n+na}{t\PYGZhy{}jquery=}\PYG{l+s}{\PYGZdq{}ul\PYGZdq{}} \PYG{n+na}{t\PYGZhy{}operation=}\PYG{l+s}{\PYGZdq{}append\PYGZdq{}}\PYG{n+nt}{\PYGZgt{}}
        \PYG{n+nt}{\PYGZlt{}li}\PYG{n+nt}{\PYGZgt{}}new element\PYG{n+nt}{\PYGZlt{}/li\PYGZgt{}}
    \PYG{n+nt}{\PYGZlt{}/t\PYGZgt{}}
\PYG{n+nt}{\PYGZlt{}/t\PYGZgt{}}
\end{sphinxVerbatim}

The \sphinxcode{\sphinxupquote{t-jquery}} directives takes a \sphinxhref{http://api.jquery.com/category/selectors/}{CSS selector}. This selector is used
on the extended template to select \sphinxstyleemphasis{context nodes} to which the specified
\sphinxcode{\sphinxupquote{t-operation}} is applied:
\begin{description}
\item[{\sphinxcode{\sphinxupquote{append}}}] \leavevmode
the node’s body is appended at the end of the context node (after the
context node’s last child)

\item[{\sphinxcode{\sphinxupquote{prepend}}}] \leavevmode
the node’s body is prepended to the context node (inserted before the
context node’s first child)

\item[{\sphinxcode{\sphinxupquote{before}}}] \leavevmode
the node’s body is inserted right before the context node

\item[{\sphinxcode{\sphinxupquote{after}}}] \leavevmode
the node’s body is inserted right after the context node

\item[{\sphinxcode{\sphinxupquote{inner}}}] \leavevmode
the node’s body replaces the context node’s children

\item[{\sphinxcode{\sphinxupquote{replace}}}] \leavevmode
the node’s body is used to replace the context node itself

\item[{No operation}] \leavevmode
if no \sphinxcode{\sphinxupquote{t-operation}} is specified, the template body is interpreted as
javascript code and executed with the context node as \sphinxcode{\sphinxupquote{this}}

\begin{sphinxadmonition}{warning}{Warning:}
while much more powerful than other operations, this mode is
also much harder to debug and maintain, it is recommended to
avoid it
\end{sphinxadmonition}

\end{description}


\subsubsection{debugging}
\label{\detokenize{reference/qweb:id4}}
The javascript QWeb implementation provides a few debugging hooks:
\begin{description}
\item[{\sphinxcode{\sphinxupquote{t-log}}}] \leavevmode
takes an expression parameter, evaluates the expression during rendering
and logs its result with \sphinxcode{\sphinxupquote{console.log}}:

\fvset{hllines={, ,}}%
\begin{sphinxVerbatim}[commandchars=\\\{\}]
\PYG{n+nt}{\PYGZlt{}t} \PYG{n+na}{t\PYGZhy{}set=}\PYG{l+s}{\PYGZdq{}foo\PYGZdq{}} \PYG{n+na}{t\PYGZhy{}value=}\PYG{l+s}{\PYGZdq{}42\PYGZdq{}}\PYG{n+nt}{/\PYGZgt{}}
\PYG{n+nt}{\PYGZlt{}t} \PYG{n+na}{t\PYGZhy{}log=}\PYG{l+s}{\PYGZdq{}foo\PYGZdq{}}\PYG{n+nt}{/\PYGZgt{}}
\end{sphinxVerbatim}

will print \sphinxcode{\sphinxupquote{42}} to the console

\item[{\sphinxcode{\sphinxupquote{t-debug}}}] \leavevmode
triggers a debugger breakpoint during template rendering:

\fvset{hllines={, ,}}%
\begin{sphinxVerbatim}[commandchars=\\\{\}]
\PYG{n+nt}{\PYGZlt{}t} \PYG{n+na}{t\PYGZhy{}if=}\PYG{l+s}{\PYGZdq{}a\PYGZus{}test\PYGZdq{}}\PYG{n+nt}{\PYGZgt{}}
    \PYG{n+nt}{\PYGZlt{}t} \PYG{n+na}{t\PYGZhy{}debug=}\PYG{l+s}{\PYGZdq{}\PYGZdq{}}\PYG{n+nt}{\PYGZgt{}}
\PYG{n+nt}{\PYGZlt{}/t\PYGZgt{}}
\end{sphinxVerbatim}

will stop execution if debugging is active (exact condition depend on the
browser and its development tools)

\item[{\sphinxcode{\sphinxupquote{t-js}}}] \leavevmode
the node’s body is javascript code executed during template rendering.
Takes a \sphinxcode{\sphinxupquote{context}} parameter, which is the name under which the rendering
context will be available in the \sphinxcode{\sphinxupquote{t-js}}’s body:

\fvset{hllines={, ,}}%
\begin{sphinxVerbatim}[commandchars=\\\{\}]
\PYG{n+nt}{\PYGZlt{}t} \PYG{n+na}{t\PYGZhy{}set=}\PYG{l+s}{\PYGZdq{}foo\PYGZdq{}} \PYG{n+na}{t\PYGZhy{}value=}\PYG{l+s}{\PYGZdq{}42\PYGZdq{}}\PYG{n+nt}{/\PYGZgt{}}
\PYG{n+nt}{\PYGZlt{}t} \PYG{n+na}{t\PYGZhy{}js=}\PYG{l+s}{\PYGZdq{}ctx\PYGZdq{}}\PYG{n+nt}{\PYGZgt{}}
    console.log(\PYGZdq{}Foo is\PYGZdq{}, ctx.foo);
\PYG{n+nt}{\PYGZlt{}/t\PYGZgt{}}
\end{sphinxVerbatim}

\end{description}


\subsubsection{Helpers}
\label{\detokenize{reference/qweb:id5}}\index{core.qweb (core attribute)}

\begin{fulllineitems}
\phantomsection\label{\detokenize{reference/qweb:core.qweb}}\pysigline{\sphinxcode{\sphinxupquote{core.}}\sphinxbfcode{\sphinxupquote{qweb}}}
(core is the \sphinxcode{\sphinxupquote{web.core}} module) An instance of {\hyperref[\detokenize{reference/qweb:QWeb2.Engine}]{\sphinxcrossref{\sphinxcode{\sphinxupquote{QWeb2.Engine()}}}}} with all module-defined template
files loaded, and references to standard helper objects \sphinxcode{\sphinxupquote{\_}}
(underscore), \sphinxcode{\sphinxupquote{\_t}} (translation function) and \sphinxhref{https://developer.mozilla.org/en-US/docs/Web/JavaScript/Reference/Global\_Objects/JSON}{JSON}.

\sphinxcode{\sphinxupquote{core.qweb.render}} can be used to
easily render basic module templates

\end{fulllineitems}



\subsubsection{API}
\label{\detokenize{reference/qweb:api}}\index{QWeb2.Engine() (class)}

\begin{fulllineitems}
\phantomsection\label{\detokenize{reference/qweb:QWeb2.Engine}}\pysiglinewithargsret{\sphinxbfcode{\sphinxupquote{class }}\sphinxcode{\sphinxupquote{QWeb2.}}\sphinxbfcode{\sphinxupquote{Engine}}}{}{}
The QWeb “renderer”, handles most of QWeb’s logic (loading,
parsing, compiling and rendering templates).

OpenERP Web instantiates one for the user in the core module, and
exports it to \sphinxcode{\sphinxupquote{core.qweb}}. It also loads all the template files
of the various modules into that QWeb instance.

A {\hyperref[\detokenize{reference/qweb:QWeb2.Engine}]{\sphinxcrossref{\sphinxcode{\sphinxupquote{QWeb2.Engine()}}}}} also serves as a “template namespace”.
\index{QWeb2.Engine.QWeb2.Engine.render() (QWeb2.Engine.QWeb2.Engine method)}

\begin{fulllineitems}
\phantomsection\label{\detokenize{reference/qweb:QWeb2.Engine.QWeb2.Engine.render}}\pysiglinewithargsret{\sphinxcode{\sphinxupquote{QWeb2.Engine.QWeb2.Engine.}}\sphinxbfcode{\sphinxupquote{render}}}{\emph{template}\sphinxoptional{, \emph{context}}}{}
Renders a previously loaded template to a String, using
\sphinxcode{\sphinxupquote{context}} (if provided) to find the variables accessed
during template rendering (e.g. strings to display).
\begin{quote}\begin{description}
\item[{Arguments}] \leavevmode\begin{itemize}
\item {} 
\sphinxstyleliteralstrong{\sphinxupquote{template}} (\sphinxstyleliteralemphasis{\sphinxupquote{String}}) \textendash{} the name of the template to render

\item {} 
\sphinxstyleliteralstrong{\sphinxupquote{context}} (\sphinxstyleliteralemphasis{\sphinxupquote{Object}}) \textendash{} the basic namespace to use for template
rendering

\end{itemize}

\item[{Returns}] \leavevmode
String

\end{description}\end{quote}

\end{fulllineitems}


The engine exposes an other method which may be useful in some
cases (e.g. if you need a separate template namespace with, in
OpenERP Web, Kanban views get their own {\hyperref[\detokenize{reference/qweb:QWeb2.Engine}]{\sphinxcrossref{\sphinxcode{\sphinxupquote{QWeb2.Engine()}}}}}
instance so their templates don’t collide with more general
“module” templates):
\index{QWeb2.Engine.QWeb2.Engine.add\_template() (QWeb2.Engine.QWeb2.Engine method)}

\begin{fulllineitems}
\phantomsection\label{\detokenize{reference/qweb:QWeb2.Engine.QWeb2.Engine.add_template}}\pysiglinewithargsret{\sphinxcode{\sphinxupquote{QWeb2.Engine.QWeb2.Engine.}}\sphinxbfcode{\sphinxupquote{add\_template}}}{\emph{templates}}{}
Loads a template file (a collection of templates) in the QWeb
instance. The templates can be specified as:
\begin{description}
\item[{An XML string}] \leavevmode
QWeb will attempt to parse it to an XML document then load
it.

\item[{A URL}] \leavevmode
QWeb will attempt to download the URL content, then load
the resulting XML string.

\item[{A \sphinxcode{\sphinxupquote{Document}} or \sphinxcode{\sphinxupquote{Node}}}] \leavevmode
QWeb will traverse the first level of the document (the
child nodes of the provided root) and load any named
template or template override.

\end{description}
\begin{quote}\begin{description}
\end{description}\end{quote}

\end{fulllineitems}


A {\hyperref[\detokenize{reference/qweb:QWeb2.Engine}]{\sphinxcrossref{\sphinxcode{\sphinxupquote{QWeb2.Engine()}}}}} also exposes various attributes for
behavior customization:
\index{QWeb2.Engine.QWeb2.Engine.prefix (QWeb2.Engine.QWeb2.Engine attribute)}

\begin{fulllineitems}
\phantomsection\label{\detokenize{reference/qweb:QWeb2.Engine.QWeb2.Engine.prefix}}\pysigline{\sphinxcode{\sphinxupquote{QWeb2.Engine.QWeb2.Engine.}}\sphinxbfcode{\sphinxupquote{prefix}}}
Prefix used to recognize directives during parsing. A string. By
default, \sphinxcode{\sphinxupquote{t}}.

\end{fulllineitems}

\index{QWeb2.Engine.QWeb2.Engine.debug (QWeb2.Engine.QWeb2.Engine attribute)}

\begin{fulllineitems}
\phantomsection\label{\detokenize{reference/qweb:QWeb2.Engine.QWeb2.Engine.debug}}\pysigline{\sphinxcode{\sphinxupquote{QWeb2.Engine.QWeb2.Engine.}}\sphinxbfcode{\sphinxupquote{debug}}}
Boolean flag putting the engine in “debug mode”. Normally,
QWeb intercepts any error raised during template execution. In
debug mode, it leaves all exceptions go through without
intercepting them.

\end{fulllineitems}

\index{QWeb2.Engine.QWeb2.Engine.jQuery (QWeb2.Engine.QWeb2.Engine attribute)}

\begin{fulllineitems}
\phantomsection\label{\detokenize{reference/qweb:QWeb2.Engine.QWeb2.Engine.jQuery}}\pysigline{\sphinxcode{\sphinxupquote{QWeb2.Engine.QWeb2.Engine.}}\sphinxbfcode{\sphinxupquote{jQuery}}}
The jQuery instance used during template inheritance processing.
Defaults to \sphinxcode{\sphinxupquote{window.jQuery}}.

\end{fulllineitems}

\index{QWeb2.Engine.QWeb2.Engine.preprocess\_node (QWeb2.Engine.QWeb2.Engine attribute)}

\begin{fulllineitems}
\phantomsection\label{\detokenize{reference/qweb:QWeb2.Engine.QWeb2.Engine.preprocess_node}}\pysigline{\sphinxcode{\sphinxupquote{QWeb2.Engine.QWeb2.Engine.}}\sphinxbfcode{\sphinxupquote{preprocess\_node}}}
A \sphinxcode{\sphinxupquote{Function}}. If present, called before compiling each DOM
node to template code. In OpenERP Web, this is used to
automatically translate text content and some attributes in
templates. Defaults to \sphinxcode{\sphinxupquote{null}}.

\end{fulllineitems}


\end{fulllineitems}



\section{Javascript Cheatsheet}
\label{\detokenize{reference/javascript_cheatsheet:javascript-cheatsheet}}\label{\detokenize{reference/javascript_cheatsheet::doc}}\label{\detokenize{reference/javascript_cheatsheet:css-selector}}
There are many ways to solve a problem in JavaScript, and in Odoo.  However, the
Odoo framework was designed to be extensible (this is a pretty big constraint),
and some common problems have a nice standard solution.  The standard solution
has probably the advantage of being easy to understand for an odoo developers,
and will probably keep working when Odoo is modified.

This document tries to explain the way one could solve some of these issues.
Note that this is not a reference.  This is just a random collection of recipes,
or explanations on how to proceed in some cases.

First of all, remember that the first rule of customizing odoo with JS is:
\sphinxstyleemphasis{try to do it in python}.  This may seem strange, but the python framework is
quite extensible, and many behaviours can be done simply with a touch of xml or
python.  This has usually a lower cost of maintenance than working with JS:
\begin{itemize}
\item {} 
the JS framework tends to change more, so JS code needs to be more frequently
updated

\item {} 
it is often more difficult to implement a customized behaviour if it needs to
communicate with the server and properly integrate with the javascript framework.
There are many small details taken care by the framework that customized code
needs to replicate.  For example, responsiveness, or updating the url, or
displaying data without flickering.

\end{itemize}

\begin{sphinxadmonition}{note}{Note:}
This document does not really explain any concepts. This is more a
cookbook.  For more details, please consult the javascript reference
page (see {\hyperref[\detokenize{reference/javascript_reference::doc}]{\sphinxcrossref{\DUrole{doc}{Javascript Reference}}}})
\end{sphinxadmonition}


\subsection{Creating a new field widget}
\label{\detokenize{reference/javascript_cheatsheet:creating-a-new-field-widget}}
This is probably a really common usecase: we want to display some information in
a form view in a really specific (maybe business dependent) way.  For example,
assume that we want to change the text color depending on some business condition.

This can be done in three steps: creating a new widget, registering it in the
field registry, then adding the widget to the field in the form view
\begin{itemize}
\item {} \begin{description}
\item[{creating a new widget:}] \leavevmode
This can be done by extending a widget:

\fvset{hllines={, ,}}%
\begin{sphinxVerbatim}[commandchars=\\\{\}]
\PYG{k+kd}{var} \PYG{n+nx}{FieldChar} \PYG{o}{=} \PYG{n+nx}{require}\PYG{p}{(}\PYG{l+s+s1}{\PYGZsq{}web.basic\PYGZus{}fields\PYGZsq{}}\PYG{p}{)}\PYG{p}{.}\PYG{n+nx}{FieldChar}\PYG{p}{;}

\PYG{k+kd}{var} \PYG{n+nx}{CustomFieldChar} \PYG{o}{=} \PYG{n+nx}{Fieldchar}\PYG{p}{.}\PYG{n+nx}{extend}\PYG{p}{(}\PYG{p}{\PYGZob{}}
    \PYG{n+nx}{renderReadonly}\PYG{o}{:} \PYG{k+kd}{function} \PYG{p}{(}\PYG{p}{)} \PYG{p}{\PYGZob{}}
        \PYG{c+c1}{// implement some custom logic here}
    \PYG{p}{\PYGZcb{}}\PYG{p}{,}
\PYG{p}{\PYGZcb{}}\PYG{p}{)}\PYG{p}{;}
\end{sphinxVerbatim}

\end{description}

\item {} \begin{description}
\item[{registering it in the field registry:}] \leavevmode
The web client needs to know the mapping between a widget name and its
actual class.  This is done by a registry:

\fvset{hllines={, ,}}%
\begin{sphinxVerbatim}[commandchars=\\\{\}]
\PYG{k+kd}{var} \PYG{n+nx}{fieldRegistry} \PYG{o}{=} \PYG{n+nx}{require}\PYG{p}{(}\PYG{l+s+s1}{\PYGZsq{}web.field\PYGZus{}registry\PYGZsq{}}\PYG{p}{)}\PYG{p}{;}

\PYG{n+nx}{fieldRegistry}\PYG{p}{.}\PYG{n+nx}{add}\PYG{p}{(}\PYG{l+s+s1}{\PYGZsq{}my\PYGZhy{}custom\PYGZhy{}field\PYGZsq{}}\PYG{p}{,} \PYG{n+nx}{CustomFieldChar}\PYG{p}{)}\PYG{p}{;}
\end{sphinxVerbatim}

\end{description}

\item {} \begin{description}
\item[{adding the widget in the form view}] \leavevmode
\fvset{hllines={, ,}}%
\begin{sphinxVerbatim}[commandchars=\\\{\}]
\PYG{n+nt}{\PYGZlt{}field} \PYG{n+na}{name=}\PYG{l+s}{\PYGZdq{}somefield\PYGZdq{}} \PYG{n+na}{widget=}\PYG{l+s}{\PYGZdq{}my\PYGZhy{}custom\PYGZhy{}field\PYGZdq{}}\PYG{n+nt}{/\PYGZgt{}}
\end{sphinxVerbatim}

Note that only the form, list and kanban views use this field widgets registry.
These views are tightly integrated, because the list and kanban views can
appear inside a form view).

\end{description}

\end{itemize}


\subsection{Modifying an existing field widget}
\label{\detokenize{reference/javascript_cheatsheet:modifying-an-existing-field-widget}}
Another use case is that we want to modify an existing field widget.  For
example, the voip addon in odoo need to modify the FieldPhone widget to add the
possibility to easily call the given number on voip. This is done by \sphinxstyleemphasis{including}
the FieldPhone widget, so there is no need to change any existing form view.

Field Widgets (instances of (subclass of) AbstractField) are like every other
widgets, so they can be monkey patched. This looks like this:

\fvset{hllines={, ,}}%
\begin{sphinxVerbatim}[commandchars=\\\{\}]
\PYG{k+kd}{var} \PYG{n+nx}{basic\PYGZus{}fields} \PYG{o}{=} \PYG{n+nx}{require}\PYG{p}{(}\PYG{l+s+s1}{\PYGZsq{}web.basic\PYGZus{}fields\PYGZsq{}}\PYG{p}{)}\PYG{p}{;}
\PYG{k+kd}{var} \PYG{n+nx}{Phone} \PYG{o}{=} \PYG{n+nx}{basic\PYGZus{}fields}\PYG{p}{.}\PYG{n+nx}{FieldPhone}\PYG{p}{;}

\PYG{n+nx}{Phone}\PYG{p}{.}\PYG{n+nx}{include}\PYG{p}{(}\PYG{p}{\PYGZob{}}
    \PYG{n+nx}{events}\PYG{o}{:} \PYG{n+nx}{\PYGZus{}}\PYG{p}{.}\PYG{n+nx}{extend}\PYG{p}{(}\PYG{p}{\PYGZob{}}\PYG{p}{\PYGZcb{}}\PYG{p}{,} \PYG{n+nx}{Phone}\PYG{p}{.}\PYG{n+nx}{prototype}\PYG{p}{.}\PYG{n+nx}{events}\PYG{p}{,} \PYG{p}{\PYGZob{}}
        \PYG{l+s+s1}{\PYGZsq{}click\PYGZsq{}}\PYG{o}{:} \PYG{l+s+s1}{\PYGZsq{}\PYGZus{}onClick\PYGZsq{}}\PYG{p}{,}
    \PYG{p}{\PYGZcb{}}\PYG{p}{)}\PYG{p}{,}

    \PYG{n+nx}{\PYGZus{}onClick}\PYG{o}{:} \PYG{k+kd}{function} \PYG{p}{(}\PYG{n+nx}{e}\PYG{p}{)} \PYG{p}{\PYGZob{}}
        \PYG{k}{if} \PYG{p}{(}\PYG{k}{this}\PYG{p}{.}\PYG{n+nx}{mode} \PYG{o}{===} \PYG{l+s+s1}{\PYGZsq{}readonly\PYGZsq{}}\PYG{p}{)} \PYG{p}{\PYGZob{}}
            \PYG{n+nx}{e}\PYG{p}{.}\PYG{n+nx}{preventDefault}\PYG{p}{(}\PYG{p}{)}\PYG{p}{;}
            \PYG{k+kd}{var} \PYG{n+nx}{phoneNumber} \PYG{o}{=} \PYG{k}{this}\PYG{p}{.}\PYG{n+nx}{value}\PYG{p}{;}
            \PYG{c+c1}{// call the number on voip...}
        \PYG{p}{\PYGZcb{}}
    \PYG{p}{\PYGZcb{}}\PYG{p}{,}
\PYG{p}{\PYGZcb{}}\PYG{p}{)}\PYG{p}{;}
\end{sphinxVerbatim}

Note that there is no need to add the widget to the registry, since it is already
registered.


\subsection{Modifying a main widget from the interface}
\label{\detokenize{reference/javascript_cheatsheet:modifying-a-main-widget-from-the-interface}}
Another common usecase is the need to customize some elements from the user
interface.  For example, adding a message in the home menu.  The usual process
in this case is again to \sphinxstyleemphasis{include} the widget.  This is the only way to do it,
since there are no registries for those widgets.

This is usually done with code looking like this:

\fvset{hllines={, ,}}%
\begin{sphinxVerbatim}[commandchars=\\\{\}]
\PYG{k+kd}{var} \PYG{n+nx}{AppSwitcher} \PYG{o}{=} \PYG{n+nx}{require}\PYG{p}{(}\PYG{l+s+s1}{\PYGZsq{}web\PYGZus{}enterprise.AppSwitcher\PYGZsq{}}\PYG{p}{)}\PYG{p}{;}

\PYG{n+nx}{AppSwitcher}\PYG{p}{.}\PYG{n+nx}{include}\PYG{p}{(}\PYG{p}{\PYGZob{}}
    \PYG{n+nx}{render}\PYG{o}{:} \PYG{k+kd}{function} \PYG{p}{(}\PYG{p}{)} \PYG{p}{\PYGZob{}}
        \PYG{k}{this}\PYG{p}{.}\PYG{n+nx}{\PYGZus{}super}\PYG{p}{(}\PYG{p}{)}\PYG{p}{;}
        \PYG{c+c1}{// do something else here...}
    \PYG{p}{\PYGZcb{}}\PYG{p}{,}
\PYG{p}{\PYGZcb{}}\PYG{p}{)}\PYG{p}{;}
\end{sphinxVerbatim}


\subsection{Adding a client action}
\label{\detokenize{reference/javascript_cheatsheet:adding-a-client-action}}
A client action is a widget which will control the part of the screen below the
menu bar.  It can have a control panel, if necessary.  Defining a client action
can be done in two steps: implementing a new widget, and registering the widget
in the action registry.
\begin{itemize}
\item {} \begin{description}
\item[{Implementing a new client action:}] \leavevmode
This is done by creating a widget:

\fvset{hllines={, ,}}%
\begin{sphinxVerbatim}[commandchars=\\\{\}]
\PYG{k+kd}{var} \PYG{n+nx}{ControlPanelMixin} \PYG{o}{=} \PYG{n+nx}{require}\PYG{p}{(}\PYG{l+s+s1}{\PYGZsq{}web.ControlPanelMixin\PYGZsq{}}\PYG{p}{)}\PYG{p}{;}
\PYG{k+kd}{var} \PYG{n+nx}{Widget} \PYG{o}{=} \PYG{n+nx}{require}\PYG{p}{(}\PYG{l+s+s1}{\PYGZsq{}web.Widget\PYGZsq{}}\PYG{p}{)}\PYG{p}{;}

\PYG{k+kd}{var} \PYG{n+nx}{ClientAction} \PYG{o}{=} \PYG{n+nx}{Widget}\PYG{p}{.}\PYG{n+nx}{extend}\PYG{p}{(}\PYG{n+nx}{ControlPanelMixin}\PYG{p}{,} \PYG{p}{\PYGZob{}}
    \PYG{p}{...}
\PYG{p}{\PYGZcb{}}\PYG{p}{)}\PYG{p}{;}
\end{sphinxVerbatim}

Do not add the controlpanel mixin if you do not need it.  Note that some
code is needed to interact with the control panel (via the
\sphinxcode{\sphinxupquote{update\_control\_panel}} method given by the mixin).

\end{description}

\item {} \begin{description}
\item[{Registering the client action:}] \leavevmode
As usual, we need to make the web client aware of the mapping between
client actions and the actual class:

\fvset{hllines={, ,}}%
\begin{sphinxVerbatim}[commandchars=\\\{\}]
\PYG{k+kd}{var} \PYG{n+nx}{core} \PYG{o}{=} \PYG{n+nx}{require}\PYG{p}{(}\PYG{l+s+s1}{\PYGZsq{}web.core\PYGZsq{}}\PYG{p}{)}\PYG{p}{;}

\PYG{n+nx}{core}\PYG{p}{.}\PYG{n+nx}{action\PYGZus{}registry}\PYG{p}{.}\PYG{n+nx}{add}\PYG{p}{(}\PYG{l+s+s1}{\PYGZsq{}my\PYGZhy{}custom\PYGZhy{}action\PYGZsq{}}\PYG{p}{,} \PYG{n+nx}{ClientAction}\PYG{p}{)}\PYG{p}{;}
\end{sphinxVerbatim}

Then, to use the client action in the web client, we need to create a client
action record (a record of the model \sphinxcode{\sphinxupquote{ir.actions.client}}) with the proper
\sphinxcode{\sphinxupquote{tag}} attribute:

\fvset{hllines={, ,}}%
\begin{sphinxVerbatim}[commandchars=\\\{\}]
\PYG{n+nt}{\PYGZlt{}record} \PYG{n+na}{id=}\PYG{l+s}{\PYGZdq{}my\PYGZus{}client\PYGZus{}action\PYGZdq{}} \PYG{n+na}{model=}\PYG{l+s}{\PYGZdq{}ir.actions.client\PYGZdq{}}\PYG{n+nt}{\PYGZgt{}}
    \PYG{n+nt}{\PYGZlt{}field} \PYG{n+na}{name=}\PYG{l+s}{\PYGZdq{}name\PYGZdq{}}\PYG{n+nt}{\PYGZgt{}}Some Name\PYG{n+nt}{\PYGZlt{}/field\PYGZgt{}}
    \PYG{n+nt}{\PYGZlt{}field} \PYG{n+na}{name=}\PYG{l+s}{\PYGZdq{}tag\PYGZdq{}}\PYG{n+nt}{\PYGZgt{}}my\PYGZhy{}custom\PYGZhy{}action\PYG{n+nt}{\PYGZlt{}/field\PYGZgt{}}
\PYG{n+nt}{\PYGZlt{}/record\PYGZgt{}}
\end{sphinxVerbatim}

\end{description}

\end{itemize}


\subsection{Creating a new view (from scratch)}
\label{\detokenize{reference/javascript_cheatsheet:creating-a-new-view-from-scratch}}
Creating a new view is a more advanced topic.  This cheatsheet will only
highlight the steps that will probably need to be done (in no particular order):
\begin{itemize}
\item {} 
adding a new view type to the field \sphinxcode{\sphinxupquote{type}} of \sphinxcode{\sphinxupquote{ir.ui.view}}:

\fvset{hllines={, ,}}%
\begin{sphinxVerbatim}[commandchars=\\\{\}]
\PYG{k}{class} \PYG{n+nc}{View}\PYG{p}{(}\PYG{n}{models}\PYG{o}{.}\PYG{n}{Model}\PYG{p}{)}\PYG{p}{:}
    \PYG{n}{\PYGZus{}inherit} \PYG{o}{=} \PYG{l+s+s1}{\PYGZsq{}}\PYG{l+s+s1}{ir.ui.view}\PYG{l+s+s1}{\PYGZsq{}}

    \PYG{n+nb}{type} \PYG{o}{=} \PYG{n}{fields}\PYG{o}{.}\PYG{n}{Selection}\PYG{p}{(}\PYG{n}{selection\PYGZus{}add}\PYG{o}{=}\PYG{p}{[}\PYG{p}{(}\PYG{l+s+s1}{\PYGZsq{}}\PYG{l+s+s1}{map}\PYG{l+s+s1}{\PYGZsq{}}\PYG{p}{,} \PYG{l+s+s2}{\PYGZdq{}}\PYG{l+s+s2}{Map}\PYG{l+s+s2}{\PYGZdq{}}\PYG{p}{)}\PYG{p}{]}\PYG{p}{)}
\end{sphinxVerbatim}

\item {} 
adding the new view type to the field \sphinxcode{\sphinxupquote{view\_mode}} of \sphinxcode{\sphinxupquote{ir.actions.act\_window.view}}:

\fvset{hllines={, ,}}%
\begin{sphinxVerbatim}[commandchars=\\\{\}]
\PYG{k}{class} \PYG{n+nc}{ActWindowView}\PYG{p}{(}\PYG{n}{models}\PYG{o}{.}\PYG{n}{Model}\PYG{p}{)}\PYG{p}{:}
    \PYG{n}{\PYGZus{}inherit} \PYG{o}{=} \PYG{l+s+s1}{\PYGZsq{}}\PYG{l+s+s1}{ir.actions.act\PYGZus{}window.view}\PYG{l+s+s1}{\PYGZsq{}}

    \PYG{n}{view\PYGZus{}mode} \PYG{o}{=} \PYG{n}{fields}\PYG{o}{.}\PYG{n}{Selection}\PYG{p}{(}\PYG{n}{selection\PYGZus{}add}\PYG{o}{=}\PYG{p}{[}\PYG{p}{(}\PYG{l+s+s1}{\PYGZsq{}}\PYG{l+s+s1}{map}\PYG{l+s+s1}{\PYGZsq{}}\PYG{p}{,} \PYG{l+s+s2}{\PYGZdq{}}\PYG{l+s+s2}{Map}\PYG{l+s+s2}{\PYGZdq{}}\PYG{p}{)}\PYG{p}{]}\PYG{p}{)}
\end{sphinxVerbatim}

\item {} \begin{description}
\item[{creating the four main pieces which makes a view (in JavaScript):}] \leavevmode
we need a view (a subclass of \sphinxcode{\sphinxupquote{AbstractView}}, this is the factory), a
renderer (from \sphinxcode{\sphinxupquote{AbstractRenderer}}), a controller (from \sphinxcode{\sphinxupquote{AbstractController}})
and a model (from \sphinxcode{\sphinxupquote{AbstractModel}}).  I suggest starting by simply
extending the superclasses:

\fvset{hllines={, ,}}%
\begin{sphinxVerbatim}[commandchars=\\\{\}]
\PYG{k+kd}{var} \PYG{n+nx}{AbstractController} \PYG{o}{=} \PYG{n+nx}{require}\PYG{p}{(}\PYG{l+s+s1}{\PYGZsq{}web.AbstractController\PYGZsq{}}\PYG{p}{)}\PYG{p}{;}
\PYG{k+kd}{var} \PYG{n+nx}{AbstractModel} \PYG{o}{=} \PYG{n+nx}{require}\PYG{p}{(}\PYG{l+s+s1}{\PYGZsq{}web.AbstractModel\PYGZsq{}}\PYG{p}{)}\PYG{p}{;}
\PYG{k+kd}{var} \PYG{n+nx}{AbstractRenderer} \PYG{o}{=} \PYG{n+nx}{require}\PYG{p}{(}\PYG{l+s+s1}{\PYGZsq{}web.AbstractRenderer\PYGZsq{}}\PYG{p}{)}\PYG{p}{;}
\PYG{k+kd}{var} \PYG{n+nx}{AbstractView} \PYG{o}{=} \PYG{n+nx}{require}\PYG{p}{(}\PYG{l+s+s1}{\PYGZsq{}web.AbstractView\PYGZsq{}}\PYG{p}{)}\PYG{p}{;}

\PYG{k+kd}{var} \PYG{n+nx}{MapController} \PYG{o}{=} \PYG{n+nx}{AbstractController}\PYG{p}{.}\PYG{n+nx}{extend}\PYG{p}{(}\PYG{p}{\PYGZob{}}\PYG{p}{\PYGZcb{}}\PYG{p}{)}\PYG{p}{;}
\PYG{k+kd}{var} \PYG{n+nx}{MapRenderer} \PYG{o}{=} \PYG{n+nx}{AbstractRenderer}\PYG{p}{.}\PYG{n+nx}{extend}\PYG{p}{(}\PYG{p}{\PYGZob{}}\PYG{p}{\PYGZcb{}}\PYG{p}{)}\PYG{p}{;}
\PYG{k+kd}{var} \PYG{n+nx}{MapModel} \PYG{o}{=} \PYG{n+nx}{AbstractModel}\PYG{p}{.}\PYG{n+nx}{extend}\PYG{p}{(}\PYG{p}{\PYGZob{}}\PYG{p}{\PYGZcb{}}\PYG{p}{)}\PYG{p}{;}

\PYG{k+kd}{var} \PYG{n+nx}{MapView} \PYG{o}{=} \PYG{n+nx}{AbstractView}\PYG{p}{.}\PYG{n+nx}{extend}\PYG{p}{(}\PYG{p}{\PYGZob{}}
    \PYG{n+nx}{config}\PYG{o}{:} \PYG{p}{\PYGZob{}}
        \PYG{n+nx}{Model}\PYG{o}{:} \PYG{n+nx}{MapModel}\PYG{p}{,}
        \PYG{n+nx}{Controller}\PYG{o}{:} \PYG{n+nx}{MapController}\PYG{p}{,}
        \PYG{n+nx}{Renderer}\PYG{o}{:} \PYG{n+nx}{MapRenderer}\PYG{p}{,}
    \PYG{p}{\PYGZcb{}}\PYG{p}{,}
\PYG{p}{\PYGZcb{}}\PYG{p}{)}\PYG{p}{;}
\end{sphinxVerbatim}

\end{description}

\item {} \begin{description}
\item[{adding the view to the registry:}] \leavevmode
As usual, the mapping between a view type and the actual class needs to be
updated:

\fvset{hllines={, ,}}%
\begin{sphinxVerbatim}[commandchars=\\\{\}]
\PYG{k+kd}{var} \PYG{n+nx}{viewRegistry} \PYG{o}{=} \PYG{n+nx}{require}\PYG{p}{(}\PYG{l+s+s1}{\PYGZsq{}web.view\PYGZus{}registry\PYGZsq{}}\PYG{p}{)}\PYG{p}{;}

\PYG{n+nx}{viewRegistry}\PYG{p}{.}\PYG{n+nx}{add}\PYG{p}{(}\PYG{l+s+s1}{\PYGZsq{}map\PYGZsq{}}\PYG{p}{,} \PYG{n+nx}{MapView}\PYG{p}{)}\PYG{p}{;}
\end{sphinxVerbatim}

\end{description}

\item {} \begin{description}
\item[{implementing the four main classes:}] \leavevmode
The \sphinxcode{\sphinxupquote{View}} class needs to parse the \sphinxcode{\sphinxupquote{arch}} field and setup the other
three classes.  The \sphinxcode{\sphinxupquote{Renderer}} is in charge of representing the data in
the user interface, the \sphinxcode{\sphinxupquote{Model}} is supposed to talk to the server, to
load data and process it.  And the \sphinxcode{\sphinxupquote{Controller}} is there to coordinate,
to talk to the web client, …

\end{description}

\item {} \begin{description}
\item[{creating some views in the database:}] \leavevmode
\fvset{hllines={, ,}}%
\begin{sphinxVerbatim}[commandchars=\\\{\}]
\PYG{n+nt}{\PYGZlt{}record} \PYG{n+na}{id=}\PYG{l+s}{\PYGZdq{}customer\PYGZus{}map\PYGZus{}view\PYGZdq{}} \PYG{n+na}{model=}\PYG{l+s}{\PYGZdq{}ir.ui.view\PYGZdq{}}\PYG{n+nt}{\PYGZgt{}}
    \PYG{n+nt}{\PYGZlt{}field} \PYG{n+na}{name=}\PYG{l+s}{\PYGZdq{}name\PYGZdq{}}\PYG{n+nt}{\PYGZgt{}}customer.map.view\PYG{n+nt}{\PYGZlt{}/field\PYGZgt{}}
    \PYG{n+nt}{\PYGZlt{}field} \PYG{n+na}{name=}\PYG{l+s}{\PYGZdq{}model\PYGZdq{}}\PYG{n+nt}{\PYGZgt{}}res.partner\PYG{n+nt}{\PYGZlt{}/field\PYGZgt{}}
    \PYG{n+nt}{\PYGZlt{}field} \PYG{n+na}{name=}\PYG{l+s}{\PYGZdq{}arch\PYGZdq{}} \PYG{n+na}{type=}\PYG{l+s}{\PYGZdq{}xml\PYGZdq{}}\PYG{n+nt}{\PYGZgt{}}
        \PYG{n+nt}{\PYGZlt{}map} \PYG{n+na}{latitude=}\PYG{l+s}{\PYGZdq{}partner\PYGZus{}latitude\PYGZdq{}} \PYG{n+na}{longitude=}\PYG{l+s}{\PYGZdq{}partner\PYGZus{}longitude\PYGZdq{}}\PYG{n+nt}{\PYGZgt{}}
            \PYG{n+nt}{\PYGZlt{}field} \PYG{n+na}{name=}\PYG{l+s}{\PYGZdq{}name\PYGZdq{}}\PYG{n+nt}{/\PYGZgt{}}
        \PYG{n+nt}{\PYGZlt{}/map\PYGZgt{}}
    \PYG{n+nt}{\PYGZlt{}/field\PYGZgt{}}
\PYG{n+nt}{\PYGZlt{}/record\PYGZgt{}}
\end{sphinxVerbatim}

\end{description}

\end{itemize}


\subsection{Customizing an existing view}
\label{\detokenize{reference/javascript_cheatsheet:customizing-an-existing-view}}
Assume we need to create a custom version of a generic view.  For example, a
kanban view with some extra \sphinxstyleemphasis{ribbon-like} widget on top (to display some
specific custom information). In that case, this can be done with 3 steps:
extend the kanban view (which also probably mean extending controllers/renderers
and/or models), then registering the view in the view registry, and finally,
using the view in the kanban arch (a specific example is the helpdesk dashboard).
\begin{itemize}
\item {} \begin{description}
\item[{extending a view:}] \leavevmode
Here is what it could look like:

\fvset{hllines={, ,}}%
\begin{sphinxVerbatim}[commandchars=\\\{\}]
\PYG{k+kd}{var} \PYG{n+nx}{HelpdeskDashboardRenderer} \PYG{o}{=} \PYG{n+nx}{KanbanRenderer}\PYG{p}{.}\PYG{n+nx}{extend}\PYG{p}{(}\PYG{p}{\PYGZob{}}
    \PYG{p}{...}
\PYG{p}{\PYGZcb{}}\PYG{p}{)}\PYG{p}{;}

\PYG{k+kd}{var} \PYG{n+nx}{HelpdeskDashboardModel} \PYG{o}{=} \PYG{n+nx}{KanbanModel}\PYG{p}{.}\PYG{n+nx}{extend}\PYG{p}{(}\PYG{p}{\PYGZob{}}
    \PYG{p}{...}
\PYG{p}{\PYGZcb{}}\PYG{p}{)}\PYG{p}{;}

\PYG{k+kd}{var} \PYG{n+nx}{HelpdeskDashboardController} \PYG{o}{=} \PYG{n+nx}{KanbanController}\PYG{p}{.}\PYG{n+nx}{extend}\PYG{p}{(}\PYG{p}{\PYGZob{}}
    \PYG{p}{...}
\PYG{p}{\PYGZcb{}}\PYG{p}{)}\PYG{p}{;}

\PYG{k+kd}{var} \PYG{n+nx}{HelpdeskDashboardView} \PYG{o}{=} \PYG{n+nx}{KanbanView}\PYG{p}{.}\PYG{n+nx}{extend}\PYG{p}{(}\PYG{p}{\PYGZob{}}
    \PYG{n+nx}{config}\PYG{o}{:} \PYG{n+nx}{\PYGZus{}}\PYG{p}{.}\PYG{n+nx}{extend}\PYG{p}{(}\PYG{p}{\PYGZob{}}\PYG{p}{\PYGZcb{}}\PYG{p}{,} \PYG{n+nx}{KanbanView}\PYG{p}{.}\PYG{n+nx}{prototype}\PYG{p}{.}\PYG{n+nx}{config}\PYG{p}{,} \PYG{p}{\PYGZob{}}
        \PYG{n+nx}{Model}\PYG{o}{:} \PYG{n+nx}{HelpdeskDashboardModel}\PYG{p}{,}
        \PYG{n+nx}{Renderer}\PYG{o}{:} \PYG{n+nx}{HelpdeskDashboardRenderer}\PYG{p}{,}
        \PYG{n+nx}{Controller}\PYG{o}{:} \PYG{n+nx}{HelpdeskDashboardController}\PYG{p}{,}
    \PYG{p}{\PYGZcb{}}\PYG{p}{)}\PYG{p}{,}
\PYG{p}{\PYGZcb{}}\PYG{p}{)}\PYG{p}{;}
\end{sphinxVerbatim}

\end{description}

\item {} \begin{description}
\item[{adding it to the view registry:}] \leavevmode
as usual, we need to inform the web client of the mapping between the name
of the views and the actual class.

\fvset{hllines={, ,}}%
\begin{sphinxVerbatim}[commandchars=\\\{\}]
\PYG{k+kd}{var} \PYG{n+nx}{viewRegistry} \PYG{o}{=} \PYG{n+nx}{require}\PYG{p}{(}\PYG{l+s+s1}{\PYGZsq{}web.view\PYGZus{}registry\PYGZsq{}}\PYG{p}{)}\PYG{p}{;}
\PYG{n+nx}{viewRegistry}\PYG{p}{.}\PYG{n+nx}{add}\PYG{p}{(}\PYG{l+s+s1}{\PYGZsq{}helpdesk\PYGZus{}dashboard\PYGZsq{}}\PYG{p}{,} \PYG{n+nx}{HelpdeskDashboardView}\PYG{p}{)}\PYG{p}{;}
\end{sphinxVerbatim}

\end{description}

\item {} \begin{description}
\item[{using it in an actual view:}] \leavevmode
we now need to inform the web client that a specific \sphinxcode{\sphinxupquote{ir.ui.view}} needs to
use our new class.  Note that this is a web client specific concern.  From
the point of view of the server, we still have a kanban view.  The proper
way to do this is by using a special attribute \sphinxcode{\sphinxupquote{js\_class}} (which will be
renamed someday into \sphinxcode{\sphinxupquote{widget}}, because this is really not a good name) on
the root node of the arch:

\fvset{hllines={, ,}}%
\begin{sphinxVerbatim}[commandchars=\\\{\}]
\PYG{n+nt}{\PYGZlt{}record} \PYG{n+na}{id=}\PYG{l+s}{\PYGZdq{}helpdesk\PYGZus{}team\PYGZus{}view\PYGZus{}kanban\PYGZdq{}} \PYG{n+na}{model=}\PYG{l+s}{\PYGZdq{}ir.ui.view\PYGZdq{}} \PYG{n+nt}{\PYGZgt{}}
    ...
    \PYG{n+nt}{\PYGZlt{}field} \PYG{n+na}{name=}\PYG{l+s}{\PYGZdq{}arch\PYGZdq{}} \PYG{n+na}{type=}\PYG{l+s}{\PYGZdq{}xml\PYGZdq{}}\PYG{n+nt}{\PYGZgt{}}
        \PYG{n+nt}{\PYGZlt{}kanban} \PYG{n+na}{js\PYGZus{}class=}\PYG{l+s}{\PYGZdq{}helpdesk\PYGZus{}dashboard\PYGZdq{}}\PYG{n+nt}{\PYGZgt{}}
            ...
        \PYG{n+nt}{\PYGZlt{}/kanban\PYGZgt{}}
    \PYG{n+nt}{\PYGZlt{}/field\PYGZgt{}}
\PYG{n+nt}{\PYGZlt{}/record\PYGZgt{}}
\end{sphinxVerbatim}

\end{description}

\end{itemize}

\begin{sphinxadmonition}{note}{Note:}
Note: you can change the way the view interprets the arch structure.  However,
from the server point of view, this is still a view of the same base type,
subjected to the same rules (rng validation, for example).  So, your views still
need to have a valid arch field.
\end{sphinxadmonition}


\section{Javascript Reference}
\label{\detokenize{reference/javascript_reference:javascript-reference}}\label{\detokenize{reference/javascript_reference::doc}}
This document presents the Odoo Javascript framework. This
framework is not a large application in term of lines of code, but it is quite
generic, because it is basically a machine to turn a declarative interface
description into a live application, able to interact with every model and
records in the database.  It is even possible to use the web client to modify
the interface of the web client.

\begin{sphinxadmonition}{note}{Note:}
An html version of all docstrings in Odoo is available at:


\subsection{Javascript API}
\label{\detokenize{reference/javascript_api:javascript-api}}\label{\detokenize{reference/javascript_api::doc}}\label{\detokenize{reference/javascript_api:module-web.crash_manager}}

\begin{fulllineitems}
\phantomsection\label{\detokenize{reference/javascript_api:web.crash_manager}}\pysigline{\sphinxbfcode{\sphinxupquote{module }}\sphinxbfcode{\sphinxupquote{web.crash\_manager}}}~~\begin{quote}\begin{description}
\item[{Exports}] \leavevmode{\hyperref[\detokenize{reference/javascript_api:web.crash_manager.}]{\sphinxcrossref{
\textless{}anonymous\textgreater{}
}}}
\item[{Depends On}] \leavevmode\begin{itemize}
\item {} {\hyperref[\detokenize{reference/javascript_api:web.CrashManager}]{\sphinxcrossref{
web.CrashManager
}}}
\end{itemize}

\end{description}\end{quote}


\begin{fulllineitems}
\phantomsection\label{\detokenize{reference/javascript_api:web.crash_manager.}}\pysigline{\sphinxbfcode{\sphinxupquote{object }}\sphinxbfcode{\sphinxupquote{}}\sphinxbfcode{\sphinxupquote{ instance of }}{\hyperref[\detokenize{reference/javascript_api:web.CrashManager.CrashManager}]{\sphinxcrossref{CrashManager}}}}
\end{fulllineitems}


\end{fulllineitems}

\phantomsection\label{\detokenize{reference/javascript_api:module-web_editor.ready}}

\begin{fulllineitems}
\phantomsection\label{\detokenize{reference/javascript_api:web_editor.ready}}\pysigline{\sphinxbfcode{\sphinxupquote{module }}\sphinxbfcode{\sphinxupquote{web\_editor.ready}}}~~\begin{quote}\begin{description}
\item[{Exports}] \leavevmode{\hyperref[\detokenize{reference/javascript_api:web_editor.ready.}]{\sphinxcrossref{
\textless{}anonymous\textgreater{}
}}}
\item[{Depends On}] \leavevmode\begin{itemize}
\item {} {\hyperref[\detokenize{reference/javascript_api:web_editor.base}]{\sphinxcrossref{
web\_editor.base
}}}
\end{itemize}

\end{description}\end{quote}


\begin{fulllineitems}
\phantomsection\label{\detokenize{reference/javascript_api:web_editor.ready.}}\pysigline{\sphinxbfcode{\sphinxupquote{namespace }}\sphinxbfcode{\sphinxupquote{}}}
\end{fulllineitems}


\end{fulllineitems}

\phantomsection\label{\detokenize{reference/javascript_api:module-web.DebugManager}}

\begin{fulllineitems}
\phantomsection\label{\detokenize{reference/javascript_api:web.DebugManager}}\pysigline{\sphinxbfcode{\sphinxupquote{module }}\sphinxbfcode{\sphinxupquote{web.DebugManager}}}~~\begin{quote}\begin{description}
\item[{Exports}] \leavevmode{\hyperref[\detokenize{reference/javascript_api:web.DebugManager.DebugManager}]{\sphinxcrossref{
DebugManager
}}}
\item[{Depends On}] \leavevmode\begin{itemize}
\item {} {\hyperref[\detokenize{reference/javascript_api:web.ActionManager}]{\sphinxcrossref{
web.ActionManager
}}}
\item {} {\hyperref[\detokenize{reference/javascript_api:web.Dialog}]{\sphinxcrossref{
web.Dialog
}}}
\item {} {\hyperref[\detokenize{reference/javascript_api:web.SystrayMenu}]{\sphinxcrossref{
web.SystrayMenu
}}}
\item {} {\hyperref[\detokenize{reference/javascript_api:web.ViewManager}]{\sphinxcrossref{
web.ViewManager
}}}
\item {} {\hyperref[\detokenize{reference/javascript_api:web.WebClient}]{\sphinxcrossref{
web.WebClient
}}}
\item {} {\hyperref[\detokenize{reference/javascript_api:web.Widget}]{\sphinxcrossref{
web.Widget
}}}
\item {} {\hyperref[\detokenize{reference/javascript_api:web.core}]{\sphinxcrossref{
web.core
}}}
\item {} {\hyperref[\detokenize{reference/javascript_api:web.field_utils}]{\sphinxcrossref{
web.field\_utils
}}}
\item {} {\hyperref[\detokenize{reference/javascript_api:web.session}]{\sphinxcrossref{
web.session
}}}
\item {} {\hyperref[\detokenize{reference/javascript_api:web.utils}]{\sphinxcrossref{
web.utils
}}}
\item {} {\hyperref[\detokenize{reference/javascript_api:web.view_dialogs}]{\sphinxcrossref{
web.view\_dialogs
}}}
\end{itemize}

\end{description}\end{quote}


\begin{fulllineitems}
\phantomsection\label{\detokenize{reference/javascript_api:DebugManager}}\pysiglinewithargsret{\sphinxbfcode{\sphinxupquote{class }}\sphinxbfcode{\sphinxupquote{DebugManager}}}{}{}~\begin{quote}\begin{description}
\item[{Extends}] \leavevmode{\hyperref[\detokenize{reference/javascript_api:web.Widget.Widget}]{\sphinxcrossref{
Widget
}}}
\end{description}\end{quote}

DebugManager base + general features (applicable to any context)


\begin{fulllineitems}
\phantomsection\label{\detokenize{reference/javascript_api:perform_callback}}\pysiglinewithargsret{\sphinxbfcode{\sphinxupquote{method }}\sphinxbfcode{\sphinxupquote{perform\_callback}}}{\emph{evt}}{}
Calls the appropriate callback when clicking on a Debug option
\begin{quote}\begin{description}
\item[{Parameters}] \leavevmode\begin{itemize}

\sphinxstylestrong{evt}
\end{itemize}

\end{description}\end{quote}

\end{fulllineitems}



\begin{fulllineitems}
\phantomsection\label{\detokenize{reference/javascript_api:perform_js_tests}}\pysiglinewithargsret{\sphinxbfcode{\sphinxupquote{method }}\sphinxbfcode{\sphinxupquote{perform\_js\_tests}}}{}{}
Runs the JS (desktop) tests

\end{fulllineitems}



\begin{fulllineitems}
\phantomsection\label{\detokenize{reference/javascript_api:perform_js_mobile_tests}}\pysiglinewithargsret{\sphinxbfcode{\sphinxupquote{method }}\sphinxbfcode{\sphinxupquote{perform\_js\_mobile\_tests}}}{}{}
Runs the JS mobile tests

\end{fulllineitems}


\end{fulllineitems}



\begin{fulllineitems}
\phantomsection\label{\detokenize{reference/javascript_api:DebugManager}}\pysiglinewithargsret{\sphinxbfcode{\sphinxupquote{class }}\sphinxbfcode{\sphinxupquote{DebugManager}}}{}{}~\begin{quote}\begin{description}
\item[{Extends}] \leavevmode{\hyperref[\detokenize{reference/javascript_api:web.Widget.Widget}]{\sphinxcrossref{
Widget
}}}
\end{description}\end{quote}

DebugManager base + general features (applicable to any context)


\begin{fulllineitems}
\phantomsection\label{\detokenize{reference/javascript_api:perform_callback}}\pysiglinewithargsret{\sphinxbfcode{\sphinxupquote{method }}\sphinxbfcode{\sphinxupquote{perform\_callback}}}{\emph{evt}}{}
Calls the appropriate callback when clicking on a Debug option
\begin{quote}\begin{description}
\item[{Parameters}] \leavevmode\begin{itemize}

\sphinxstylestrong{evt}
\end{itemize}

\end{description}\end{quote}

\end{fulllineitems}



\begin{fulllineitems}
\phantomsection\label{\detokenize{reference/javascript_api:perform_js_tests}}\pysiglinewithargsret{\sphinxbfcode{\sphinxupquote{method }}\sphinxbfcode{\sphinxupquote{perform\_js\_tests}}}{}{}
Runs the JS (desktop) tests

\end{fulllineitems}



\begin{fulllineitems}
\phantomsection\label{\detokenize{reference/javascript_api:perform_js_mobile_tests}}\pysiglinewithargsret{\sphinxbfcode{\sphinxupquote{method }}\sphinxbfcode{\sphinxupquote{perform\_js\_mobile\_tests}}}{}{}
Runs the JS mobile tests

\end{fulllineitems}


\end{fulllineitems}


\end{fulllineitems}

\phantomsection\label{\detokenize{reference/javascript_api:module-web.translation}}

\begin{fulllineitems}
\phantomsection\label{\detokenize{reference/javascript_api:web.translation}}\pysigline{\sphinxbfcode{\sphinxupquote{module }}\sphinxbfcode{\sphinxupquote{web.translation}}}~~\begin{quote}\begin{description}
\item[{Exports}] \leavevmode{\hyperref[\detokenize{reference/javascript_api:web.translation.}]{\sphinxcrossref{
\textless{}anonymous\textgreater{}
}}}
\item[{Depends On}] \leavevmode\begin{itemize}
\item {} {\hyperref[\detokenize{reference/javascript_api:web.Class}]{\sphinxcrossref{
web.Class
}}}
\end{itemize}

\end{description}\end{quote}


\begin{fulllineitems}
\phantomsection\label{\detokenize{reference/javascript_api:_t}}\pysiglinewithargsret{\sphinxbfcode{\sphinxupquote{function }}\sphinxbfcode{\sphinxupquote{\_t}}}{\emph{source}}{{ $\rightarrow$ String}}
Eager translation function, performs translation immediately at call
site. Beware using this outside of method bodies (before the
translation database is loaded), you probably want {\hyperref[\detokenize{reference/javascript_api:_lt}]{\sphinxcrossref{\sphinxcode{\sphinxupquote{\_lt()}}}}}
instead.
\begin{quote}\begin{description}
\item[{Parameters}] \leavevmode\begin{itemize}

\sphinxstylestrong{source} (\sphinxstyleliteralemphasis{\sphinxupquote{String}}) \textendash{} string to translate
\end{itemize}

\item[{Returns}] \leavevmode
source translated into the current locale

\item[{Return Type}] \leavevmode
\sphinxstyleliteralemphasis{\sphinxupquote{String}}

\end{description}\end{quote}

\end{fulllineitems}



\begin{fulllineitems}
\phantomsection\label{\detokenize{reference/javascript_api:_lt}}\pysiglinewithargsret{\sphinxbfcode{\sphinxupquote{function }}\sphinxbfcode{\sphinxupquote{\_lt}}}{\emph{s}}{{ $\rightarrow$ Object}}
Lazy translation function, only performs the translation when actually
printed (e.g. inserted into a template)

Useful when defining translatable strings in code evaluated before the
translation database is loaded, as class attributes or at the top-level of
an OpenERP Web module
\begin{quote}\begin{description}
\item[{Parameters}] \leavevmode\begin{itemize}

\sphinxstylestrong{s} (\sphinxstyleliteralemphasis{\sphinxupquote{String}}) \textendash{} string to translate
\end{itemize}

\item[{Returns}] \leavevmode
lazy translation object

\item[{Return Type}] \leavevmode
\sphinxstyleliteralemphasis{\sphinxupquote{Object}}

\end{description}\end{quote}

\end{fulllineitems}



\begin{fulllineitems}
\phantomsection\label{\detokenize{reference/javascript_api:web.translation.}}\pysigline{\sphinxbfcode{\sphinxupquote{namespace }}\sphinxbfcode{\sphinxupquote{}}}~

\begin{fulllineitems}
\phantomsection\label{\detokenize{reference/javascript_api:_t}}\pysiglinewithargsret{\sphinxbfcode{\sphinxupquote{function }}\sphinxbfcode{\sphinxupquote{\_t}}}{\emph{source}}{{ $\rightarrow$ String}}
Eager translation function, performs translation immediately at call
site. Beware using this outside of method bodies (before the
translation database is loaded), you probably want {\hyperref[\detokenize{reference/javascript_api:_lt}]{\sphinxcrossref{\sphinxcode{\sphinxupquote{\_lt()}}}}}
instead.
\begin{quote}\begin{description}
\item[{Parameters}] \leavevmode\begin{itemize}

\sphinxstylestrong{source} (\sphinxstyleliteralemphasis{\sphinxupquote{String}}) \textendash{} string to translate
\end{itemize}

\item[{Returns}] \leavevmode
source translated into the current locale

\item[{Return Type}] \leavevmode
\sphinxstyleliteralemphasis{\sphinxupquote{String}}

\end{description}\end{quote}

\end{fulllineitems}



\begin{fulllineitems}
\phantomsection\label{\detokenize{reference/javascript_api:_lt}}\pysiglinewithargsret{\sphinxbfcode{\sphinxupquote{function }}\sphinxbfcode{\sphinxupquote{\_lt}}}{\emph{s}}{{ $\rightarrow$ Object}}
Lazy translation function, only performs the translation when actually
printed (e.g. inserted into a template)

Useful when defining translatable strings in code evaluated before the
translation database is loaded, as class attributes or at the top-level of
an OpenERP Web module
\begin{quote}\begin{description}
\item[{Parameters}] \leavevmode\begin{itemize}

\sphinxstylestrong{s} (\sphinxstyleliteralemphasis{\sphinxupquote{String}}) \textendash{} string to translate
\end{itemize}

\item[{Returns}] \leavevmode
lazy translation object

\item[{Return Type}] \leavevmode
\sphinxstyleliteralemphasis{\sphinxupquote{Object}}

\end{description}\end{quote}

\end{fulllineitems}


\end{fulllineitems}


\end{fulllineitems}

\phantomsection\label{\detokenize{reference/javascript_api:module-survey.result}}

\begin{fulllineitems}
\phantomsection\label{\detokenize{reference/javascript_api:survey.result}}\pysigline{\sphinxbfcode{\sphinxupquote{module }}\sphinxbfcode{\sphinxupquote{survey.result}}}~~\begin{quote}\begin{description}
\item[{Exports}] \leavevmode{\hyperref[\detokenize{reference/javascript_api:survey.result.}]{\sphinxcrossref{
\textless{}anonymous\textgreater{}
}}}
\end{description}\end{quote}


\begin{fulllineitems}
\phantomsection\label{\detokenize{reference/javascript_api:survey.result.}}\pysigline{\sphinxbfcode{\sphinxupquote{namespace }}\sphinxbfcode{\sphinxupquote{}}}
\end{fulllineitems}


\end{fulllineitems}

\phantomsection\label{\detokenize{reference/javascript_api:module-web.AbstractService}}

\begin{fulllineitems}
\phantomsection\label{\detokenize{reference/javascript_api:web.AbstractService}}\pysigline{\sphinxbfcode{\sphinxupquote{module }}\sphinxbfcode{\sphinxupquote{web.AbstractService}}}~~\begin{quote}\begin{description}
\item[{Exports}] \leavevmode{\hyperref[\detokenize{reference/javascript_api:web.AbstractService.AbstractService}]{\sphinxcrossref{
AbstractService
}}}
\item[{Depends On}] \leavevmode\begin{itemize}
\item {} {\hyperref[\detokenize{reference/javascript_api:web.Class}]{\sphinxcrossref{
web.Class
}}}
\end{itemize}

\end{description}\end{quote}


\begin{fulllineitems}
\phantomsection\label{\detokenize{reference/javascript_api:AbstractService}}\pysiglinewithargsret{\sphinxbfcode{\sphinxupquote{class }}\sphinxbfcode{\sphinxupquote{AbstractService}}}{}{}~\begin{quote}\begin{description}
\item[{Extends}] \leavevmode{\hyperref[\detokenize{reference/javascript_api:web.Class.Class}]{\sphinxcrossref{
Class
}}}
\end{description}\end{quote}

\end{fulllineitems}



\begin{fulllineitems}
\pysiglinewithargsret{\sphinxbfcode{\sphinxupquote{function }}\sphinxbfcode{\sphinxupquote{}}}{}{}
Create subclass for AbstractService

\end{fulllineitems}


\end{fulllineitems}

\phantomsection\label{\detokenize{reference/javascript_api:module-web.AbstractField}}

\begin{fulllineitems}
\phantomsection\label{\detokenize{reference/javascript_api:web.AbstractField}}\pysigline{\sphinxbfcode{\sphinxupquote{module }}\sphinxbfcode{\sphinxupquote{web.AbstractField}}}~~\begin{quote}\begin{description}
\item[{Exports}] \leavevmode{\hyperref[\detokenize{reference/javascript_api:web.AbstractField.AbstractField}]{\sphinxcrossref{
AbstractField
}}}
\item[{Depends On}] \leavevmode\begin{itemize}
\item {} {\hyperref[\detokenize{reference/javascript_api:web.Widget}]{\sphinxcrossref{
web.Widget
}}}
\item {} {\hyperref[\detokenize{reference/javascript_api:web.ajax}]{\sphinxcrossref{
web.ajax
}}}
\item {} {\hyperref[\detokenize{reference/javascript_api:web.field_utils}]{\sphinxcrossref{
web.field\_utils
}}}
\end{itemize}

\end{description}\end{quote}


\begin{fulllineitems}
\phantomsection\label{\detokenize{reference/javascript_api:AbstractField}}\pysiglinewithargsret{\sphinxbfcode{\sphinxupquote{class }}\sphinxbfcode{\sphinxupquote{AbstractField}}}{\emph{parent}, \emph{name}, \emph{record}\sphinxoptional{, \emph{options}}}{}~\begin{quote}\begin{description}
\item[{Extends}] \leavevmode{\hyperref[\detokenize{reference/javascript_api:web.Widget.Widget}]{\sphinxcrossref{
Widget
}}}
\item[{Parameters}] \leavevmode\begin{itemize}

\sphinxstylestrong{parent} ({\hyperref[\detokenize{reference/javascript_api:Widget}]{\sphinxcrossref{\sphinxstyleliteralemphasis{\sphinxupquote{Widget}}}}})

\sphinxstylestrong{name} (\sphinxstyleliteralemphasis{\sphinxupquote{string}}) \textendash{} The field name defined in the model

\sphinxstylestrong{record} (\sphinxstyleliteralemphasis{\sphinxupquote{Object}}) \textendash{} A record object (result of the get method of
  a basic model)

\sphinxstylestrong{options} ({\hyperref[\detokenize{reference/javascript_api:web.AbstractField.AbstractFieldOptions}]{\sphinxcrossref{\sphinxstyleliteralemphasis{\sphinxupquote{AbstractFieldOptions}}}}})
\end{itemize}

\end{description}\end{quote}


\begin{fulllineitems}
\phantomsection\label{\detokenize{reference/javascript_api:fieldDependencies}}\pysigline{\sphinxbfcode{\sphinxupquote{namespace }}\sphinxbfcode{\sphinxupquote{fieldDependencies}}}
An object representing fields to be fetched by the model eventhough not present in the view
This object contains “field name” as key and an object as value.
That value object must contain the key “type”
see FieldBinaryImage for an example.

\end{fulllineitems}



\begin{fulllineitems}
\phantomsection\label{\detokenize{reference/javascript_api:resetOnAnyFieldChange}}\pysigline{\sphinxbfcode{\sphinxupquote{attribute }}\sphinxbfcode{\sphinxupquote{resetOnAnyFieldChange}} Boolean}
If this flag is set to true, the field widget will be reset on every
change which is made in the view (if the view supports it). This is
currently a form view feature.

\end{fulllineitems}



\begin{fulllineitems}
\phantomsection\label{\detokenize{reference/javascript_api:specialData}}\pysigline{\sphinxbfcode{\sphinxupquote{attribute }}\sphinxbfcode{\sphinxupquote{specialData}} Boolean}
If this flag is given a string, the related BasicModel will be used to
initialize specialData the field might need. This data will be available
through this.record.specialData{[}this.name{]}.

\end{fulllineitems}



\begin{fulllineitems}
\phantomsection\label{\detokenize{reference/javascript_api:supportedFieldTypes}}\pysigline{\sphinxbfcode{\sphinxupquote{attribute }}\sphinxbfcode{\sphinxupquote{supportedFieldTypes}} Array\textless{}String\textgreater{}}
to override to indicate which field types are supported by the widget

\end{fulllineitems}



\begin{fulllineitems}
\phantomsection\label{\detokenize{reference/javascript_api:start}}\pysiglinewithargsret{\sphinxbfcode{\sphinxupquote{method }}\sphinxbfcode{\sphinxupquote{start}}}{}{{ $\rightarrow$ Deferred}}
When a field widget is appended to the DOM, its start method is called,
and will automatically call render. Most widgets should not override this.
\begin{quote}\begin{description}
\item[{Return Type}] \leavevmode
\sphinxstyleliteralemphasis{\sphinxupquote{Deferred}}

\end{description}\end{quote}

\end{fulllineitems}



\begin{fulllineitems}
\phantomsection\label{\detokenize{reference/javascript_api:willStart}}\pysiglinewithargsret{\sphinxbfcode{\sphinxupquote{method }}\sphinxbfcode{\sphinxupquote{willStart}}}{}{}
Loads the libraries listed in this.jsLibs and this.cssLibs

\end{fulllineitems}



\begin{fulllineitems}
\phantomsection\label{\detokenize{reference/javascript_api:activate}}\pysiglinewithargsret{\sphinxbfcode{\sphinxupquote{method }}\sphinxbfcode{\sphinxupquote{activate}}}{\sphinxoptional{\emph{options}}}{{ $\rightarrow$ boolean}}
Activates the field widget. By default, activation means focusing and
selecting (if possible) the associated focusable element. The selecting
part can be disabled.  In that case, note that the focused input/textarea
will have the cursor at the very end.
\begin{quote}\begin{description}
\item[{Parameters}] \leavevmode\begin{itemize}

\sphinxstylestrong{options} ({\hyperref[\detokenize{reference/javascript_api:web.AbstractField.ActivateOptions}]{\sphinxcrossref{\sphinxstyleliteralemphasis{\sphinxupquote{ActivateOptions}}}}})
\end{itemize}

\item[{Returns}] \leavevmode
true if the widget was activated, false if the
                   focusable element was not found or invisible

\item[{Return Type}] \leavevmode
\sphinxstyleliteralemphasis{\sphinxupquote{boolean}}

\end{description}\end{quote}


\begin{fulllineitems}
\phantomsection\label{\detokenize{reference/javascript_api:ActivateOptions}}\pysiglinewithargsret{\sphinxbfcode{\sphinxupquote{class }}\sphinxbfcode{\sphinxupquote{ActivateOptions}}}{}{}~

\begin{fulllineitems}
\phantomsection\label{\detokenize{reference/javascript_api:noselect}}\pysigline{\sphinxbfcode{\sphinxupquote{attribute }}\sphinxbfcode{\sphinxupquote{noselect}} boolean}~\begin{description}
\item[{if false and the input}] \leavevmode
is of type text or textarea, the content will also be selected

\end{description}

\end{fulllineitems}



\begin{fulllineitems}
\phantomsection\label{\detokenize{reference/javascript_api:event}}\pysigline{\sphinxbfcode{\sphinxupquote{attribute }}\sphinxbfcode{\sphinxupquote{event}} Event}
the event which fired this activation

\end{fulllineitems}


\end{fulllineitems}


\end{fulllineitems}



\begin{fulllineitems}
\phantomsection\label{\detokenize{reference/javascript_api:commitChanges}}\pysiglinewithargsret{\sphinxbfcode{\sphinxupquote{function }}\sphinxbfcode{\sphinxupquote{commitChanges}}}{}{{ $\rightarrow$ Deferred\textbar{}undefined}}
This function should be implemented by widgets that are not able to
notify their environment when their value changes (maybe because their
are not aware of the changes) or that may have a value in a temporary
state (maybe because some action should be performed to validate it
before notifying it). This is typically called before trying to save the
widget’s value, so it should call \_setValue() to notify the environment
if the value changed but was not notified.
\begin{quote}\begin{description}
\item[{Return Type}] \leavevmode
\sphinxstyleliteralemphasis{\sphinxupquote{Deferred}}\sphinxstyleemphasis{ or }undefined

\end{description}\end{quote}

\end{fulllineitems}



\begin{fulllineitems}
\phantomsection\label{\detokenize{reference/javascript_api:getFocusableElement}}\pysiglinewithargsret{\sphinxbfcode{\sphinxupquote{function }}\sphinxbfcode{\sphinxupquote{getFocusableElement}}}{}{{ $\rightarrow$ jQuery}}
Returns the main field’s DOM element (jQuery form) which can be focused
by the browser.
\begin{quote}\begin{description}
\item[{Returns}] \leavevmode
main focusable element inside the widget

\item[{Return Type}] \leavevmode
\sphinxstyleliteralemphasis{\sphinxupquote{jQuery}}

\end{description}\end{quote}

\end{fulllineitems}



\begin{fulllineitems}
\phantomsection\label{\detokenize{reference/javascript_api:isFocusable}}\pysiglinewithargsret{\sphinxbfcode{\sphinxupquote{function }}\sphinxbfcode{\sphinxupquote{isFocusable}}}{}{{ $\rightarrow$ boolean}}
Returns true iff the widget has a visible element that can take the focus
\begin{quote}\begin{description}
\item[{Return Type}] \leavevmode
\sphinxstyleliteralemphasis{\sphinxupquote{boolean}}

\end{description}\end{quote}

\end{fulllineitems}



\begin{fulllineitems}
\phantomsection\label{\detokenize{reference/javascript_api:isSet}}\pysiglinewithargsret{\sphinxbfcode{\sphinxupquote{function }}\sphinxbfcode{\sphinxupquote{isSet}}}{}{{ $\rightarrow$ boolean}}
this method is used to determine if the field value is set to a meaningful
value.  This is useful to determine if a field should be displayed as empty
\begin{quote}\begin{description}
\item[{Return Type}] \leavevmode
\sphinxstyleliteralemphasis{\sphinxupquote{boolean}}

\end{description}\end{quote}

\end{fulllineitems}



\begin{fulllineitems}
\phantomsection\label{\detokenize{reference/javascript_api:isValid}}\pysiglinewithargsret{\sphinxbfcode{\sphinxupquote{function }}\sphinxbfcode{\sphinxupquote{isValid}}}{}{{ $\rightarrow$ boolean}}
A field widget is valid if it was checked as valid the last time its
value was changed by the user. This is checked before saving a record, by
the view.

Note: this is the responsability of the view to check that required
fields have a set value.
\begin{quote}\begin{description}
\item[{Returns}] \leavevmode
true/false if the widget is valid

\item[{Return Type}] \leavevmode
\sphinxstyleliteralemphasis{\sphinxupquote{boolean}}

\end{description}\end{quote}

\end{fulllineitems}



\begin{fulllineitems}
\phantomsection\label{\detokenize{reference/javascript_api:reset}}\pysiglinewithargsret{\sphinxbfcode{\sphinxupquote{function }}\sphinxbfcode{\sphinxupquote{reset}}}{\emph{record}\sphinxoptional{, \emph{event}}}{{ $\rightarrow$ Deferred}}
this method is supposed to be called from the outside of field widgets.
The typical use case is when an onchange has changed the widget value.
It will reset the widget to the values that could have changed, then will
rerender the widget.
\begin{quote}\begin{description}
\item[{Parameters}] \leavevmode\begin{itemize}

\sphinxstylestrong{record} (\sphinxstyleliteralemphasis{\sphinxupquote{any}})

\sphinxstylestrong{event} (\sphinxstyleliteralemphasis{\sphinxupquote{OdooEvent}}) \textendash{} an event that triggered the reset action. It
  is optional, and may be used by a widget to share information from the
  moment a field change event is triggered to the moment a reset
  operation is applied.
\end{itemize}

\item[{Returns}] \leavevmode
A Deferred, which resolves when the widget rendering
  is complete

\item[{Return Type}] \leavevmode
\sphinxstyleliteralemphasis{\sphinxupquote{Deferred}}

\end{description}\end{quote}

\end{fulllineitems}



\begin{fulllineitems}
\phantomsection\label{\detokenize{reference/javascript_api:AbstractFieldOptions}}\pysiglinewithargsret{\sphinxbfcode{\sphinxupquote{class }}\sphinxbfcode{\sphinxupquote{AbstractFieldOptions}}}{}{}~

\begin{fulllineitems}
\phantomsection\label{\detokenize{reference/javascript_api:mode}}\pysigline{\sphinxbfcode{\sphinxupquote{attribute }}\sphinxbfcode{\sphinxupquote{mode}} string}
should be ‘readonly’ or ‘edit’

\end{fulllineitems}


\end{fulllineitems}


\end{fulllineitems}


\end{fulllineitems}

\phantomsection\label{\detokenize{reference/javascript_api:module-hr_attendance.greeting_message}}

\begin{fulllineitems}
\phantomsection\label{\detokenize{reference/javascript_api:hr_attendance.greeting_message}}\pysigline{\sphinxbfcode{\sphinxupquote{module }}\sphinxbfcode{\sphinxupquote{hr\_attendance.greeting\_message}}}~~\begin{quote}\begin{description}
\item[{Exports}] \leavevmode{\hyperref[\detokenize{reference/javascript_api:hr_attendance.greeting_message.GreetingMessage}]{\sphinxcrossref{
GreetingMessage
}}}
\item[{Depends On}] \leavevmode\begin{itemize}
\item {} {\hyperref[\detokenize{reference/javascript_api:web.Widget}]{\sphinxcrossref{
web.Widget
}}}
\item {} {\hyperref[\detokenize{reference/javascript_api:web.core}]{\sphinxcrossref{
web.core
}}}
\end{itemize}

\end{description}\end{quote}


\begin{fulllineitems}
\phantomsection\label{\detokenize{reference/javascript_api:GreetingMessage}}\pysiglinewithargsret{\sphinxbfcode{\sphinxupquote{class }}\sphinxbfcode{\sphinxupquote{GreetingMessage}}}{\emph{parent}, \emph{action}}{}~\begin{quote}\begin{description}
\item[{Extends}] \leavevmode{\hyperref[\detokenize{reference/javascript_api:web.Widget.Widget}]{\sphinxcrossref{
Widget
}}}
\item[{Parameters}] \leavevmode\begin{itemize}

\sphinxstylestrong{parent}

\sphinxstylestrong{action}
\end{itemize}

\end{description}\end{quote}

\end{fulllineitems}


\end{fulllineitems}

\phantomsection\label{\detokenize{reference/javascript_api:module-website_sale_wishlist.wishlist}}

\begin{fulllineitems}
\phantomsection\label{\detokenize{reference/javascript_api:website_sale_wishlist.wishlist}}\pysigline{\sphinxbfcode{\sphinxupquote{module }}\sphinxbfcode{\sphinxupquote{website\_sale\_wishlist.wishlist}}}~~\begin{quote}\begin{description}
\item[{Exports}] \leavevmode{\hyperref[\detokenize{reference/javascript_api:website_sale_wishlist.wishlist.}]{\sphinxcrossref{
\textless{}anonymous\textgreater{}
}}}
\item[{Depends On}] \leavevmode\begin{itemize}
\item {} {\hyperref[\detokenize{reference/javascript_api:web.Widget}]{\sphinxcrossref{
web.Widget
}}}
\item {} {\hyperref[\detokenize{reference/javascript_api:web.ajax}]{\sphinxcrossref{
web.ajax
}}}
\item {} {\hyperref[\detokenize{reference/javascript_api:web_editor.base}]{\sphinxcrossref{
web\_editor.base
}}}
\item {} {\hyperref[\detokenize{reference/javascript_api:website_sale.utils}]{\sphinxcrossref{
website\_sale.utils
}}}
\end{itemize}

\end{description}\end{quote}


\begin{fulllineitems}
\phantomsection\label{\detokenize{reference/javascript_api:website_sale_wishlist.wishlist.}}\pysigline{\sphinxbfcode{\sphinxupquote{namespace }}\sphinxbfcode{\sphinxupquote{}}}
\end{fulllineitems}


\end{fulllineitems}

\phantomsection\label{\detokenize{reference/javascript_api:module-web.DataExport}}

\begin{fulllineitems}
\phantomsection\label{\detokenize{reference/javascript_api:web.DataExport}}\pysigline{\sphinxbfcode{\sphinxupquote{module }}\sphinxbfcode{\sphinxupquote{web.DataExport}}}~~\begin{quote}\begin{description}
\item[{Exports}] \leavevmode{\hyperref[\detokenize{reference/javascript_api:web.DataExport.DataExport}]{\sphinxcrossref{
DataExport
}}}
\item[{Depends On}] \leavevmode\begin{itemize}
\item {} {\hyperref[\detokenize{reference/javascript_api:web.Dialog}]{\sphinxcrossref{
web.Dialog
}}}
\item {} {\hyperref[\detokenize{reference/javascript_api:web.core}]{\sphinxcrossref{
web.core
}}}
\item {} {\hyperref[\detokenize{reference/javascript_api:web.crash_manager}]{\sphinxcrossref{
web.crash\_manager
}}}
\item {} {\hyperref[\detokenize{reference/javascript_api:web.data}]{\sphinxcrossref{
web.data
}}}
\item {} {\hyperref[\detokenize{reference/javascript_api:web.framework}]{\sphinxcrossref{
web.framework
}}}
\item {} {\hyperref[\detokenize{reference/javascript_api:web.pyeval}]{\sphinxcrossref{
web.pyeval
}}}
\end{itemize}

\end{description}\end{quote}


\begin{fulllineitems}
\phantomsection\label{\detokenize{reference/javascript_api:DataExport}}\pysiglinewithargsret{\sphinxbfcode{\sphinxupquote{class }}\sphinxbfcode{\sphinxupquote{DataExport}}}{\emph{parent}, \emph{record}}{}~\begin{quote}\begin{description}
\item[{Extends}] \leavevmode{\hyperref[\detokenize{reference/javascript_api:web.Dialog.Dialog}]{\sphinxcrossref{
Dialog
}}}
\item[{Parameters}] \leavevmode\begin{itemize}

\sphinxstylestrong{parent}

\sphinxstylestrong{record}
\end{itemize}

\end{description}\end{quote}

\end{fulllineitems}


\end{fulllineitems}

\phantomsection\label{\detokenize{reference/javascript_api:module-web.SwitchCompanyMenu}}

\begin{fulllineitems}
\phantomsection\label{\detokenize{reference/javascript_api:web.SwitchCompanyMenu}}\pysigline{\sphinxbfcode{\sphinxupquote{module }}\sphinxbfcode{\sphinxupquote{web.SwitchCompanyMenu}}}~~\begin{quote}\begin{description}
\item[{Exports}] \leavevmode{\hyperref[\detokenize{reference/javascript_api:web.SwitchCompanyMenu.SwitchCompanyMenu}]{\sphinxcrossref{
SwitchCompanyMenu
}}}
\item[{Depends On}] \leavevmode\begin{itemize}
\item {} {\hyperref[\detokenize{reference/javascript_api:web.SystrayMenu}]{\sphinxcrossref{
web.SystrayMenu
}}}
\item {} {\hyperref[\detokenize{reference/javascript_api:web.Widget}]{\sphinxcrossref{
web.Widget
}}}
\item {} {\hyperref[\detokenize{reference/javascript_api:web.config}]{\sphinxcrossref{
web.config
}}}
\item {} {\hyperref[\detokenize{reference/javascript_api:web.core}]{\sphinxcrossref{
web.core
}}}
\item {} {\hyperref[\detokenize{reference/javascript_api:web.session}]{\sphinxcrossref{
web.session
}}}
\end{itemize}

\end{description}\end{quote}


\begin{fulllineitems}
\phantomsection\label{\detokenize{reference/javascript_api:SwitchCompanyMenu}}\pysiglinewithargsret{\sphinxbfcode{\sphinxupquote{class }}\sphinxbfcode{\sphinxupquote{SwitchCompanyMenu}}}{}{}~\begin{quote}\begin{description}
\item[{Extends}] \leavevmode{\hyperref[\detokenize{reference/javascript_api:web.Widget.Widget}]{\sphinxcrossref{
Widget
}}}
\end{description}\end{quote}

\end{fulllineitems}


\end{fulllineitems}

\phantomsection\label{\detokenize{reference/javascript_api:module-web.ControlPanelMixin}}

\begin{fulllineitems}
\phantomsection\label{\detokenize{reference/javascript_api:web.ControlPanelMixin}}\pysigline{\sphinxbfcode{\sphinxupquote{module }}\sphinxbfcode{\sphinxupquote{web.ControlPanelMixin}}}~~\begin{quote}\begin{description}
\item[{Exports}] \leavevmode{\hyperref[\detokenize{reference/javascript_api:web.ControlPanelMixin.ControlPanelMixin}]{\sphinxcrossref{
ControlPanelMixin
}}}
\end{description}\end{quote}


\begin{fulllineitems}
\phantomsection\label{\detokenize{reference/javascript_api:ControlPanelMixin}}\pysigline{\sphinxbfcode{\sphinxupquote{namespace }}\sphinxbfcode{\sphinxupquote{ControlPanelMixin}}}
Mixin allowing widgets to communicate with the ControlPanel. Widgets needing a
ControlPanel should use this mixin and call update\_control\_panel(cp\_status) where
cp\_status contains information for the ControlPanel to update itself.

Note that the API is slightly awkward.  Hopefully we will improve this when
we get the time to refactor the control panel.

For example, here is what a typical client action would need to do to add
support for a control panel with some buttons:

\fvset{hllines={, ,}}%
\begin{sphinxVerbatim}[commandchars=\\\{\}]
\PYG{k+kd}{var} \PYG{n+nx}{ControlPanelMixin} \PYG{o}{=} \PYG{n+nx}{require}\PYG{p}{(}\PYG{l+s+s1}{\PYGZsq{}web.ControlPanelMixin\PYGZsq{}}\PYG{p}{)}\PYG{p}{;}

\PYG{k+kd}{var} \PYG{n+nx}{SomeClientAction} \PYG{o}{=} \PYG{n+nx}{Widget}\PYG{p}{.}\PYG{n+nx}{extend}\PYG{p}{(}\PYG{n+nx}{ControlPanelMixin}\PYG{p}{,} \PYG{p}{\PYGZob{}}
    \PYG{p}{...}
    \PYG{n+nx}{start}\PYG{o}{:} \PYG{k+kd}{function} \PYG{p}{(}\PYG{p}{)} \PYG{p}{\PYGZob{}}
        \PYG{k}{this}\PYG{p}{.}\PYG{n+nx}{\PYGZus{}renderButtons}\PYG{p}{(}\PYG{p}{)}\PYG{p}{;}
        \PYG{k}{this}\PYG{p}{.}\PYG{n+nx}{\PYGZus{}updateControlPanel}\PYG{p}{(}\PYG{p}{)}\PYG{p}{;}
        \PYG{p}{...}
    \PYG{p}{\PYGZcb{}}\PYG{p}{,}
    \PYG{n+nx}{do\PYGZus{}show}\PYG{o}{:} \PYG{k+kd}{function} \PYG{p}{(}\PYG{p}{)} \PYG{p}{\PYGZob{}}
         \PYG{p}{...}
         \PYG{k}{this}\PYG{p}{.}\PYG{n+nx}{\PYGZus{}updateControlPanel}\PYG{p}{(}\PYG{p}{)}\PYG{p}{;}
    \PYG{p}{\PYGZcb{}}\PYG{p}{,}
    \PYG{n+nx}{\PYGZus{}renderButtons}\PYG{o}{:} \PYG{k+kd}{function} \PYG{p}{(}\PYG{p}{)} \PYG{p}{\PYGZob{}}
        \PYG{k}{this}\PYG{p}{.}\PYG{n+nx}{\PYGZdl{}buttons} \PYG{o}{=} \PYG{n+nx}{\PYGZdl{}}\PYG{p}{(}\PYG{n+nx}{QWeb}\PYG{p}{.}\PYG{n+nx}{render}\PYG{p}{(}\PYG{l+s+s1}{\PYGZsq{}SomeTemplate.Buttons\PYGZsq{}}\PYG{p}{)}\PYG{p}{)}\PYG{p}{;}
        \PYG{k}{this}\PYG{p}{.}\PYG{n+nx}{\PYGZdl{}buttons}\PYG{p}{.}\PYG{n+nx}{on}\PYG{p}{(}\PYG{l+s+s1}{\PYGZsq{}click\PYGZsq{}}\PYG{p}{,} \PYG{p}{...}\PYG{p}{)}\PYG{p}{;}
    \PYG{p}{\PYGZcb{}}\PYG{p}{,}
    \PYG{n+nx}{\PYGZus{}updateControlPanel}\PYG{o}{:} \PYG{k+kd}{function} \PYG{p}{(}\PYG{p}{)} \PYG{p}{\PYGZob{}}
        \PYG{k}{this}\PYG{p}{.}\PYG{n+nx}{update\PYGZus{}control\PYGZus{}panel}\PYG{p}{(}\PYG{p}{\PYGZob{}}
            \PYG{n+nx}{breadcrumbs}\PYG{o}{:} \PYG{k}{this}\PYG{p}{.}\PYG{n+nx}{action\PYGZus{}manager}\PYG{p}{.}\PYG{n+nx}{get\PYGZus{}breadcrumbs}\PYG{p}{(}\PYG{p}{)}\PYG{p}{,}
            \PYG{n+nx}{cp\PYGZus{}content}\PYG{o}{:} \PYG{p}{\PYGZob{}}
               \PYG{n+nx}{\PYGZdl{}buttons}\PYG{o}{:} \PYG{k}{this}\PYG{p}{.}\PYG{n+nx}{\PYGZdl{}buttons}\PYG{p}{,}
            \PYG{p}{\PYGZcb{}}\PYG{p}{,}
     \PYG{p}{\PYGZcb{}}\PYG{p}{)}\PYG{p}{;}
\end{sphinxVerbatim}


\begin{fulllineitems}
\phantomsection\label{\detokenize{reference/javascript_api:update_control_panel}}\pysiglinewithargsret{\sphinxbfcode{\sphinxupquote{function }}\sphinxbfcode{\sphinxupquote{update\_control\_panel}}}{\sphinxoptional{\emph{cp\_status}}\sphinxoptional{, \emph{options}}}{}
Triggers ‘update’ on the cp\_bus to update the ControlPanel according to cp\_status
\begin{quote}\begin{description}
\item[{Parameters}] \leavevmode\begin{itemize}

\sphinxstylestrong{cp\_status} (\sphinxstyleliteralemphasis{\sphinxupquote{Object}}) \textendash{} see web.ControlPanel.update() for a description

\sphinxstylestrong{options} (\sphinxstyleliteralemphasis{\sphinxupquote{Object}}) \textendash{} see web.ControlPanel.update() for a description
\end{itemize}

\end{description}\end{quote}

\end{fulllineitems}


\end{fulllineitems}



\begin{fulllineitems}
\phantomsection\label{\detokenize{reference/javascript_api:ControlPanelMixin}}\pysigline{\sphinxbfcode{\sphinxupquote{namespace }}\sphinxbfcode{\sphinxupquote{ControlPanelMixin}}}
Mixin allowing widgets to communicate with the ControlPanel. Widgets needing a
ControlPanel should use this mixin and call update\_control\_panel(cp\_status) where
cp\_status contains information for the ControlPanel to update itself.

Note that the API is slightly awkward.  Hopefully we will improve this when
we get the time to refactor the control panel.

For example, here is what a typical client action would need to do to add
support for a control panel with some buttons:

\fvset{hllines={, ,}}%
\begin{sphinxVerbatim}[commandchars=\\\{\}]
\PYG{k+kd}{var} \PYG{n+nx}{ControlPanelMixin} \PYG{o}{=} \PYG{n+nx}{require}\PYG{p}{(}\PYG{l+s+s1}{\PYGZsq{}web.ControlPanelMixin\PYGZsq{}}\PYG{p}{)}\PYG{p}{;}

\PYG{k+kd}{var} \PYG{n+nx}{SomeClientAction} \PYG{o}{=} \PYG{n+nx}{Widget}\PYG{p}{.}\PYG{n+nx}{extend}\PYG{p}{(}\PYG{n+nx}{ControlPanelMixin}\PYG{p}{,} \PYG{p}{\PYGZob{}}
    \PYG{p}{...}
    \PYG{n+nx}{start}\PYG{o}{:} \PYG{k+kd}{function} \PYG{p}{(}\PYG{p}{)} \PYG{p}{\PYGZob{}}
        \PYG{k}{this}\PYG{p}{.}\PYG{n+nx}{\PYGZus{}renderButtons}\PYG{p}{(}\PYG{p}{)}\PYG{p}{;}
        \PYG{k}{this}\PYG{p}{.}\PYG{n+nx}{\PYGZus{}updateControlPanel}\PYG{p}{(}\PYG{p}{)}\PYG{p}{;}
        \PYG{p}{...}
    \PYG{p}{\PYGZcb{}}\PYG{p}{,}
    \PYG{n+nx}{do\PYGZus{}show}\PYG{o}{:} \PYG{k+kd}{function} \PYG{p}{(}\PYG{p}{)} \PYG{p}{\PYGZob{}}
         \PYG{p}{...}
         \PYG{k}{this}\PYG{p}{.}\PYG{n+nx}{\PYGZus{}updateControlPanel}\PYG{p}{(}\PYG{p}{)}\PYG{p}{;}
    \PYG{p}{\PYGZcb{}}\PYG{p}{,}
    \PYG{n+nx}{\PYGZus{}renderButtons}\PYG{o}{:} \PYG{k+kd}{function} \PYG{p}{(}\PYG{p}{)} \PYG{p}{\PYGZob{}}
        \PYG{k}{this}\PYG{p}{.}\PYG{n+nx}{\PYGZdl{}buttons} \PYG{o}{=} \PYG{n+nx}{\PYGZdl{}}\PYG{p}{(}\PYG{n+nx}{QWeb}\PYG{p}{.}\PYG{n+nx}{render}\PYG{p}{(}\PYG{l+s+s1}{\PYGZsq{}SomeTemplate.Buttons\PYGZsq{}}\PYG{p}{)}\PYG{p}{)}\PYG{p}{;}
        \PYG{k}{this}\PYG{p}{.}\PYG{n+nx}{\PYGZdl{}buttons}\PYG{p}{.}\PYG{n+nx}{on}\PYG{p}{(}\PYG{l+s+s1}{\PYGZsq{}click\PYGZsq{}}\PYG{p}{,} \PYG{p}{...}\PYG{p}{)}\PYG{p}{;}
    \PYG{p}{\PYGZcb{}}\PYG{p}{,}
    \PYG{n+nx}{\PYGZus{}updateControlPanel}\PYG{o}{:} \PYG{k+kd}{function} \PYG{p}{(}\PYG{p}{)} \PYG{p}{\PYGZob{}}
        \PYG{k}{this}\PYG{p}{.}\PYG{n+nx}{update\PYGZus{}control\PYGZus{}panel}\PYG{p}{(}\PYG{p}{\PYGZob{}}
            \PYG{n+nx}{breadcrumbs}\PYG{o}{:} \PYG{k}{this}\PYG{p}{.}\PYG{n+nx}{action\PYGZus{}manager}\PYG{p}{.}\PYG{n+nx}{get\PYGZus{}breadcrumbs}\PYG{p}{(}\PYG{p}{)}\PYG{p}{,}
            \PYG{n+nx}{cp\PYGZus{}content}\PYG{o}{:} \PYG{p}{\PYGZob{}}
               \PYG{n+nx}{\PYGZdl{}buttons}\PYG{o}{:} \PYG{k}{this}\PYG{p}{.}\PYG{n+nx}{\PYGZdl{}buttons}\PYG{p}{,}
            \PYG{p}{\PYGZcb{}}\PYG{p}{,}
     \PYG{p}{\PYGZcb{}}\PYG{p}{)}\PYG{p}{;}
\end{sphinxVerbatim}


\begin{fulllineitems}
\phantomsection\label{\detokenize{reference/javascript_api:update_control_panel}}\pysiglinewithargsret{\sphinxbfcode{\sphinxupquote{function }}\sphinxbfcode{\sphinxupquote{update\_control\_panel}}}{\sphinxoptional{\emph{cp\_status}}\sphinxoptional{, \emph{options}}}{}
Triggers ‘update’ on the cp\_bus to update the ControlPanel according to cp\_status
\begin{quote}\begin{description}
\item[{Parameters}] \leavevmode\begin{itemize}

\sphinxstylestrong{cp\_status} (\sphinxstyleliteralemphasis{\sphinxupquote{Object}}) \textendash{} see web.ControlPanel.update() for a description

\sphinxstylestrong{options} (\sphinxstyleliteralemphasis{\sphinxupquote{Object}}) \textendash{} see web.ControlPanel.update() for a description
\end{itemize}

\end{description}\end{quote}

\end{fulllineitems}


\end{fulllineitems}


\end{fulllineitems}

\phantomsection\label{\detokenize{reference/javascript_api:module-web.CalendarView}}

\begin{fulllineitems}
\phantomsection\label{\detokenize{reference/javascript_api:web.CalendarView}}\pysigline{\sphinxbfcode{\sphinxupquote{module }}\sphinxbfcode{\sphinxupquote{web.CalendarView}}}~~\begin{quote}\begin{description}
\item[{Exports}] \leavevmode{\hyperref[\detokenize{reference/javascript_api:web.CalendarView.CalendarView}]{\sphinxcrossref{
CalendarView
}}}
\item[{Depends On}] \leavevmode\begin{itemize}
\item {} {\hyperref[\detokenize{reference/javascript_api:web.AbstractView}]{\sphinxcrossref{
web.AbstractView
}}}
\item {} {\hyperref[\detokenize{reference/javascript_api:web.CalendarController}]{\sphinxcrossref{
web.CalendarController
}}}
\item {} {\hyperref[\detokenize{reference/javascript_api:web.CalendarModel}]{\sphinxcrossref{
web.CalendarModel
}}}
\item {} {\hyperref[\detokenize{reference/javascript_api:web.CalendarRenderer}]{\sphinxcrossref{
web.CalendarRenderer
}}}
\item {} {\hyperref[\detokenize{reference/javascript_api:web.core}]{\sphinxcrossref{
web.core
}}}
\item {} {\hyperref[\detokenize{reference/javascript_api:web.utils}]{\sphinxcrossref{
web.utils
}}}
\end{itemize}

\end{description}\end{quote}


\begin{fulllineitems}
\phantomsection\label{\detokenize{reference/javascript_api:CalendarView}}\pysiglinewithargsret{\sphinxbfcode{\sphinxupquote{class }}\sphinxbfcode{\sphinxupquote{CalendarView}}}{\emph{viewInfo}, \emph{params}}{}~\begin{quote}\begin{description}
\item[{Extends}] \leavevmode{\hyperref[\detokenize{reference/javascript_api:web.AbstractView.AbstractView}]{\sphinxcrossref{
AbstractView
}}}
\item[{Parameters}] \leavevmode\begin{itemize}

\sphinxstylestrong{viewInfo}

\sphinxstylestrong{params}
\end{itemize}

\end{description}\end{quote}

\end{fulllineitems}


\end{fulllineitems}

\phantomsection\label{\detokenize{reference/javascript_api:module-point_of_sale.keyboard}}

\begin{fulllineitems}
\phantomsection\label{\detokenize{reference/javascript_api:point_of_sale.keyboard}}\pysigline{\sphinxbfcode{\sphinxupquote{module }}\sphinxbfcode{\sphinxupquote{point\_of\_sale.keyboard}}}~~\begin{quote}\begin{description}
\item[{Exports}] \leavevmode{\hyperref[\detokenize{reference/javascript_api:point_of_sale.keyboard.}]{\sphinxcrossref{
\textless{}anonymous\textgreater{}
}}}
\item[{Depends On}] \leavevmode\begin{itemize}
\item {} {\hyperref[\detokenize{reference/javascript_api:web.Widget}]{\sphinxcrossref{
web.Widget
}}}
\end{itemize}

\end{description}\end{quote}


\begin{fulllineitems}
\phantomsection\label{\detokenize{reference/javascript_api:point_of_sale.keyboard.}}\pysigline{\sphinxbfcode{\sphinxupquote{namespace }}\sphinxbfcode{\sphinxupquote{}}}
\end{fulllineitems}


\end{fulllineitems}

\phantomsection\label{\detokenize{reference/javascript_api:module-web.AbstractRenderer}}

\begin{fulllineitems}
\phantomsection\label{\detokenize{reference/javascript_api:web.AbstractRenderer}}\pysigline{\sphinxbfcode{\sphinxupquote{module }}\sphinxbfcode{\sphinxupquote{web.AbstractRenderer}}}~~\begin{quote}\begin{description}
\item[{Exports}] \leavevmode
\textless{}anonymous\textgreater{}

\item[{Depends On}] \leavevmode\begin{itemize}
\item {} {\hyperref[\detokenize{reference/javascript_api:web.Widget}]{\sphinxcrossref{
web.Widget
}}}
\end{itemize}

\end{description}\end{quote}


\begin{fulllineitems}
\phantomsection\label{\detokenize{reference/javascript_api:AbstractRenderer}}\pysiglinewithargsret{\sphinxbfcode{\sphinxupquote{class }}\sphinxbfcode{\sphinxupquote{AbstractRenderer}}}{\emph{parent}, \emph{state}, \emph{params}}{}~\begin{quote}\begin{description}
\item[{Extends}] \leavevmode{\hyperref[\detokenize{reference/javascript_api:web.Widget.Widget}]{\sphinxcrossref{
Widget
}}}
\item[{Parameters}] \leavevmode\begin{itemize}

\sphinxstylestrong{parent} ({\hyperref[\detokenize{reference/javascript_api:Widget}]{\sphinxcrossref{\sphinxstyleliteralemphasis{\sphinxupquote{Widget}}}}})

\sphinxstylestrong{state} (\sphinxstyleliteralemphasis{\sphinxupquote{any}})

\sphinxstylestrong{params} ({\hyperref[\detokenize{reference/javascript_api:web.AbstractRenderer.AbstractRendererParams}]{\sphinxcrossref{\sphinxstyleliteralemphasis{\sphinxupquote{AbstractRendererParams}}}}})
\end{itemize}

\end{description}\end{quote}


\begin{fulllineitems}
\phantomsection\label{\detokenize{reference/javascript_api:start}}\pysiglinewithargsret{\sphinxbfcode{\sphinxupquote{method }}\sphinxbfcode{\sphinxupquote{start}}}{}{{ $\rightarrow$ Deferred}}
The rendering can be asynchronous (but it is not encouraged). The start
method simply makes sure that we render the view.
\begin{quote}\begin{description}
\item[{Return Type}] \leavevmode
\sphinxstyleliteralemphasis{\sphinxupquote{Deferred}}

\end{description}\end{quote}

\end{fulllineitems}



\begin{fulllineitems}
\phantomsection\label{\detokenize{reference/javascript_api:on_attach_callback}}\pysiglinewithargsret{\sphinxbfcode{\sphinxupquote{method }}\sphinxbfcode{\sphinxupquote{on\_attach\_callback}}}{}{}
Called each time the renderer is attached into the DOM.

\end{fulllineitems}



\begin{fulllineitems}
\phantomsection\label{\detokenize{reference/javascript_api:on_detach_callback}}\pysiglinewithargsret{\sphinxbfcode{\sphinxupquote{method }}\sphinxbfcode{\sphinxupquote{on\_detach\_callback}}}{}{}
Called each time the renderer is detached from the DOM.

\end{fulllineitems}



\begin{fulllineitems}
\phantomsection\label{\detokenize{reference/javascript_api:getLocalState}}\pysiglinewithargsret{\sphinxbfcode{\sphinxupquote{method }}\sphinxbfcode{\sphinxupquote{getLocalState}}}{}{{ $\rightarrow$ any}}
Returns any relevant state that the renderer might want to keep.

The idea is that a renderer can be destroyed, then be replaced by another
one instantiated with the state from the model and the localState from
the renderer, and the end result should be the same.

The kind of state that we expect the renderer to have is mostly DOM state
such as the scroll position, the currently active tab page, …

This method is called before each updateState, by the controller.
\begin{quote}\begin{description}
\item[{Return Type}] \leavevmode
\sphinxstyleliteralemphasis{\sphinxupquote{any}}

\end{description}\end{quote}

\end{fulllineitems}



\begin{fulllineitems}
\phantomsection\label{\detokenize{reference/javascript_api:setLocalState}}\pysiglinewithargsret{\sphinxbfcode{\sphinxupquote{method }}\sphinxbfcode{\sphinxupquote{setLocalState}}}{\emph{localState}}{}
This is the reverse operation from getLocalState.  With this method, we
expect the renderer to restore all DOM state, if it is relevant.

This method is called after each updateState, by the controller.
\begin{quote}\begin{description}
\item[{Parameters}] \leavevmode\begin{itemize}

\sphinxstylestrong{localState} (\sphinxstyleliteralemphasis{\sphinxupquote{any}}) \textendash{} the result of a call to getLocalState
\end{itemize}

\end{description}\end{quote}

\end{fulllineitems}



\begin{fulllineitems}
\phantomsection\label{\detokenize{reference/javascript_api:updateState}}\pysiglinewithargsret{\sphinxbfcode{\sphinxupquote{method }}\sphinxbfcode{\sphinxupquote{updateState}}}{\emph{state}, \emph{params}}{{ $\rightarrow$ Deferred}}
Updates the state of the view. It retriggers a full rerender, unless told
otherwise (for optimization for example).
\begin{quote}\begin{description}
\item[{Parameters}] \leavevmode\begin{itemize}

\sphinxstylestrong{state} (\sphinxstyleliteralemphasis{\sphinxupquote{any}})

\sphinxstylestrong{params} ({\hyperref[\detokenize{reference/javascript_api:web.AbstractRenderer.UpdateStateParams}]{\sphinxcrossref{\sphinxstyleliteralemphasis{\sphinxupquote{UpdateStateParams}}}}})
\end{itemize}

\item[{Return Type}] \leavevmode
\sphinxstyleliteralemphasis{\sphinxupquote{Deferred}}

\end{description}\end{quote}


\begin{fulllineitems}
\phantomsection\label{\detokenize{reference/javascript_api:UpdateStateParams}}\pysiglinewithargsret{\sphinxbfcode{\sphinxupquote{class }}\sphinxbfcode{\sphinxupquote{UpdateStateParams}}}{}{}~

\begin{fulllineitems}
\phantomsection\label{\detokenize{reference/javascript_api:noRender}}\pysigline{\sphinxbfcode{\sphinxupquote{attribute }}\sphinxbfcode{\sphinxupquote{noRender}} boolean}
if true, the method only updates the state without rerendering

\end{fulllineitems}


\end{fulllineitems}


\end{fulllineitems}



\begin{fulllineitems}
\phantomsection\label{\detokenize{reference/javascript_api:AbstractRendererParams}}\pysiglinewithargsret{\sphinxbfcode{\sphinxupquote{class }}\sphinxbfcode{\sphinxupquote{AbstractRendererParams}}}{}{}~

\begin{fulllineitems}
\phantomsection\label{\detokenize{reference/javascript_api:noContentHelp}}\pysigline{\sphinxbfcode{\sphinxupquote{attribute }}\sphinxbfcode{\sphinxupquote{noContentHelp}} string}
\end{fulllineitems}


\end{fulllineitems}


\end{fulllineitems}


\end{fulllineitems}

\phantomsection\label{\detokenize{reference/javascript_api:module-website_forum.share}}

\begin{fulllineitems}
\phantomsection\label{\detokenize{reference/javascript_api:website_forum.share}}\pysigline{\sphinxbfcode{\sphinxupquote{module }}\sphinxbfcode{\sphinxupquote{website\_forum.share}}}~~\begin{quote}\begin{description}
\item[{Exports}] \leavevmode{\hyperref[\detokenize{reference/javascript_api:website_forum.share.}]{\sphinxcrossref{
\textless{}anonymous\textgreater{}
}}}
\item[{Depends On}] \leavevmode\begin{itemize}
\item {} {\hyperref[\detokenize{reference/javascript_api:web.core}]{\sphinxcrossref{
web.core
}}}
\item {} {\hyperref[\detokenize{reference/javascript_api:web_editor.base}]{\sphinxcrossref{
web\_editor.base
}}}
\item {} {\hyperref[\detokenize{reference/javascript_api:website.content.snippets.animation}]{\sphinxcrossref{
website.content.snippets.animation
}}}
\end{itemize}

\end{description}\end{quote}


\begin{fulllineitems}
\phantomsection\label{\detokenize{reference/javascript_api:website_forum.share.}}\pysigline{\sphinxbfcode{\sphinxupquote{namespace }}\sphinxbfcode{\sphinxupquote{}}}
\end{fulllineitems}


\end{fulllineitems}

\phantomsection\label{\detokenize{reference/javascript_api:module-web.ViewManager}}

\begin{fulllineitems}
\phantomsection\label{\detokenize{reference/javascript_api:web.ViewManager}}\pysigline{\sphinxbfcode{\sphinxupquote{module }}\sphinxbfcode{\sphinxupquote{web.ViewManager}}}~~\begin{quote}\begin{description}
\item[{Exports}] \leavevmode{\hyperref[\detokenize{reference/javascript_api:web.ViewManager.ViewManager}]{\sphinxcrossref{
ViewManager
}}}
\item[{Depends On}] \leavevmode\begin{itemize}
\item {} {\hyperref[\detokenize{reference/javascript_api:web.Context}]{\sphinxcrossref{
web.Context
}}}
\item {} {\hyperref[\detokenize{reference/javascript_api:web.ControlPanelMixin}]{\sphinxcrossref{
web.ControlPanelMixin
}}}
\item {} {\hyperref[\detokenize{reference/javascript_api:web.SearchView}]{\sphinxcrossref{
web.SearchView
}}}
\item {} {\hyperref[\detokenize{reference/javascript_api:web.Widget}]{\sphinxcrossref{
web.Widget
}}}
\item {} {\hyperref[\detokenize{reference/javascript_api:web.core}]{\sphinxcrossref{
web.core
}}}
\item {} {\hyperref[\detokenize{reference/javascript_api:web.data}]{\sphinxcrossref{
web.data
}}}
\item {} {\hyperref[\detokenize{reference/javascript_api:web.data_manager}]{\sphinxcrossref{
web.data\_manager
}}}
\item {} {\hyperref[\detokenize{reference/javascript_api:web.dom}]{\sphinxcrossref{
web.dom
}}}
\item {} {\hyperref[\detokenize{reference/javascript_api:web.pyeval}]{\sphinxcrossref{
web.pyeval
}}}
\item {} {\hyperref[\detokenize{reference/javascript_api:web.view_registry}]{\sphinxcrossref{
web.view\_registry
}}}
\end{itemize}

\end{description}\end{quote}


\begin{fulllineitems}
\phantomsection\label{\detokenize{reference/javascript_api:ViewManager}}\pysiglinewithargsret{\sphinxbfcode{\sphinxupquote{class }}\sphinxbfcode{\sphinxupquote{ViewManager}}}{\emph{parent}\sphinxoptional{, \emph{dataset}}\sphinxoptional{, \emph{views}}\sphinxoptional{, \emph{flags}}, \emph{options}}{}~\begin{quote}\begin{description}
\item[{Extends}] \leavevmode{\hyperref[\detokenize{reference/javascript_api:web.Widget.Widget}]{\sphinxcrossref{
Widget
}}}
\item[{Mixes}] \leavevmode\begin{itemize}
\item {} {\hyperref[\detokenize{reference/javascript_api:web.ControlPanelMixin.ControlPanelMixin}]{\sphinxcrossref{
ControlPanelMixin
}}}
\end{itemize}

\item[{Parameters}] \leavevmode\begin{itemize}

\sphinxstylestrong{parent}

\sphinxstylestrong{dataset} (\sphinxstyleliteralemphasis{\sphinxupquote{Object}})

\sphinxstylestrong{views} (\sphinxstyleliteralemphasis{\sphinxupquote{Array}}) \textendash{} List of {[}view\_id, view\_type{[}, fields\_view{]}{]}

\sphinxstylestrong{flags} (\sphinxstyleliteralemphasis{\sphinxupquote{Object}}) \textendash{} various boolean describing UI state

\sphinxstylestrong{options}
\end{itemize}

\end{description}\end{quote}


\begin{fulllineitems}
\phantomsection\label{\detokenize{reference/javascript_api:on_attach_callback}}\pysiglinewithargsret{\sphinxbfcode{\sphinxupquote{method }}\sphinxbfcode{\sphinxupquote{on\_attach\_callback}}}{}{}
Called each time the view manager is attached into the DOM

\end{fulllineitems}



\begin{fulllineitems}
\phantomsection\label{\detokenize{reference/javascript_api:on_detach_callback}}\pysiglinewithargsret{\sphinxbfcode{\sphinxupquote{method }}\sphinxbfcode{\sphinxupquote{on\_detach\_callback}}}{}{}
Called each time the view manager is detached from the DOM

\end{fulllineitems}



\begin{fulllineitems}
\phantomsection\label{\detokenize{reference/javascript_api:load_views}}\pysiglinewithargsret{\sphinxbfcode{\sphinxupquote{method }}\sphinxbfcode{\sphinxupquote{load\_views}}}{}{{ $\rightarrow$ Deferred}}
Loads all missing field\_views of views in this.views and the search view.
\begin{quote}\begin{description}
\item[{Return Type}] \leavevmode
\sphinxstyleliteralemphasis{\sphinxupquote{Deferred}}

\end{description}\end{quote}

\end{fulllineitems}



\begin{fulllineitems}
\phantomsection\label{\detokenize{reference/javascript_api:get_default_view}}\pysiglinewithargsret{\sphinxbfcode{\sphinxupquote{method }}\sphinxbfcode{\sphinxupquote{get\_default\_view}}}{}{{ $\rightarrow$ Object}}
Returns the default view with the following fallbacks:
\begin{itemize}
\item {} 
use the default\_view defined in the flags, if any

\item {} 
use the first view in the view\_order

\end{itemize}
\begin{quote}\begin{description}
\item[{Returns}] \leavevmode
the default view

\item[{Return Type}] \leavevmode
\sphinxstyleliteralemphasis{\sphinxupquote{Object}}

\end{description}\end{quote}

\end{fulllineitems}



\begin{fulllineitems}
\phantomsection\label{\detokenize{reference/javascript_api:render_switch_buttons}}\pysiglinewithargsret{\sphinxbfcode{\sphinxupquote{method }}\sphinxbfcode{\sphinxupquote{render\_switch\_buttons}}}{}{}
Renders the switch buttons for multi- and mono-record views and adds
listeners on them, but does not append them to the DOM
Sets switch\_buttons.\$mono and switch\_buttons.\$multi to send to the ControlPanel

\end{fulllineitems}



\begin{fulllineitems}
\phantomsection\label{\detokenize{reference/javascript_api:render_view_control_elements}}\pysiglinewithargsret{\sphinxbfcode{\sphinxupquote{method }}\sphinxbfcode{\sphinxupquote{render\_view\_control\_elements}}}{}{}
Renders the control elements (buttons, sidebar, pager) of the current view.
Fills this.active\_view.control\_elements dictionnary with the rendered
elements and the adequate view switcher, to send to the ControlPanel.
Warning: it should be called before calling do\_show on the view as the
sidebar is extended to listen on the load\_record event triggered as soon
as do\_show is done (the sidebar should thus be instantiated before).

\end{fulllineitems}



\begin{fulllineitems}
\phantomsection\label{\detokenize{reference/javascript_api:setup_search_view}}\pysiglinewithargsret{\sphinxbfcode{\sphinxupquote{method }}\sphinxbfcode{\sphinxupquote{setup\_search\_view}}}{}{}
Sets up the current viewmanager’s search view.
Sets \$searchview and \$searchview\_buttons in searchview\_elements to send to the ControlPanel

\end{fulllineitems}



\begin{fulllineitems}
\phantomsection\label{\detokenize{reference/javascript_api:do_execute_action}}\pysiglinewithargsret{\sphinxbfcode{\sphinxupquote{method }}\sphinxbfcode{\sphinxupquote{do\_execute\_action}}}{\emph{action\_data}, \emph{env}, \emph{on\_closed}}{}
Fetches and executes the action identified by \sphinxcode{\sphinxupquote{action\_data}}.
\begin{quote}\begin{description}
\item[{Parameters}] \leavevmode\begin{itemize}

\sphinxstylestrong{action\_data} ({\hyperref[\detokenize{reference/javascript_api:web.ViewManager.DoExecuteActionActionData}]{\sphinxcrossref{\sphinxstyleliteralemphasis{\sphinxupquote{DoExecuteActionActionData}}}}}) \textendash{} the action descriptor data

\sphinxstylestrong{env} ({\hyperref[\detokenize{reference/javascript_api:web.ViewManager.DoExecuteActionEnv}]{\sphinxcrossref{\sphinxstyleliteralemphasis{\sphinxupquote{DoExecuteActionEnv}}}}})

\sphinxstylestrong{on\_closed} (\sphinxstyleliteralemphasis{\sphinxupquote{Function}}) \textendash{} callback to execute when dialog is closed or when the action does not generate any result (no new action)
\end{itemize}

\end{description}\end{quote}


\begin{fulllineitems}
\phantomsection\label{\detokenize{reference/javascript_api:DoExecuteActionEnv}}\pysiglinewithargsret{\sphinxbfcode{\sphinxupquote{class }}\sphinxbfcode{\sphinxupquote{DoExecuteActionEnv}}}{}{}~

\begin{fulllineitems}
\phantomsection\label{\detokenize{reference/javascript_api:model}}\pysigline{\sphinxbfcode{\sphinxupquote{attribute }}\sphinxbfcode{\sphinxupquote{model}} string}
the model of the record(s) triggering the action

\end{fulllineitems}



\begin{fulllineitems}
\phantomsection\label{\detokenize{reference/javascript_api:resIDs}}\pysigline{\sphinxbfcode{\sphinxupquote{attribute }}\sphinxbfcode{\sphinxupquote{resIDs}} integer{[}{]}}
the current ids in the environment where the action is triggered

\end{fulllineitems}



\begin{fulllineitems}
\phantomsection\label{\detokenize{reference/javascript_api:currentID}}\pysigline{\sphinxbfcode{\sphinxupquote{attribute }}\sphinxbfcode{\sphinxupquote{currentID}} integer}
the id of the record triggering the action

\end{fulllineitems}



\begin{fulllineitems}
\phantomsection\label{\detokenize{reference/javascript_api:context}}\pysigline{\sphinxbfcode{\sphinxupquote{attribute }}\sphinxbfcode{\sphinxupquote{context}} Object}
a context to pass to the action

\end{fulllineitems}


\end{fulllineitems}



\begin{fulllineitems}
\phantomsection\label{\detokenize{reference/javascript_api:DoExecuteActionActionData}}\pysiglinewithargsret{\sphinxbfcode{\sphinxupquote{class }}\sphinxbfcode{\sphinxupquote{DoExecuteActionActionData}}}{}{}
the action descriptor data


\begin{fulllineitems}
\phantomsection\label{\detokenize{reference/javascript_api:name}}\pysigline{\sphinxbfcode{\sphinxupquote{attribute }}\sphinxbfcode{\sphinxupquote{name}} String}
the action name, used to uniquely identify the action to find and execute it

\end{fulllineitems}



\begin{fulllineitems}
\phantomsection\label{\detokenize{reference/javascript_api:special}}\pysigline{\sphinxbfcode{\sphinxupquote{attribute }}\sphinxbfcode{\sphinxupquote{special}} String}
special action handlers, closes the dialog if set

\end{fulllineitems}



\begin{fulllineitems}
\phantomsection\label{\detokenize{reference/javascript_api:type}}\pysigline{\sphinxbfcode{\sphinxupquote{attribute }}\sphinxbfcode{\sphinxupquote{type}} String}
the action type, if present, one of \sphinxcode{\sphinxupquote{'object'}}, \sphinxcode{\sphinxupquote{'action'}} or \sphinxcode{\sphinxupquote{'workflow'}}

\end{fulllineitems}



\begin{fulllineitems}
\phantomsection\label{\detokenize{reference/javascript_api:context}}\pysigline{\sphinxbfcode{\sphinxupquote{attribute }}\sphinxbfcode{\sphinxupquote{context}} Object}
additional action context, to add to the current context

\end{fulllineitems}



\begin{fulllineitems}
\phantomsection\label{\detokenize{reference/javascript_api:effect}}\pysigline{\sphinxbfcode{\sphinxupquote{attribute }}\sphinxbfcode{\sphinxupquote{effect}} string}~\begin{description}
\item[{if given, a visual effect (a}] \leavevmode
rainbowman by default) will be displayed when the action is complete,
with the string (evaluated) given as options.

\end{description}

\end{fulllineitems}


\end{fulllineitems}


\end{fulllineitems}


\end{fulllineitems}


\end{fulllineitems}

\phantomsection\label{\detokenize{reference/javascript_api:module-web_tour.tour}}

\begin{fulllineitems}
\phantomsection\label{\detokenize{reference/javascript_api:web_tour.tour}}\pysigline{\sphinxbfcode{\sphinxupquote{module }}\sphinxbfcode{\sphinxupquote{web\_tour.tour}}}~~\begin{quote}\begin{description}
\item[{Exports}] \leavevmode{\hyperref[\detokenize{reference/javascript_api:web_tour.tour.}]{\sphinxcrossref{
\textless{}anonymous\textgreater{}
}}}
\item[{Depends On}] \leavevmode\begin{itemize}
\item {} {\hyperref[\detokenize{reference/javascript_api:web.Class}]{\sphinxcrossref{
web.Class
}}}
\item {} {\hyperref[\detokenize{reference/javascript_api:web.ajax}]{\sphinxcrossref{
web.ajax
}}}
\item {} {\hyperref[\detokenize{reference/javascript_api:web.config}]{\sphinxcrossref{
web.config
}}}
\item {} {\hyperref[\detokenize{reference/javascript_api:web.core}]{\sphinxcrossref{
web.core
}}}
\item {} {\hyperref[\detokenize{reference/javascript_api:web.mixins}]{\sphinxcrossref{
web.mixins
}}}
\item {} {\hyperref[\detokenize{reference/javascript_api:web.rpc}]{\sphinxcrossref{
web.rpc
}}}
\item {} {\hyperref[\detokenize{reference/javascript_api:web.session}]{\sphinxcrossref{
web.session
}}}
\item {} {\hyperref[\detokenize{reference/javascript_api:web_tour.TourManager}]{\sphinxcrossref{
web\_tour.TourManager
}}}
\end{itemize}

\end{description}\end{quote}


\begin{fulllineitems}
\phantomsection\label{\detokenize{reference/javascript_api:web_tour.tour.}}\pysigline{\sphinxbfcode{\sphinxupquote{namespace }}\sphinxbfcode{\sphinxupquote{}}}
\end{fulllineitems}


\end{fulllineitems}

\phantomsection\label{\detokenize{reference/javascript_api:module-web.Menu}}

\begin{fulllineitems}
\phantomsection\label{\detokenize{reference/javascript_api:web.Menu}}\pysigline{\sphinxbfcode{\sphinxupquote{module }}\sphinxbfcode{\sphinxupquote{web.Menu}}}~~\begin{quote}\begin{description}
\item[{Exports}] \leavevmode{\hyperref[\detokenize{reference/javascript_api:web.Menu.Menu}]{\sphinxcrossref{
Menu
}}}
\item[{Depends On}] \leavevmode\begin{itemize}
\item {} {\hyperref[\detokenize{reference/javascript_api:web.Widget}]{\sphinxcrossref{
web.Widget
}}}
\item {} {\hyperref[\detokenize{reference/javascript_api:web.core}]{\sphinxcrossref{
web.core
}}}
\item {} {\hyperref[\detokenize{reference/javascript_api:web.session}]{\sphinxcrossref{
web.session
}}}
\end{itemize}

\end{description}\end{quote}


\begin{fulllineitems}
\phantomsection\label{\detokenize{reference/javascript_api:Menu}}\pysiglinewithargsret{\sphinxbfcode{\sphinxupquote{class }}\sphinxbfcode{\sphinxupquote{Menu}}}{}{}~\begin{quote}\begin{description}
\item[{Extends}] \leavevmode{\hyperref[\detokenize{reference/javascript_api:web.Widget.Widget}]{\sphinxcrossref{
Widget
}}}
\end{description}\end{quote}


\begin{fulllineitems}
\phantomsection\label{\detokenize{reference/javascript_api:reflow}}\pysiglinewithargsret{\sphinxbfcode{\sphinxupquote{method }}\sphinxbfcode{\sphinxupquote{reflow}}}{\emph{behavior}}{}
Reflow the menu items and dock overflowing items into a “More” menu item.
Automatically called when ‘menu\_bound’ event is triggered and on window resizing.
\begin{quote}\begin{description}
\item[{Parameters}] \leavevmode\begin{itemize}

\sphinxstylestrong{behavior} (\sphinxstyleliteralemphasis{\sphinxupquote{string}}) \textendash{} If set to ‘all\_outside’, all the items are displayed.
If not set, only the overflowing items are hidden.
\end{itemize}

\end{description}\end{quote}

\end{fulllineitems}



\begin{fulllineitems}
\phantomsection\label{\detokenize{reference/javascript_api:open_menu}}\pysiglinewithargsret{\sphinxbfcode{\sphinxupquote{method }}\sphinxbfcode{\sphinxupquote{open\_menu}}}{\emph{id}}{}
Opens a given menu by id, as if a user had browsed to that menu by hand
except does not trigger any event on the way
\begin{quote}\begin{description}
\item[{Parameters}] \leavevmode\begin{itemize}

\sphinxstylestrong{id} (\sphinxstyleliteralemphasis{\sphinxupquote{Number}}) \textendash{} database id of the terminal menu to select
\end{itemize}

\end{description}\end{quote}

\end{fulllineitems}



\begin{fulllineitems}
\phantomsection\label{\detokenize{reference/javascript_api:open_action}}\pysiglinewithargsret{\sphinxbfcode{\sphinxupquote{method }}\sphinxbfcode{\sphinxupquote{open\_action}}}{\emph{id}\sphinxoptional{, \emph{menuID}}}{}
Call open\_menu on a menu\_item that matches the action\_id

If \sphinxcode{\sphinxupquote{menuID}} is a match on this action, open this menu\_item.
Otherwise open the first menu\_item that matches the action\_id.
\begin{quote}\begin{description}
\item[{Parameters}] \leavevmode\begin{itemize}

\sphinxstylestrong{id} (\sphinxstyleliteralemphasis{\sphinxupquote{Number}}) \textendash{} the action\_id to match

\sphinxstylestrong{menuID} (\sphinxstyleliteralemphasis{\sphinxupquote{Number}}) \textendash{} a menu ID that may match with provided action
\end{itemize}

\end{description}\end{quote}

\end{fulllineitems}



\begin{fulllineitems}
\phantomsection\label{\detokenize{reference/javascript_api:menu_click}}\pysiglinewithargsret{\sphinxbfcode{\sphinxupquote{method }}\sphinxbfcode{\sphinxupquote{menu\_click}}}{\emph{id}}{}
Process a click on a menu item
\begin{quote}\begin{description}
\item[{Parameters}] \leavevmode\begin{itemize}

\sphinxstylestrong{id} (\sphinxstyleliteralemphasis{\sphinxupquote{Number}}) \textendash{} the menu\_id
\end{itemize}

\end{description}\end{quote}

\end{fulllineitems}



\begin{fulllineitems}
\phantomsection\label{\detokenize{reference/javascript_api:on_change_top_menu}}\pysiglinewithargsret{\sphinxbfcode{\sphinxupquote{method }}\sphinxbfcode{\sphinxupquote{on\_change\_top\_menu}}}{\sphinxoptional{\emph{menu\_id}}}{}
Change the current top menu
\begin{quote}\begin{description}
\item[{Parameters}] \leavevmode\begin{itemize}

\sphinxstylestrong{menu\_id} (\sphinxstyleliteralemphasis{\sphinxupquote{int}}) \textendash{} the top menu id
\end{itemize}

\end{description}\end{quote}

\end{fulllineitems}


\end{fulllineitems}


\end{fulllineitems}

\phantomsection\label{\detokenize{reference/javascript_api:module-web_diagram.DiagramController}}

\begin{fulllineitems}
\phantomsection\label{\detokenize{reference/javascript_api:web_diagram.DiagramController}}\pysigline{\sphinxbfcode{\sphinxupquote{module }}\sphinxbfcode{\sphinxupquote{web\_diagram.DiagramController}}}~~\begin{quote}\begin{description}
\item[{Exports}] \leavevmode{\hyperref[\detokenize{reference/javascript_api:web_diagram.DiagramController.DiagramController}]{\sphinxcrossref{
DiagramController
}}}
\item[{Depends On}] \leavevmode\begin{itemize}
\item {} {\hyperref[\detokenize{reference/javascript_api:web.AbstractController}]{\sphinxcrossref{
web.AbstractController
}}}
\item {} {\hyperref[\detokenize{reference/javascript_api:web.Dialog}]{\sphinxcrossref{
web.Dialog
}}}
\item {} {\hyperref[\detokenize{reference/javascript_api:web.core}]{\sphinxcrossref{
web.core
}}}
\item {} {\hyperref[\detokenize{reference/javascript_api:web.view_dialogs}]{\sphinxcrossref{
web.view\_dialogs
}}}
\end{itemize}

\end{description}\end{quote}


\begin{fulllineitems}
\phantomsection\label{\detokenize{reference/javascript_api:DiagramController}}\pysiglinewithargsret{\sphinxbfcode{\sphinxupquote{class }}\sphinxbfcode{\sphinxupquote{DiagramController}}}{\emph{parent}, \emph{model}, \emph{renderer}, \emph{params}}{}~\begin{quote}\begin{description}
\item[{Extends}] \leavevmode{\hyperref[\detokenize{reference/javascript_api:web.AbstractController.AbstractController}]{\sphinxcrossref{
AbstractController
}}}
\item[{Parameters}] \leavevmode\begin{itemize}

\sphinxstylestrong{parent} ({\hyperref[\detokenize{reference/javascript_api:Widget}]{\sphinxcrossref{\sphinxstyleliteralemphasis{\sphinxupquote{Widget}}}}})

\sphinxstylestrong{model} ({\hyperref[\detokenize{reference/javascript_api:DiagramModel}]{\sphinxcrossref{\sphinxstyleliteralemphasis{\sphinxupquote{DiagramModel}}}}})

\sphinxstylestrong{renderer} ({\hyperref[\detokenize{reference/javascript_api:DiagramRenderer}]{\sphinxcrossref{\sphinxstyleliteralemphasis{\sphinxupquote{DiagramRenderer}}}}})

\sphinxstylestrong{params} (\sphinxstyleliteralemphasis{\sphinxupquote{Object}})
\end{itemize}

\end{description}\end{quote}

Diagram Controller


\begin{fulllineitems}
\phantomsection\label{\detokenize{reference/javascript_api:renderButtons}}\pysiglinewithargsret{\sphinxbfcode{\sphinxupquote{method }}\sphinxbfcode{\sphinxupquote{renderButtons}}}{\sphinxoptional{\emph{\$node}}}{}
Render the buttons according to the DiagramView.buttons template and add
listeners on it. Set this.\$buttons with the produced jQuery element
\begin{quote}\begin{description}
\item[{Parameters}] \leavevmode\begin{itemize}

\sphinxstylestrong{\$node} (\sphinxstyleliteralemphasis{\sphinxupquote{jQuery}}) \textendash{} a jQuery node where the rendered buttons should
  be inserted \$node may be undefined, in which case they are inserted
  into this.options.\$buttons
\end{itemize}

\end{description}\end{quote}

\end{fulllineitems}


\end{fulllineitems}



\begin{fulllineitems}
\phantomsection\label{\detokenize{reference/javascript_api:DiagramController}}\pysiglinewithargsret{\sphinxbfcode{\sphinxupquote{class }}\sphinxbfcode{\sphinxupquote{DiagramController}}}{\emph{parent}, \emph{model}, \emph{renderer}, \emph{params}}{}~\begin{quote}\begin{description}
\item[{Extends}] \leavevmode{\hyperref[\detokenize{reference/javascript_api:web.AbstractController.AbstractController}]{\sphinxcrossref{
AbstractController
}}}
\item[{Parameters}] \leavevmode\begin{itemize}

\sphinxstylestrong{parent} ({\hyperref[\detokenize{reference/javascript_api:Widget}]{\sphinxcrossref{\sphinxstyleliteralemphasis{\sphinxupquote{Widget}}}}})

\sphinxstylestrong{model} ({\hyperref[\detokenize{reference/javascript_api:DiagramModel}]{\sphinxcrossref{\sphinxstyleliteralemphasis{\sphinxupquote{DiagramModel}}}}})

\sphinxstylestrong{renderer} ({\hyperref[\detokenize{reference/javascript_api:DiagramRenderer}]{\sphinxcrossref{\sphinxstyleliteralemphasis{\sphinxupquote{DiagramRenderer}}}}})

\sphinxstylestrong{params} (\sphinxstyleliteralemphasis{\sphinxupquote{Object}})
\end{itemize}

\end{description}\end{quote}

Diagram Controller


\begin{fulllineitems}
\phantomsection\label{\detokenize{reference/javascript_api:renderButtons}}\pysiglinewithargsret{\sphinxbfcode{\sphinxupquote{method }}\sphinxbfcode{\sphinxupquote{renderButtons}}}{\sphinxoptional{\emph{\$node}}}{}
Render the buttons according to the DiagramView.buttons template and add
listeners on it. Set this.\$buttons with the produced jQuery element
\begin{quote}\begin{description}
\item[{Parameters}] \leavevmode\begin{itemize}

\sphinxstylestrong{\$node} (\sphinxstyleliteralemphasis{\sphinxupquote{jQuery}}) \textendash{} a jQuery node where the rendered buttons should
  be inserted \$node may be undefined, in which case they are inserted
  into this.options.\$buttons
\end{itemize}

\end{description}\end{quote}

\end{fulllineitems}


\end{fulllineitems}


\end{fulllineitems}

\phantomsection\label{\detokenize{reference/javascript_api:module-website_links.code_editor}}

\begin{fulllineitems}
\phantomsection\label{\detokenize{reference/javascript_api:website_links.code_editor}}\pysigline{\sphinxbfcode{\sphinxupquote{module }}\sphinxbfcode{\sphinxupquote{website\_links.code\_editor}}}~~\begin{quote}\begin{description}
\item[{Exports}] \leavevmode{\hyperref[\detokenize{reference/javascript_api:website_links.code_editor.}]{\sphinxcrossref{
\textless{}anonymous\textgreater{}
}}}
\item[{Depends On}] \leavevmode\begin{itemize}
\item {} {\hyperref[\detokenize{reference/javascript_api:web.ajax}]{\sphinxcrossref{
web.ajax
}}}
\end{itemize}

\end{description}\end{quote}


\begin{fulllineitems}
\phantomsection\label{\detokenize{reference/javascript_api:website_links.code_editor.}}\pysigline{\sphinxbfcode{\sphinxupquote{namespace }}\sphinxbfcode{\sphinxupquote{}}}
\end{fulllineitems}


\end{fulllineitems}

\phantomsection\label{\detokenize{reference/javascript_api:module-web_settings_dashboard}}

\begin{fulllineitems}
\phantomsection\label{\detokenize{reference/javascript_api:web_settings_dashboard}}\pysigline{\sphinxbfcode{\sphinxupquote{module }}\sphinxbfcode{\sphinxupquote{web\_settings\_dashboard}}}~~\begin{quote}\begin{description}
\item[{Exports}] \leavevmode{\hyperref[\detokenize{reference/javascript_api:web_settings_dashboard.}]{\sphinxcrossref{
\textless{}anonymous\textgreater{}
}}}
\item[{Depends On}] \leavevmode\begin{itemize}
\item {} {\hyperref[\detokenize{reference/javascript_api:web.Widget}]{\sphinxcrossref{
web.Widget
}}}
\item {} {\hyperref[\detokenize{reference/javascript_api:web.core}]{\sphinxcrossref{
web.core
}}}
\item {} {\hyperref[\detokenize{reference/javascript_api:web.framework}]{\sphinxcrossref{
web.framework
}}}
\item {} {\hyperref[\detokenize{reference/javascript_api:web.planner.common}]{\sphinxcrossref{
web.planner.common
}}}
\end{itemize}

\end{description}\end{quote}


\begin{fulllineitems}
\phantomsection\label{\detokenize{reference/javascript_api:web_settings_dashboard.}}\pysigline{\sphinxbfcode{\sphinxupquote{namespace }}\sphinxbfcode{\sphinxupquote{}}}
\end{fulllineitems}


\end{fulllineitems}

\phantomsection\label{\detokenize{reference/javascript_api:module-mail.Followers}}

\begin{fulllineitems}
\phantomsection\label{\detokenize{reference/javascript_api:mail.Followers}}\pysigline{\sphinxbfcode{\sphinxupquote{module }}\sphinxbfcode{\sphinxupquote{mail.Followers}}}~~\begin{quote}\begin{description}
\item[{Exports}] \leavevmode{\hyperref[\detokenize{reference/javascript_api:mail.Followers.Followers}]{\sphinxcrossref{
Followers
}}}
\item[{Depends On}] \leavevmode\begin{itemize}
\item {} {\hyperref[\detokenize{reference/javascript_api:web.AbstractField}]{\sphinxcrossref{
web.AbstractField
}}}
\item {} {\hyperref[\detokenize{reference/javascript_api:web.Dialog}]{\sphinxcrossref{
web.Dialog
}}}
\item {} {\hyperref[\detokenize{reference/javascript_api:web.concurrency}]{\sphinxcrossref{
web.concurrency
}}}
\item {} {\hyperref[\detokenize{reference/javascript_api:web.core}]{\sphinxcrossref{
web.core
}}}
\item {} {\hyperref[\detokenize{reference/javascript_api:web.field_registry}]{\sphinxcrossref{
web.field\_registry
}}}
\end{itemize}

\end{description}\end{quote}


\begin{fulllineitems}
\phantomsection\label{\detokenize{reference/javascript_api:Followers}}\pysiglinewithargsret{\sphinxbfcode{\sphinxupquote{class }}\sphinxbfcode{\sphinxupquote{Followers}}}{\emph{parent}, \emph{name}, \emph{record}, \emph{options}}{}~\begin{quote}\begin{description}
\item[{Extends}] \leavevmode{\hyperref[\detokenize{reference/javascript_api:web.AbstractField.AbstractField}]{\sphinxcrossref{
AbstractField
}}}
\item[{Parameters}] \leavevmode\begin{itemize}

\sphinxstylestrong{parent}

\sphinxstylestrong{name}

\sphinxstylestrong{record}

\sphinxstylestrong{options}
\end{itemize}

\end{description}\end{quote}

\end{fulllineitems}


\end{fulllineitems}

\phantomsection\label{\detokenize{reference/javascript_api:module-web.FormRenderer}}

\begin{fulllineitems}
\phantomsection\label{\detokenize{reference/javascript_api:web.FormRenderer}}\pysigline{\sphinxbfcode{\sphinxupquote{module }}\sphinxbfcode{\sphinxupquote{web.FormRenderer}}}~~\begin{quote}\begin{description}
\item[{Exports}] \leavevmode{\hyperref[\detokenize{reference/javascript_api:web.FormRenderer.FormRenderer}]{\sphinxcrossref{
FormRenderer
}}}
\item[{Depends On}] \leavevmode\begin{itemize}
\item {} {\hyperref[\detokenize{reference/javascript_api:web.BasicRenderer}]{\sphinxcrossref{
web.BasicRenderer
}}}
\item {} {\hyperref[\detokenize{reference/javascript_api:web.config}]{\sphinxcrossref{
web.config
}}}
\item {} {\hyperref[\detokenize{reference/javascript_api:web.core}]{\sphinxcrossref{
web.core
}}}
\item {} {\hyperref[\detokenize{reference/javascript_api:web.dom}]{\sphinxcrossref{
web.dom
}}}
\end{itemize}

\end{description}\end{quote}


\begin{fulllineitems}
\phantomsection\label{\detokenize{reference/javascript_api:FormRenderer}}\pysiglinewithargsret{\sphinxbfcode{\sphinxupquote{class }}\sphinxbfcode{\sphinxupquote{FormRenderer}}}{\emph{parent}, \emph{state}, \emph{params}}{}~\begin{quote}\begin{description}
\item[{Extends}] \leavevmode{\hyperref[\detokenize{reference/javascript_api:web.BasicRenderer.BasicRenderer}]{\sphinxcrossref{
BasicRenderer
}}}
\item[{Parameters}] \leavevmode\begin{itemize}

\sphinxstylestrong{parent}

\sphinxstylestrong{state}

\sphinxstylestrong{params}
\end{itemize}

\end{description}\end{quote}


\begin{fulllineitems}
\phantomsection\label{\detokenize{reference/javascript_api:autofocus}}\pysiglinewithargsret{\sphinxbfcode{\sphinxupquote{method }}\sphinxbfcode{\sphinxupquote{autofocus}}}{}{}
Focuses the field having attribute ‘default\_focus’ set, if any, or the
first focusable field otherwise.

\end{fulllineitems}



\begin{fulllineitems}
\phantomsection\label{\detokenize{reference/javascript_api:canBeSaved}}\pysiglinewithargsret{\sphinxbfcode{\sphinxupquote{method }}\sphinxbfcode{\sphinxupquote{canBeSaved}}}{\emph{recordID}}{{ $\rightarrow$ string{[}{]}}}
Extend the method so that labels also receive the ‘o\_field\_invalid’ class
if necessary.
\begin{quote}\begin{description}
\item[{Parameters}] \leavevmode\begin{itemize}

\sphinxstylestrong{recordID} (\sphinxstyleliteralemphasis{\sphinxupquote{string}})
\end{itemize}

\item[{Return Type}] \leavevmode
\sphinxstyleliteralemphasis{\sphinxupquote{Array}}\textless{}\sphinxstyleliteralemphasis{\sphinxupquote{string}}\textgreater{}

\end{description}\end{quote}

\end{fulllineitems}



\begin{fulllineitems}
\phantomsection\label{\detokenize{reference/javascript_api:displayTranslationAlert}}\pysiglinewithargsret{\sphinxbfcode{\sphinxupquote{method }}\sphinxbfcode{\sphinxupquote{displayTranslationAlert}}}{\emph{alertFields}}{}
Show a warning message if the user modified a translated field.  For each
field, the notification provides a link to edit the field’s translations.
\begin{quote}\begin{description}
\item[{Parameters}] \leavevmode\begin{itemize}

\sphinxstylestrong{alertFields} (\sphinxstyleliteralemphasis{\sphinxupquote{Array}}\textless{}\sphinxstyleliteralemphasis{\sphinxupquote{Object}}\textgreater{}) \textendash{} field list
\end{itemize}

\end{description}\end{quote}

\end{fulllineitems}



\begin{fulllineitems}
\phantomsection\label{\detokenize{reference/javascript_api:confirmChange}}\pysiglinewithargsret{\sphinxbfcode{\sphinxupquote{method }}\sphinxbfcode{\sphinxupquote{confirmChange}}}{\emph{state}, \emph{id}, \emph{fields}}{}
Updates the chatter area with the new state if its fields has changed
\begin{quote}\begin{description}
\item[{Parameters}] \leavevmode\begin{itemize}

\sphinxstylestrong{state}

\sphinxstylestrong{id}

\sphinxstylestrong{fields}
\end{itemize}

\end{description}\end{quote}

\end{fulllineitems}



\begin{fulllineitems}
\phantomsection\label{\detokenize{reference/javascript_api:disableButtons}}\pysiglinewithargsret{\sphinxbfcode{\sphinxupquote{method }}\sphinxbfcode{\sphinxupquote{disableButtons}}}{}{}
Disable statusbar buttons and stat buttons so that they can’t be clicked anymore

\end{fulllineitems}



\begin{fulllineitems}
\phantomsection\label{\detokenize{reference/javascript_api:enableButtons}}\pysiglinewithargsret{\sphinxbfcode{\sphinxupquote{method }}\sphinxbfcode{\sphinxupquote{enableButtons}}}{}{}
Enable statusbar buttons and stat buttons so they can be clicked again

\end{fulllineitems}



\begin{fulllineitems}
\phantomsection\label{\detokenize{reference/javascript_api:getLocalState}}\pysiglinewithargsret{\sphinxbfcode{\sphinxupquote{method }}\sphinxbfcode{\sphinxupquote{getLocalState}}}{}{{ $\rightarrow$ Object}}
returns the active tab pages for each notebook
\begin{quote}\begin{description}
\item[{Returns}] \leavevmode
a map from notebook name to the active tab index

\item[{Return Type}] \leavevmode
\sphinxstyleliteralemphasis{\sphinxupquote{Object}}

\end{description}\end{quote}

\end{fulllineitems}



\begin{fulllineitems}
\phantomsection\label{\detokenize{reference/javascript_api:setLocalState}}\pysiglinewithargsret{\sphinxbfcode{\sphinxupquote{method }}\sphinxbfcode{\sphinxupquote{setLocalState}}}{\emph{state}}{}
restore active tab pages for each notebook
\begin{quote}\begin{description}
\item[{Parameters}] \leavevmode\begin{itemize}

\sphinxstylestrong{state} (\sphinxstyleliteralemphasis{\sphinxupquote{Object}}) \textendash{} the result from a getLocalState call
\end{itemize}

\end{description}\end{quote}

\end{fulllineitems}



\begin{fulllineitems}
\phantomsection\label{\detokenize{reference/javascript_api:on_attach_callback}}\pysiglinewithargsret{\sphinxbfcode{\sphinxupquote{method }}\sphinxbfcode{\sphinxupquote{on\_attach\_callback}}}{}{}
Call \sphinxcode{\sphinxupquote{on\_attach\_callback}} for each subview

\end{fulllineitems}



\begin{fulllineitems}
\phantomsection\label{\detokenize{reference/javascript_api:on_detach_callback}}\pysiglinewithargsret{\sphinxbfcode{\sphinxupquote{method }}\sphinxbfcode{\sphinxupquote{on\_detach\_callback}}}{}{}
Call \sphinxcode{\sphinxupquote{on\_detach\_callback}} for each subview

\end{fulllineitems}



\begin{fulllineitems}
\phantomsection\label{\detokenize{reference/javascript_api:getBoard}}\pysiglinewithargsret{\sphinxbfcode{\sphinxupquote{method }}\sphinxbfcode{\sphinxupquote{getBoard}}}{}{{ $\rightarrow$ Object}}
Returns a representation of the current dashboard
\begin{quote}\begin{description}
\item[{Return Type}] \leavevmode
\sphinxstyleliteralemphasis{\sphinxupquote{Object}}

\end{description}\end{quote}

\end{fulllineitems}


\end{fulllineitems}


\end{fulllineitems}

\phantomsection\label{\detokenize{reference/javascript_api:module-web.KanbanRecord}}

\begin{fulllineitems}
\phantomsection\label{\detokenize{reference/javascript_api:web.KanbanRecord}}\pysigline{\sphinxbfcode{\sphinxupquote{module }}\sphinxbfcode{\sphinxupquote{web.KanbanRecord}}}~~\begin{quote}\begin{description}
\item[{Exports}] \leavevmode{\hyperref[\detokenize{reference/javascript_api:web.KanbanRecord.KanbanRecord}]{\sphinxcrossref{
KanbanRecord
}}}
\item[{Depends On}] \leavevmode\begin{itemize}
\item {} {\hyperref[\detokenize{reference/javascript_api:web.Domain}]{\sphinxcrossref{
web.Domain
}}}
\item {} {\hyperref[\detokenize{reference/javascript_api:web.Widget}]{\sphinxcrossref{
web.Widget
}}}
\item {} {\hyperref[\detokenize{reference/javascript_api:web.core}]{\sphinxcrossref{
web.core
}}}
\item {} {\hyperref[\detokenize{reference/javascript_api:web.field_utils}]{\sphinxcrossref{
web.field\_utils
}}}
\item {} {\hyperref[\detokenize{reference/javascript_api:web.utils}]{\sphinxcrossref{
web.utils
}}}
\item {} {\hyperref[\detokenize{reference/javascript_api:web.widget_registry}]{\sphinxcrossref{
web.widget\_registry
}}}
\end{itemize}

\end{description}\end{quote}


\begin{fulllineitems}
\phantomsection\label{\detokenize{reference/javascript_api:KanbanRecord}}\pysiglinewithargsret{\sphinxbfcode{\sphinxupquote{class }}\sphinxbfcode{\sphinxupquote{KanbanRecord}}}{\emph{parent}, \emph{state}, \emph{options}}{}~\begin{quote}\begin{description}
\item[{Extends}] \leavevmode{\hyperref[\detokenize{reference/javascript_api:web.Widget.Widget}]{\sphinxcrossref{
Widget
}}}
\item[{Parameters}] \leavevmode\begin{itemize}

\sphinxstylestrong{parent}

\sphinxstylestrong{state}

\sphinxstylestrong{options}
\end{itemize}

\end{description}\end{quote}


\begin{fulllineitems}
\phantomsection\label{\detokenize{reference/javascript_api:update}}\pysiglinewithargsret{\sphinxbfcode{\sphinxupquote{method }}\sphinxbfcode{\sphinxupquote{update}}}{\emph{state}}{}
Re-renders the record with a new state
\begin{quote}\begin{description}
\item[{Parameters}] \leavevmode\begin{itemize}

\sphinxstylestrong{state} (\sphinxstyleliteralemphasis{\sphinxupquote{Object}})
\end{itemize}

\end{description}\end{quote}

\end{fulllineitems}


\end{fulllineitems}


\end{fulllineitems}

\phantomsection\label{\detokenize{reference/javascript_api:module-point_of_sale.screens}}

\begin{fulllineitems}
\phantomsection\label{\detokenize{reference/javascript_api:point_of_sale.screens}}\pysigline{\sphinxbfcode{\sphinxupquote{module }}\sphinxbfcode{\sphinxupquote{point\_of\_sale.screens}}}~~\begin{quote}\begin{description}
\item[{Exports}] \leavevmode{\hyperref[\detokenize{reference/javascript_api:point_of_sale.screens.}]{\sphinxcrossref{
\textless{}anonymous\textgreater{}
}}}
\item[{Depends On}] \leavevmode\begin{itemize}
\item {} {\hyperref[\detokenize{reference/javascript_api:point_of_sale.BaseWidget}]{\sphinxcrossref{
point\_of\_sale.BaseWidget
}}}
\item {} {\hyperref[\detokenize{reference/javascript_api:point_of_sale.gui}]{\sphinxcrossref{
point\_of\_sale.gui
}}}
\item {} {\hyperref[\detokenize{reference/javascript_api:point_of_sale.models}]{\sphinxcrossref{
point\_of\_sale.models
}}}
\item {} {\hyperref[\detokenize{reference/javascript_api:web.core}]{\sphinxcrossref{
web.core
}}}
\item {} {\hyperref[\detokenize{reference/javascript_api:web.field_utils}]{\sphinxcrossref{
web.field\_utils
}}}
\item {} {\hyperref[\detokenize{reference/javascript_api:web.rpc}]{\sphinxcrossref{
web.rpc
}}}
\item {} {\hyperref[\detokenize{reference/javascript_api:web.utils}]{\sphinxcrossref{
web.utils
}}}
\end{itemize}

\end{description}\end{quote}


\begin{fulllineitems}
\phantomsection\label{\detokenize{reference/javascript_api:point_of_sale.screens.}}\pysigline{\sphinxbfcode{\sphinxupquote{namespace }}\sphinxbfcode{\sphinxupquote{}}}
\end{fulllineitems}


\end{fulllineitems}

\phantomsection\label{\detokenize{reference/javascript_api:module-web_tour.Tip}}

\begin{fulllineitems}
\phantomsection\label{\detokenize{reference/javascript_api:web_tour.Tip}}\pysigline{\sphinxbfcode{\sphinxupquote{module }}\sphinxbfcode{\sphinxupquote{web\_tour.Tip}}}~~\begin{quote}\begin{description}
\item[{Exports}] \leavevmode{\hyperref[\detokenize{reference/javascript_api:web_tour.Tip.Tip}]{\sphinxcrossref{
Tip
}}}
\item[{Depends On}] \leavevmode\begin{itemize}
\item {} {\hyperref[\detokenize{reference/javascript_api:web.Widget}]{\sphinxcrossref{
web.Widget
}}}
\item {} {\hyperref[\detokenize{reference/javascript_api:web.core}]{\sphinxcrossref{
web.core
}}}
\end{itemize}

\end{description}\end{quote}


\begin{fulllineitems}
\phantomsection\label{\detokenize{reference/javascript_api:Tip}}\pysiglinewithargsret{\sphinxbfcode{\sphinxupquote{class }}\sphinxbfcode{\sphinxupquote{Tip}}}{\emph{parent}\sphinxoptional{, \emph{info}}}{}~\begin{quote}\begin{description}
\item[{Extends}] \leavevmode{\hyperref[\detokenize{reference/javascript_api:web.Widget.Widget}]{\sphinxcrossref{
Widget
}}}
\item[{Parameters}] \leavevmode\begin{itemize}

\sphinxstylestrong{parent} ({\hyperref[\detokenize{reference/javascript_api:Widget}]{\sphinxcrossref{\sphinxstyleliteralemphasis{\sphinxupquote{Widget}}}}})

\sphinxstylestrong{info} (\sphinxstyleliteralemphasis{\sphinxupquote{Object}}) \textendash{} description of the tip, containing the following keys:
 - content {[}String{]} the html content of the tip
 - event\_handlers {[}Object{]} description of optional event handlers to bind to the tip:
   - event {[}String{]} the event name
   - selector {[}String{]} the jQuery selector on which the event should be bound
   - handler {[}function{]} the handler
 - position {[}String{]} tip’s position (‘top’, ‘right’, ‘left’ or ‘bottom’), default ‘right’
 - width {[}int{]} the width in px of the tip when opened, default 270
 - space {[}int{]} space in px between anchor and tip, default 10
 - overlay {[}Object{]} x and y values for the number of pixels the mouseout detection area
   overlaps the opened tip, default \{x: 50, y: 50\}
\end{itemize}

\end{description}\end{quote}

\end{fulllineitems}


\end{fulllineitems}

\phantomsection\label{\detokenize{reference/javascript_api:module-barcodes.BarcodeParser}}

\begin{fulllineitems}
\phantomsection\label{\detokenize{reference/javascript_api:barcodes.BarcodeParser}}\pysigline{\sphinxbfcode{\sphinxupquote{module }}\sphinxbfcode{\sphinxupquote{barcodes.BarcodeParser}}}~~\begin{quote}\begin{description}
\item[{Exports}] \leavevmode{\hyperref[\detokenize{reference/javascript_api:barcodes.BarcodeParser.BarcodeParser}]{\sphinxcrossref{
BarcodeParser
}}}
\item[{Depends On}] \leavevmode\begin{itemize}
\item {} {\hyperref[\detokenize{reference/javascript_api:web.Class}]{\sphinxcrossref{
web.Class
}}}
\item {} {\hyperref[\detokenize{reference/javascript_api:web.rpc}]{\sphinxcrossref{
web.rpc
}}}
\end{itemize}

\end{description}\end{quote}


\begin{fulllineitems}
\phantomsection\label{\detokenize{reference/javascript_api:BarcodeParser}}\pysiglinewithargsret{\sphinxbfcode{\sphinxupquote{class }}\sphinxbfcode{\sphinxupquote{BarcodeParser}}}{\emph{attributes}}{}~\begin{quote}\begin{description}
\item[{Extends}] \leavevmode{\hyperref[\detokenize{reference/javascript_api:web.Class.Class}]{\sphinxcrossref{
Class
}}}
\item[{Parameters}] \leavevmode\begin{itemize}

\sphinxstylestrong{attributes}
\end{itemize}

\end{description}\end{quote}

\end{fulllineitems}


\end{fulllineitems}

\phantomsection\label{\detokenize{reference/javascript_api:module-web.dom_ready}}

\begin{fulllineitems}
\phantomsection\label{\detokenize{reference/javascript_api:web.dom_ready}}\pysigline{\sphinxbfcode{\sphinxupquote{module }}\sphinxbfcode{\sphinxupquote{web.dom\_ready}}}~~\begin{quote}\begin{description}
\item[{Exports}] \leavevmode{\hyperref[\detokenize{reference/javascript_api:web.dom_ready.def}]{\sphinxcrossref{
def
}}}
\end{description}\end{quote}


\begin{fulllineitems}
\phantomsection\label{\detokenize{reference/javascript_api:def}}\pysigline{\sphinxbfcode{\sphinxupquote{namespace }}\sphinxbfcode{\sphinxupquote{def}}}
\end{fulllineitems}


\end{fulllineitems}

\phantomsection\label{\detokenize{reference/javascript_api:module-web.ServicesMixin}}

\begin{fulllineitems}
\phantomsection\label{\detokenize{reference/javascript_api:web.ServicesMixin}}\pysigline{\sphinxbfcode{\sphinxupquote{module }}\sphinxbfcode{\sphinxupquote{web.ServicesMixin}}}~~\begin{quote}\begin{description}
\item[{Exports}] \leavevmode{\hyperref[\detokenize{reference/javascript_api:web.ServicesMixin.ServicesMixin}]{\sphinxcrossref{
ServicesMixin
}}}
\item[{Depends On}] \leavevmode\begin{itemize}
\item {} {\hyperref[\detokenize{reference/javascript_api:web.rpc}]{\sphinxcrossref{
web.rpc
}}}
\end{itemize}

\end{description}\end{quote}


\begin{fulllineitems}
\phantomsection\label{\detokenize{reference/javascript_api:ServicesMixin}}\pysigline{\sphinxbfcode{\sphinxupquote{mixin }}\sphinxbfcode{\sphinxupquote{ServicesMixin}}}~

\begin{fulllineitems}
\phantomsection\label{\detokenize{reference/javascript_api:do_action}}\pysiglinewithargsret{\sphinxbfcode{\sphinxupquote{function }}\sphinxbfcode{\sphinxupquote{do\_action}}}{\emph{action}, \emph{options}}{{ $\rightarrow$ Deferred}}
Informs the action manager to do an action. This supposes that the action
manager can be found amongst the ancestors of the current widget.
If that’s not the case this method will simply return an unresolved
deferred.
\begin{quote}\begin{description}
\item[{Parameters}] \leavevmode\begin{itemize}

\sphinxstylestrong{action} (\sphinxstyleliteralemphasis{\sphinxupquote{any}})

\sphinxstylestrong{options} (\sphinxstyleliteralemphasis{\sphinxupquote{any}})
\end{itemize}

\item[{Return Type}] \leavevmode
\sphinxstyleliteralemphasis{\sphinxupquote{Deferred}}

\end{description}\end{quote}

\end{fulllineitems}


\end{fulllineitems}


\end{fulllineitems}

\phantomsection\label{\detokenize{reference/javascript_api:module-website.content.snippets.animation}}

\begin{fulllineitems}
\phantomsection\label{\detokenize{reference/javascript_api:website.content.snippets.animation}}\pysigline{\sphinxbfcode{\sphinxupquote{module }}\sphinxbfcode{\sphinxupquote{website.content.snippets.animation}}}~~\begin{quote}\begin{description}
\item[{Exports}] \leavevmode{\hyperref[\detokenize{reference/javascript_api:website.content.snippets.animation.}]{\sphinxcrossref{
\textless{}anonymous\textgreater{}
}}}
\item[{Depends On}] \leavevmode\begin{itemize}
\item {} {\hyperref[\detokenize{reference/javascript_api:web.Class}]{\sphinxcrossref{
web.Class
}}}
\item {} {\hyperref[\detokenize{reference/javascript_api:web.Widget}]{\sphinxcrossref{
web.Widget
}}}
\item {} {\hyperref[\detokenize{reference/javascript_api:web.core}]{\sphinxcrossref{
web.core
}}}
\item {} {\hyperref[\detokenize{reference/javascript_api:web.mixins}]{\sphinxcrossref{
web.mixins
}}}
\end{itemize}

\end{description}\end{quote}


\begin{fulllineitems}
\phantomsection\label{\detokenize{reference/javascript_api:Animation}}\pysiglinewithargsret{\sphinxbfcode{\sphinxupquote{class }}\sphinxbfcode{\sphinxupquote{Animation}}}{\emph{parent}, \emph{editableMode}}{}~\begin{quote}\begin{description}
\item[{Extends}] \leavevmode{\hyperref[\detokenize{reference/javascript_api:web.Widget.Widget}]{\sphinxcrossref{
Widget
}}}
\item[{Parameters}] \leavevmode\begin{itemize}

\sphinxstylestrong{parent} (\sphinxstyleliteralemphasis{\sphinxupquote{Object}})

\sphinxstylestrong{editableMode} (\sphinxstyleliteralemphasis{\sphinxupquote{boolean}}) \textendash{} true if the page is in edition mode
\end{itemize}

\end{description}\end{quote}

Provides a way for executing code once a website DOM element is loaded in the
dom and handle the case where the website edit mode is triggered.

Also register AnimationEffect automatically (@see effects, \_prepareEffects).


\begin{fulllineitems}
\phantomsection\label{\detokenize{reference/javascript_api:selector}}\pysigline{\sphinxbfcode{\sphinxupquote{attribute }}\sphinxbfcode{\sphinxupquote{selector}} Boolean}
The selector attribute, if defined, allows to automatically create an
instance of this animation on page load for each DOM element which
matches this selector. The \sphinxcode{\sphinxupquote{Animation.\$target}} element will then be that
particular DOM element. This should be the main way of instantiating
\sphinxcode{\sphinxupquote{Animation}} elements.

\end{fulllineitems}



\begin{fulllineitems}
\phantomsection\label{\detokenize{reference/javascript_api:edit_events}}\pysigline{\sphinxbfcode{\sphinxupquote{namespace }}\sphinxbfcode{\sphinxupquote{edit\_events}}}
Acts as @see Widget.events except that the events are only binded if the
Animation instance is instanciated in edit mode.

\end{fulllineitems}



\begin{fulllineitems}
\phantomsection\label{\detokenize{reference/javascript_api:read_events}}\pysigline{\sphinxbfcode{\sphinxupquote{namespace }}\sphinxbfcode{\sphinxupquote{read\_events}}}
Acts as @see Widget.events except that the events are only binded if the
Animation instance is instanciated in readonly mode.

\end{fulllineitems}



\begin{fulllineitems}
\phantomsection\label{\detokenize{reference/javascript_api:maxFPS}}\pysigline{\sphinxbfcode{\sphinxupquote{attribute }}\sphinxbfcode{\sphinxupquote{maxFPS}} Number}
The max FPS at which all the automatic animation effects will be
running by default.

\end{fulllineitems}



\begin{fulllineitems}
\phantomsection\label{\detokenize{reference/javascript_api:start}}\pysiglinewithargsret{\sphinxbfcode{\sphinxupquote{method }}\sphinxbfcode{\sphinxupquote{start}}}{}{}
Initializes the animation. The method should not be called directly as
called automatically on animation instantiation and on restart.

Also, prepares animation’s effects and start them if any.

\end{fulllineitems}



\begin{fulllineitems}
\phantomsection\label{\detokenize{reference/javascript_api:destroy}}\pysiglinewithargsret{\sphinxbfcode{\sphinxupquote{method }}\sphinxbfcode{\sphinxupquote{destroy}}}{}{}
Destroys the animation and basically restores the target to the state it
was before the start method was called (unlike standard widget, the
associated \$el DOM is not removed).

Also stops animation effects and destroys them if any.

\end{fulllineitems}


\end{fulllineitems}



\begin{fulllineitems}
\phantomsection\label{\detokenize{reference/javascript_api:AnimationEffect}}\pysiglinewithargsret{\sphinxbfcode{\sphinxupquote{class }}\sphinxbfcode{\sphinxupquote{AnimationEffect}}}{\emph{parent}, \emph{updateCallback}\sphinxoptional{, \emph{startEvents}}\sphinxoptional{, \emph{\$startTarget}}\sphinxoptional{, \emph{options}}}{}~\begin{quote}\begin{description}
\item[{Extends}] \leavevmode{\hyperref[\detokenize{reference/javascript_api:web.Class.Class}]{\sphinxcrossref{
Class
}}}
\item[{Mixes}] \leavevmode\begin{itemize}
\item {} {\hyperref[\detokenize{reference/javascript_api:web.mixins.ParentedMixin}]{\sphinxcrossref{
ParentedMixin
}}}
\end{itemize}

\item[{Parameters}] \leavevmode\begin{itemize}

\sphinxstylestrong{parent} (\sphinxstyleliteralemphasis{\sphinxupquote{Object}})

\sphinxstylestrong{updateCallback} (\sphinxstyleliteralemphasis{\sphinxupquote{function}}) \textendash{} the animation update callback

\sphinxstylestrong{startEvents}=\sphinxstyleemphasis{scroll} (\sphinxstyleliteralemphasis{\sphinxupquote{string}}) \textendash{} space separated list of events which starts the animation loop

\sphinxstylestrong{\$startTarget}=\sphinxstyleemphasis{window} (\sphinxstyleliteralemphasis{\sphinxupquote{jQuery}}\sphinxstyleemphasis{ or }\sphinxstyleliteralemphasis{\sphinxupquote{DOMElement}}) \textendash{} the element(s) on which the startEvents are listened

\sphinxstylestrong{options} ({\hyperref[\detokenize{reference/javascript_api:website.content.snippets.animation.AnimationEffectOptions}]{\sphinxcrossref{\sphinxstyleliteralemphasis{\sphinxupquote{AnimationEffectOptions}}}}})
\end{itemize}

\end{description}\end{quote}

In charge of handling one animation loop using the requestAnimationFrame
feature. This is used by the \sphinxcode{\sphinxupquote{Animation}} class below and should not be called
directly by an end developer.

This uses a simple API: it can be started, stopped, played and paused.


\begin{fulllineitems}
\phantomsection\label{\detokenize{reference/javascript_api:start}}\pysiglinewithargsret{\sphinxbfcode{\sphinxupquote{method }}\sphinxbfcode{\sphinxupquote{start}}}{}{}
Initializes when the animation must be played and paused and initializes
the animation first frame.

\end{fulllineitems}



\begin{fulllineitems}
\phantomsection\label{\detokenize{reference/javascript_api:stop}}\pysiglinewithargsret{\sphinxbfcode{\sphinxupquote{method }}\sphinxbfcode{\sphinxupquote{stop}}}{}{}
Pauses the animation and destroys the attached events which trigger the
animation to be played or paused.

\end{fulllineitems}



\begin{fulllineitems}
\phantomsection\label{\detokenize{reference/javascript_api:play}}\pysiglinewithargsret{\sphinxbfcode{\sphinxupquote{method }}\sphinxbfcode{\sphinxupquote{play}}}{\emph{e}}{}
Forces the requestAnimationFrame loop to start.
\begin{quote}\begin{description}
\item[{Parameters}] \leavevmode\begin{itemize}

\sphinxstylestrong{e} (\sphinxstyleliteralemphasis{\sphinxupquote{Event}}) \textendash{} the event which triggered the animation to play
\end{itemize}

\end{description}\end{quote}

\end{fulllineitems}



\begin{fulllineitems}
\phantomsection\label{\detokenize{reference/javascript_api:pause}}\pysiglinewithargsret{\sphinxbfcode{\sphinxupquote{method }}\sphinxbfcode{\sphinxupquote{pause}}}{}{}
Forces the requestAnimationFrame loop to stop.

\end{fulllineitems}



\begin{fulllineitems}
\phantomsection\label{\detokenize{reference/javascript_api:AnimationEffectOptions}}\pysiglinewithargsret{\sphinxbfcode{\sphinxupquote{class }}\sphinxbfcode{\sphinxupquote{AnimationEffectOptions}}}{}{}~

\begin{fulllineitems}
\phantomsection\label{\detokenize{reference/javascript_api:getStateCallback}}\pysigline{\sphinxbfcode{\sphinxupquote{attribute }}\sphinxbfcode{\sphinxupquote{getStateCallback}} function}~\begin{description}
\item[{a function which returns a value which represents the state of the}] \leavevmode
animation, i.e. for two same value, no refreshing of the animation
is needed. Can be used for optimization. If the \$startTarget is
the window element, this defaults to returning the current
scoll offset of the window or the size of the window for the
scroll and resize events respectively.

\end{description}

\end{fulllineitems}



\begin{fulllineitems}
\phantomsection\label{\detokenize{reference/javascript_api:endEvents}}\pysigline{\sphinxbfcode{\sphinxupquote{attribute }}\sphinxbfcode{\sphinxupquote{endEvents}} string}~\begin{description}
\item[{space separated list of events which pause the animation loop. If}] \leavevmode
not given, the animation is stopped after a while (if no
startEvents is received again)

\end{description}

\end{fulllineitems}



\begin{fulllineitems}
\phantomsection\label{\detokenize{reference/javascript_api:_endTarget}}\pysigline{\sphinxbfcode{\sphinxupquote{attribute }}\sphinxbfcode{\sphinxupquote{\$endTarget}} jQuery\textbar{}DOMElement}
the element(s) on which the endEvents are listened

\end{fulllineitems}


\end{fulllineitems}


\end{fulllineitems}



\begin{fulllineitems}
\phantomsection\label{\detokenize{reference/javascript_api:registry}}\pysigline{\sphinxbfcode{\sphinxupquote{namespace }}\sphinxbfcode{\sphinxupquote{registry}}}
The registry object contains the list of available animations.


\begin{fulllineitems}
\phantomsection\label{\detokenize{reference/javascript_api:_fixAppleCollapse}}\pysiglinewithargsret{\sphinxbfcode{\sphinxupquote{class }}\sphinxbfcode{\sphinxupquote{\_fixAppleCollapse}}}{}{}~\begin{quote}\begin{description}
\item[{Extends}] \leavevmode{\hyperref[\detokenize{reference/javascript_api:website.content.snippets.animation.Animation}]{\sphinxcrossref{
Animation
}}}
\end{description}\end{quote}

This is a fix for apple device (\textless{}= IPhone 4, IPad 2)
Standard bootstrap requires data-toggle=’collapse’ element to be \textless{}a/\textgreater{} tags.
Unfortunatly one snippet uses a \textless{}div/\textgreater{} tag instead. The fix forces an empty
click handler on these div, which allows standard bootstrap to work.

This should be removed in a future odoo snippets refactoring.

\end{fulllineitems}


\end{fulllineitems}



\begin{fulllineitems}
\phantomsection\label{\detokenize{reference/javascript_api:website.content.snippets.animation.}}\pysigline{\sphinxbfcode{\sphinxupquote{namespace }}\sphinxbfcode{\sphinxupquote{}}}~

\begin{fulllineitems}
\phantomsection\label{\detokenize{reference/javascript_api:Animation}}\pysiglinewithargsret{\sphinxbfcode{\sphinxupquote{class }}\sphinxbfcode{\sphinxupquote{Animation}}}{\emph{parent}, \emph{editableMode}}{}~\begin{quote}\begin{description}
\item[{Extends}] \leavevmode{\hyperref[\detokenize{reference/javascript_api:web.Widget.Widget}]{\sphinxcrossref{
Widget
}}}
\item[{Parameters}] \leavevmode\begin{itemize}

\sphinxstylestrong{parent} (\sphinxstyleliteralemphasis{\sphinxupquote{Object}})

\sphinxstylestrong{editableMode} (\sphinxstyleliteralemphasis{\sphinxupquote{boolean}}) \textendash{} true if the page is in edition mode
\end{itemize}

\end{description}\end{quote}

Provides a way for executing code once a website DOM element is loaded in the
dom and handle the case where the website edit mode is triggered.

Also register AnimationEffect automatically (@see effects, \_prepareEffects).


\begin{fulllineitems}
\phantomsection\label{\detokenize{reference/javascript_api:selector}}\pysigline{\sphinxbfcode{\sphinxupquote{attribute }}\sphinxbfcode{\sphinxupquote{selector}} Boolean}
The selector attribute, if defined, allows to automatically create an
instance of this animation on page load for each DOM element which
matches this selector. The \sphinxcode{\sphinxupquote{Animation.\$target}} element will then be that
particular DOM element. This should be the main way of instantiating
\sphinxcode{\sphinxupquote{Animation}} elements.

\end{fulllineitems}



\begin{fulllineitems}
\phantomsection\label{\detokenize{reference/javascript_api:edit_events}}\pysigline{\sphinxbfcode{\sphinxupquote{namespace }}\sphinxbfcode{\sphinxupquote{edit\_events}}}
Acts as @see Widget.events except that the events are only binded if the
Animation instance is instanciated in edit mode.

\end{fulllineitems}



\begin{fulllineitems}
\phantomsection\label{\detokenize{reference/javascript_api:read_events}}\pysigline{\sphinxbfcode{\sphinxupquote{namespace }}\sphinxbfcode{\sphinxupquote{read\_events}}}
Acts as @see Widget.events except that the events are only binded if the
Animation instance is instanciated in readonly mode.

\end{fulllineitems}



\begin{fulllineitems}
\phantomsection\label{\detokenize{reference/javascript_api:maxFPS}}\pysigline{\sphinxbfcode{\sphinxupquote{attribute }}\sphinxbfcode{\sphinxupquote{maxFPS}} Number}
The max FPS at which all the automatic animation effects will be
running by default.

\end{fulllineitems}



\begin{fulllineitems}
\phantomsection\label{\detokenize{reference/javascript_api:start}}\pysiglinewithargsret{\sphinxbfcode{\sphinxupquote{method }}\sphinxbfcode{\sphinxupquote{start}}}{}{}
Initializes the animation. The method should not be called directly as
called automatically on animation instantiation and on restart.

Also, prepares animation’s effects and start them if any.

\end{fulllineitems}



\begin{fulllineitems}
\phantomsection\label{\detokenize{reference/javascript_api:destroy}}\pysiglinewithargsret{\sphinxbfcode{\sphinxupquote{method }}\sphinxbfcode{\sphinxupquote{destroy}}}{}{}
Destroys the animation and basically restores the target to the state it
was before the start method was called (unlike standard widget, the
associated \$el DOM is not removed).

Also stops animation effects and destroys them if any.

\end{fulllineitems}


\end{fulllineitems}



\begin{fulllineitems}
\phantomsection\label{\detokenize{reference/javascript_api:registry}}\pysigline{\sphinxbfcode{\sphinxupquote{namespace }}\sphinxbfcode{\sphinxupquote{registry}}}
The registry object contains the list of available animations.


\begin{fulllineitems}
\phantomsection\label{\detokenize{reference/javascript_api:_fixAppleCollapse}}\pysiglinewithargsret{\sphinxbfcode{\sphinxupquote{class }}\sphinxbfcode{\sphinxupquote{\_fixAppleCollapse}}}{}{}~\begin{quote}\begin{description}
\item[{Extends}] \leavevmode{\hyperref[\detokenize{reference/javascript_api:website.content.snippets.animation.Animation}]{\sphinxcrossref{
Animation
}}}
\end{description}\end{quote}

This is a fix for apple device (\textless{}= IPhone 4, IPad 2)
Standard bootstrap requires data-toggle=’collapse’ element to be \textless{}a/\textgreater{} tags.
Unfortunatly one snippet uses a \textless{}div/\textgreater{} tag instead. The fix forces an empty
click handler on these div, which allows standard bootstrap to work.

This should be removed in a future odoo snippets refactoring.

\end{fulllineitems}


\end{fulllineitems}


\end{fulllineitems}


\end{fulllineitems}

\phantomsection\label{\detokenize{reference/javascript_api:module-web.CrashManager}}

\begin{fulllineitems}
\phantomsection\label{\detokenize{reference/javascript_api:web.CrashManager}}\pysigline{\sphinxbfcode{\sphinxupquote{module }}\sphinxbfcode{\sphinxupquote{web.CrashManager}}}~~\begin{quote}\begin{description}
\item[{Exports}] \leavevmode{\hyperref[\detokenize{reference/javascript_api:web.CrashManager.CrashManager}]{\sphinxcrossref{
CrashManager
}}}
\item[{Depends On}] \leavevmode\begin{itemize}
\item {} {\hyperref[\detokenize{reference/javascript_api:web.Dialog}]{\sphinxcrossref{
web.Dialog
}}}
\item {} {\hyperref[\detokenize{reference/javascript_api:web.ajax}]{\sphinxcrossref{
web.ajax
}}}
\item {} {\hyperref[\detokenize{reference/javascript_api:web.core}]{\sphinxcrossref{
web.core
}}}
\end{itemize}

\end{description}\end{quote}


\begin{fulllineitems}
\phantomsection\label{\detokenize{reference/javascript_api:ExceptionHandler}}\pysigline{\sphinxbfcode{\sphinxupquote{namespace }}\sphinxbfcode{\sphinxupquote{ExceptionHandler}}}
An interface to implement to handle exceptions. Register implementation in instance.web.crash\_manager\_registry.


\begin{fulllineitems}
\phantomsection\label{\detokenize{reference/javascript_api:display}}\pysiglinewithargsret{\sphinxbfcode{\sphinxupquote{function }}\sphinxbfcode{\sphinxupquote{display}}}{}{}
Called to inform to display the widget, if necessary. A typical way would be to implement
this interface in a class extending instance.web.Dialog and simply display the dialog in this
method.

\end{fulllineitems}


\end{fulllineitems}



\begin{fulllineitems}
\phantomsection\label{\detokenize{reference/javascript_api:RedirectWarningHandler}}\pysiglinewithargsret{\sphinxbfcode{\sphinxupquote{class }}\sphinxbfcode{\sphinxupquote{RedirectWarningHandler}}}{\emph{parent}, \emph{error}}{}~\begin{quote}\begin{description}
\item[{Extends}] \leavevmode{\hyperref[\detokenize{reference/javascript_api:web.Dialog.Dialog}]{\sphinxcrossref{
Dialog
}}}
\item[{Mixes}] \leavevmode\begin{itemize}
\item {} {\hyperref[\detokenize{reference/javascript_api:web.CrashManager.ExceptionHandler}]{\sphinxcrossref{
ExceptionHandler
}}}
\end{itemize}

\item[{Parameters}] \leavevmode\begin{itemize}

\sphinxstylestrong{parent}

\sphinxstylestrong{error}
\end{itemize}

\end{description}\end{quote}

Handle redirection warnings, which behave more or less like a regular
warning, with an additional redirection button.

\end{fulllineitems}



\begin{fulllineitems}
\phantomsection\label{\detokenize{reference/javascript_api:CrashManager}}\pysiglinewithargsret{\sphinxbfcode{\sphinxupquote{class }}\sphinxbfcode{\sphinxupquote{CrashManager}}}{}{}~\begin{quote}\begin{description}
\item[{Extends}] \leavevmode{\hyperref[\detokenize{reference/javascript_api:web.Class.Class}]{\sphinxcrossref{
Class
}}}
\end{description}\end{quote}

\end{fulllineitems}


\end{fulllineitems}

\phantomsection\label{\detokenize{reference/javascript_api:module-mail.ThreadField}}

\begin{fulllineitems}
\phantomsection\label{\detokenize{reference/javascript_api:mail.ThreadField}}\pysigline{\sphinxbfcode{\sphinxupquote{module }}\sphinxbfcode{\sphinxupquote{mail.ThreadField}}}~~\begin{quote}\begin{description}
\item[{Exports}] \leavevmode{\hyperref[\detokenize{reference/javascript_api:mail.ThreadField.ThreadField}]{\sphinxcrossref{
ThreadField
}}}
\item[{Depends On}] \leavevmode\begin{itemize}
\item {} {\hyperref[\detokenize{reference/javascript_api:mail.ChatThread}]{\sphinxcrossref{
mail.ChatThread
}}}
\item {} {\hyperref[\detokenize{reference/javascript_api:mail.chat_mixin}]{\sphinxcrossref{
mail.chat\_mixin
}}}
\item {} {\hyperref[\detokenize{reference/javascript_api:web.AbstractField}]{\sphinxcrossref{
web.AbstractField
}}}
\item {} {\hyperref[\detokenize{reference/javascript_api:web.concurrency}]{\sphinxcrossref{
web.concurrency
}}}
\item {} {\hyperref[\detokenize{reference/javascript_api:web.core}]{\sphinxcrossref{
web.core
}}}
\item {} {\hyperref[\detokenize{reference/javascript_api:web.field_registry}]{\sphinxcrossref{
web.field\_registry
}}}
\end{itemize}

\end{description}\end{quote}


\begin{fulllineitems}
\phantomsection\label{\detokenize{reference/javascript_api:ThreadField}}\pysiglinewithargsret{\sphinxbfcode{\sphinxupquote{class }}\sphinxbfcode{\sphinxupquote{ThreadField}}}{}{}~\begin{quote}\begin{description}
\item[{Extends}] \leavevmode{\hyperref[\detokenize{reference/javascript_api:web.AbstractField.AbstractField}]{\sphinxcrossref{
AbstractField
}}}
\item[{Mixes}] \leavevmode\begin{itemize}
\item {} {\hyperref[\detokenize{reference/javascript_api:mail.chat_mixin.ChatMixin}]{\sphinxcrossref{
ChatMixin
}}}
\end{itemize}

\end{description}\end{quote}

\end{fulllineitems}


\end{fulllineitems}

\phantomsection\label{\detokenize{reference/javascript_api:module-web.KanbanRenderer}}

\begin{fulllineitems}
\phantomsection\label{\detokenize{reference/javascript_api:web.KanbanRenderer}}\pysigline{\sphinxbfcode{\sphinxupquote{module }}\sphinxbfcode{\sphinxupquote{web.KanbanRenderer}}}~~\begin{quote}\begin{description}
\item[{Exports}] \leavevmode{\hyperref[\detokenize{reference/javascript_api:web.KanbanRenderer.KanbanRenderer}]{\sphinxcrossref{
KanbanRenderer
}}}
\item[{Depends On}] \leavevmode\begin{itemize}
\item {} {\hyperref[\detokenize{reference/javascript_api:web.BasicRenderer}]{\sphinxcrossref{
web.BasicRenderer
}}}
\item {} {\hyperref[\detokenize{reference/javascript_api:web.KanbanColumn}]{\sphinxcrossref{
web.KanbanColumn
}}}
\item {} {\hyperref[\detokenize{reference/javascript_api:web.KanbanRecord}]{\sphinxcrossref{
web.KanbanRecord
}}}
\item {} {\hyperref[\detokenize{reference/javascript_api:web.QWeb}]{\sphinxcrossref{
web.QWeb
}}}
\item {} {\hyperref[\detokenize{reference/javascript_api:web.core}]{\sphinxcrossref{
web.core
}}}
\item {} {\hyperref[\detokenize{reference/javascript_api:web.kanban_quick_create}]{\sphinxcrossref{
web.kanban\_quick\_create
}}}
\item {} {\hyperref[\detokenize{reference/javascript_api:web.session}]{\sphinxcrossref{
web.session
}}}
\item {} {\hyperref[\detokenize{reference/javascript_api:web.utils}]{\sphinxcrossref{
web.utils
}}}
\end{itemize}

\end{description}\end{quote}


\begin{fulllineitems}
\phantomsection\label{\detokenize{reference/javascript_api:KanbanRenderer}}\pysiglinewithargsret{\sphinxbfcode{\sphinxupquote{class }}\sphinxbfcode{\sphinxupquote{KanbanRenderer}}}{}{}~\begin{quote}\begin{description}
\item[{Extends}] \leavevmode{\hyperref[\detokenize{reference/javascript_api:web.BasicRenderer.BasicRenderer}]{\sphinxcrossref{
BasicRenderer
}}}
\end{description}\end{quote}


\begin{fulllineitems}
\phantomsection\label{\detokenize{reference/javascript_api:addQuickCreate}}\pysiglinewithargsret{\sphinxbfcode{\sphinxupquote{method }}\sphinxbfcode{\sphinxupquote{addQuickCreate}}}{}{}
Displays the quick create record in the active column

\end{fulllineitems}



\begin{fulllineitems}
\phantomsection\label{\detokenize{reference/javascript_api:quickCreateToggleFold}}\pysiglinewithargsret{\sphinxbfcode{\sphinxupquote{method }}\sphinxbfcode{\sphinxupquote{quickCreateToggleFold}}}{}{}
Toggle fold/unfold the Column quick create widget

\end{fulllineitems}



\begin{fulllineitems}
\phantomsection\label{\detokenize{reference/javascript_api:removeWidget}}\pysiglinewithargsret{\sphinxbfcode{\sphinxupquote{method }}\sphinxbfcode{\sphinxupquote{removeWidget}}}{\emph{widget}}{}
Removes a widget (record if ungrouped, column if grouped) from the view.
\begin{quote}\begin{description}
\item[{Parameters}] \leavevmode\begin{itemize}

\sphinxstylestrong{widget} ({\hyperref[\detokenize{reference/javascript_api:Widget}]{\sphinxcrossref{\sphinxstyleliteralemphasis{\sphinxupquote{Widget}}}}}) \textendash{} the instance of the widget to remove
\end{itemize}

\end{description}\end{quote}

\end{fulllineitems}



\begin{fulllineitems}
\phantomsection\label{\detokenize{reference/javascript_api:updateColumn}}\pysiglinewithargsret{\sphinxbfcode{\sphinxupquote{method }}\sphinxbfcode{\sphinxupquote{updateColumn}}}{\emph{localID}}{}
Overrides to restore the left property and the scrollTop on the updated
column, and to enable the swipe handlers
\begin{quote}\begin{description}
\item[{Parameters}] \leavevmode\begin{itemize}

\sphinxstylestrong{localID}
\end{itemize}

\end{description}\end{quote}

\end{fulllineitems}



\begin{fulllineitems}
\phantomsection\label{\detokenize{reference/javascript_api:updateRecord}}\pysiglinewithargsret{\sphinxbfcode{\sphinxupquote{method }}\sphinxbfcode{\sphinxupquote{updateRecord}}}{\emph{recordState}}{}
Updates a given record with its new state.
\begin{quote}\begin{description}
\item[{Parameters}] \leavevmode\begin{itemize}

\sphinxstylestrong{recordState} (\sphinxstyleliteralemphasis{\sphinxupquote{Object}})
\end{itemize}

\end{description}\end{quote}

\end{fulllineitems}


\end{fulllineitems}


\end{fulllineitems}

\phantomsection\label{\detokenize{reference/javascript_api:module-web.ajax}}

\begin{fulllineitems}
\phantomsection\label{\detokenize{reference/javascript_api:web.ajax}}\pysigline{\sphinxbfcode{\sphinxupquote{module }}\sphinxbfcode{\sphinxupquote{web.ajax}}}~~\begin{quote}\begin{description}
\item[{Exports}] \leavevmode{\hyperref[\detokenize{reference/javascript_api:web.ajax.ajax}]{\sphinxcrossref{
ajax
}}}
\item[{Depends On}] \leavevmode\begin{itemize}
\item {} {\hyperref[\detokenize{reference/javascript_api:web.core}]{\sphinxcrossref{
web.core
}}}
\item {} {\hyperref[\detokenize{reference/javascript_api:web.time}]{\sphinxcrossref{
web.time
}}}
\item {} {\hyperref[\detokenize{reference/javascript_api:web.utils}]{\sphinxcrossref{
web.utils
}}}
\end{itemize}

\end{description}\end{quote}


\begin{fulllineitems}
\phantomsection\label{\detokenize{reference/javascript_api:loadXML}}\pysigline{\sphinxbfcode{\sphinxupquote{namespace }}\sphinxbfcode{\sphinxupquote{loadXML}}}
Loads an XML file according to the given URL and adds its associated qweb
templates to the given qweb engine. The function can also be used to get
the deferred which indicates when all the calls to the function are finished.

Note: “all the calls” = the calls that happened before the current no-args
one + the calls that will happen after but when the previous ones are not
finished yet.

\end{fulllineitems}



\begin{fulllineitems}
\phantomsection\label{\detokenize{reference/javascript_api:loadLibs}}\pysiglinewithargsret{\sphinxbfcode{\sphinxupquote{function }}\sphinxbfcode{\sphinxupquote{loadLibs}}}{\emph{libs}}{{ $\rightarrow$ Deferred}}
Loads the given js and css libraries. Note that the ajax loadJS and loadCSS methods
don’t do anything if the given file is already loaded.
\begin{quote}\begin{description}
\item[{Parameters}] \leavevmode\begin{itemize}

\sphinxstylestrong{libs} ({\hyperref[\detokenize{reference/javascript_api:web.ajax.LoadLibsLibs}]{\sphinxcrossref{\sphinxstyleliteralemphasis{\sphinxupquote{LoadLibsLibs}}}}})
\end{itemize}

\item[{Return Type}] \leavevmode
\sphinxstyleliteralemphasis{\sphinxupquote{Deferred}}

\end{description}\end{quote}


\begin{fulllineitems}
\phantomsection\label{\detokenize{reference/javascript_api:LoadLibsLibs}}\pysiglinewithargsret{\sphinxbfcode{\sphinxupquote{class }}\sphinxbfcode{\sphinxupquote{LoadLibsLibs}}}{}{}~

\begin{fulllineitems}
\phantomsection\label{\detokenize{reference/javascript_api:cssLibs}}\pysigline{\sphinxbfcode{\sphinxupquote{attribute }}\sphinxbfcode{\sphinxupquote{cssLibs}} Array\textless{}string\textgreater{}}~\begin{description}
\item[{A list of css files, to be loaded in}] \leavevmode
parallel

\end{description}

\end{fulllineitems}


\end{fulllineitems}


\end{fulllineitems}



\begin{fulllineitems}
\phantomsection\label{\detokenize{reference/javascript_api:ajax}}\pysigline{\sphinxbfcode{\sphinxupquote{namespace }}\sphinxbfcode{\sphinxupquote{ajax}}}~

\begin{fulllineitems}
\phantomsection\label{\detokenize{reference/javascript_api:loadXML}}\pysigline{\sphinxbfcode{\sphinxupquote{namespace }}\sphinxbfcode{\sphinxupquote{loadXML}}}
Loads an XML file according to the given URL and adds its associated qweb
templates to the given qweb engine. The function can also be used to get
the deferred which indicates when all the calls to the function are finished.

Note: “all the calls” = the calls that happened before the current no-args
one + the calls that will happen after but when the previous ones are not
finished yet.

\end{fulllineitems}



\begin{fulllineitems}
\phantomsection\label{\detokenize{reference/javascript_api:loadLibs}}\pysiglinewithargsret{\sphinxbfcode{\sphinxupquote{function }}\sphinxbfcode{\sphinxupquote{loadLibs}}}{\emph{libs}}{{ $\rightarrow$ Deferred}}
Loads the given js and css libraries. Note that the ajax loadJS and loadCSS methods
don’t do anything if the given file is already loaded.
\begin{quote}\begin{description}
\item[{Parameters}] \leavevmode\begin{itemize}

\sphinxstylestrong{libs} ({\hyperref[\detokenize{reference/javascript_api:web.ajax.LoadLibsLibs}]{\sphinxcrossref{\sphinxstyleliteralemphasis{\sphinxupquote{LoadLibsLibs}}}}})
\end{itemize}

\item[{Return Type}] \leavevmode
\sphinxstyleliteralemphasis{\sphinxupquote{Deferred}}

\end{description}\end{quote}


\begin{fulllineitems}
\phantomsection\label{\detokenize{reference/javascript_api:LoadLibsLibs}}\pysiglinewithargsret{\sphinxbfcode{\sphinxupquote{class }}\sphinxbfcode{\sphinxupquote{LoadLibsLibs}}}{}{}~

\begin{fulllineitems}
\phantomsection\label{\detokenize{reference/javascript_api:cssLibs}}\pysigline{\sphinxbfcode{\sphinxupquote{attribute }}\sphinxbfcode{\sphinxupquote{cssLibs}} Array\textless{}string\textgreater{}}~\begin{description}
\item[{A list of css files, to be loaded in}] \leavevmode
parallel

\end{description}

\end{fulllineitems}


\end{fulllineitems}


\end{fulllineitems}



\begin{fulllineitems}
\phantomsection\label{\detokenize{reference/javascript_api:get_file}}\pysiglinewithargsret{\sphinxbfcode{\sphinxupquote{function }}\sphinxbfcode{\sphinxupquote{get\_file}}}{\emph{options}}{{ $\rightarrow$ boolean}}
Cooperative file download implementation, for ajaxy APIs.

Requires that the server side implements an httprequest correctly
setting the \sphinxcode{\sphinxupquote{fileToken}} cookie to the value provided as the \sphinxcode{\sphinxupquote{token}}
parameter. The cookie \sphinxstyleemphasis{must} be set on the \sphinxcode{\sphinxupquote{/}} path and \sphinxstyleemphasis{must not} be
\sphinxcode{\sphinxupquote{httpOnly}}.

It would probably also be a good idea for the response to use a
\sphinxcode{\sphinxupquote{Content-Disposition: attachment}} header, especially if the MIME is a
“known” type (e.g. text/plain, or for some browsers application/json
\begin{quote}\begin{description}
\item[{Parameters}] \leavevmode\begin{itemize}

\sphinxstylestrong{options} ({\hyperref[\detokenize{reference/javascript_api:web.ajax.GetFileOptions}]{\sphinxcrossref{\sphinxstyleliteralemphasis{\sphinxupquote{GetFileOptions}}}}})
\end{itemize}

\item[{Returns}] \leavevmode
a false value means that a popup window was blocked. This
  mean that we probably need to inform the user that something needs to be
  changed to make it work.

\item[{Return Type}] \leavevmode
\sphinxstyleliteralemphasis{\sphinxupquote{boolean}}

\end{description}\end{quote}


\begin{fulllineitems}
\phantomsection\label{\detokenize{reference/javascript_api:GetFileOptions}}\pysiglinewithargsret{\sphinxbfcode{\sphinxupquote{class }}\sphinxbfcode{\sphinxupquote{GetFileOptions}}}{}{}~

\begin{fulllineitems}
\phantomsection\label{\detokenize{reference/javascript_api:url}}\pysigline{\sphinxbfcode{\sphinxupquote{attribute }}\sphinxbfcode{\sphinxupquote{url}} String}
used to dynamically create a form

\end{fulllineitems}



\begin{fulllineitems}
\phantomsection\label{\detokenize{reference/javascript_api:data}}\pysigline{\sphinxbfcode{\sphinxupquote{attribute }}\sphinxbfcode{\sphinxupquote{data}} Object}
data to add to the form submission. If can be used without a form, in which case a form is created from scratch. Otherwise, added to form data

\end{fulllineitems}



\begin{fulllineitems}
\phantomsection\label{\detokenize{reference/javascript_api:form}}\pysigline{\sphinxbfcode{\sphinxupquote{attribute }}\sphinxbfcode{\sphinxupquote{form}} HTMLFormElement}
the form to submit in order to fetch the file

\end{fulllineitems}



\begin{fulllineitems}
\phantomsection\label{\detokenize{reference/javascript_api:success}}\pysigline{\sphinxbfcode{\sphinxupquote{attribute }}\sphinxbfcode{\sphinxupquote{success}} Function}
callback in case of download success

\end{fulllineitems}



\begin{fulllineitems}
\phantomsection\label{\detokenize{reference/javascript_api:error}}\pysigline{\sphinxbfcode{\sphinxupquote{attribute }}\sphinxbfcode{\sphinxupquote{error}} Function}
callback in case of request error, provided with the error body

\end{fulllineitems}



\begin{fulllineitems}
\phantomsection\label{\detokenize{reference/javascript_api:complete}}\pysigline{\sphinxbfcode{\sphinxupquote{attribute }}\sphinxbfcode{\sphinxupquote{complete}} Function}
called after both \sphinxcode{\sphinxupquote{success}} and \sphinxcode{\sphinxupquote{error}} callbacks have executed

\end{fulllineitems}


\end{fulllineitems}


\end{fulllineitems}


\end{fulllineitems}



\begin{fulllineitems}
\phantomsection\label{\detokenize{reference/javascript_api:get_file}}\pysiglinewithargsret{\sphinxbfcode{\sphinxupquote{function }}\sphinxbfcode{\sphinxupquote{get\_file}}}{\emph{options}}{{ $\rightarrow$ boolean}}
Cooperative file download implementation, for ajaxy APIs.

Requires that the server side implements an httprequest correctly
setting the \sphinxcode{\sphinxupquote{fileToken}} cookie to the value provided as the \sphinxcode{\sphinxupquote{token}}
parameter. The cookie \sphinxstyleemphasis{must} be set on the \sphinxcode{\sphinxupquote{/}} path and \sphinxstyleemphasis{must not} be
\sphinxcode{\sphinxupquote{httpOnly}}.

It would probably also be a good idea for the response to use a
\sphinxcode{\sphinxupquote{Content-Disposition: attachment}} header, especially if the MIME is a
“known” type (e.g. text/plain, or for some browsers application/json
\begin{quote}\begin{description}
\item[{Parameters}] \leavevmode\begin{itemize}

\sphinxstylestrong{options} ({\hyperref[\detokenize{reference/javascript_api:web.ajax.GetFileOptions}]{\sphinxcrossref{\sphinxstyleliteralemphasis{\sphinxupquote{GetFileOptions}}}}})
\end{itemize}

\item[{Returns}] \leavevmode
a false value means that a popup window was blocked. This
  mean that we probably need to inform the user that something needs to be
  changed to make it work.

\item[{Return Type}] \leavevmode
\sphinxstyleliteralemphasis{\sphinxupquote{boolean}}

\end{description}\end{quote}


\begin{fulllineitems}
\phantomsection\label{\detokenize{reference/javascript_api:GetFileOptions}}\pysiglinewithargsret{\sphinxbfcode{\sphinxupquote{class }}\sphinxbfcode{\sphinxupquote{GetFileOptions}}}{}{}~

\begin{fulllineitems}
\phantomsection\label{\detokenize{reference/javascript_api:url}}\pysigline{\sphinxbfcode{\sphinxupquote{attribute }}\sphinxbfcode{\sphinxupquote{url}} String}
used to dynamically create a form

\end{fulllineitems}



\begin{fulllineitems}
\phantomsection\label{\detokenize{reference/javascript_api:data}}\pysigline{\sphinxbfcode{\sphinxupquote{attribute }}\sphinxbfcode{\sphinxupquote{data}} Object}
data to add to the form submission. If can be used without a form, in which case a form is created from scratch. Otherwise, added to form data

\end{fulllineitems}



\begin{fulllineitems}
\phantomsection\label{\detokenize{reference/javascript_api:form}}\pysigline{\sphinxbfcode{\sphinxupquote{attribute }}\sphinxbfcode{\sphinxupquote{form}} HTMLFormElement}
the form to submit in order to fetch the file

\end{fulllineitems}



\begin{fulllineitems}
\phantomsection\label{\detokenize{reference/javascript_api:success}}\pysigline{\sphinxbfcode{\sphinxupquote{attribute }}\sphinxbfcode{\sphinxupquote{success}} Function}
callback in case of download success

\end{fulllineitems}



\begin{fulllineitems}
\phantomsection\label{\detokenize{reference/javascript_api:error}}\pysigline{\sphinxbfcode{\sphinxupquote{attribute }}\sphinxbfcode{\sphinxupquote{error}} Function}
callback in case of request error, provided with the error body

\end{fulllineitems}



\begin{fulllineitems}
\phantomsection\label{\detokenize{reference/javascript_api:complete}}\pysigline{\sphinxbfcode{\sphinxupquote{attribute }}\sphinxbfcode{\sphinxupquote{complete}} Function}
called after both \sphinxcode{\sphinxupquote{success}} and \sphinxcode{\sphinxupquote{error}} callbacks have executed

\end{fulllineitems}


\end{fulllineitems}


\end{fulllineitems}


\end{fulllineitems}

\phantomsection\label{\detokenize{reference/javascript_api:module-website.backend.dashboard}}

\begin{fulllineitems}
\phantomsection\label{\detokenize{reference/javascript_api:website.backend.dashboard}}\pysigline{\sphinxbfcode{\sphinxupquote{module }}\sphinxbfcode{\sphinxupquote{website.backend.dashboard}}}~~\begin{quote}\begin{description}
\item[{Exports}] \leavevmode{\hyperref[\detokenize{reference/javascript_api:website.backend.dashboard.Dashboard}]{\sphinxcrossref{
Dashboard
}}}
\item[{Depends On}] \leavevmode\begin{itemize}
\item {} {\hyperref[\detokenize{reference/javascript_api:web.ControlPanelMixin}]{\sphinxcrossref{
web.ControlPanelMixin
}}}
\item {} {\hyperref[\detokenize{reference/javascript_api:web.Dialog}]{\sphinxcrossref{
web.Dialog
}}}
\item {} {\hyperref[\detokenize{reference/javascript_api:web.Widget}]{\sphinxcrossref{
web.Widget
}}}
\item {} {\hyperref[\detokenize{reference/javascript_api:web.ajax}]{\sphinxcrossref{
web.ajax
}}}
\item {} {\hyperref[\detokenize{reference/javascript_api:web.core}]{\sphinxcrossref{
web.core
}}}
\item {} {\hyperref[\detokenize{reference/javascript_api:web.field_utils}]{\sphinxcrossref{
web.field\_utils
}}}
\item {} {\hyperref[\detokenize{reference/javascript_api:web.local_storage}]{\sphinxcrossref{
web.local\_storage
}}}
\item {} {\hyperref[\detokenize{reference/javascript_api:web.session}]{\sphinxcrossref{
web.session
}}}
\item {} 
web.web\_client

\end{itemize}

\end{description}\end{quote}


\begin{fulllineitems}
\phantomsection\label{\detokenize{reference/javascript_api:Dashboard}}\pysiglinewithargsret{\sphinxbfcode{\sphinxupquote{class }}\sphinxbfcode{\sphinxupquote{Dashboard}}}{\emph{parent}, \emph{context}}{}~\begin{quote}\begin{description}
\item[{Extends}] \leavevmode{\hyperref[\detokenize{reference/javascript_api:web.Widget.Widget}]{\sphinxcrossref{
Widget
}}}
\item[{Mixes}] \leavevmode\begin{itemize}
\item {} {\hyperref[\detokenize{reference/javascript_api:web.ControlPanelMixin.ControlPanelMixin}]{\sphinxcrossref{
ControlPanelMixin
}}}
\end{itemize}

\item[{Parameters}] \leavevmode\begin{itemize}

\sphinxstylestrong{parent}

\sphinxstylestrong{context}
\end{itemize}

\end{description}\end{quote}


\begin{fulllineitems}
\phantomsection\label{\detokenize{reference/javascript_api:fetch_data}}\pysiglinewithargsret{\sphinxbfcode{\sphinxupquote{method }}\sphinxbfcode{\sphinxupquote{fetch\_data}}}{}{}
Fetches dashboard data

\end{fulllineitems}


\end{fulllineitems}


\end{fulllineitems}

\phantomsection\label{\detokenize{reference/javascript_api:module-website_sale_comparison.comparison}}

\begin{fulllineitems}
\phantomsection\label{\detokenize{reference/javascript_api:website_sale_comparison.comparison}}\pysigline{\sphinxbfcode{\sphinxupquote{module }}\sphinxbfcode{\sphinxupquote{website\_sale\_comparison.comparison}}}~~\begin{quote}\begin{description}
\item[{Exports}] \leavevmode{\hyperref[\detokenize{reference/javascript_api:website_sale_comparison.comparison.}]{\sphinxcrossref{
\textless{}anonymous\textgreater{}
}}}
\item[{Depends On}] \leavevmode\begin{itemize}
\item {} {\hyperref[\detokenize{reference/javascript_api:web.Widget}]{\sphinxcrossref{
web.Widget
}}}
\item {} {\hyperref[\detokenize{reference/javascript_api:web.ajax}]{\sphinxcrossref{
web.ajax
}}}
\item {} {\hyperref[\detokenize{reference/javascript_api:web.core}]{\sphinxcrossref{
web.core
}}}
\item {} {\hyperref[\detokenize{reference/javascript_api:web.utils}]{\sphinxcrossref{
web.utils
}}}
\item {} {\hyperref[\detokenize{reference/javascript_api:web_editor.base}]{\sphinxcrossref{
web\_editor.base
}}}
\item {} {\hyperref[\detokenize{reference/javascript_api:website_sale.utils}]{\sphinxcrossref{
website\_sale.utils
}}}
\end{itemize}

\end{description}\end{quote}


\begin{fulllineitems}
\phantomsection\label{\detokenize{reference/javascript_api:website_sale_comparison.comparison.}}\pysigline{\sphinxbfcode{\sphinxupquote{namespace }}\sphinxbfcode{\sphinxupquote{}}}
\end{fulllineitems}


\end{fulllineitems}

\phantomsection\label{\detokenize{reference/javascript_api:module-website_event.registration_form.instance}}

\begin{fulllineitems}
\phantomsection\label{\detokenize{reference/javascript_api:website_event.registration_form.instance}}\pysigline{\sphinxbfcode{\sphinxupquote{module }}\sphinxbfcode{\sphinxupquote{website\_event.registration\_form.instance}}}~~\begin{quote}\begin{description}
\item[{Exports}] \leavevmode{\hyperref[\detokenize{reference/javascript_api:website_event.registration_form.instance.}]{\sphinxcrossref{
\textless{}anonymous\textgreater{}
}}}
\item[{Depends On}] \leavevmode\begin{itemize}
\item {} {\hyperref[\detokenize{reference/javascript_api:website_event.website_event}]{\sphinxcrossref{
website\_event.website\_event
}}}
\end{itemize}

\end{description}\end{quote}


\begin{fulllineitems}
\phantomsection\label{\detokenize{reference/javascript_api:website_event.registration_form.instance.}}\pysigline{\sphinxbfcode{\sphinxupquote{namespace }}\sphinxbfcode{\sphinxupquote{}}}
\end{fulllineitems}


\end{fulllineitems}

\phantomsection\label{\detokenize{reference/javascript_api:module-web_editor.editor}}

\begin{fulllineitems}
\phantomsection\label{\detokenize{reference/javascript_api:web_editor.editor}}\pysigline{\sphinxbfcode{\sphinxupquote{module }}\sphinxbfcode{\sphinxupquote{web\_editor.editor}}}~~\begin{quote}\begin{description}
\item[{Exports}] \leavevmode{\hyperref[\detokenize{reference/javascript_api:web_editor.editor.}]{\sphinxcrossref{
\textless{}anonymous\textgreater{}
}}}
\item[{Depends On}] \leavevmode\begin{itemize}
\item {} {\hyperref[\detokenize{reference/javascript_api:web.Dialog}]{\sphinxcrossref{
web.Dialog
}}}
\item {} {\hyperref[\detokenize{reference/javascript_api:web.Widget}]{\sphinxcrossref{
web.Widget
}}}
\item {} {\hyperref[\detokenize{reference/javascript_api:web.core}]{\sphinxcrossref{
web.core
}}}
\item {} {\hyperref[\detokenize{reference/javascript_api:web_editor.rte}]{\sphinxcrossref{
web\_editor.rte
}}}
\item {} {\hyperref[\detokenize{reference/javascript_api:web_editor.snippet.editor}]{\sphinxcrossref{
web\_editor.snippet.editor
}}}
\end{itemize}

\end{description}\end{quote}


\begin{fulllineitems}
\phantomsection\label{\detokenize{reference/javascript_api:web_editor.editor.}}\pysigline{\sphinxbfcode{\sphinxupquote{namespace }}\sphinxbfcode{\sphinxupquote{}}}
\end{fulllineitems}


\end{fulllineitems}

\phantomsection\label{\detokenize{reference/javascript_api:module-web.BasicRenderer}}

\begin{fulllineitems}
\phantomsection\label{\detokenize{reference/javascript_api:web.BasicRenderer}}\pysigline{\sphinxbfcode{\sphinxupquote{module }}\sphinxbfcode{\sphinxupquote{web.BasicRenderer}}}~~\begin{quote}\begin{description}
\item[{Exports}] \leavevmode{\hyperref[\detokenize{reference/javascript_api:web.BasicRenderer.BasicRenderer}]{\sphinxcrossref{
BasicRenderer
}}}
\item[{Depends On}] \leavevmode\begin{itemize}
\item {} {\hyperref[\detokenize{reference/javascript_api:web.AbstractRenderer}]{\sphinxcrossref{
web.AbstractRenderer
}}}
\item {} {\hyperref[\detokenize{reference/javascript_api:web.config}]{\sphinxcrossref{
web.config
}}}
\item {} {\hyperref[\detokenize{reference/javascript_api:web.core}]{\sphinxcrossref{
web.core
}}}
\item {} {\hyperref[\detokenize{reference/javascript_api:web.dom}]{\sphinxcrossref{
web.dom
}}}
\item {} {\hyperref[\detokenize{reference/javascript_api:web.widget_registry}]{\sphinxcrossref{
web.widget\_registry
}}}
\end{itemize}

\end{description}\end{quote}


\begin{fulllineitems}
\phantomsection\label{\detokenize{reference/javascript_api:BasicRenderer}}\pysiglinewithargsret{\sphinxbfcode{\sphinxupquote{class }}\sphinxbfcode{\sphinxupquote{BasicRenderer}}}{\emph{parent}, \emph{state}, \emph{params}}{}~\begin{quote}\begin{description}
\item[{Extends}] \leavevmode{\hyperref[\detokenize{reference/javascript_api:web.AbstractRenderer.AbstractRenderer}]{\sphinxcrossref{
AbstractRenderer
}}}
\item[{Parameters}] \leavevmode\begin{itemize}

\sphinxstylestrong{parent}

\sphinxstylestrong{state}

\sphinxstylestrong{params}
\end{itemize}

\end{description}\end{quote}


\begin{fulllineitems}
\phantomsection\label{\detokenize{reference/javascript_api:canBeSaved}}\pysiglinewithargsret{\sphinxbfcode{\sphinxupquote{method }}\sphinxbfcode{\sphinxupquote{canBeSaved}}}{\emph{recordID}}{{ $\rightarrow$ string{[}{]}}}
This method has two responsabilities: find every invalid fields in the
current view, and making sure that they are displayed as invalid, by
toggling the o\_form\_invalid css class. It has to be done both on the
widget, and on the label, if any.
\begin{quote}\begin{description}
\item[{Parameters}] \leavevmode\begin{itemize}

\sphinxstylestrong{recordID} (\sphinxstyleliteralemphasis{\sphinxupquote{string}})
\end{itemize}

\item[{Returns}] \leavevmode
the list of invalid field names

\item[{Return Type}] \leavevmode
\sphinxstyleliteralemphasis{\sphinxupquote{Array}}\textless{}\sphinxstyleliteralemphasis{\sphinxupquote{string}}\textgreater{}

\end{description}\end{quote}

\end{fulllineitems}



\begin{fulllineitems}
\phantomsection\label{\detokenize{reference/javascript_api:commitChanges}}\pysiglinewithargsret{\sphinxbfcode{\sphinxupquote{method }}\sphinxbfcode{\sphinxupquote{commitChanges}}}{\emph{recordID}}{{ $\rightarrow$ Deferred}}
Calls ‘commitChanges’ on all field widgets, so that they can notify the
environment with their current value (useful for widgets that can’t
detect when their value changes or that have to validate their changes
before notifying them).
\begin{quote}\begin{description}
\item[{Parameters}] \leavevmode\begin{itemize}

\sphinxstylestrong{recordID} (\sphinxstyleliteralemphasis{\sphinxupquote{string}})
\end{itemize}

\item[{Return Type}] \leavevmode
\sphinxstyleliteralemphasis{\sphinxupquote{Deferred}}

\end{description}\end{quote}

\end{fulllineitems}



\begin{fulllineitems}
\phantomsection\label{\detokenize{reference/javascript_api:confirmChange}}\pysiglinewithargsret{\sphinxbfcode{\sphinxupquote{method }}\sphinxbfcode{\sphinxupquote{confirmChange}}}{\emph{state}, \emph{id}, \emph{fields}, \emph{ev}}{{ $\rightarrow$ Deferred\textless{}AbstractField{[}{]}\textgreater{}}}
Updates the internal state of the renderer to the new state. By default,
this also implements the recomputation of the modifiers and their
application to the DOM and the reset of the field widgets if needed.

In case the given record is not found anymore, a whole re-rendering is
completed (possible if a change in a record caused an onchange which
erased the current record).

We could always rerender the view from scratch, but then it would not be
as efficient, and we might lose some local state, such as the input focus
cursor, or the scrolling position.
\begin{quote}\begin{description}
\item[{Parameters}] \leavevmode\begin{itemize}

\sphinxstylestrong{state} (\sphinxstyleliteralemphasis{\sphinxupquote{Object}})

\sphinxstylestrong{id} (\sphinxstyleliteralemphasis{\sphinxupquote{string}})

\sphinxstylestrong{fields} (\sphinxstyleliteralemphasis{\sphinxupquote{Array}}\textless{}\sphinxstyleliteralemphasis{\sphinxupquote{string}}\textgreater{})

\sphinxstylestrong{ev} (\sphinxstyleliteralemphasis{\sphinxupquote{OdooEvent}})
\end{itemize}

\item[{Returns}] \leavevmode
resolved with the list of widgets
                                     that have been reset

\item[{Return Type}] \leavevmode
\sphinxstyleliteralemphasis{\sphinxupquote{Deferred}}\textless{}\sphinxstyleliteralemphasis{\sphinxupquote{Array}}\textless{}{\hyperref[\detokenize{reference/javascript_api:AbstractField}]{\sphinxcrossref{\sphinxstyleliteralemphasis{\sphinxupquote{AbstractField}}}}}\textgreater{}\textgreater{}

\end{description}\end{quote}

\end{fulllineitems}



\begin{fulllineitems}
\phantomsection\label{\detokenize{reference/javascript_api:focusField}}\pysiglinewithargsret{\sphinxbfcode{\sphinxupquote{method }}\sphinxbfcode{\sphinxupquote{focusField}}}{\emph{id}, \emph{fieldName}, \emph{offset}}{}
Activates the widget and move the cursor to the given offset
\begin{quote}\begin{description}
\item[{Parameters}] \leavevmode\begin{itemize}

\sphinxstylestrong{id} (\sphinxstyleliteralemphasis{\sphinxupquote{string}})

\sphinxstylestrong{fieldName} (\sphinxstyleliteralemphasis{\sphinxupquote{string}})

\sphinxstylestrong{offset} (\sphinxstyleliteralemphasis{\sphinxupquote{integer}})
\end{itemize}

\end{description}\end{quote}

\end{fulllineitems}


\end{fulllineitems}


\end{fulllineitems}

\phantomsection\label{\detokenize{reference/javascript_api:module-website_event.website_event}}

\begin{fulllineitems}
\phantomsection\label{\detokenize{reference/javascript_api:website_event.website_event}}\pysigline{\sphinxbfcode{\sphinxupquote{module }}\sphinxbfcode{\sphinxupquote{website\_event.website\_event}}}~~\begin{quote}\begin{description}
\item[{Exports}] \leavevmode{\hyperref[\detokenize{reference/javascript_api:website_event.website_event.EventRegistrationForm}]{\sphinxcrossref{
EventRegistrationForm
}}}
\item[{Depends On}] \leavevmode\begin{itemize}
\item {} {\hyperref[\detokenize{reference/javascript_api:web.Widget}]{\sphinxcrossref{
web.Widget
}}}
\item {} {\hyperref[\detokenize{reference/javascript_api:web.ajax}]{\sphinxcrossref{
web.ajax
}}}
\end{itemize}

\end{description}\end{quote}


\begin{fulllineitems}
\phantomsection\label{\detokenize{reference/javascript_api:EventRegistrationForm}}\pysiglinewithargsret{\sphinxbfcode{\sphinxupquote{class }}\sphinxbfcode{\sphinxupquote{EventRegistrationForm}}}{}{}~\begin{quote}\begin{description}
\item[{Extends}] \leavevmode{\hyperref[\detokenize{reference/javascript_api:web.Widget.Widget}]{\sphinxcrossref{
Widget
}}}
\end{description}\end{quote}

\end{fulllineitems}


\end{fulllineitems}

\phantomsection\label{\detokenize{reference/javascript_api:module-web.config}}

\begin{fulllineitems}
\phantomsection\label{\detokenize{reference/javascript_api:web.config}}\pysigline{\sphinxbfcode{\sphinxupquote{module }}\sphinxbfcode{\sphinxupquote{web.config}}}~~\begin{quote}\begin{description}
\item[{Exports}] \leavevmode{\hyperref[\detokenize{reference/javascript_api:web.config.config}]{\sphinxcrossref{
config
}}}
\item[{Depends On}] \leavevmode\begin{itemize}
\item {} {\hyperref[\detokenize{reference/javascript_api:web.core}]{\sphinxcrossref{
web.core
}}}
\end{itemize}

\end{description}\end{quote}


\begin{fulllineitems}
\phantomsection\label{\detokenize{reference/javascript_api:_updateSizeProps}}\pysiglinewithargsret{\sphinxbfcode{\sphinxupquote{function }}\sphinxbfcode{\sphinxupquote{\_updateSizeProps}}}{}{}
Update the size dependant properties in the config object.  This method
should be called every time the size class changes.

\end{fulllineitems}



\begin{fulllineitems}
\phantomsection\label{\detokenize{reference/javascript_api:_getSizeClass}}\pysiglinewithargsret{\sphinxbfcode{\sphinxupquote{function }}\sphinxbfcode{\sphinxupquote{\_getSizeClass}}}{}{{ $\rightarrow$ integer}}
Return the current size class
\begin{quote}\begin{description}
\item[{Returns}] \leavevmode
a number between 0 and 3, included

\item[{Return Type}] \leavevmode
\sphinxstyleliteralemphasis{\sphinxupquote{integer}}

\end{description}\end{quote}

\end{fulllineitems}



\begin{fulllineitems}
\phantomsection\label{\detokenize{reference/javascript_api:config}}\pysigline{\sphinxbfcode{\sphinxupquote{namespace }}\sphinxbfcode{\sphinxupquote{config}}}~

\begin{fulllineitems}
\phantomsection\label{\detokenize{reference/javascript_api:debug}}\pysigline{\sphinxbfcode{\sphinxupquote{attribute }}\sphinxbfcode{\sphinxupquote{debug}} Boolean}
debug is a boolean flag.  It is only considered true if the flag is set
in the url

\end{fulllineitems}


\end{fulllineitems}


\end{fulllineitems}

\phantomsection\label{\detokenize{reference/javascript_api:module-web.FieldManagerMixin}}

\begin{fulllineitems}
\phantomsection\label{\detokenize{reference/javascript_api:web.FieldManagerMixin}}\pysigline{\sphinxbfcode{\sphinxupquote{module }}\sphinxbfcode{\sphinxupquote{web.FieldManagerMixin}}}~~\begin{quote}\begin{description}
\item[{Exports}] \leavevmode{\hyperref[\detokenize{reference/javascript_api:web.FieldManagerMixin.FieldManagerMixin}]{\sphinxcrossref{
FieldManagerMixin
}}}
\item[{Depends On}] \leavevmode\begin{itemize}
\item {} {\hyperref[\detokenize{reference/javascript_api:web.BasicModel}]{\sphinxcrossref{
web.BasicModel
}}}
\item {} {\hyperref[\detokenize{reference/javascript_api:web.concurrency}]{\sphinxcrossref{
web.concurrency
}}}
\end{itemize}

\end{description}\end{quote}


\begin{fulllineitems}
\phantomsection\label{\detokenize{reference/javascript_api:FieldManagerMixin}}\pysigline{\sphinxbfcode{\sphinxupquote{namespace }}\sphinxbfcode{\sphinxupquote{FieldManagerMixin}}}
\end{fulllineitems}


\end{fulllineitems}

\phantomsection\label{\detokenize{reference/javascript_api:module-payment_stripe.stripe}}

\begin{fulllineitems}
\phantomsection\label{\detokenize{reference/javascript_api:payment_stripe.stripe}}\pysigline{\sphinxbfcode{\sphinxupquote{module }}\sphinxbfcode{\sphinxupquote{payment\_stripe.stripe}}}~~\begin{quote}\begin{description}
\item[{Exports}] \leavevmode{\hyperref[\detokenize{reference/javascript_api:payment_stripe.stripe.}]{\sphinxcrossref{
\textless{}anonymous\textgreater{}
}}}
\item[{Depends On}] \leavevmode\begin{itemize}
\item {} {\hyperref[\detokenize{reference/javascript_api:web.ajax}]{\sphinxcrossref{
web.ajax
}}}
\item {} {\hyperref[\detokenize{reference/javascript_api:web.core}]{\sphinxcrossref{
web.core
}}}
\end{itemize}

\end{description}\end{quote}


\begin{fulllineitems}
\phantomsection\label{\detokenize{reference/javascript_api:payment_stripe.stripe.}}\pysigline{\sphinxbfcode{\sphinxupquote{namespace }}\sphinxbfcode{\sphinxupquote{}}}
\end{fulllineitems}


\end{fulllineitems}

\phantomsection\label{\detokenize{reference/javascript_api:module-web.rpc}}

\begin{fulllineitems}
\phantomsection\label{\detokenize{reference/javascript_api:web.rpc}}\pysigline{\sphinxbfcode{\sphinxupquote{module }}\sphinxbfcode{\sphinxupquote{web.rpc}}}~~\begin{quote}\begin{description}
\item[{Exports}] \leavevmode{\hyperref[\detokenize{reference/javascript_api:web.rpc.}]{\sphinxcrossref{
\textless{}anonymous\textgreater{}
}}}
\item[{Depends On}] \leavevmode\begin{itemize}
\item {} {\hyperref[\detokenize{reference/javascript_api:web.ajax}]{\sphinxcrossref{
web.ajax
}}}
\end{itemize}

\end{description}\end{quote}


\begin{fulllineitems}
\phantomsection\label{\detokenize{reference/javascript_api:web.rpc.}}\pysigline{\sphinxbfcode{\sphinxupquote{namespace }}\sphinxbfcode{\sphinxupquote{}}}~

\begin{fulllineitems}
\phantomsection\label{\detokenize{reference/javascript_api:query}}\pysiglinewithargsret{\sphinxbfcode{\sphinxupquote{function }}\sphinxbfcode{\sphinxupquote{query}}}{\emph{params}, \emph{options}}{{ $\rightarrow$ Deferred\textless{}any\textgreater{}}}
Perform a RPC.  Please note that this is not the preferred way to do a
rpc if you are in the context of a widget.  In that case, you should use
the this.\_rpc method.
\begin{quote}\begin{description}
\item[{Parameters}] \leavevmode\begin{itemize}

\sphinxstylestrong{params} (\sphinxstyleliteralemphasis{\sphinxupquote{Object}}) \textendash{} @see buildQuery for a description

\sphinxstylestrong{options} (\sphinxstyleliteralemphasis{\sphinxupquote{Object}})
\end{itemize}

\item[{Return Type}] \leavevmode
\sphinxstyleliteralemphasis{\sphinxupquote{Deferred}}\textless{}\sphinxstyleliteralemphasis{\sphinxupquote{any}}\textgreater{}

\end{description}\end{quote}

\end{fulllineitems}


\end{fulllineitems}


\end{fulllineitems}

\phantomsection\label{\detokenize{reference/javascript_api:module-hr_attendance.kiosk_confirm}}

\begin{fulllineitems}
\phantomsection\label{\detokenize{reference/javascript_api:hr_attendance.kiosk_confirm}}\pysigline{\sphinxbfcode{\sphinxupquote{module }}\sphinxbfcode{\sphinxupquote{hr\_attendance.kiosk\_confirm}}}~~\begin{quote}\begin{description}
\item[{Exports}] \leavevmode{\hyperref[\detokenize{reference/javascript_api:hr_attendance.kiosk_confirm.KioskConfirm}]{\sphinxcrossref{
KioskConfirm
}}}
\item[{Depends On}] \leavevmode\begin{itemize}
\item {} {\hyperref[\detokenize{reference/javascript_api:web.Widget}]{\sphinxcrossref{
web.Widget
}}}
\item {} {\hyperref[\detokenize{reference/javascript_api:web.core}]{\sphinxcrossref{
web.core
}}}
\end{itemize}

\end{description}\end{quote}


\begin{fulllineitems}
\phantomsection\label{\detokenize{reference/javascript_api:KioskConfirm}}\pysiglinewithargsret{\sphinxbfcode{\sphinxupquote{class }}\sphinxbfcode{\sphinxupquote{KioskConfirm}}}{\emph{parent}, \emph{action}}{}~\begin{quote}\begin{description}
\item[{Extends}] \leavevmode{\hyperref[\detokenize{reference/javascript_api:web.Widget.Widget}]{\sphinxcrossref{
Widget
}}}
\item[{Parameters}] \leavevmode\begin{itemize}

\sphinxstylestrong{parent}

\sphinxstylestrong{action}
\end{itemize}

\end{description}\end{quote}

\end{fulllineitems}


\end{fulllineitems}

\phantomsection\label{\detokenize{reference/javascript_api:module-point_of_sale.chrome}}

\begin{fulllineitems}
\phantomsection\label{\detokenize{reference/javascript_api:point_of_sale.chrome}}\pysigline{\sphinxbfcode{\sphinxupquote{module }}\sphinxbfcode{\sphinxupquote{point\_of\_sale.chrome}}}~~\begin{quote}\begin{description}
\item[{Exports}] \leavevmode{\hyperref[\detokenize{reference/javascript_api:point_of_sale.chrome.}]{\sphinxcrossref{
\textless{}anonymous\textgreater{}
}}}
\item[{Depends On}] \leavevmode\begin{itemize}
\item {} {\hyperref[\detokenize{reference/javascript_api:barcodes.BarcodeEvents}]{\sphinxcrossref{
barcodes.BarcodeEvents
}}}
\item {} {\hyperref[\detokenize{reference/javascript_api:point_of_sale.BaseWidget}]{\sphinxcrossref{
point\_of\_sale.BaseWidget
}}}
\item {} {\hyperref[\detokenize{reference/javascript_api:point_of_sale.gui}]{\sphinxcrossref{
point\_of\_sale.gui
}}}
\item {} {\hyperref[\detokenize{reference/javascript_api:point_of_sale.keyboard}]{\sphinxcrossref{
point\_of\_sale.keyboard
}}}
\item {} {\hyperref[\detokenize{reference/javascript_api:point_of_sale.models}]{\sphinxcrossref{
point\_of\_sale.models
}}}
\item {} {\hyperref[\detokenize{reference/javascript_api:web.CrashManager}]{\sphinxcrossref{
web.CrashManager
}}}
\item {} {\hyperref[\detokenize{reference/javascript_api:web.ajax}]{\sphinxcrossref{
web.ajax
}}}
\item {} {\hyperref[\detokenize{reference/javascript_api:web.core}]{\sphinxcrossref{
web.core
}}}
\end{itemize}

\end{description}\end{quote}


\begin{fulllineitems}
\phantomsection\label{\detokenize{reference/javascript_api:point_of_sale.chrome.}}\pysigline{\sphinxbfcode{\sphinxupquote{namespace }}\sphinxbfcode{\sphinxupquote{}}}
\end{fulllineitems}


\end{fulllineitems}

\phantomsection\label{\detokenize{reference/javascript_api:module-web.pyeval}}

\begin{fulllineitems}
\phantomsection\label{\detokenize{reference/javascript_api:web.pyeval}}\pysigline{\sphinxbfcode{\sphinxupquote{module }}\sphinxbfcode{\sphinxupquote{web.pyeval}}}~~\begin{quote}\begin{description}
\item[{Exports}] \leavevmode{\hyperref[\detokenize{reference/javascript_api:web.pyeval.}]{\sphinxcrossref{
\textless{}anonymous\textgreater{}
}}}
\item[{Depends On}] \leavevmode\begin{itemize}
\item {} {\hyperref[\detokenize{reference/javascript_api:web.core}]{\sphinxcrossref{
web.core
}}}
\item {} {\hyperref[\detokenize{reference/javascript_api:web.utils}]{\sphinxcrossref{
web.utils
}}}
\end{itemize}

\end{description}\end{quote}


\begin{fulllineitems}
\phantomsection\label{\detokenize{reference/javascript_api:tmxxx}}\pysiglinewithargsret{\sphinxbfcode{\sphinxupquote{function }}\sphinxbfcode{\sphinxupquote{tmxxx}}}{\emph{year}, \emph{month}, \emph{day}, \emph{hour}, \emph{minute}, \emph{second}, \emph{microsecond}}{}
Converts the stuff passed in into a valid date, applying overflows as needed
\begin{quote}\begin{description}
\item[{Parameters}] \leavevmode\begin{itemize}

\sphinxstylestrong{year}

\sphinxstylestrong{month}

\sphinxstylestrong{day}

\sphinxstylestrong{hour}

\sphinxstylestrong{minute}

\sphinxstylestrong{second}

\sphinxstylestrong{microsecond}
\end{itemize}

\end{description}\end{quote}

\end{fulllineitems}



\begin{fulllineitems}
\phantomsection\label{\detokenize{reference/javascript_api:ensure_evaluated}}\pysiglinewithargsret{\sphinxbfcode{\sphinxupquote{function }}\sphinxbfcode{\sphinxupquote{ensure\_evaluated}}}{\emph{args}, \emph{kwargs}}{}
If args or kwargs are unevaluated contexts or domains (compound or not),
evaluated them in-place.

Potentially mutates both parameters.
\begin{quote}\begin{description}
\item[{Parameters}] \leavevmode\begin{itemize}

\sphinxstylestrong{args}

\sphinxstylestrong{kwargs}
\end{itemize}

\end{description}\end{quote}

\end{fulllineitems}



\begin{fulllineitems}
\phantomsection\label{\detokenize{reference/javascript_api:get_normalized_domain}}\pysiglinewithargsret{\sphinxbfcode{\sphinxupquote{function }}\sphinxbfcode{\sphinxupquote{get\_normalized\_domain}}}{\emph{domain\_array}}{{ $\rightarrow$ Array}}
Returns a normalized copy of the given domain array. Normalization is
is making the implicit “\&” at the start of the domain explicit, e.g.
{[}A, B, C{]} would become {[}“\&”, “\&”, A, B, C{]}.
\begin{quote}\begin{description}
\item[{Parameters}] \leavevmode\begin{itemize}

\sphinxstylestrong{domain\_array} (\sphinxstyleliteralemphasis{\sphinxupquote{Array}})
\end{itemize}

\item[{Returns}] \leavevmode
normalized copy of the given array

\item[{Return Type}] \leavevmode
\sphinxstyleliteralemphasis{\sphinxupquote{Array}}

\end{description}\end{quote}

\end{fulllineitems}



\begin{fulllineitems}
\phantomsection\label{\detokenize{reference/javascript_api:context_today}}\pysiglinewithargsret{\sphinxbfcode{\sphinxupquote{function }}\sphinxbfcode{\sphinxupquote{context\_today}}}{}{{ $\rightarrow$ datetime.date}}
Returns the current local date, which means the date on the client (which can be different
compared to the date of the server).
\begin{quote}\begin{description}
\item[{Return Type}] \leavevmode
\sphinxstyleliteralemphasis{\sphinxupquote{datetime.date}}

\end{description}\end{quote}

\end{fulllineitems}



\begin{fulllineitems}
\phantomsection\label{\detokenize{reference/javascript_api:web.pyeval.}}\pysigline{\sphinxbfcode{\sphinxupquote{namespace }}\sphinxbfcode{\sphinxupquote{}}}~

\begin{fulllineitems}
\phantomsection\label{\detokenize{reference/javascript_api:ensure_evaluated}}\pysiglinewithargsret{\sphinxbfcode{\sphinxupquote{function }}\sphinxbfcode{\sphinxupquote{ensure\_evaluated}}}{\emph{args}, \emph{kwargs}}{}
If args or kwargs are unevaluated contexts or domains (compound or not),
evaluated them in-place.

Potentially mutates both parameters.
\begin{quote}\begin{description}
\item[{Parameters}] \leavevmode\begin{itemize}

\sphinxstylestrong{args}

\sphinxstylestrong{kwargs}
\end{itemize}

\end{description}\end{quote}

\end{fulllineitems}


\end{fulllineitems}


\end{fulllineitems}

\phantomsection\label{\detokenize{reference/javascript_api:module-web.AjaxService}}

\begin{fulllineitems}
\phantomsection\label{\detokenize{reference/javascript_api:web.AjaxService}}\pysigline{\sphinxbfcode{\sphinxupquote{module }}\sphinxbfcode{\sphinxupquote{web.AjaxService}}}~~\begin{quote}\begin{description}
\item[{Exports}] \leavevmode{\hyperref[\detokenize{reference/javascript_api:web.AjaxService.AjaxService}]{\sphinxcrossref{
AjaxService
}}}
\item[{Depends On}] \leavevmode\begin{itemize}
\item {} {\hyperref[\detokenize{reference/javascript_api:web.AbstractService}]{\sphinxcrossref{
web.AbstractService
}}}
\item {} {\hyperref[\detokenize{reference/javascript_api:web.session}]{\sphinxcrossref{
web.session
}}}
\end{itemize}

\end{description}\end{quote}


\begin{fulllineitems}
\phantomsection\label{\detokenize{reference/javascript_api:AjaxService}}\pysiglinewithargsret{\sphinxbfcode{\sphinxupquote{class }}\sphinxbfcode{\sphinxupquote{AjaxService}}}{}{}~\begin{quote}\begin{description}
\item[{Extends}] \leavevmode{\hyperref[\detokenize{reference/javascript_api:web.AbstractService.AbstractService}]{\sphinxcrossref{
AbstractService
}}}
\end{description}\end{quote}

\end{fulllineitems}


\end{fulllineitems}

\phantomsection\label{\detokenize{reference/javascript_api:module-web.BasicModel}}

\begin{fulllineitems}
\phantomsection\label{\detokenize{reference/javascript_api:web.BasicModel}}\pysigline{\sphinxbfcode{\sphinxupquote{module }}\sphinxbfcode{\sphinxupquote{web.BasicModel}}}~~\begin{quote}\begin{description}
\item[{Exports}] \leavevmode{\hyperref[\detokenize{reference/javascript_api:web.BasicModel.BasicModel}]{\sphinxcrossref{
BasicModel
}}}
\item[{Depends On}] \leavevmode\begin{itemize}
\item {} {\hyperref[\detokenize{reference/javascript_api:web.AbstractModel}]{\sphinxcrossref{
web.AbstractModel
}}}
\item {} {\hyperref[\detokenize{reference/javascript_api:web.Context}]{\sphinxcrossref{
web.Context
}}}
\item {} {\hyperref[\detokenize{reference/javascript_api:web.Domain}]{\sphinxcrossref{
web.Domain
}}}
\item {} {\hyperref[\detokenize{reference/javascript_api:web.concurrency}]{\sphinxcrossref{
web.concurrency
}}}
\item {} {\hyperref[\detokenize{reference/javascript_api:web.core}]{\sphinxcrossref{
web.core
}}}
\item {} {\hyperref[\detokenize{reference/javascript_api:web.session}]{\sphinxcrossref{
web.session
}}}
\item {} {\hyperref[\detokenize{reference/javascript_api:web.utils}]{\sphinxcrossref{
web.utils
}}}
\end{itemize}

\end{description}\end{quote}


\begin{fulllineitems}
\phantomsection\label{\detokenize{reference/javascript_api:BasicModel}}\pysiglinewithargsret{\sphinxbfcode{\sphinxupquote{class }}\sphinxbfcode{\sphinxupquote{BasicModel}}}{}{}~\begin{quote}\begin{description}
\item[{Extends}] \leavevmode{\hyperref[\detokenize{reference/javascript_api:web.AbstractModel.AbstractModel}]{\sphinxcrossref{
AbstractModel
}}}
\end{description}\end{quote}


\begin{fulllineitems}
\phantomsection\label{\detokenize{reference/javascript_api:addDefaultRecord}}\pysiglinewithargsret{\sphinxbfcode{\sphinxupquote{method }}\sphinxbfcode{\sphinxupquote{addDefaultRecord}}}{\emph{listID}\sphinxoptional{, \emph{options}}}{{ $\rightarrow$ Deferred\textless{}string\textgreater{}}}
Add a default record to a list object. This method actually makes a new
record with the \_makeDefaultRecord method, then adds it to the list object.
The default record is added in the data directly. This is meant to be used
by list or kanban controllers (i.e. not for x2manys in form views, as in
this case, we store changes as commands).
\begin{quote}\begin{description}
\item[{Parameters}] \leavevmode\begin{itemize}

\sphinxstylestrong{listID} (\sphinxstyleliteralemphasis{\sphinxupquote{string}}) \textendash{} a valid handle for a list object

\sphinxstylestrong{options} ({\hyperref[\detokenize{reference/javascript_api:web.BasicModel.AddDefaultRecordOptions}]{\sphinxcrossref{\sphinxstyleliteralemphasis{\sphinxupquote{AddDefaultRecordOptions}}}}})
\end{itemize}

\item[{Returns}] \leavevmode
resolves to the id of the new created record

\item[{Return Type}] \leavevmode
\sphinxstyleliteralemphasis{\sphinxupquote{Deferred}}\textless{}\sphinxstyleliteralemphasis{\sphinxupquote{string}}\textgreater{}

\end{description}\end{quote}


\begin{fulllineitems}
\phantomsection\label{\detokenize{reference/javascript_api:AddDefaultRecordOptions}}\pysiglinewithargsret{\sphinxbfcode{\sphinxupquote{class }}\sphinxbfcode{\sphinxupquote{AddDefaultRecordOptions}}}{}{}~

\begin{fulllineitems}
\phantomsection\label{\detokenize{reference/javascript_api:position}}\pysigline{\sphinxbfcode{\sphinxupquote{attribute }}\sphinxbfcode{\sphinxupquote{position}} string}~\begin{description}
\item[{if the new record should be added}] \leavevmode
on top or on bottom of the list

\end{description}

\end{fulllineitems}


\end{fulllineitems}


\end{fulllineitems}



\begin{fulllineitems}
\phantomsection\label{\detokenize{reference/javascript_api:applyDefaultValues}}\pysiglinewithargsret{\sphinxbfcode{\sphinxupquote{function }}\sphinxbfcode{\sphinxupquote{applyDefaultValues}}}{\emph{recordID}, \emph{values}\sphinxoptional{, \emph{options}}}{{ $\rightarrow$ Deferred}}
Add and process default values for a given record. Those values are
parsed and stored in the ‘\_changes’ key of the record. For relational
fields, sub-dataPoints are created, and missing relational data is
fetched. Also generate default values for fields with no given value.
Typically, this function is called with the result of a ‘default\_get’
RPC, to populate a newly created dataPoint. It may also be called when a
one2many subrecord is open in a form view (dialog), to generate the
default values for the fields displayed in the o2m form view, but not in
the list or kanban (mainly to correctly create sub-dataPoints for
relational fields).
\begin{quote}\begin{description}
\item[{Parameters}] \leavevmode\begin{itemize}

\sphinxstylestrong{recordID} (\sphinxstyleliteralemphasis{\sphinxupquote{string}}) \textendash{} local id for a record

\sphinxstylestrong{values} (\sphinxstyleliteralemphasis{\sphinxupquote{Object}}) \textendash{} dict of default values for the given record

\sphinxstylestrong{options} ({\hyperref[\detokenize{reference/javascript_api:web.BasicModel.ApplyDefaultValuesOptions}]{\sphinxcrossref{\sphinxstyleliteralemphasis{\sphinxupquote{ApplyDefaultValuesOptions}}}}})
\end{itemize}

\item[{Return Type}] \leavevmode
\sphinxstyleliteralemphasis{\sphinxupquote{Deferred}}

\end{description}\end{quote}


\begin{fulllineitems}
\phantomsection\label{\detokenize{reference/javascript_api:ApplyDefaultValuesOptions}}\pysiglinewithargsret{\sphinxbfcode{\sphinxupquote{class }}\sphinxbfcode{\sphinxupquote{ApplyDefaultValuesOptions}}}{}{}~

\begin{fulllineitems}
\phantomsection\label{\detokenize{reference/javascript_api:viewType}}\pysigline{\sphinxbfcode{\sphinxupquote{attribute }}\sphinxbfcode{\sphinxupquote{viewType}} string}~\begin{description}
\item[{current viewType. If not set, we will}] \leavevmode
assume main viewType from the record

\end{description}

\end{fulllineitems}



\begin{fulllineitems}
\phantomsection\label{\detokenize{reference/javascript_api:fieldNames}}\pysigline{\sphinxbfcode{\sphinxupquote{attribute }}\sphinxbfcode{\sphinxupquote{fieldNames}} Array}~\begin{description}
\item[{list of field names for which a}] \leavevmode
default value must be generated (used to complete the values dict)

\end{description}

\end{fulllineitems}


\end{fulllineitems}


\end{fulllineitems}



\begin{fulllineitems}
\phantomsection\label{\detokenize{reference/javascript_api:applyRawChanges}}\pysiglinewithargsret{\sphinxbfcode{\sphinxupquote{function }}\sphinxbfcode{\sphinxupquote{applyRawChanges}}}{\emph{recordID}, \emph{viewType}}{{ $\rightarrow$ Deferred\textless{}string\textgreater{}}}
Onchange RPCs may return values for fields that are not in the current
view. Those fields might even be unknown when the onchange returns (e.g.
in x2manys, we only know the fields that are used in the inner view, but
not those used in the potential form view opened in a dialog when a sub-
record is clicked). When this happens, we can’t infer their type, so the
given value can’t be processed. It is instead stored in the ‘\_rawChanges’
key of the record, without any processing. Later on, if this record is
displayed in another view (e.g. the user clicked on it in the x2many
list, and the record opens in a dialog), those changes that were left
behind must be applied. This function applies changes stored in
‘\_rawChanges’ for a given viewType.
\begin{quote}\begin{description}
\item[{Parameters}] \leavevmode\begin{itemize}

\sphinxstylestrong{recordID} (\sphinxstyleliteralemphasis{\sphinxupquote{string}}) \textendash{} local resource id of a record

\sphinxstylestrong{viewType} (\sphinxstyleliteralemphasis{\sphinxupquote{string}}) \textendash{} the current viewType
\end{itemize}

\item[{Returns}] \leavevmode
resolves to the id of the record

\item[{Return Type}] \leavevmode
\sphinxstyleliteralemphasis{\sphinxupquote{Deferred}}\textless{}\sphinxstyleliteralemphasis{\sphinxupquote{string}}\textgreater{}

\end{description}\end{quote}

\end{fulllineitems}



\begin{fulllineitems}
\phantomsection\label{\detokenize{reference/javascript_api:deleteRecords}}\pysiglinewithargsret{\sphinxbfcode{\sphinxupquote{function }}\sphinxbfcode{\sphinxupquote{deleteRecords}}}{\emph{recordIds}, \emph{modelName}}{{ $\rightarrow$ Deferred}}
Delete a list of records, then, if the records have a parent, reload it.
\begin{quote}\begin{description}
\item[{Parameters}] \leavevmode\begin{itemize}

\sphinxstylestrong{recordIds} (\sphinxstyleliteralemphasis{\sphinxupquote{Array}}\textless{}\sphinxstyleliteralemphasis{\sphinxupquote{string}}\textgreater{}) \textendash{} list of local resources ids. They should all
  be of type ‘record’, be of the same model and have the same parent.

\sphinxstylestrong{modelName} (\sphinxstyleliteralemphasis{\sphinxupquote{string}}) \textendash{} mode name used to unlink the records
\end{itemize}

\item[{Return Type}] \leavevmode
\sphinxstyleliteralemphasis{\sphinxupquote{Deferred}}

\end{description}\end{quote}

\end{fulllineitems}



\begin{fulllineitems}
\phantomsection\label{\detokenize{reference/javascript_api:discardChanges}}\pysiglinewithargsret{\sphinxbfcode{\sphinxupquote{function }}\sphinxbfcode{\sphinxupquote{discardChanges}}}{\emph{id}\sphinxoptional{, \emph{options}}}{}
Discard all changes in a local resource.  Basically, it removes
everything that was stored in a \_changes key.
\begin{quote}\begin{description}
\item[{Parameters}] \leavevmode\begin{itemize}

\sphinxstylestrong{id} (\sphinxstyleliteralemphasis{\sphinxupquote{string}}) \textendash{} local resource id

\sphinxstylestrong{options} ({\hyperref[\detokenize{reference/javascript_api:web.BasicModel.DiscardChangesOptions}]{\sphinxcrossref{\sphinxstyleliteralemphasis{\sphinxupquote{DiscardChangesOptions}}}}})
\end{itemize}

\end{description}\end{quote}


\begin{fulllineitems}
\phantomsection\label{\detokenize{reference/javascript_api:DiscardChangesOptions}}\pysiglinewithargsret{\sphinxbfcode{\sphinxupquote{class }}\sphinxbfcode{\sphinxupquote{DiscardChangesOptions}}}{}{}~

\begin{fulllineitems}
\phantomsection\label{\detokenize{reference/javascript_api:rollback}}\pysigline{\sphinxbfcode{\sphinxupquote{attribute }}\sphinxbfcode{\sphinxupquote{rollback}} boolean}~\begin{description}
\item[{if true, the changes will}] \leavevmode
be reset to the last \_savePoint, otherwise, they are reset to null

\end{description}

\end{fulllineitems}


\end{fulllineitems}


\end{fulllineitems}



\begin{fulllineitems}
\phantomsection\label{\detokenize{reference/javascript_api:duplicateRecord}}\pysiglinewithargsret{\sphinxbfcode{\sphinxupquote{function }}\sphinxbfcode{\sphinxupquote{duplicateRecord}}}{\emph{recordID}}{{ $\rightarrow$ Deferred\textless{}string\textgreater{}}}
Duplicate a record (by calling the ‘copy’ route)
\begin{quote}\begin{description}
\item[{Parameters}] \leavevmode\begin{itemize}

\sphinxstylestrong{recordID} (\sphinxstyleliteralemphasis{\sphinxupquote{string}}) \textendash{} id for a local resource
\end{itemize}

\item[{Returns}] \leavevmode
resolves to the id of duplicate record

\item[{Return Type}] \leavevmode
\sphinxstyleliteralemphasis{\sphinxupquote{Deferred}}\textless{}\sphinxstyleliteralemphasis{\sphinxupquote{string}}\textgreater{}

\end{description}\end{quote}

\end{fulllineitems}



\begin{fulllineitems}
\phantomsection\label{\detokenize{reference/javascript_api:get}}\pysiglinewithargsret{\sphinxbfcode{\sphinxupquote{function }}\sphinxbfcode{\sphinxupquote{get}}}{\emph{id}, \emph{options}}{{ $\rightarrow$ Object}}
The get method first argument is the handle returned by the load method.
It is optional (the handle can be undefined).  In some case, it makes
sense to use the handle as a key, for example the BasicModel holds the
data for various records, each with its local ID.

synchronous method, it assumes that the resource has already been loaded.
\begin{quote}\begin{description}
\item[{Parameters}] \leavevmode\begin{itemize}

\sphinxstylestrong{id} (\sphinxstyleliteralemphasis{\sphinxupquote{string}}) \textendash{} local id for the resource

\sphinxstylestrong{options} ({\hyperref[\detokenize{reference/javascript_api:web.BasicModel.GetOptions}]{\sphinxcrossref{\sphinxstyleliteralemphasis{\sphinxupquote{GetOptions}}}}})
\end{itemize}

\item[{Return Type}] \leavevmode
\sphinxstyleliteralemphasis{\sphinxupquote{Object}}

\end{description}\end{quote}


\begin{fulllineitems}
\phantomsection\label{\detokenize{reference/javascript_api:GetOptions}}\pysiglinewithargsret{\sphinxbfcode{\sphinxupquote{class }}\sphinxbfcode{\sphinxupquote{GetOptions}}}{}{}~

\begin{fulllineitems}
\phantomsection\label{\detokenize{reference/javascript_api:env}}\pysigline{\sphinxbfcode{\sphinxupquote{attribute }}\sphinxbfcode{\sphinxupquote{env}} boolean}~\begin{description}
\item[{if true, will only  return res\_id}] \leavevmode
(if record) or res\_ids (if list)

\end{description}

\end{fulllineitems}



\begin{fulllineitems}
\phantomsection\label{\detokenize{reference/javascript_api:raw}}\pysigline{\sphinxbfcode{\sphinxupquote{attribute }}\sphinxbfcode{\sphinxupquote{raw}} boolean}
if true, will not follow relations

\end{fulllineitems}


\end{fulllineitems}


\end{fulllineitems}



\begin{fulllineitems}
\phantomsection\label{\detokenize{reference/javascript_api:getName}}\pysiglinewithargsret{\sphinxbfcode{\sphinxupquote{function }}\sphinxbfcode{\sphinxupquote{getName}}}{\emph{id}}{{ $\rightarrow$ string}}
Returns the current display\_name for the record.
\begin{quote}\begin{description}
\item[{Parameters}] \leavevmode\begin{itemize}

\sphinxstylestrong{id} (\sphinxstyleliteralemphasis{\sphinxupquote{string}}) \textendash{} the localID for a valid record element
\end{itemize}

\item[{Return Type}] \leavevmode
\sphinxstyleliteralemphasis{\sphinxupquote{string}}

\end{description}\end{quote}

\end{fulllineitems}



\begin{fulllineitems}
\phantomsection\label{\detokenize{reference/javascript_api:canBeAbandoned}}\pysiglinewithargsret{\sphinxbfcode{\sphinxupquote{function }}\sphinxbfcode{\sphinxupquote{canBeAbandoned}}}{\emph{id}}{{ $\rightarrow$ boolean}}
Returns true if a record can be abandoned.

A record can be abandonned if it is a new record, except if
this datapoint has been specifically tagged as “do not abandon”.

Example:
\begin{itemize}
\item {} 
Discard record from “Add an item” =\textgreater{} “New” record =\textgreater{} abandon

\item {} 
Discard record from \sphinxcode{\sphinxupquote{default\_get}}/\sphinxcode{\sphinxupquote{onchange}} =\textgreater{} do not abandon

\end{itemize}

This is useful when discarding changes on this record, as it means that
we must keep the record even if some fields are invalids (e.g. required
field is empty).
\begin{quote}\begin{description}
\item[{Parameters}] \leavevmode\begin{itemize}

\sphinxstylestrong{id} (\sphinxstyleliteralemphasis{\sphinxupquote{string}}) \textendash{} id for a local resource
\end{itemize}

\item[{Return Type}] \leavevmode
\sphinxstyleliteralemphasis{\sphinxupquote{boolean}}

\end{description}\end{quote}

\end{fulllineitems}



\begin{fulllineitems}
\phantomsection\label{\detokenize{reference/javascript_api:isDirty}}\pysiglinewithargsret{\sphinxbfcode{\sphinxupquote{function }}\sphinxbfcode{\sphinxupquote{isDirty}}}{\emph{id}}{{ $\rightarrow$ boolean}}
Returns true if a record is dirty. A record is considered dirty if it has
some unsaved changes, marked by the \_isDirty property on the record or
one of its subrecords.
\begin{quote}\begin{description}
\item[{Parameters}] \leavevmode\begin{itemize}

\sphinxstylestrong{id} (\sphinxstyleliteralemphasis{\sphinxupquote{string}}) \textendash{} the local resource id
\end{itemize}

\item[{Return Type}] \leavevmode
\sphinxstyleliteralemphasis{\sphinxupquote{boolean}}

\end{description}\end{quote}

\end{fulllineitems}



\begin{fulllineitems}
\phantomsection\label{\detokenize{reference/javascript_api:isNew}}\pysiglinewithargsret{\sphinxbfcode{\sphinxupquote{function }}\sphinxbfcode{\sphinxupquote{isNew}}}{\emph{id}}{{ $\rightarrow$ boolean}}
Check if a localData is new, meaning if it is in the process of being
created and no actual record exists in db. Note: if the localData is not
of the “record” type, then it is always considered as not new.

Note: A virtual id is a character string composed of an integer and has
a dash and other information.
E.g: in calendar, the recursive event have virtual id linked to a real id
virtual event id “23-20170418020000” is linked to the event id 23
\begin{quote}\begin{description}
\item[{Parameters}] \leavevmode\begin{itemize}

\sphinxstylestrong{id} (\sphinxstyleliteralemphasis{\sphinxupquote{string}}) \textendash{} id for a local resource
\end{itemize}

\item[{Return Type}] \leavevmode
\sphinxstyleliteralemphasis{\sphinxupquote{boolean}}

\end{description}\end{quote}

\end{fulllineitems}



\begin{fulllineitems}
\phantomsection\label{\detokenize{reference/javascript_api:load}}\pysiglinewithargsret{\sphinxbfcode{\sphinxupquote{function }}\sphinxbfcode{\sphinxupquote{load}}}{\emph{params}}{{ $\rightarrow$ Deferred\textless{}string\textgreater{}}}
Main entry point, the goal of this method is to fetch and process all
data (following relations if necessary) for a given record/list.
\begin{quote}\begin{description}
\item[{Parameters}] \leavevmode\begin{itemize}

\sphinxstylestrong{params} ({\hyperref[\detokenize{reference/javascript_api:web.BasicModel.LoadParams}]{\sphinxcrossref{\sphinxstyleliteralemphasis{\sphinxupquote{LoadParams}}}}})
\end{itemize}

\item[{Returns}] \leavevmode
resolves to a local id, or handle

\item[{Return Type}] \leavevmode
\sphinxstyleliteralemphasis{\sphinxupquote{Deferred}}\textless{}\sphinxstyleliteralemphasis{\sphinxupquote{string}}\textgreater{}

\end{description}\end{quote}


\begin{fulllineitems}
\phantomsection\label{\detokenize{reference/javascript_api:LoadParams}}\pysiglinewithargsret{\sphinxbfcode{\sphinxupquote{class }}\sphinxbfcode{\sphinxupquote{LoadParams}}}{}{}~

\begin{fulllineitems}
\phantomsection\label{\detokenize{reference/javascript_api:fieldsInfo}}\pysigline{\sphinxbfcode{\sphinxupquote{attribute }}\sphinxbfcode{\sphinxupquote{fieldsInfo}} Object}
contains the fieldInfo of each field

\end{fulllineitems}



\begin{fulllineitems}
\phantomsection\label{\detokenize{reference/javascript_api:fields}}\pysigline{\sphinxbfcode{\sphinxupquote{attribute }}\sphinxbfcode{\sphinxupquote{fields}} Object}
contains the description of each field

\end{fulllineitems}



\begin{fulllineitems}
\phantomsection\label{\detokenize{reference/javascript_api:type}}\pysigline{\sphinxbfcode{\sphinxupquote{attribute }}\sphinxbfcode{\sphinxupquote{type}} string}
‘record’ or ‘list’

\end{fulllineitems}



\begin{fulllineitems}
\phantomsection\label{\detokenize{reference/javascript_api:recordID}}\pysigline{\sphinxbfcode{\sphinxupquote{attribute }}\sphinxbfcode{\sphinxupquote{recordID}} string}
an ID for an existing resource.

\end{fulllineitems}


\end{fulllineitems}


\end{fulllineitems}



\begin{fulllineitems}
\phantomsection\label{\detokenize{reference/javascript_api:makeRecord}}\pysiglinewithargsret{\sphinxbfcode{\sphinxupquote{function }}\sphinxbfcode{\sphinxupquote{makeRecord}}}{\emph{model}, \emph{fields}\sphinxoptional{, \emph{fieldInfo}}}{{ $\rightarrow$ string}}
This helper method is designed to help developpers that want to use a
field widget outside of a view.  In that case, we want a way to create
data without actually performing a fetch.
\begin{quote}\begin{description}
\item[{Parameters}] \leavevmode\begin{itemize}

\sphinxstylestrong{model} (\sphinxstyleliteralemphasis{\sphinxupquote{string}}) \textendash{} name of the model

\sphinxstylestrong{fields} (\sphinxstyleliteralemphasis{\sphinxupquote{Array}}\textless{}\sphinxstyleliteralemphasis{\sphinxupquote{Object}}\textgreater{}) \textendash{} a description of field properties

\sphinxstylestrong{fieldInfo} (\sphinxstyleliteralemphasis{\sphinxupquote{Object}}) \textendash{} various field info that we want to set
\end{itemize}

\item[{Returns}] \leavevmode
the local id for the created resource

\item[{Return Type}] \leavevmode
\sphinxstyleliteralemphasis{\sphinxupquote{string}}

\end{description}\end{quote}

\end{fulllineitems}



\begin{fulllineitems}
\phantomsection\label{\detokenize{reference/javascript_api:notifyChanges}}\pysiglinewithargsret{\sphinxbfcode{\sphinxupquote{function }}\sphinxbfcode{\sphinxupquote{notifyChanges}}}{\emph{record\_id}, \emph{changes}\sphinxoptional{, \emph{options}}}{{ $\rightarrow$ string{[}{]}}}
This is an extremely important method.  All changes in any field go
through this method.  It will then apply them in the local state, check
if onchanges needs to be applied, actually do them if necessary, then
resolves with the list of changed fields.
\begin{quote}\begin{description}
\item[{Parameters}] \leavevmode\begin{itemize}

\sphinxstylestrong{record\_id} (\sphinxstyleliteralemphasis{\sphinxupquote{string}})

\sphinxstylestrong{changes} (\sphinxstyleliteralemphasis{\sphinxupquote{Object}}) \textendash{} a map field =\textgreater{} new value

\sphinxstylestrong{options} (\sphinxstyleliteralemphasis{\sphinxupquote{Object}}) \textendash{} will be transferred to the applyChange method
\end{itemize}

\item[{Returns}] \leavevmode
list of changed fields

\item[{Return Type}] \leavevmode
\sphinxstyleliteralemphasis{\sphinxupquote{Array}}\textless{}\sphinxstyleliteralemphasis{\sphinxupquote{string}}\textgreater{}

\end{description}\end{quote}

\end{fulllineitems}



\begin{fulllineitems}
\phantomsection\label{\detokenize{reference/javascript_api:reload}}\pysiglinewithargsret{\sphinxbfcode{\sphinxupquote{function }}\sphinxbfcode{\sphinxupquote{reload}}}{\emph{id}\sphinxoptional{, \emph{options}}}{{ $\rightarrow$ Deferred\textless{}string\textgreater{}}}
Reload all data for a given resource
\begin{quote}\begin{description}
\item[{Parameters}] \leavevmode\begin{itemize}

\sphinxstylestrong{id} (\sphinxstyleliteralemphasis{\sphinxupquote{string}}) \textendash{} local id for a resource

\sphinxstylestrong{options} ({\hyperref[\detokenize{reference/javascript_api:web.BasicModel.ReloadOptions}]{\sphinxcrossref{\sphinxstyleliteralemphasis{\sphinxupquote{ReloadOptions}}}}})
\end{itemize}

\item[{Returns}] \leavevmode
resolves to the id of the resource

\item[{Return Type}] \leavevmode
\sphinxstyleliteralemphasis{\sphinxupquote{Deferred}}\textless{}\sphinxstyleliteralemphasis{\sphinxupquote{string}}\textgreater{}

\end{description}\end{quote}


\begin{fulllineitems}
\phantomsection\label{\detokenize{reference/javascript_api:ReloadOptions}}\pysiglinewithargsret{\sphinxbfcode{\sphinxupquote{class }}\sphinxbfcode{\sphinxupquote{ReloadOptions}}}{}{}~

\begin{fulllineitems}
\phantomsection\label{\detokenize{reference/javascript_api:keepChanges}}\pysigline{\sphinxbfcode{\sphinxupquote{attribute }}\sphinxbfcode{\sphinxupquote{keepChanges}} boolean}~\begin{description}
\item[{if true, doesn’t discard the}] \leavevmode
changes on the record before reloading it

\end{description}

\end{fulllineitems}


\end{fulllineitems}


\end{fulllineitems}



\begin{fulllineitems}
\phantomsection\label{\detokenize{reference/javascript_api:removeLine}}\pysiglinewithargsret{\sphinxbfcode{\sphinxupquote{function }}\sphinxbfcode{\sphinxupquote{removeLine}}}{\emph{elementID}}{}
In some case, we may need to remove an element from a list, without going
through the notifyChanges machinery.  The motivation for this is when the
user click on ‘Add an item’ in a field one2many with a required field,
then clicks somewhere else.  The new line need to be discarded, but we
don’t want to trigger a real notifyChanges (no need for that, and also,
we don’t want to rerender the UI).
\begin{quote}\begin{description}
\item[{Parameters}] \leavevmode\begin{itemize}

\sphinxstylestrong{elementID} (\sphinxstyleliteralemphasis{\sphinxupquote{string}}) \textendash{} some valid element id. It is necessary that the
  corresponding element has a parent.
\end{itemize}

\end{description}\end{quote}

\end{fulllineitems}



\begin{fulllineitems}
\phantomsection\label{\detokenize{reference/javascript_api:resequence}}\pysiglinewithargsret{\sphinxbfcode{\sphinxupquote{function }}\sphinxbfcode{\sphinxupquote{resequence}}}{\emph{modelName}, \emph{resIDs}, \emph{parentID}\sphinxoptional{, \emph{options}}}{{ $\rightarrow$ Deferred\textless{}string\textgreater{}}}
Resequences records.
\begin{quote}\begin{description}
\item[{Parameters}] \leavevmode\begin{itemize}

\sphinxstylestrong{modelName} (\sphinxstyleliteralemphasis{\sphinxupquote{string}}) \textendash{} the resIDs model

\sphinxstylestrong{resIDs} (\sphinxstyleliteralemphasis{\sphinxupquote{Array}}\textless{}\sphinxstyleliteralemphasis{\sphinxupquote{integer}}\textgreater{}) \textendash{} the new sequence of ids

\sphinxstylestrong{parentID} (\sphinxstyleliteralemphasis{\sphinxupquote{string}}) \textendash{} the localID of the parent

\sphinxstylestrong{options} ({\hyperref[\detokenize{reference/javascript_api:web.BasicModel.ResequenceOptions}]{\sphinxcrossref{\sphinxstyleliteralemphasis{\sphinxupquote{ResequenceOptions}}}}})
\end{itemize}

\item[{Returns}] \leavevmode
resolves to the local id of the parent

\item[{Return Type}] \leavevmode
\sphinxstyleliteralemphasis{\sphinxupquote{Deferred}}\textless{}\sphinxstyleliteralemphasis{\sphinxupquote{string}}\textgreater{}

\end{description}\end{quote}


\begin{fulllineitems}
\phantomsection\label{\detokenize{reference/javascript_api:ResequenceOptions}}\pysiglinewithargsret{\sphinxbfcode{\sphinxupquote{class }}\sphinxbfcode{\sphinxupquote{ResequenceOptions}}}{}{}~

\begin{fulllineitems}
\phantomsection\label{\detokenize{reference/javascript_api:offset}}\pysigline{\sphinxbfcode{\sphinxupquote{attribute }}\sphinxbfcode{\sphinxupquote{offset}} integer}
\end{fulllineitems}



\begin{fulllineitems}
\phantomsection\label{\detokenize{reference/javascript_api:field}}\pysigline{\sphinxbfcode{\sphinxupquote{attribute }}\sphinxbfcode{\sphinxupquote{field}} string}
the field name used as sequence

\end{fulllineitems}


\end{fulllineitems}


\end{fulllineitems}



\begin{fulllineitems}
\phantomsection\label{\detokenize{reference/javascript_api:save}}\pysiglinewithargsret{\sphinxbfcode{\sphinxupquote{function }}\sphinxbfcode{\sphinxupquote{save}}}{\emph{recordID}\sphinxoptional{, \emph{options}}}{{ $\rightarrow$ Deferred}}
Save a local resource, if needed.  This is a complicated operation,
- it needs to check all changes,
- generate commands for x2many fields,
- call the /create or /write method according to the record status
- After that, it has to reload all data, in case something changed, server side.
\begin{quote}\begin{description}
\item[{Parameters}] \leavevmode\begin{itemize}

\sphinxstylestrong{recordID}

\sphinxstylestrong{options} ({\hyperref[\detokenize{reference/javascript_api:web.BasicModel.SaveOptions}]{\sphinxcrossref{\sphinxstyleliteralemphasis{\sphinxupquote{SaveOptions}}}}})
\end{itemize}

\item[{Returns}] \leavevmode
Resolved with the list of field names (whose value has been modified)

\item[{Return Type}] \leavevmode
\sphinxstyleliteralemphasis{\sphinxupquote{Deferred}}

\end{description}\end{quote}


\begin{fulllineitems}
\phantomsection\label{\detokenize{reference/javascript_api:SaveOptions}}\pysiglinewithargsret{\sphinxbfcode{\sphinxupquote{class }}\sphinxbfcode{\sphinxupquote{SaveOptions}}}{}{}~

\begin{fulllineitems}
\phantomsection\label{\detokenize{reference/javascript_api:reload}}\pysigline{\sphinxbfcode{\sphinxupquote{attribute }}\sphinxbfcode{\sphinxupquote{reload}} boolean}
if true, data will be reloaded

\end{fulllineitems}



\begin{fulllineitems}
\phantomsection\label{\detokenize{reference/javascript_api:savePoint}}\pysigline{\sphinxbfcode{\sphinxupquote{attribute }}\sphinxbfcode{\sphinxupquote{savePoint}} boolean}~\begin{description}
\item[{if true, the record will only}] \leavevmode
be ‘locally’ saved: its changes written in a \_savePoint key that can
be restored later by call discardChanges with option rollback to true

\end{description}

\end{fulllineitems}



\begin{fulllineitems}
\phantomsection\label{\detokenize{reference/javascript_api:viewType}}\pysigline{\sphinxbfcode{\sphinxupquote{attribute }}\sphinxbfcode{\sphinxupquote{viewType}} string}~\begin{description}
\item[{current viewType. If not set, we will}] \leavevmode
assume main viewType from the record

\end{description}

\end{fulllineitems}


\end{fulllineitems}


\end{fulllineitems}



\begin{fulllineitems}
\phantomsection\label{\detokenize{reference/javascript_api:addFieldsInfo}}\pysiglinewithargsret{\sphinxbfcode{\sphinxupquote{function }}\sphinxbfcode{\sphinxupquote{addFieldsInfo}}}{\emph{recordID}, \emph{viewInfo}}{}
Completes the fields and fieldsInfo of a dataPoint with the given ones.
It is useful for the cases where a record element is shared between
various views, such as a one2many with a tree and a form view.
\begin{quote}\begin{description}
\item[{Parameters}] \leavevmode\begin{itemize}

\sphinxstylestrong{recordID} (\sphinxstyleliteralemphasis{\sphinxupquote{string}}) \textendash{} a valid element ID

\sphinxstylestrong{viewInfo} ({\hyperref[\detokenize{reference/javascript_api:web.BasicModel.AddFieldsInfoViewInfo}]{\sphinxcrossref{\sphinxstyleliteralemphasis{\sphinxupquote{AddFieldsInfoViewInfo}}}}})
\end{itemize}

\end{description}\end{quote}


\begin{fulllineitems}
\phantomsection\label{\detokenize{reference/javascript_api:AddFieldsInfoViewInfo}}\pysiglinewithargsret{\sphinxbfcode{\sphinxupquote{class }}\sphinxbfcode{\sphinxupquote{AddFieldsInfoViewInfo}}}{}{}~

\begin{fulllineitems}
\phantomsection\label{\detokenize{reference/javascript_api:fields}}\pysigline{\sphinxbfcode{\sphinxupquote{attribute }}\sphinxbfcode{\sphinxupquote{fields}} Object}
\end{fulllineitems}



\begin{fulllineitems}
\phantomsection\label{\detokenize{reference/javascript_api:fieldsInfo}}\pysigline{\sphinxbfcode{\sphinxupquote{attribute }}\sphinxbfcode{\sphinxupquote{fieldsInfo}} Object}
\end{fulllineitems}


\end{fulllineitems}


\end{fulllineitems}



\begin{fulllineitems}
\phantomsection\label{\detokenize{reference/javascript_api:freezeOrder}}\pysiglinewithargsret{\sphinxbfcode{\sphinxupquote{function }}\sphinxbfcode{\sphinxupquote{freezeOrder}}}{\emph{listID}}{}
For list resources, this freezes the current records order.
\begin{quote}\begin{description}
\item[{Parameters}] \leavevmode\begin{itemize}

\sphinxstylestrong{listID} (\sphinxstyleliteralemphasis{\sphinxupquote{string}}) \textendash{} a valid element ID of type list
\end{itemize}

\end{description}\end{quote}

\end{fulllineitems}



\begin{fulllineitems}
\phantomsection\label{\detokenize{reference/javascript_api:setDirty}}\pysiglinewithargsret{\sphinxbfcode{\sphinxupquote{function }}\sphinxbfcode{\sphinxupquote{setDirty}}}{\emph{id}}{}
Manually sets a resource as dirty. This is used to notify that a field
has been modified, but with an invalid value. In that case, the value is
not sent to the basic model, but the record should still be flagged as
dirty so that it isn’t discarded without any warning.
\begin{quote}\begin{description}
\item[{Parameters}] \leavevmode\begin{itemize}

\sphinxstylestrong{id} (\sphinxstyleliteralemphasis{\sphinxupquote{string}}) \textendash{} a resource id
\end{itemize}

\end{description}\end{quote}

\end{fulllineitems}



\begin{fulllineitems}
\phantomsection\label{\detokenize{reference/javascript_api:setSort}}\pysiglinewithargsret{\sphinxbfcode{\sphinxupquote{function }}\sphinxbfcode{\sphinxupquote{setSort}}}{\emph{list\_id}, \emph{fieldName}}{{ $\rightarrow$ Deferred}}
For list resources, this changes the orderedBy key.
\begin{quote}\begin{description}
\item[{Parameters}] \leavevmode\begin{itemize}

\sphinxstylestrong{list\_id} (\sphinxstyleliteralemphasis{\sphinxupquote{string}}) \textendash{} id for the list resource

\sphinxstylestrong{fieldName} (\sphinxstyleliteralemphasis{\sphinxupquote{string}}) \textendash{} valid field name
\end{itemize}

\item[{Return Type}] \leavevmode
\sphinxstyleliteralemphasis{\sphinxupquote{Deferred}}

\end{description}\end{quote}

\end{fulllineitems}



\begin{fulllineitems}
\phantomsection\label{\detokenize{reference/javascript_api:toggleActive}}\pysiglinewithargsret{\sphinxbfcode{\sphinxupquote{function }}\sphinxbfcode{\sphinxupquote{toggleActive}}}{\emph{recordIDs}, \emph{value}, \emph{parentID}}{{ $\rightarrow$ Deferred\textless{}string\textgreater{}}}
Toggle the active value of given records (to archive/unarchive them)
\begin{quote}\begin{description}
\item[{Parameters}] \leavevmode\begin{itemize}

\sphinxstylestrong{recordIDs} (\sphinxstyleliteralemphasis{\sphinxupquote{Array}}) \textendash{} local ids of the records to (un)archive

\sphinxstylestrong{value} (\sphinxstyleliteralemphasis{\sphinxupquote{boolean}}) \textendash{} false to archive, true to unarchive (value of the active field)

\sphinxstylestrong{parentID} (\sphinxstyleliteralemphasis{\sphinxupquote{string}}) \textendash{} id of the parent resource to reload
\end{itemize}

\item[{Returns}] \leavevmode
resolves to the parent id

\item[{Return Type}] \leavevmode
\sphinxstyleliteralemphasis{\sphinxupquote{Deferred}}\textless{}\sphinxstyleliteralemphasis{\sphinxupquote{string}}\textgreater{}

\end{description}\end{quote}

\end{fulllineitems}



\begin{fulllineitems}
\phantomsection\label{\detokenize{reference/javascript_api:toggleGroup}}\pysiglinewithargsret{\sphinxbfcode{\sphinxupquote{function }}\sphinxbfcode{\sphinxupquote{toggleGroup}}}{\emph{groupId}}{{ $\rightarrow$ Deferred\textless{}string\textgreater{}}}
Toggle (open/close) a group in a grouped list, then fetches relevant
data
\begin{quote}\begin{description}
\item[{Parameters}] \leavevmode\begin{itemize}

\sphinxstylestrong{groupId} (\sphinxstyleliteralemphasis{\sphinxupquote{string}})
\end{itemize}

\item[{Returns}] \leavevmode
resolves to the group id

\item[{Return Type}] \leavevmode
\sphinxstyleliteralemphasis{\sphinxupquote{Deferred}}\textless{}\sphinxstyleliteralemphasis{\sphinxupquote{string}}\textgreater{}

\end{description}\end{quote}

\end{fulllineitems}



\begin{fulllineitems}
\phantomsection\label{\detokenize{reference/javascript_api:unfreezeOrder}}\pysiglinewithargsret{\sphinxbfcode{\sphinxupquote{function }}\sphinxbfcode{\sphinxupquote{unfreezeOrder}}}{\emph{elementID}}{}
For a list datapoint, unfreezes the current records order and sorts it.
For a record datapoint, unfreezes the x2many list datapoints.
\begin{quote}\begin{description}
\item[{Parameters}] \leavevmode\begin{itemize}

\sphinxstylestrong{elementID} (\sphinxstyleliteralemphasis{\sphinxupquote{string}}) \textendash{} a valid element ID
\end{itemize}

\end{description}\end{quote}

\end{fulllineitems}



\begin{fulllineitems}
\phantomsection\label{\detokenize{reference/javascript_api:updateMessageIDs}}\pysiglinewithargsret{\sphinxbfcode{\sphinxupquote{function }}\sphinxbfcode{\sphinxupquote{updateMessageIDs}}}{\emph{id}, \emph{msgIDs}}{}
Update the message ids on a datapoint.

Note that we directly update the res\_ids on the datapoint as the message
has already been posted ; this change can’t be handled ‘normally’ with
x2m commands because the change won’t be saved as a normal field.
\begin{quote}\begin{description}
\item[{Parameters}] \leavevmode\begin{itemize}

\sphinxstylestrong{id} (\sphinxstyleliteralemphasis{\sphinxupquote{string}})

\sphinxstylestrong{msgIDs} (\sphinxstyleliteralemphasis{\sphinxupquote{Array}}\textless{}\sphinxstyleliteralemphasis{\sphinxupquote{integer}}\textgreater{})
\end{itemize}

\end{description}\end{quote}

\end{fulllineitems}


\end{fulllineitems}


\end{fulllineitems}

\phantomsection\label{\detokenize{reference/javascript_api:module-web.FavoriteMenu}}

\begin{fulllineitems}
\phantomsection\label{\detokenize{reference/javascript_api:web.FavoriteMenu}}\pysigline{\sphinxbfcode{\sphinxupquote{module }}\sphinxbfcode{\sphinxupquote{web.FavoriteMenu}}}~~\begin{quote}\begin{description}
\item[{Exports}] \leavevmode{\hyperref[\detokenize{reference/javascript_api:web.FavoriteMenu.}]{\sphinxcrossref{
\textless{}anonymous\textgreater{}
}}}
\item[{Depends On}] \leavevmode\begin{itemize}
\item {} {\hyperref[\detokenize{reference/javascript_api:web.Widget}]{\sphinxcrossref{
web.Widget
}}}
\item {} {\hyperref[\detokenize{reference/javascript_api:web.core}]{\sphinxcrossref{
web.core
}}}
\item {} {\hyperref[\detokenize{reference/javascript_api:web.data_manager}]{\sphinxcrossref{
web.data\_manager
}}}
\item {} {\hyperref[\detokenize{reference/javascript_api:web.pyeval}]{\sphinxcrossref{
web.pyeval
}}}
\item {} {\hyperref[\detokenize{reference/javascript_api:web.session}]{\sphinxcrossref{
web.session
}}}
\end{itemize}

\end{description}\end{quote}


\begin{fulllineitems}
\phantomsection\label{\detokenize{reference/javascript_api:web.FavoriteMenu.}}\pysiglinewithargsret{\sphinxbfcode{\sphinxupquote{class }}\sphinxbfcode{\sphinxupquote{}}}{\emph{parent}, \emph{query}, \emph{target\_model}, \emph{action\_id}, \emph{filters}}{}~\begin{quote}\begin{description}
\item[{Extends}] \leavevmode{\hyperref[\detokenize{reference/javascript_api:web.Widget.Widget}]{\sphinxcrossref{
Widget
}}}
\item[{Parameters}] \leavevmode\begin{itemize}

\sphinxstylestrong{parent}

\sphinxstylestrong{query}

\sphinxstylestrong{target\_model}

\sphinxstylestrong{action\_id}

\sphinxstylestrong{filters}
\end{itemize}

\end{description}\end{quote}


\begin{fulllineitems}
\phantomsection\label{\detokenize{reference/javascript_api:start}}\pysiglinewithargsret{\sphinxbfcode{\sphinxupquote{function }}\sphinxbfcode{\sphinxupquote{start}}}{}{}
We manually add the ‘add to dashboard’ feature in the searchview.

\end{fulllineitems}



\begin{fulllineitems}
\phantomsection\label{\detokenize{reference/javascript_api:key_for}}\pysiglinewithargsret{\sphinxbfcode{\sphinxupquote{function }}\sphinxbfcode{\sphinxupquote{key\_for}}}{\emph{filter}}{{ $\rightarrow$ String}}
Generates a mapping key (in the filters and \$filter mappings) for the
filter descriptor object provided (as returned by \sphinxcode{\sphinxupquote{get\_filters}}).

The mapping key is guaranteed to be unique for a given (user\_id, name)
pair.
\begin{quote}\begin{description}
\item[{Parameters}] \leavevmode\begin{itemize}

\sphinxstylestrong{filter} ({\hyperref[\detokenize{reference/javascript_api:web.FavoriteMenu.KeyForFilter}]{\sphinxcrossref{\sphinxstyleliteralemphasis{\sphinxupquote{KeyForFilter}}}}})
\end{itemize}

\item[{Returns}] \leavevmode
mapping key corresponding to the filter

\item[{Return Type}] \leavevmode
\sphinxstyleliteralemphasis{\sphinxupquote{String}}

\end{description}\end{quote}


\begin{fulllineitems}
\phantomsection\label{\detokenize{reference/javascript_api:KeyForFilter}}\pysiglinewithargsret{\sphinxbfcode{\sphinxupquote{class }}\sphinxbfcode{\sphinxupquote{KeyForFilter}}}{}{}~

\begin{fulllineitems}
\phantomsection\label{\detokenize{reference/javascript_api:name}}\pysigline{\sphinxbfcode{\sphinxupquote{attribute }}\sphinxbfcode{\sphinxupquote{name}} String}
\end{fulllineitems}



\begin{fulllineitems}
\phantomsection\label{\detokenize{reference/javascript_api:user_id}}\pysigline{\sphinxbfcode{\sphinxupquote{attribute }}\sphinxbfcode{\sphinxupquote{user\_id}} Number\textbar{}Pair\textless{}Number, String\textgreater{}}
\end{fulllineitems}


\end{fulllineitems}


\end{fulllineitems}



\begin{fulllineitems}
\phantomsection\label{\detokenize{reference/javascript_api:facet_for}}\pysiglinewithargsret{\sphinxbfcode{\sphinxupquote{function }}\sphinxbfcode{\sphinxupquote{facet\_for}}}{\emph{filter}}{{ $\rightarrow$ Object}}
Generates a \sphinxcode{\sphinxupquote{Facet()}} descriptor from a
filter descriptor
\begin{quote}\begin{description}
\item[{Parameters}] \leavevmode\begin{itemize}

\sphinxstylestrong{filter} ({\hyperref[\detokenize{reference/javascript_api:web.FavoriteMenu.FacetForFilter}]{\sphinxcrossref{\sphinxstyleliteralemphasis{\sphinxupquote{FacetForFilter}}}}})
\end{itemize}

\item[{Return Type}] \leavevmode
\sphinxstyleliteralemphasis{\sphinxupquote{Object}}

\end{description}\end{quote}


\begin{fulllineitems}
\phantomsection\label{\detokenize{reference/javascript_api:FacetForFilter}}\pysiglinewithargsret{\sphinxbfcode{\sphinxupquote{class }}\sphinxbfcode{\sphinxupquote{FacetForFilter}}}{}{}~

\begin{fulllineitems}
\phantomsection\label{\detokenize{reference/javascript_api:name}}\pysigline{\sphinxbfcode{\sphinxupquote{attribute }}\sphinxbfcode{\sphinxupquote{name}} String}
\end{fulllineitems}



\begin{fulllineitems}
\phantomsection\label{\detokenize{reference/javascript_api:context}}\pysigline{\sphinxbfcode{\sphinxupquote{attribute }}\sphinxbfcode{\sphinxupquote{context}} Object}
\end{fulllineitems}



\begin{fulllineitems}
\phantomsection\label{\detokenize{reference/javascript_api:domain}}\pysigline{\sphinxbfcode{\sphinxupquote{attribute }}\sphinxbfcode{\sphinxupquote{domain}} Array}
\end{fulllineitems}


\end{fulllineitems}


\end{fulllineitems}



\begin{fulllineitems}
\phantomsection\label{\detokenize{reference/javascript_api:add_filter}}\pysiglinewithargsret{\sphinxbfcode{\sphinxupquote{function }}\sphinxbfcode{\sphinxupquote{add\_filter}}}{\sphinxoptional{\emph{filter}}}{}
Adds a filter description to the filters dict
\begin{quote}\begin{description}
\item[{Parameters}] \leavevmode\begin{itemize}

\sphinxstylestrong{filter} (\sphinxstyleliteralemphasis{\sphinxupquote{Object}}) \textendash{} the filter description
\end{itemize}

\end{description}\end{quote}

\end{fulllineitems}



\begin{fulllineitems}
\phantomsection\label{\detokenize{reference/javascript_api:append_filter}}\pysiglinewithargsret{\sphinxbfcode{\sphinxupquote{function }}\sphinxbfcode{\sphinxupquote{append\_filter}}}{\sphinxoptional{\emph{filter}}}{}
Creates a \$filter JQuery node, adds it to the \$filters dict and appends it to the filter menu
\begin{quote}\begin{description}
\item[{Parameters}] \leavevmode\begin{itemize}

\sphinxstylestrong{filter} (\sphinxstyleliteralemphasis{\sphinxupquote{Object}}) \textendash{} the filter description
\end{itemize}

\end{description}\end{quote}

\end{fulllineitems}


\end{fulllineitems}


\end{fulllineitems}

\phantomsection\label{\detokenize{reference/javascript_api:module-web.field_utils}}

\begin{fulllineitems}
\phantomsection\label{\detokenize{reference/javascript_api:web.field_utils}}\pysigline{\sphinxbfcode{\sphinxupquote{module }}\sphinxbfcode{\sphinxupquote{web.field\_utils}}}~~\begin{quote}\begin{description}
\item[{Exports}] \leavevmode{\hyperref[\detokenize{reference/javascript_api:web.field_utils.}]{\sphinxcrossref{
\textless{}anonymous\textgreater{}
}}}
\item[{Depends On}] \leavevmode\begin{itemize}
\item {} {\hyperref[\detokenize{reference/javascript_api:web.core}]{\sphinxcrossref{
web.core
}}}
\item {} {\hyperref[\detokenize{reference/javascript_api:web.dom}]{\sphinxcrossref{
web.dom
}}}
\item {} {\hyperref[\detokenize{reference/javascript_api:web.session}]{\sphinxcrossref{
web.session
}}}
\item {} {\hyperref[\detokenize{reference/javascript_api:web.time}]{\sphinxcrossref{
web.time
}}}
\item {} {\hyperref[\detokenize{reference/javascript_api:web.utils}]{\sphinxcrossref{
web.utils
}}}
\end{itemize}

\end{description}\end{quote}


\begin{fulllineitems}
\phantomsection\label{\detokenize{reference/javascript_api:formatMany2one}}\pysiglinewithargsret{\sphinxbfcode{\sphinxupquote{function }}\sphinxbfcode{\sphinxupquote{formatMany2one}}}{\emph{value}\sphinxoptional{, \emph{field}}\sphinxoptional{, \emph{options}}}{{ $\rightarrow$ string}}
Returns a string representing an many2one.  If the value is false, then we
return an empty string.  Note that it accepts two types of input parameters:
an array, in that case we assume that the many2one value is of the form
{[}id, nameget{]}, and we return the nameget, or it can be an object, and in that
case, we assume that it is a record from a BasicModel.
\begin{quote}\begin{description}
\item[{Parameters}] \leavevmode\begin{itemize}

\sphinxstylestrong{value} (\sphinxstyleliteralemphasis{\sphinxupquote{Array}}\sphinxstyleemphasis{ or }\sphinxstyleliteralemphasis{\sphinxupquote{Object}}\sphinxstyleemphasis{ or }false)

\sphinxstylestrong{field} (\sphinxstyleliteralemphasis{\sphinxupquote{Object}}) \textendash{} a description of the field (note: this parameter is ignored)

\sphinxstylestrong{options} ({\hyperref[\detokenize{reference/javascript_api:web.field_utils.FormatMany2oneOptions}]{\sphinxcrossref{\sphinxstyleliteralemphasis{\sphinxupquote{FormatMany2oneOptions}}}}}) \textendash{} additional options
\end{itemize}

\item[{Return Type}] \leavevmode
\sphinxstyleliteralemphasis{\sphinxupquote{string}}

\end{description}\end{quote}


\begin{fulllineitems}
\phantomsection\label{\detokenize{reference/javascript_api:FormatMany2oneOptions}}\pysiglinewithargsret{\sphinxbfcode{\sphinxupquote{class }}\sphinxbfcode{\sphinxupquote{FormatMany2oneOptions}}}{}{}
additional options


\begin{fulllineitems}
\phantomsection\label{\detokenize{reference/javascript_api:escape}}\pysigline{\sphinxbfcode{\sphinxupquote{attribute }}\sphinxbfcode{\sphinxupquote{escape}} boolean}
if true, escapes the formatted value

\end{fulllineitems}


\end{fulllineitems}


\end{fulllineitems}



\begin{fulllineitems}
\phantomsection\label{\detokenize{reference/javascript_api:formatInteger}}\pysiglinewithargsret{\sphinxbfcode{\sphinxupquote{function }}\sphinxbfcode{\sphinxupquote{formatInteger}}}{\emph{value}\sphinxoptional{, \emph{field}}\sphinxoptional{, \emph{options}}}{{ $\rightarrow$ string}}
Returns a string representing an integer.  If the value is false, then we
return an empty string.
\begin{quote}\begin{description}
\item[{Parameters}] \leavevmode\begin{itemize}

\sphinxstylestrong{value} (\sphinxstyleliteralemphasis{\sphinxupquote{integer}}\sphinxstyleemphasis{ or }false)

\sphinxstylestrong{field} (\sphinxstyleliteralemphasis{\sphinxupquote{Object}}) \textendash{} a description of the field (note: this parameter is ignored)

\sphinxstylestrong{options} ({\hyperref[\detokenize{reference/javascript_api:web.field_utils.FormatIntegerOptions}]{\sphinxcrossref{\sphinxstyleliteralemphasis{\sphinxupquote{FormatIntegerOptions}}}}}) \textendash{} additional options
\end{itemize}

\item[{Return Type}] \leavevmode
\sphinxstyleliteralemphasis{\sphinxupquote{string}}

\end{description}\end{quote}


\begin{fulllineitems}
\phantomsection\label{\detokenize{reference/javascript_api:FormatIntegerOptions}}\pysiglinewithargsret{\sphinxbfcode{\sphinxupquote{class }}\sphinxbfcode{\sphinxupquote{FormatIntegerOptions}}}{}{}
additional options


\begin{fulllineitems}
\phantomsection\label{\detokenize{reference/javascript_api:isPassword}}\pysigline{\sphinxbfcode{\sphinxupquote{attribute }}\sphinxbfcode{\sphinxupquote{isPassword}} boolean}
if true, returns ‘\sphinxstylestrong{****}’

\end{fulllineitems}


\end{fulllineitems}


\end{fulllineitems}



\begin{fulllineitems}
\phantomsection\label{\detokenize{reference/javascript_api:web.field_utils.}}\pysigline{\sphinxbfcode{\sphinxupquote{namespace }}\sphinxbfcode{\sphinxupquote{}}}
\end{fulllineitems}



\begin{fulllineitems}
\phantomsection\label{\detokenize{reference/javascript_api:formatMonetary}}\pysiglinewithargsret{\sphinxbfcode{\sphinxupquote{function }}\sphinxbfcode{\sphinxupquote{formatMonetary}}}{\emph{value}\sphinxoptional{, \emph{field}}\sphinxoptional{, \emph{options}}}{{ $\rightarrow$ string}}
Returns a string representing a monetary value. The result takes into account
the user settings (to display the correct decimal separator, currency, …).
\begin{quote}\begin{description}
\item[{Parameters}] \leavevmode\begin{itemize}

\sphinxstylestrong{value} (\sphinxstyleliteralemphasis{\sphinxupquote{float}}\sphinxstyleemphasis{ or }false) \textendash{} the value that should be formatted

\sphinxstylestrong{field} (\sphinxstyleliteralemphasis{\sphinxupquote{Object}}) \textendash{} a description of the field (returned by fields\_get for example). It
       may contain a description of the number of digits that should be used.

\sphinxstylestrong{options} ({\hyperref[\detokenize{reference/javascript_api:web.field_utils.FormatMonetaryOptions}]{\sphinxcrossref{\sphinxstyleliteralemphasis{\sphinxupquote{FormatMonetaryOptions}}}}}) \textendash{} additional options to override the values in the python description of
       the field.
\end{itemize}

\item[{Return Type}] \leavevmode
\sphinxstyleliteralemphasis{\sphinxupquote{string}}

\end{description}\end{quote}


\begin{fulllineitems}
\phantomsection\label{\detokenize{reference/javascript_api:FormatMonetaryOptions}}\pysiglinewithargsret{\sphinxbfcode{\sphinxupquote{class }}\sphinxbfcode{\sphinxupquote{FormatMonetaryOptions}}}{}{}~\begin{description}
\item[{additional options to override the values in the python description of}] \leavevmode
the field.

\end{description}


\begin{fulllineitems}
\phantomsection\label{\detokenize{reference/javascript_api:currency}}\pysigline{\sphinxbfcode{\sphinxupquote{attribute }}\sphinxbfcode{\sphinxupquote{currency}} Object}
the description of the currency to use

\end{fulllineitems}



\begin{fulllineitems}
\phantomsection\label{\detokenize{reference/javascript_api:currency_id}}\pysigline{\sphinxbfcode{\sphinxupquote{attribute }}\sphinxbfcode{\sphinxupquote{currency\_id}} integer}
the id of the ‘res.currency’ to use (ignored if options.currency)

\end{fulllineitems}



\begin{fulllineitems}
\phantomsection\label{\detokenize{reference/javascript_api:currency_field}}\pysigline{\sphinxbfcode{\sphinxupquote{attribute }}\sphinxbfcode{\sphinxupquote{currency\_field}} string}~\begin{description}
\item[{the name of the field whose value is the currency id}] \leavevmode
(ignore if options.currency or options.currency\_id)
Note: if not given it will default to the field currency\_field value
or to ‘currency\_id’.

\end{description}

\end{fulllineitems}



\begin{fulllineitems}
\phantomsection\label{\detokenize{reference/javascript_api:data}}\pysigline{\sphinxbfcode{\sphinxupquote{attribute }}\sphinxbfcode{\sphinxupquote{data}} Object}~\begin{description}
\item[{a mapping of field name to field value, required with}] \leavevmode
options.currency\_field

\end{description}

\end{fulllineitems}



\begin{fulllineitems}
\phantomsection\label{\detokenize{reference/javascript_api:digits}}\pysigline{\sphinxbfcode{\sphinxupquote{attribute }}\sphinxbfcode{\sphinxupquote{digits}} integer{[}{]}}~\begin{description}
\item[{the number of digits that should be used, instead of the default}] \leavevmode
digits precision in the field. Note: if the currency defines a
precision, the currency’s one is used.

\end{description}

\end{fulllineitems}


\end{fulllineitems}


\end{fulllineitems}



\begin{fulllineitems}
\phantomsection\label{\detokenize{reference/javascript_api:formatDate}}\pysiglinewithargsret{\sphinxbfcode{\sphinxupquote{function }}\sphinxbfcode{\sphinxupquote{formatDate}}}{\emph{value}\sphinxoptional{, \emph{field}}\sphinxoptional{, \emph{options}}}{{ $\rightarrow$ string}}
Returns a string representing a date.  If the value is false, then we return
an empty string. Note that this is dependant on the localization settings
\begin{quote}\begin{description}
\item[{Parameters}] \leavevmode\begin{itemize}

\sphinxstylestrong{value} (\sphinxstyleliteralemphasis{\sphinxupquote{Moment}}\sphinxstyleemphasis{ or }false)

\sphinxstylestrong{field} (\sphinxstyleliteralemphasis{\sphinxupquote{Object}}) \textendash{} a description of the field (note: this parameter is ignored)

\sphinxstylestrong{options} ({\hyperref[\detokenize{reference/javascript_api:web.field_utils.FormatDateOptions}]{\sphinxcrossref{\sphinxstyleliteralemphasis{\sphinxupquote{FormatDateOptions}}}}}) \textendash{} additional options
\end{itemize}

\item[{Return Type}] \leavevmode
\sphinxstyleliteralemphasis{\sphinxupquote{string}}

\end{description}\end{quote}


\begin{fulllineitems}
\phantomsection\label{\detokenize{reference/javascript_api:FormatDateOptions}}\pysiglinewithargsret{\sphinxbfcode{\sphinxupquote{class }}\sphinxbfcode{\sphinxupquote{FormatDateOptions}}}{}{}
additional options


\begin{fulllineitems}
\phantomsection\label{\detokenize{reference/javascript_api:timezone}}\pysigline{\sphinxbfcode{\sphinxupquote{attribute }}\sphinxbfcode{\sphinxupquote{timezone}} boolean}~\begin{description}
\item[{use the user timezone when formating the}] \leavevmode
date

\end{description}

\end{fulllineitems}


\end{fulllineitems}


\end{fulllineitems}



\begin{fulllineitems}
\phantomsection\label{\detokenize{reference/javascript_api:parseMonetary}}\pysiglinewithargsret{\sphinxbfcode{\sphinxupquote{function }}\sphinxbfcode{\sphinxupquote{parseMonetary}}}{\emph{value}\sphinxoptional{, \emph{field}}\sphinxoptional{, \emph{options}}}{{ $\rightarrow$ float}}
Parse a String containing currency symbol and returns amount
\begin{quote}\begin{description}
\item[{Parameters}] \leavevmode\begin{itemize}

\sphinxstylestrong{value} (\sphinxstyleliteralemphasis{\sphinxupquote{string}}) \textendash{} The string to be parsed
               We assume that a monetary is always a pair (symbol, amount) separated
               by a non breaking space. A simple float can also be accepted as value

\sphinxstylestrong{field} (\sphinxstyleliteralemphasis{\sphinxupquote{Object}}) \textendash{} a description of the field (returned by fields\_get for example).

\sphinxstylestrong{options} ({\hyperref[\detokenize{reference/javascript_api:web.field_utils.ParseMonetaryOptions}]{\sphinxcrossref{\sphinxstyleliteralemphasis{\sphinxupquote{ParseMonetaryOptions}}}}}) \textendash{} additional options.
\end{itemize}

\item[{Returns}] \leavevmode
the float value contained in the string representation

\item[{Return Type}] \leavevmode
\sphinxstyleliteralemphasis{\sphinxupquote{float}}

\end{description}\end{quote}


\begin{fulllineitems}
\phantomsection\label{\detokenize{reference/javascript_api:ParseMonetaryOptions}}\pysiglinewithargsret{\sphinxbfcode{\sphinxupquote{class }}\sphinxbfcode{\sphinxupquote{ParseMonetaryOptions}}}{}{}
additional options.


\begin{fulllineitems}
\phantomsection\label{\detokenize{reference/javascript_api:currency}}\pysigline{\sphinxbfcode{\sphinxupquote{attribute }}\sphinxbfcode{\sphinxupquote{currency}} Object}
the description of the currency to use

\end{fulllineitems}



\begin{fulllineitems}
\phantomsection\label{\detokenize{reference/javascript_api:currency_id}}\pysigline{\sphinxbfcode{\sphinxupquote{attribute }}\sphinxbfcode{\sphinxupquote{currency\_id}} integer}
the id of the ‘res.currency’ to use (ignored if options.currency)

\end{fulllineitems}



\begin{fulllineitems}
\phantomsection\label{\detokenize{reference/javascript_api:currency_field}}\pysigline{\sphinxbfcode{\sphinxupquote{attribute }}\sphinxbfcode{\sphinxupquote{currency\_field}} string}~\begin{description}
\item[{the name of the field whose value is the currency id}] \leavevmode
(ignore if options.currency or options.currency\_id)
Note: if not given it will default to the field currency\_field value
or to ‘currency\_id’.

\end{description}

\end{fulllineitems}



\begin{fulllineitems}
\phantomsection\label{\detokenize{reference/javascript_api:data}}\pysigline{\sphinxbfcode{\sphinxupquote{attribute }}\sphinxbfcode{\sphinxupquote{data}} Object}~\begin{description}
\item[{a mapping of field name to field value, required with}] \leavevmode
options.currency\_field

\end{description}

\end{fulllineitems}


\end{fulllineitems}


\end{fulllineitems}



\begin{fulllineitems}
\phantomsection\label{\detokenize{reference/javascript_api:parseDate}}\pysiglinewithargsret{\sphinxbfcode{\sphinxupquote{function }}\sphinxbfcode{\sphinxupquote{parseDate}}}{\emph{value}\sphinxoptional{, \emph{field}}\sphinxoptional{, \emph{options}}}{{ $\rightarrow$ Moment\textbar{}false}}
Create an Date object
The method toJSON return the formated value to send value server side
\begin{quote}\begin{description}
\item[{Parameters}] \leavevmode\begin{itemize}

\sphinxstylestrong{value} (\sphinxstyleliteralemphasis{\sphinxupquote{string}})

\sphinxstylestrong{field} (\sphinxstyleliteralemphasis{\sphinxupquote{Object}}) \textendash{} a description of the field (note: this parameter is ignored)

\sphinxstylestrong{options} ({\hyperref[\detokenize{reference/javascript_api:web.field_utils.ParseDateOptions}]{\sphinxcrossref{\sphinxstyleliteralemphasis{\sphinxupquote{ParseDateOptions}}}}}) \textendash{} additional options
\end{itemize}

\item[{Returns}] \leavevmode
Moment date object

\item[{Return Type}] \leavevmode
\sphinxstyleliteralemphasis{\sphinxupquote{Moment}}\sphinxstyleemphasis{ or }false

\end{description}\end{quote}


\begin{fulllineitems}
\phantomsection\label{\detokenize{reference/javascript_api:ParseDateOptions}}\pysiglinewithargsret{\sphinxbfcode{\sphinxupquote{class }}\sphinxbfcode{\sphinxupquote{ParseDateOptions}}}{}{}
additional options


\begin{fulllineitems}
\phantomsection\label{\detokenize{reference/javascript_api:isUTC}}\pysigline{\sphinxbfcode{\sphinxupquote{attribute }}\sphinxbfcode{\sphinxupquote{isUTC}} boolean}
the formatted date is utc

\end{fulllineitems}



\begin{fulllineitems}
\phantomsection\label{\detokenize{reference/javascript_api:timezone}}\pysigline{\sphinxbfcode{\sphinxupquote{attribute }}\sphinxbfcode{\sphinxupquote{timezone}} boolean}~\begin{description}
\item[{format the date after apply the timezone}] \leavevmode
offset

\end{description}

\end{fulllineitems}


\end{fulllineitems}


\end{fulllineitems}



\begin{fulllineitems}
\phantomsection\label{\detokenize{reference/javascript_api:formatX2Many}}\pysiglinewithargsret{\sphinxbfcode{\sphinxupquote{function }}\sphinxbfcode{\sphinxupquote{formatX2Many}}}{\emph{value}}{{ $\rightarrow$ string}}
Returns a string indicating the number of records in the relation.
\begin{quote}\begin{description}
\item[{Parameters}] \leavevmode\begin{itemize}

\sphinxstylestrong{value} (\sphinxstyleliteralemphasis{\sphinxupquote{Object}}) \textendash{} a valid element from a BasicModel, that represents a
  list of values
\end{itemize}

\item[{Return Type}] \leavevmode
\sphinxstyleliteralemphasis{\sphinxupquote{string}}

\end{description}\end{quote}

\end{fulllineitems}



\begin{fulllineitems}
\phantomsection\label{\detokenize{reference/javascript_api:formatSelection}}\pysiglinewithargsret{\sphinxbfcode{\sphinxupquote{function }}\sphinxbfcode{\sphinxupquote{formatSelection}}}{\emph{value}\sphinxoptional{, \emph{field}}\sphinxoptional{, \emph{options}}}{}
Returns a string representing the value of the selection.
\begin{quote}\begin{description}
\item[{Parameters}] \leavevmode\begin{itemize}

\sphinxstylestrong{value} (\sphinxstyleliteralemphasis{\sphinxupquote{string}}\sphinxstyleemphasis{ or }false)

\sphinxstylestrong{field} (\sphinxstyleliteralemphasis{\sphinxupquote{Object}}) \textendash{} a description of the field (note: this parameter is ignored)

\sphinxstylestrong{options} ({\hyperref[\detokenize{reference/javascript_api:web.field_utils.FormatSelectionOptions}]{\sphinxcrossref{\sphinxstyleliteralemphasis{\sphinxupquote{FormatSelectionOptions}}}}}) \textendash{} additional options
\end{itemize}

\end{description}\end{quote}


\begin{fulllineitems}
\phantomsection\label{\detokenize{reference/javascript_api:FormatSelectionOptions}}\pysiglinewithargsret{\sphinxbfcode{\sphinxupquote{class }}\sphinxbfcode{\sphinxupquote{FormatSelectionOptions}}}{}{}
additional options


\begin{fulllineitems}
\phantomsection\label{\detokenize{reference/javascript_api:escape}}\pysigline{\sphinxbfcode{\sphinxupquote{attribute }}\sphinxbfcode{\sphinxupquote{escape}} boolean}
if true, escapes the formatted value

\end{fulllineitems}


\end{fulllineitems}


\end{fulllineitems}



\begin{fulllineitems}
\phantomsection\label{\detokenize{reference/javascript_api:formatBinary}}\pysiglinewithargsret{\sphinxbfcode{\sphinxupquote{function }}\sphinxbfcode{\sphinxupquote{formatBinary}}}{\sphinxoptional{\emph{value}}\sphinxoptional{, \emph{field}}\sphinxoptional{, \emph{options}}}{{ $\rightarrow$ string}}
Convert binary to bin\_size
\begin{quote}\begin{description}
\item[{Parameters}] \leavevmode\begin{itemize}

\sphinxstylestrong{value} (\sphinxstyleliteralemphasis{\sphinxupquote{string}}) \textendash{} base64 representation of the binary (might be already a bin\_size!)

\sphinxstylestrong{field} (\sphinxstyleliteralemphasis{\sphinxupquote{Object}}) \textendash{} a description of the field (note: this parameter is ignored)

\sphinxstylestrong{options} (\sphinxstyleliteralemphasis{\sphinxupquote{Object}}) \textendash{} additional options (note: this parameter is ignored)
\end{itemize}

\item[{Returns}] \leavevmode
bin\_size (which is human-readable)

\item[{Return Type}] \leavevmode
\sphinxstyleliteralemphasis{\sphinxupquote{string}}

\end{description}\end{quote}

\end{fulllineitems}



\begin{fulllineitems}
\phantomsection\label{\detokenize{reference/javascript_api:formatDateTime}}\pysiglinewithargsret{\sphinxbfcode{\sphinxupquote{function }}\sphinxbfcode{\sphinxupquote{formatDateTime}}}{\emph{value}\sphinxoptional{, \emph{field}}\sphinxoptional{, \emph{options}}}{{ $\rightarrow$ string}}
Returns a string representing a datetime.  If the value is false, then we
return an empty string.  Note that this is dependant on the localization
settings
\begin{quote}\begin{description}
\item[{Parameters}] \leavevmode\begin{itemize}

\sphinxstylestrong{value}

\sphinxstylestrong{field} (\sphinxstyleliteralemphasis{\sphinxupquote{Object}}) \textendash{} a description of the field (note: this parameter is ignored)

\sphinxstylestrong{options} ({\hyperref[\detokenize{reference/javascript_api:web.field_utils.FormatDateTimeOptions}]{\sphinxcrossref{\sphinxstyleliteralemphasis{\sphinxupquote{FormatDateTimeOptions}}}}}) \textendash{} additional options
\end{itemize}

\item[{Return Type}] \leavevmode
\sphinxstyleliteralemphasis{\sphinxupquote{string}}

\end{description}\end{quote}


\begin{fulllineitems}
\phantomsection\label{\detokenize{reference/javascript_api:FormatDateTimeOptions}}\pysiglinewithargsret{\sphinxbfcode{\sphinxupquote{class }}\sphinxbfcode{\sphinxupquote{FormatDateTimeOptions}}}{}{}
additional options


\begin{fulllineitems}
\phantomsection\label{\detokenize{reference/javascript_api:timezone}}\pysigline{\sphinxbfcode{\sphinxupquote{attribute }}\sphinxbfcode{\sphinxupquote{timezone}} boolean}~\begin{description}
\item[{use the user timezone when formating the}] \leavevmode
date

\end{description}

\end{fulllineitems}


\end{fulllineitems}


\end{fulllineitems}



\begin{fulllineitems}
\phantomsection\label{\detokenize{reference/javascript_api:parseFloat}}\pysiglinewithargsret{\sphinxbfcode{\sphinxupquote{function }}\sphinxbfcode{\sphinxupquote{parseFloat}}}{\emph{value}}{{ $\rightarrow$ float}}
Parse a String containing float in language formating
\begin{quote}\begin{description}
\item[{Parameters}] \leavevmode\begin{itemize}

\sphinxstylestrong{value} (\sphinxstyleliteralemphasis{\sphinxupquote{string}}) \textendash{} The string to be parsed with the setting of thousands and
               decimal separator
\end{itemize}

\item[{Return Type}] \leavevmode
\sphinxstyleliteralemphasis{\sphinxupquote{float}}

\end{description}\end{quote}

\end{fulllineitems}



\begin{fulllineitems}
\phantomsection\label{\detokenize{reference/javascript_api:formatFloat}}\pysiglinewithargsret{\sphinxbfcode{\sphinxupquote{function }}\sphinxbfcode{\sphinxupquote{formatFloat}}}{\emph{value}\sphinxoptional{, \emph{field}}\sphinxoptional{, \emph{options}}}{{ $\rightarrow$ string}}
Returns a string representing a float.  The result takes into account the
user settings (to display the correct decimal separator).
\begin{quote}\begin{description}
\item[{Parameters}] \leavevmode\begin{itemize}

\sphinxstylestrong{value} (\sphinxstyleliteralemphasis{\sphinxupquote{float}}\sphinxstyleemphasis{ or }false) \textendash{} the value that should be formatted

\sphinxstylestrong{field} (\sphinxstyleliteralemphasis{\sphinxupquote{Object}}) \textendash{} a description of the field (returned by fields\_get
  for example).  It may contain a description of the number of digits that
  should be used.

\sphinxstylestrong{options} ({\hyperref[\detokenize{reference/javascript_api:web.field_utils.FormatFloatOptions}]{\sphinxcrossref{\sphinxstyleliteralemphasis{\sphinxupquote{FormatFloatOptions}}}}}) \textendash{} additional options to override the values in the
  python description of the field.
\end{itemize}

\item[{Return Type}] \leavevmode
\sphinxstyleliteralemphasis{\sphinxupquote{string}}

\end{description}\end{quote}


\begin{fulllineitems}
\phantomsection\label{\detokenize{reference/javascript_api:FormatFloatOptions}}\pysiglinewithargsret{\sphinxbfcode{\sphinxupquote{class }}\sphinxbfcode{\sphinxupquote{FormatFloatOptions}}}{}{}~\begin{description}
\item[{additional options to override the values in the}] \leavevmode
python description of the field.

\end{description}


\begin{fulllineitems}
\phantomsection\label{\detokenize{reference/javascript_api:digits}}\pysigline{\sphinxbfcode{\sphinxupquote{attribute }}\sphinxbfcode{\sphinxupquote{digits}} integer{[}{]}}~\begin{description}
\item[{the number of digits that should be used,}] \leavevmode
instead of the default digits precision in the field.

\end{description}

\end{fulllineitems}


\end{fulllineitems}


\end{fulllineitems}



\begin{fulllineitems}
\phantomsection\label{\detokenize{reference/javascript_api:parseDateTime}}\pysiglinewithargsret{\sphinxbfcode{\sphinxupquote{function }}\sphinxbfcode{\sphinxupquote{parseDateTime}}}{\emph{value}\sphinxoptional{, \emph{field}}\sphinxoptional{, \emph{options}}}{{ $\rightarrow$ Moment\textbar{}false}}
Create an Date object
The method toJSON return the formated value to send value server side
\begin{quote}\begin{description}
\item[{Parameters}] \leavevmode\begin{itemize}

\sphinxstylestrong{value} (\sphinxstyleliteralemphasis{\sphinxupquote{string}})

\sphinxstylestrong{field} (\sphinxstyleliteralemphasis{\sphinxupquote{Object}}) \textendash{} a description of the field (note: this parameter is ignored)

\sphinxstylestrong{options} ({\hyperref[\detokenize{reference/javascript_api:web.field_utils.ParseDateTimeOptions}]{\sphinxcrossref{\sphinxstyleliteralemphasis{\sphinxupquote{ParseDateTimeOptions}}}}}) \textendash{} additional options
\end{itemize}

\item[{Returns}] \leavevmode
Moment date object

\item[{Return Type}] \leavevmode
\sphinxstyleliteralemphasis{\sphinxupquote{Moment}}\sphinxstyleemphasis{ or }false

\end{description}\end{quote}


\begin{fulllineitems}
\phantomsection\label{\detokenize{reference/javascript_api:ParseDateTimeOptions}}\pysiglinewithargsret{\sphinxbfcode{\sphinxupquote{class }}\sphinxbfcode{\sphinxupquote{ParseDateTimeOptions}}}{}{}
additional options


\begin{fulllineitems}
\phantomsection\label{\detokenize{reference/javascript_api:isUTC}}\pysigline{\sphinxbfcode{\sphinxupquote{attribute }}\sphinxbfcode{\sphinxupquote{isUTC}} boolean}
the formatted date is utc

\end{fulllineitems}



\begin{fulllineitems}
\phantomsection\label{\detokenize{reference/javascript_api:timezone}}\pysigline{\sphinxbfcode{\sphinxupquote{attribute }}\sphinxbfcode{\sphinxupquote{timezone}} boolean}~\begin{description}
\item[{format the date after apply the timezone}] \leavevmode
offset

\end{description}

\end{fulllineitems}


\end{fulllineitems}


\end{fulllineitems}



\begin{fulllineitems}
\phantomsection\label{\detokenize{reference/javascript_api:parseMany2one}}\pysiglinewithargsret{\sphinxbfcode{\sphinxupquote{function }}\sphinxbfcode{\sphinxupquote{parseMany2one}}}{\emph{value}}{{ $\rightarrow$ Object}}
Creates an object with id and display\_name.
\begin{quote}\begin{description}
\item[{Parameters}] \leavevmode\begin{itemize}

\sphinxstylestrong{value} (\sphinxstyleliteralemphasis{\sphinxupquote{Array}}\sphinxstyleemphasis{ or }\sphinxstyleliteralemphasis{\sphinxupquote{number}}\sphinxstyleemphasis{ or }\sphinxstyleliteralemphasis{\sphinxupquote{string}}\sphinxstyleemphasis{ or }\sphinxstyleliteralemphasis{\sphinxupquote{Object}}) \textendash{} The given value can be :
       - an array with id as first element and display\_name as second element
       - a number or a string representing the id (the display\_name will be
         returned as undefined)
       - an object, simply returned untouched
\end{itemize}

\item[{Returns}] \leavevmode
(contains the id and display\_name)
                  Note: if the given value is not an array, a string or a
                  number, the value is returned untouched.

\item[{Return Type}] \leavevmode
\sphinxstyleliteralemphasis{\sphinxupquote{Object}}

\end{description}\end{quote}

\end{fulllineitems}



\begin{fulllineitems}
\phantomsection\label{\detokenize{reference/javascript_api:parseNumber}}\pysiglinewithargsret{\sphinxbfcode{\sphinxupquote{function }}\sphinxbfcode{\sphinxupquote{parseNumber}}}{\emph{value}}{{ $\rightarrow$ float\textbar{}NaN}}
Parse a String containing number in language formating
\begin{quote}\begin{description}
\item[{Parameters}] \leavevmode\begin{itemize}

\sphinxstylestrong{value} (\sphinxstyleliteralemphasis{\sphinxupquote{string}}) \textendash{} The string to be parsed with the setting of thousands and
               decimal separator
\end{itemize}

\item[{Returns}] \leavevmode
the number value contained in the string representation

\item[{Return Type}] \leavevmode
\sphinxstyleliteralemphasis{\sphinxupquote{float}}\sphinxstyleemphasis{ or }\sphinxstyleliteralemphasis{\sphinxupquote{NaN}}

\end{description}\end{quote}

\end{fulllineitems}



\begin{fulllineitems}
\phantomsection\label{\detokenize{reference/javascript_api:formatChar}}\pysiglinewithargsret{\sphinxbfcode{\sphinxupquote{function }}\sphinxbfcode{\sphinxupquote{formatChar}}}{\emph{value}\sphinxoptional{, \emph{field}}\sphinxoptional{, \emph{options}}}{{ $\rightarrow$ string}}
Returns a string representing a char.  If the value is false, then we return
an empty string.
\begin{quote}\begin{description}
\item[{Parameters}] \leavevmode\begin{itemize}

\sphinxstylestrong{value} (\sphinxstyleliteralemphasis{\sphinxupquote{string}}\sphinxstyleemphasis{ or }false)

\sphinxstylestrong{field} (\sphinxstyleliteralemphasis{\sphinxupquote{Object}}) \textendash{} a description of the field (note: this parameter is ignored)

\sphinxstylestrong{options} ({\hyperref[\detokenize{reference/javascript_api:web.field_utils.FormatCharOptions}]{\sphinxcrossref{\sphinxstyleliteralemphasis{\sphinxupquote{FormatCharOptions}}}}}) \textendash{} additional options
\end{itemize}

\item[{Return Type}] \leavevmode
\sphinxstyleliteralemphasis{\sphinxupquote{string}}

\end{description}\end{quote}


\begin{fulllineitems}
\phantomsection\label{\detokenize{reference/javascript_api:FormatCharOptions}}\pysiglinewithargsret{\sphinxbfcode{\sphinxupquote{class }}\sphinxbfcode{\sphinxupquote{FormatCharOptions}}}{}{}
additional options


\begin{fulllineitems}
\phantomsection\label{\detokenize{reference/javascript_api:escape}}\pysigline{\sphinxbfcode{\sphinxupquote{attribute }}\sphinxbfcode{\sphinxupquote{escape}} boolean}
if true, escapes the formatted value

\end{fulllineitems}



\begin{fulllineitems}
\phantomsection\label{\detokenize{reference/javascript_api:isPassword}}\pysigline{\sphinxbfcode{\sphinxupquote{attribute }}\sphinxbfcode{\sphinxupquote{isPassword}} boolean}~\begin{description}
\item[{if true, returns ‘\sphinxstylestrong{****}’}] \leavevmode
instead of the formatted value

\end{description}

\end{fulllineitems}


\end{fulllineitems}


\end{fulllineitems}



\begin{fulllineitems}
\phantomsection\label{\detokenize{reference/javascript_api:parseInteger}}\pysiglinewithargsret{\sphinxbfcode{\sphinxupquote{function }}\sphinxbfcode{\sphinxupquote{parseInteger}}}{\emph{value}}{{ $\rightarrow$ integer}}
Parse a String containing integer with language formating
\begin{quote}\begin{description}
\item[{Parameters}] \leavevmode\begin{itemize}

\sphinxstylestrong{value} (\sphinxstyleliteralemphasis{\sphinxupquote{string}}) \textendash{} The string to be parsed with the setting of thousands and
               decimal separator
\end{itemize}

\item[{Return Type}] \leavevmode
\sphinxstyleliteralemphasis{\sphinxupquote{integer}}

\end{description}\end{quote}

\end{fulllineitems}



\begin{fulllineitems}
\phantomsection\label{\detokenize{reference/javascript_api:formatFloatTime}}\pysiglinewithargsret{\sphinxbfcode{\sphinxupquote{function }}\sphinxbfcode{\sphinxupquote{formatFloatTime}}}{\emph{value}}{{ $\rightarrow$ string}}
Returns a string representing a time value, from a float.  The idea is that
we sometimes want to display something like 1:45 instead of 1.75, or 0:15
instead of 0.25.
\begin{quote}\begin{description}
\item[{Parameters}] \leavevmode\begin{itemize}

\sphinxstylestrong{value} (\sphinxstyleliteralemphasis{\sphinxupquote{float}})
\end{itemize}

\item[{Return Type}] \leavevmode
\sphinxstyleliteralemphasis{\sphinxupquote{string}}

\end{description}\end{quote}

\end{fulllineitems}


\end{fulllineitems}

\phantomsection\label{\detokenize{reference/javascript_api:module-website_slides.upload}}

\begin{fulllineitems}
\phantomsection\label{\detokenize{reference/javascript_api:website_slides.upload}}\pysigline{\sphinxbfcode{\sphinxupquote{module }}\sphinxbfcode{\sphinxupquote{website\_slides.upload}}}~~\begin{quote}\begin{description}
\item[{Exports}] \leavevmode{\hyperref[\detokenize{reference/javascript_api:website_slides.upload.}]{\sphinxcrossref{
\textless{}anonymous\textgreater{}
}}}
\item[{Depends On}] \leavevmode\begin{itemize}
\item {} {\hyperref[\detokenize{reference/javascript_api:web.Widget}]{\sphinxcrossref{
web.Widget
}}}
\item {} {\hyperref[\detokenize{reference/javascript_api:web.ajax}]{\sphinxcrossref{
web.ajax
}}}
\item {} {\hyperref[\detokenize{reference/javascript_api:web.core}]{\sphinxcrossref{
web.core
}}}
\item {} {\hyperref[\detokenize{reference/javascript_api:web_editor.context}]{\sphinxcrossref{
web\_editor.context
}}}
\item {} {\hyperref[\detokenize{reference/javascript_api:website_slides.slides}]{\sphinxcrossref{
website\_slides.slides
}}}
\end{itemize}

\end{description}\end{quote}


\begin{fulllineitems}
\phantomsection\label{\detokenize{reference/javascript_api:website_slides.upload.}}\pysigline{\sphinxbfcode{\sphinxupquote{namespace }}\sphinxbfcode{\sphinxupquote{}}}
\end{fulllineitems}


\end{fulllineitems}

\phantomsection\label{\detokenize{reference/javascript_api:module-web.search_inputs}}

\begin{fulllineitems}
\phantomsection\label{\detokenize{reference/javascript_api:web.search_inputs}}\pysigline{\sphinxbfcode{\sphinxupquote{module }}\sphinxbfcode{\sphinxupquote{web.search\_inputs}}}~~\begin{quote}\begin{description}
\item[{Exports}] \leavevmode{\hyperref[\detokenize{reference/javascript_api:web.search_inputs.}]{\sphinxcrossref{
\textless{}anonymous\textgreater{}
}}}
\item[{Depends On}] \leavevmode\begin{itemize}
\item {} {\hyperref[\detokenize{reference/javascript_api:web.Context}]{\sphinxcrossref{
web.Context
}}}
\item {} {\hyperref[\detokenize{reference/javascript_api:web.Domain}]{\sphinxcrossref{
web.Domain
}}}
\item {} {\hyperref[\detokenize{reference/javascript_api:web.Widget}]{\sphinxcrossref{
web.Widget
}}}
\item {} {\hyperref[\detokenize{reference/javascript_api:web.core}]{\sphinxcrossref{
web.core
}}}
\item {} {\hyperref[\detokenize{reference/javascript_api:web.field_utils}]{\sphinxcrossref{
web.field\_utils
}}}
\item {} {\hyperref[\detokenize{reference/javascript_api:web.pyeval}]{\sphinxcrossref{
web.pyeval
}}}
\item {} {\hyperref[\detokenize{reference/javascript_api:web.time}]{\sphinxcrossref{
web.time
}}}
\item {} {\hyperref[\detokenize{reference/javascript_api:web.utils}]{\sphinxcrossref{
web.utils
}}}
\end{itemize}

\end{description}\end{quote}


\begin{fulllineitems}
\phantomsection\label{\detokenize{reference/javascript_api:web.search_inputs.}}\pysigline{\sphinxbfcode{\sphinxupquote{namespace }}\sphinxbfcode{\sphinxupquote{}}}
\end{fulllineitems}



\begin{fulllineitems}
\phantomsection\label{\detokenize{reference/javascript_api:CharField}}\pysiglinewithargsret{\sphinxbfcode{\sphinxupquote{class }}\sphinxbfcode{\sphinxupquote{CharField}}}{}{}~\begin{quote}\begin{description}
\item[{Extends}] \leavevmode
Field

\end{description}\end{quote}

Implementation of the \sphinxcode{\sphinxupquote{char}} OpenERP field type:
\begin{itemize}
\item {} 
Default operator is \sphinxcode{\sphinxupquote{ilike}} rather than \sphinxcode{\sphinxupquote{=}}

\item {} 
The Javascript and the HTML values are identical (strings)

\end{itemize}

\end{fulllineitems}



\begin{fulllineitems}
\phantomsection\label{\detokenize{reference/javascript_api:DateTimeField}}\pysiglinewithargsret{\sphinxbfcode{\sphinxupquote{class }}\sphinxbfcode{\sphinxupquote{DateTimeField}}}{}{}~\begin{quote}\begin{description}
\item[{Extends}] \leavevmode
DateField

\end{description}\end{quote}

Implementation of the \sphinxcode{\sphinxupquote{datetime}} openerp field type:
\begin{itemize}
\item {} 
Uses the same widget as the \sphinxcode{\sphinxupquote{date}} field type (a simple date)

\item {} 
Builds a slighly more complex, it’s a datetime range (includes time)
spanning the whole day selected by the date widget

\end{itemize}

\end{fulllineitems}



\begin{fulllineitems}
\phantomsection\label{\detokenize{reference/javascript_api:facet_from}}\pysiglinewithargsret{\sphinxbfcode{\sphinxupquote{function }}\sphinxbfcode{\sphinxupquote{facet\_from}}}{\emph{field}, \emph{pair}}{}
Utility function for m2o \& selection fields taking a selection/name\_get pair
(value, name) and converting it to a Facet descriptor
\begin{quote}\begin{description}
\item[{Parameters}] \leavevmode\begin{itemize}

\sphinxstylestrong{field} (\sphinxstyleliteralemphasis{\sphinxupquote{instance.web.search.Field}}) \textendash{} holder field

\sphinxstylestrong{pair} (\sphinxstyleliteralemphasis{\sphinxupquote{Array}}) \textendash{} pair value to convert
\end{itemize}

\end{description}\end{quote}

\end{fulllineitems}


\end{fulllineitems}

\phantomsection\label{\detokenize{reference/javascript_api:module-web_diagram.DiagramModel}}

\begin{fulllineitems}
\phantomsection\label{\detokenize{reference/javascript_api:web_diagram.DiagramModel}}\pysigline{\sphinxbfcode{\sphinxupquote{module }}\sphinxbfcode{\sphinxupquote{web\_diagram.DiagramModel}}}~~\begin{quote}\begin{description}
\item[{Exports}] \leavevmode{\hyperref[\detokenize{reference/javascript_api:web_diagram.DiagramModel.DiagramModel}]{\sphinxcrossref{
DiagramModel
}}}
\item[{Depends On}] \leavevmode\begin{itemize}
\item {} {\hyperref[\detokenize{reference/javascript_api:web.AbstractModel}]{\sphinxcrossref{
web.AbstractModel
}}}
\end{itemize}

\end{description}\end{quote}


\begin{fulllineitems}
\phantomsection\label{\detokenize{reference/javascript_api:DiagramModel}}\pysiglinewithargsret{\sphinxbfcode{\sphinxupquote{class }}\sphinxbfcode{\sphinxupquote{DiagramModel}}}{}{}~\begin{quote}\begin{description}
\item[{Extends}] \leavevmode{\hyperref[\detokenize{reference/javascript_api:web.AbstractModel.AbstractModel}]{\sphinxcrossref{
AbstractModel
}}}
\end{description}\end{quote}

DiagramModel

\end{fulllineitems}



\begin{fulllineitems}
\phantomsection\label{\detokenize{reference/javascript_api:DiagramModel}}\pysiglinewithargsret{\sphinxbfcode{\sphinxupquote{class }}\sphinxbfcode{\sphinxupquote{DiagramModel}}}{}{}~\begin{quote}\begin{description}
\item[{Extends}] \leavevmode{\hyperref[\detokenize{reference/javascript_api:web.AbstractModel.AbstractModel}]{\sphinxcrossref{
AbstractModel
}}}
\end{description}\end{quote}

DiagramModel

\end{fulllineitems}


\end{fulllineitems}

\phantomsection\label{\detokenize{reference/javascript_api:module-web_editor.widget}}

\begin{fulllineitems}
\phantomsection\label{\detokenize{reference/javascript_api:web_editor.widget}}\pysigline{\sphinxbfcode{\sphinxupquote{module }}\sphinxbfcode{\sphinxupquote{web\_editor.widget}}}~~\begin{quote}\begin{description}
\item[{Exports}] \leavevmode{\hyperref[\detokenize{reference/javascript_api:web_editor.widget.}]{\sphinxcrossref{
\textless{}anonymous\textgreater{}
}}}
\item[{Depends On}] \leavevmode\begin{itemize}
\item {} {\hyperref[\detokenize{reference/javascript_api:web.Dialog}]{\sphinxcrossref{
web.Dialog
}}}
\item {} {\hyperref[\detokenize{reference/javascript_api:web.Widget}]{\sphinxcrossref{
web.Widget
}}}
\item {} {\hyperref[\detokenize{reference/javascript_api:web.core}]{\sphinxcrossref{
web.core
}}}
\item {} {\hyperref[\detokenize{reference/javascript_api:web.utils}]{\sphinxcrossref{
web.utils
}}}
\item {} {\hyperref[\detokenize{reference/javascript_api:web_editor.context}]{\sphinxcrossref{
web\_editor.context
}}}
\end{itemize}

\end{description}\end{quote}


\begin{fulllineitems}
\phantomsection\label{\detokenize{reference/javascript_api:web_editor.widget.}}\pysigline{\sphinxbfcode{\sphinxupquote{namespace }}\sphinxbfcode{\sphinxupquote{}}}~

\begin{fulllineitems}
\phantomsection\label{\detokenize{reference/javascript_api:Dialog}}\pysiglinewithargsret{\sphinxbfcode{\sphinxupquote{class }}\sphinxbfcode{\sphinxupquote{Dialog}}}{\emph{parent}, \emph{options}}{}~\begin{quote}\begin{description}
\item[{Extends}] \leavevmode{\hyperref[\detokenize{reference/javascript_api:web.Dialog.Dialog}]{\sphinxcrossref{
Dialog
}}}
\item[{Parameters}] \leavevmode\begin{itemize}

\sphinxstylestrong{parent}

\sphinxstylestrong{options}
\end{itemize}

\end{description}\end{quote}

Extend Dialog class to handle save/cancel of edition components.

\end{fulllineitems}



\begin{fulllineitems}
\phantomsection\label{\detokenize{reference/javascript_api:alt}}\pysiglinewithargsret{\sphinxbfcode{\sphinxupquote{class }}\sphinxbfcode{\sphinxupquote{alt}}}{\emph{parent}, \emph{options}, \emph{\$editable}, \emph{media}}{}~\begin{quote}\begin{description}
\item[{Extends}] \leavevmode{\hyperref[\detokenize{reference/javascript_api:web_editor.widget.Dialog}]{\sphinxcrossref{
Dialog
}}}
\item[{Parameters}] \leavevmode\begin{itemize}

\sphinxstylestrong{parent}

\sphinxstylestrong{options}

\sphinxstylestrong{\$editable}

\sphinxstylestrong{media}
\end{itemize}

\end{description}\end{quote}

alt widget. Lets users change a alt \& title on a media

\end{fulllineitems}



\begin{fulllineitems}
\phantomsection\label{\detokenize{reference/javascript_api:MediaDialog}}\pysiglinewithargsret{\sphinxbfcode{\sphinxupquote{class }}\sphinxbfcode{\sphinxupquote{MediaDialog}}}{\emph{parent}, \emph{options}, \emph{\$editable}, \emph{media}}{}~\begin{quote}\begin{description}
\item[{Extends}] \leavevmode{\hyperref[\detokenize{reference/javascript_api:web_editor.widget.Dialog}]{\sphinxcrossref{
Dialog
}}}
\item[{Parameters}] \leavevmode\begin{itemize}

\sphinxstylestrong{parent}

\sphinxstylestrong{options}

\sphinxstylestrong{\$editable}

\sphinxstylestrong{media}
\end{itemize}

\end{description}\end{quote}

MediaDialog widget. Lets users change a media, including uploading a
new image, font awsome or video and can change a media into an other
media

options: select\_images: allow the selection of more of one image

\end{fulllineitems}



\begin{fulllineitems}
\phantomsection\label{\detokenize{reference/javascript_api:LinkDialog}}\pysiglinewithargsret{\sphinxbfcode{\sphinxupquote{class }}\sphinxbfcode{\sphinxupquote{LinkDialog}}}{\emph{parent}, \emph{options}, \emph{editable}, \emph{linkInfo}}{}~\begin{quote}\begin{description}
\item[{Extends}] \leavevmode{\hyperref[\detokenize{reference/javascript_api:web_editor.widget.Dialog}]{\sphinxcrossref{
Dialog
}}}
\item[{Parameters}] \leavevmode\begin{itemize}

\sphinxstylestrong{parent}

\sphinxstylestrong{options}

\sphinxstylestrong{editable}

\sphinxstylestrong{linkInfo}
\end{itemize}

\end{description}\end{quote}

The Link Dialog allows to customize link content and style.


\begin{fulllineitems}
\phantomsection\label{\detokenize{reference/javascript_api:bind_data}}\pysiglinewithargsret{\sphinxbfcode{\sphinxupquote{method }}\sphinxbfcode{\sphinxupquote{bind\_data}}}{}{}
Allows the URL input to propose existing website pages.

\end{fulllineitems}


\end{fulllineitems}



\begin{fulllineitems}
\phantomsection\label{\detokenize{reference/javascript_api:fontIcons}}\pysigline{\sphinxbfcode{\sphinxupquote{attribute }}\sphinxbfcode{\sphinxupquote{fontIcons}} Array}
list of font icons to load by editor. The icons are displayed in the media editor and
identified like font and image (can be colored, spinned, resized with fa classes).
To add font, push a new object \{base, parser\}
\begin{itemize}
\item {} 
base: class who appear on all fonts (eg: fa fa-refresh)

\item {} 
parser: regular expression used to select all font in css style sheets

\end{itemize}

\end{fulllineitems}



\begin{fulllineitems}
\phantomsection\label{\detokenize{reference/javascript_api:fontIconsDialog}}\pysiglinewithargsret{\sphinxbfcode{\sphinxupquote{class }}\sphinxbfcode{\sphinxupquote{fontIconsDialog}}}{\emph{parent}, \emph{media}}{}~\begin{quote}\begin{description}
\item[{Extends}] \leavevmode{\hyperref[\detokenize{reference/javascript_api:web.Widget.Widget}]{\sphinxcrossref{
Widget
}}}
\item[{Parameters}] \leavevmode\begin{itemize}

\sphinxstylestrong{parent}

\sphinxstylestrong{media}
\end{itemize}

\end{description}\end{quote}

FontIconsDialog widget. Lets users change a font awesome, support all
font awesome loaded in the css files.


\begin{fulllineitems}
\phantomsection\label{\detokenize{reference/javascript_api:save}}\pysiglinewithargsret{\sphinxbfcode{\sphinxupquote{method }}\sphinxbfcode{\sphinxupquote{save}}}{}{}
Removes existing FontAwesome classes on the bound element, and sets
all the new ones if necessary.

\end{fulllineitems}



\begin{fulllineitems}
\phantomsection\label{\detokenize{reference/javascript_api:getFont}}\pysiglinewithargsret{\sphinxbfcode{\sphinxupquote{method }}\sphinxbfcode{\sphinxupquote{getFont}}}{\emph{classNames}}{}
return the data font object (with base, parser and icons) or null
\begin{quote}\begin{description}
\item[{Parameters}] \leavevmode\begin{itemize}

\sphinxstylestrong{classNames}
\end{itemize}

\end{description}\end{quote}

\end{fulllineitems}



\begin{fulllineitems}
\phantomsection\label{\detokenize{reference/javascript_api:load_data}}\pysiglinewithargsret{\sphinxbfcode{\sphinxupquote{method }}\sphinxbfcode{\sphinxupquote{load\_data}}}{}{}
Looks up the various FontAwesome classes on the bound element and
sets the corresponding template/form elements to the right state.
If multiple classes of the same category are present on an element
(e.g. fa-lg and fa-3x) the last one occurring will be selected,
which may not match the visual look of the element.

\end{fulllineitems}



\begin{fulllineitems}
\phantomsection\label{\detokenize{reference/javascript_api:get_fa_classes}}\pysiglinewithargsret{\sphinxbfcode{\sphinxupquote{method }}\sphinxbfcode{\sphinxupquote{get\_fa\_classes}}}{}{}
Serializes the dialog to an array of FontAwesome classes. Includes the base \sphinxcode{\sphinxupquote{fa}}.

\end{fulllineitems}


\end{fulllineitems}



\begin{fulllineitems}
\phantomsection\label{\detokenize{reference/javascript_api:ImageDialog}}\pysiglinewithargsret{\sphinxbfcode{\sphinxupquote{class }}\sphinxbfcode{\sphinxupquote{ImageDialog}}}{\emph{parent}, \emph{media}, \emph{options}}{}~\begin{quote}\begin{description}
\item[{Extends}] \leavevmode{\hyperref[\detokenize{reference/javascript_api:web.Widget.Widget}]{\sphinxcrossref{
Widget
}}}
\item[{Parameters}] \leavevmode\begin{itemize}

\sphinxstylestrong{parent}

\sphinxstylestrong{media}

\sphinxstylestrong{options}
\end{itemize}

\end{description}\end{quote}

ImageDialog widget. Let users change an image, including uploading a
new image in odoo or selecting the image style (if supported by
the caller).

\end{fulllineitems}



\begin{fulllineitems}
\phantomsection\label{\detokenize{reference/javascript_api:VideoDialog}}\pysiglinewithargsret{\sphinxbfcode{\sphinxupquote{class }}\sphinxbfcode{\sphinxupquote{VideoDialog}}}{\emph{parent}, \emph{media}}{}~\begin{quote}\begin{description}
\item[{Extends}] \leavevmode{\hyperref[\detokenize{reference/javascript_api:web.Widget.Widget}]{\sphinxcrossref{
Widget
}}}
\item[{Parameters}] \leavevmode\begin{itemize}

\sphinxstylestrong{parent}

\sphinxstylestrong{media}
\end{itemize}

\end{description}\end{quote}

VideoDialog widget. Let users change a video, support all summernote
video, and embed iframe.

\end{fulllineitems}


\end{fulllineitems}



\begin{fulllineitems}
\phantomsection\label{\detokenize{reference/javascript_api:MediaDialog}}\pysiglinewithargsret{\sphinxbfcode{\sphinxupquote{class }}\sphinxbfcode{\sphinxupquote{MediaDialog}}}{\emph{parent}, \emph{options}, \emph{\$editable}, \emph{media}}{}~\begin{quote}\begin{description}
\item[{Extends}] \leavevmode{\hyperref[\detokenize{reference/javascript_api:web_editor.widget.Dialog}]{\sphinxcrossref{
Dialog
}}}
\item[{Parameters}] \leavevmode\begin{itemize}

\sphinxstylestrong{parent}

\sphinxstylestrong{options}

\sphinxstylestrong{\$editable}

\sphinxstylestrong{media}
\end{itemize}

\end{description}\end{quote}

MediaDialog widget. Lets users change a media, including uploading a
new image, font awsome or video and can change a media into an other
media

options: select\_images: allow the selection of more of one image

\end{fulllineitems}



\begin{fulllineitems}
\phantomsection\label{\detokenize{reference/javascript_api:alt}}\pysiglinewithargsret{\sphinxbfcode{\sphinxupquote{class }}\sphinxbfcode{\sphinxupquote{alt}}}{\emph{parent}, \emph{options}, \emph{\$editable}, \emph{media}}{}~\begin{quote}\begin{description}
\item[{Extends}] \leavevmode{\hyperref[\detokenize{reference/javascript_api:web_editor.widget.Dialog}]{\sphinxcrossref{
Dialog
}}}
\item[{Parameters}] \leavevmode\begin{itemize}

\sphinxstylestrong{parent}

\sphinxstylestrong{options}

\sphinxstylestrong{\$editable}

\sphinxstylestrong{media}
\end{itemize}

\end{description}\end{quote}

alt widget. Lets users change a alt \& title on a media

\end{fulllineitems}



\begin{fulllineitems}
\phantomsection\label{\detokenize{reference/javascript_api:VideoDialog}}\pysiglinewithargsret{\sphinxbfcode{\sphinxupquote{class }}\sphinxbfcode{\sphinxupquote{VideoDialog}}}{\emph{parent}, \emph{media}}{}~\begin{quote}\begin{description}
\item[{Extends}] \leavevmode{\hyperref[\detokenize{reference/javascript_api:web.Widget.Widget}]{\sphinxcrossref{
Widget
}}}
\item[{Parameters}] \leavevmode\begin{itemize}

\sphinxstylestrong{parent}

\sphinxstylestrong{media}
\end{itemize}

\end{description}\end{quote}

VideoDialog widget. Let users change a video, support all summernote
video, and embed iframe.

\end{fulllineitems}



\begin{fulllineitems}
\phantomsection\label{\detokenize{reference/javascript_api:fontIconsDialog}}\pysiglinewithargsret{\sphinxbfcode{\sphinxupquote{class }}\sphinxbfcode{\sphinxupquote{fontIconsDialog}}}{\emph{parent}, \emph{media}}{}~\begin{quote}\begin{description}
\item[{Extends}] \leavevmode{\hyperref[\detokenize{reference/javascript_api:web.Widget.Widget}]{\sphinxcrossref{
Widget
}}}
\item[{Parameters}] \leavevmode\begin{itemize}

\sphinxstylestrong{parent}

\sphinxstylestrong{media}
\end{itemize}

\end{description}\end{quote}

FontIconsDialog widget. Lets users change a font awesome, support all
font awesome loaded in the css files.


\begin{fulllineitems}
\phantomsection\label{\detokenize{reference/javascript_api:save}}\pysiglinewithargsret{\sphinxbfcode{\sphinxupquote{method }}\sphinxbfcode{\sphinxupquote{save}}}{}{}
Removes existing FontAwesome classes on the bound element, and sets
all the new ones if necessary.

\end{fulllineitems}



\begin{fulllineitems}
\phantomsection\label{\detokenize{reference/javascript_api:getFont}}\pysiglinewithargsret{\sphinxbfcode{\sphinxupquote{method }}\sphinxbfcode{\sphinxupquote{getFont}}}{\emph{classNames}}{}
return the data font object (with base, parser and icons) or null
\begin{quote}\begin{description}
\item[{Parameters}] \leavevmode\begin{itemize}

\sphinxstylestrong{classNames}
\end{itemize}

\end{description}\end{quote}

\end{fulllineitems}



\begin{fulllineitems}
\phantomsection\label{\detokenize{reference/javascript_api:load_data}}\pysiglinewithargsret{\sphinxbfcode{\sphinxupquote{method }}\sphinxbfcode{\sphinxupquote{load\_data}}}{}{}
Looks up the various FontAwesome classes on the bound element and
sets the corresponding template/form elements to the right state.
If multiple classes of the same category are present on an element
(e.g. fa-lg and fa-3x) the last one occurring will be selected,
which may not match the visual look of the element.

\end{fulllineitems}



\begin{fulllineitems}
\phantomsection\label{\detokenize{reference/javascript_api:get_fa_classes}}\pysiglinewithargsret{\sphinxbfcode{\sphinxupquote{method }}\sphinxbfcode{\sphinxupquote{get\_fa\_classes}}}{}{}
Serializes the dialog to an array of FontAwesome classes. Includes the base \sphinxcode{\sphinxupquote{fa}}.

\end{fulllineitems}


\end{fulllineitems}



\begin{fulllineitems}
\phantomsection\label{\detokenize{reference/javascript_api:LinkDialog}}\pysiglinewithargsret{\sphinxbfcode{\sphinxupquote{class }}\sphinxbfcode{\sphinxupquote{LinkDialog}}}{\emph{parent}, \emph{options}, \emph{editable}, \emph{linkInfo}}{}~\begin{quote}\begin{description}
\item[{Extends}] \leavevmode{\hyperref[\detokenize{reference/javascript_api:web_editor.widget.Dialog}]{\sphinxcrossref{
Dialog
}}}
\item[{Parameters}] \leavevmode\begin{itemize}

\sphinxstylestrong{parent}

\sphinxstylestrong{options}

\sphinxstylestrong{editable}

\sphinxstylestrong{linkInfo}
\end{itemize}

\end{description}\end{quote}

The Link Dialog allows to customize link content and style.


\begin{fulllineitems}
\phantomsection\label{\detokenize{reference/javascript_api:bind_data}}\pysiglinewithargsret{\sphinxbfcode{\sphinxupquote{method }}\sphinxbfcode{\sphinxupquote{bind\_data}}}{}{}
Allows the URL input to propose existing website pages.

\end{fulllineitems}


\end{fulllineitems}



\begin{fulllineitems}
\phantomsection\label{\detokenize{reference/javascript_api:fontIcons}}\pysigline{\sphinxbfcode{\sphinxupquote{attribute }}\sphinxbfcode{\sphinxupquote{fontIcons}} Array}
list of font icons to load by editor. The icons are displayed in the media editor and
identified like font and image (can be colored, spinned, resized with fa classes).
To add font, push a new object \{base, parser\}
\begin{itemize}
\item {} 
base: class who appear on all fonts (eg: fa fa-refresh)

\item {} 
parser: regular expression used to select all font in css style sheets

\end{itemize}

\end{fulllineitems}



\begin{fulllineitems}
\phantomsection\label{\detokenize{reference/javascript_api:Dialog}}\pysiglinewithargsret{\sphinxbfcode{\sphinxupquote{class }}\sphinxbfcode{\sphinxupquote{Dialog}}}{\emph{parent}, \emph{options}}{}~\begin{quote}\begin{description}
\item[{Extends}] \leavevmode{\hyperref[\detokenize{reference/javascript_api:web.Dialog.Dialog}]{\sphinxcrossref{
Dialog
}}}
\item[{Parameters}] \leavevmode\begin{itemize}

\sphinxstylestrong{parent}

\sphinxstylestrong{options}
\end{itemize}

\end{description}\end{quote}

Extend Dialog class to handle save/cancel of edition components.

\end{fulllineitems}



\begin{fulllineitems}
\phantomsection\label{\detokenize{reference/javascript_api:ImageDialog}}\pysiglinewithargsret{\sphinxbfcode{\sphinxupquote{class }}\sphinxbfcode{\sphinxupquote{ImageDialog}}}{\emph{parent}, \emph{media}, \emph{options}}{}~\begin{quote}\begin{description}
\item[{Extends}] \leavevmode{\hyperref[\detokenize{reference/javascript_api:web.Widget.Widget}]{\sphinxcrossref{
Widget
}}}
\item[{Parameters}] \leavevmode\begin{itemize}

\sphinxstylestrong{parent}

\sphinxstylestrong{media}

\sphinxstylestrong{options}
\end{itemize}

\end{description}\end{quote}

ImageDialog widget. Let users change an image, including uploading a
new image in odoo or selecting the image style (if supported by
the caller).

\end{fulllineitems}


\end{fulllineitems}

\phantomsection\label{\detokenize{reference/javascript_api:module-website_slides.slides}}

\begin{fulllineitems}
\phantomsection\label{\detokenize{reference/javascript_api:website_slides.slides}}\pysigline{\sphinxbfcode{\sphinxupquote{module }}\sphinxbfcode{\sphinxupquote{website\_slides.slides}}}~~\begin{quote}\begin{description}
\item[{Exports}] \leavevmode{\hyperref[\detokenize{reference/javascript_api:website_slides.slides.}]{\sphinxcrossref{
\textless{}anonymous\textgreater{}
}}}
\item[{Depends On}] \leavevmode\begin{itemize}
\item {} {\hyperref[\detokenize{reference/javascript_api:web.Widget}]{\sphinxcrossref{
web.Widget
}}}
\item {} {\hyperref[\detokenize{reference/javascript_api:web.ajax}]{\sphinxcrossref{
web.ajax
}}}
\item {} {\hyperref[\detokenize{reference/javascript_api:web.core}]{\sphinxcrossref{
web.core
}}}
\item {} {\hyperref[\detokenize{reference/javascript_api:web.local_storage}]{\sphinxcrossref{
web.local\_storage
}}}
\item {} {\hyperref[\detokenize{reference/javascript_api:web.time}]{\sphinxcrossref{
web.time
}}}
\item {} {\hyperref[\detokenize{reference/javascript_api:website.WebsiteRoot.instance}]{\sphinxcrossref{
website.WebsiteRoot.instance
}}}
\end{itemize}

\end{description}\end{quote}


\begin{fulllineitems}
\phantomsection\label{\detokenize{reference/javascript_api:website_slides.slides.}}\pysigline{\sphinxbfcode{\sphinxupquote{namespace }}\sphinxbfcode{\sphinxupquote{}}}
\end{fulllineitems}


\end{fulllineitems}

\phantomsection\label{\detokenize{reference/javascript_api:module-web_editor.rte.summernote}}

\begin{fulllineitems}
\phantomsection\label{\detokenize{reference/javascript_api:web_editor.rte.summernote}}\pysigline{\sphinxbfcode{\sphinxupquote{module }}\sphinxbfcode{\sphinxupquote{web\_editor.rte.summernote}}}~~\begin{quote}\begin{description}
\item[{Exports}] \leavevmode{\hyperref[\detokenize{reference/javascript_api:web_editor.rte.summernote.SummernoteManager}]{\sphinxcrossref{
SummernoteManager
}}}
\item[{Depends On}] \leavevmode\begin{itemize}
\item {} {\hyperref[\detokenize{reference/javascript_api:web.Class}]{\sphinxcrossref{
web.Class
}}}
\item {} {\hyperref[\detokenize{reference/javascript_api:web.ajax}]{\sphinxcrossref{
web.ajax
}}}
\item {} {\hyperref[\detokenize{reference/javascript_api:web.core}]{\sphinxcrossref{
web.core
}}}
\item {} {\hyperref[\detokenize{reference/javascript_api:web.mixins}]{\sphinxcrossref{
web.mixins
}}}
\item {} {\hyperref[\detokenize{reference/javascript_api:web_editor.context}]{\sphinxcrossref{
web\_editor.context
}}}
\item {} {\hyperref[\detokenize{reference/javascript_api:web_editor.rte}]{\sphinxcrossref{
web\_editor.rte
}}}
\item {} {\hyperref[\detokenize{reference/javascript_api:web_editor.widget}]{\sphinxcrossref{
web\_editor.widget
}}}
\end{itemize}

\end{description}\end{quote}


\begin{fulllineitems}
\phantomsection\label{\detokenize{reference/javascript_api:SummernoteManager}}\pysiglinewithargsret{\sphinxbfcode{\sphinxupquote{class }}\sphinxbfcode{\sphinxupquote{SummernoteManager}}}{\emph{parent}}{}~\begin{quote}\begin{description}
\item[{Extends}] \leavevmode{\hyperref[\detokenize{reference/javascript_api:web.Class.Class}]{\sphinxcrossref{
Class
}}}
\item[{Mixes}] \leavevmode\begin{itemize}
\item {} {\hyperref[\detokenize{reference/javascript_api:web.mixins.EventDispatcherMixin}]{\sphinxcrossref{
EventDispatcherMixin
}}}
\end{itemize}

\item[{Parameters}] \leavevmode\begin{itemize}

\sphinxstylestrong{parent}
\end{itemize}

\end{description}\end{quote}

\end{fulllineitems}


\end{fulllineitems}

\phantomsection\label{\detokenize{reference/javascript_api:module-web_diagram.DiagramRenderer}}

\begin{fulllineitems}
\phantomsection\label{\detokenize{reference/javascript_api:web_diagram.DiagramRenderer}}\pysigline{\sphinxbfcode{\sphinxupquote{module }}\sphinxbfcode{\sphinxupquote{web\_diagram.DiagramRenderer}}}~~\begin{quote}\begin{description}
\item[{Exports}] \leavevmode{\hyperref[\detokenize{reference/javascript_api:web_diagram.DiagramRenderer.DiagramRenderer}]{\sphinxcrossref{
DiagramRenderer
}}}
\item[{Depends On}] \leavevmode\begin{itemize}
\item {} {\hyperref[\detokenize{reference/javascript_api:web.AbstractRenderer}]{\sphinxcrossref{
web.AbstractRenderer
}}}
\end{itemize}

\end{description}\end{quote}


\begin{fulllineitems}
\phantomsection\label{\detokenize{reference/javascript_api:DiagramRenderer}}\pysiglinewithargsret{\sphinxbfcode{\sphinxupquote{class }}\sphinxbfcode{\sphinxupquote{DiagramRenderer}}}{}{}~\begin{quote}\begin{description}
\item[{Extends}] \leavevmode{\hyperref[\detokenize{reference/javascript_api:web.AbstractRenderer.AbstractRenderer}]{\sphinxcrossref{
AbstractRenderer
}}}
\end{description}\end{quote}

Diagram Renderer

The diagram renderer responsability is to render a diagram view, that is, a
set of (labelled) nodes and edges.  To do that, it uses the Raphael.js
library.

\end{fulllineitems}



\begin{fulllineitems}
\phantomsection\label{\detokenize{reference/javascript_api:DiagramRenderer}}\pysiglinewithargsret{\sphinxbfcode{\sphinxupquote{class }}\sphinxbfcode{\sphinxupquote{DiagramRenderer}}}{}{}~\begin{quote}\begin{description}
\item[{Extends}] \leavevmode{\hyperref[\detokenize{reference/javascript_api:web.AbstractRenderer.AbstractRenderer}]{\sphinxcrossref{
AbstractRenderer
}}}
\end{description}\end{quote}

Diagram Renderer

The diagram renderer responsability is to render a diagram view, that is, a
set of (labelled) nodes and edges.  To do that, it uses the Raphael.js
library.

\end{fulllineitems}


\end{fulllineitems}

\phantomsection\label{\detokenize{reference/javascript_api:module-point_of_sale.DB}}

\begin{fulllineitems}
\phantomsection\label{\detokenize{reference/javascript_api:point_of_sale.DB}}\pysigline{\sphinxbfcode{\sphinxupquote{module }}\sphinxbfcode{\sphinxupquote{point\_of\_sale.DB}}}~~\begin{quote}\begin{description}
\item[{Exports}] \leavevmode{\hyperref[\detokenize{reference/javascript_api:point_of_sale.DB.PosDB}]{\sphinxcrossref{
PosDB
}}}
\item[{Depends On}] \leavevmode\begin{itemize}
\item {} {\hyperref[\detokenize{reference/javascript_api:web.core}]{\sphinxcrossref{
web.core
}}}
\end{itemize}

\end{description}\end{quote}


\begin{fulllineitems}
\phantomsection\label{\detokenize{reference/javascript_api:PosDB}}\pysiglinewithargsret{\sphinxbfcode{\sphinxupquote{class }}\sphinxbfcode{\sphinxupquote{PosDB}}}{\emph{options}}{}~\begin{quote}\begin{description}
\item[{Extends}] \leavevmode{\hyperref[\detokenize{reference/javascript_api:web.Class.Class}]{\sphinxcrossref{
Class
}}}
\item[{Parameters}] \leavevmode\begin{itemize}

\sphinxstylestrong{options}
\end{itemize}

\end{description}\end{quote}

\end{fulllineitems}


\end{fulllineitems}

\phantomsection\label{\detokenize{reference/javascript_api:module-web.CalendarRenderer}}

\begin{fulllineitems}
\phantomsection\label{\detokenize{reference/javascript_api:web.CalendarRenderer}}\pysigline{\sphinxbfcode{\sphinxupquote{module }}\sphinxbfcode{\sphinxupquote{web.CalendarRenderer}}}~~\begin{quote}\begin{description}
\item[{Exports}] \leavevmode{\hyperref[\detokenize{reference/javascript_api:web.CalendarRenderer.}]{\sphinxcrossref{
\textless{}anonymous\textgreater{}
}}}
\item[{Depends On}] \leavevmode\begin{itemize}
\item {} {\hyperref[\detokenize{reference/javascript_api:web.AbstractRenderer}]{\sphinxcrossref{
web.AbstractRenderer
}}}
\item {} {\hyperref[\detokenize{reference/javascript_api:web.Dialog}]{\sphinxcrossref{
web.Dialog
}}}
\item {} {\hyperref[\detokenize{reference/javascript_api:web.FieldManagerMixin}]{\sphinxcrossref{
web.FieldManagerMixin
}}}
\item {} {\hyperref[\detokenize{reference/javascript_api:web.QWeb}]{\sphinxcrossref{
web.QWeb
}}}
\item {} {\hyperref[\detokenize{reference/javascript_api:web.Widget}]{\sphinxcrossref{
web.Widget
}}}
\item {} {\hyperref[\detokenize{reference/javascript_api:web.core}]{\sphinxcrossref{
web.core
}}}
\item {} {\hyperref[\detokenize{reference/javascript_api:web.field_utils}]{\sphinxcrossref{
web.field\_utils
}}}
\item {} {\hyperref[\detokenize{reference/javascript_api:web.relational_fields}]{\sphinxcrossref{
web.relational\_fields
}}}
\item {} {\hyperref[\detokenize{reference/javascript_api:web.session}]{\sphinxcrossref{
web.session
}}}
\item {} {\hyperref[\detokenize{reference/javascript_api:web.utils}]{\sphinxcrossref{
web.utils
}}}
\end{itemize}

\end{description}\end{quote}


\begin{fulllineitems}
\phantomsection\label{\detokenize{reference/javascript_api:web.CalendarRenderer.}}\pysiglinewithargsret{\sphinxbfcode{\sphinxupquote{class }}\sphinxbfcode{\sphinxupquote{}}}{\emph{parent}, \emph{state}, \emph{params}}{}~\begin{quote}\begin{description}
\item[{Extends}] \leavevmode{\hyperref[\detokenize{reference/javascript_api:web.AbstractRenderer.AbstractRenderer}]{\sphinxcrossref{
AbstractRenderer
}}}
\item[{Parameters}] \leavevmode\begin{itemize}

\sphinxstylestrong{parent} ({\hyperref[\detokenize{reference/javascript_api:Widget}]{\sphinxcrossref{\sphinxstyleliteralemphasis{\sphinxupquote{Widget}}}}})

\sphinxstylestrong{state} (\sphinxstyleliteralemphasis{\sphinxupquote{Object}})

\sphinxstylestrong{params} (\sphinxstyleliteralemphasis{\sphinxupquote{Object}})
\end{itemize}

\end{description}\end{quote}


\begin{fulllineitems}
\phantomsection\label{\detokenize{reference/javascript_api:getAvatars}}\pysiglinewithargsret{\sphinxbfcode{\sphinxupquote{function }}\sphinxbfcode{\sphinxupquote{getAvatars}}}{\emph{record}, \emph{fieldName}, \emph{imageField}}{{ $\rightarrow$ string{[}{]}}}
Note: this is not dead code, it is called by the calendar-box template
\begin{quote}\begin{description}
\item[{Parameters}] \leavevmode\begin{itemize}

\sphinxstylestrong{record} (\sphinxstyleliteralemphasis{\sphinxupquote{any}})

\sphinxstylestrong{fieldName} (\sphinxstyleliteralemphasis{\sphinxupquote{any}})

\sphinxstylestrong{imageField} (\sphinxstyleliteralemphasis{\sphinxupquote{any}})
\end{itemize}

\item[{Return Type}] \leavevmode
\sphinxstyleliteralemphasis{\sphinxupquote{Array}}\textless{}\sphinxstyleliteralemphasis{\sphinxupquote{string}}\textgreater{}

\end{description}\end{quote}

\end{fulllineitems}



\begin{fulllineitems}
\phantomsection\label{\detokenize{reference/javascript_api:getColor}}\pysiglinewithargsret{\sphinxbfcode{\sphinxupquote{function }}\sphinxbfcode{\sphinxupquote{getColor}}}{\emph{key}}{{ $\rightarrow$ integer}}
Note: this is not dead code, it is called by two template
\begin{quote}\begin{description}
\item[{Parameters}] \leavevmode\begin{itemize}

\sphinxstylestrong{key} (\sphinxstyleliteralemphasis{\sphinxupquote{any}})
\end{itemize}

\item[{Return Type}] \leavevmode
\sphinxstyleliteralemphasis{\sphinxupquote{integer}}

\end{description}\end{quote}

\end{fulllineitems}


\end{fulllineitems}


\end{fulllineitems}

\phantomsection\label{\detokenize{reference/javascript_api:module-point_of_sale.BaseWidget}}

\begin{fulllineitems}
\phantomsection\label{\detokenize{reference/javascript_api:point_of_sale.BaseWidget}}\pysigline{\sphinxbfcode{\sphinxupquote{module }}\sphinxbfcode{\sphinxupquote{point\_of\_sale.BaseWidget}}}~~\begin{quote}\begin{description}
\item[{Exports}] \leavevmode{\hyperref[\detokenize{reference/javascript_api:point_of_sale.BaseWidget.PosBaseWidget}]{\sphinxcrossref{
PosBaseWidget
}}}
\item[{Depends On}] \leavevmode\begin{itemize}
\item {} {\hyperref[\detokenize{reference/javascript_api:web.Widget}]{\sphinxcrossref{
web.Widget
}}}
\item {} {\hyperref[\detokenize{reference/javascript_api:web.field_utils}]{\sphinxcrossref{
web.field\_utils
}}}
\item {} {\hyperref[\detokenize{reference/javascript_api:web.utils}]{\sphinxcrossref{
web.utils
}}}
\end{itemize}

\end{description}\end{quote}


\begin{fulllineitems}
\phantomsection\label{\detokenize{reference/javascript_api:PosBaseWidget}}\pysiglinewithargsret{\sphinxbfcode{\sphinxupquote{class }}\sphinxbfcode{\sphinxupquote{PosBaseWidget}}}{\emph{parent}, \emph{options}}{}~\begin{quote}\begin{description}
\item[{Extends}] \leavevmode{\hyperref[\detokenize{reference/javascript_api:web.Widget.Widget}]{\sphinxcrossref{
Widget
}}}
\item[{Parameters}] \leavevmode\begin{itemize}

\sphinxstylestrong{parent}

\sphinxstylestrong{options}
\end{itemize}

\end{description}\end{quote}

\end{fulllineitems}


\end{fulllineitems}

\phantomsection\label{\detokenize{reference/javascript_api:module-web.PivotController}}

\begin{fulllineitems}
\phantomsection\label{\detokenize{reference/javascript_api:web.PivotController}}\pysigline{\sphinxbfcode{\sphinxupquote{module }}\sphinxbfcode{\sphinxupquote{web.PivotController}}}~~\begin{quote}\begin{description}
\item[{Exports}] \leavevmode{\hyperref[\detokenize{reference/javascript_api:web.PivotController.PivotController}]{\sphinxcrossref{
PivotController
}}}
\item[{Depends On}] \leavevmode\begin{itemize}
\item {} {\hyperref[\detokenize{reference/javascript_api:web.AbstractController}]{\sphinxcrossref{
web.AbstractController
}}}
\item {} {\hyperref[\detokenize{reference/javascript_api:web.core}]{\sphinxcrossref{
web.core
}}}
\item {} {\hyperref[\detokenize{reference/javascript_api:web.crash_manager}]{\sphinxcrossref{
web.crash\_manager
}}}
\item {} {\hyperref[\detokenize{reference/javascript_api:web.framework}]{\sphinxcrossref{
web.framework
}}}
\item {} {\hyperref[\detokenize{reference/javascript_api:web.session}]{\sphinxcrossref{
web.session
}}}
\end{itemize}

\end{description}\end{quote}


\begin{fulllineitems}
\phantomsection\label{\detokenize{reference/javascript_api:PivotController}}\pysiglinewithargsret{\sphinxbfcode{\sphinxupquote{class }}\sphinxbfcode{\sphinxupquote{PivotController}}}{\emph{parent}, \emph{model}, \emph{renderer}, \emph{params}}{}~\begin{quote}\begin{description}
\item[{Extends}] \leavevmode{\hyperref[\detokenize{reference/javascript_api:web.AbstractController.AbstractController}]{\sphinxcrossref{
AbstractController
}}}
\item[{Parameters}] \leavevmode\begin{itemize}

\sphinxstylestrong{parent}

\sphinxstylestrong{model}

\sphinxstylestrong{renderer}

\sphinxstylestrong{params} ({\hyperref[\detokenize{reference/javascript_api:web.PivotController.PivotControllerParams}]{\sphinxcrossref{\sphinxstyleliteralemphasis{\sphinxupquote{PivotControllerParams}}}}})
\end{itemize}

\end{description}\end{quote}


\begin{fulllineitems}
\phantomsection\label{\detokenize{reference/javascript_api:getContext}}\pysiglinewithargsret{\sphinxbfcode{\sphinxupquote{method }}\sphinxbfcode{\sphinxupquote{getContext}}}{}{{ $\rightarrow$ Object}}
Returns the current measures and groupbys, so we can restore the view
when we save the current state in the search view, or when we add it to
the dashboard.
\begin{quote}\begin{description}
\item[{Return Type}] \leavevmode
\sphinxstyleliteralemphasis{\sphinxupquote{Object}}

\end{description}\end{quote}

\end{fulllineitems}



\begin{fulllineitems}
\phantomsection\label{\detokenize{reference/javascript_api:renderButtons}}\pysiglinewithargsret{\sphinxbfcode{\sphinxupquote{method }}\sphinxbfcode{\sphinxupquote{renderButtons}}}{\sphinxoptional{\emph{\$node}}}{}
Render the buttons according to the PivotView.buttons template and
add listeners on it.
Set this.\$buttons with the produced jQuery element
\begin{quote}\begin{description}
\item[{Parameters}] \leavevmode\begin{itemize}

\sphinxstylestrong{\$node} (\sphinxstyleliteralemphasis{\sphinxupquote{jQuery}}) \textendash{} a jQuery node where the rendered buttons should
  be inserted. \$node may be undefined, in which case the PivotView
  does nothing
\end{itemize}

\end{description}\end{quote}

\end{fulllineitems}



\begin{fulllineitems}
\phantomsection\label{\detokenize{reference/javascript_api:PivotControllerParams}}\pysiglinewithargsret{\sphinxbfcode{\sphinxupquote{class }}\sphinxbfcode{\sphinxupquote{PivotControllerParams}}}{}{}~

\begin{fulllineitems}
\phantomsection\label{\detokenize{reference/javascript_api:groupableFields}}\pysigline{\sphinxbfcode{\sphinxupquote{attribute }}\sphinxbfcode{\sphinxupquote{groupableFields}} Object}~\begin{description}
\item[{a map from field name to field}] \leavevmode
props

\end{description}

\end{fulllineitems}



\begin{fulllineitems}
\phantomsection\label{\detokenize{reference/javascript_api:enableLinking}}\pysigline{\sphinxbfcode{\sphinxupquote{attribute }}\sphinxbfcode{\sphinxupquote{enableLinking}} boolean}~\begin{description}
\item[{configure the pivot view to allow}] \leavevmode
opening a list view by clicking on a cell with some data.

\end{description}

\end{fulllineitems}


\end{fulllineitems}


\end{fulllineitems}


\end{fulllineitems}

\phantomsection\label{\detokenize{reference/javascript_api:module-web.field_registry}}

\begin{fulllineitems}
\phantomsection\label{\detokenize{reference/javascript_api:web.field_registry}}\pysigline{\sphinxbfcode{\sphinxupquote{module }}\sphinxbfcode{\sphinxupquote{web.field\_registry}}}~~\begin{quote}\begin{description}
\item[{Exports}] \leavevmode{\hyperref[\detokenize{reference/javascript_api:web.field_registry.}]{\sphinxcrossref{
\textless{}anonymous\textgreater{}
}}}
\item[{Depends On}] \leavevmode\begin{itemize}
\item {} {\hyperref[\detokenize{reference/javascript_api:web.Registry}]{\sphinxcrossref{
web.Registry
}}}
\end{itemize}

\end{description}\end{quote}


\begin{fulllineitems}
\phantomsection\label{\detokenize{reference/javascript_api:web.field_registry.}}\pysigline{\sphinxbfcode{\sphinxupquote{object }}\sphinxbfcode{\sphinxupquote{}}\sphinxbfcode{\sphinxupquote{ instance of }}{\hyperref[\detokenize{reference/javascript_api:web.Registry.Registry}]{\sphinxcrossref{Registry}}}}
\end{fulllineitems}


\end{fulllineitems}

\phantomsection\label{\detokenize{reference/javascript_api:module-web.dom}}

\begin{fulllineitems}
\phantomsection\label{\detokenize{reference/javascript_api:web.dom}}\pysigline{\sphinxbfcode{\sphinxupquote{module }}\sphinxbfcode{\sphinxupquote{web.dom}}}~~\begin{quote}\begin{description}
\item[{Exports}] \leavevmode{\hyperref[\detokenize{reference/javascript_api:web.dom.}]{\sphinxcrossref{
\textless{}anonymous\textgreater{}
}}}
\item[{Depends On}] \leavevmode\begin{itemize}
\item {} {\hyperref[\detokenize{reference/javascript_api:web.core}]{\sphinxcrossref{
web.core
}}}
\end{itemize}

\end{description}\end{quote}


\begin{fulllineitems}
\phantomsection\label{\detokenize{reference/javascript_api:_notify}}\pysiglinewithargsret{\sphinxbfcode{\sphinxupquote{function }}\sphinxbfcode{\sphinxupquote{\_notify}}}{\sphinxoptional{\emph{content}}, \emph{callbacks}}{}
Private function to notify that something has been attached in the DOM
\begin{quote}\begin{description}
\item[{Parameters}] \leavevmode\begin{itemize}

\sphinxstylestrong{content} (\sphinxstyleliteralemphasis{\sphinxupquote{htmlString}}\sphinxstyleemphasis{ or }\sphinxstyleliteralemphasis{\sphinxupquote{Element}}\sphinxstyleemphasis{ or }\sphinxstyleliteralemphasis{\sphinxupquote{Array}}\sphinxstyleemphasis{ or }\sphinxstyleliteralemphasis{\sphinxupquote{jQuery}}) \textendash{} the content that
has been attached in the DOM

\sphinxstylestrong{callbacks}
\end{itemize}

\end{description}\end{quote}

\end{fulllineitems}



\begin{fulllineitems}
\phantomsection\label{\detokenize{reference/javascript_api:web.dom.}}\pysigline{\sphinxbfcode{\sphinxupquote{namespace }}\sphinxbfcode{\sphinxupquote{}}}~

\begin{fulllineitems}
\phantomsection\label{\detokenize{reference/javascript_api:append}}\pysiglinewithargsret{\sphinxbfcode{\sphinxupquote{function }}\sphinxbfcode{\sphinxupquote{append}}}{\sphinxoptional{\emph{\$target}}\sphinxoptional{, \emph{content}}, \emph{options}}{}
Appends content in a jQuery object and optionnally triggers an event
\begin{quote}\begin{description}
\item[{Parameters}] \leavevmode\begin{itemize}

\sphinxstylestrong{\$target} (\sphinxstyleliteralemphasis{\sphinxupquote{jQuery}}) \textendash{} the node where content will be appended

\sphinxstylestrong{content} (\sphinxstyleliteralemphasis{\sphinxupquote{htmlString}}\sphinxstyleemphasis{ or }\sphinxstyleliteralemphasis{\sphinxupquote{Element}}\sphinxstyleemphasis{ or }\sphinxstyleliteralemphasis{\sphinxupquote{Array}}\sphinxstyleemphasis{ or }\sphinxstyleliteralemphasis{\sphinxupquote{jQuery}}) \textendash{} DOM element,
  array of elements, HTML string or jQuery object to append to \$target

\sphinxstylestrong{options} ({\hyperref[\detokenize{reference/javascript_api:web.dom.AppendOptions}]{\sphinxcrossref{\sphinxstyleliteralemphasis{\sphinxupquote{AppendOptions}}}}})
\end{itemize}

\end{description}\end{quote}


\begin{fulllineitems}
\phantomsection\label{\detokenize{reference/javascript_api:AppendOptions}}\pysiglinewithargsret{\sphinxbfcode{\sphinxupquote{class }}\sphinxbfcode{\sphinxupquote{AppendOptions}}}{}{}~

\begin{fulllineitems}
\phantomsection\label{\detokenize{reference/javascript_api:in_DOM}}\pysigline{\sphinxbfcode{\sphinxupquote{attribute }}\sphinxbfcode{\sphinxupquote{in\_DOM}} Boolean}
true if \$target is in the DOM

\end{fulllineitems}



\begin{fulllineitems}
\phantomsection\label{\detokenize{reference/javascript_api:callbacks}}\pysigline{\sphinxbfcode{\sphinxupquote{attribute }}\sphinxbfcode{\sphinxupquote{callbacks}} Array}~\begin{description}
\item[{array of objects describing the}] \leavevmode
callbacks to perform (see \_notify for a complete description)

\end{description}

\end{fulllineitems}


\end{fulllineitems}


\end{fulllineitems}



\begin{fulllineitems}
\phantomsection\label{\detokenize{reference/javascript_api:autoresize}}\pysiglinewithargsret{\sphinxbfcode{\sphinxupquote{function }}\sphinxbfcode{\sphinxupquote{autoresize}}}{\emph{\$textarea}, \emph{options}}{}
Autoresize a \$textarea node, by recomputing its height when necessary
\begin{quote}\begin{description}
\item[{Parameters}] \leavevmode\begin{itemize}

\sphinxstylestrong{\$textarea}

\sphinxstylestrong{options} ({\hyperref[\detokenize{reference/javascript_api:web.dom.AutoresizeOptions}]{\sphinxcrossref{\sphinxstyleliteralemphasis{\sphinxupquote{AutoresizeOptions}}}}})
\end{itemize}

\end{description}\end{quote}


\begin{fulllineitems}
\phantomsection\label{\detokenize{reference/javascript_api:AutoresizeOptions}}\pysiglinewithargsret{\sphinxbfcode{\sphinxupquote{class }}\sphinxbfcode{\sphinxupquote{AutoresizeOptions}}}{}{}~

\begin{fulllineitems}
\phantomsection\label{\detokenize{reference/javascript_api:min_height}}\pysigline{\sphinxbfcode{\sphinxupquote{attribute }}\sphinxbfcode{\sphinxupquote{min\_height}} number}
by default, 50.

\end{fulllineitems}



\begin{fulllineitems}
\phantomsection\label{\detokenize{reference/javascript_api:parent}}\pysigline{\sphinxbfcode{\sphinxupquote{attribute }}\sphinxbfcode{\sphinxupquote{parent}} {\hyperref[\detokenize{reference/javascript_api:Widget}]{\sphinxcrossref{Widget}}}}~\begin{description}
\item[{if set, autoresize will listen to some}] \leavevmode
extra events to decide when to resize itself.  This is useful for
widgets that are not in the dom when the autoresize is declared.

\end{description}

\end{fulllineitems}


\end{fulllineitems}


\end{fulllineitems}



\begin{fulllineitems}
\phantomsection\label{\detokenize{reference/javascript_api:cssFind}}\pysiglinewithargsret{\sphinxbfcode{\sphinxupquote{function }}\sphinxbfcode{\sphinxupquote{cssFind}}}{\emph{\$from}, \emph{selector}}{{ $\rightarrow$ jQuery}}
jQuery find function behavior is:

\fvset{hllines={, ,}}%
\begin{sphinxVerbatim}[commandchars=\\\{\}]
\PYG{n+nx}{\PYGZdl{}}\PYG{p}{(}\PYG{l+s+s1}{\PYGZsq{}A\PYGZsq{}}\PYG{p}{)}\PYG{p}{.}\PYG{n+nx}{find}\PYG{p}{(}\PYG{l+s+s1}{\PYGZsq{}A B\PYGZsq{}}\PYG{p}{)} \PYG{o}{\PYGZlt{}=}\PYG{o}{\PYGZgt{}} \PYG{n+nx}{\PYGZdl{}}\PYG{p}{(}\PYG{l+s+s1}{\PYGZsq{}A A B\PYGZsq{}}\PYG{p}{)}
\end{sphinxVerbatim}

The searches behavior to find options’ DOM needs to be:

\fvset{hllines={, ,}}%
\begin{sphinxVerbatim}[commandchars=\\\{\}]
\PYG{n+nx}{\PYGZdl{}}\PYG{p}{(}\PYG{l+s+s1}{\PYGZsq{}A\PYGZsq{}}\PYG{p}{)}\PYG{p}{.}\PYG{n+nx}{find}\PYG{p}{(}\PYG{l+s+s1}{\PYGZsq{}A B\PYGZsq{}}\PYG{p}{)} \PYG{o}{\PYGZlt{}=}\PYG{o}{\PYGZgt{}} \PYG{n+nx}{\PYGZdl{}}\PYG{p}{(}\PYG{l+s+s1}{\PYGZsq{}A B\PYGZsq{}}\PYG{p}{)}
\end{sphinxVerbatim}

This is what this function does.
\begin{quote}\begin{description}
\item[{Parameters}] \leavevmode\begin{itemize}

\sphinxstylestrong{\$from} (\sphinxstyleliteralemphasis{\sphinxupquote{jQuery}}) \textendash{} the jQuery element(s) from which to search

\sphinxstylestrong{selector} (\sphinxstyleliteralemphasis{\sphinxupquote{string}}) \textendash{} the CSS selector to match
\end{itemize}

\item[{Return Type}] \leavevmode
\sphinxstyleliteralemphasis{\sphinxupquote{jQuery}}

\end{description}\end{quote}

\end{fulllineitems}



\begin{fulllineitems}
\phantomsection\label{\detokenize{reference/javascript_api:detach}}\pysiglinewithargsret{\sphinxbfcode{\sphinxupquote{function }}\sphinxbfcode{\sphinxupquote{detach}}}{\sphinxoptional{\emph{to\_detach}}, \emph{options}}{{ $\rightarrow$ jQuery}}
Detaches widgets from the DOM and performs their on\_detach\_callback()
\begin{quote}\begin{description}
\item[{Parameters}] \leavevmode\begin{itemize}

\sphinxstylestrong{to\_detach} (\sphinxstyleliteralemphasis{\sphinxupquote{Array}}) \textendash{} array of \{widget: w, callback\_args: args\} such
  that w.\$el will be detached and w.on\_detach\_callback(args) will be
  called

\sphinxstylestrong{options} ({\hyperref[\detokenize{reference/javascript_api:web.dom.DetachOptions}]{\sphinxcrossref{\sphinxstyleliteralemphasis{\sphinxupquote{DetachOptions}}}}})
\end{itemize}

\item[{Returns}] \leavevmode
the detached elements

\item[{Return Type}] \leavevmode
\sphinxstyleliteralemphasis{\sphinxupquote{jQuery}}

\end{description}\end{quote}


\begin{fulllineitems}
\phantomsection\label{\detokenize{reference/javascript_api:DetachOptions}}\pysiglinewithargsret{\sphinxbfcode{\sphinxupquote{class }}\sphinxbfcode{\sphinxupquote{DetachOptions}}}{}{}~

\begin{fulllineitems}
\phantomsection\label{\detokenize{reference/javascript_api:_to_detach}}\pysigline{\sphinxbfcode{\sphinxupquote{attribute }}\sphinxbfcode{\sphinxupquote{\$to\_detach}} jQuery}~\begin{description}
\item[{if given, detached instead of}] \leavevmode
widgets’ \$el

\end{description}

\end{fulllineitems}


\end{fulllineitems}


\end{fulllineitems}



\begin{fulllineitems}
\phantomsection\label{\detokenize{reference/javascript_api:getSelectionRange}}\pysiglinewithargsret{\sphinxbfcode{\sphinxupquote{function }}\sphinxbfcode{\sphinxupquote{getSelectionRange}}}{\emph{node}}{{ $\rightarrow$ Object}}
Returns the selection range of an input or textarea
\begin{quote}\begin{description}
\item[{Parameters}] \leavevmode\begin{itemize}

\sphinxstylestrong{node} (\sphinxstyleliteralemphasis{\sphinxupquote{Object}}) \textendash{} DOM item input or texteara
\end{itemize}

\item[{Returns}] \leavevmode
range

\item[{Return Type}] \leavevmode
\sphinxstyleliteralemphasis{\sphinxupquote{Object}}

\end{description}\end{quote}

\end{fulllineitems}



\begin{fulllineitems}
\phantomsection\label{\detokenize{reference/javascript_api:getPosition}}\pysiglinewithargsret{\sphinxbfcode{\sphinxupquote{function }}\sphinxbfcode{\sphinxupquote{getPosition}}}{\emph{e}}{{ $\rightarrow$ Object}}
Returns the distance between a DOM element and the top-left corner of the
window
\begin{quote}\begin{description}
\item[{Parameters}] \leavevmode\begin{itemize}

\sphinxstylestrong{e} (\sphinxstyleliteralemphasis{\sphinxupquote{Object}}) \textendash{} DOM element (input or texteara)
\end{itemize}

\item[{Returns}] \leavevmode
the left and top distances in pixels

\item[{Return Type}] \leavevmode
\sphinxstyleliteralemphasis{\sphinxupquote{Object}}

\end{description}\end{quote}

\end{fulllineitems}



\begin{fulllineitems}
\phantomsection\label{\detokenize{reference/javascript_api:prepend}}\pysiglinewithargsret{\sphinxbfcode{\sphinxupquote{function }}\sphinxbfcode{\sphinxupquote{prepend}}}{\sphinxoptional{\emph{\$target}}\sphinxoptional{, \emph{content}}, \emph{options}}{}
Prepends content in a jQuery object and optionnally triggers an event
\begin{quote}\begin{description}
\item[{Parameters}] \leavevmode\begin{itemize}

\sphinxstylestrong{\$target} (\sphinxstyleliteralemphasis{\sphinxupquote{jQuery}}) \textendash{} the node where content will be prepended

\sphinxstylestrong{content} (\sphinxstyleliteralemphasis{\sphinxupquote{htmlString}}\sphinxstyleemphasis{ or }\sphinxstyleliteralemphasis{\sphinxupquote{Element}}\sphinxstyleemphasis{ or }\sphinxstyleliteralemphasis{\sphinxupquote{Array}}\sphinxstyleemphasis{ or }\sphinxstyleliteralemphasis{\sphinxupquote{jQuery}}) \textendash{} DOM element,
  array of elements, HTML string or jQuery object to prepend to \$target

\sphinxstylestrong{options} ({\hyperref[\detokenize{reference/javascript_api:web.dom.PrependOptions}]{\sphinxcrossref{\sphinxstyleliteralemphasis{\sphinxupquote{PrependOptions}}}}})
\end{itemize}

\end{description}\end{quote}


\begin{fulllineitems}
\phantomsection\label{\detokenize{reference/javascript_api:PrependOptions}}\pysiglinewithargsret{\sphinxbfcode{\sphinxupquote{class }}\sphinxbfcode{\sphinxupquote{PrependOptions}}}{}{}~

\begin{fulllineitems}
\phantomsection\label{\detokenize{reference/javascript_api:in_DOM}}\pysigline{\sphinxbfcode{\sphinxupquote{attribute }}\sphinxbfcode{\sphinxupquote{in\_DOM}} Boolean}
true if \$target is in the DOM

\end{fulllineitems}



\begin{fulllineitems}
\phantomsection\label{\detokenize{reference/javascript_api:callbacks}}\pysigline{\sphinxbfcode{\sphinxupquote{attribute }}\sphinxbfcode{\sphinxupquote{callbacks}} Array}~\begin{description}
\item[{array of objects describing the}] \leavevmode
callbacks to perform (see \_notify for a complete description)

\end{description}

\end{fulllineitems}


\end{fulllineitems}


\end{fulllineitems}



\begin{fulllineitems}
\phantomsection\label{\detokenize{reference/javascript_api:renderButton}}\pysiglinewithargsret{\sphinxbfcode{\sphinxupquote{function }}\sphinxbfcode{\sphinxupquote{renderButton}}}{\emph{options}}{{ $\rightarrow$ jQuery}}
Renders a button with standard odoo template. This does not use any xml
template to avoid forcing the frontend part to lazy load a xml file for
each widget which might want to create a simple button.
\begin{quote}\begin{description}
\item[{Parameters}] \leavevmode\begin{itemize}

\sphinxstylestrong{options} ({\hyperref[\detokenize{reference/javascript_api:web.dom.RenderButtonOptions}]{\sphinxcrossref{\sphinxstyleliteralemphasis{\sphinxupquote{RenderButtonOptions}}}}})
\end{itemize}

\item[{Return Type}] \leavevmode
\sphinxstyleliteralemphasis{\sphinxupquote{jQuery}}

\end{description}\end{quote}


\begin{fulllineitems}
\phantomsection\label{\detokenize{reference/javascript_api:RenderButtonOptions}}\pysiglinewithargsret{\sphinxbfcode{\sphinxupquote{class }}\sphinxbfcode{\sphinxupquote{RenderButtonOptions}}}{}{}~

\begin{fulllineitems}
\phantomsection\label{\detokenize{reference/javascript_api:attrs}}\pysigline{\sphinxbfcode{\sphinxupquote{attribute }}\sphinxbfcode{\sphinxupquote{attrs}} Object}
Attributes to put on the button element

\end{fulllineitems}



\begin{fulllineitems}
\phantomsection\label{\detokenize{reference/javascript_api:attrs.type}}\pysigline{\sphinxbfcode{\sphinxupquote{attribute }}\sphinxbfcode{\sphinxupquote{attrs.type}} string}
\end{fulllineitems}



\begin{fulllineitems}
\phantomsection\label{\detokenize{reference/javascript_api:attrs.class}}\pysigline{\sphinxbfcode{\sphinxupquote{attribute }}\sphinxbfcode{\sphinxupquote{attrs.class}} string}~\begin{description}
\item[{Note: automatically completed with “btn btn-X” (@see options.size}] \leavevmode
for the value of X)

\end{description}

\end{fulllineitems}



\begin{fulllineitems}
\phantomsection\label{\detokenize{reference/javascript_api:size}}\pysigline{\sphinxbfcode{\sphinxupquote{attribute }}\sphinxbfcode{\sphinxupquote{size}} string}
@see options.attrs.class

\end{fulllineitems}



\begin{fulllineitems}
\phantomsection\label{\detokenize{reference/javascript_api:icon}}\pysigline{\sphinxbfcode{\sphinxupquote{attribute }}\sphinxbfcode{\sphinxupquote{icon}} string}~\begin{description}
\item[{The specific fa icon class (for example “fa-home”) or an URL for}] \leavevmode
an image to use as icon.

\end{description}

\end{fulllineitems}



\begin{fulllineitems}
\phantomsection\label{\detokenize{reference/javascript_api:text}}\pysigline{\sphinxbfcode{\sphinxupquote{attribute }}\sphinxbfcode{\sphinxupquote{text}} string}
the button’s text

\end{fulllineitems}


\end{fulllineitems}


\end{fulllineitems}



\begin{fulllineitems}
\phantomsection\label{\detokenize{reference/javascript_api:renderCheckbox}}\pysiglinewithargsret{\sphinxbfcode{\sphinxupquote{function }}\sphinxbfcode{\sphinxupquote{renderCheckbox}}}{\sphinxoptional{\emph{options}}}{{ $\rightarrow$ jQuery}}
Renders a checkbox with standard odoo template. This does not use any xml
template to avoid forcing the frontend part to lazy load a xml file for
each widget which might want to create a simple checkbox.
\begin{quote}\begin{description}
\item[{Parameters}] \leavevmode\begin{itemize}

\sphinxstylestrong{options} ({\hyperref[\detokenize{reference/javascript_api:web.dom.RenderCheckboxOptions}]{\sphinxcrossref{\sphinxstyleliteralemphasis{\sphinxupquote{RenderCheckboxOptions}}}}})
\end{itemize}

\item[{Return Type}] \leavevmode
\sphinxstyleliteralemphasis{\sphinxupquote{jQuery}}

\end{description}\end{quote}


\begin{fulllineitems}
\phantomsection\label{\detokenize{reference/javascript_api:RenderCheckboxOptions}}\pysiglinewithargsret{\sphinxbfcode{\sphinxupquote{class }}\sphinxbfcode{\sphinxupquote{RenderCheckboxOptions}}}{}{}~

\begin{fulllineitems}
\phantomsection\label{\detokenize{reference/javascript_api:prop}}\pysigline{\sphinxbfcode{\sphinxupquote{attribute }}\sphinxbfcode{\sphinxupquote{prop}} Object}
Allows to set the input properties (disabled and checked states).

\end{fulllineitems}



\begin{fulllineitems}
\phantomsection\label{\detokenize{reference/javascript_api:text}}\pysigline{\sphinxbfcode{\sphinxupquote{attribute }}\sphinxbfcode{\sphinxupquote{text}} string}~\begin{description}
\item[{The checkbox’s associated text. If none is given then a simple}] \leavevmode
checkbox without label structure is rendered.

\end{description}

\end{fulllineitems}


\end{fulllineitems}


\end{fulllineitems}



\begin{fulllineitems}
\phantomsection\label{\detokenize{reference/javascript_api:setSelectionRange}}\pysiglinewithargsret{\sphinxbfcode{\sphinxupquote{function }}\sphinxbfcode{\sphinxupquote{setSelectionRange}}}{\emph{node}, \emph{range}}{}
Sets the selection range of a given input or textarea
\begin{quote}\begin{description}
\item[{Parameters}] \leavevmode\begin{itemize}

\sphinxstylestrong{node} (\sphinxstyleliteralemphasis{\sphinxupquote{Object}}) \textendash{} DOM element (input or textarea)

\sphinxstylestrong{range} ({\hyperref[\detokenize{reference/javascript_api:web.dom.SetSelectionRangeRange}]{\sphinxcrossref{\sphinxstyleliteralemphasis{\sphinxupquote{SetSelectionRangeRange}}}}})
\end{itemize}

\end{description}\end{quote}


\begin{fulllineitems}
\phantomsection\label{\detokenize{reference/javascript_api:SetSelectionRangeRange}}\pysiglinewithargsret{\sphinxbfcode{\sphinxupquote{class }}\sphinxbfcode{\sphinxupquote{SetSelectionRangeRange}}}{}{}~

\begin{fulllineitems}
\phantomsection\label{\detokenize{reference/javascript_api:start}}\pysigline{\sphinxbfcode{\sphinxupquote{attribute }}\sphinxbfcode{\sphinxupquote{start}} integer}
\end{fulllineitems}



\begin{fulllineitems}
\phantomsection\label{\detokenize{reference/javascript_api:end}}\pysigline{\sphinxbfcode{\sphinxupquote{attribute }}\sphinxbfcode{\sphinxupquote{end}} integer}
\end{fulllineitems}


\end{fulllineitems}


\end{fulllineitems}


\end{fulllineitems}


\end{fulllineitems}

\phantomsection\label{\detokenize{reference/javascript_api:module-web.datepicker}}

\begin{fulllineitems}
\phantomsection\label{\detokenize{reference/javascript_api:web.datepicker}}\pysigline{\sphinxbfcode{\sphinxupquote{module }}\sphinxbfcode{\sphinxupquote{web.datepicker}}}~~\begin{quote}\begin{description}
\item[{Exports}] \leavevmode{\hyperref[\detokenize{reference/javascript_api:web.datepicker.}]{\sphinxcrossref{
\textless{}anonymous\textgreater{}
}}}
\item[{Depends On}] \leavevmode\begin{itemize}
\item {} {\hyperref[\detokenize{reference/javascript_api:web.Widget}]{\sphinxcrossref{
web.Widget
}}}
\item {} {\hyperref[\detokenize{reference/javascript_api:web.core}]{\sphinxcrossref{
web.core
}}}
\item {} {\hyperref[\detokenize{reference/javascript_api:web.field_utils}]{\sphinxcrossref{
web.field\_utils
}}}
\item {} {\hyperref[\detokenize{reference/javascript_api:web.time}]{\sphinxcrossref{
web.time
}}}
\end{itemize}

\end{description}\end{quote}


\begin{fulllineitems}
\phantomsection\label{\detokenize{reference/javascript_api:web.datepicker.}}\pysigline{\sphinxbfcode{\sphinxupquote{namespace }}\sphinxbfcode{\sphinxupquote{}}}
\end{fulllineitems}


\end{fulllineitems}

\phantomsection\label{\detokenize{reference/javascript_api:module-web.ModelFieldSelector}}

\begin{fulllineitems}
\phantomsection\label{\detokenize{reference/javascript_api:web.ModelFieldSelector}}\pysigline{\sphinxbfcode{\sphinxupquote{module }}\sphinxbfcode{\sphinxupquote{web.ModelFieldSelector}}}~~\begin{quote}\begin{description}
\item[{Exports}] \leavevmode{\hyperref[\detokenize{reference/javascript_api:web.ModelFieldSelector.ModelFieldSelector}]{\sphinxcrossref{
ModelFieldSelector
}}}
\item[{Depends On}] \leavevmode\begin{itemize}
\item {} {\hyperref[\detokenize{reference/javascript_api:web.Widget}]{\sphinxcrossref{
web.Widget
}}}
\item {} {\hyperref[\detokenize{reference/javascript_api:web.core}]{\sphinxcrossref{
web.core
}}}
\end{itemize}

\end{description}\end{quote}


\begin{fulllineitems}
\phantomsection\label{\detokenize{reference/javascript_api:ModelFieldSelector}}\pysiglinewithargsret{\sphinxbfcode{\sphinxupquote{class }}\sphinxbfcode{\sphinxupquote{ModelFieldSelector}}}{\emph{parent}, \emph{model}, \emph{chain}\sphinxoptional{, \emph{options}}}{}~\begin{quote}\begin{description}
\item[{Extends}] \leavevmode{\hyperref[\detokenize{reference/javascript_api:web.Widget.Widget}]{\sphinxcrossref{
Widget
}}}
\item[{Parameters}] \leavevmode\begin{itemize}

\sphinxstylestrong{parent}

\sphinxstylestrong{model} (\sphinxstyleliteralemphasis{\sphinxupquote{string}}) \textendash{} the model name (e.g. “res.partner”)

\sphinxstylestrong{chain} (\sphinxstyleliteralemphasis{\sphinxupquote{Array}}\textless{}\sphinxstyleliteralemphasis{\sphinxupquote{string}}\textgreater{}) \textendash{} list of the initial field chain parts

\sphinxstylestrong{options} ({\hyperref[\detokenize{reference/javascript_api:web.ModelFieldSelector.ModelFieldSelectorOptions}]{\sphinxcrossref{\sphinxstyleliteralemphasis{\sphinxupquote{ModelFieldSelectorOptions}}}}}) \textendash{} some key-value options
\end{itemize}

\end{description}\end{quote}

The ModelFieldSelector widget can be used to display/select a particular
field chain from a given model.


\begin{fulllineitems}
\phantomsection\label{\detokenize{reference/javascript_api:getSelectedField}}\pysiglinewithargsret{\sphinxbfcode{\sphinxupquote{method }}\sphinxbfcode{\sphinxupquote{getSelectedField}}}{}{{ $\rightarrow$ Object}}
Returns the field information selected by the field chain.
\begin{quote}\begin{description}
\item[{Return Type}] \leavevmode
\sphinxstyleliteralemphasis{\sphinxupquote{Object}}

\end{description}\end{quote}

\end{fulllineitems}



\begin{fulllineitems}
\phantomsection\label{\detokenize{reference/javascript_api:isValid}}\pysiglinewithargsret{\sphinxbfcode{\sphinxupquote{method }}\sphinxbfcode{\sphinxupquote{isValid}}}{}{{ $\rightarrow$ boolean}}
Indicates if the field chain is valid. If the field chain has not been
processed yet (the widget is not ready), this method will return
undefined.
\begin{quote}\begin{description}
\item[{Return Type}] \leavevmode
\sphinxstyleliteralemphasis{\sphinxupquote{boolean}}

\end{description}\end{quote}

\end{fulllineitems}



\begin{fulllineitems}
\phantomsection\label{\detokenize{reference/javascript_api:setChain}}\pysiglinewithargsret{\sphinxbfcode{\sphinxupquote{method }}\sphinxbfcode{\sphinxupquote{setChain}}}{\emph{chain}}{{ $\rightarrow$ Deferred}}
Saves a new field chain (array) and re-render.
\begin{quote}\begin{description}
\item[{Parameters}] \leavevmode\begin{itemize}

\sphinxstylestrong{chain} (\sphinxstyleliteralemphasis{\sphinxupquote{Array}}\textless{}\sphinxstyleliteralemphasis{\sphinxupquote{string}}\textgreater{}) \textendash{} the new field chain
\end{itemize}

\item[{Returns}] \leavevmode
resolved once the re-rendering is finished

\item[{Return Type}] \leavevmode
\sphinxstyleliteralemphasis{\sphinxupquote{Deferred}}

\end{description}\end{quote}

\end{fulllineitems}



\begin{fulllineitems}
\phantomsection\label{\detokenize{reference/javascript_api:ModelFieldSelectorOptions}}\pysiglinewithargsret{\sphinxbfcode{\sphinxupquote{class }}\sphinxbfcode{\sphinxupquote{ModelFieldSelectorOptions}}}{}{}
some key-value options


\begin{fulllineitems}
\phantomsection\label{\detokenize{reference/javascript_api:readonly}}\pysigline{\sphinxbfcode{\sphinxupquote{attribute }}\sphinxbfcode{\sphinxupquote{readonly}} boolean}
true if should be readonly

\end{fulllineitems}



\begin{fulllineitems}
\phantomsection\label{\detokenize{reference/javascript_api:filters}}\pysigline{\sphinxbfcode{\sphinxupquote{attribute }}\sphinxbfcode{\sphinxupquote{filters}} Object}
some key-value options to filter the fetched fields

\end{fulllineitems}



\begin{fulllineitems}
\phantomsection\label{\detokenize{reference/javascript_api:filters.searchable}}\pysigline{\sphinxbfcode{\sphinxupquote{attribute }}\sphinxbfcode{\sphinxupquote{filters.searchable}} boolean}
true if only the searchable fields have to be used

\end{fulllineitems}



\begin{fulllineitems}
\phantomsection\label{\detokenize{reference/javascript_api:fields}}\pysigline{\sphinxbfcode{\sphinxupquote{attribute }}\sphinxbfcode{\sphinxupquote{fields}} Object{[}{]}}~\begin{description}
\item[{the list of fields info to use when no relation has}] \leavevmode
been followed (null indicates the widget has to request
the fields itself)

\end{description}

\end{fulllineitems}



\begin{fulllineitems}
\phantomsection\label{\detokenize{reference/javascript_api:followRelations}}\pysigline{\sphinxbfcode{\sphinxupquote{attribute }}\sphinxbfcode{\sphinxupquote{followRelations}} boolean}
true if can follow relation when building the chain

\end{fulllineitems}



\begin{fulllineitems}
\phantomsection\label{\detokenize{reference/javascript_api:debugMode}}\pysigline{\sphinxbfcode{\sphinxupquote{attribute }}\sphinxbfcode{\sphinxupquote{debugMode}} boolean}
true if the widget is in debug mode, false otherwise

\end{fulllineitems}


\end{fulllineitems}


\end{fulllineitems}



\begin{fulllineitems}
\phantomsection\label{\detokenize{reference/javascript_api:ModelFieldSelector}}\pysiglinewithargsret{\sphinxbfcode{\sphinxupquote{class }}\sphinxbfcode{\sphinxupquote{ModelFieldSelector}}}{\emph{parent}, \emph{model}, \emph{chain}\sphinxoptional{, \emph{options}}}{}~\begin{quote}\begin{description}
\item[{Extends}] \leavevmode{\hyperref[\detokenize{reference/javascript_api:web.Widget.Widget}]{\sphinxcrossref{
Widget
}}}
\item[{Parameters}] \leavevmode\begin{itemize}

\sphinxstylestrong{parent}

\sphinxstylestrong{model} (\sphinxstyleliteralemphasis{\sphinxupquote{string}}) \textendash{} the model name (e.g. “res.partner”)

\sphinxstylestrong{chain} (\sphinxstyleliteralemphasis{\sphinxupquote{Array}}\textless{}\sphinxstyleliteralemphasis{\sphinxupquote{string}}\textgreater{}) \textendash{} list of the initial field chain parts

\sphinxstylestrong{options} ({\hyperref[\detokenize{reference/javascript_api:web.ModelFieldSelector.ModelFieldSelectorOptions}]{\sphinxcrossref{\sphinxstyleliteralemphasis{\sphinxupquote{ModelFieldSelectorOptions}}}}}) \textendash{} some key-value options
\end{itemize}

\end{description}\end{quote}

The ModelFieldSelector widget can be used to display/select a particular
field chain from a given model.


\begin{fulllineitems}
\phantomsection\label{\detokenize{reference/javascript_api:getSelectedField}}\pysiglinewithargsret{\sphinxbfcode{\sphinxupquote{method }}\sphinxbfcode{\sphinxupquote{getSelectedField}}}{}{{ $\rightarrow$ Object}}
Returns the field information selected by the field chain.
\begin{quote}\begin{description}
\item[{Return Type}] \leavevmode
\sphinxstyleliteralemphasis{\sphinxupquote{Object}}

\end{description}\end{quote}

\end{fulllineitems}



\begin{fulllineitems}
\phantomsection\label{\detokenize{reference/javascript_api:isValid}}\pysiglinewithargsret{\sphinxbfcode{\sphinxupquote{method }}\sphinxbfcode{\sphinxupquote{isValid}}}{}{{ $\rightarrow$ boolean}}
Indicates if the field chain is valid. If the field chain has not been
processed yet (the widget is not ready), this method will return
undefined.
\begin{quote}\begin{description}
\item[{Return Type}] \leavevmode
\sphinxstyleliteralemphasis{\sphinxupquote{boolean}}

\end{description}\end{quote}

\end{fulllineitems}



\begin{fulllineitems}
\phantomsection\label{\detokenize{reference/javascript_api:setChain}}\pysiglinewithargsret{\sphinxbfcode{\sphinxupquote{method }}\sphinxbfcode{\sphinxupquote{setChain}}}{\emph{chain}}{{ $\rightarrow$ Deferred}}
Saves a new field chain (array) and re-render.
\begin{quote}\begin{description}
\item[{Parameters}] \leavevmode\begin{itemize}

\sphinxstylestrong{chain} (\sphinxstyleliteralemphasis{\sphinxupquote{Array}}\textless{}\sphinxstyleliteralemphasis{\sphinxupquote{string}}\textgreater{}) \textendash{} the new field chain
\end{itemize}

\item[{Returns}] \leavevmode
resolved once the re-rendering is finished

\item[{Return Type}] \leavevmode
\sphinxstyleliteralemphasis{\sphinxupquote{Deferred}}

\end{description}\end{quote}

\end{fulllineitems}



\begin{fulllineitems}
\phantomsection\label{\detokenize{reference/javascript_api:ModelFieldSelectorOptions}}\pysiglinewithargsret{\sphinxbfcode{\sphinxupquote{class }}\sphinxbfcode{\sphinxupquote{ModelFieldSelectorOptions}}}{}{}
some key-value options


\begin{fulllineitems}
\phantomsection\label{\detokenize{reference/javascript_api:readonly}}\pysigline{\sphinxbfcode{\sphinxupquote{attribute }}\sphinxbfcode{\sphinxupquote{readonly}} boolean}
true if should be readonly

\end{fulllineitems}



\begin{fulllineitems}
\phantomsection\label{\detokenize{reference/javascript_api:filters}}\pysigline{\sphinxbfcode{\sphinxupquote{attribute }}\sphinxbfcode{\sphinxupquote{filters}} Object}
some key-value options to filter the fetched fields

\end{fulllineitems}



\begin{fulllineitems}
\phantomsection\label{\detokenize{reference/javascript_api:filters.searchable}}\pysigline{\sphinxbfcode{\sphinxupquote{attribute }}\sphinxbfcode{\sphinxupquote{filters.searchable}} boolean}
true if only the searchable fields have to be used

\end{fulllineitems}



\begin{fulllineitems}
\phantomsection\label{\detokenize{reference/javascript_api:fields}}\pysigline{\sphinxbfcode{\sphinxupquote{attribute }}\sphinxbfcode{\sphinxupquote{fields}} Object{[}{]}}~\begin{description}
\item[{the list of fields info to use when no relation has}] \leavevmode
been followed (null indicates the widget has to request
the fields itself)

\end{description}

\end{fulllineitems}



\begin{fulllineitems}
\phantomsection\label{\detokenize{reference/javascript_api:followRelations}}\pysigline{\sphinxbfcode{\sphinxupquote{attribute }}\sphinxbfcode{\sphinxupquote{followRelations}} boolean}
true if can follow relation when building the chain

\end{fulllineitems}



\begin{fulllineitems}
\phantomsection\label{\detokenize{reference/javascript_api:debugMode}}\pysigline{\sphinxbfcode{\sphinxupquote{attribute }}\sphinxbfcode{\sphinxupquote{debugMode}} boolean}
true if the widget is in debug mode, false otherwise

\end{fulllineitems}


\end{fulllineitems}


\end{fulllineitems}



\begin{fulllineitems}
\phantomsection\label{\detokenize{reference/javascript_api:sortFields}}\pysiglinewithargsret{\sphinxbfcode{\sphinxupquote{function }}\sphinxbfcode{\sphinxupquote{sortFields}}}{\emph{fields}, \emph{model}}{{ $\rightarrow$ Object{[}{]}}}
Allows to transform a mapping field name -\textgreater{} field info in an array of the
field infos, sorted by field user name (“string” value). The field infos in
the final array contain an additional key “name” with the field name.
\begin{quote}\begin{description}
\item[{Parameters}] \leavevmode\begin{itemize}

\sphinxstylestrong{fields} (\sphinxstyleliteralemphasis{\sphinxupquote{Object}}) \textendash{} the mapping field name -\textgreater{} field info

\sphinxstylestrong{model} (\sphinxstyleliteralemphasis{\sphinxupquote{string}})
\end{itemize}

\item[{Returns}] \leavevmode
the field infos sorted by field “string” (field infos
                    contain additional keys “model” and “name” with the field
                    name)

\item[{Return Type}] \leavevmode
\sphinxstyleliteralemphasis{\sphinxupquote{Array}}\textless{}\sphinxstyleliteralemphasis{\sphinxupquote{Object}}\textgreater{}

\end{description}\end{quote}

\end{fulllineitems}



\begin{fulllineitems}
\phantomsection\label{\detokenize{reference/javascript_api:modelFieldsCache}}\pysigline{\sphinxbfcode{\sphinxupquote{namespace }}\sphinxbfcode{\sphinxupquote{modelFieldsCache}}}
Field Selector Cache - TODO Should be improved to use external cache ?
- Stores fields per model used in field selector

\end{fulllineitems}


\end{fulllineitems}

\phantomsection\label{\detokenize{reference/javascript_api:module-hr_attendance.my_attendances}}

\begin{fulllineitems}
\phantomsection\label{\detokenize{reference/javascript_api:hr_attendance.my_attendances}}\pysigline{\sphinxbfcode{\sphinxupquote{module }}\sphinxbfcode{\sphinxupquote{hr\_attendance.my\_attendances}}}~~\begin{quote}\begin{description}
\item[{Exports}] \leavevmode{\hyperref[\detokenize{reference/javascript_api:hr_attendance.my_attendances.MyAttendances}]{\sphinxcrossref{
MyAttendances
}}}
\item[{Depends On}] \leavevmode\begin{itemize}
\item {} {\hyperref[\detokenize{reference/javascript_api:web.Widget}]{\sphinxcrossref{
web.Widget
}}}
\item {} {\hyperref[\detokenize{reference/javascript_api:web.core}]{\sphinxcrossref{
web.core
}}}
\end{itemize}

\end{description}\end{quote}


\begin{fulllineitems}
\phantomsection\label{\detokenize{reference/javascript_api:MyAttendances}}\pysiglinewithargsret{\sphinxbfcode{\sphinxupquote{class }}\sphinxbfcode{\sphinxupquote{MyAttendances}}}{}{}~\begin{quote}\begin{description}
\item[{Extends}] \leavevmode{\hyperref[\detokenize{reference/javascript_api:web.Widget.Widget}]{\sphinxcrossref{
Widget
}}}
\end{description}\end{quote}

\end{fulllineitems}


\end{fulllineitems}

\phantomsection\label{\detokenize{reference/javascript_api:module-report.client_action}}

\begin{fulllineitems}
\phantomsection\label{\detokenize{reference/javascript_api:report.client_action}}\pysigline{\sphinxbfcode{\sphinxupquote{module }}\sphinxbfcode{\sphinxupquote{report.client\_action}}}~~\begin{quote}\begin{description}
\item[{Exports}] \leavevmode{\hyperref[\detokenize{reference/javascript_api:report.client_action.ReportAction}]{\sphinxcrossref{
ReportAction
}}}
\item[{Depends On}] \leavevmode\begin{itemize}
\item {} {\hyperref[\detokenize{reference/javascript_api:report.utils}]{\sphinxcrossref{
report.utils
}}}
\item {} {\hyperref[\detokenize{reference/javascript_api:web.ControlPanelMixin}]{\sphinxcrossref{
web.ControlPanelMixin
}}}
\item {} {\hyperref[\detokenize{reference/javascript_api:web.Widget}]{\sphinxcrossref{
web.Widget
}}}
\item {} {\hyperref[\detokenize{reference/javascript_api:web.core}]{\sphinxcrossref{
web.core
}}}
\item {} {\hyperref[\detokenize{reference/javascript_api:web.session}]{\sphinxcrossref{
web.session
}}}
\end{itemize}

\end{description}\end{quote}


\begin{fulllineitems}
\phantomsection\label{\detokenize{reference/javascript_api:ReportAction}}\pysiglinewithargsret{\sphinxbfcode{\sphinxupquote{class }}\sphinxbfcode{\sphinxupquote{ReportAction}}}{\emph{parent}, \emph{action}, \emph{options}}{}~\begin{quote}\begin{description}
\item[{Extends}] \leavevmode{\hyperref[\detokenize{reference/javascript_api:web.Widget.Widget}]{\sphinxcrossref{
Widget
}}}
\item[{Mixes}] \leavevmode\begin{itemize}
\item {} {\hyperref[\detokenize{reference/javascript_api:web.ControlPanelMixin.ControlPanelMixin}]{\sphinxcrossref{
ControlPanelMixin
}}}
\end{itemize}

\item[{Parameters}] \leavevmode\begin{itemize}

\sphinxstylestrong{parent}

\sphinxstylestrong{action}

\sphinxstylestrong{options}
\end{itemize}

\end{description}\end{quote}


\begin{fulllineitems}
\phantomsection\label{\detokenize{reference/javascript_api:on_message_received}}\pysiglinewithargsret{\sphinxbfcode{\sphinxupquote{method }}\sphinxbfcode{\sphinxupquote{on\_message\_received}}}{\emph{ev}}{}
Event handler of the message post. We only handle them if they’re from
\sphinxcode{\sphinxupquote{web.base.url}} host and protocol and if they’re part of \sphinxcode{\sphinxupquote{AUTHORIZED\_MESSAGES}}.
\begin{quote}\begin{description}
\item[{Parameters}] \leavevmode\begin{itemize}

\sphinxstylestrong{ev}
\end{itemize}

\end{description}\end{quote}

\end{fulllineitems}


\end{fulllineitems}


\end{fulllineitems}

\phantomsection\label{\detokenize{reference/javascript_api:module-web.Sidebar}}

\begin{fulllineitems}
\phantomsection\label{\detokenize{reference/javascript_api:web.Sidebar}}\pysigline{\sphinxbfcode{\sphinxupquote{module }}\sphinxbfcode{\sphinxupquote{web.Sidebar}}}~~\begin{quote}\begin{description}
\item[{Exports}] \leavevmode{\hyperref[\detokenize{reference/javascript_api:web.Sidebar.Sidebar}]{\sphinxcrossref{
Sidebar
}}}
\item[{Depends On}] \leavevmode\begin{itemize}
\item {} {\hyperref[\detokenize{reference/javascript_api:web.Context}]{\sphinxcrossref{
web.Context
}}}
\item {} {\hyperref[\detokenize{reference/javascript_api:web.Widget}]{\sphinxcrossref{
web.Widget
}}}
\item {} {\hyperref[\detokenize{reference/javascript_api:web.core}]{\sphinxcrossref{
web.core
}}}
\item {} {\hyperref[\detokenize{reference/javascript_api:web.pyeval}]{\sphinxcrossref{
web.pyeval
}}}
\end{itemize}

\end{description}\end{quote}


\begin{fulllineitems}
\phantomsection\label{\detokenize{reference/javascript_api:Sidebar}}\pysiglinewithargsret{\sphinxbfcode{\sphinxupquote{class }}\sphinxbfcode{\sphinxupquote{Sidebar}}}{\emph{parent}, \emph{options}}{}~\begin{quote}\begin{description}
\item[{Extends}] \leavevmode{\hyperref[\detokenize{reference/javascript_api:web.Widget.Widget}]{\sphinxcrossref{
Widget
}}}
\item[{Parameters}] \leavevmode\begin{itemize}

\sphinxstylestrong{parent}

\sphinxstylestrong{options}
\end{itemize}

\end{description}\end{quote}


\begin{fulllineitems}
\phantomsection\label{\detokenize{reference/javascript_api:start}}\pysiglinewithargsret{\sphinxbfcode{\sphinxupquote{method }}\sphinxbfcode{\sphinxupquote{start}}}{}{}
Get the attachment linked to the record when the toolbar started

\end{fulllineitems}


\end{fulllineitems}


\end{fulllineitems}

\phantomsection\label{\detokenize{reference/javascript_api:module-mass_mailing.unsubscribe}}

\begin{fulllineitems}
\phantomsection\label{\detokenize{reference/javascript_api:mass_mailing.unsubscribe}}\pysigline{\sphinxbfcode{\sphinxupquote{module }}\sphinxbfcode{\sphinxupquote{mass\_mailing.unsubscribe}}}~~\begin{quote}\begin{description}
\item[{Exports}] \leavevmode{\hyperref[\detokenize{reference/javascript_api:mass_mailing.unsubscribe.}]{\sphinxcrossref{
\textless{}anonymous\textgreater{}
}}}
\item[{Depends On}] \leavevmode\begin{itemize}
\item {} {\hyperref[\detokenize{reference/javascript_api:web.ajax}]{\sphinxcrossref{
web.ajax
}}}
\item {} {\hyperref[\detokenize{reference/javascript_api:web.core}]{\sphinxcrossref{
web.core
}}}
\end{itemize}

\end{description}\end{quote}


\begin{fulllineitems}
\phantomsection\label{\detokenize{reference/javascript_api:mass_mailing.unsubscribe.}}\pysigline{\sphinxbfcode{\sphinxupquote{namespace }}\sphinxbfcode{\sphinxupquote{}}}
\end{fulllineitems}


\end{fulllineitems}

\phantomsection\label{\detokenize{reference/javascript_api:module-survey.survey}}

\begin{fulllineitems}
\phantomsection\label{\detokenize{reference/javascript_api:survey.survey}}\pysigline{\sphinxbfcode{\sphinxupquote{module }}\sphinxbfcode{\sphinxupquote{survey.survey}}}~~\begin{quote}\begin{description}
\item[{Exports}] \leavevmode{\hyperref[\detokenize{reference/javascript_api:survey.survey.}]{\sphinxcrossref{
\textless{}anonymous\textgreater{}
}}}
\item[{Depends On}] \leavevmode\begin{itemize}
\item {} {\hyperref[\detokenize{reference/javascript_api:web.ajax}]{\sphinxcrossref{
web.ajax
}}}
\item {} {\hyperref[\detokenize{reference/javascript_api:web.core}]{\sphinxcrossref{
web.core
}}}
\item {} {\hyperref[\detokenize{reference/javascript_api:web.field_utils}]{\sphinxcrossref{
web.field\_utils
}}}
\item {} {\hyperref[\detokenize{reference/javascript_api:web.time}]{\sphinxcrossref{
web.time
}}}
\item {} {\hyperref[\detokenize{reference/javascript_api:web_editor.base}]{\sphinxcrossref{
web\_editor.base
}}}
\item {} {\hyperref[\detokenize{reference/javascript_api:web_editor.context}]{\sphinxcrossref{
web\_editor.context
}}}
\end{itemize}

\end{description}\end{quote}


\begin{fulllineitems}
\phantomsection\label{\detokenize{reference/javascript_api:survey.survey.}}\pysigline{\sphinxbfcode{\sphinxupquote{namespace }}\sphinxbfcode{\sphinxupquote{}}}
\end{fulllineitems}


\end{fulllineitems}

\phantomsection\label{\detokenize{reference/javascript_api:module-web.PivotRenderer}}

\begin{fulllineitems}
\phantomsection\label{\detokenize{reference/javascript_api:web.PivotRenderer}}\pysigline{\sphinxbfcode{\sphinxupquote{module }}\sphinxbfcode{\sphinxupquote{web.PivotRenderer}}}~~\begin{quote}\begin{description}
\item[{Exports}] \leavevmode{\hyperref[\detokenize{reference/javascript_api:web.PivotRenderer.PivotRenderer}]{\sphinxcrossref{
PivotRenderer
}}}
\item[{Depends On}] \leavevmode\begin{itemize}
\item {} {\hyperref[\detokenize{reference/javascript_api:web.AbstractRenderer}]{\sphinxcrossref{
web.AbstractRenderer
}}}
\item {} {\hyperref[\detokenize{reference/javascript_api:web.core}]{\sphinxcrossref{
web.core
}}}
\item {} {\hyperref[\detokenize{reference/javascript_api:web.field_utils}]{\sphinxcrossref{
web.field\_utils
}}}
\end{itemize}

\end{description}\end{quote}


\begin{fulllineitems}
\phantomsection\label{\detokenize{reference/javascript_api:PivotRenderer}}\pysiglinewithargsret{\sphinxbfcode{\sphinxupquote{class }}\sphinxbfcode{\sphinxupquote{PivotRenderer}}}{}{}~\begin{quote}\begin{description}
\item[{Extends}] \leavevmode{\hyperref[\detokenize{reference/javascript_api:web.AbstractRenderer.AbstractRenderer}]{\sphinxcrossref{
AbstractRenderer
}}}
\end{description}\end{quote}

\end{fulllineitems}


\end{fulllineitems}

\phantomsection\label{\detokenize{reference/javascript_api:module-web.QWeb}}

\begin{fulllineitems}
\phantomsection\label{\detokenize{reference/javascript_api:web.QWeb}}\pysigline{\sphinxbfcode{\sphinxupquote{module }}\sphinxbfcode{\sphinxupquote{web.QWeb}}}~~\begin{quote}\begin{description}
\item[{Exports}] \leavevmode{\hyperref[\detokenize{reference/javascript_api:web.QWeb.QWeb}]{\sphinxcrossref{
QWeb
}}}
\item[{Depends On}] \leavevmode\begin{itemize}
\item {} {\hyperref[\detokenize{reference/javascript_api:web.translation}]{\sphinxcrossref{
web.translation
}}}
\end{itemize}

\end{description}\end{quote}


\begin{fulllineitems}
\phantomsection\label{\detokenize{reference/javascript_api:QWeb}}\pysiglinewithargsret{\sphinxbfcode{\sphinxupquote{function }}\sphinxbfcode{\sphinxupquote{QWeb}}}{\emph{debug}, \emph{default\_dict}\sphinxoptional{, \emph{enableTranslation}}}{}~\begin{quote}\begin{description}
\item[{Parameters}] \leavevmode\begin{itemize}

\sphinxstylestrong{debug} (\sphinxstyleliteralemphasis{\sphinxupquote{boolean}})

\sphinxstylestrong{default\_dict} (\sphinxstyleliteralemphasis{\sphinxupquote{Object}})

\sphinxstylestrong{enableTranslation}=\sphinxstyleemphasis{true} (\sphinxstyleliteralemphasis{\sphinxupquote{boolean}}) \textendash{} if true (this is the default),
  the rendering will translate all strings that are not marked with
  t-translation=off.  This is useful for the kanban view, which uses a
  template which is already translated by the server
\end{itemize}

\end{description}\end{quote}

\end{fulllineitems}


\end{fulllineitems}

\phantomsection\label{\detokenize{reference/javascript_api:module-web_editor.IframeRoot}}

\begin{fulllineitems}
\phantomsection\label{\detokenize{reference/javascript_api:web_editor.IframeRoot}}\pysigline{\sphinxbfcode{\sphinxupquote{module }}\sphinxbfcode{\sphinxupquote{web\_editor.IframeRoot}}}~~\begin{quote}\begin{description}
\item[{Exports}] \leavevmode{\hyperref[\detokenize{reference/javascript_api:web_editor.IframeRoot.}]{\sphinxcrossref{
\textless{}anonymous\textgreater{}
}}}
\item[{Depends On}] \leavevmode\begin{itemize}
\item {} {\hyperref[\detokenize{reference/javascript_api:web_editor.BodyManager}]{\sphinxcrossref{
web\_editor.BodyManager
}}}
\item {} {\hyperref[\detokenize{reference/javascript_api:web_editor.context}]{\sphinxcrossref{
web\_editor.context
}}}
\item {} {\hyperref[\detokenize{reference/javascript_api:web_editor.editor}]{\sphinxcrossref{
web\_editor.editor
}}}
\item {} {\hyperref[\detokenize{reference/javascript_api:web_editor.root_widget}]{\sphinxcrossref{
web\_editor.root\_widget
}}}
\item {} {\hyperref[\detokenize{reference/javascript_api:web_editor.translate}]{\sphinxcrossref{
web\_editor.translate
}}}
\end{itemize}

\end{description}\end{quote}


\begin{fulllineitems}
\phantomsection\label{\detokenize{reference/javascript_api:web_editor.IframeRoot.}}\pysigline{\sphinxbfcode{\sphinxupquote{namespace }}\sphinxbfcode{\sphinxupquote{}}}
\end{fulllineitems}


\end{fulllineitems}

\phantomsection\label{\detokenize{reference/javascript_api:module-web.Bus}}

\begin{fulllineitems}
\phantomsection\label{\detokenize{reference/javascript_api:web.Bus}}\pysigline{\sphinxbfcode{\sphinxupquote{module }}\sphinxbfcode{\sphinxupquote{web.Bus}}}~~\begin{quote}\begin{description}
\item[{Exports}] \leavevmode
\textless{}anonymous\textgreater{}

\item[{Depends On}] \leavevmode\begin{itemize}
\item {} {\hyperref[\detokenize{reference/javascript_api:web.Class}]{\sphinxcrossref{
web.Class
}}}
\item {} {\hyperref[\detokenize{reference/javascript_api:web.mixins}]{\sphinxcrossref{
web.mixins
}}}
\end{itemize}

\end{description}\end{quote}


\begin{fulllineitems}
\phantomsection\label{\detokenize{reference/javascript_api:Bus}}\pysiglinewithargsret{\sphinxbfcode{\sphinxupquote{class }}\sphinxbfcode{\sphinxupquote{Bus}}}{\emph{parent}}{}~\begin{quote}\begin{description}
\item[{Extends}] \leavevmode{\hyperref[\detokenize{reference/javascript_api:web.Class.Class}]{\sphinxcrossref{
Class
}}}
\item[{Mixes}] \leavevmode\begin{itemize}
\item {} {\hyperref[\detokenize{reference/javascript_api:web.mixins.EventDispatcherMixin}]{\sphinxcrossref{
EventDispatcherMixin
}}}
\end{itemize}

\item[{Parameters}] \leavevmode\begin{itemize}

\sphinxstylestrong{parent}
\end{itemize}

\end{description}\end{quote}

Event Bus used to bind events scoped in the current instance

\end{fulllineitems}


\end{fulllineitems}

\phantomsection\label{\detokenize{reference/javascript_api:module-web_editor.rte}}

\begin{fulllineitems}
\phantomsection\label{\detokenize{reference/javascript_api:web_editor.rte}}\pysigline{\sphinxbfcode{\sphinxupquote{module }}\sphinxbfcode{\sphinxupquote{web\_editor.rte}}}~~\begin{quote}\begin{description}
\item[{Exports}] \leavevmode{\hyperref[\detokenize{reference/javascript_api:web_editor.rte.}]{\sphinxcrossref{
\textless{}anonymous\textgreater{}
}}}
\item[{Depends On}] \leavevmode\begin{itemize}
\item {} {\hyperref[\detokenize{reference/javascript_api:web.Widget}]{\sphinxcrossref{
web.Widget
}}}
\item {} {\hyperref[\detokenize{reference/javascript_api:web.concurrency}]{\sphinxcrossref{
web.concurrency
}}}
\item {} {\hyperref[\detokenize{reference/javascript_api:web.core}]{\sphinxcrossref{
web.core
}}}
\item {} {\hyperref[\detokenize{reference/javascript_api:web_editor.context}]{\sphinxcrossref{
web\_editor.context
}}}
\item {} {\hyperref[\detokenize{reference/javascript_api:web_editor.summernote}]{\sphinxcrossref{
web\_editor.summernote
}}}
\item {} {\hyperref[\detokenize{reference/javascript_api:web_editor.widget}]{\sphinxcrossref{
web\_editor.widget
}}}
\end{itemize}

\end{description}\end{quote}


\begin{fulllineitems}
\phantomsection\label{\detokenize{reference/javascript_api:web_editor.rte.}}\pysigline{\sphinxbfcode{\sphinxupquote{namespace }}\sphinxbfcode{\sphinxupquote{}}}
\end{fulllineitems}


\end{fulllineitems}

\phantomsection\label{\detokenize{reference/javascript_api:module-web.widget_registry}}

\begin{fulllineitems}
\phantomsection\label{\detokenize{reference/javascript_api:web.widget_registry}}\pysigline{\sphinxbfcode{\sphinxupquote{module }}\sphinxbfcode{\sphinxupquote{web.widget\_registry}}}~~\begin{quote}\begin{description}
\item[{Exports}] \leavevmode{\hyperref[\detokenize{reference/javascript_api:web.widget_registry.}]{\sphinxcrossref{
\textless{}anonymous\textgreater{}
}}}
\item[{Depends On}] \leavevmode\begin{itemize}
\item {} {\hyperref[\detokenize{reference/javascript_api:web.Registry}]{\sphinxcrossref{
web.Registry
}}}
\end{itemize}

\end{description}\end{quote}


\begin{fulllineitems}
\phantomsection\label{\detokenize{reference/javascript_api:web.widget_registry.}}\pysigline{\sphinxbfcode{\sphinxupquote{object }}\sphinxbfcode{\sphinxupquote{}}\sphinxbfcode{\sphinxupquote{ instance of }}{\hyperref[\detokenize{reference/javascript_api:web.Registry.Registry}]{\sphinxcrossref{Registry}}}}
\end{fulllineitems}


\end{fulllineitems}

\phantomsection\label{\detokenize{reference/javascript_api:module-mail.ExtendedChatWindow}}

\begin{fulllineitems}
\phantomsection\label{\detokenize{reference/javascript_api:mail.ExtendedChatWindow}}\pysigline{\sphinxbfcode{\sphinxupquote{module }}\sphinxbfcode{\sphinxupquote{mail.ExtendedChatWindow}}}~~\begin{quote}\begin{description}
\item[{Exports}] \leavevmode{\hyperref[\detokenize{reference/javascript_api:mail.ExtendedChatWindow.}]{\sphinxcrossref{
\textless{}anonymous\textgreater{}
}}}
\item[{Depends On}] \leavevmode\begin{itemize}
\item {} {\hyperref[\detokenize{reference/javascript_api:mail.ChatWindow}]{\sphinxcrossref{
mail.ChatWindow
}}}
\item {} {\hyperref[\detokenize{reference/javascript_api:mail.chat_manager}]{\sphinxcrossref{
mail.chat\_manager
}}}
\item {} {\hyperref[\detokenize{reference/javascript_api:mail.composer}]{\sphinxcrossref{
mail.composer
}}}
\item {} {\hyperref[\detokenize{reference/javascript_api:web.core}]{\sphinxcrossref{
web.core
}}}
\end{itemize}

\end{description}\end{quote}


\begin{fulllineitems}
\phantomsection\label{\detokenize{reference/javascript_api:mail.ExtendedChatWindow.}}\pysiglinewithargsret{\sphinxbfcode{\sphinxupquote{class }}\sphinxbfcode{\sphinxupquote{}}}{}{}~\begin{quote}\begin{description}
\item[{Extends}] \leavevmode{\hyperref[\detokenize{reference/javascript_api:mail.ChatWindow.}]{\sphinxcrossref{

}}}
\end{description}\end{quote}

\end{fulllineitems}


\end{fulllineitems}

\phantomsection\label{\detokenize{reference/javascript_api:module-web.FormView}}

\begin{fulllineitems}
\phantomsection\label{\detokenize{reference/javascript_api:web.FormView}}\pysigline{\sphinxbfcode{\sphinxupquote{module }}\sphinxbfcode{\sphinxupquote{web.FormView}}}~~\begin{quote}\begin{description}
\item[{Exports}] \leavevmode{\hyperref[\detokenize{reference/javascript_api:web.FormView.FormView}]{\sphinxcrossref{
FormView
}}}
\item[{Depends On}] \leavevmode\begin{itemize}
\item {} {\hyperref[\detokenize{reference/javascript_api:web.BasicView}]{\sphinxcrossref{
web.BasicView
}}}
\item {} {\hyperref[\detokenize{reference/javascript_api:web.Context}]{\sphinxcrossref{
web.Context
}}}
\item {} {\hyperref[\detokenize{reference/javascript_api:web.FormController}]{\sphinxcrossref{
web.FormController
}}}
\item {} {\hyperref[\detokenize{reference/javascript_api:web.FormRenderer}]{\sphinxcrossref{
web.FormRenderer
}}}
\item {} {\hyperref[\detokenize{reference/javascript_api:web.core}]{\sphinxcrossref{
web.core
}}}
\end{itemize}

\end{description}\end{quote}


\begin{fulllineitems}
\phantomsection\label{\detokenize{reference/javascript_api:FormView}}\pysiglinewithargsret{\sphinxbfcode{\sphinxupquote{class }}\sphinxbfcode{\sphinxupquote{FormView}}}{\emph{viewInfo}}{}~\begin{quote}\begin{description}
\item[{Extends}] \leavevmode{\hyperref[\detokenize{reference/javascript_api:web.BasicView.BasicView}]{\sphinxcrossref{
BasicView
}}}
\item[{Parameters}] \leavevmode\begin{itemize}

\sphinxstylestrong{viewInfo}
\end{itemize}

\end{description}\end{quote}

\end{fulllineitems}


\end{fulllineitems}

\phantomsection\label{\detokenize{reference/javascript_api:module-portal.portal}}

\begin{fulllineitems}
\phantomsection\label{\detokenize{reference/javascript_api:portal.portal}}\pysigline{\sphinxbfcode{\sphinxupquote{module }}\sphinxbfcode{\sphinxupquote{portal.portal}}}~~\begin{quote}\begin{description}
\item[{Exports}] \leavevmode{\hyperref[\detokenize{reference/javascript_api:portal.portal.}]{\sphinxcrossref{
\textless{}anonymous\textgreater{}
}}}
\end{description}\end{quote}


\begin{fulllineitems}
\phantomsection\label{\detokenize{reference/javascript_api:portal.portal.}}\pysigline{\sphinxbfcode{\sphinxupquote{namespace }}\sphinxbfcode{\sphinxupquote{}}}
\end{fulllineitems}


\end{fulllineitems}

\phantomsection\label{\detokenize{reference/javascript_api:module-web_editor.base}}

\begin{fulllineitems}
\phantomsection\label{\detokenize{reference/javascript_api:web_editor.base}}\pysigline{\sphinxbfcode{\sphinxupquote{module }}\sphinxbfcode{\sphinxupquote{web\_editor.base}}}~~\begin{quote}\begin{description}
\item[{Exports}] \leavevmode{\hyperref[\detokenize{reference/javascript_api:web_editor.base.}]{\sphinxcrossref{
\textless{}anonymous\textgreater{}
}}}
\item[{Depends On}] \leavevmode\begin{itemize}
\item {} {\hyperref[\detokenize{reference/javascript_api:web.ajax}]{\sphinxcrossref{
web.ajax
}}}
\item {} {\hyperref[\detokenize{reference/javascript_api:web.session}]{\sphinxcrossref{
web.session
}}}
\end{itemize}

\end{description}\end{quote}


\begin{fulllineitems}
\phantomsection\label{\detokenize{reference/javascript_api:web_editor.base.}}\pysigline{\sphinxbfcode{\sphinxupquote{namespace }}\sphinxbfcode{\sphinxupquote{}}}~

\begin{fulllineitems}
\phantomsection\label{\detokenize{reference/javascript_api:ready}}\pysiglinewithargsret{\sphinxbfcode{\sphinxupquote{function }}\sphinxbfcode{\sphinxupquote{ready}}}{}{{ $\rightarrow$ Deferred}}
If a widget needs to be instantiated on page loading, it needs to wait
for appropriate resources to be loaded. This function returns a Deferred
which is resolved when the dom is ready, the session is bound
(translations loaded) and the XML is loaded. This should however not be
necessary anymore as widgets should not be parentless and should then be
instantiated (directly or not) by the page main component (webclient,
website root, editor bar, …). The DOM will be ready then, the main
component is in charge of waiting for the session and the XML can be
lazy loaded thanks to the @see Widget.xmlDependencies key.
\begin{quote}\begin{description}
\item[{Return Type}] \leavevmode
\sphinxstyleliteralemphasis{\sphinxupquote{Deferred}}

\end{description}\end{quote}

\end{fulllineitems}


\end{fulllineitems}


\end{fulllineitems}

\phantomsection\label{\detokenize{reference/javascript_api:module-web.framework}}

\begin{fulllineitems}
\phantomsection\label{\detokenize{reference/javascript_api:web.framework}}\pysigline{\sphinxbfcode{\sphinxupquote{module }}\sphinxbfcode{\sphinxupquote{web.framework}}}~~\begin{quote}\begin{description}
\item[{Exports}] \leavevmode{\hyperref[\detokenize{reference/javascript_api:web.framework.}]{\sphinxcrossref{
\textless{}anonymous\textgreater{}
}}}
\item[{Depends On}] \leavevmode\begin{itemize}
\item {} {\hyperref[\detokenize{reference/javascript_api:web.Widget}]{\sphinxcrossref{
web.Widget
}}}
\item {} {\hyperref[\detokenize{reference/javascript_api:web.ajax}]{\sphinxcrossref{
web.ajax
}}}
\item {} {\hyperref[\detokenize{reference/javascript_api:web.core}]{\sphinxcrossref{
web.core
}}}
\item {} {\hyperref[\detokenize{reference/javascript_api:web.crash_manager}]{\sphinxcrossref{
web.crash\_manager
}}}
\end{itemize}

\end{description}\end{quote}


\begin{fulllineitems}
\phantomsection\label{\detokenize{reference/javascript_api:redirect}}\pysiglinewithargsret{\sphinxbfcode{\sphinxupquote{function }}\sphinxbfcode{\sphinxupquote{redirect}}}{\emph{url}, \emph{wait}}{}
Redirect to url by replacing window.location
If wait is true, sleep 1s and wait for the server i.e. after a restart.
\begin{quote}\begin{description}
\item[{Parameters}] \leavevmode\begin{itemize}

\sphinxstylestrong{url}

\sphinxstylestrong{wait}
\end{itemize}

\end{description}\end{quote}

\end{fulllineitems}



\begin{fulllineitems}
\phantomsection\label{\detokenize{reference/javascript_api:HistoryBack}}\pysiglinewithargsret{\sphinxbfcode{\sphinxupquote{function }}\sphinxbfcode{\sphinxupquote{HistoryBack}}}{\emph{parent}}{}
Client action to go back in breadcrumb history.
If can’t go back in history stack, will go back to home.
\begin{quote}\begin{description}
\item[{Parameters}] \leavevmode\begin{itemize}

\sphinxstylestrong{parent}
\end{itemize}

\end{description}\end{quote}

\end{fulllineitems}



\begin{fulllineitems}
\phantomsection\label{\detokenize{reference/javascript_api:ReloadContext}}\pysiglinewithargsret{\sphinxbfcode{\sphinxupquote{function }}\sphinxbfcode{\sphinxupquote{ReloadContext}}}{\emph{parent}, \emph{action}}{}
Client action to refresh the session context (making sure
HTTP requests will have the right one) then reload the
whole interface.
\begin{quote}\begin{description}
\item[{Parameters}] \leavevmode\begin{itemize}

\sphinxstylestrong{parent}

\sphinxstylestrong{action}
\end{itemize}

\end{description}\end{quote}

\end{fulllineitems}



\begin{fulllineitems}
\phantomsection\label{\detokenize{reference/javascript_api:Home}}\pysiglinewithargsret{\sphinxbfcode{\sphinxupquote{function }}\sphinxbfcode{\sphinxupquote{Home}}}{\emph{parent}, \emph{action}}{}
Client action to go back home.
\begin{quote}\begin{description}
\item[{Parameters}] \leavevmode\begin{itemize}

\sphinxstylestrong{parent}

\sphinxstylestrong{action}
\end{itemize}

\end{description}\end{quote}

\end{fulllineitems}



\begin{fulllineitems}
\phantomsection\label{\detokenize{reference/javascript_api:web.framework.}}\pysigline{\sphinxbfcode{\sphinxupquote{namespace }}\sphinxbfcode{\sphinxupquote{}}}~

\begin{fulllineitems}
\phantomsection\label{\detokenize{reference/javascript_api:redirect}}\pysiglinewithargsret{\sphinxbfcode{\sphinxupquote{function }}\sphinxbfcode{\sphinxupquote{redirect}}}{\emph{url}, \emph{wait}}{}
Redirect to url by replacing window.location
If wait is true, sleep 1s and wait for the server i.e. after a restart.
\begin{quote}\begin{description}
\item[{Parameters}] \leavevmode\begin{itemize}

\sphinxstylestrong{url}

\sphinxstylestrong{wait}
\end{itemize}

\end{description}\end{quote}

\end{fulllineitems}


\end{fulllineitems}


\end{fulllineitems}

\phantomsection\label{\detokenize{reference/javascript_api:module-web.relational_fields}}

\begin{fulllineitems}
\phantomsection\label{\detokenize{reference/javascript_api:web.relational_fields}}\pysigline{\sphinxbfcode{\sphinxupquote{module }}\sphinxbfcode{\sphinxupquote{web.relational\_fields}}}~~\begin{quote}\begin{description}
\item[{Exports}] \leavevmode{\hyperref[\detokenize{reference/javascript_api:web.relational_fields.}]{\sphinxcrossref{
\textless{}anonymous\textgreater{}
}}}
\item[{Depends On}] \leavevmode\begin{itemize}
\item {} {\hyperref[\detokenize{reference/javascript_api:web.AbstractField}]{\sphinxcrossref{
web.AbstractField
}}}
\item {} {\hyperref[\detokenize{reference/javascript_api:web.ControlPanel}]{\sphinxcrossref{
web.ControlPanel
}}}
\item {} {\hyperref[\detokenize{reference/javascript_api:web.Dialog}]{\sphinxcrossref{
web.Dialog
}}}
\item {} {\hyperref[\detokenize{reference/javascript_api:web.KanbanRenderer}]{\sphinxcrossref{
web.KanbanRenderer
}}}
\item {} {\hyperref[\detokenize{reference/javascript_api:web.ListRenderer}]{\sphinxcrossref{
web.ListRenderer
}}}
\item {} {\hyperref[\detokenize{reference/javascript_api:web.Pager}]{\sphinxcrossref{
web.Pager
}}}
\item {} {\hyperref[\detokenize{reference/javascript_api:web.basic_fields}]{\sphinxcrossref{
web.basic\_fields
}}}
\item {} {\hyperref[\detokenize{reference/javascript_api:web.concurrency}]{\sphinxcrossref{
web.concurrency
}}}
\item {} {\hyperref[\detokenize{reference/javascript_api:web.core}]{\sphinxcrossref{
web.core
}}}
\item {} {\hyperref[\detokenize{reference/javascript_api:web.data}]{\sphinxcrossref{
web.data
}}}
\item {} {\hyperref[\detokenize{reference/javascript_api:web.view_dialogs}]{\sphinxcrossref{
web.view\_dialogs
}}}
\end{itemize}

\end{description}\end{quote}


\begin{fulllineitems}
\phantomsection\label{\detokenize{reference/javascript_api:web.relational_fields.}}\pysigline{\sphinxbfcode{\sphinxupquote{namespace }}\sphinxbfcode{\sphinxupquote{}}}~

\begin{fulllineitems}
\phantomsection\label{\detokenize{reference/javascript_api:FieldMany2ManyBinaryMultiFiles}}\pysiglinewithargsret{\sphinxbfcode{\sphinxupquote{class }}\sphinxbfcode{\sphinxupquote{FieldMany2ManyBinaryMultiFiles}}}{}{}~\begin{quote}\begin{description}
\item[{Extends}] \leavevmode{\hyperref[\detokenize{reference/javascript_api:web.AbstractField.AbstractField}]{\sphinxcrossref{
AbstractField
}}}
\end{description}\end{quote}

Widget to upload or delete one or more files at the same time.

\end{fulllineitems}



\begin{fulllineitems}
\phantomsection\label{\detokenize{reference/javascript_api:FieldSelection}}\pysiglinewithargsret{\sphinxbfcode{\sphinxupquote{class }}\sphinxbfcode{\sphinxupquote{FieldSelection}}}{}{}~\begin{quote}\begin{description}
\item[{Extends}] \leavevmode{\hyperref[\detokenize{reference/javascript_api:web.AbstractField.AbstractField}]{\sphinxcrossref{
AbstractField
}}}
\end{description}\end{quote}

The FieldSelection widget is a simple select tag with a dropdown menu to
allow the selection of a range of values.  It is designed to work with fields
of type ‘selection’ and ‘many2one’.

\end{fulllineitems}



\begin{fulllineitems}
\phantomsection\label{\detokenize{reference/javascript_api:FieldReference}}\pysiglinewithargsret{\sphinxbfcode{\sphinxupquote{class }}\sphinxbfcode{\sphinxupquote{FieldReference}}}{}{}~\begin{quote}\begin{description}
\item[{Extends}] \leavevmode
FieldMany2One

\end{description}\end{quote}

The FieldReference is a combination of a select (for the model) and
a FieldMany2one for its value.
Its intern representation is similar to the many2one (a datapoint with a
\sphinxcode{\sphinxupquote{name\_get}} as data).

\end{fulllineitems}


\end{fulllineitems}



\begin{fulllineitems}
\phantomsection\label{\detokenize{reference/javascript_api:FieldMany2ManyBinaryMultiFiles}}\pysiglinewithargsret{\sphinxbfcode{\sphinxupquote{class }}\sphinxbfcode{\sphinxupquote{FieldMany2ManyBinaryMultiFiles}}}{}{}~\begin{quote}\begin{description}
\item[{Extends}] \leavevmode{\hyperref[\detokenize{reference/javascript_api:web.AbstractField.AbstractField}]{\sphinxcrossref{
AbstractField
}}}
\end{description}\end{quote}

Widget to upload or delete one or more files at the same time.

\end{fulllineitems}



\begin{fulllineitems}
\phantomsection\label{\detokenize{reference/javascript_api:FieldReference}}\pysiglinewithargsret{\sphinxbfcode{\sphinxupquote{class }}\sphinxbfcode{\sphinxupquote{FieldReference}}}{}{}~\begin{quote}\begin{description}
\item[{Extends}] \leavevmode
FieldMany2One

\end{description}\end{quote}

The FieldReference is a combination of a select (for the model) and
a FieldMany2one for its value.
Its intern representation is similar to the many2one (a datapoint with a
\sphinxcode{\sphinxupquote{name\_get}} as data).

\end{fulllineitems}



\begin{fulllineitems}
\phantomsection\label{\detokenize{reference/javascript_api:FieldSelection}}\pysiglinewithargsret{\sphinxbfcode{\sphinxupquote{class }}\sphinxbfcode{\sphinxupquote{FieldSelection}}}{}{}~\begin{quote}\begin{description}
\item[{Extends}] \leavevmode{\hyperref[\detokenize{reference/javascript_api:web.AbstractField.AbstractField}]{\sphinxcrossref{
AbstractField
}}}
\end{description}\end{quote}

The FieldSelection widget is a simple select tag with a dropdown menu to
allow the selection of a range of values.  It is designed to work with fields
of type ‘selection’ and ‘many2one’.

\end{fulllineitems}


\end{fulllineitems}

\phantomsection\label{\detokenize{reference/javascript_api:module-mail.utils}}

\begin{fulllineitems}
\phantomsection\label{\detokenize{reference/javascript_api:mail.utils}}\pysigline{\sphinxbfcode{\sphinxupquote{module }}\sphinxbfcode{\sphinxupquote{mail.utils}}}~~\begin{quote}\begin{description}
\item[{Exports}] \leavevmode{\hyperref[\detokenize{reference/javascript_api:mail.utils.}]{\sphinxcrossref{
\textless{}anonymous\textgreater{}
}}}
\item[{Depends On}] \leavevmode\begin{itemize}
\item {} {\hyperref[\detokenize{reference/javascript_api:bus.bus}]{\sphinxcrossref{
bus.bus
}}}
\end{itemize}

\end{description}\end{quote}


\begin{fulllineitems}
\phantomsection\label{\detokenize{reference/javascript_api:mail.utils.}}\pysigline{\sphinxbfcode{\sphinxupquote{namespace }}\sphinxbfcode{\sphinxupquote{}}}
\end{fulllineitems}


\end{fulllineitems}

\phantomsection\label{\detokenize{reference/javascript_api:module-web.collections}}

\begin{fulllineitems}
\phantomsection\label{\detokenize{reference/javascript_api:web.collections}}\pysigline{\sphinxbfcode{\sphinxupquote{module }}\sphinxbfcode{\sphinxupquote{web.collections}}}~~\begin{quote}\begin{description}
\item[{Exports}] \leavevmode{\hyperref[\detokenize{reference/javascript_api:web.collections.}]{\sphinxcrossref{
\textless{}anonymous\textgreater{}
}}}
\item[{Depends On}] \leavevmode\begin{itemize}
\item {} {\hyperref[\detokenize{reference/javascript_api:web.Class}]{\sphinxcrossref{
web.Class
}}}
\end{itemize}

\end{description}\end{quote}


\begin{fulllineitems}
\phantomsection\label{\detokenize{reference/javascript_api:Tree}}\pysiglinewithargsret{\sphinxbfcode{\sphinxupquote{class }}\sphinxbfcode{\sphinxupquote{Tree}}}{\emph{data}}{}~\begin{quote}\begin{description}
\item[{Extends}] \leavevmode{\hyperref[\detokenize{reference/javascript_api:web.Class.Class}]{\sphinxcrossref{
Class
}}}
\item[{Parameters}] \leavevmode\begin{itemize}

\sphinxstylestrong{data} (\sphinxstyleliteralemphasis{\sphinxupquote{any}}) \textendash{} the data associated to the root node
\end{itemize}

\end{description}\end{quote}

Allows to build a tree representation of a data.


\begin{fulllineitems}
\phantomsection\label{\detokenize{reference/javascript_api:getData}}\pysiglinewithargsret{\sphinxbfcode{\sphinxupquote{method }}\sphinxbfcode{\sphinxupquote{getData}}}{}{{ $\rightarrow$ *}}
Returns the root’s associated data.
\begin{quote}\begin{description}
\item[{Return Type}] \leavevmode
\sphinxstyleliteralemphasis{\sphinxupquote{any}}

\end{description}\end{quote}

\end{fulllineitems}



\begin{fulllineitems}
\phantomsection\label{\detokenize{reference/javascript_api:addChild}}\pysiglinewithargsret{\sphinxbfcode{\sphinxupquote{method }}\sphinxbfcode{\sphinxupquote{addChild}}}{\emph{tree}}{}
Adds a child tree.
\begin{quote}\begin{description}
\item[{Parameters}] \leavevmode\begin{itemize}

\sphinxstylestrong{tree} ({\hyperref[\detokenize{reference/javascript_api:web.collections.Tree}]{\sphinxcrossref{\sphinxstyleliteralemphasis{\sphinxupquote{Tree}}}}})
\end{itemize}

\end{description}\end{quote}

\end{fulllineitems}


\end{fulllineitems}



\begin{fulllineitems}
\phantomsection\label{\detokenize{reference/javascript_api:web.collections.}}\pysigline{\sphinxbfcode{\sphinxupquote{namespace }}\sphinxbfcode{\sphinxupquote{}}}~

\begin{fulllineitems}
\phantomsection\label{\detokenize{reference/javascript_api:Tree}}\pysiglinewithargsret{\sphinxbfcode{\sphinxupquote{class }}\sphinxbfcode{\sphinxupquote{Tree}}}{\emph{data}}{}~\begin{quote}\begin{description}
\item[{Extends}] \leavevmode{\hyperref[\detokenize{reference/javascript_api:web.Class.Class}]{\sphinxcrossref{
Class
}}}
\item[{Parameters}] \leavevmode\begin{itemize}

\sphinxstylestrong{data} (\sphinxstyleliteralemphasis{\sphinxupquote{any}}) \textendash{} the data associated to the root node
\end{itemize}

\end{description}\end{quote}

Allows to build a tree representation of a data.


\begin{fulllineitems}
\phantomsection\label{\detokenize{reference/javascript_api:getData}}\pysiglinewithargsret{\sphinxbfcode{\sphinxupquote{method }}\sphinxbfcode{\sphinxupquote{getData}}}{}{{ $\rightarrow$ *}}
Returns the root’s associated data.
\begin{quote}\begin{description}
\item[{Return Type}] \leavevmode
\sphinxstyleliteralemphasis{\sphinxupquote{any}}

\end{description}\end{quote}

\end{fulllineitems}



\begin{fulllineitems}
\phantomsection\label{\detokenize{reference/javascript_api:addChild}}\pysiglinewithargsret{\sphinxbfcode{\sphinxupquote{method }}\sphinxbfcode{\sphinxupquote{addChild}}}{\emph{tree}}{}
Adds a child tree.
\begin{quote}\begin{description}
\item[{Parameters}] \leavevmode\begin{itemize}

\sphinxstylestrong{tree} ({\hyperref[\detokenize{reference/javascript_api:web.collections.Tree}]{\sphinxcrossref{\sphinxstyleliteralemphasis{\sphinxupquote{Tree}}}}})
\end{itemize}

\end{description}\end{quote}

\end{fulllineitems}


\end{fulllineitems}


\end{fulllineitems}


\end{fulllineitems}

\phantomsection\label{\detokenize{reference/javascript_api:module-web.Class}}

\begin{fulllineitems}
\phantomsection\label{\detokenize{reference/javascript_api:web.Class}}\pysigline{\sphinxbfcode{\sphinxupquote{module }}\sphinxbfcode{\sphinxupquote{web.Class}}}~~\begin{quote}\begin{description}
\item[{Exports}] \leavevmode
OdooClass

\end{description}\end{quote}


\begin{fulllineitems}
\phantomsection\label{\detokenize{reference/javascript_api:Class}}\pysiglinewithargsret{\sphinxbfcode{\sphinxupquote{class }}\sphinxbfcode{\sphinxupquote{Class}}}{}{}
Improved John Resig’s inheritance, based on:

Simple JavaScript Inheritance
By John Resig \sphinxurl{http://ejohn.org/}
MIT Licensed.

Adds “include()”

Defines The Class object. That object can be used to define and inherit classes using
the extend() method.

Example:

\fvset{hllines={, ,}}%
\begin{sphinxVerbatim}[commandchars=\\\{\}]
\PYG{k+kd}{var} \PYG{n+nx}{Person} \PYG{o}{=} \PYG{n+nx}{Class}\PYG{p}{.}\PYG{n+nx}{extend}\PYG{p}{(}\PYG{p}{\PYGZob{}}
 \PYG{n+nx}{init}\PYG{o}{:} \PYG{k+kd}{function}\PYG{p}{(}\PYG{n+nx}{isDancing}\PYG{p}{)}\PYG{p}{\PYGZob{}}
    \PYG{k}{this}\PYG{p}{.}\PYG{n+nx}{dancing} \PYG{o}{=} \PYG{n+nx}{isDancing}\PYG{p}{;}
  \PYG{p}{\PYGZcb{}}\PYG{p}{,}
  \PYG{n+nx}{dance}\PYG{o}{:} \PYG{k+kd}{function}\PYG{p}{(}\PYG{p}{)}\PYG{p}{\PYGZob{}}
    \PYG{k}{return} \PYG{k}{this}\PYG{p}{.}\PYG{n+nx}{dancing}\PYG{p}{;}
  \PYG{p}{\PYGZcb{}}
\PYG{p}{\PYGZcb{}}\PYG{p}{)}\PYG{p}{;}
\end{sphinxVerbatim}

The init() method act as a constructor. This class can be instanced this way:

\fvset{hllines={, ,}}%
\begin{sphinxVerbatim}[commandchars=\\\{\}]
\PYG{k+kd}{var} \PYG{n+nx}{person} \PYG{o}{=} \PYG{k}{new} \PYG{n+nx}{Person}\PYG{p}{(}\PYG{k+kc}{true}\PYG{p}{)}\PYG{p}{;}
\PYG{n+nx}{person}\PYG{p}{.}\PYG{n+nx}{dance}\PYG{p}{(}\PYG{p}{)}\PYG{p}{;}

\PYG{n+nx}{The} \PYG{n+nx}{Person} \PYG{k+kr}{class} \PYG{n+nx}{can} \PYG{n+nx}{also} \PYG{n+nx}{be} \PYG{n+nx}{extended} \PYG{n+nx}{again}\PYG{o}{:}

\PYG{k+kd}{var} \PYG{n+nx}{Ninja} \PYG{o}{=} \PYG{n+nx}{Person}\PYG{p}{.}\PYG{n+nx}{extend}\PYG{p}{(}\PYG{p}{\PYGZob{}}
  \PYG{n+nx}{init}\PYG{o}{:} \PYG{k+kd}{function}\PYG{p}{(}\PYG{p}{)}\PYG{p}{\PYGZob{}}
    \PYG{k}{this}\PYG{p}{.}\PYG{n+nx}{\PYGZus{}super}\PYG{p}{(} \PYG{k+kc}{false} \PYG{p}{)}\PYG{p}{;}
  \PYG{p}{\PYGZcb{}}\PYG{p}{,}
  \PYG{n+nx}{dance}\PYG{o}{:} \PYG{k+kd}{function}\PYG{p}{(}\PYG{p}{)}\PYG{p}{\PYGZob{}}
    \PYG{c+c1}{// Call the inherited version of dance()}
    \PYG{k}{return} \PYG{k}{this}\PYG{p}{.}\PYG{n+nx}{\PYGZus{}super}\PYG{p}{(}\PYG{p}{)}\PYG{p}{;}
  \PYG{p}{\PYGZcb{}}\PYG{p}{,}
  \PYG{n+nx}{swingSword}\PYG{o}{:} \PYG{k+kd}{function}\PYG{p}{(}\PYG{p}{)}\PYG{p}{\PYGZob{}}
    \PYG{k}{return} \PYG{k+kc}{true}\PYG{p}{;}
  \PYG{p}{\PYGZcb{}}
\PYG{p}{\PYGZcb{}}\PYG{p}{)}\PYG{p}{;}
\end{sphinxVerbatim}

When extending a class, each re-defined method can use this.\_super() to call the previous
implementation of that method.


\begin{fulllineitems}
\phantomsection\label{\detokenize{reference/javascript_api:extend}}\pysiglinewithargsret{\sphinxbfcode{\sphinxupquote{method }}\sphinxbfcode{\sphinxupquote{extend}}}{\emph{prop}}{}
Subclass an existing class
\begin{quote}\begin{description}
\item[{Parameters}] \leavevmode\begin{itemize}

\sphinxstylestrong{prop} (\sphinxstyleliteralemphasis{\sphinxupquote{Object}}) \textendash{} class-level properties (class attributes and instance methods) to set on the new class
\end{itemize}

\end{description}\end{quote}

\end{fulllineitems}


\end{fulllineitems}



\begin{fulllineitems}
\phantomsection\label{\detokenize{reference/javascript_api:Class}}\pysiglinewithargsret{\sphinxbfcode{\sphinxupquote{class }}\sphinxbfcode{\sphinxupquote{Class}}}{}{}
Improved John Resig’s inheritance, based on:

Simple JavaScript Inheritance
By John Resig \sphinxurl{http://ejohn.org/}
MIT Licensed.

Adds “include()”

Defines The Class object. That object can be used to define and inherit classes using
the extend() method.

Example:

\fvset{hllines={, ,}}%
\begin{sphinxVerbatim}[commandchars=\\\{\}]
\PYG{k+kd}{var} \PYG{n+nx}{Person} \PYG{o}{=} \PYG{n+nx}{Class}\PYG{p}{.}\PYG{n+nx}{extend}\PYG{p}{(}\PYG{p}{\PYGZob{}}
 \PYG{n+nx}{init}\PYG{o}{:} \PYG{k+kd}{function}\PYG{p}{(}\PYG{n+nx}{isDancing}\PYG{p}{)}\PYG{p}{\PYGZob{}}
    \PYG{k}{this}\PYG{p}{.}\PYG{n+nx}{dancing} \PYG{o}{=} \PYG{n+nx}{isDancing}\PYG{p}{;}
  \PYG{p}{\PYGZcb{}}\PYG{p}{,}
  \PYG{n+nx}{dance}\PYG{o}{:} \PYG{k+kd}{function}\PYG{p}{(}\PYG{p}{)}\PYG{p}{\PYGZob{}}
    \PYG{k}{return} \PYG{k}{this}\PYG{p}{.}\PYG{n+nx}{dancing}\PYG{p}{;}
  \PYG{p}{\PYGZcb{}}
\PYG{p}{\PYGZcb{}}\PYG{p}{)}\PYG{p}{;}
\end{sphinxVerbatim}

The init() method act as a constructor. This class can be instanced this way:

\fvset{hllines={, ,}}%
\begin{sphinxVerbatim}[commandchars=\\\{\}]
\PYG{k+kd}{var} \PYG{n+nx}{person} \PYG{o}{=} \PYG{k}{new} \PYG{n+nx}{Person}\PYG{p}{(}\PYG{k+kc}{true}\PYG{p}{)}\PYG{p}{;}
\PYG{n+nx}{person}\PYG{p}{.}\PYG{n+nx}{dance}\PYG{p}{(}\PYG{p}{)}\PYG{p}{;}

\PYG{n+nx}{The} \PYG{n+nx}{Person} \PYG{k+kr}{class} \PYG{n+nx}{can} \PYG{n+nx}{also} \PYG{n+nx}{be} \PYG{n+nx}{extended} \PYG{n+nx}{again}\PYG{o}{:}

\PYG{k+kd}{var} \PYG{n+nx}{Ninja} \PYG{o}{=} \PYG{n+nx}{Person}\PYG{p}{.}\PYG{n+nx}{extend}\PYG{p}{(}\PYG{p}{\PYGZob{}}
  \PYG{n+nx}{init}\PYG{o}{:} \PYG{k+kd}{function}\PYG{p}{(}\PYG{p}{)}\PYG{p}{\PYGZob{}}
    \PYG{k}{this}\PYG{p}{.}\PYG{n+nx}{\PYGZus{}super}\PYG{p}{(} \PYG{k+kc}{false} \PYG{p}{)}\PYG{p}{;}
  \PYG{p}{\PYGZcb{}}\PYG{p}{,}
  \PYG{n+nx}{dance}\PYG{o}{:} \PYG{k+kd}{function}\PYG{p}{(}\PYG{p}{)}\PYG{p}{\PYGZob{}}
    \PYG{c+c1}{// Call the inherited version of dance()}
    \PYG{k}{return} \PYG{k}{this}\PYG{p}{.}\PYG{n+nx}{\PYGZus{}super}\PYG{p}{(}\PYG{p}{)}\PYG{p}{;}
  \PYG{p}{\PYGZcb{}}\PYG{p}{,}
  \PYG{n+nx}{swingSword}\PYG{o}{:} \PYG{k+kd}{function}\PYG{p}{(}\PYG{p}{)}\PYG{p}{\PYGZob{}}
    \PYG{k}{return} \PYG{k+kc}{true}\PYG{p}{;}
  \PYG{p}{\PYGZcb{}}
\PYG{p}{\PYGZcb{}}\PYG{p}{)}\PYG{p}{;}
\end{sphinxVerbatim}

When extending a class, each re-defined method can use this.\_super() to call the previous
implementation of that method.


\begin{fulllineitems}
\phantomsection\label{\detokenize{reference/javascript_api:extend}}\pysiglinewithargsret{\sphinxbfcode{\sphinxupquote{method }}\sphinxbfcode{\sphinxupquote{extend}}}{\emph{prop}}{}
Subclass an existing class
\begin{quote}\begin{description}
\item[{Parameters}] \leavevmode\begin{itemize}

\sphinxstylestrong{prop} (\sphinxstyleliteralemphasis{\sphinxupquote{Object}}) \textendash{} class-level properties (class attributes and instance methods) to set on the new class
\end{itemize}

\end{description}\end{quote}

\end{fulllineitems}


\end{fulllineitems}


\end{fulllineitems}

\phantomsection\label{\detokenize{reference/javascript_api:module-web_editor.ace}}

\begin{fulllineitems}
\phantomsection\label{\detokenize{reference/javascript_api:web_editor.ace}}\pysigline{\sphinxbfcode{\sphinxupquote{module }}\sphinxbfcode{\sphinxupquote{web\_editor.ace}}}~~\begin{quote}\begin{description}
\item[{Exports}] \leavevmode{\hyperref[\detokenize{reference/javascript_api:web_editor.ace.ViewEditor}]{\sphinxcrossref{
ViewEditor
}}}
\item[{Depends On}] \leavevmode\begin{itemize}
\item {} {\hyperref[\detokenize{reference/javascript_api:web.Dialog}]{\sphinxcrossref{
web.Dialog
}}}
\item {} {\hyperref[\detokenize{reference/javascript_api:web.Widget}]{\sphinxcrossref{
web.Widget
}}}
\item {} {\hyperref[\detokenize{reference/javascript_api:web.ajax}]{\sphinxcrossref{
web.ajax
}}}
\item {} {\hyperref[\detokenize{reference/javascript_api:web.core}]{\sphinxcrossref{
web.core
}}}
\item {} {\hyperref[\detokenize{reference/javascript_api:web.local_storage}]{\sphinxcrossref{
web.local\_storage
}}}
\item {} {\hyperref[\detokenize{reference/javascript_api:web.session}]{\sphinxcrossref{
web.session
}}}
\item {} {\hyperref[\detokenize{reference/javascript_api:web_editor.context}]{\sphinxcrossref{
web\_editor.context
}}}
\end{itemize}

\end{description}\end{quote}


\begin{fulllineitems}
\phantomsection\label{\detokenize{reference/javascript_api:checkLESS}}\pysigline{\sphinxbfcode{\sphinxupquote{namespace }}\sphinxbfcode{\sphinxupquote{checkLESS}}}
Checks the syntax validity of some LESS.

\end{fulllineitems}



\begin{fulllineitems}
\phantomsection\label{\detokenize{reference/javascript_api:formatXML}}\pysiglinewithargsret{\sphinxbfcode{\sphinxupquote{function }}\sphinxbfcode{\sphinxupquote{formatXML}}}{\emph{xml}}{{ $\rightarrow$ string}}
Formats some XML so that it has proper indentation and structure.
\begin{quote}\begin{description}
\item[{Parameters}] \leavevmode\begin{itemize}

\sphinxstylestrong{xml} (\sphinxstyleliteralemphasis{\sphinxupquote{string}})
\end{itemize}

\item[{Returns}] \leavevmode
formatted xml

\item[{Return Type}] \leavevmode
\sphinxstyleliteralemphasis{\sphinxupquote{string}}

\end{description}\end{quote}

\end{fulllineitems}



\begin{fulllineitems}
\phantomsection\label{\detokenize{reference/javascript_api:ViewEditor}}\pysiglinewithargsret{\sphinxbfcode{\sphinxupquote{class }}\sphinxbfcode{\sphinxupquote{ViewEditor}}}{\emph{parent}, \emph{viewKey}\sphinxoptional{, \emph{options}}}{}~\begin{quote}\begin{description}
\item[{Extends}] \leavevmode{\hyperref[\detokenize{reference/javascript_api:web.Widget.Widget}]{\sphinxcrossref{
Widget
}}}
\item[{Parameters}] \leavevmode\begin{itemize}

\sphinxstylestrong{parent} ({\hyperref[\detokenize{reference/javascript_api:Widget}]{\sphinxcrossref{\sphinxstyleliteralemphasis{\sphinxupquote{Widget}}}}})

\sphinxstylestrong{viewKey} (\sphinxstyleliteralemphasis{\sphinxupquote{string}}\sphinxstyleemphasis{ or }\sphinxstyleliteralemphasis{\sphinxupquote{integer}}) \textendash{} xml\_id or id of the view whose linked resources have to be loaded.

\sphinxstylestrong{options} ({\hyperref[\detokenize{reference/javascript_api:web_editor.ace.ViewEditorOptions}]{\sphinxcrossref{\sphinxstyleliteralemphasis{\sphinxupquote{ViewEditorOptions}}}}})
\end{itemize}

\end{description}\end{quote}

Allows to visualize resources (by default, XML views) and edit them.


\begin{fulllineitems}
\phantomsection\label{\detokenize{reference/javascript_api:willStart}}\pysiglinewithargsret{\sphinxbfcode{\sphinxupquote{method }}\sphinxbfcode{\sphinxupquote{willStart}}}{}{}
Loads everything the ace library needs to work.
It also loads the resources to visualize (@see \_loadResources).

\end{fulllineitems}



\begin{fulllineitems}
\phantomsection\label{\detokenize{reference/javascript_api:start}}\pysiglinewithargsret{\sphinxbfcode{\sphinxupquote{method }}\sphinxbfcode{\sphinxupquote{start}}}{}{}
Initializes the library and initial view once the DOM is ready. It also
initializes the resize feature of the ace editor.

\end{fulllineitems}



\begin{fulllineitems}
\phantomsection\label{\detokenize{reference/javascript_api:ViewEditorOptions}}\pysiglinewithargsret{\sphinxbfcode{\sphinxupquote{class }}\sphinxbfcode{\sphinxupquote{ViewEditorOptions}}}{}{}~

\begin{fulllineitems}
\phantomsection\label{\detokenize{reference/javascript_api:initialResID}}\pysigline{\sphinxbfcode{\sphinxupquote{attribute }}\sphinxbfcode{\sphinxupquote{initialResID}} string\textbar{}integer}~\begin{description}
\item[{a specific view ID / LESS URL to load on start (otherwise the main}] \leavevmode
view ID associated with the specified viewKey will be used)

\end{description}

\end{fulllineitems}



\begin{fulllineitems}
\phantomsection\label{\detokenize{reference/javascript_api:position}}\pysigline{\sphinxbfcode{\sphinxupquote{attribute }}\sphinxbfcode{\sphinxupquote{position}} string}
\end{fulllineitems}



\begin{fulllineitems}
\phantomsection\label{\detokenize{reference/javascript_api:doNotLoadViews}}\pysigline{\sphinxbfcode{\sphinxupquote{attribute }}\sphinxbfcode{\sphinxupquote{doNotLoadViews}} boolean}
\end{fulllineitems}



\begin{fulllineitems}
\phantomsection\label{\detokenize{reference/javascript_api:doNotLoadLess}}\pysigline{\sphinxbfcode{\sphinxupquote{attribute }}\sphinxbfcode{\sphinxupquote{doNotLoadLess}} boolean}
\end{fulllineitems}



\begin{fulllineitems}
\phantomsection\label{\detokenize{reference/javascript_api:includeBundles}}\pysigline{\sphinxbfcode{\sphinxupquote{attribute }}\sphinxbfcode{\sphinxupquote{includeBundles}} boolean}
\end{fulllineitems}



\begin{fulllineitems}
\phantomsection\label{\detokenize{reference/javascript_api:includeAllLess}}\pysigline{\sphinxbfcode{\sphinxupquote{attribute }}\sphinxbfcode{\sphinxupquote{includeAllLess}} boolean}
\end{fulllineitems}



\begin{fulllineitems}
\phantomsection\label{\detokenize{reference/javascript_api:defaultBundlesRestriction}}\pysigline{\sphinxbfcode{\sphinxupquote{attribute }}\sphinxbfcode{\sphinxupquote{defaultBundlesRestriction}} string{[}{]}}
\end{fulllineitems}


\end{fulllineitems}


\end{fulllineitems}



\begin{fulllineitems}
\phantomsection\label{\detokenize{reference/javascript_api:_getCheckReturn}}\pysiglinewithargsret{\sphinxbfcode{\sphinxupquote{function }}\sphinxbfcode{\sphinxupquote{\_getCheckReturn}}}{\emph{isValid}\sphinxoptional{, \emph{errorLine}}\sphinxoptional{, \emph{errorMessage}}}{{ $\rightarrow$ Object}}
Formats a content-check result (@see checkXML, checkLESS).
\begin{quote}\begin{description}
\item[{Parameters}] \leavevmode\begin{itemize}

\sphinxstylestrong{isValid} (\sphinxstyleliteralemphasis{\sphinxupquote{boolean}})

\sphinxstylestrong{errorLine} (\sphinxstyleliteralemphasis{\sphinxupquote{integer}}) \textendash{} needed if isValid is false

\sphinxstylestrong{errorMessage} (\sphinxstyleliteralemphasis{\sphinxupquote{string}}) \textendash{} needed if isValid is false
\end{itemize}

\item[{Return Type}] \leavevmode
\sphinxstyleliteralemphasis{\sphinxupquote{Object}}

\end{description}\end{quote}

\end{fulllineitems}



\begin{fulllineitems}
\phantomsection\label{\detokenize{reference/javascript_api:ViewEditor}}\pysiglinewithargsret{\sphinxbfcode{\sphinxupquote{class }}\sphinxbfcode{\sphinxupquote{ViewEditor}}}{\emph{parent}, \emph{viewKey}\sphinxoptional{, \emph{options}}}{}~\begin{quote}\begin{description}
\item[{Extends}] \leavevmode{\hyperref[\detokenize{reference/javascript_api:web.Widget.Widget}]{\sphinxcrossref{
Widget
}}}
\item[{Parameters}] \leavevmode\begin{itemize}

\sphinxstylestrong{parent} ({\hyperref[\detokenize{reference/javascript_api:Widget}]{\sphinxcrossref{\sphinxstyleliteralemphasis{\sphinxupquote{Widget}}}}})

\sphinxstylestrong{viewKey} (\sphinxstyleliteralemphasis{\sphinxupquote{string}}\sphinxstyleemphasis{ or }\sphinxstyleliteralemphasis{\sphinxupquote{integer}}) \textendash{} xml\_id or id of the view whose linked resources have to be loaded.

\sphinxstylestrong{options} ({\hyperref[\detokenize{reference/javascript_api:web_editor.ace.ViewEditorOptions}]{\sphinxcrossref{\sphinxstyleliteralemphasis{\sphinxupquote{ViewEditorOptions}}}}})
\end{itemize}

\end{description}\end{quote}

Allows to visualize resources (by default, XML views) and edit them.


\begin{fulllineitems}
\phantomsection\label{\detokenize{reference/javascript_api:willStart}}\pysiglinewithargsret{\sphinxbfcode{\sphinxupquote{method }}\sphinxbfcode{\sphinxupquote{willStart}}}{}{}
Loads everything the ace library needs to work.
It also loads the resources to visualize (@see \_loadResources).

\end{fulllineitems}



\begin{fulllineitems}
\phantomsection\label{\detokenize{reference/javascript_api:start}}\pysiglinewithargsret{\sphinxbfcode{\sphinxupquote{method }}\sphinxbfcode{\sphinxupquote{start}}}{}{}
Initializes the library and initial view once the DOM is ready. It also
initializes the resize feature of the ace editor.

\end{fulllineitems}



\begin{fulllineitems}
\phantomsection\label{\detokenize{reference/javascript_api:ViewEditorOptions}}\pysiglinewithargsret{\sphinxbfcode{\sphinxupquote{class }}\sphinxbfcode{\sphinxupquote{ViewEditorOptions}}}{}{}~

\begin{fulllineitems}
\phantomsection\label{\detokenize{reference/javascript_api:initialResID}}\pysigline{\sphinxbfcode{\sphinxupquote{attribute }}\sphinxbfcode{\sphinxupquote{initialResID}} string\textbar{}integer}~\begin{description}
\item[{a specific view ID / LESS URL to load on start (otherwise the main}] \leavevmode
view ID associated with the specified viewKey will be used)

\end{description}

\end{fulllineitems}



\begin{fulllineitems}
\phantomsection\label{\detokenize{reference/javascript_api:position}}\pysigline{\sphinxbfcode{\sphinxupquote{attribute }}\sphinxbfcode{\sphinxupquote{position}} string}
\end{fulllineitems}



\begin{fulllineitems}
\phantomsection\label{\detokenize{reference/javascript_api:doNotLoadViews}}\pysigline{\sphinxbfcode{\sphinxupquote{attribute }}\sphinxbfcode{\sphinxupquote{doNotLoadViews}} boolean}
\end{fulllineitems}



\begin{fulllineitems}
\phantomsection\label{\detokenize{reference/javascript_api:doNotLoadLess}}\pysigline{\sphinxbfcode{\sphinxupquote{attribute }}\sphinxbfcode{\sphinxupquote{doNotLoadLess}} boolean}
\end{fulllineitems}



\begin{fulllineitems}
\phantomsection\label{\detokenize{reference/javascript_api:includeBundles}}\pysigline{\sphinxbfcode{\sphinxupquote{attribute }}\sphinxbfcode{\sphinxupquote{includeBundles}} boolean}
\end{fulllineitems}



\begin{fulllineitems}
\phantomsection\label{\detokenize{reference/javascript_api:includeAllLess}}\pysigline{\sphinxbfcode{\sphinxupquote{attribute }}\sphinxbfcode{\sphinxupquote{includeAllLess}} boolean}
\end{fulllineitems}



\begin{fulllineitems}
\phantomsection\label{\detokenize{reference/javascript_api:defaultBundlesRestriction}}\pysigline{\sphinxbfcode{\sphinxupquote{attribute }}\sphinxbfcode{\sphinxupquote{defaultBundlesRestriction}} string{[}{]}}
\end{fulllineitems}


\end{fulllineitems}


\end{fulllineitems}



\begin{fulllineitems}
\phantomsection\label{\detokenize{reference/javascript_api:formatLESS}}\pysiglinewithargsret{\sphinxbfcode{\sphinxupquote{function }}\sphinxbfcode{\sphinxupquote{formatLESS}}}{\emph{less}}{{ $\rightarrow$ string}}
Formats some LESS so that it has proper indentation and structure.
\begin{quote}\begin{description}
\item[{Parameters}] \leavevmode\begin{itemize}

\sphinxstylestrong{less} (\sphinxstyleliteralemphasis{\sphinxupquote{string}})
\end{itemize}

\item[{Returns}] \leavevmode
formatted less

\item[{Return Type}] \leavevmode
\sphinxstyleliteralemphasis{\sphinxupquote{string}}

\end{description}\end{quote}

\end{fulllineitems}



\begin{fulllineitems}
\phantomsection\label{\detokenize{reference/javascript_api:checkXML}}\pysiglinewithargsret{\sphinxbfcode{\sphinxupquote{function }}\sphinxbfcode{\sphinxupquote{checkXML}}}{\emph{xml}}{{ $\rightarrow$ Object}}
Checks the syntax validity of some XML.
\begin{quote}\begin{description}
\item[{Parameters}] \leavevmode\begin{itemize}

\sphinxstylestrong{xml} (\sphinxstyleliteralemphasis{\sphinxupquote{string}})
\end{itemize}

\item[{Returns}] \leavevmode
@see \_getCheckReturn

\item[{Return Type}] \leavevmode
\sphinxstyleliteralemphasis{\sphinxupquote{Object}}

\end{description}\end{quote}

\end{fulllineitems}


\end{fulllineitems}

\phantomsection\label{\detokenize{reference/javascript_api:module-website.WebsiteRoot}}

\begin{fulllineitems}
\phantomsection\label{\detokenize{reference/javascript_api:website.WebsiteRoot}}\pysigline{\sphinxbfcode{\sphinxupquote{module }}\sphinxbfcode{\sphinxupquote{website.WebsiteRoot}}}~~\begin{quote}\begin{description}
\item[{Exports}] \leavevmode{\hyperref[\detokenize{reference/javascript_api:website.WebsiteRoot.}]{\sphinxcrossref{
\textless{}anonymous\textgreater{}
}}}
\item[{Depends On}] \leavevmode\begin{itemize}
\item {} {\hyperref[\detokenize{reference/javascript_api:web.Dialog}]{\sphinxcrossref{
web.Dialog
}}}
\item {} {\hyperref[\detokenize{reference/javascript_api:web.ajax}]{\sphinxcrossref{
web.ajax
}}}
\item {} {\hyperref[\detokenize{reference/javascript_api:web.core}]{\sphinxcrossref{
web.core
}}}
\item {} {\hyperref[\detokenize{reference/javascript_api:web.utils}]{\sphinxcrossref{
web.utils
}}}
\item {} {\hyperref[\detokenize{reference/javascript_api:web_editor.BodyManager}]{\sphinxcrossref{
web\_editor.BodyManager
}}}
\item {} {\hyperref[\detokenize{reference/javascript_api:web_editor.context}]{\sphinxcrossref{
web\_editor.context
}}}
\item {} {\hyperref[\detokenize{reference/javascript_api:web_editor.root_widget}]{\sphinxcrossref{
web\_editor.root\_widget
}}}
\item {} {\hyperref[\detokenize{reference/javascript_api:website.content.snippets.animation}]{\sphinxcrossref{
website.content.snippets.animation
}}}
\end{itemize}

\end{description}\end{quote}


\begin{fulllineitems}
\phantomsection\label{\detokenize{reference/javascript_api:website.WebsiteRoot.}}\pysigline{\sphinxbfcode{\sphinxupquote{namespace }}\sphinxbfcode{\sphinxupquote{}}}
\end{fulllineitems}


\end{fulllineitems}

\phantomsection\label{\detokenize{reference/javascript_api:module-web.DomainSelector}}

\begin{fulllineitems}
\phantomsection\label{\detokenize{reference/javascript_api:web.DomainSelector}}\pysigline{\sphinxbfcode{\sphinxupquote{module }}\sphinxbfcode{\sphinxupquote{web.DomainSelector}}}~~\begin{quote}\begin{description}
\item[{Exports}] \leavevmode{\hyperref[\detokenize{reference/javascript_api:web.DomainSelector.DomainSelector}]{\sphinxcrossref{
DomainSelector
}}}
\item[{Depends On}] \leavevmode\begin{itemize}
\item {} {\hyperref[\detokenize{reference/javascript_api:web.Domain}]{\sphinxcrossref{
web.Domain
}}}
\item {} {\hyperref[\detokenize{reference/javascript_api:web.ModelFieldSelector}]{\sphinxcrossref{
web.ModelFieldSelector
}}}
\item {} {\hyperref[\detokenize{reference/javascript_api:web.Widget}]{\sphinxcrossref{
web.Widget
}}}
\item {} {\hyperref[\detokenize{reference/javascript_api:web.core}]{\sphinxcrossref{
web.core
}}}
\item {} {\hyperref[\detokenize{reference/javascript_api:web.datepicker}]{\sphinxcrossref{
web.datepicker
}}}
\item {} {\hyperref[\detokenize{reference/javascript_api:web.field_utils}]{\sphinxcrossref{
web.field\_utils
}}}
\end{itemize}

\end{description}\end{quote}


\begin{fulllineitems}
\phantomsection\label{\detokenize{reference/javascript_api:DomainSelector}}\pysiglinewithargsret{\sphinxbfcode{\sphinxupquote{class }}\sphinxbfcode{\sphinxupquote{DomainSelector}}}{}{}~\begin{quote}\begin{description}
\item[{Extends}] \leavevmode{\hyperref[\detokenize{reference/javascript_api:web.DomainSelector.DomainTree}]{\sphinxcrossref{
DomainTree
}}}
\end{description}\end{quote}

The DomainSelector widget can be used to build prefix char domain. It is the
DomainTree specialization to use to have a fully working widget.

Known limitations:
\begin{itemize}
\item {} 
Some operators like “child\_of”, “parent\_of”, “like”, “not like”,
“=like”, “=ilike” will come only if you use them from demo data or
debug input.

\item {} 
Some kind of domain can not be build right now
e.g (“country\_id”, “in”, {[}1,2,3{]}) but you can insert from debug input.

\end{itemize}


\begin{fulllineitems}
\phantomsection\label{\detokenize{reference/javascript_api:setDomain}}\pysiglinewithargsret{\sphinxbfcode{\sphinxupquote{method }}\sphinxbfcode{\sphinxupquote{setDomain}}}{\emph{domain}}{{ $\rightarrow$ Deferred}}
Changes the internal domain value and forces a reparsing and rerendering.
If the internal domain value was already equal to the given one, this
does nothing.
\begin{quote}\begin{description}
\item[{Parameters}] \leavevmode\begin{itemize}

\sphinxstylestrong{domain} (\sphinxstyleliteralemphasis{\sphinxupquote{Array}}\sphinxstyleemphasis{ or }\sphinxstyleliteralemphasis{\sphinxupquote{string}})
\end{itemize}

\item[{Returns}] \leavevmode
resolved when the rerendering is finished

\item[{Return Type}] \leavevmode
\sphinxstyleliteralemphasis{\sphinxupquote{Deferred}}

\end{description}\end{quote}

\end{fulllineitems}


\end{fulllineitems}



\begin{fulllineitems}
\phantomsection\label{\detokenize{reference/javascript_api:DomainSelector}}\pysiglinewithargsret{\sphinxbfcode{\sphinxupquote{class }}\sphinxbfcode{\sphinxupquote{DomainSelector}}}{}{}~\begin{quote}\begin{description}
\item[{Extends}] \leavevmode{\hyperref[\detokenize{reference/javascript_api:web.DomainSelector.DomainTree}]{\sphinxcrossref{
DomainTree
}}}
\end{description}\end{quote}

The DomainSelector widget can be used to build prefix char domain. It is the
DomainTree specialization to use to have a fully working widget.

Known limitations:
\begin{itemize}
\item {} 
Some operators like “child\_of”, “parent\_of”, “like”, “not like”,
“=like”, “=ilike” will come only if you use them from demo data or
debug input.

\item {} 
Some kind of domain can not be build right now
e.g (“country\_id”, “in”, {[}1,2,3{]}) but you can insert from debug input.

\end{itemize}


\begin{fulllineitems}
\phantomsection\label{\detokenize{reference/javascript_api:setDomain}}\pysiglinewithargsret{\sphinxbfcode{\sphinxupquote{method }}\sphinxbfcode{\sphinxupquote{setDomain}}}{\emph{domain}}{{ $\rightarrow$ Deferred}}
Changes the internal domain value and forces a reparsing and rerendering.
If the internal domain value was already equal to the given one, this
does nothing.
\begin{quote}\begin{description}
\item[{Parameters}] \leavevmode\begin{itemize}

\sphinxstylestrong{domain} (\sphinxstyleliteralemphasis{\sphinxupquote{Array}}\sphinxstyleemphasis{ or }\sphinxstyleliteralemphasis{\sphinxupquote{string}})
\end{itemize}

\item[{Returns}] \leavevmode
resolved when the rerendering is finished

\item[{Return Type}] \leavevmode
\sphinxstyleliteralemphasis{\sphinxupquote{Deferred}}

\end{description}\end{quote}

\end{fulllineitems}


\end{fulllineitems}



\begin{fulllineitems}
\phantomsection\label{\detokenize{reference/javascript_api:DomainLeaf}}\pysiglinewithargsret{\sphinxbfcode{\sphinxupquote{class }}\sphinxbfcode{\sphinxupquote{DomainLeaf}}}{\emph{parent}, \emph{model}, \emph{domain}, \emph{options}}{}~\begin{quote}\begin{description}
\item[{Extends}] \leavevmode{\hyperref[\detokenize{reference/javascript_api:web.DomainSelector.DomainNode}]{\sphinxcrossref{
DomainNode
}}}
\item[{Parameters}] \leavevmode\begin{itemize}

\sphinxstylestrong{parent}

\sphinxstylestrong{model}

\sphinxstylestrong{domain}

\sphinxstylestrong{options}
\end{itemize}

\end{description}\end{quote}

DomainNode which handles a domain which cannot be split in another
subdomains, i.e. composed of a field chain, an operator and a value.


\begin{fulllineitems}
\phantomsection\label{\detokenize{reference/javascript_api:willStart}}\pysiglinewithargsret{\sphinxbfcode{\sphinxupquote{method }}\sphinxbfcode{\sphinxupquote{willStart}}}{}{{ $\rightarrow$ Deferred}}
Prepares the information the rendering of the widget will need by
pre-instantiating its internal field selector widget.
\begin{quote}\begin{description}
\item[{Return Type}] \leavevmode
\sphinxstyleliteralemphasis{\sphinxupquote{Deferred}}

\end{description}\end{quote}

\end{fulllineitems}


\end{fulllineitems}



\begin{fulllineitems}
\phantomsection\label{\detokenize{reference/javascript_api:DomainTree}}\pysiglinewithargsret{\sphinxbfcode{\sphinxupquote{class }}\sphinxbfcode{\sphinxupquote{DomainTree}}}{\emph{parent}, \emph{model}, \emph{domain}, \emph{options}}{}~\begin{quote}\begin{description}
\item[{Extends}] \leavevmode{\hyperref[\detokenize{reference/javascript_api:web.DomainSelector.DomainNode}]{\sphinxcrossref{
DomainNode
}}}
\item[{Parameters}] \leavevmode\begin{itemize}

\sphinxstylestrong{parent}

\sphinxstylestrong{model}

\sphinxstylestrong{domain}

\sphinxstylestrong{options}
\end{itemize}

\end{description}\end{quote}

DomainNode which can handle subdomains (a domain which is composed of
multiple parts). It thus will be composed of other DomainTree instances
and/or leaf parts of a domain (@see DomainLeaf).

\end{fulllineitems}



\begin{fulllineitems}
\phantomsection\label{\detokenize{reference/javascript_api:instantiateNode}}\pysiglinewithargsret{\sphinxbfcode{\sphinxupquote{function }}\sphinxbfcode{\sphinxupquote{instantiateNode}}}{\emph{parent}, \emph{model}, \emph{domain}, \emph{options}}{{ $\rightarrow$ DomainTree\textbar{}DomainLeaf\textbar{}null}}
Instantiates a DomainTree if the given domain contains several parts and a
DomainLeaf if it only contains one part. Returns null otherwise.
\begin{quote}\begin{description}
\item[{Parameters}] \leavevmode\begin{itemize}

\sphinxstylestrong{parent} (\sphinxstyleliteralemphasis{\sphinxupquote{Object}})

\sphinxstylestrong{model} (\sphinxstyleliteralemphasis{\sphinxupquote{string}}) \textendash{} the model name

\sphinxstylestrong{domain} (\sphinxstyleliteralemphasis{\sphinxupquote{Array}}\sphinxstyleemphasis{ or }\sphinxstyleliteralemphasis{\sphinxupquote{string}}) \textendash{} the prefix representation of the domain

\sphinxstylestrong{options} (\sphinxstyleliteralemphasis{\sphinxupquote{Object}}) \textendash{} @see DomainNode.init.options
\end{itemize}

\item[{Return Type}] \leavevmode
{\hyperref[\detokenize{reference/javascript_api:web.DomainSelector.DomainTree}]{\sphinxcrossref{\sphinxstyleliteralemphasis{\sphinxupquote{DomainTree}}}}}\sphinxstyleemphasis{ or }{\hyperref[\detokenize{reference/javascript_api:web.DomainSelector.DomainLeaf}]{\sphinxcrossref{\sphinxstyleliteralemphasis{\sphinxupquote{DomainLeaf}}}}}\sphinxstyleemphasis{ or }null

\end{description}\end{quote}

\end{fulllineitems}



\begin{fulllineitems}
\phantomsection\label{\detokenize{reference/javascript_api:DomainNode}}\pysiglinewithargsret{\sphinxbfcode{\sphinxupquote{class }}\sphinxbfcode{\sphinxupquote{DomainNode}}}{\emph{parent}, \emph{model}, \emph{domain}\sphinxoptional{, \emph{options}}}{}~\begin{quote}\begin{description}
\item[{Extends}] \leavevmode{\hyperref[\detokenize{reference/javascript_api:web.Widget.Widget}]{\sphinxcrossref{
Widget
}}}
\item[{Parameters}] \leavevmode\begin{itemize}

\sphinxstylestrong{parent} (\sphinxstyleliteralemphasis{\sphinxupquote{Object}})

\sphinxstylestrong{model} (\sphinxstyleliteralemphasis{\sphinxupquote{string}}) \textendash{} the model name

\sphinxstylestrong{domain} (\sphinxstyleliteralemphasis{\sphinxupquote{Array}}\sphinxstyleemphasis{ or }\sphinxstyleliteralemphasis{\sphinxupquote{string}}) \textendash{} the prefix representation of the domain

\sphinxstylestrong{options} ({\hyperref[\detokenize{reference/javascript_api:web.DomainSelector.DomainNodeOptions}]{\sphinxcrossref{\sphinxstyleliteralemphasis{\sphinxupquote{DomainNodeOptions}}}}}) \textendash{} an object with possible values:
\end{itemize}

\end{description}\end{quote}

Abstraction for widgets which can represent and allow edition of a domain.


\begin{fulllineitems}
\phantomsection\label{\detokenize{reference/javascript_api:isValid}}\pysiglinewithargsret{\sphinxbfcode{\sphinxupquote{method }}\sphinxbfcode{\sphinxupquote{isValid}}}{}{{ $\rightarrow$ boolean}}
Should return if the node is representing a well-formed domain, whose
field chains properly belong to the associated model.
\begin{quote}\begin{description}
\item[{Return Type}] \leavevmode
\sphinxstyleliteralemphasis{\sphinxupquote{boolean}}

\end{description}\end{quote}

\end{fulllineitems}



\begin{fulllineitems}
\phantomsection\label{\detokenize{reference/javascript_api:getDomain}}\pysiglinewithargsret{\sphinxbfcode{\sphinxupquote{method }}\sphinxbfcode{\sphinxupquote{getDomain}}}{}{{ $\rightarrow$ Array}}
Should return the prefix domain the widget is currently representing
(an array).
\begin{quote}\begin{description}
\item[{Return Type}] \leavevmode
\sphinxstyleliteralemphasis{\sphinxupquote{Array}}

\end{description}\end{quote}

\end{fulllineitems}



\begin{fulllineitems}
\phantomsection\label{\detokenize{reference/javascript_api:DomainNodeOptions}}\pysiglinewithargsret{\sphinxbfcode{\sphinxupquote{class }}\sphinxbfcode{\sphinxupquote{DomainNodeOptions}}}{}{}
an object with possible values:


\begin{fulllineitems}
\phantomsection\label{\detokenize{reference/javascript_api:readonly}}\pysigline{\sphinxbfcode{\sphinxupquote{attribute }}\sphinxbfcode{\sphinxupquote{readonly}} boolean}
true if is readonly

\end{fulllineitems}



\begin{fulllineitems}
\phantomsection\label{\detokenize{reference/javascript_api:operators}}\pysigline{\sphinxbfcode{\sphinxupquote{attribute }}\sphinxbfcode{\sphinxupquote{operators}} string{[}{]}}
a list of available operators (null = all of supported ones)

\end{fulllineitems}



\begin{fulllineitems}
\phantomsection\label{\detokenize{reference/javascript_api:debugMode}}\pysigline{\sphinxbfcode{\sphinxupquote{attribute }}\sphinxbfcode{\sphinxupquote{debugMode}} boolean}
true if should be in debug

\end{fulllineitems}


\end{fulllineitems}


\end{fulllineitems}


\end{fulllineitems}

\phantomsection\label{\detokenize{reference/javascript_api:module-website.content.lazy_template_call}}

\begin{fulllineitems}
\phantomsection\label{\detokenize{reference/javascript_api:website.content.lazy_template_call}}\pysigline{\sphinxbfcode{\sphinxupquote{module }}\sphinxbfcode{\sphinxupquote{website.content.lazy\_template\_call}}}~~\begin{quote}\begin{description}
\item[{Exports}] \leavevmode{\hyperref[\detokenize{reference/javascript_api:website.content.lazy_template_call.LazyTemplateRenderer}]{\sphinxcrossref{
LazyTemplateRenderer
}}}
\item[{Depends On}] \leavevmode\begin{itemize}
\item {} {\hyperref[\detokenize{reference/javascript_api:web.Widget}]{\sphinxcrossref{
web.Widget
}}}
\item {} {\hyperref[\detokenize{reference/javascript_api:website.WebsiteRoot}]{\sphinxcrossref{
website.WebsiteRoot
}}}
\end{itemize}

\end{description}\end{quote}


\begin{fulllineitems}
\phantomsection\label{\detokenize{reference/javascript_api:LazyTemplateRenderer}}\pysiglinewithargsret{\sphinxbfcode{\sphinxupquote{class }}\sphinxbfcode{\sphinxupquote{LazyTemplateRenderer}}}{}{}~\begin{quote}\begin{description}
\item[{Extends}] \leavevmode{\hyperref[\detokenize{reference/javascript_api:web.Widget.Widget}]{\sphinxcrossref{
Widget
}}}
\end{description}\end{quote}


\begin{fulllineitems}
\phantomsection\label{\detokenize{reference/javascript_api:start}}\pysiglinewithargsret{\sphinxbfcode{\sphinxupquote{method }}\sphinxbfcode{\sphinxupquote{start}}}{}{}
Lazy replaces the \sphinxcode{\sphinxupquote{{[}data-oe-call{]}}} elements by their corresponding
template content.

\end{fulllineitems}


\end{fulllineitems}


\end{fulllineitems}

\phantomsection\label{\detokenize{reference/javascript_api:module-web.GraphModel}}

\begin{fulllineitems}
\phantomsection\label{\detokenize{reference/javascript_api:web.GraphModel}}\pysigline{\sphinxbfcode{\sphinxupquote{module }}\sphinxbfcode{\sphinxupquote{web.GraphModel}}}~~\begin{quote}\begin{description}
\item[{Exports}] \leavevmode{\hyperref[\detokenize{reference/javascript_api:web.GraphModel.}]{\sphinxcrossref{
\textless{}anonymous\textgreater{}
}}}
\item[{Depends On}] \leavevmode\begin{itemize}
\item {} {\hyperref[\detokenize{reference/javascript_api:web.AbstractModel}]{\sphinxcrossref{
web.AbstractModel
}}}
\item {} {\hyperref[\detokenize{reference/javascript_api:web.core}]{\sphinxcrossref{
web.core
}}}
\end{itemize}

\end{description}\end{quote}


\begin{fulllineitems}
\phantomsection\label{\detokenize{reference/javascript_api:web.GraphModel.}}\pysiglinewithargsret{\sphinxbfcode{\sphinxupquote{class }}\sphinxbfcode{\sphinxupquote{}}}{\emph{parent}}{}~\begin{quote}\begin{description}
\item[{Extends}] \leavevmode{\hyperref[\detokenize{reference/javascript_api:web.AbstractModel.AbstractModel}]{\sphinxcrossref{
AbstractModel
}}}
\item[{Parameters}] \leavevmode\begin{itemize}

\sphinxstylestrong{parent} ({\hyperref[\detokenize{reference/javascript_api:Widget}]{\sphinxcrossref{\sphinxstyleliteralemphasis{\sphinxupquote{Widget}}}}})
\end{itemize}

\end{description}\end{quote}


\begin{fulllineitems}
\phantomsection\label{\detokenize{reference/javascript_api:get}}\pysiglinewithargsret{\sphinxbfcode{\sphinxupquote{function }}\sphinxbfcode{\sphinxupquote{get}}}{}{{ $\rightarrow$ Object}}
We defend against outside modifications by extending the chart data. It
may be overkill.
\begin{quote}\begin{description}
\item[{Return Type}] \leavevmode
\sphinxstyleliteralemphasis{\sphinxupquote{Object}}

\end{description}\end{quote}

\end{fulllineitems}



\begin{fulllineitems}
\phantomsection\label{\detokenize{reference/javascript_api:load}}\pysiglinewithargsret{\sphinxbfcode{\sphinxupquote{function }}\sphinxbfcode{\sphinxupquote{load}}}{\emph{params}}{{ $\rightarrow$ Deferred}}
Initial loading.
\begin{quote}\begin{description}
\item[{Parameters}] \leavevmode\begin{itemize}

\sphinxstylestrong{params} ({\hyperref[\detokenize{reference/javascript_api:web.GraphModel.LoadParams}]{\sphinxcrossref{\sphinxstyleliteralemphasis{\sphinxupquote{LoadParams}}}}})
\end{itemize}

\item[{Returns}] \leavevmode
The deferred does not return a handle, we don’t need
  to keep track of various entities.

\item[{Return Type}] \leavevmode
\sphinxstyleliteralemphasis{\sphinxupquote{Deferred}}

\end{description}\end{quote}


\begin{fulllineitems}
\phantomsection\label{\detokenize{reference/javascript_api:LoadParams}}\pysiglinewithargsret{\sphinxbfcode{\sphinxupquote{class }}\sphinxbfcode{\sphinxupquote{LoadParams}}}{}{}~

\begin{fulllineitems}
\phantomsection\label{\detokenize{reference/javascript_api:mode}}\pysigline{\sphinxbfcode{\sphinxupquote{attribute }}\sphinxbfcode{\sphinxupquote{mode}} string}
one of ‘pie’, ‘bar’, ‘line

\end{fulllineitems}



\begin{fulllineitems}
\phantomsection\label{\detokenize{reference/javascript_api:measure}}\pysigline{\sphinxbfcode{\sphinxupquote{attribute }}\sphinxbfcode{\sphinxupquote{measure}} string}
a valid field name

\end{fulllineitems}



\begin{fulllineitems}
\phantomsection\label{\detokenize{reference/javascript_api:groupBys}}\pysigline{\sphinxbfcode{\sphinxupquote{attribute }}\sphinxbfcode{\sphinxupquote{groupBys}} string{[}{]}}
a list of valid field names

\end{fulllineitems}



\begin{fulllineitems}
\phantomsection\label{\detokenize{reference/javascript_api:context}}\pysigline{\sphinxbfcode{\sphinxupquote{attribute }}\sphinxbfcode{\sphinxupquote{context}} Object}
\end{fulllineitems}



\begin{fulllineitems}
\phantomsection\label{\detokenize{reference/javascript_api:domain}}\pysigline{\sphinxbfcode{\sphinxupquote{attribute }}\sphinxbfcode{\sphinxupquote{domain}} string{[}{]}}
\end{fulllineitems}


\end{fulllineitems}


\end{fulllineitems}



\begin{fulllineitems}
\phantomsection\label{\detokenize{reference/javascript_api:reload}}\pysiglinewithargsret{\sphinxbfcode{\sphinxupquote{function }}\sphinxbfcode{\sphinxupquote{reload}}}{\emph{handle}, \emph{params}}{{ $\rightarrow$ Deferred}}
Reload data.  It is similar to the load function. Note that we ignore the
handle parameter, we always expect our data to be in this.chart object.
\begin{quote}\begin{description}
\item[{Parameters}] \leavevmode\begin{itemize}

\sphinxstylestrong{handle} (\sphinxstyleliteralemphasis{\sphinxupquote{any}}) \textendash{} ignored!

\sphinxstylestrong{params} ({\hyperref[\detokenize{reference/javascript_api:web.GraphModel.ReloadParams}]{\sphinxcrossref{\sphinxstyleliteralemphasis{\sphinxupquote{ReloadParams}}}}})
\end{itemize}

\item[{Return Type}] \leavevmode
\sphinxstyleliteralemphasis{\sphinxupquote{Deferred}}

\end{description}\end{quote}


\begin{fulllineitems}
\phantomsection\label{\detokenize{reference/javascript_api:ReloadParams}}\pysiglinewithargsret{\sphinxbfcode{\sphinxupquote{class }}\sphinxbfcode{\sphinxupquote{ReloadParams}}}{}{}~

\begin{fulllineitems}
\phantomsection\label{\detokenize{reference/javascript_api:domain}}\pysigline{\sphinxbfcode{\sphinxupquote{attribute }}\sphinxbfcode{\sphinxupquote{domain}} string{[}{]}}
\end{fulllineitems}



\begin{fulllineitems}
\phantomsection\label{\detokenize{reference/javascript_api:groupBy}}\pysigline{\sphinxbfcode{\sphinxupquote{attribute }}\sphinxbfcode{\sphinxupquote{groupBy}} string{[}{]}}
\end{fulllineitems}



\begin{fulllineitems}
\phantomsection\label{\detokenize{reference/javascript_api:mode}}\pysigline{\sphinxbfcode{\sphinxupquote{attribute }}\sphinxbfcode{\sphinxupquote{mode}} string}
one of ‘bar’, ‘pie’, ‘line’

\end{fulllineitems}



\begin{fulllineitems}
\phantomsection\label{\detokenize{reference/javascript_api:measure}}\pysigline{\sphinxbfcode{\sphinxupquote{attribute }}\sphinxbfcode{\sphinxupquote{measure}} string}
a valid field name

\end{fulllineitems}


\end{fulllineitems}


\end{fulllineitems}


\end{fulllineitems}


\end{fulllineitems}

\phantomsection\label{\detokenize{reference/javascript_api:module-web.Dialog}}

\begin{fulllineitems}
\phantomsection\label{\detokenize{reference/javascript_api:web.Dialog}}\pysigline{\sphinxbfcode{\sphinxupquote{module }}\sphinxbfcode{\sphinxupquote{web.Dialog}}}~~\begin{quote}\begin{description}
\item[{Exports}] \leavevmode{\hyperref[\detokenize{reference/javascript_api:web.Dialog.Dialog}]{\sphinxcrossref{
Dialog
}}}
\item[{Depends On}] \leavevmode\begin{itemize}
\item {} {\hyperref[\detokenize{reference/javascript_api:web.Widget}]{\sphinxcrossref{
web.Widget
}}}
\item {} {\hyperref[\detokenize{reference/javascript_api:web.core}]{\sphinxcrossref{
web.core
}}}
\item {} {\hyperref[\detokenize{reference/javascript_api:web.dom}]{\sphinxcrossref{
web.dom
}}}
\end{itemize}

\end{description}\end{quote}


\begin{fulllineitems}
\phantomsection\label{\detokenize{reference/javascript_api:Dialog}}\pysiglinewithargsret{\sphinxbfcode{\sphinxupquote{class }}\sphinxbfcode{\sphinxupquote{Dialog}}}{\emph{parent}\sphinxoptional{, \emph{options}}}{}~\begin{quote}\begin{description}
\item[{Extends}] \leavevmode{\hyperref[\detokenize{reference/javascript_api:web.Widget.Widget}]{\sphinxcrossref{
Widget
}}}
\item[{Parameters}] \leavevmode\begin{itemize}

\sphinxstylestrong{parent} ({\hyperref[\detokenize{reference/javascript_api:Widget}]{\sphinxcrossref{\sphinxstyleliteralemphasis{\sphinxupquote{Widget}}}}})

\sphinxstylestrong{options} ({\hyperref[\detokenize{reference/javascript_api:web.Dialog.DialogOptions}]{\sphinxcrossref{\sphinxstyleliteralemphasis{\sphinxupquote{DialogOptions}}}}})
\end{itemize}

\end{description}\end{quote}

A useful class to handle dialogs.
Attributes:
\begin{description}
\item[{\sphinxcode{\sphinxupquote{\$footer}}}] \leavevmode
A jQuery element targeting a dom part where buttons can be added. It
always exists during the lifecycle of the dialog.

\end{description}


\begin{fulllineitems}
\phantomsection\label{\detokenize{reference/javascript_api:willStart}}\pysiglinewithargsret{\sphinxbfcode{\sphinxupquote{method }}\sphinxbfcode{\sphinxupquote{willStart}}}{}{}
Wait for XML dependencies and instantiate the modal structure (except
modal-body).

\end{fulllineitems}



\begin{fulllineitems}
\phantomsection\label{\detokenize{reference/javascript_api:safeConfirm}}\pysiglinewithargsret{\sphinxbfcode{\sphinxupquote{method }}\sphinxbfcode{\sphinxupquote{safeConfirm}}}{\emph{owner}, \emph{message}\sphinxoptional{, \emph{options}}}{{ $\rightarrow$ Dialog}}
Static method to open double confirmation dialog.
\begin{quote}\begin{description}
\item[{Parameters}] \leavevmode\begin{itemize}

\sphinxstylestrong{owner} ({\hyperref[\detokenize{reference/javascript_api:Widget}]{\sphinxcrossref{\sphinxstyleliteralemphasis{\sphinxupquote{Widget}}}}})

\sphinxstylestrong{message} (\sphinxstyleliteralemphasis{\sphinxupquote{string}})

\sphinxstylestrong{options} ({\hyperref[\detokenize{reference/javascript_api:web.Dialog.SafeConfirmOptions}]{\sphinxcrossref{\sphinxstyleliteralemphasis{\sphinxupquote{SafeConfirmOptions}}}}}) \textendash{} @see Dialog.init @see Dialog.confirm
\end{itemize}

\item[{Returns}] \leavevmode
(open() is automatically called)

\item[{Return Type}] \leavevmode
{\hyperref[\detokenize{reference/javascript_api:web.Dialog.Dialog}]{\sphinxcrossref{\sphinxstyleliteralemphasis{\sphinxupquote{Dialog}}}}}

\end{description}\end{quote}


\begin{fulllineitems}
\phantomsection\label{\detokenize{reference/javascript_api:SafeConfirmOptions}}\pysiglinewithargsret{\sphinxbfcode{\sphinxupquote{class }}\sphinxbfcode{\sphinxupquote{SafeConfirmOptions}}}{}{}
@see Dialog.init @see Dialog.confirm


\begin{fulllineitems}
\phantomsection\label{\detokenize{reference/javascript_api:securityLevel}}\pysigline{\sphinxbfcode{\sphinxupquote{attribute }}\sphinxbfcode{\sphinxupquote{securityLevel}} string}
bootstrap color

\end{fulllineitems}



\begin{fulllineitems}
\phantomsection\label{\detokenize{reference/javascript_api:securityMessage}}\pysigline{\sphinxbfcode{\sphinxupquote{attribute }}\sphinxbfcode{\sphinxupquote{securityMessage}} string}
am sure about this”{]}

\end{fulllineitems}


\end{fulllineitems}


\end{fulllineitems}



\begin{fulllineitems}
\phantomsection\label{\detokenize{reference/javascript_api:DialogOptions}}\pysiglinewithargsret{\sphinxbfcode{\sphinxupquote{class }}\sphinxbfcode{\sphinxupquote{DialogOptions}}}{}{}~

\begin{fulllineitems}
\phantomsection\label{\detokenize{reference/javascript_api:title}}\pysigline{\sphinxbfcode{\sphinxupquote{attribute }}\sphinxbfcode{\sphinxupquote{title}} string}
\end{fulllineitems}



\begin{fulllineitems}
\phantomsection\label{\detokenize{reference/javascript_api:subtitle}}\pysigline{\sphinxbfcode{\sphinxupquote{attribute }}\sphinxbfcode{\sphinxupquote{subtitle}} string}
\end{fulllineitems}



\begin{fulllineitems}
\phantomsection\label{\detokenize{reference/javascript_api:size}}\pysigline{\sphinxbfcode{\sphinxupquote{attribute }}\sphinxbfcode{\sphinxupquote{size}} string}
‘large’, ‘medium’ or ‘small’

\end{fulllineitems}



\begin{fulllineitems}
\phantomsection\label{\detokenize{reference/javascript_api:dialogClass}}\pysigline{\sphinxbfcode{\sphinxupquote{attribute }}\sphinxbfcode{\sphinxupquote{dialogClass}} string}
class to add to the modal-body

\end{fulllineitems}



\begin{fulllineitems}
\phantomsection\label{\detokenize{reference/javascript_api:_content}}\pysigline{\sphinxbfcode{\sphinxupquote{attribute }}\sphinxbfcode{\sphinxupquote{\$content}} jQuery}~\begin{description}
\item[{Element which will be the \$el, replace the .modal-body and get the}] \leavevmode
modal-body class

\end{description}

\end{fulllineitems}



\begin{fulllineitems}
\phantomsection\label{\detokenize{reference/javascript_api:buttons}}\pysigline{\sphinxbfcode{\sphinxupquote{attribute }}\sphinxbfcode{\sphinxupquote{buttons}} Object{[}{]}}~\begin{description}
\item[{List of button descriptions. Note: if no buttons, a “ok” primary}] \leavevmode
button is added to allow closing the dialog

\end{description}

\end{fulllineitems}



\begin{fulllineitems}
\phantomsection\label{\detokenize{reference/javascript_api:buttons__.text}}\pysigline{\sphinxbfcode{\sphinxupquote{attribute }}\sphinxbfcode{\sphinxupquote{buttons{[}{]}.text}} string}
\end{fulllineitems}



\begin{fulllineitems}
\phantomsection\label{\detokenize{reference/javascript_api:buttons__.classes}}\pysigline{\sphinxbfcode{\sphinxupquote{attribute }}\sphinxbfcode{\sphinxupquote{buttons{[}{]}.classes}} string}~\begin{description}
\item[{Default to ‘btn-primary’ if only one button, ‘btn-default’}] \leavevmode
otherwise

\end{description}

\end{fulllineitems}



\begin{fulllineitems}
\phantomsection\label{\detokenize{reference/javascript_api:buttons__.close}}\pysigline{\sphinxbfcode{\sphinxupquote{attribute }}\sphinxbfcode{\sphinxupquote{buttons{[}{]}.close}} boolean}
\end{fulllineitems}



\begin{fulllineitems}
\phantomsection\label{\detokenize{reference/javascript_api:buttons__.click}}\pysigline{\sphinxbfcode{\sphinxupquote{attribute }}\sphinxbfcode{\sphinxupquote{buttons{[}{]}.click}} function}
\end{fulllineitems}



\begin{fulllineitems}
\phantomsection\label{\detokenize{reference/javascript_api:buttons__.disabled}}\pysigline{\sphinxbfcode{\sphinxupquote{attribute }}\sphinxbfcode{\sphinxupquote{buttons{[}{]}.disabled}} boolean}
\end{fulllineitems}



\begin{fulllineitems}
\phantomsection\label{\detokenize{reference/javascript_api:technical}}\pysigline{\sphinxbfcode{\sphinxupquote{attribute }}\sphinxbfcode{\sphinxupquote{technical}} boolean}~\begin{description}
\item[{If set to false, the modal will have the standard frontend style}] \leavevmode
(use this for non-editor frontend features)

\end{description}

\end{fulllineitems}


\end{fulllineitems}


\end{fulllineitems}



\begin{fulllineitems}
\phantomsection\label{\detokenize{reference/javascript_api:Dialog}}\pysiglinewithargsret{\sphinxbfcode{\sphinxupquote{class }}\sphinxbfcode{\sphinxupquote{Dialog}}}{\emph{parent}\sphinxoptional{, \emph{options}}}{}~\begin{quote}\begin{description}
\item[{Extends}] \leavevmode{\hyperref[\detokenize{reference/javascript_api:web.Widget.Widget}]{\sphinxcrossref{
Widget
}}}
\item[{Parameters}] \leavevmode\begin{itemize}

\sphinxstylestrong{parent} ({\hyperref[\detokenize{reference/javascript_api:Widget}]{\sphinxcrossref{\sphinxstyleliteralemphasis{\sphinxupquote{Widget}}}}})

\sphinxstylestrong{options} ({\hyperref[\detokenize{reference/javascript_api:web.Dialog.DialogOptions}]{\sphinxcrossref{\sphinxstyleliteralemphasis{\sphinxupquote{DialogOptions}}}}})
\end{itemize}

\end{description}\end{quote}

A useful class to handle dialogs.
Attributes:
\begin{description}
\item[{\sphinxcode{\sphinxupquote{\$footer}}}] \leavevmode
A jQuery element targeting a dom part where buttons can be added. It
always exists during the lifecycle of the dialog.

\end{description}


\begin{fulllineitems}
\phantomsection\label{\detokenize{reference/javascript_api:willStart}}\pysiglinewithargsret{\sphinxbfcode{\sphinxupquote{method }}\sphinxbfcode{\sphinxupquote{willStart}}}{}{}
Wait for XML dependencies and instantiate the modal structure (except
modal-body).

\end{fulllineitems}



\begin{fulllineitems}
\phantomsection\label{\detokenize{reference/javascript_api:safeConfirm}}\pysiglinewithargsret{\sphinxbfcode{\sphinxupquote{method }}\sphinxbfcode{\sphinxupquote{safeConfirm}}}{\emph{owner}, \emph{message}\sphinxoptional{, \emph{options}}}{{ $\rightarrow$ Dialog}}
Static method to open double confirmation dialog.
\begin{quote}\begin{description}
\item[{Parameters}] \leavevmode\begin{itemize}

\sphinxstylestrong{owner} ({\hyperref[\detokenize{reference/javascript_api:Widget}]{\sphinxcrossref{\sphinxstyleliteralemphasis{\sphinxupquote{Widget}}}}})

\sphinxstylestrong{message} (\sphinxstyleliteralemphasis{\sphinxupquote{string}})

\sphinxstylestrong{options} ({\hyperref[\detokenize{reference/javascript_api:web.Dialog.SafeConfirmOptions}]{\sphinxcrossref{\sphinxstyleliteralemphasis{\sphinxupquote{SafeConfirmOptions}}}}}) \textendash{} @see Dialog.init @see Dialog.confirm
\end{itemize}

\item[{Returns}] \leavevmode
(open() is automatically called)

\item[{Return Type}] \leavevmode
{\hyperref[\detokenize{reference/javascript_api:web.Dialog.Dialog}]{\sphinxcrossref{\sphinxstyleliteralemphasis{\sphinxupquote{Dialog}}}}}

\end{description}\end{quote}


\begin{fulllineitems}
\phantomsection\label{\detokenize{reference/javascript_api:SafeConfirmOptions}}\pysiglinewithargsret{\sphinxbfcode{\sphinxupquote{class }}\sphinxbfcode{\sphinxupquote{SafeConfirmOptions}}}{}{}
@see Dialog.init @see Dialog.confirm


\begin{fulllineitems}
\phantomsection\label{\detokenize{reference/javascript_api:securityLevel}}\pysigline{\sphinxbfcode{\sphinxupquote{attribute }}\sphinxbfcode{\sphinxupquote{securityLevel}} string}
bootstrap color

\end{fulllineitems}



\begin{fulllineitems}
\phantomsection\label{\detokenize{reference/javascript_api:securityMessage}}\pysigline{\sphinxbfcode{\sphinxupquote{attribute }}\sphinxbfcode{\sphinxupquote{securityMessage}} string}
am sure about this”{]}

\end{fulllineitems}


\end{fulllineitems}


\end{fulllineitems}



\begin{fulllineitems}
\phantomsection\label{\detokenize{reference/javascript_api:DialogOptions}}\pysiglinewithargsret{\sphinxbfcode{\sphinxupquote{class }}\sphinxbfcode{\sphinxupquote{DialogOptions}}}{}{}~

\begin{fulllineitems}
\phantomsection\label{\detokenize{reference/javascript_api:title}}\pysigline{\sphinxbfcode{\sphinxupquote{attribute }}\sphinxbfcode{\sphinxupquote{title}} string}
\end{fulllineitems}



\begin{fulllineitems}
\phantomsection\label{\detokenize{reference/javascript_api:subtitle}}\pysigline{\sphinxbfcode{\sphinxupquote{attribute }}\sphinxbfcode{\sphinxupquote{subtitle}} string}
\end{fulllineitems}



\begin{fulllineitems}
\phantomsection\label{\detokenize{reference/javascript_api:size}}\pysigline{\sphinxbfcode{\sphinxupquote{attribute }}\sphinxbfcode{\sphinxupquote{size}} string}
‘large’, ‘medium’ or ‘small’

\end{fulllineitems}



\begin{fulllineitems}
\phantomsection\label{\detokenize{reference/javascript_api:dialogClass}}\pysigline{\sphinxbfcode{\sphinxupquote{attribute }}\sphinxbfcode{\sphinxupquote{dialogClass}} string}
class to add to the modal-body

\end{fulllineitems}



\begin{fulllineitems}
\phantomsection\label{\detokenize{reference/javascript_api:_content}}\pysigline{\sphinxbfcode{\sphinxupquote{attribute }}\sphinxbfcode{\sphinxupquote{\$content}} jQuery}~\begin{description}
\item[{Element which will be the \$el, replace the .modal-body and get the}] \leavevmode
modal-body class

\end{description}

\end{fulllineitems}



\begin{fulllineitems}
\phantomsection\label{\detokenize{reference/javascript_api:buttons}}\pysigline{\sphinxbfcode{\sphinxupquote{attribute }}\sphinxbfcode{\sphinxupquote{buttons}} Object{[}{]}}~\begin{description}
\item[{List of button descriptions. Note: if no buttons, a “ok” primary}] \leavevmode
button is added to allow closing the dialog

\end{description}

\end{fulllineitems}



\begin{fulllineitems}
\phantomsection\label{\detokenize{reference/javascript_api:buttons__.text}}\pysigline{\sphinxbfcode{\sphinxupquote{attribute }}\sphinxbfcode{\sphinxupquote{buttons{[}{]}.text}} string}
\end{fulllineitems}



\begin{fulllineitems}
\phantomsection\label{\detokenize{reference/javascript_api:buttons__.classes}}\pysigline{\sphinxbfcode{\sphinxupquote{attribute }}\sphinxbfcode{\sphinxupquote{buttons{[}{]}.classes}} string}~\begin{description}
\item[{Default to ‘btn-primary’ if only one button, ‘btn-default’}] \leavevmode
otherwise

\end{description}

\end{fulllineitems}



\begin{fulllineitems}
\phantomsection\label{\detokenize{reference/javascript_api:buttons__.close}}\pysigline{\sphinxbfcode{\sphinxupquote{attribute }}\sphinxbfcode{\sphinxupquote{buttons{[}{]}.close}} boolean}
\end{fulllineitems}



\begin{fulllineitems}
\phantomsection\label{\detokenize{reference/javascript_api:buttons__.click}}\pysigline{\sphinxbfcode{\sphinxupquote{attribute }}\sphinxbfcode{\sphinxupquote{buttons{[}{]}.click}} function}
\end{fulllineitems}



\begin{fulllineitems}
\phantomsection\label{\detokenize{reference/javascript_api:buttons__.disabled}}\pysigline{\sphinxbfcode{\sphinxupquote{attribute }}\sphinxbfcode{\sphinxupquote{buttons{[}{]}.disabled}} boolean}
\end{fulllineitems}



\begin{fulllineitems}
\phantomsection\label{\detokenize{reference/javascript_api:technical}}\pysigline{\sphinxbfcode{\sphinxupquote{attribute }}\sphinxbfcode{\sphinxupquote{technical}} boolean}~\begin{description}
\item[{If set to false, the modal will have the standard frontend style}] \leavevmode
(use this for non-editor frontend features)

\end{description}

\end{fulllineitems}


\end{fulllineitems}


\end{fulllineitems}


\end{fulllineitems}

\phantomsection\label{\detokenize{reference/javascript_api:module-web.planner}}

\begin{fulllineitems}
\phantomsection\label{\detokenize{reference/javascript_api:web.planner}}\pysigline{\sphinxbfcode{\sphinxupquote{module }}\sphinxbfcode{\sphinxupquote{web.planner}}}~~\begin{quote}\begin{description}
\item[{Exports}] \leavevmode{\hyperref[\detokenize{reference/javascript_api:web.planner.}]{\sphinxcrossref{
\textless{}anonymous\textgreater{}
}}}
\item[{Depends On}] \leavevmode\begin{itemize}
\item {} {\hyperref[\detokenize{reference/javascript_api:web.SystrayMenu}]{\sphinxcrossref{
web.SystrayMenu
}}}
\item {} {\hyperref[\detokenize{reference/javascript_api:web.core}]{\sphinxcrossref{
web.core
}}}
\item {} {\hyperref[\detokenize{reference/javascript_api:web.planner.common}]{\sphinxcrossref{
web.planner.common
}}}
\item {} {\hyperref[\detokenize{reference/javascript_api:web.session}]{\sphinxcrossref{
web.session
}}}
\end{itemize}

\end{description}\end{quote}


\begin{fulllineitems}
\phantomsection\label{\detokenize{reference/javascript_api:web.planner.}}\pysigline{\sphinxbfcode{\sphinxupquote{namespace }}\sphinxbfcode{\sphinxupquote{}}}
\end{fulllineitems}


\end{fulllineitems}

\phantomsection\label{\detokenize{reference/javascript_api:module-barcodes.field}}

\begin{fulllineitems}
\phantomsection\label{\detokenize{reference/javascript_api:barcodes.field}}\pysigline{\sphinxbfcode{\sphinxupquote{module }}\sphinxbfcode{\sphinxupquote{barcodes.field}}}~~\begin{quote}\begin{description}
\item[{Exports}] \leavevmode{\hyperref[\detokenize{reference/javascript_api:barcodes.field.}]{\sphinxcrossref{
\textless{}anonymous\textgreater{}
}}}
\item[{Depends On}] \leavevmode\begin{itemize}
\item {} {\hyperref[\detokenize{reference/javascript_api:web.AbstractField}]{\sphinxcrossref{
web.AbstractField
}}}
\item {} {\hyperref[\detokenize{reference/javascript_api:web.basic_fields}]{\sphinxcrossref{
web.basic\_fields
}}}
\item {} {\hyperref[\detokenize{reference/javascript_api:web.field_registry}]{\sphinxcrossref{
web.field\_registry
}}}
\end{itemize}

\end{description}\end{quote}


\begin{fulllineitems}
\phantomsection\label{\detokenize{reference/javascript_api:barcodes.field.}}\pysigline{\sphinxbfcode{\sphinxupquote{namespace }}\sphinxbfcode{\sphinxupquote{}}}
\end{fulllineitems}


\end{fulllineitems}

\phantomsection\label{\detokenize{reference/javascript_api:module-web.utils}}

\begin{fulllineitems}
\phantomsection\label{\detokenize{reference/javascript_api:web.utils}}\pysigline{\sphinxbfcode{\sphinxupquote{module }}\sphinxbfcode{\sphinxupquote{web.utils}}}~~\begin{quote}\begin{description}
\item[{Exports}] \leavevmode{\hyperref[\detokenize{reference/javascript_api:web.utils.utils}]{\sphinxcrossref{
utils
}}}
\item[{Depends On}] \leavevmode\begin{itemize}
\item {} {\hyperref[\detokenize{reference/javascript_api:web.translation}]{\sphinxcrossref{
web.translation
}}}
\end{itemize}

\end{description}\end{quote}


\begin{fulllineitems}
\phantomsection\label{\detokenize{reference/javascript_api:utils}}\pysigline{\sphinxbfcode{\sphinxupquote{namespace }}\sphinxbfcode{\sphinxupquote{utils}}}~

\begin{fulllineitems}
\phantomsection\label{\detokenize{reference/javascript_api:assert}}\pysiglinewithargsret{\sphinxbfcode{\sphinxupquote{function }}\sphinxbfcode{\sphinxupquote{assert}}}{\emph{bool}}{}
Throws an error if the given condition is not true
\begin{quote}\begin{description}
\item[{Parameters}] \leavevmode\begin{itemize}

\sphinxstylestrong{bool} (\sphinxstyleliteralemphasis{\sphinxupquote{any}})
\end{itemize}

\end{description}\end{quote}

\end{fulllineitems}



\begin{fulllineitems}
\phantomsection\label{\detokenize{reference/javascript_api:binaryToBinsize}}\pysiglinewithargsret{\sphinxbfcode{\sphinxupquote{function }}\sphinxbfcode{\sphinxupquote{binaryToBinsize}}}{\emph{value}}{{ $\rightarrow$ string}}
Check if the value is a bin\_size or not.
If not, compute an approximate size out of the base64 encoded string.
\begin{quote}\begin{description}
\item[{Parameters}] \leavevmode\begin{itemize}

\sphinxstylestrong{value} (\sphinxstyleliteralemphasis{\sphinxupquote{string}}) \textendash{} original format
\end{itemize}

\item[{Returns}] \leavevmode
bin\_size (human-readable)

\item[{Return Type}] \leavevmode
\sphinxstyleliteralemphasis{\sphinxupquote{string}}

\end{description}\end{quote}

\end{fulllineitems}



\begin{fulllineitems}
\phantomsection\label{\detokenize{reference/javascript_api:confine}}\pysiglinewithargsret{\sphinxbfcode{\sphinxupquote{function }}\sphinxbfcode{\sphinxupquote{confine}}}{\sphinxoptional{\emph{val}}\sphinxoptional{, \emph{min}}\sphinxoptional{, \emph{max}}}{{ $\rightarrow$ number}}
Confines a value inside an interval
\begin{quote}\begin{description}
\item[{Parameters}] \leavevmode\begin{itemize}

\sphinxstylestrong{val} (\sphinxstyleliteralemphasis{\sphinxupquote{number}}) \textendash{} the value to confine

\sphinxstylestrong{min} (\sphinxstyleliteralemphasis{\sphinxupquote{number}}) \textendash{} the minimum of the interval

\sphinxstylestrong{max} (\sphinxstyleliteralemphasis{\sphinxupquote{number}}) \textendash{} the maximum of the interval
\end{itemize}

\item[{Returns}] \leavevmode
val if val is in {[}min, max{]}, min if val \textless{} min and max
  otherwise

\item[{Return Type}] \leavevmode
\sphinxstyleliteralemphasis{\sphinxupquote{number}}

\end{description}\end{quote}

\end{fulllineitems}



\begin{fulllineitems}
\phantomsection\label{\detokenize{reference/javascript_api:divmod}}\pysiglinewithargsret{\sphinxbfcode{\sphinxupquote{function }}\sphinxbfcode{\sphinxupquote{divmod}}}{\emph{a}, \emph{b}, \emph{fn}}{}
computes (Math.floor(a/b), a\%b and passes that to the callback.

returns the callback’s result
\begin{quote}\begin{description}
\item[{Parameters}] \leavevmode\begin{itemize}

\sphinxstylestrong{a}

\sphinxstylestrong{b}

\sphinxstylestrong{fn}
\end{itemize}

\end{description}\end{quote}

\end{fulllineitems}



\begin{fulllineitems}
\phantomsection\label{\detokenize{reference/javascript_api:generateID}}\pysiglinewithargsret{\sphinxbfcode{\sphinxupquote{function }}\sphinxbfcode{\sphinxupquote{generateID}}}{}{{ $\rightarrow$ integer}}
Generate a unique numerical ID
\begin{quote}\begin{description}
\item[{Return Type}] \leavevmode
\sphinxstyleliteralemphasis{\sphinxupquote{integer}}

\end{description}\end{quote}

\end{fulllineitems}



\begin{fulllineitems}
\phantomsection\label{\detokenize{reference/javascript_api:get_cookie}}\pysiglinewithargsret{\sphinxbfcode{\sphinxupquote{function }}\sphinxbfcode{\sphinxupquote{get\_cookie}}}{\emph{c\_name}}{{ $\rightarrow$ string}}
Read the cookie described by c\_name
\begin{quote}\begin{description}
\item[{Parameters}] \leavevmode\begin{itemize}

\sphinxstylestrong{c\_name} (\sphinxstyleliteralemphasis{\sphinxupquote{string}})
\end{itemize}

\item[{Return Type}] \leavevmode
\sphinxstyleliteralemphasis{\sphinxupquote{string}}

\end{description}\end{quote}

\end{fulllineitems}



\begin{fulllineitems}
\phantomsection\label{\detokenize{reference/javascript_api:human_number}}\pysiglinewithargsret{\sphinxbfcode{\sphinxupquote{function }}\sphinxbfcode{\sphinxupquote{human\_number}}}{\emph{number}\sphinxoptional{, \emph{decimals}}\sphinxoptional{, \emph{minDigits}}\sphinxoptional{, \emph{formatterCallback}}}{{ $\rightarrow$ string}}
Returns a human readable number (e.g. 34000 -\textgreater{} 34k).
\begin{quote}\begin{description}
\item[{Parameters}] \leavevmode\begin{itemize}

\sphinxstylestrong{number} (\sphinxstyleliteralemphasis{\sphinxupquote{number}})

\sphinxstylestrong{decimals}=\sphinxstyleemphasis{0} (\sphinxstyleliteralemphasis{\sphinxupquote{integer}}) \textendash{} maximum number of decimals to use in human readable representation

\sphinxstylestrong{minDigits}=\sphinxstyleemphasis{1} (\sphinxstyleliteralemphasis{\sphinxupquote{integer}}) \textendash{} the minimum number of digits to preserve when switching to another
       level of thousands (e.g. with a value of ‘2’, 4321 will still be
       represented as 4321 otherwise it will be down to one digit (4k))

\sphinxstylestrong{formatterCallback} (\sphinxstyleliteralemphasis{\sphinxupquote{function}}) \textendash{} a callback to transform the final number before adding the
       thousands symbol (default to adding thousands separators (useful
       if minDigits \textgreater{} 1))
\end{itemize}

\item[{Return Type}] \leavevmode
\sphinxstyleliteralemphasis{\sphinxupquote{string}}

\end{description}\end{quote}

\end{fulllineitems}



\begin{fulllineitems}
\phantomsection\label{\detokenize{reference/javascript_api:human_size}}\pysiglinewithargsret{\sphinxbfcode{\sphinxupquote{function }}\sphinxbfcode{\sphinxupquote{human\_size}}}{\emph{size}}{}
Returns a human readable size
\begin{quote}\begin{description}
\item[{Parameters}] \leavevmode\begin{itemize}

\sphinxstylestrong{size} (\sphinxstyleliteralemphasis{\sphinxupquote{Number}}) \textendash{} number of bytes
\end{itemize}

\end{description}\end{quote}

\end{fulllineitems}



\begin{fulllineitems}
\phantomsection\label{\detokenize{reference/javascript_api:insert_thousand_seps}}\pysiglinewithargsret{\sphinxbfcode{\sphinxupquote{function }}\sphinxbfcode{\sphinxupquote{insert\_thousand\_seps}}}{\emph{num}}{{ $\rightarrow$ String}}
Insert “thousands” separators in the provided number (which is actually
a string)
\begin{quote}\begin{description}
\item[{Parameters}] \leavevmode\begin{itemize}

\sphinxstylestrong{num} (\sphinxstyleliteralemphasis{\sphinxupquote{String}})
\end{itemize}

\item[{Return Type}] \leavevmode
\sphinxstyleliteralemphasis{\sphinxupquote{String}}

\end{description}\end{quote}

\end{fulllineitems}



\begin{fulllineitems}
\phantomsection\label{\detokenize{reference/javascript_api:intersperse}}\pysiglinewithargsret{\sphinxbfcode{\sphinxupquote{function }}\sphinxbfcode{\sphinxupquote{intersperse}}}{\emph{str}, \emph{indices}, \emph{separator}}{{ $\rightarrow$ String}}
Intersperses \sphinxcode{\sphinxupquote{separator}} in \sphinxcode{\sphinxupquote{str}} at the positions indicated by
\sphinxcode{\sphinxupquote{indices}}.

\sphinxcode{\sphinxupquote{indices}} is an array of relative offsets (from the previous insertion
position, starting from the end of the string) at which to insert
\sphinxcode{\sphinxupquote{separator}}.

There are two special values:
\begin{description}
\item[{\sphinxcode{\sphinxupquote{-1}}}] \leavevmode
indicates the insertion should end now

\item[{\sphinxcode{\sphinxupquote{0}}}] \leavevmode
indicates that the previous section pattern should be repeated (until all
of \sphinxcode{\sphinxupquote{str}} is consumed)

\end{description}
\begin{quote}\begin{description}
\item[{Parameters}] \leavevmode\begin{itemize}

\sphinxstylestrong{str} (\sphinxstyleliteralemphasis{\sphinxupquote{String}})

\sphinxstylestrong{indices} (\sphinxstyleliteralemphasis{\sphinxupquote{Array}}\textless{}\sphinxstyleliteralemphasis{\sphinxupquote{Number}}\textgreater{})

\sphinxstylestrong{separator} (\sphinxstyleliteralemphasis{\sphinxupquote{String}})
\end{itemize}

\item[{Return Type}] \leavevmode
\sphinxstyleliteralemphasis{\sphinxupquote{String}}

\end{description}\end{quote}

\end{fulllineitems}



\begin{fulllineitems}
\phantomsection\label{\detokenize{reference/javascript_api:lpad}}\pysiglinewithargsret{\sphinxbfcode{\sphinxupquote{function }}\sphinxbfcode{\sphinxupquote{lpad}}}{\emph{str}, \emph{size}}{{ $\rightarrow$ string}}
Left-pad provided arg 1 with zeroes until reaching size provided by second
argument.
\begin{quote}\begin{description}
\item[{Parameters}] \leavevmode\begin{itemize}

\sphinxstylestrong{str} (\sphinxstyleliteralemphasis{\sphinxupquote{number}}\sphinxstyleemphasis{ or }\sphinxstyleliteralemphasis{\sphinxupquote{string}}) \textendash{} value to pad

\sphinxstylestrong{size} (\sphinxstyleliteralemphasis{\sphinxupquote{number}}) \textendash{} size to reach on the final padded value
\end{itemize}

\item[{Returns}] \leavevmode
padded string

\item[{Return Type}] \leavevmode
\sphinxstyleliteralemphasis{\sphinxupquote{string}}

\end{description}\end{quote}

\end{fulllineitems}



\begin{fulllineitems}
\phantomsection\label{\detokenize{reference/javascript_api:modf}}\pysiglinewithargsret{\sphinxbfcode{\sphinxupquote{function }}\sphinxbfcode{\sphinxupquote{modf}}}{\emph{x}, \emph{fn}}{}
Passes the fractional and integer parts of x to the callback, returns
the callback’s result
\begin{quote}\begin{description}
\item[{Parameters}] \leavevmode\begin{itemize}

\sphinxstylestrong{x}

\sphinxstylestrong{fn}
\end{itemize}

\end{description}\end{quote}

\end{fulllineitems}



\begin{fulllineitems}
\phantomsection\label{\detokenize{reference/javascript_api:round_decimals}}\pysiglinewithargsret{\sphinxbfcode{\sphinxupquote{function }}\sphinxbfcode{\sphinxupquote{round\_decimals}}}{\emph{value}, \emph{decimals}}{}
performs a half up rounding with a fixed amount of decimals, correcting for float loss of precision
See the corresponding float\_round() in server/tools/float\_utils.py for more info
\begin{quote}\begin{description}
\item[{Parameters}] \leavevmode\begin{itemize}

\sphinxstylestrong{value} (\sphinxstyleliteralemphasis{\sphinxupquote{Number}}) \textendash{} the value to be rounded

\sphinxstylestrong{decimals} (\sphinxstyleliteralemphasis{\sphinxupquote{Number}}) \textendash{} the number of decimals. eg: round\_decimals(3.141592,2) -\textgreater{} 3.14
\end{itemize}

\end{description}\end{quote}

\end{fulllineitems}



\begin{fulllineitems}
\phantomsection\label{\detokenize{reference/javascript_api:round_precision}}\pysiglinewithargsret{\sphinxbfcode{\sphinxupquote{function }}\sphinxbfcode{\sphinxupquote{round\_precision}}}{\emph{value}, \emph{precision}}{}
performs a half up rounding with arbitrary precision, correcting for float loss of precision
See the corresponding float\_round() in server/tools/float\_utils.py for more info
\begin{quote}\begin{description}
\item[{Parameters}] \leavevmode\begin{itemize}

\sphinxstylestrong{value} (\sphinxstyleliteralemphasis{\sphinxupquote{number}}) \textendash{} the value to be rounded

\sphinxstylestrong{precision} (\sphinxstyleliteralemphasis{\sphinxupquote{number}}) \textendash{} a precision parameter. eg: 0.01 rounds to two digits.
\end{itemize}

\end{description}\end{quote}

\end{fulllineitems}



\begin{fulllineitems}
\phantomsection\label{\detokenize{reference/javascript_api:set_cookie}}\pysiglinewithargsret{\sphinxbfcode{\sphinxupquote{function }}\sphinxbfcode{\sphinxupquote{set\_cookie}}}{\emph{name}, \emph{value}, \emph{ttl}}{}
Create a cookie
\begin{quote}\begin{description}
\item[{Parameters}] \leavevmode\begin{itemize}

\sphinxstylestrong{name} (\sphinxstyleliteralemphasis{\sphinxupquote{String}}) \textendash{} the name of the cookie

\sphinxstylestrong{value} (\sphinxstyleliteralemphasis{\sphinxupquote{String}}) \textendash{} the value stored in the cookie

\sphinxstylestrong{ttl} (\sphinxstyleliteralemphasis{\sphinxupquote{Integer}}) \textendash{} time to live of the cookie in millis. -1 to erase the cookie.
\end{itemize}

\end{description}\end{quote}

\end{fulllineitems}



\begin{fulllineitems}
\phantomsection\label{\detokenize{reference/javascript_api:stableSort}}\pysiglinewithargsret{\sphinxbfcode{\sphinxupquote{function }}\sphinxbfcode{\sphinxupquote{stableSort}}}{\emph{array}, \emph{iteratee}}{}
Sort an array in place, keeping the initial order for identical values.
\begin{quote}\begin{description}
\item[{Parameters}] \leavevmode\begin{itemize}

\sphinxstylestrong{array} (\sphinxstyleliteralemphasis{\sphinxupquote{Array}})

\sphinxstylestrong{iteratee} (\sphinxstyleliteralemphasis{\sphinxupquote{function}})
\end{itemize}

\end{description}\end{quote}

\end{fulllineitems}



\begin{fulllineitems}
\phantomsection\label{\detokenize{reference/javascript_api:traverse}}\pysiglinewithargsret{\sphinxbfcode{\sphinxupquote{function }}\sphinxbfcode{\sphinxupquote{traverse}}}{\emph{tree}, \emph{f}}{}
Visit a tree of objects, where each children are in an attribute ‘children’.
For each children, we call the callback function given in arguments.
\begin{quote}\begin{description}
\item[{Parameters}] \leavevmode\begin{itemize}

\sphinxstylestrong{tree} (\sphinxstyleliteralemphasis{\sphinxupquote{Object}}) \textendash{} an object describing a tree structure

\sphinxstylestrong{f} (\sphinxstyleliteralemphasis{\sphinxupquote{function}}) \textendash{} a callback
\end{itemize}

\end{description}\end{quote}

\end{fulllineitems}



\begin{fulllineitems}
\phantomsection\label{\detokenize{reference/javascript_api:deepFreeze}}\pysiglinewithargsret{\sphinxbfcode{\sphinxupquote{function }}\sphinxbfcode{\sphinxupquote{deepFreeze}}}{\emph{obj}}{}
Visit a tree of objects and freeze all
\begin{quote}\begin{description}
\item[{Parameters}] \leavevmode\begin{itemize}

\sphinxstylestrong{obj} (\sphinxstyleliteralemphasis{\sphinxupquote{Object}})
\end{itemize}

\end{description}\end{quote}

\end{fulllineitems}


\end{fulllineitems}


\end{fulllineitems}

\phantomsection\label{\detokenize{reference/javascript_api:module-portal.signature_form}}

\begin{fulllineitems}
\phantomsection\label{\detokenize{reference/javascript_api:portal.signature_form}}\pysigline{\sphinxbfcode{\sphinxupquote{module }}\sphinxbfcode{\sphinxupquote{portal.signature\_form}}}~~\begin{quote}\begin{description}
\item[{Exports}] \leavevmode{\hyperref[\detokenize{reference/javascript_api:portal.signature_form.}]{\sphinxcrossref{
\textless{}anonymous\textgreater{}
}}}
\item[{Depends On}] \leavevmode\begin{itemize}
\item {} {\hyperref[\detokenize{reference/javascript_api:web.Widget}]{\sphinxcrossref{
web.Widget
}}}
\item {} {\hyperref[\detokenize{reference/javascript_api:web.ajax}]{\sphinxcrossref{
web.ajax
}}}
\item {} {\hyperref[\detokenize{reference/javascript_api:web.core}]{\sphinxcrossref{
web.core
}}}
\item {} {\hyperref[\detokenize{reference/javascript_api:web.rpc}]{\sphinxcrossref{
web.rpc
}}}
\item {} {\hyperref[\detokenize{reference/javascript_api:web_editor.base}]{\sphinxcrossref{
web\_editor.base
}}}
\end{itemize}

\end{description}\end{quote}


\begin{fulllineitems}
\phantomsection\label{\detokenize{reference/javascript_api:portal.signature_form.}}\pysigline{\sphinxbfcode{\sphinxupquote{namespace }}\sphinxbfcode{\sphinxupquote{}}}
\end{fulllineitems}


\end{fulllineitems}

\phantomsection\label{\detokenize{reference/javascript_api:module-web_diagram.DiagramView}}

\begin{fulllineitems}
\phantomsection\label{\detokenize{reference/javascript_api:web_diagram.DiagramView}}\pysigline{\sphinxbfcode{\sphinxupquote{module }}\sphinxbfcode{\sphinxupquote{web\_diagram.DiagramView}}}~~\begin{quote}\begin{description}
\item[{Exports}] \leavevmode{\hyperref[\detokenize{reference/javascript_api:web_diagram.DiagramView.DiagramView}]{\sphinxcrossref{
DiagramView
}}}
\item[{Depends On}] \leavevmode\begin{itemize}
\item {} {\hyperref[\detokenize{reference/javascript_api:web.BasicView}]{\sphinxcrossref{
web.BasicView
}}}
\item {} {\hyperref[\detokenize{reference/javascript_api:web.core}]{\sphinxcrossref{
web.core
}}}
\item {} {\hyperref[\detokenize{reference/javascript_api:web_diagram.DiagramController}]{\sphinxcrossref{
web\_diagram.DiagramController
}}}
\item {} {\hyperref[\detokenize{reference/javascript_api:web_diagram.DiagramModel}]{\sphinxcrossref{
web\_diagram.DiagramModel
}}}
\item {} {\hyperref[\detokenize{reference/javascript_api:web_diagram.DiagramRenderer}]{\sphinxcrossref{
web\_diagram.DiagramRenderer
}}}
\end{itemize}

\end{description}\end{quote}


\begin{fulllineitems}
\phantomsection\label{\detokenize{reference/javascript_api:DiagramView}}\pysiglinewithargsret{\sphinxbfcode{\sphinxupquote{class }}\sphinxbfcode{\sphinxupquote{DiagramView}}}{\emph{viewInfo}, \emph{params}}{}~\begin{quote}\begin{description}
\item[{Extends}] \leavevmode{\hyperref[\detokenize{reference/javascript_api:web.BasicView.BasicView}]{\sphinxcrossref{
BasicView
}}}
\item[{Parameters}] \leavevmode\begin{itemize}

\sphinxstylestrong{viewInfo} (\sphinxstyleliteralemphasis{\sphinxupquote{Object}})

\sphinxstylestrong{params} (\sphinxstyleliteralemphasis{\sphinxupquote{Object}})
\end{itemize}

\end{description}\end{quote}

Diagram View


\begin{fulllineitems}
\phantomsection\label{\detokenize{reference/javascript_api:getController}}\pysiglinewithargsret{\sphinxbfcode{\sphinxupquote{method }}\sphinxbfcode{\sphinxupquote{getController}}}{}{}
This override is quite tricky: the graph renderer uses Raphael.js to
render itself, so it needs it to be loaded in the window before rendering
However, the raphael.js library is built in such a way that if it detects
that a module system is present, it will try to use it.  So, in that
case, it is not available on window.Raphael.  This means that the diagram
view is then broken.

As a workaround, we simply remove and restore the define function, if
present, while we are loading Raphael.

\end{fulllineitems}


\end{fulllineitems}



\begin{fulllineitems}
\phantomsection\label{\detokenize{reference/javascript_api:DiagramView}}\pysiglinewithargsret{\sphinxbfcode{\sphinxupquote{class }}\sphinxbfcode{\sphinxupquote{DiagramView}}}{\emph{viewInfo}, \emph{params}}{}~\begin{quote}\begin{description}
\item[{Extends}] \leavevmode{\hyperref[\detokenize{reference/javascript_api:web.BasicView.BasicView}]{\sphinxcrossref{
BasicView
}}}
\item[{Parameters}] \leavevmode\begin{itemize}

\sphinxstylestrong{viewInfo} (\sphinxstyleliteralemphasis{\sphinxupquote{Object}})

\sphinxstylestrong{params} (\sphinxstyleliteralemphasis{\sphinxupquote{Object}})
\end{itemize}

\end{description}\end{quote}

Diagram View


\begin{fulllineitems}
\phantomsection\label{\detokenize{reference/javascript_api:getController}}\pysiglinewithargsret{\sphinxbfcode{\sphinxupquote{method }}\sphinxbfcode{\sphinxupquote{getController}}}{}{}
This override is quite tricky: the graph renderer uses Raphael.js to
render itself, so it needs it to be loaded in the window before rendering
However, the raphael.js library is built in such a way that if it detects
that a module system is present, it will try to use it.  So, in that
case, it is not available on window.Raphael.  This means that the diagram
view is then broken.

As a workaround, we simply remove and restore the define function, if
present, while we are loading Raphael.

\end{fulllineitems}


\end{fulllineitems}


\end{fulllineitems}

\phantomsection\label{\detokenize{reference/javascript_api:module-web.GraphController}}

\begin{fulllineitems}
\phantomsection\label{\detokenize{reference/javascript_api:web.GraphController}}\pysigline{\sphinxbfcode{\sphinxupquote{module }}\sphinxbfcode{\sphinxupquote{web.GraphController}}}~~\begin{quote}\begin{description}
\item[{Exports}] \leavevmode{\hyperref[\detokenize{reference/javascript_api:web.GraphController.GraphController}]{\sphinxcrossref{
GraphController
}}}
\item[{Depends On}] \leavevmode\begin{itemize}
\item {} {\hyperref[\detokenize{reference/javascript_api:web.AbstractController}]{\sphinxcrossref{
web.AbstractController
}}}
\item {} {\hyperref[\detokenize{reference/javascript_api:web.core}]{\sphinxcrossref{
web.core
}}}
\end{itemize}

\end{description}\end{quote}


\begin{fulllineitems}
\phantomsection\label{\detokenize{reference/javascript_api:GraphController}}\pysiglinewithargsret{\sphinxbfcode{\sphinxupquote{class }}\sphinxbfcode{\sphinxupquote{GraphController}}}{\emph{parent}, \emph{model}, \emph{renderer}, \emph{params}}{}~\begin{quote}\begin{description}
\item[{Extends}] \leavevmode{\hyperref[\detokenize{reference/javascript_api:web.AbstractController.AbstractController}]{\sphinxcrossref{
AbstractController
}}}
\item[{Parameters}] \leavevmode\begin{itemize}

\sphinxstylestrong{parent} ({\hyperref[\detokenize{reference/javascript_api:Widget}]{\sphinxcrossref{\sphinxstyleliteralemphasis{\sphinxupquote{Widget}}}}})

\sphinxstylestrong{model} (\sphinxstyleliteralemphasis{\sphinxupquote{GraphModel}})

\sphinxstylestrong{renderer} (\sphinxstyleliteralemphasis{\sphinxupquote{GraphRenderer}})

\sphinxstylestrong{params} ({\hyperref[\detokenize{reference/javascript_api:web.GraphController.GraphControllerParams}]{\sphinxcrossref{\sphinxstyleliteralemphasis{\sphinxupquote{GraphControllerParams}}}}})
\end{itemize}

\end{description}\end{quote}


\begin{fulllineitems}
\phantomsection\label{\detokenize{reference/javascript_api:getContext}}\pysiglinewithargsret{\sphinxbfcode{\sphinxupquote{method }}\sphinxbfcode{\sphinxupquote{getContext}}}{}{{ $\rightarrow$ Object}}
Returns the current mode, measure and groupbys, so we can restore the
view when we save the current state in the search view, or when we add it
to the dashboard.
\begin{quote}\begin{description}
\item[{Return Type}] \leavevmode
\sphinxstyleliteralemphasis{\sphinxupquote{Object}}

\end{description}\end{quote}

\end{fulllineitems}



\begin{fulllineitems}
\phantomsection\label{\detokenize{reference/javascript_api:renderButtons}}\pysiglinewithargsret{\sphinxbfcode{\sphinxupquote{method }}\sphinxbfcode{\sphinxupquote{renderButtons}}}{\sphinxoptional{\emph{\$node}}}{}
Render the buttons according to the GraphView.buttons and
add listeners on it.
Set this.\$buttons with the produced jQuery element
\begin{quote}\begin{description}
\item[{Parameters}] \leavevmode\begin{itemize}

\sphinxstylestrong{\$node} (\sphinxstyleliteralemphasis{\sphinxupquote{jQuery}}) \textendash{} a jQuery node where the rendered buttons should
be inserted \$node may be undefined, in which case the GraphView does
nothing
\end{itemize}

\end{description}\end{quote}

\end{fulllineitems}



\begin{fulllineitems}
\phantomsection\label{\detokenize{reference/javascript_api:GraphControllerParams}}\pysiglinewithargsret{\sphinxbfcode{\sphinxupquote{class }}\sphinxbfcode{\sphinxupquote{GraphControllerParams}}}{}{}~

\begin{fulllineitems}
\phantomsection\label{\detokenize{reference/javascript_api:measures}}\pysigline{\sphinxbfcode{\sphinxupquote{attribute }}\sphinxbfcode{\sphinxupquote{measures}} string{[}{]}}
\end{fulllineitems}


\end{fulllineitems}


\end{fulllineitems}


\end{fulllineitems}

\phantomsection\label{\detokenize{reference/javascript_api:module-mail.chat_manager}}

\begin{fulllineitems}
\phantomsection\label{\detokenize{reference/javascript_api:mail.chat_manager}}\pysigline{\sphinxbfcode{\sphinxupquote{module }}\sphinxbfcode{\sphinxupquote{mail.chat\_manager}}}~~\begin{quote}\begin{description}
\item[{Exports}] \leavevmode{\hyperref[\detokenize{reference/javascript_api:mail.chat_manager.chat_manager}]{\sphinxcrossref{
chat\_manager
}}}
\item[{Depends On}] \leavevmode\begin{itemize}
\item {} {\hyperref[\detokenize{reference/javascript_api:bus.bus}]{\sphinxcrossref{
bus.bus
}}}
\item {} {\hyperref[\detokenize{reference/javascript_api:mail.utils}]{\sphinxcrossref{
mail.utils
}}}
\item {} {\hyperref[\detokenize{reference/javascript_api:web.Bus}]{\sphinxcrossref{
web.Bus
}}}
\item {} {\hyperref[\detokenize{reference/javascript_api:web.Class}]{\sphinxcrossref{
web.Class
}}}
\item {} {\hyperref[\detokenize{reference/javascript_api:web.ServicesMixin}]{\sphinxcrossref{
web.ServicesMixin
}}}
\item {} {\hyperref[\detokenize{reference/javascript_api:web.config}]{\sphinxcrossref{
web.config
}}}
\item {} {\hyperref[\detokenize{reference/javascript_api:web.core}]{\sphinxcrossref{
web.core
}}}
\item {} {\hyperref[\detokenize{reference/javascript_api:web.mixins}]{\sphinxcrossref{
web.mixins
}}}
\item {} {\hyperref[\detokenize{reference/javascript_api:web.session}]{\sphinxcrossref{
web.session
}}}
\item {} {\hyperref[\detokenize{reference/javascript_api:web.time}]{\sphinxcrossref{
web.time
}}}
\item {} 
web.web\_client

\end{itemize}

\end{description}\end{quote}


\begin{fulllineitems}
\phantomsection\label{\detokenize{reference/javascript_api:chat_manager}}\pysigline{\sphinxbfcode{\sphinxupquote{object }}\sphinxbfcode{\sphinxupquote{chat\_manager}}\sphinxbfcode{\sphinxupquote{ instance of }}ChatManager}
\end{fulllineitems}


\end{fulllineitems}

\phantomsection\label{\detokenize{reference/javascript_api:module-web.SearchView}}

\begin{fulllineitems}
\phantomsection\label{\detokenize{reference/javascript_api:web.SearchView}}\pysigline{\sphinxbfcode{\sphinxupquote{module }}\sphinxbfcode{\sphinxupquote{web.SearchView}}}~~\begin{quote}\begin{description}
\item[{Exports}] \leavevmode{\hyperref[\detokenize{reference/javascript_api:web.SearchView.SearchView}]{\sphinxcrossref{
SearchView
}}}
\item[{Depends On}] \leavevmode\begin{itemize}
\item {} {\hyperref[\detokenize{reference/javascript_api:web.AutoComplete}]{\sphinxcrossref{
web.AutoComplete
}}}
\item {} {\hyperref[\detokenize{reference/javascript_api:web.FavoriteMenu}]{\sphinxcrossref{
web.FavoriteMenu
}}}
\item {} {\hyperref[\detokenize{reference/javascript_api:web.FilterMenu}]{\sphinxcrossref{
web.FilterMenu
}}}
\item {} {\hyperref[\detokenize{reference/javascript_api:web.GroupByMenu}]{\sphinxcrossref{
web.GroupByMenu
}}}
\item {} {\hyperref[\detokenize{reference/javascript_api:web.Widget}]{\sphinxcrossref{
web.Widget
}}}
\item {} {\hyperref[\detokenize{reference/javascript_api:web.config}]{\sphinxcrossref{
web.config
}}}
\item {} {\hyperref[\detokenize{reference/javascript_api:web.core}]{\sphinxcrossref{
web.core
}}}
\item {} {\hyperref[\detokenize{reference/javascript_api:web.pyeval}]{\sphinxcrossref{
web.pyeval
}}}
\item {} {\hyperref[\detokenize{reference/javascript_api:web.search_inputs}]{\sphinxcrossref{
web.search\_inputs
}}}
\end{itemize}

\end{description}\end{quote}


\begin{fulllineitems}
\phantomsection\label{\detokenize{reference/javascript_api:SearchView}}\pysiglinewithargsret{\sphinxbfcode{\sphinxupquote{class }}\sphinxbfcode{\sphinxupquote{SearchView}}}{\emph{parent}, \emph{dataset}, \emph{fvg}\sphinxoptional{, \emph{options}}}{}~\begin{quote}\begin{description}
\item[{Extends}] \leavevmode{\hyperref[\detokenize{reference/javascript_api:web.Widget.Widget}]{\sphinxcrossref{
Widget
}}}
\item[{Parameters}] \leavevmode\begin{itemize}

\sphinxstylestrong{parent}

\sphinxstylestrong{dataset}

\sphinxstylestrong{fvg}

\sphinxstylestrong{options} ({\hyperref[\detokenize{reference/javascript_api:web.SearchView.SearchViewOptions}]{\sphinxcrossref{\sphinxstyleliteralemphasis{\sphinxupquote{SearchViewOptions}}}}})
\end{itemize}

\end{description}\end{quote}


\begin{fulllineitems}
\phantomsection\label{\detokenize{reference/javascript_api:do_search}}\pysiglinewithargsret{\sphinxbfcode{\sphinxupquote{method }}\sphinxbfcode{\sphinxupquote{do\_search}}}{\sphinxoptional{\emph{\_query}}\sphinxoptional{, \emph{options}}}{}
Performs the search view collection of widget data.

If the collection went well (all fields are valid), then triggers
\sphinxcode{\sphinxupquote{instance.web.SearchView.on\_search()}}.

If at least one field failed its validation, triggers
\sphinxcode{\sphinxupquote{instance.web.SearchView.on\_invalid()}} instead.
\begin{quote}\begin{description}
\item[{Parameters}] \leavevmode\begin{itemize}

\sphinxstylestrong{\_query}

\sphinxstylestrong{options} (\sphinxstyleliteralemphasis{\sphinxupquote{Object}})
\end{itemize}

\end{description}\end{quote}

\end{fulllineitems}



\begin{fulllineitems}
\phantomsection\label{\detokenize{reference/javascript_api:build_search_data}}\pysiglinewithargsret{\sphinxbfcode{\sphinxupquote{method }}\sphinxbfcode{\sphinxupquote{build\_search\_data}}}{}{{ $\rightarrow$ Object}}
Extract search data from the view’s facets.

Result is an object with 3 (own) properties:
\begin{description}
\item[{domains}] \leavevmode
Array of domains

\item[{contexts}] \leavevmode
Array of contexts

\item[{groupbys}] \leavevmode
Array of domains, in groupby order rather than view order

\end{description}
\begin{quote}\begin{description}
\item[{Return Type}] \leavevmode
\sphinxstyleliteralemphasis{\sphinxupquote{Object}}

\end{description}\end{quote}

\end{fulllineitems}



\begin{fulllineitems}
\phantomsection\label{\detokenize{reference/javascript_api:setup_global_completion}}\pysiglinewithargsret{\sphinxbfcode{\sphinxupquote{method }}\sphinxbfcode{\sphinxupquote{setup\_global\_completion}}}{}{}
Sets up search view’s view-wide auto-completion widget

\end{fulllineitems}



\begin{fulllineitems}
\phantomsection\label{\detokenize{reference/javascript_api:complete_global_search}}\pysiglinewithargsret{\sphinxbfcode{\sphinxupquote{method }}\sphinxbfcode{\sphinxupquote{complete\_global\_search}}}{\emph{req}, \emph{resp}}{}
Provide auto-completion result for req.term (an array to \sphinxcode{\sphinxupquote{resp}})
\begin{quote}\begin{description}
\item[{Parameters}] \leavevmode\begin{itemize}

\sphinxstylestrong{req} ({\hyperref[\detokenize{reference/javascript_api:web.SearchView.CompleteGlobalSearchReq}]{\sphinxcrossref{\sphinxstyleliteralemphasis{\sphinxupquote{CompleteGlobalSearchReq}}}}}) \textendash{} request to complete

\sphinxstylestrong{resp} (\sphinxstyleliteralemphasis{\sphinxupquote{Function}}) \textendash{} response callback
\end{itemize}

\end{description}\end{quote}


\begin{fulllineitems}
\phantomsection\label{\detokenize{reference/javascript_api:CompleteGlobalSearchReq}}\pysiglinewithargsret{\sphinxbfcode{\sphinxupquote{class }}\sphinxbfcode{\sphinxupquote{CompleteGlobalSearchReq}}}{}{}
request to complete


\begin{fulllineitems}
\phantomsection\label{\detokenize{reference/javascript_api:term}}\pysigline{\sphinxbfcode{\sphinxupquote{attribute }}\sphinxbfcode{\sphinxupquote{term}} String}
searched term to complete

\end{fulllineitems}


\end{fulllineitems}


\end{fulllineitems}



\begin{fulllineitems}
\phantomsection\label{\detokenize{reference/javascript_api:select_completion}}\pysiglinewithargsret{\sphinxbfcode{\sphinxupquote{function }}\sphinxbfcode{\sphinxupquote{select\_completion}}}{\emph{e}, \emph{ui}}{}
Action to perform in case of selection: create a facet (model)
and add it to the search collection
\begin{quote}\begin{description}
\item[{Parameters}] \leavevmode\begin{itemize}

\sphinxstylestrong{e} (\sphinxstyleliteralemphasis{\sphinxupquote{Object}}) \textendash{} selection event, preventDefault to avoid setting value on object

\sphinxstylestrong{ui} ({\hyperref[\detokenize{reference/javascript_api:web.SearchView.SelectCompletionUi}]{\sphinxcrossref{\sphinxstyleliteralemphasis{\sphinxupquote{SelectCompletionUi}}}}}) \textendash{} selection information
\end{itemize}

\end{description}\end{quote}


\begin{fulllineitems}
\phantomsection\label{\detokenize{reference/javascript_api:SelectCompletionUi}}\pysiglinewithargsret{\sphinxbfcode{\sphinxupquote{class }}\sphinxbfcode{\sphinxupquote{SelectCompletionUi}}}{}{}
selection information


\begin{fulllineitems}
\phantomsection\label{\detokenize{reference/javascript_api:item}}\pysigline{\sphinxbfcode{\sphinxupquote{attribute }}\sphinxbfcode{\sphinxupquote{item}} Object}
selected completion item

\end{fulllineitems}


\end{fulllineitems}


\end{fulllineitems}



\begin{fulllineitems}
\phantomsection\label{\detokenize{reference/javascript_api:renderChangedFacets}}\pysiglinewithargsret{\sphinxbfcode{\sphinxupquote{function }}\sphinxbfcode{\sphinxupquote{renderChangedFacets}}}{\emph{model}, \emph{options}}{}
Call the renderFacets method with the correct arguments.
This is due to the fact that change events are called with two arguments
(model, options) while add, reset and remove events are called with
(collection, model, options) as arguments
\begin{quote}\begin{description}
\item[{Parameters}] \leavevmode\begin{itemize}

\sphinxstylestrong{model}

\sphinxstylestrong{options}
\end{itemize}

\end{description}\end{quote}

\end{fulllineitems}



\begin{fulllineitems}
\phantomsection\label{\detokenize{reference/javascript_api:SearchViewOptions}}\pysiglinewithargsret{\sphinxbfcode{\sphinxupquote{class }}\sphinxbfcode{\sphinxupquote{SearchViewOptions}}}{}{}~

\begin{fulllineitems}
\phantomsection\label{\detokenize{reference/javascript_api:hidden}}\pysigline{\sphinxbfcode{\sphinxupquote{attribute }}\sphinxbfcode{\sphinxupquote{hidden}} Boolean}
hide the search view

\end{fulllineitems}



\begin{fulllineitems}
\phantomsection\label{\detokenize{reference/javascript_api:disable_custom_filters}}\pysigline{\sphinxbfcode{\sphinxupquote{attribute }}\sphinxbfcode{\sphinxupquote{disable\_custom\_filters}} Boolean}
do not load custom filters from ir.filters

\end{fulllineitems}


\end{fulllineitems}


\end{fulllineitems}


\end{fulllineitems}

\phantomsection\label{\detokenize{reference/javascript_api:module-web_editor.BodyManager}}

\begin{fulllineitems}
\phantomsection\label{\detokenize{reference/javascript_api:web_editor.BodyManager}}\pysigline{\sphinxbfcode{\sphinxupquote{module }}\sphinxbfcode{\sphinxupquote{web\_editor.BodyManager}}}~~\begin{quote}\begin{description}
\item[{Exports}] \leavevmode{\hyperref[\detokenize{reference/javascript_api:web_editor.BodyManager.BodyManager}]{\sphinxcrossref{
BodyManager
}}}
\item[{Depends On}] \leavevmode\begin{itemize}
\item {} {\hyperref[\detokenize{reference/javascript_api:web.mixins}]{\sphinxcrossref{
web.mixins
}}}
\item {} {\hyperref[\detokenize{reference/javascript_api:web.session}]{\sphinxcrossref{
web.session
}}}
\item {} {\hyperref[\detokenize{reference/javascript_api:web_editor.root_widget}]{\sphinxcrossref{
web\_editor.root\_widget
}}}
\end{itemize}

\end{description}\end{quote}


\begin{fulllineitems}
\phantomsection\label{\detokenize{reference/javascript_api:BodyManager}}\pysiglinewithargsret{\sphinxbfcode{\sphinxupquote{class }}\sphinxbfcode{\sphinxupquote{BodyManager}}}{}{}~\begin{quote}\begin{description}
\item[{Extends}] \leavevmode{\hyperref[\detokenize{reference/javascript_api:web_editor.root_widget.RootWidget}]{\sphinxcrossref{
RootWidget
}}}
\item[{Mixes}] \leavevmode\begin{itemize}
\item {} 
ServiceProvider

\end{itemize}

\end{description}\end{quote}

Element which is designed to be unique and that will be the top-most element
in the widget hierarchy. So, all other widgets will be indirectly linked to
this Class instance. Its main role will be to retrieve RPC demands from its
children and handle them.

\end{fulllineitems}



\begin{fulllineitems}
\phantomsection\label{\detokenize{reference/javascript_api:BodyManager}}\pysiglinewithargsret{\sphinxbfcode{\sphinxupquote{class }}\sphinxbfcode{\sphinxupquote{BodyManager}}}{}{}~\begin{quote}\begin{description}
\item[{Extends}] \leavevmode{\hyperref[\detokenize{reference/javascript_api:web_editor.root_widget.RootWidget}]{\sphinxcrossref{
RootWidget
}}}
\item[{Mixes}] \leavevmode\begin{itemize}
\item {} 
ServiceProvider

\end{itemize}

\end{description}\end{quote}

Element which is designed to be unique and that will be the top-most element
in the widget hierarchy. So, all other widgets will be indirectly linked to
this Class instance. Its main role will be to retrieve RPC demands from its
children and handle them.

\end{fulllineitems}


\end{fulllineitems}

\phantomsection\label{\detokenize{reference/javascript_api:module-web.OrgChart}}

\begin{fulllineitems}
\phantomsection\label{\detokenize{reference/javascript_api:web.OrgChart}}\pysigline{\sphinxbfcode{\sphinxupquote{module }}\sphinxbfcode{\sphinxupquote{web.OrgChart}}}~~\begin{quote}\begin{description}
\item[{Exports}] \leavevmode{\hyperref[\detokenize{reference/javascript_api:web.OrgChart.FieldOrgChart}]{\sphinxcrossref{
FieldOrgChart
}}}
\item[{Depends On}] \leavevmode\begin{itemize}
\item {} {\hyperref[\detokenize{reference/javascript_api:web.AbstractField}]{\sphinxcrossref{
web.AbstractField
}}}
\item {} {\hyperref[\detokenize{reference/javascript_api:web.concurrency}]{\sphinxcrossref{
web.concurrency
}}}
\item {} {\hyperref[\detokenize{reference/javascript_api:web.core}]{\sphinxcrossref{
web.core
}}}
\item {} {\hyperref[\detokenize{reference/javascript_api:web.field_registry}]{\sphinxcrossref{
web.field\_registry
}}}
\end{itemize}

\end{description}\end{quote}


\begin{fulllineitems}
\phantomsection\label{\detokenize{reference/javascript_api:FieldOrgChart}}\pysiglinewithargsret{\sphinxbfcode{\sphinxupquote{class }}\sphinxbfcode{\sphinxupquote{FieldOrgChart}}}{}{}~\begin{quote}\begin{description}
\item[{Extends}] \leavevmode{\hyperref[\detokenize{reference/javascript_api:web.AbstractField.AbstractField}]{\sphinxcrossref{
AbstractField
}}}
\end{description}\end{quote}

\end{fulllineitems}


\end{fulllineitems}

\phantomsection\label{\detokenize{reference/javascript_api:module-web.GraphRenderer}}

\begin{fulllineitems}
\phantomsection\label{\detokenize{reference/javascript_api:web.GraphRenderer}}\pysigline{\sphinxbfcode{\sphinxupquote{module }}\sphinxbfcode{\sphinxupquote{web.GraphRenderer}}}~~\begin{quote}\begin{description}
\item[{Exports}] \leavevmode{\hyperref[\detokenize{reference/javascript_api:web.GraphRenderer.}]{\sphinxcrossref{
\textless{}anonymous\textgreater{}
}}}
\item[{Depends On}] \leavevmode\begin{itemize}
\item {} {\hyperref[\detokenize{reference/javascript_api:web.AbstractRenderer}]{\sphinxcrossref{
web.AbstractRenderer
}}}
\item {} {\hyperref[\detokenize{reference/javascript_api:web.config}]{\sphinxcrossref{
web.config
}}}
\item {} {\hyperref[\detokenize{reference/javascript_api:web.core}]{\sphinxcrossref{
web.core
}}}
\item {} {\hyperref[\detokenize{reference/javascript_api:web.field_utils}]{\sphinxcrossref{
web.field\_utils
}}}
\end{itemize}

\end{description}\end{quote}


\begin{fulllineitems}
\phantomsection\label{\detokenize{reference/javascript_api:web.GraphRenderer.}}\pysiglinewithargsret{\sphinxbfcode{\sphinxupquote{class }}\sphinxbfcode{\sphinxupquote{}}}{\emph{parent}, \emph{state}, \emph{params}}{}~\begin{quote}\begin{description}
\item[{Extends}] \leavevmode{\hyperref[\detokenize{reference/javascript_api:web.AbstractRenderer.AbstractRenderer}]{\sphinxcrossref{
AbstractRenderer
}}}
\item[{Parameters}] \leavevmode\begin{itemize}

\sphinxstylestrong{parent} ({\hyperref[\detokenize{reference/javascript_api:Widget}]{\sphinxcrossref{\sphinxstyleliteralemphasis{\sphinxupquote{Widget}}}}})

\sphinxstylestrong{state} (\sphinxstyleliteralemphasis{\sphinxupquote{Object}})

\sphinxstylestrong{params} ({\hyperref[\detokenize{reference/javascript_api:web.GraphRenderer.Params}]{\sphinxcrossref{\sphinxstyleliteralemphasis{\sphinxupquote{Params}}}}})
\end{itemize}

\end{description}\end{quote}


\begin{fulllineitems}
\phantomsection\label{\detokenize{reference/javascript_api:on_attach_callback}}\pysiglinewithargsret{\sphinxbfcode{\sphinxupquote{function }}\sphinxbfcode{\sphinxupquote{on\_attach\_callback}}}{}{}
The graph view uses the nv(d3) lib to render the graph. This lib requires
that the rendering is done directly into the DOM (so that it can correctly
compute positions). However, the views are always rendered in fragments,
and appended to the DOM once ready (to prevent them from flickering). We
here use the on\_attach\_callback hook, called when the widget is attached
to the DOM, to perform the rendering. This ensures that the rendering is
always done in the DOM.

\end{fulllineitems}



\begin{fulllineitems}
\phantomsection\label{\detokenize{reference/javascript_api:Params}}\pysiglinewithargsret{\sphinxbfcode{\sphinxupquote{class }}\sphinxbfcode{\sphinxupquote{Params}}}{}{}~

\begin{fulllineitems}
\phantomsection\label{\detokenize{reference/javascript_api:stacked}}\pysigline{\sphinxbfcode{\sphinxupquote{attribute }}\sphinxbfcode{\sphinxupquote{stacked}} boolean}
\end{fulllineitems}


\end{fulllineitems}


\end{fulllineitems}


\end{fulllineitems}

\phantomsection\label{\detokenize{reference/javascript_api:module-web.special_fields}}

\begin{fulllineitems}
\phantomsection\label{\detokenize{reference/javascript_api:web.special_fields}}\pysigline{\sphinxbfcode{\sphinxupquote{module }}\sphinxbfcode{\sphinxupquote{web.special\_fields}}}~~\begin{quote}\begin{description}
\item[{Exports}] \leavevmode{\hyperref[\detokenize{reference/javascript_api:web.special_fields.}]{\sphinxcrossref{
\textless{}anonymous\textgreater{}
}}}
\item[{Depends On}] \leavevmode\begin{itemize}
\item {} {\hyperref[\detokenize{reference/javascript_api:web.core}]{\sphinxcrossref{
web.core
}}}
\item {} {\hyperref[\detokenize{reference/javascript_api:web.field_utils}]{\sphinxcrossref{
web.field\_utils
}}}
\item {} {\hyperref[\detokenize{reference/javascript_api:web.relational_fields}]{\sphinxcrossref{
web.relational\_fields
}}}
\end{itemize}

\end{description}\end{quote}


\begin{fulllineitems}
\phantomsection\label{\detokenize{reference/javascript_api:web.special_fields.}}\pysigline{\sphinxbfcode{\sphinxupquote{namespace }}\sphinxbfcode{\sphinxupquote{}}}~

\begin{fulllineitems}
\phantomsection\label{\detokenize{reference/javascript_api:FieldTimezoneMismatch}}\pysiglinewithargsret{\sphinxbfcode{\sphinxupquote{class }}\sphinxbfcode{\sphinxupquote{FieldTimezoneMismatch}}}{}{}~\begin{quote}\begin{description}
\item[{Extends}] \leavevmode{\hyperref[\detokenize{reference/javascript_reference:web.relational_fields.FieldSelection}]{\sphinxcrossref{
FieldSelection
}}}
\end{description}\end{quote}

This widget is intended to display a warning near a label of a ‘timezone’ field
indicating if the browser timezone is identical (or not) to the selected timezone.
This widget depends on a field given with the param ‘tz\_offset\_field’, which contains
the time difference between UTC time and local time, in minutes.

\end{fulllineitems}


\end{fulllineitems}



\begin{fulllineitems}
\phantomsection\label{\detokenize{reference/javascript_api:FieldTimezoneMismatch}}\pysiglinewithargsret{\sphinxbfcode{\sphinxupquote{class }}\sphinxbfcode{\sphinxupquote{FieldTimezoneMismatch}}}{}{}~\begin{quote}\begin{description}
\item[{Extends}] \leavevmode{\hyperref[\detokenize{reference/javascript_reference:web.relational_fields.FieldSelection}]{\sphinxcrossref{
FieldSelection
}}}
\end{description}\end{quote}

This widget is intended to display a warning near a label of a ‘timezone’ field
indicating if the browser timezone is identical (or not) to the selected timezone.
This widget depends on a field given with the param ‘tz\_offset\_field’, which contains
the time difference between UTC time and local time, in minutes.

\end{fulllineitems}


\end{fulllineitems}

\phantomsection\label{\detokenize{reference/javascript_api:module-web_editor.summernote}}

\begin{fulllineitems}
\phantomsection\label{\detokenize{reference/javascript_api:web_editor.summernote}}\pysigline{\sphinxbfcode{\sphinxupquote{module }}\sphinxbfcode{\sphinxupquote{web\_editor.summernote}}}~~\begin{quote}\begin{description}
\item[{Exports}] \leavevmode
summernote

\item[{Depends On}] \leavevmode\begin{itemize}
\item {} {\hyperref[\detokenize{reference/javascript_api:web.core}]{\sphinxcrossref{
web.core
}}}
\end{itemize}

\end{description}\end{quote}

\end{fulllineitems}

\phantomsection\label{\detokenize{reference/javascript_api:module-website.mobile}}

\begin{fulllineitems}
\phantomsection\label{\detokenize{reference/javascript_api:website.mobile}}\pysigline{\sphinxbfcode{\sphinxupquote{module }}\sphinxbfcode{\sphinxupquote{website.mobile}}}~~\begin{quote}\begin{description}
\item[{Exports}] \leavevmode{\hyperref[\detokenize{reference/javascript_api:website.mobile.}]{\sphinxcrossref{
\textless{}anonymous\textgreater{}
}}}
\item[{Depends On}] \leavevmode\begin{itemize}
\item {} {\hyperref[\detokenize{reference/javascript_api:web.Dialog}]{\sphinxcrossref{
web.Dialog
}}}
\item {} {\hyperref[\detokenize{reference/javascript_api:web.core}]{\sphinxcrossref{
web.core
}}}
\item {} {\hyperref[\detokenize{reference/javascript_api:website.navbar}]{\sphinxcrossref{
website.navbar
}}}
\end{itemize}

\end{description}\end{quote}


\begin{fulllineitems}
\phantomsection\label{\detokenize{reference/javascript_api:website.mobile.}}\pysigline{\sphinxbfcode{\sphinxupquote{namespace }}\sphinxbfcode{\sphinxupquote{}}}
\end{fulllineitems}


\end{fulllineitems}

\phantomsection\label{\detokenize{reference/javascript_api:module-web.AbstractView}}

\begin{fulllineitems}
\phantomsection\label{\detokenize{reference/javascript_api:web.AbstractView}}\pysigline{\sphinxbfcode{\sphinxupquote{module }}\sphinxbfcode{\sphinxupquote{web.AbstractView}}}~~\begin{quote}\begin{description}
\item[{Exports}] \leavevmode{\hyperref[\detokenize{reference/javascript_api:web.AbstractView.AbstractView}]{\sphinxcrossref{
AbstractView
}}}
\item[{Depends On}] \leavevmode\begin{itemize}
\item {} {\hyperref[\detokenize{reference/javascript_api:web.AbstractController}]{\sphinxcrossref{
web.AbstractController
}}}
\item {} {\hyperref[\detokenize{reference/javascript_api:web.AbstractModel}]{\sphinxcrossref{
web.AbstractModel
}}}
\item {} {\hyperref[\detokenize{reference/javascript_api:web.AbstractRenderer}]{\sphinxcrossref{
web.AbstractRenderer
}}}
\item {} {\hyperref[\detokenize{reference/javascript_api:web.Class}]{\sphinxcrossref{
web.Class
}}}
\item {} {\hyperref[\detokenize{reference/javascript_api:web.Context}]{\sphinxcrossref{
web.Context
}}}
\item {} {\hyperref[\detokenize{reference/javascript_api:web.ajax}]{\sphinxcrossref{
web.ajax
}}}
\end{itemize}

\end{description}\end{quote}


\begin{fulllineitems}
\phantomsection\label{\detokenize{reference/javascript_api:AbstractView}}\pysiglinewithargsret{\sphinxbfcode{\sphinxupquote{class }}\sphinxbfcode{\sphinxupquote{AbstractView}}}{\emph{viewInfo}, \emph{params}}{}~\begin{quote}\begin{description}
\item[{Extends}] \leavevmode{\hyperref[\detokenize{reference/javascript_api:web.Class.Class}]{\sphinxcrossref{
Class
}}}
\item[{Parameters}] \leavevmode\begin{itemize}

\sphinxstylestrong{viewInfo} ({\hyperref[\detokenize{reference/javascript_api:web.AbstractView.AbstractViewViewInfo}]{\sphinxcrossref{\sphinxstyleliteralemphasis{\sphinxupquote{AbstractViewViewInfo}}}}})

\sphinxstylestrong{params} ({\hyperref[\detokenize{reference/javascript_api:web.AbstractView.AbstractViewParams}]{\sphinxcrossref{\sphinxstyleliteralemphasis{\sphinxupquote{AbstractViewParams}}}}})
\end{itemize}

\end{description}\end{quote}


\begin{fulllineitems}
\phantomsection\label{\detokenize{reference/javascript_api:getController}}\pysiglinewithargsret{\sphinxbfcode{\sphinxupquote{method }}\sphinxbfcode{\sphinxupquote{getController}}}{\emph{parent}}{{ $\rightarrow$ Deferred}}
Main method of the view class. Create a controller, and make sure that
data and libraries are loaded.

There is a unusual thing going in this method with parents: we create
renderer/model with parent as parent, then we have to reassign them at
the end to make sure that we have the proper relationships.  This is
necessary to solve the problem that the controller need the model and the
renderer to be instantiated, but the model need a parent to be able to
load itself, and the renderer needs the data in its constructor.
\begin{quote}\begin{description}
\item[{Parameters}] \leavevmode\begin{itemize}

\sphinxstylestrong{parent} ({\hyperref[\detokenize{reference/javascript_api:Widget}]{\sphinxcrossref{\sphinxstyleliteralemphasis{\sphinxupquote{Widget}}}}}) \textendash{} The parent of the resulting Controller (most
     likely a view manager)
\end{itemize}

\item[{Returns}] \leavevmode
The deferred resolves to a controller

\item[{Return Type}] \leavevmode
\sphinxstyleliteralemphasis{\sphinxupquote{Deferred}}

\end{description}\end{quote}

\end{fulllineitems}



\begin{fulllineitems}
\phantomsection\label{\detokenize{reference/javascript_api:getModel}}\pysiglinewithargsret{\sphinxbfcode{\sphinxupquote{method }}\sphinxbfcode{\sphinxupquote{getModel}}}{\emph{parent}}{{ $\rightarrow$ Object}}
Returns the view model or create an instance of it if none
\begin{quote}\begin{description}
\item[{Parameters}] \leavevmode\begin{itemize}

\sphinxstylestrong{parent} ({\hyperref[\detokenize{reference/javascript_api:Widget}]{\sphinxcrossref{\sphinxstyleliteralemphasis{\sphinxupquote{Widget}}}}}) \textendash{} the parent of the model, if it has to be created
\end{itemize}

\item[{Returns}] \leavevmode
instance of the view model

\item[{Return Type}] \leavevmode
\sphinxstyleliteralemphasis{\sphinxupquote{Object}}

\end{description}\end{quote}

\end{fulllineitems}



\begin{fulllineitems}
\phantomsection\label{\detokenize{reference/javascript_api:getRenderer}}\pysiglinewithargsret{\sphinxbfcode{\sphinxupquote{method }}\sphinxbfcode{\sphinxupquote{getRenderer}}}{\emph{parent}, \emph{state}}{{ $\rightarrow$ Object}}
Returns the a new view renderer instance
\begin{quote}\begin{description}
\item[{Parameters}] \leavevmode\begin{itemize}

\sphinxstylestrong{parent} ({\hyperref[\detokenize{reference/javascript_api:Widget}]{\sphinxcrossref{\sphinxstyleliteralemphasis{\sphinxupquote{Widget}}}}}) \textendash{} the parent of the model, if it has to be created

\sphinxstylestrong{state} (\sphinxstyleliteralemphasis{\sphinxupquote{Object}}) \textendash{} the information related to the rendered view
\end{itemize}

\item[{Returns}] \leavevmode
instance of the view renderer

\item[{Return Type}] \leavevmode
\sphinxstyleliteralemphasis{\sphinxupquote{Object}}

\end{description}\end{quote}

\end{fulllineitems}



\begin{fulllineitems}
\phantomsection\label{\detokenize{reference/javascript_api:setController}}\pysiglinewithargsret{\sphinxbfcode{\sphinxupquote{method }}\sphinxbfcode{\sphinxupquote{setController}}}{\emph{Controller}}{}
this is useful to customize the actual class to use before calling
createView.
\begin{quote}\begin{description}
\item[{Parameters}] \leavevmode\begin{itemize}

\sphinxstylestrong{Controller} (\sphinxstyleliteralemphasis{\sphinxupquote{Controller}})
\end{itemize}

\end{description}\end{quote}

\end{fulllineitems}



\begin{fulllineitems}
\phantomsection\label{\detokenize{reference/javascript_api:AbstractViewViewInfo}}\pysiglinewithargsret{\sphinxbfcode{\sphinxupquote{class }}\sphinxbfcode{\sphinxupquote{AbstractViewViewInfo}}}{}{}~

\begin{fulllineitems}
\phantomsection\label{\detokenize{reference/javascript_api:arch}}\pysigline{\sphinxbfcode{\sphinxupquote{attribute }}\sphinxbfcode{\sphinxupquote{arch}} Object}
\end{fulllineitems}



\begin{fulllineitems}
\phantomsection\label{\detokenize{reference/javascript_api:fields}}\pysigline{\sphinxbfcode{\sphinxupquote{attribute }}\sphinxbfcode{\sphinxupquote{fields}} Object}
\end{fulllineitems}



\begin{fulllineitems}
\phantomsection\label{\detokenize{reference/javascript_api:fieldsInfo}}\pysigline{\sphinxbfcode{\sphinxupquote{attribute }}\sphinxbfcode{\sphinxupquote{fieldsInfo}} Object}
\end{fulllineitems}


\end{fulllineitems}



\begin{fulllineitems}
\phantomsection\label{\detokenize{reference/javascript_api:AbstractViewParams}}\pysiglinewithargsret{\sphinxbfcode{\sphinxupquote{class }}\sphinxbfcode{\sphinxupquote{AbstractViewParams}}}{}{}~

\begin{fulllineitems}
\phantomsection\label{\detokenize{reference/javascript_api:modelName}}\pysigline{\sphinxbfcode{\sphinxupquote{attribute }}\sphinxbfcode{\sphinxupquote{modelName}} string}
The actual model name

\end{fulllineitems}



\begin{fulllineitems}
\phantomsection\label{\detokenize{reference/javascript_api:context}}\pysigline{\sphinxbfcode{\sphinxupquote{attribute }}\sphinxbfcode{\sphinxupquote{context}} Object}
\end{fulllineitems}



\begin{fulllineitems}
\phantomsection\label{\detokenize{reference/javascript_api:count}}\pysigline{\sphinxbfcode{\sphinxupquote{attribute }}\sphinxbfcode{\sphinxupquote{count}} number}
\end{fulllineitems}



\begin{fulllineitems}
\phantomsection\label{\detokenize{reference/javascript_api:domain}}\pysigline{\sphinxbfcode{\sphinxupquote{attribute }}\sphinxbfcode{\sphinxupquote{domain}} string{[}{]}}
\end{fulllineitems}



\begin{fulllineitems}
\phantomsection\label{\detokenize{reference/javascript_api:groupBy}}\pysigline{\sphinxbfcode{\sphinxupquote{attribute }}\sphinxbfcode{\sphinxupquote{groupBy}} string{[}{]}}
\end{fulllineitems}



\begin{fulllineitems}
\phantomsection\label{\detokenize{reference/javascript_api:currentId}}\pysigline{\sphinxbfcode{\sphinxupquote{attribute }}\sphinxbfcode{\sphinxupquote{currentId}} number}
\end{fulllineitems}



\begin{fulllineitems}
\phantomsection\label{\detokenize{reference/javascript_api:ids}}\pysigline{\sphinxbfcode{\sphinxupquote{attribute }}\sphinxbfcode{\sphinxupquote{ids}} number{[}{]}}
\end{fulllineitems}



\begin{fulllineitems}
\phantomsection\label{\detokenize{reference/javascript_api:action.help}}\pysigline{\sphinxbfcode{\sphinxupquote{attribute }}\sphinxbfcode{\sphinxupquote{action.help}} string}
\end{fulllineitems}


\end{fulllineitems}


\end{fulllineitems}


\end{fulllineitems}

\phantomsection\label{\detokenize{reference/javascript_api:module-web_editor.backend}}

\begin{fulllineitems}
\phantomsection\label{\detokenize{reference/javascript_api:web_editor.backend}}\pysigline{\sphinxbfcode{\sphinxupquote{module }}\sphinxbfcode{\sphinxupquote{web\_editor.backend}}}~~\begin{quote}\begin{description}
\item[{Exports}] \leavevmode{\hyperref[\detokenize{reference/javascript_api:web_editor.backend.}]{\sphinxcrossref{
\textless{}anonymous\textgreater{}
}}}
\item[{Depends On}] \leavevmode\begin{itemize}
\item {} {\hyperref[\detokenize{reference/javascript_api:web.AbstractField}]{\sphinxcrossref{
web.AbstractField
}}}
\item {} {\hyperref[\detokenize{reference/javascript_api:web.basic_fields}]{\sphinxcrossref{
web.basic\_fields
}}}
\item {} {\hyperref[\detokenize{reference/javascript_api:web.config}]{\sphinxcrossref{
web.config
}}}
\item {} {\hyperref[\detokenize{reference/javascript_api:web.core}]{\sphinxcrossref{
web.core
}}}
\item {} {\hyperref[\detokenize{reference/javascript_api:web.field_registry}]{\sphinxcrossref{
web.field\_registry
}}}
\item {} {\hyperref[\detokenize{reference/javascript_api:web.session}]{\sphinxcrossref{
web.session
}}}
\item {} {\hyperref[\detokenize{reference/javascript_api:web_editor.rte.summernote}]{\sphinxcrossref{
web\_editor.rte.summernote
}}}
\item {} {\hyperref[\detokenize{reference/javascript_api:web_editor.transcoder}]{\sphinxcrossref{
web\_editor.transcoder
}}}
\end{itemize}

\end{description}\end{quote}


\begin{fulllineitems}
\phantomsection\label{\detokenize{reference/javascript_api:web_editor.backend.}}\pysigline{\sphinxbfcode{\sphinxupquote{namespace }}\sphinxbfcode{\sphinxupquote{}}}~

\begin{fulllineitems}
\phantomsection\label{\detokenize{reference/javascript_api:FieldTextHtmlSimple}}\pysiglinewithargsret{\sphinxbfcode{\sphinxupquote{class }}\sphinxbfcode{\sphinxupquote{FieldTextHtmlSimple}}}{}{}~\begin{quote}\begin{description}
\item[{Extends}] \leavevmode
DebouncedField

\item[{Mixes}] \leavevmode\begin{itemize}
\item {} 
TranslatableFieldMixin

\end{itemize}

\end{description}\end{quote}

FieldTextHtmlSimple Widget
Intended to display HTML content. This widget uses the summernote editor
improved by odoo.


\begin{fulllineitems}
\phantomsection\label{\detokenize{reference/javascript_api:commitChanges}}\pysiglinewithargsret{\sphinxbfcode{\sphinxupquote{method }}\sphinxbfcode{\sphinxupquote{commitChanges}}}{}{}
Summernote doesn’t notify for changes done in code mode. We override
commitChanges to manually switch back to normal mode before committing
changes, so that the widget is aware of the changes done in code mode.

\end{fulllineitems}



\begin{fulllineitems}
\phantomsection\label{\detokenize{reference/javascript_api:reset}}\pysiglinewithargsret{\sphinxbfcode{\sphinxupquote{method }}\sphinxbfcode{\sphinxupquote{reset}}}{\emph{record}, \emph{event}}{}
Do not re-render this field if it was the origin of the onchange call.
\begin{quote}\begin{description}
\item[{Parameters}] \leavevmode\begin{itemize}

\sphinxstylestrong{record}

\sphinxstylestrong{event}
\end{itemize}

\end{description}\end{quote}

\end{fulllineitems}


\end{fulllineitems}


\end{fulllineitems}



\begin{fulllineitems}
\phantomsection\label{\detokenize{reference/javascript_api:FieldTextHtmlSimple}}\pysiglinewithargsret{\sphinxbfcode{\sphinxupquote{class }}\sphinxbfcode{\sphinxupquote{FieldTextHtmlSimple}}}{}{}~\begin{quote}\begin{description}
\item[{Extends}] \leavevmode
DebouncedField

\item[{Mixes}] \leavevmode\begin{itemize}
\item {} 
TranslatableFieldMixin

\end{itemize}

\end{description}\end{quote}

FieldTextHtmlSimple Widget
Intended to display HTML content. This widget uses the summernote editor
improved by odoo.


\begin{fulllineitems}
\phantomsection\label{\detokenize{reference/javascript_api:commitChanges}}\pysiglinewithargsret{\sphinxbfcode{\sphinxupquote{method }}\sphinxbfcode{\sphinxupquote{commitChanges}}}{}{}
Summernote doesn’t notify for changes done in code mode. We override
commitChanges to manually switch back to normal mode before committing
changes, so that the widget is aware of the changes done in code mode.

\end{fulllineitems}



\begin{fulllineitems}
\phantomsection\label{\detokenize{reference/javascript_api:reset}}\pysiglinewithargsret{\sphinxbfcode{\sphinxupquote{method }}\sphinxbfcode{\sphinxupquote{reset}}}{\emph{record}, \emph{event}}{}
Do not re-render this field if it was the origin of the onchange call.
\begin{quote}\begin{description}
\item[{Parameters}] \leavevmode\begin{itemize}

\sphinxstylestrong{record}

\sphinxstylestrong{event}
\end{itemize}

\end{description}\end{quote}

\end{fulllineitems}


\end{fulllineitems}


\end{fulllineitems}

\phantomsection\label{\detokenize{reference/javascript_api:module-web.FilterMenu}}

\begin{fulllineitems}
\phantomsection\label{\detokenize{reference/javascript_api:web.FilterMenu}}\pysigline{\sphinxbfcode{\sphinxupquote{module }}\sphinxbfcode{\sphinxupquote{web.FilterMenu}}}~~\begin{quote}\begin{description}
\item[{Exports}] \leavevmode{\hyperref[\detokenize{reference/javascript_api:web.FilterMenu.}]{\sphinxcrossref{
\textless{}anonymous\textgreater{}
}}}
\item[{Depends On}] \leavevmode\begin{itemize}
\item {} {\hyperref[\detokenize{reference/javascript_api:web.Widget}]{\sphinxcrossref{
web.Widget
}}}
\item {} {\hyperref[\detokenize{reference/javascript_api:web.search_filters}]{\sphinxcrossref{
web.search\_filters
}}}
\item {} {\hyperref[\detokenize{reference/javascript_api:web.search_inputs}]{\sphinxcrossref{
web.search\_inputs
}}}
\end{itemize}

\end{description}\end{quote}


\begin{fulllineitems}
\phantomsection\label{\detokenize{reference/javascript_api:web.FilterMenu.}}\pysiglinewithargsret{\sphinxbfcode{\sphinxupquote{class }}\sphinxbfcode{\sphinxupquote{}}}{\emph{parent}, \emph{filters}, \emph{fields}}{}~\begin{quote}\begin{description}
\item[{Extends}] \leavevmode{\hyperref[\detokenize{reference/javascript_api:web.Widget.Widget}]{\sphinxcrossref{
Widget
}}}
\item[{Parameters}] \leavevmode\begin{itemize}

\sphinxstylestrong{parent}

\sphinxstylestrong{filters}

\sphinxstylestrong{fields}
\end{itemize}

\end{description}\end{quote}

\end{fulllineitems}


\end{fulllineitems}

\phantomsection\label{\detokenize{reference/javascript_api:module-web.core}}

\begin{fulllineitems}
\phantomsection\label{\detokenize{reference/javascript_api:web.core}}\pysigline{\sphinxbfcode{\sphinxupquote{module }}\sphinxbfcode{\sphinxupquote{web.core}}}~~\begin{quote}\begin{description}
\item[{Exports}] \leavevmode{\hyperref[\detokenize{reference/javascript_api:web.core.}]{\sphinxcrossref{
\textless{}anonymous\textgreater{}
}}}
\item[{Depends On}] \leavevmode\begin{itemize}
\item {} {\hyperref[\detokenize{reference/javascript_api:web.Bus}]{\sphinxcrossref{
web.Bus
}}}
\item {} {\hyperref[\detokenize{reference/javascript_api:web.Class}]{\sphinxcrossref{
web.Class
}}}
\item {} {\hyperref[\detokenize{reference/javascript_api:web.QWeb}]{\sphinxcrossref{
web.QWeb
}}}
\item {} {\hyperref[\detokenize{reference/javascript_api:web.Registry}]{\sphinxcrossref{
web.Registry
}}}
\item {} {\hyperref[\detokenize{reference/javascript_api:web.translation}]{\sphinxcrossref{
web.translation
}}}
\end{itemize}

\end{description}\end{quote}


\begin{fulllineitems}
\phantomsection\label{\detokenize{reference/javascript_api:debug}}\pysigline{\sphinxbfcode{\sphinxupquote{attribute }}\sphinxbfcode{\sphinxupquote{debug}} Boolean}
Whether the client is currently in “debug” mode

\end{fulllineitems}



\begin{fulllineitems}
\phantomsection\label{\detokenize{reference/javascript_api:web.core.}}\pysigline{\sphinxbfcode{\sphinxupquote{namespace }}\sphinxbfcode{\sphinxupquote{}}}~

\begin{fulllineitems}
\phantomsection\label{\detokenize{reference/javascript_api:debug}}\pysigline{\sphinxbfcode{\sphinxupquote{attribute }}\sphinxbfcode{\sphinxupquote{debug}} Boolean}
Whether the client is currently in “debug” mode

\end{fulllineitems}



\begin{fulllineitems}
\phantomsection\label{\detokenize{reference/javascript_api:Class}}\pysiglinewithargsret{\sphinxbfcode{\sphinxupquote{class }}\sphinxbfcode{\sphinxupquote{Class}}}{}{}
Improved John Resig’s inheritance, based on:

Simple JavaScript Inheritance
By John Resig \sphinxurl{http://ejohn.org/}
MIT Licensed.

Adds “include()”

Defines The Class object. That object can be used to define and inherit classes using
the extend() method.

Example:

\fvset{hllines={, ,}}%
\begin{sphinxVerbatim}[commandchars=\\\{\}]
\PYG{k+kd}{var} \PYG{n+nx}{Person} \PYG{o}{=} \PYG{n+nx}{Class}\PYG{p}{.}\PYG{n+nx}{extend}\PYG{p}{(}\PYG{p}{\PYGZob{}}
 \PYG{n+nx}{init}\PYG{o}{:} \PYG{k+kd}{function}\PYG{p}{(}\PYG{n+nx}{isDancing}\PYG{p}{)}\PYG{p}{\PYGZob{}}
    \PYG{k}{this}\PYG{p}{.}\PYG{n+nx}{dancing} \PYG{o}{=} \PYG{n+nx}{isDancing}\PYG{p}{;}
  \PYG{p}{\PYGZcb{}}\PYG{p}{,}
  \PYG{n+nx}{dance}\PYG{o}{:} \PYG{k+kd}{function}\PYG{p}{(}\PYG{p}{)}\PYG{p}{\PYGZob{}}
    \PYG{k}{return} \PYG{k}{this}\PYG{p}{.}\PYG{n+nx}{dancing}\PYG{p}{;}
  \PYG{p}{\PYGZcb{}}
\PYG{p}{\PYGZcb{}}\PYG{p}{)}\PYG{p}{;}
\end{sphinxVerbatim}

The init() method act as a constructor. This class can be instanced this way:

\fvset{hllines={, ,}}%
\begin{sphinxVerbatim}[commandchars=\\\{\}]
\PYG{k+kd}{var} \PYG{n+nx}{person} \PYG{o}{=} \PYG{k}{new} \PYG{n+nx}{Person}\PYG{p}{(}\PYG{k+kc}{true}\PYG{p}{)}\PYG{p}{;}
\PYG{n+nx}{person}\PYG{p}{.}\PYG{n+nx}{dance}\PYG{p}{(}\PYG{p}{)}\PYG{p}{;}

\PYG{n+nx}{The} \PYG{n+nx}{Person} \PYG{k+kr}{class} \PYG{n+nx}{can} \PYG{n+nx}{also} \PYG{n+nx}{be} \PYG{n+nx}{extended} \PYG{n+nx}{again}\PYG{o}{:}

\PYG{k+kd}{var} \PYG{n+nx}{Ninja} \PYG{o}{=} \PYG{n+nx}{Person}\PYG{p}{.}\PYG{n+nx}{extend}\PYG{p}{(}\PYG{p}{\PYGZob{}}
  \PYG{n+nx}{init}\PYG{o}{:} \PYG{k+kd}{function}\PYG{p}{(}\PYG{p}{)}\PYG{p}{\PYGZob{}}
    \PYG{k}{this}\PYG{p}{.}\PYG{n+nx}{\PYGZus{}super}\PYG{p}{(} \PYG{k+kc}{false} \PYG{p}{)}\PYG{p}{;}
  \PYG{p}{\PYGZcb{}}\PYG{p}{,}
  \PYG{n+nx}{dance}\PYG{o}{:} \PYG{k+kd}{function}\PYG{p}{(}\PYG{p}{)}\PYG{p}{\PYGZob{}}
    \PYG{c+c1}{// Call the inherited version of dance()}
    \PYG{k}{return} \PYG{k}{this}\PYG{p}{.}\PYG{n+nx}{\PYGZus{}super}\PYG{p}{(}\PYG{p}{)}\PYG{p}{;}
  \PYG{p}{\PYGZcb{}}\PYG{p}{,}
  \PYG{n+nx}{swingSword}\PYG{o}{:} \PYG{k+kd}{function}\PYG{p}{(}\PYG{p}{)}\PYG{p}{\PYGZob{}}
    \PYG{k}{return} \PYG{k+kc}{true}\PYG{p}{;}
  \PYG{p}{\PYGZcb{}}
\PYG{p}{\PYGZcb{}}\PYG{p}{)}\PYG{p}{;}
\end{sphinxVerbatim}

When extending a class, each re-defined method can use this.\_super() to call the previous
implementation of that method.


\begin{fulllineitems}
\phantomsection\label{\detokenize{reference/javascript_api:extend}}\pysiglinewithargsret{\sphinxbfcode{\sphinxupquote{method }}\sphinxbfcode{\sphinxupquote{extend}}}{\emph{prop}}{}
Subclass an existing class
\begin{quote}\begin{description}
\item[{Parameters}] \leavevmode\begin{itemize}

\sphinxstylestrong{prop} (\sphinxstyleliteralemphasis{\sphinxupquote{Object}}) \textendash{} class-level properties (class attributes and instance methods) to set on the new class
\end{itemize}

\end{description}\end{quote}

\end{fulllineitems}


\end{fulllineitems}



\begin{fulllineitems}
\phantomsection\label{\detokenize{reference/javascript_api:_t}}\pysiglinewithargsret{\sphinxbfcode{\sphinxupquote{function }}\sphinxbfcode{\sphinxupquote{\_t}}}{\emph{source}}{{ $\rightarrow$ String}}
Eager translation function, performs translation immediately at call
site. Beware using this outside of method bodies (before the
translation database is loaded), you probably want {\hyperref[\detokenize{reference/javascript_api:_lt}]{\sphinxcrossref{\sphinxcode{\sphinxupquote{\_lt()}}}}}
instead.
\begin{quote}\begin{description}
\item[{Parameters}] \leavevmode\begin{itemize}

\sphinxstylestrong{source} (\sphinxstyleliteralemphasis{\sphinxupquote{String}}) \textendash{} string to translate
\end{itemize}

\item[{Returns}] \leavevmode
source translated into the current locale

\item[{Return Type}] \leavevmode
\sphinxstyleliteralemphasis{\sphinxupquote{String}}

\end{description}\end{quote}

\end{fulllineitems}



\begin{fulllineitems}
\phantomsection\label{\detokenize{reference/javascript_api:_lt}}\pysiglinewithargsret{\sphinxbfcode{\sphinxupquote{function }}\sphinxbfcode{\sphinxupquote{\_lt}}}{\emph{s}}{{ $\rightarrow$ Object}}
Lazy translation function, only performs the translation when actually
printed (e.g. inserted into a template)

Useful when defining translatable strings in code evaluated before the
translation database is loaded, as class attributes or at the top-level of
an OpenERP Web module
\begin{quote}\begin{description}
\item[{Parameters}] \leavevmode\begin{itemize}

\sphinxstylestrong{s} (\sphinxstyleliteralemphasis{\sphinxupquote{String}}) \textendash{} string to translate
\end{itemize}

\item[{Returns}] \leavevmode
lazy translation object

\item[{Return Type}] \leavevmode
\sphinxstyleliteralemphasis{\sphinxupquote{Object}}

\end{description}\end{quote}

\end{fulllineitems}


\end{fulllineitems}


\end{fulllineitems}

\phantomsection\label{\detokenize{reference/javascript_api:module-bus.bus}}

\begin{fulllineitems}
\phantomsection\label{\detokenize{reference/javascript_api:bus.bus}}\pysigline{\sphinxbfcode{\sphinxupquote{module }}\sphinxbfcode{\sphinxupquote{bus.bus}}}~~\begin{quote}\begin{description}
\item[{Exports}] \leavevmode{\hyperref[\detokenize{reference/javascript_api:bus.bus.bus}]{\sphinxcrossref{
bus
}}}
\item[{Depends On}] \leavevmode\begin{itemize}
\item {} {\hyperref[\detokenize{reference/javascript_api:web.Widget}]{\sphinxcrossref{
web.Widget
}}}
\item {} {\hyperref[\detokenize{reference/javascript_api:web.local_storage}]{\sphinxcrossref{
web.local\_storage
}}}
\item {} {\hyperref[\detokenize{reference/javascript_api:web.session}]{\sphinxcrossref{
web.session
}}}
\end{itemize}

\end{description}\end{quote}


\begin{fulllineitems}
\phantomsection\label{\detokenize{reference/javascript_api:bus}}\pysigline{\sphinxbfcode{\sphinxupquote{namespace }}\sphinxbfcode{\sphinxupquote{bus}}}
\end{fulllineitems}



\begin{fulllineitems}
\phantomsection\label{\detokenize{reference/javascript_api:CrossTabBus}}\pysiglinewithargsret{\sphinxbfcode{\sphinxupquote{class }}\sphinxbfcode{\sphinxupquote{CrossTabBus}}}{}{}~\begin{quote}\begin{description}
\item[{Extends}] \leavevmode{\hyperref[\detokenize{reference/javascript_api:Bus}]{\sphinxcrossref{
Bus
}}}
\end{description}\end{quote}

CrossTabBus Widget

Manage the communication before browser tab to allow only one tab polling for the others (performance improvement)
When a tab is opened, and the start\_polling method is called, the tab is signaling through the localStorage to the
others. When a tab is closed, it signals its removing. If he was the master tab (the polling one), he choose another
one in the list of open tabs. This one start polling for the other. When a notification is recieved from the poll, it
is signaling through the localStorage too.

localStorage used keys are:
\begin{itemize}
\item {} 
bus.channels : shared public channel list to listen during the poll

\item {} 
bus.options : shared options

\item {} 
bus.notification : the received notifications from the last poll

\item {} 
bus.tab\_list : list of opened tab ids

\item {} 
bus.tab\_master : generated id of the master tab

\end{itemize}

\end{fulllineitems}


\end{fulllineitems}

\phantomsection\label{\detokenize{reference/javascript_api:module-sms.sms_widget}}

\begin{fulllineitems}
\phantomsection\label{\detokenize{reference/javascript_api:sms.sms_widget}}\pysigline{\sphinxbfcode{\sphinxupquote{module }}\sphinxbfcode{\sphinxupquote{sms.sms\_widget}}}~~\begin{quote}\begin{description}
\item[{Exports}] \leavevmode{\hyperref[\detokenize{reference/javascript_api:sms.sms_widget.SmsWidget}]{\sphinxcrossref{
SmsWidget
}}}
\item[{Depends On}] \leavevmode\begin{itemize}
\item {} {\hyperref[\detokenize{reference/javascript_api:web.basic_fields}]{\sphinxcrossref{
web.basic\_fields
}}}
\item {} {\hyperref[\detokenize{reference/javascript_api:web.core}]{\sphinxcrossref{
web.core
}}}
\item {} {\hyperref[\detokenize{reference/javascript_api:web.field_registry}]{\sphinxcrossref{
web.field\_registry
}}}
\item {} {\hyperref[\detokenize{reference/javascript_api:web.framework}]{\sphinxcrossref{
web.framework
}}}
\end{itemize}

\end{description}\end{quote}


\begin{fulllineitems}
\phantomsection\label{\detokenize{reference/javascript_api:SmsWidget}}\pysiglinewithargsret{\sphinxbfcode{\sphinxupquote{class }}\sphinxbfcode{\sphinxupquote{SmsWidget}}}{}{}~\begin{quote}\begin{description}
\item[{Extends}] \leavevmode
FieldText

\end{description}\end{quote}

SmsWidget is a widget to display a textarea (the body) and a text representing
the number of SMS and the number of characters. This text is computed every
time the user changes the body.

\end{fulllineitems}



\begin{fulllineitems}
\phantomsection\label{\detokenize{reference/javascript_api:SmsWidget}}\pysiglinewithargsret{\sphinxbfcode{\sphinxupquote{class }}\sphinxbfcode{\sphinxupquote{SmsWidget}}}{}{}~\begin{quote}\begin{description}
\item[{Extends}] \leavevmode
FieldText

\end{description}\end{quote}

SmsWidget is a widget to display a textarea (the body) and a text representing
the number of SMS and the number of characters. This text is computed every
time the user changes the body.

\end{fulllineitems}


\end{fulllineitems}

\phantomsection\label{\detokenize{reference/javascript_api:module-web.KanbanModel}}

\begin{fulllineitems}
\phantomsection\label{\detokenize{reference/javascript_api:web.KanbanModel}}\pysigline{\sphinxbfcode{\sphinxupquote{module }}\sphinxbfcode{\sphinxupquote{web.KanbanModel}}}~~\begin{quote}\begin{description}
\item[{Exports}] \leavevmode{\hyperref[\detokenize{reference/javascript_api:web.KanbanModel.KanbanModel}]{\sphinxcrossref{
KanbanModel
}}}
\item[{Depends On}] \leavevmode\begin{itemize}
\item {} {\hyperref[\detokenize{reference/javascript_api:web.BasicModel}]{\sphinxcrossref{
web.BasicModel
}}}
\end{itemize}

\end{description}\end{quote}


\begin{fulllineitems}
\phantomsection\label{\detokenize{reference/javascript_api:KanbanModel}}\pysiglinewithargsret{\sphinxbfcode{\sphinxupquote{class }}\sphinxbfcode{\sphinxupquote{KanbanModel}}}{}{}~\begin{quote}\begin{description}
\item[{Extends}] \leavevmode{\hyperref[\detokenize{reference/javascript_api:web.BasicModel.BasicModel}]{\sphinxcrossref{
BasicModel
}}}
\end{description}\end{quote}


\begin{fulllineitems}
\phantomsection\label{\detokenize{reference/javascript_api:addRecordToGroup}}\pysiglinewithargsret{\sphinxbfcode{\sphinxupquote{method }}\sphinxbfcode{\sphinxupquote{addRecordToGroup}}}{\emph{groupID}, \emph{resId}}{{ $\rightarrow$ Deferred\textless{}string\textgreater{}}}
Adds a record to a group in the localData, and fetch the record.
\begin{quote}\begin{description}
\item[{Parameters}] \leavevmode\begin{itemize}

\sphinxstylestrong{groupID} (\sphinxstyleliteralemphasis{\sphinxupquote{string}}) \textendash{} localID of the group

\sphinxstylestrong{resId} (\sphinxstyleliteralemphasis{\sphinxupquote{integer}}) \textendash{} id of the record
\end{itemize}

\item[{Returns}] \leavevmode
resolves to the local id of the new record

\item[{Return Type}] \leavevmode
\sphinxstyleliteralemphasis{\sphinxupquote{Deferred}}\textless{}\sphinxstyleliteralemphasis{\sphinxupquote{string}}\textgreater{}

\end{description}\end{quote}

\end{fulllineitems}



\begin{fulllineitems}
\phantomsection\label{\detokenize{reference/javascript_api:createGroup}}\pysiglinewithargsret{\sphinxbfcode{\sphinxupquote{method }}\sphinxbfcode{\sphinxupquote{createGroup}}}{\emph{name}, \emph{parentID}}{{ $\rightarrow$ Deferred\textless{}string\textgreater{}}}
Creates a new group from a name (performs a name\_create).
\begin{quote}\begin{description}
\item[{Parameters}] \leavevmode\begin{itemize}

\sphinxstylestrong{name} (\sphinxstyleliteralemphasis{\sphinxupquote{string}})

\sphinxstylestrong{parentID} (\sphinxstyleliteralemphasis{\sphinxupquote{string}}) \textendash{} localID of the parent of the group
\end{itemize}

\item[{Returns}] \leavevmode
resolves to the local id of the new group

\item[{Return Type}] \leavevmode
\sphinxstyleliteralemphasis{\sphinxupquote{Deferred}}\textless{}\sphinxstyleliteralemphasis{\sphinxupquote{string}}\textgreater{}

\end{description}\end{quote}

\end{fulllineitems}



\begin{fulllineitems}
\phantomsection\label{\detokenize{reference/javascript_api:get}}\pysiglinewithargsret{\sphinxbfcode{\sphinxupquote{method }}\sphinxbfcode{\sphinxupquote{get}}}{}{{ $\rightarrow$ Object}}
Add the key \sphinxcode{\sphinxupquote{tooltipData}} (kanban specific) when performing a \sphinxcode{\sphinxupquote{geŧ}}.
\begin{quote}\begin{description}
\item[{Return Type}] \leavevmode
\sphinxstyleliteralemphasis{\sphinxupquote{Object}}

\end{description}\end{quote}

\end{fulllineitems}



\begin{fulllineitems}
\phantomsection\label{\detokenize{reference/javascript_api:getColumn}}\pysiglinewithargsret{\sphinxbfcode{\sphinxupquote{method }}\sphinxbfcode{\sphinxupquote{getColumn}}}{\emph{id}}{{ $\rightarrow$ Object}}
Same as @see get but getting the parent element whose ID is given.
\begin{quote}\begin{description}
\item[{Parameters}] \leavevmode\begin{itemize}

\sphinxstylestrong{id} (\sphinxstyleliteralemphasis{\sphinxupquote{string}})
\end{itemize}

\item[{Return Type}] \leavevmode
\sphinxstyleliteralemphasis{\sphinxupquote{Object}}

\end{description}\end{quote}

\end{fulllineitems}



\begin{fulllineitems}
\phantomsection\label{\detokenize{reference/javascript_api:loadColumnRecords}}\pysiglinewithargsret{\sphinxbfcode{\sphinxupquote{method }}\sphinxbfcode{\sphinxupquote{loadColumnRecords}}}{\emph{groupID}}{{ $\rightarrow$ Deferred}}
Opens a given group and loads its \textless{}limit\textgreater{} first records
\begin{quote}\begin{description}
\item[{Parameters}] \leavevmode\begin{itemize}

\sphinxstylestrong{groupID} (\sphinxstyleliteralemphasis{\sphinxupquote{string}})
\end{itemize}

\item[{Return Type}] \leavevmode
\sphinxstyleliteralemphasis{\sphinxupquote{Deferred}}

\end{description}\end{quote}

\end{fulllineitems}



\begin{fulllineitems}
\phantomsection\label{\detokenize{reference/javascript_api:loadMore}}\pysiglinewithargsret{\sphinxbfcode{\sphinxupquote{method }}\sphinxbfcode{\sphinxupquote{loadMore}}}{\emph{groupID}}{{ $\rightarrow$ Deferred\textless{}string\textgreater{}}}
Load more records in a group.
\begin{quote}\begin{description}
\item[{Parameters}] \leavevmode\begin{itemize}

\sphinxstylestrong{groupID} (\sphinxstyleliteralemphasis{\sphinxupquote{string}}) \textendash{} localID of the group
\end{itemize}

\item[{Returns}] \leavevmode
resolves to the localID of the group

\item[{Return Type}] \leavevmode
\sphinxstyleliteralemphasis{\sphinxupquote{Deferred}}\textless{}\sphinxstyleliteralemphasis{\sphinxupquote{string}}\textgreater{}

\end{description}\end{quote}

\end{fulllineitems}



\begin{fulllineitems}
\phantomsection\label{\detokenize{reference/javascript_api:moveRecord}}\pysiglinewithargsret{\sphinxbfcode{\sphinxupquote{method }}\sphinxbfcode{\sphinxupquote{moveRecord}}}{\emph{recordID}, \emph{groupID}, \emph{parentID}}{{ $\rightarrow$ Deferred\textless{}string{[}{]}\textgreater{}}}
Moves a record from a group to another.
\begin{quote}\begin{description}
\item[{Parameters}] \leavevmode\begin{itemize}

\sphinxstylestrong{recordID} (\sphinxstyleliteralemphasis{\sphinxupquote{string}}) \textendash{} localID of the record

\sphinxstylestrong{groupID} (\sphinxstyleliteralemphasis{\sphinxupquote{string}}) \textendash{} localID of the new group of the record

\sphinxstylestrong{parentID} (\sphinxstyleliteralemphasis{\sphinxupquote{string}}) \textendash{} localID of the parent
\end{itemize}

\item[{Returns}] \leavevmode
resolves to a pair {[}oldGroupID, newGroupID{]}

\item[{Return Type}] \leavevmode
\sphinxstyleliteralemphasis{\sphinxupquote{Deferred}}\textless{}\sphinxstyleliteralemphasis{\sphinxupquote{Array}}\textless{}\sphinxstyleliteralemphasis{\sphinxupquote{string}}\textgreater{}\textgreater{}

\end{description}\end{quote}

\end{fulllineitems}


\end{fulllineitems}


\end{fulllineitems}

\phantomsection\label{\detokenize{reference/javascript_api:module-web.session}}

\begin{fulllineitems}
\phantomsection\label{\detokenize{reference/javascript_api:web.session}}\pysigline{\sphinxbfcode{\sphinxupquote{module }}\sphinxbfcode{\sphinxupquote{web.session}}}~~\begin{quote}\begin{description}
\item[{Exports}] \leavevmode{\hyperref[\detokenize{reference/javascript_api:web.session.session}]{\sphinxcrossref{
session
}}}
\item[{Depends On}] \leavevmode\begin{itemize}
\item {} {\hyperref[\detokenize{reference/javascript_api:web.Session}]{\sphinxcrossref{
web.Session
}}}
\end{itemize}

\end{description}\end{quote}


\begin{fulllineitems}
\phantomsection\label{\detokenize{reference/javascript_api:session}}\pysigline{\sphinxbfcode{\sphinxupquote{object }}\sphinxbfcode{\sphinxupquote{session}}\sphinxbfcode{\sphinxupquote{ instance of }}{\hyperref[\detokenize{reference/javascript_api:web.Session.Session}]{\sphinxcrossref{Session}}}}
\end{fulllineitems}


\end{fulllineitems}

\phantomsection\label{\detokenize{reference/javascript_api:module-web.ActionManager}}

\begin{fulllineitems}
\phantomsection\label{\detokenize{reference/javascript_api:web.ActionManager}}\pysigline{\sphinxbfcode{\sphinxupquote{module }}\sphinxbfcode{\sphinxupquote{web.ActionManager}}}~~\begin{quote}\begin{description}
\item[{Exports}] \leavevmode{\hyperref[\detokenize{reference/javascript_api:web.ActionManager.ActionManager}]{\sphinxcrossref{
ActionManager
}}}
\item[{Depends On}] \leavevmode\begin{itemize}
\item {} {\hyperref[\detokenize{reference/javascript_api:web.Bus}]{\sphinxcrossref{
web.Bus
}}}
\item {} {\hyperref[\detokenize{reference/javascript_api:web.Context}]{\sphinxcrossref{
web.Context
}}}
\item {} {\hyperref[\detokenize{reference/javascript_api:web.ControlPanel}]{\sphinxcrossref{
web.ControlPanel
}}}
\item {} {\hyperref[\detokenize{reference/javascript_api:web.Dialog}]{\sphinxcrossref{
web.Dialog
}}}
\item {} {\hyperref[\detokenize{reference/javascript_api:web.ViewManager}]{\sphinxcrossref{
web.ViewManager
}}}
\item {} {\hyperref[\detokenize{reference/javascript_api:web.Widget}]{\sphinxcrossref{
web.Widget
}}}
\item {} {\hyperref[\detokenize{reference/javascript_api:web.core}]{\sphinxcrossref{
web.core
}}}
\item {} {\hyperref[\detokenize{reference/javascript_api:web.crash_manager}]{\sphinxcrossref{
web.crash\_manager
}}}
\item {} {\hyperref[\detokenize{reference/javascript_api:web.data}]{\sphinxcrossref{
web.data
}}}
\item {} {\hyperref[\detokenize{reference/javascript_api:web.data_manager}]{\sphinxcrossref{
web.data\_manager
}}}
\item {} {\hyperref[\detokenize{reference/javascript_api:web.dom}]{\sphinxcrossref{
web.dom
}}}
\item {} {\hyperref[\detokenize{reference/javascript_api:web.framework}]{\sphinxcrossref{
web.framework
}}}
\item {} {\hyperref[\detokenize{reference/javascript_api:web.pyeval}]{\sphinxcrossref{
web.pyeval
}}}
\item {} {\hyperref[\detokenize{reference/javascript_api:web.session}]{\sphinxcrossref{
web.session
}}}
\end{itemize}

\end{description}\end{quote}


\begin{fulllineitems}
\phantomsection\label{\detokenize{reference/javascript_api:WidgetAction}}\pysiglinewithargsret{\sphinxbfcode{\sphinxupquote{class }}\sphinxbfcode{\sphinxupquote{WidgetAction}}}{\emph{action}, \emph{widget}}{}~\begin{quote}\begin{description}
\item[{Extends}] \leavevmode{\hyperref[\detokenize{reference/javascript_api:web.ActionManager.Action}]{\sphinxcrossref{
Action
}}}
\item[{Parameters}] \leavevmode\begin{itemize}

\sphinxstylestrong{action}

\sphinxstylestrong{widget}
\end{itemize}

\end{description}\end{quote}

Specialization of Action for client actions that are Widgets


\begin{fulllineitems}
\phantomsection\label{\detokenize{reference/javascript_api:restore}}\pysiglinewithargsret{\sphinxbfcode{\sphinxupquote{method }}\sphinxbfcode{\sphinxupquote{restore}}}{}{}
Restores WidgetAction by calling do\_show on its widget

\end{fulllineitems}



\begin{fulllineitems}
\phantomsection\label{\detokenize{reference/javascript_api:detach}}\pysiglinewithargsret{\sphinxbfcode{\sphinxupquote{method }}\sphinxbfcode{\sphinxupquote{detach}}}{}{{ $\rightarrow$ *}}
Detaches the action’s widget from the DOM
\begin{quote}\begin{description}
\item[{Returns}] \leavevmode
widget’s \$el

\end{description}\end{quote}

\end{fulllineitems}



\begin{fulllineitems}
\phantomsection\label{\detokenize{reference/javascript_api:destroy}}\pysiglinewithargsret{\sphinxbfcode{\sphinxupquote{method }}\sphinxbfcode{\sphinxupquote{destroy}}}{}{}
Destroys the widget

\end{fulllineitems}


\end{fulllineitems}



\begin{fulllineitems}
\phantomsection\label{\detokenize{reference/javascript_api:ViewManagerAction}}\pysiglinewithargsret{\sphinxbfcode{\sphinxupquote{class }}\sphinxbfcode{\sphinxupquote{ViewManagerAction}}}{}{}~\begin{quote}\begin{description}
\item[{Extends}] \leavevmode{\hyperref[\detokenize{reference/javascript_api:web.ActionManager.WidgetAction}]{\sphinxcrossref{
WidgetAction
}}}
\end{description}\end{quote}

Specialization of WidgetAction for window actions (i.e. ViewManagers)


\begin{fulllineitems}
\phantomsection\label{\detokenize{reference/javascript_api:restore}}\pysiglinewithargsret{\sphinxbfcode{\sphinxupquote{method }}\sphinxbfcode{\sphinxupquote{restore}}}{\sphinxoptional{\emph{view\_index}}}{}
Restores a ViewManagerAction
Switches to the requested view by calling select\_view on the ViewManager
\begin{quote}\begin{description}
\item[{Parameters}] \leavevmode\begin{itemize}

\sphinxstylestrong{view\_index} (\sphinxstyleliteralemphasis{\sphinxupquote{int}}) \textendash{} the index of the view to select
\end{itemize}

\end{description}\end{quote}

\end{fulllineitems}



\begin{fulllineitems}
\phantomsection\label{\detokenize{reference/javascript_api:set_on_reverse_breadcrumb}}\pysiglinewithargsret{\sphinxbfcode{\sphinxupquote{method }}\sphinxbfcode{\sphinxupquote{set\_on\_reverse\_breadcrumb}}}{\sphinxoptional{\emph{callback}}\sphinxoptional{, \emph{scrollTop}}}{}
Sets the on\_reverse\_breadcrumb\_callback and the scrollTop to apply when
coming back to that action
\begin{quote}\begin{description}
\item[{Parameters}] \leavevmode\begin{itemize}

\sphinxstylestrong{callback} (\sphinxstyleliteralemphasis{\sphinxupquote{Function}}) \textendash{} the callback

\sphinxstylestrong{scrollTop} (\sphinxstyleliteralemphasis{\sphinxupquote{int}}) \textendash{} the number of pixels to scroll
\end{itemize}

\end{description}\end{quote}

\end{fulllineitems}



\begin{fulllineitems}
\phantomsection\label{\detokenize{reference/javascript_api:setScrollTop}}\pysiglinewithargsret{\sphinxbfcode{\sphinxupquote{method }}\sphinxbfcode{\sphinxupquote{setScrollTop}}}{\sphinxoptional{\emph{scrollTop}}}{}
Sets the scroll position of the widgets’s active\_view
\begin{quote}\begin{description}
\item[{Parameters}] \leavevmode\begin{itemize}

\sphinxstylestrong{scrollTop} (\sphinxstyleliteralemphasis{\sphinxupquote{integer}}) \textendash{} the number of pixels to scroll
\end{itemize}

\end{description}\end{quote}

\end{fulllineitems}



\begin{fulllineitems}
\phantomsection\label{\detokenize{reference/javascript_api:getScrollTop}}\pysiglinewithargsret{\sphinxbfcode{\sphinxupquote{method }}\sphinxbfcode{\sphinxupquote{getScrollTop}}}{}{{ $\rightarrow$ integer}}
Returns the current scrolling offset for the current action.  We have to
ask nicely the question to the active view, because the answer depends
on the view.
\begin{quote}\begin{description}
\item[{Returns}] \leavevmode
the number of pixels the webclient is currently
 scrolled

\item[{Return Type}] \leavevmode
\sphinxstyleliteralemphasis{\sphinxupquote{integer}}

\end{description}\end{quote}

\end{fulllineitems}


\end{fulllineitems}



\begin{fulllineitems}
\phantomsection\label{\detokenize{reference/javascript_api:ActionManager}}\pysiglinewithargsret{\sphinxbfcode{\sphinxupquote{class }}\sphinxbfcode{\sphinxupquote{ActionManager}}}{\emph{parent}, \emph{options}}{}~\begin{quote}\begin{description}
\item[{Extends}] \leavevmode{\hyperref[\detokenize{reference/javascript_api:web.Widget.Widget}]{\sphinxcrossref{
Widget
}}}
\item[{Parameters}] \leavevmode\begin{itemize}

\sphinxstylestrong{parent}

\sphinxstylestrong{options}
\end{itemize}

\end{description}\end{quote}


\begin{fulllineitems}
\phantomsection\label{\detokenize{reference/javascript_api:on_attach_callback}}\pysiglinewithargsret{\sphinxbfcode{\sphinxupquote{method }}\sphinxbfcode{\sphinxupquote{on\_attach\_callback}}}{}{}
Called each time the action manager is attached into the DOM

\end{fulllineitems}



\begin{fulllineitems}
\phantomsection\label{\detokenize{reference/javascript_api:on_detach_callback}}\pysiglinewithargsret{\sphinxbfcode{\sphinxupquote{method }}\sphinxbfcode{\sphinxupquote{on\_detach\_callback}}}{}{}
Called each time the action manager is detached from the DOM

\end{fulllineitems}



\begin{fulllineitems}
\phantomsection\label{\detokenize{reference/javascript_api:push_action}}\pysiglinewithargsret{\sphinxbfcode{\sphinxupquote{method }}\sphinxbfcode{\sphinxupquote{push\_action}}}{\emph{widget}, \emph{action\_descr}, \emph{options}}{}
Add a new action to the action manager
\begin{quote}\begin{description}
\item[{Parameters}] \leavevmode\begin{itemize}

\sphinxstylestrong{widget} ({\hyperref[\detokenize{reference/javascript_api:Widget}]{\sphinxcrossref{\sphinxstyleliteralemphasis{\sphinxupquote{Widget}}}}}) \textendash{} typically, widgets added are openerp.web.ViewManager. The action manager uses the stack of actions to handle the breadcrumbs.

\sphinxstylestrong{action\_descr} (\sphinxstyleliteralemphasis{\sphinxupquote{Object}}) \textendash{} new action description

\sphinxstylestrong{options} ({\hyperref[\detokenize{reference/javascript_api:web.ActionManager.PushActionOptions}]{\sphinxcrossref{\sphinxstyleliteralemphasis{\sphinxupquote{PushActionOptions}}}}})
\end{itemize}

\end{description}\end{quote}


\begin{fulllineitems}
\phantomsection\label{\detokenize{reference/javascript_api:PushActionOptions}}\pysiglinewithargsret{\sphinxbfcode{\sphinxupquote{class }}\sphinxbfcode{\sphinxupquote{PushActionOptions}}}{}{}~

\begin{fulllineitems}
\phantomsection\label{\detokenize{reference/javascript_api:on_reverse_breadcrumb}}\pysigline{\sphinxbfcode{\sphinxupquote{attribute }}\sphinxbfcode{\sphinxupquote{on\_reverse\_breadcrumb}}}
will be called when breadcrumb is clicked on

\end{fulllineitems}



\begin{fulllineitems}
\phantomsection\label{\detokenize{reference/javascript_api:clear_breadcrumbs}}\pysigline{\sphinxbfcode{\sphinxupquote{attribute }}\sphinxbfcode{\sphinxupquote{clear\_breadcrumbs}}}
boolean, if true, action stack is destroyed

\end{fulllineitems}


\end{fulllineitems}


\end{fulllineitems}



\begin{fulllineitems}
\phantomsection\label{\detokenize{reference/javascript_api:do_action}}\pysiglinewithargsret{\sphinxbfcode{\sphinxupquote{function }}\sphinxbfcode{\sphinxupquote{do\_action}}}{\emph{action}\sphinxoptional{, \emph{options}}}{{ $\rightarrow$ jQuery.Deferred}}
Execute an OpenERP action
\begin{quote}\begin{description}
\item[{Parameters}] \leavevmode\begin{itemize}

\sphinxstylestrong{action} (\sphinxstyleliteralemphasis{\sphinxupquote{Number}}\sphinxstyleemphasis{ or }\sphinxstyleliteralemphasis{\sphinxupquote{String}}\sphinxstyleemphasis{ or }\sphinxstyleliteralemphasis{\sphinxupquote{String}}\sphinxstyleemphasis{ or }\sphinxstyleliteralemphasis{\sphinxupquote{Object}}) \textendash{} Can be either an action id, an action XML id, a client action tag or an action descriptor.

\sphinxstylestrong{options} ({\hyperref[\detokenize{reference/javascript_api:web.ActionManager.DoActionOptions}]{\sphinxcrossref{\sphinxstyleliteralemphasis{\sphinxupquote{DoActionOptions}}}}})
\end{itemize}

\item[{Returns}] \leavevmode
Action loaded

\item[{Return Type}] \leavevmode
\sphinxstyleliteralemphasis{\sphinxupquote{jQuery.Deferred}}

\end{description}\end{quote}


\begin{fulllineitems}
\phantomsection\label{\detokenize{reference/javascript_api:DoActionOptions}}\pysiglinewithargsret{\sphinxbfcode{\sphinxupquote{class }}\sphinxbfcode{\sphinxupquote{DoActionOptions}}}{}{}~

\begin{fulllineitems}
\phantomsection\label{\detokenize{reference/javascript_api:clear_breadcrumbs}}\pysigline{\sphinxbfcode{\sphinxupquote{attribute }}\sphinxbfcode{\sphinxupquote{clear\_breadcrumbs}} Boolean}
Clear the breadcrumbs history list

\end{fulllineitems}



\begin{fulllineitems}
\phantomsection\label{\detokenize{reference/javascript_api:replace_breadcrumb}}\pysigline{\sphinxbfcode{\sphinxupquote{attribute }}\sphinxbfcode{\sphinxupquote{replace\_breadcrumb}} Boolean}
Replace the current breadcrumb with the action

\end{fulllineitems}



\begin{fulllineitems}
\phantomsection\label{\detokenize{reference/javascript_api:on_reverse_breadcrumb}}\pysigline{\sphinxbfcode{\sphinxupquote{attribute }}\sphinxbfcode{\sphinxupquote{on\_reverse\_breadcrumb}} Function}
Callback to be executed whenever an anterior breadcrumb item is clicked on.

\end{fulllineitems}



\begin{fulllineitems}
\phantomsection\label{\detokenize{reference/javascript_api:hide_breadcrumb}}\pysigline{\sphinxbfcode{\sphinxupquote{attribute }}\sphinxbfcode{\sphinxupquote{hide\_breadcrumb}} Function}
Do not display this widget’s title in the breadcrumb

\end{fulllineitems}



\begin{fulllineitems}
\phantomsection\label{\detokenize{reference/javascript_api:on_close}}\pysigline{\sphinxbfcode{\sphinxupquote{attribute }}\sphinxbfcode{\sphinxupquote{on\_close}} Function}
Callback to be executed when the dialog is closed (only relevant for target=new actions)

\end{fulllineitems}



\begin{fulllineitems}
\phantomsection\label{\detokenize{reference/javascript_api:action_menu_id}}\pysigline{\sphinxbfcode{\sphinxupquote{attribute }}\sphinxbfcode{\sphinxupquote{action\_menu\_id}} Function}
Manually set the menu id on the fly.

\end{fulllineitems}



\begin{fulllineitems}
\phantomsection\label{\detokenize{reference/javascript_api:additional_context}}\pysigline{\sphinxbfcode{\sphinxupquote{attribute }}\sphinxbfcode{\sphinxupquote{additional\_context}} Object}
Additional context to be merged with the action’s context.

\end{fulllineitems}


\end{fulllineitems}


\end{fulllineitems}


\end{fulllineitems}



\begin{fulllineitems}
\phantomsection\label{\detokenize{reference/javascript_api:Action}}\pysiglinewithargsret{\sphinxbfcode{\sphinxupquote{class }}\sphinxbfcode{\sphinxupquote{Action}}}{\emph{action}}{}~\begin{quote}\begin{description}
\item[{Extends}] \leavevmode{\hyperref[\detokenize{reference/javascript_api:web.Class.Class}]{\sphinxcrossref{
Class
}}}
\item[{Parameters}] \leavevmode\begin{itemize}

\sphinxstylestrong{action}
\end{itemize}

\end{description}\end{quote}

Class representing the actions of the ActionManager
Basic implementation for client actions that are functions


\begin{fulllineitems}
\phantomsection\label{\detokenize{reference/javascript_api:restore}}\pysiglinewithargsret{\sphinxbfcode{\sphinxupquote{method }}\sphinxbfcode{\sphinxupquote{restore}}}{}{{ $\rightarrow$ Deferred}}
This method should restore this previously loaded action
Calls on\_reverse\_breadcrumb\_callback if defined
\begin{quote}\begin{description}
\item[{Returns}] \leavevmode
resolved when widget is enabled

\item[{Return Type}] \leavevmode
\sphinxstyleliteralemphasis{\sphinxupquote{Deferred}}

\end{description}\end{quote}

\end{fulllineitems}



\begin{fulllineitems}
\phantomsection\label{\detokenize{reference/javascript_api:detach}}\pysiglinewithargsret{\sphinxbfcode{\sphinxupquote{method }}\sphinxbfcode{\sphinxupquote{detach}}}{}{}
There is nothing to detach in the case of a client function

\end{fulllineitems}



\begin{fulllineitems}
\phantomsection\label{\detokenize{reference/javascript_api:destroy}}\pysiglinewithargsret{\sphinxbfcode{\sphinxupquote{method }}\sphinxbfcode{\sphinxupquote{destroy}}}{}{}
Destroyer: there is nothing to destroy in the case of a client function

\end{fulllineitems}



\begin{fulllineitems}
\phantomsection\label{\detokenize{reference/javascript_api:set_on_reverse_breadcrumb}}\pysiglinewithargsret{\sphinxbfcode{\sphinxupquote{method }}\sphinxbfcode{\sphinxupquote{set\_on\_reverse\_breadcrumb}}}{\sphinxoptional{\emph{callback}}}{}
Sets the on\_reverse\_breadcrumb\_callback to be called when coming back to that action
\begin{quote}\begin{description}
\item[{Parameters}] \leavevmode\begin{itemize}

\sphinxstylestrong{callback} (\sphinxstyleliteralemphasis{\sphinxupquote{Function}}) \textendash{} the callback
\end{itemize}

\end{description}\end{quote}

\end{fulllineitems}



\begin{fulllineitems}
\phantomsection\label{\detokenize{reference/javascript_api:setScrollTop}}\pysiglinewithargsret{\sphinxbfcode{\sphinxupquote{method }}\sphinxbfcode{\sphinxupquote{setScrollTop}}}{}{}
Not implemented for client actions

\end{fulllineitems}



\begin{fulllineitems}
\phantomsection\label{\detokenize{reference/javascript_api:set_fragment}}\pysiglinewithargsret{\sphinxbfcode{\sphinxupquote{method }}\sphinxbfcode{\sphinxupquote{set\_fragment}}}{\sphinxoptional{\emph{\$fragment}}}{}
Stores the DOM fragment of the action
\begin{quote}\begin{description}
\item[{Parameters}] \leavevmode\begin{itemize}

\sphinxstylestrong{\$fragment} (\sphinxstyleliteralemphasis{\sphinxupquote{jQuery}}) \textendash{} the DOM fragment
\end{itemize}

\end{description}\end{quote}

\end{fulllineitems}



\begin{fulllineitems}
\phantomsection\label{\detokenize{reference/javascript_api:getScrollTop}}\pysiglinewithargsret{\sphinxbfcode{\sphinxupquote{method }}\sphinxbfcode{\sphinxupquote{getScrollTop}}}{}{{ $\rightarrow$ int}}
Not implemented for client actions
\begin{quote}\begin{description}
\item[{Returns}] \leavevmode
the number of pixels the webclient is scrolled when leaving the action

\item[{Return Type}] \leavevmode
\sphinxstyleliteralemphasis{\sphinxupquote{int}}

\end{description}\end{quote}

\end{fulllineitems}


\end{fulllineitems}


\end{fulllineitems}

\phantomsection\label{\detokenize{reference/javascript_api:module-web.WebClient}}

\begin{fulllineitems}
\phantomsection\label{\detokenize{reference/javascript_api:web.WebClient}}\pysigline{\sphinxbfcode{\sphinxupquote{module }}\sphinxbfcode{\sphinxupquote{web.WebClient}}}~~\begin{quote}\begin{description}
\item[{Exports}] \leavevmode{\hyperref[\detokenize{reference/javascript_api:web.WebClient.}]{\sphinxcrossref{
\textless{}anonymous\textgreater{}
}}}
\item[{Depends On}] \leavevmode\begin{itemize}
\item {} {\hyperref[\detokenize{reference/javascript_api:web.AbstractWebClient}]{\sphinxcrossref{
web.AbstractWebClient
}}}
\item {} {\hyperref[\detokenize{reference/javascript_api:web.Menu}]{\sphinxcrossref{
web.Menu
}}}
\item {} {\hyperref[\detokenize{reference/javascript_api:web.SystrayMenu}]{\sphinxcrossref{
web.SystrayMenu
}}}
\item {} {\hyperref[\detokenize{reference/javascript_api:web.UserMenu}]{\sphinxcrossref{
web.UserMenu
}}}
\item {} {\hyperref[\detokenize{reference/javascript_api:web.core}]{\sphinxcrossref{
web.core
}}}
\item {} {\hyperref[\detokenize{reference/javascript_api:web.data_manager}]{\sphinxcrossref{
web.data\_manager
}}}
\item {} {\hyperref[\detokenize{reference/javascript_api:web.framework}]{\sphinxcrossref{
web.framework
}}}
\item {} {\hyperref[\detokenize{reference/javascript_api:web.session}]{\sphinxcrossref{
web.session
}}}
\end{itemize}

\end{description}\end{quote}


\begin{fulllineitems}
\phantomsection\label{\detokenize{reference/javascript_api:web.WebClient.}}\pysiglinewithargsret{\sphinxbfcode{\sphinxupquote{class }}\sphinxbfcode{\sphinxupquote{}}}{}{}~\begin{quote}\begin{description}
\item[{Extends}] \leavevmode{\hyperref[\detokenize{reference/javascript_api:web.AbstractWebClient.AbstractWebClient}]{\sphinxcrossref{
AbstractWebClient
}}}
\end{description}\end{quote}

\end{fulllineitems}


\end{fulllineitems}

\phantomsection\label{\detokenize{reference/javascript_api:module-web.Widget}}

\begin{fulllineitems}
\phantomsection\label{\detokenize{reference/javascript_api:web.Widget}}\pysigline{\sphinxbfcode{\sphinxupquote{module }}\sphinxbfcode{\sphinxupquote{web.Widget}}}~~\begin{quote}\begin{description}
\item[{Exports}] \leavevmode{\hyperref[\detokenize{reference/javascript_api:web.Widget.Widget}]{\sphinxcrossref{
Widget
}}}
\item[{Depends On}] \leavevmode\begin{itemize}
\item {} {\hyperref[\detokenize{reference/javascript_api:web.ServicesMixin}]{\sphinxcrossref{
web.ServicesMixin
}}}
\item {} {\hyperref[\detokenize{reference/javascript_api:web.ajax}]{\sphinxcrossref{
web.ajax
}}}
\item {} {\hyperref[\detokenize{reference/javascript_api:web.core}]{\sphinxcrossref{
web.core
}}}
\item {} {\hyperref[\detokenize{reference/javascript_api:web.mixins}]{\sphinxcrossref{
web.mixins
}}}
\end{itemize}

\end{description}\end{quote}


\begin{fulllineitems}
\phantomsection\label{\detokenize{reference/javascript_api:Widget}}\pysiglinewithargsret{\sphinxbfcode{\sphinxupquote{class }}\sphinxbfcode{\sphinxupquote{Widget}}}{\emph{parent}}{}~\begin{quote}\begin{description}
\item[{Extends}] \leavevmode{\hyperref[\detokenize{reference/javascript_api:web.Class.Class}]{\sphinxcrossref{
Class
}}}
\item[{Mixes}] \leavevmode\begin{itemize}
\item {} 
PropertiesMixin

\item {} {\hyperref[\detokenize{reference/javascript_api:web.ServicesMixin.ServicesMixin}]{\sphinxcrossref{
ServicesMixin
}}}
\end{itemize}

\item[{Parameters}] \leavevmode\begin{itemize}

\sphinxstylestrong{parent} (\sphinxstyleliteralemphasis{\sphinxupquote{openerp.Widget}}) \textendash{} Binds the current instance to the given Widget instance.
When that widget is destroyed by calling destroy(), the current instance will be
destroyed too. Can be null.
\end{itemize}

\end{description}\end{quote}

Base class for all visual components. Provides a lot of functions helpful
for the management of a part of the DOM.

Widget handles:
\begin{itemize}
\item {} 
Rendering with QWeb.

\item {} 
Life-cycle management and parenting (when a parent is destroyed, all its
children are destroyed too).

\item {} 
Insertion in DOM.

\end{itemize}

\sphinxstylestrong{Guide to create implementations of the Widget class}

Here is a sample child class:

\fvset{hllines={, ,}}%
\begin{sphinxVerbatim}[commandchars=\\\{\}]
\PYG{k+kd}{var} \PYG{n+nx}{MyWidget} \PYG{o}{=} \PYG{n+nx}{openerp}\PYG{p}{.}\PYG{n+nx}{base}\PYG{p}{.}\PYG{n+nx}{Widget}\PYG{p}{.}\PYG{n+nx}{extend}\PYG{p}{(}\PYG{p}{\PYGZob{}}
    \PYG{c+c1}{// the name of the QWeb template to use for rendering}
    \PYG{n+nx}{template}\PYG{o}{:} \PYG{l+s+s2}{\PYGZdq{}MyQWebTemplate\PYGZdq{}}\PYG{p}{,}

    \PYG{n+nx}{init}\PYG{o}{:} \PYG{k+kd}{function}\PYG{p}{(}\PYG{n+nx}{parent}\PYG{p}{)} \PYG{p}{\PYGZob{}}
        \PYG{k}{this}\PYG{p}{.}\PYG{n+nx}{\PYGZus{}super}\PYG{p}{(}\PYG{n+nx}{parent}\PYG{p}{)}\PYG{p}{;}
        \PYG{c+c1}{// stuff that you want to init before the rendering}
    \PYG{p}{\PYGZcb{}}\PYG{p}{,}
    \PYG{n+nx}{start}\PYG{o}{:} \PYG{k+kd}{function}\PYG{p}{(}\PYG{p}{)} \PYG{p}{\PYGZob{}}
        \PYG{c+c1}{// stuff you want to make after the rendering, {}`this.\PYGZdl{}el{}` holds a correct value}
        \PYG{k}{this}\PYG{p}{.}\PYG{n+nx}{\PYGZdl{}}\PYG{p}{(}\PYG{l+s+s2}{\PYGZdq{}.my\PYGZus{}button\PYGZdq{}}\PYG{p}{)}\PYG{p}{.}\PYG{n+nx}{click}\PYG{p}{(}\PYG{l+s+sr}{/* an example of event binding * /}\PYG{p}{)}\PYG{p}{;}

        \PYG{c+c1}{// if you have some asynchronous operations, it\PYGZsq{}s a good idea to return}
        \PYG{c+c1}{// a promise in start()}
        \PYG{k+kd}{var} \PYG{n+nx}{promise} \PYG{o}{=} \PYG{k}{this}\PYG{p}{.}\PYG{n+nx}{\PYGZus{}rpc}\PYG{p}{(}\PYG{p}{...}\PYG{p}{)}\PYG{p}{;}
        \PYG{k}{return} \PYG{n+nx}{promise}\PYG{p}{;}
    \PYG{p}{\PYGZcb{}}
\PYG{p}{\PYGZcb{}}\PYG{p}{)}\PYG{p}{;}
\end{sphinxVerbatim}

Now this class can simply be used with the following syntax:

\fvset{hllines={, ,}}%
\begin{sphinxVerbatim}[commandchars=\\\{\}]
\PYG{k+kd}{var} \PYG{n+nx}{my\PYGZus{}widget} \PYG{o}{=} \PYG{k}{new} \PYG{n+nx}{MyWidget}\PYG{p}{(}\PYG{k}{this}\PYG{p}{)}\PYG{p}{;}
\PYG{n+nx}{my\PYGZus{}widget}\PYG{p}{.}\PYG{n+nx}{appendTo}\PYG{p}{(}\PYG{n+nx}{\PYGZdl{}}\PYG{p}{(}\PYG{l+s+s2}{\PYGZdq{}.some\PYGZhy{}div\PYGZdq{}}\PYG{p}{)}\PYG{p}{)}\PYG{p}{;}
\end{sphinxVerbatim}

With these two lines, the MyWidget instance was initialized, rendered,
inserted into the DOM inside the \sphinxcode{\sphinxupquote{.some-div}} div and its events were
bound.

And of course, when you don’t need that widget anymore, just do:

\fvset{hllines={, ,}}%
\begin{sphinxVerbatim}[commandchars=\\\{\}]
\PYG{n+nx}{my\PYGZus{}widget}\PYG{p}{.}\PYG{n+nx}{destroy}\PYG{p}{(}\PYG{p}{)}\PYG{p}{;}
\end{sphinxVerbatim}

That will kill the widget in a clean way and erase its content from the dom.


\begin{fulllineitems}
\phantomsection\label{\detokenize{reference/javascript_api:template}}\pysigline{\sphinxbfcode{\sphinxupquote{attribute }}\sphinxbfcode{\sphinxupquote{template}} String}
The name of the QWeb template that will be used for rendering. Must be
redefined in subclasses or the default render() method can not be used.

\end{fulllineitems}



\begin{fulllineitems}
\phantomsection\label{\detokenize{reference/javascript_api:xmlDependencies}}\pysigline{\sphinxbfcode{\sphinxupquote{attribute }}\sphinxbfcode{\sphinxupquote{xmlDependencies}} string{[}{]}}
List of paths to xml files that need to be loaded before the widget can
be rendered. This will not induce loading anything that has already been
loaded.

\end{fulllineitems}



\begin{fulllineitems}
\phantomsection\label{\detokenize{reference/javascript_api:willStart}}\pysiglinewithargsret{\sphinxbfcode{\sphinxupquote{method }}\sphinxbfcode{\sphinxupquote{willStart}}}{}{{ $\rightarrow$ Deferred}}
Method called between @see init and @see start. Performs asynchronous
calls required by the rendering and the start method.

This method should return a Deferred which is resolved when start can be
executed.
\begin{quote}\begin{description}
\item[{Return Type}] \leavevmode
\sphinxstyleliteralemphasis{\sphinxupquote{Deferred}}

\end{description}\end{quote}

\end{fulllineitems}



\begin{fulllineitems}
\phantomsection\label{\detokenize{reference/javascript_api:destroy}}\pysiglinewithargsret{\sphinxbfcode{\sphinxupquote{method }}\sphinxbfcode{\sphinxupquote{destroy}}}{}{}
Destroys the current widget, also destroys all its children before destroying itself.

\end{fulllineitems}



\begin{fulllineitems}
\phantomsection\label{\detokenize{reference/javascript_api:appendTo}}\pysiglinewithargsret{\sphinxbfcode{\sphinxupquote{method }}\sphinxbfcode{\sphinxupquote{appendTo}}}{\emph{target}}{}
Renders the current widget and appends it to the given jQuery object or Widget.
\begin{quote}\begin{description}
\item[{Parameters}] \leavevmode\begin{itemize}

\sphinxstylestrong{target} \textendash{} A jQuery object or a Widget instance.
\end{itemize}

\end{description}\end{quote}

\end{fulllineitems}



\begin{fulllineitems}
\phantomsection\label{\detokenize{reference/javascript_api:prependTo}}\pysiglinewithargsret{\sphinxbfcode{\sphinxupquote{method }}\sphinxbfcode{\sphinxupquote{prependTo}}}{\emph{target}}{}
Renders the current widget and prepends it to the given jQuery object or Widget.
\begin{quote}\begin{description}
\item[{Parameters}] \leavevmode\begin{itemize}

\sphinxstylestrong{target} \textendash{} A jQuery object or a Widget instance.
\end{itemize}

\end{description}\end{quote}

\end{fulllineitems}



\begin{fulllineitems}
\phantomsection\label{\detokenize{reference/javascript_api:insertAfter}}\pysiglinewithargsret{\sphinxbfcode{\sphinxupquote{method }}\sphinxbfcode{\sphinxupquote{insertAfter}}}{\emph{target}}{}
Renders the current widget and inserts it after to the given jQuery object or Widget.
\begin{quote}\begin{description}
\item[{Parameters}] \leavevmode\begin{itemize}

\sphinxstylestrong{target} \textendash{} A jQuery object or a Widget instance.
\end{itemize}

\end{description}\end{quote}

\end{fulllineitems}



\begin{fulllineitems}
\phantomsection\label{\detokenize{reference/javascript_api:insertBefore}}\pysiglinewithargsret{\sphinxbfcode{\sphinxupquote{method }}\sphinxbfcode{\sphinxupquote{insertBefore}}}{\emph{target}}{}
Renders the current widget and inserts it before to the given jQuery object or Widget.
\begin{quote}\begin{description}
\item[{Parameters}] \leavevmode\begin{itemize}

\sphinxstylestrong{target} \textendash{} A jQuery object or a Widget instance.
\end{itemize}

\end{description}\end{quote}

\end{fulllineitems}



\begin{fulllineitems}
\phantomsection\label{\detokenize{reference/javascript_api:attachTo}}\pysiglinewithargsret{\sphinxbfcode{\sphinxupquote{method }}\sphinxbfcode{\sphinxupquote{attachTo}}}{\emph{target}}{}
Attach the current widget to a dom element
\begin{quote}\begin{description}
\item[{Parameters}] \leavevmode\begin{itemize}

\sphinxstylestrong{target} \textendash{} A jQuery object or a Widget instance.
\end{itemize}

\end{description}\end{quote}

\end{fulllineitems}



\begin{fulllineitems}
\phantomsection\label{\detokenize{reference/javascript_api:replace}}\pysiglinewithargsret{\sphinxbfcode{\sphinxupquote{method }}\sphinxbfcode{\sphinxupquote{replace}}}{\emph{target}}{}
Renders the current widget and replaces the given jQuery object.
\begin{quote}\begin{description}
\item[{Parameters}] \leavevmode\begin{itemize}

\sphinxstylestrong{target} \textendash{} A jQuery object or a Widget instance.
\end{itemize}

\end{description}\end{quote}

\end{fulllineitems}



\begin{fulllineitems}
\phantomsection\label{\detokenize{reference/javascript_api:start}}\pysiglinewithargsret{\sphinxbfcode{\sphinxupquote{method }}\sphinxbfcode{\sphinxupquote{start}}}{}{{ $\rightarrow$ jQuery.Deferred or any}}
Method called after rendering. Mostly used to bind actions, perform asynchronous
calls, etc…

By convention, this method should return an object that can be passed to \$.when()
to inform the caller when this widget has been initialized.
\begin{quote}\begin{description}
\item[{Return Type}] \leavevmode
\sphinxstyleliteralemphasis{\sphinxupquote{jQuery.Deferred}}\sphinxstyleemphasis{ or }\sphinxstyleliteralemphasis{\sphinxupquote{any}}

\end{description}\end{quote}

\end{fulllineitems}



\begin{fulllineitems}
\phantomsection\label{\detokenize{reference/javascript_api:renderElement}}\pysiglinewithargsret{\sphinxbfcode{\sphinxupquote{method }}\sphinxbfcode{\sphinxupquote{renderElement}}}{}{}
Renders the element. The default implementation renders the widget using QWeb,
\sphinxcode{\sphinxupquote{this.template}} must be defined. The context given to QWeb contains the “widget”
key that references \sphinxcode{\sphinxupquote{this}}.

\end{fulllineitems}



\begin{fulllineitems}
\phantomsection\label{\detokenize{reference/javascript_api:replaceElement}}\pysiglinewithargsret{\sphinxbfcode{\sphinxupquote{method }}\sphinxbfcode{\sphinxupquote{replaceElement}}}{\emph{\$el}}{{ $\rightarrow$ Widget}}
Re-sets the widget’s root element and replaces the old root element
(if any) by the new one in the DOM.
\begin{quote}\begin{description}
\item[{Parameters}] \leavevmode\begin{itemize}

\sphinxstylestrong{\$el} (\sphinxstyleliteralemphasis{\sphinxupquote{HTMLElement}}\sphinxstyleemphasis{ or }\sphinxstyleliteralemphasis{\sphinxupquote{jQuery}})
\end{itemize}

\item[{Returns}] \leavevmode
this

\item[{Return Type}] \leavevmode
{\hyperref[\detokenize{reference/javascript_api:web.Widget.Widget}]{\sphinxcrossref{\sphinxstyleliteralemphasis{\sphinxupquote{Widget}}}}}

\end{description}\end{quote}

\end{fulllineitems}



\begin{fulllineitems}
\phantomsection\label{\detokenize{reference/javascript_api:setElement}}\pysiglinewithargsret{\sphinxbfcode{\sphinxupquote{method }}\sphinxbfcode{\sphinxupquote{setElement}}}{\emph{element}}{{ $\rightarrow$ Widget}}
Re-sets the widget’s root element (el/\$el/\$el).

Includes:
\begin{itemize}
\item {} 
re-delegating events

\item {} 
re-binding sub-elements

\item {} 
if the widget already had a root element, replacing the pre-existing
element in the DOM

\end{itemize}
\begin{quote}\begin{description}
\item[{Parameters}] \leavevmode\begin{itemize}

\sphinxstylestrong{element} (\sphinxstyleliteralemphasis{\sphinxupquote{HTMLElement}}\sphinxstyleemphasis{ or }\sphinxstyleliteralemphasis{\sphinxupquote{jQuery}}) \textendash{} new root element for the widget
\end{itemize}

\item[{Returns}] \leavevmode
this

\item[{Return Type}] \leavevmode
{\hyperref[\detokenize{reference/javascript_api:web.Widget.Widget}]{\sphinxcrossref{\sphinxstyleliteralemphasis{\sphinxupquote{Widget}}}}}

\end{description}\end{quote}

\end{fulllineitems}



\begin{fulllineitems}
\phantomsection\label{\detokenize{reference/javascript_api:make}}\pysiglinewithargsret{\sphinxbfcode{\sphinxupquote{method }}\sphinxbfcode{\sphinxupquote{make}}}{\emph{tagName}\sphinxoptional{, \emph{attributes}}\sphinxoptional{, \emph{content}}}{{ $\rightarrow$ Element}}
Utility function to build small DOM elements.
\begin{quote}\begin{description}
\item[{Parameters}] \leavevmode\begin{itemize}

\sphinxstylestrong{tagName} (\sphinxstyleliteralemphasis{\sphinxupquote{String}}) \textendash{} name of the DOM element to create

\sphinxstylestrong{attributes} (\sphinxstyleliteralemphasis{\sphinxupquote{Object}}) \textendash{} map of DOM attributes to set on the element

\sphinxstylestrong{content} (\sphinxstyleliteralemphasis{\sphinxupquote{String}}) \textendash{} HTML content to set on the element
\end{itemize}

\item[{Return Type}] \leavevmode
\sphinxstyleliteralemphasis{\sphinxupquote{Element}}

\end{description}\end{quote}

\end{fulllineitems}



\begin{fulllineitems}
\phantomsection\label{\detokenize{reference/javascript_api:_}}\pysiglinewithargsret{\sphinxbfcode{\sphinxupquote{method }}\sphinxbfcode{\sphinxupquote{\$}}}{\emph{selector}}{{ $\rightarrow$ jQuery}}
Shortcut for \sphinxcode{\sphinxupquote{this.\$el.find(selector)}}
\begin{quote}\begin{description}
\item[{Parameters}] \leavevmode\begin{itemize}

\sphinxstylestrong{selector} (\sphinxstyleliteralemphasis{\sphinxupquote{String}}) \textendash{} CSS selector, rooted in \$el
\end{itemize}

\item[{Returns}] \leavevmode
selector match

\item[{Return Type}] \leavevmode
\sphinxstyleliteralemphasis{\sphinxupquote{jQuery}}

\end{description}\end{quote}

\end{fulllineitems}



\begin{fulllineitems}
\phantomsection\label{\detokenize{reference/javascript_api:do_show}}\pysiglinewithargsret{\sphinxbfcode{\sphinxupquote{method }}\sphinxbfcode{\sphinxupquote{do\_show}}}{}{}
Displays the widget

\end{fulllineitems}



\begin{fulllineitems}
\phantomsection\label{\detokenize{reference/javascript_api:do_hide}}\pysiglinewithargsret{\sphinxbfcode{\sphinxupquote{method }}\sphinxbfcode{\sphinxupquote{do\_hide}}}{}{}
Hides the widget

\end{fulllineitems}



\begin{fulllineitems}
\phantomsection\label{\detokenize{reference/javascript_api:do_toggle}}\pysiglinewithargsret{\sphinxbfcode{\sphinxupquote{method }}\sphinxbfcode{\sphinxupquote{do\_toggle}}}{\sphinxoptional{\emph{display}}}{}
Displays or hides the widget
\begin{quote}\begin{description}
\item[{Parameters}] \leavevmode\begin{itemize}

\sphinxstylestrong{display} (\sphinxstyleliteralemphasis{\sphinxupquote{Boolean}}) \textendash{} use true to show the widget or false to hide it
\end{itemize}

\end{description}\end{quote}

\end{fulllineitems}


\end{fulllineitems}



\begin{fulllineitems}
\phantomsection\label{\detokenize{reference/javascript_api:Widget}}\pysiglinewithargsret{\sphinxbfcode{\sphinxupquote{class }}\sphinxbfcode{\sphinxupquote{Widget}}}{\emph{parent}}{}~\begin{quote}\begin{description}
\item[{Extends}] \leavevmode{\hyperref[\detokenize{reference/javascript_api:web.Class.Class}]{\sphinxcrossref{
Class
}}}
\item[{Mixes}] \leavevmode\begin{itemize}
\item {} 
PropertiesMixin

\item {} {\hyperref[\detokenize{reference/javascript_api:web.ServicesMixin.ServicesMixin}]{\sphinxcrossref{
ServicesMixin
}}}
\end{itemize}

\item[{Parameters}] \leavevmode\begin{itemize}

\sphinxstylestrong{parent} (\sphinxstyleliteralemphasis{\sphinxupquote{openerp.Widget}}) \textendash{} Binds the current instance to the given Widget instance.
When that widget is destroyed by calling destroy(), the current instance will be
destroyed too. Can be null.
\end{itemize}

\end{description}\end{quote}

Base class for all visual components. Provides a lot of functions helpful
for the management of a part of the DOM.

Widget handles:
\begin{itemize}
\item {} 
Rendering with QWeb.

\item {} 
Life-cycle management and parenting (when a parent is destroyed, all its
children are destroyed too).

\item {} 
Insertion in DOM.

\end{itemize}

\sphinxstylestrong{Guide to create implementations of the Widget class}

Here is a sample child class:

\fvset{hllines={, ,}}%
\begin{sphinxVerbatim}[commandchars=\\\{\}]
\PYG{k+kd}{var} \PYG{n+nx}{MyWidget} \PYG{o}{=} \PYG{n+nx}{openerp}\PYG{p}{.}\PYG{n+nx}{base}\PYG{p}{.}\PYG{n+nx}{Widget}\PYG{p}{.}\PYG{n+nx}{extend}\PYG{p}{(}\PYG{p}{\PYGZob{}}
    \PYG{c+c1}{// the name of the QWeb template to use for rendering}
    \PYG{n+nx}{template}\PYG{o}{:} \PYG{l+s+s2}{\PYGZdq{}MyQWebTemplate\PYGZdq{}}\PYG{p}{,}

    \PYG{n+nx}{init}\PYG{o}{:} \PYG{k+kd}{function}\PYG{p}{(}\PYG{n+nx}{parent}\PYG{p}{)} \PYG{p}{\PYGZob{}}
        \PYG{k}{this}\PYG{p}{.}\PYG{n+nx}{\PYGZus{}super}\PYG{p}{(}\PYG{n+nx}{parent}\PYG{p}{)}\PYG{p}{;}
        \PYG{c+c1}{// stuff that you want to init before the rendering}
    \PYG{p}{\PYGZcb{}}\PYG{p}{,}
    \PYG{n+nx}{start}\PYG{o}{:} \PYG{k+kd}{function}\PYG{p}{(}\PYG{p}{)} \PYG{p}{\PYGZob{}}
        \PYG{c+c1}{// stuff you want to make after the rendering, {}`this.\PYGZdl{}el{}` holds a correct value}
        \PYG{k}{this}\PYG{p}{.}\PYG{n+nx}{\PYGZdl{}}\PYG{p}{(}\PYG{l+s+s2}{\PYGZdq{}.my\PYGZus{}button\PYGZdq{}}\PYG{p}{)}\PYG{p}{.}\PYG{n+nx}{click}\PYG{p}{(}\PYG{l+s+sr}{/* an example of event binding * /}\PYG{p}{)}\PYG{p}{;}

        \PYG{c+c1}{// if you have some asynchronous operations, it\PYGZsq{}s a good idea to return}
        \PYG{c+c1}{// a promise in start()}
        \PYG{k+kd}{var} \PYG{n+nx}{promise} \PYG{o}{=} \PYG{k}{this}\PYG{p}{.}\PYG{n+nx}{\PYGZus{}rpc}\PYG{p}{(}\PYG{p}{...}\PYG{p}{)}\PYG{p}{;}
        \PYG{k}{return} \PYG{n+nx}{promise}\PYG{p}{;}
    \PYG{p}{\PYGZcb{}}
\PYG{p}{\PYGZcb{}}\PYG{p}{)}\PYG{p}{;}
\end{sphinxVerbatim}

Now this class can simply be used with the following syntax:

\fvset{hllines={, ,}}%
\begin{sphinxVerbatim}[commandchars=\\\{\}]
\PYG{k+kd}{var} \PYG{n+nx}{my\PYGZus{}widget} \PYG{o}{=} \PYG{k}{new} \PYG{n+nx}{MyWidget}\PYG{p}{(}\PYG{k}{this}\PYG{p}{)}\PYG{p}{;}
\PYG{n+nx}{my\PYGZus{}widget}\PYG{p}{.}\PYG{n+nx}{appendTo}\PYG{p}{(}\PYG{n+nx}{\PYGZdl{}}\PYG{p}{(}\PYG{l+s+s2}{\PYGZdq{}.some\PYGZhy{}div\PYGZdq{}}\PYG{p}{)}\PYG{p}{)}\PYG{p}{;}
\end{sphinxVerbatim}

With these two lines, the MyWidget instance was initialized, rendered,
inserted into the DOM inside the \sphinxcode{\sphinxupquote{.some-div}} div and its events were
bound.

And of course, when you don’t need that widget anymore, just do:

\fvset{hllines={, ,}}%
\begin{sphinxVerbatim}[commandchars=\\\{\}]
\PYG{n+nx}{my\PYGZus{}widget}\PYG{p}{.}\PYG{n+nx}{destroy}\PYG{p}{(}\PYG{p}{)}\PYG{p}{;}
\end{sphinxVerbatim}

That will kill the widget in a clean way and erase its content from the dom.


\begin{fulllineitems}
\phantomsection\label{\detokenize{reference/javascript_api:template}}\pysigline{\sphinxbfcode{\sphinxupquote{attribute }}\sphinxbfcode{\sphinxupquote{template}} String}
The name of the QWeb template that will be used for rendering. Must be
redefined in subclasses or the default render() method can not be used.

\end{fulllineitems}



\begin{fulllineitems}
\phantomsection\label{\detokenize{reference/javascript_api:xmlDependencies}}\pysigline{\sphinxbfcode{\sphinxupquote{attribute }}\sphinxbfcode{\sphinxupquote{xmlDependencies}} string{[}{]}}
List of paths to xml files that need to be loaded before the widget can
be rendered. This will not induce loading anything that has already been
loaded.

\end{fulllineitems}



\begin{fulllineitems}
\phantomsection\label{\detokenize{reference/javascript_api:willStart}}\pysiglinewithargsret{\sphinxbfcode{\sphinxupquote{method }}\sphinxbfcode{\sphinxupquote{willStart}}}{}{{ $\rightarrow$ Deferred}}
Method called between @see init and @see start. Performs asynchronous
calls required by the rendering and the start method.

This method should return a Deferred which is resolved when start can be
executed.
\begin{quote}\begin{description}
\item[{Return Type}] \leavevmode
\sphinxstyleliteralemphasis{\sphinxupquote{Deferred}}

\end{description}\end{quote}

\end{fulllineitems}



\begin{fulllineitems}
\phantomsection\label{\detokenize{reference/javascript_api:destroy}}\pysiglinewithargsret{\sphinxbfcode{\sphinxupquote{method }}\sphinxbfcode{\sphinxupquote{destroy}}}{}{}
Destroys the current widget, also destroys all its children before destroying itself.

\end{fulllineitems}



\begin{fulllineitems}
\phantomsection\label{\detokenize{reference/javascript_api:appendTo}}\pysiglinewithargsret{\sphinxbfcode{\sphinxupquote{method }}\sphinxbfcode{\sphinxupquote{appendTo}}}{\emph{target}}{}
Renders the current widget and appends it to the given jQuery object or Widget.
\begin{quote}\begin{description}
\item[{Parameters}] \leavevmode\begin{itemize}

\sphinxstylestrong{target} \textendash{} A jQuery object or a Widget instance.
\end{itemize}

\end{description}\end{quote}

\end{fulllineitems}



\begin{fulllineitems}
\phantomsection\label{\detokenize{reference/javascript_api:prependTo}}\pysiglinewithargsret{\sphinxbfcode{\sphinxupquote{method }}\sphinxbfcode{\sphinxupquote{prependTo}}}{\emph{target}}{}
Renders the current widget and prepends it to the given jQuery object or Widget.
\begin{quote}\begin{description}
\item[{Parameters}] \leavevmode\begin{itemize}

\sphinxstylestrong{target} \textendash{} A jQuery object or a Widget instance.
\end{itemize}

\end{description}\end{quote}

\end{fulllineitems}



\begin{fulllineitems}
\phantomsection\label{\detokenize{reference/javascript_api:insertAfter}}\pysiglinewithargsret{\sphinxbfcode{\sphinxupquote{method }}\sphinxbfcode{\sphinxupquote{insertAfter}}}{\emph{target}}{}
Renders the current widget and inserts it after to the given jQuery object or Widget.
\begin{quote}\begin{description}
\item[{Parameters}] \leavevmode\begin{itemize}

\sphinxstylestrong{target} \textendash{} A jQuery object or a Widget instance.
\end{itemize}

\end{description}\end{quote}

\end{fulllineitems}



\begin{fulllineitems}
\phantomsection\label{\detokenize{reference/javascript_api:insertBefore}}\pysiglinewithargsret{\sphinxbfcode{\sphinxupquote{method }}\sphinxbfcode{\sphinxupquote{insertBefore}}}{\emph{target}}{}
Renders the current widget and inserts it before to the given jQuery object or Widget.
\begin{quote}\begin{description}
\item[{Parameters}] \leavevmode\begin{itemize}

\sphinxstylestrong{target} \textendash{} A jQuery object or a Widget instance.
\end{itemize}

\end{description}\end{quote}

\end{fulllineitems}



\begin{fulllineitems}
\phantomsection\label{\detokenize{reference/javascript_api:attachTo}}\pysiglinewithargsret{\sphinxbfcode{\sphinxupquote{method }}\sphinxbfcode{\sphinxupquote{attachTo}}}{\emph{target}}{}
Attach the current widget to a dom element
\begin{quote}\begin{description}
\item[{Parameters}] \leavevmode\begin{itemize}

\sphinxstylestrong{target} \textendash{} A jQuery object or a Widget instance.
\end{itemize}

\end{description}\end{quote}

\end{fulllineitems}



\begin{fulllineitems}
\phantomsection\label{\detokenize{reference/javascript_api:replace}}\pysiglinewithargsret{\sphinxbfcode{\sphinxupquote{method }}\sphinxbfcode{\sphinxupquote{replace}}}{\emph{target}}{}
Renders the current widget and replaces the given jQuery object.
\begin{quote}\begin{description}
\item[{Parameters}] \leavevmode\begin{itemize}

\sphinxstylestrong{target} \textendash{} A jQuery object or a Widget instance.
\end{itemize}

\end{description}\end{quote}

\end{fulllineitems}



\begin{fulllineitems}
\phantomsection\label{\detokenize{reference/javascript_api:start}}\pysiglinewithargsret{\sphinxbfcode{\sphinxupquote{method }}\sphinxbfcode{\sphinxupquote{start}}}{}{{ $\rightarrow$ jQuery.Deferred or any}}
Method called after rendering. Mostly used to bind actions, perform asynchronous
calls, etc…

By convention, this method should return an object that can be passed to \$.when()
to inform the caller when this widget has been initialized.
\begin{quote}\begin{description}
\item[{Return Type}] \leavevmode
\sphinxstyleliteralemphasis{\sphinxupquote{jQuery.Deferred}}\sphinxstyleemphasis{ or }\sphinxstyleliteralemphasis{\sphinxupquote{any}}

\end{description}\end{quote}

\end{fulllineitems}



\begin{fulllineitems}
\phantomsection\label{\detokenize{reference/javascript_api:renderElement}}\pysiglinewithargsret{\sphinxbfcode{\sphinxupquote{method }}\sphinxbfcode{\sphinxupquote{renderElement}}}{}{}
Renders the element. The default implementation renders the widget using QWeb,
\sphinxcode{\sphinxupquote{this.template}} must be defined. The context given to QWeb contains the “widget”
key that references \sphinxcode{\sphinxupquote{this}}.

\end{fulllineitems}



\begin{fulllineitems}
\phantomsection\label{\detokenize{reference/javascript_api:replaceElement}}\pysiglinewithargsret{\sphinxbfcode{\sphinxupquote{method }}\sphinxbfcode{\sphinxupquote{replaceElement}}}{\emph{\$el}}{{ $\rightarrow$ Widget}}
Re-sets the widget’s root element and replaces the old root element
(if any) by the new one in the DOM.
\begin{quote}\begin{description}
\item[{Parameters}] \leavevmode\begin{itemize}

\sphinxstylestrong{\$el} (\sphinxstyleliteralemphasis{\sphinxupquote{HTMLElement}}\sphinxstyleemphasis{ or }\sphinxstyleliteralemphasis{\sphinxupquote{jQuery}})
\end{itemize}

\item[{Returns}] \leavevmode
this

\item[{Return Type}] \leavevmode
{\hyperref[\detokenize{reference/javascript_api:web.Widget.Widget}]{\sphinxcrossref{\sphinxstyleliteralemphasis{\sphinxupquote{Widget}}}}}

\end{description}\end{quote}

\end{fulllineitems}



\begin{fulllineitems}
\phantomsection\label{\detokenize{reference/javascript_api:setElement}}\pysiglinewithargsret{\sphinxbfcode{\sphinxupquote{method }}\sphinxbfcode{\sphinxupquote{setElement}}}{\emph{element}}{{ $\rightarrow$ Widget}}
Re-sets the widget’s root element (el/\$el/\$el).

Includes:
\begin{itemize}
\item {} 
re-delegating events

\item {} 
re-binding sub-elements

\item {} 
if the widget already had a root element, replacing the pre-existing
element in the DOM

\end{itemize}
\begin{quote}\begin{description}
\item[{Parameters}] \leavevmode\begin{itemize}

\sphinxstylestrong{element} (\sphinxstyleliteralemphasis{\sphinxupquote{HTMLElement}}\sphinxstyleemphasis{ or }\sphinxstyleliteralemphasis{\sphinxupquote{jQuery}}) \textendash{} new root element for the widget
\end{itemize}

\item[{Returns}] \leavevmode
this

\item[{Return Type}] \leavevmode
{\hyperref[\detokenize{reference/javascript_api:web.Widget.Widget}]{\sphinxcrossref{\sphinxstyleliteralemphasis{\sphinxupquote{Widget}}}}}

\end{description}\end{quote}

\end{fulllineitems}



\begin{fulllineitems}
\phantomsection\label{\detokenize{reference/javascript_api:make}}\pysiglinewithargsret{\sphinxbfcode{\sphinxupquote{method }}\sphinxbfcode{\sphinxupquote{make}}}{\emph{tagName}\sphinxoptional{, \emph{attributes}}\sphinxoptional{, \emph{content}}}{{ $\rightarrow$ Element}}
Utility function to build small DOM elements.
\begin{quote}\begin{description}
\item[{Parameters}] \leavevmode\begin{itemize}

\sphinxstylestrong{tagName} (\sphinxstyleliteralemphasis{\sphinxupquote{String}}) \textendash{} name of the DOM element to create

\sphinxstylestrong{attributes} (\sphinxstyleliteralemphasis{\sphinxupquote{Object}}) \textendash{} map of DOM attributes to set on the element

\sphinxstylestrong{content} (\sphinxstyleliteralemphasis{\sphinxupquote{String}}) \textendash{} HTML content to set on the element
\end{itemize}

\item[{Return Type}] \leavevmode
\sphinxstyleliteralemphasis{\sphinxupquote{Element}}

\end{description}\end{quote}

\end{fulllineitems}



\begin{fulllineitems}
\phantomsection\label{\detokenize{reference/javascript_api:_}}\pysiglinewithargsret{\sphinxbfcode{\sphinxupquote{method }}\sphinxbfcode{\sphinxupquote{\$}}}{\emph{selector}}{{ $\rightarrow$ jQuery}}
Shortcut for \sphinxcode{\sphinxupquote{this.\$el.find(selector)}}
\begin{quote}\begin{description}
\item[{Parameters}] \leavevmode\begin{itemize}

\sphinxstylestrong{selector} (\sphinxstyleliteralemphasis{\sphinxupquote{String}}) \textendash{} CSS selector, rooted in \$el
\end{itemize}

\item[{Returns}] \leavevmode
selector match

\item[{Return Type}] \leavevmode
\sphinxstyleliteralemphasis{\sphinxupquote{jQuery}}

\end{description}\end{quote}

\end{fulllineitems}



\begin{fulllineitems}
\phantomsection\label{\detokenize{reference/javascript_api:do_show}}\pysiglinewithargsret{\sphinxbfcode{\sphinxupquote{method }}\sphinxbfcode{\sphinxupquote{do\_show}}}{}{}
Displays the widget

\end{fulllineitems}



\begin{fulllineitems}
\phantomsection\label{\detokenize{reference/javascript_api:do_hide}}\pysiglinewithargsret{\sphinxbfcode{\sphinxupquote{method }}\sphinxbfcode{\sphinxupquote{do\_hide}}}{}{}
Hides the widget

\end{fulllineitems}



\begin{fulllineitems}
\phantomsection\label{\detokenize{reference/javascript_api:do_toggle}}\pysiglinewithargsret{\sphinxbfcode{\sphinxupquote{method }}\sphinxbfcode{\sphinxupquote{do\_toggle}}}{\sphinxoptional{\emph{display}}}{}
Displays or hides the widget
\begin{quote}\begin{description}
\item[{Parameters}] \leavevmode\begin{itemize}

\sphinxstylestrong{display} (\sphinxstyleliteralemphasis{\sphinxupquote{Boolean}}) \textendash{} use true to show the widget or false to hide it
\end{itemize}

\end{description}\end{quote}

\end{fulllineitems}


\end{fulllineitems}


\end{fulllineitems}

\phantomsection\label{\detokenize{reference/javascript_api:module-web.concurrency}}

\begin{fulllineitems}
\phantomsection\label{\detokenize{reference/javascript_api:web.concurrency}}\pysigline{\sphinxbfcode{\sphinxupquote{module }}\sphinxbfcode{\sphinxupquote{web.concurrency}}}~~\begin{quote}\begin{description}
\item[{Exports}] \leavevmode{\hyperref[\detokenize{reference/javascript_api:web.concurrency.}]{\sphinxcrossref{
\textless{}anonymous\textgreater{}
}}}
\item[{Depends On}] \leavevmode\begin{itemize}
\item {} {\hyperref[\detokenize{reference/javascript_api:web.Class}]{\sphinxcrossref{
web.Class
}}}
\end{itemize}

\end{description}\end{quote}


\begin{fulllineitems}
\phantomsection\label{\detokenize{reference/javascript_api:web.concurrency.}}\pysigline{\sphinxbfcode{\sphinxupquote{namespace }}\sphinxbfcode{\sphinxupquote{}}}~

\begin{fulllineitems}
\phantomsection\label{\detokenize{reference/javascript_api:asyncWhen}}\pysiglinewithargsret{\sphinxbfcode{\sphinxupquote{function }}\sphinxbfcode{\sphinxupquote{asyncWhen}}}{}{{ $\rightarrow$ Deferred}}
The jquery implementation for \$.when has a (most of the time) useful
property: it is synchronous, if the deferred is resolved immediately.

This means that when we execute \$.when(def), then all registered
callbacks will be executed before the next line is executed.  This is
useful quite often, but in some rare cases, we might want to force an
async behavior. This is the purpose of this function, which simply adds a
setTimeout before resolving the deferred.
\begin{quote}\begin{description}
\item[{Return Type}] \leavevmode
\sphinxstyleliteralemphasis{\sphinxupquote{Deferred}}

\end{description}\end{quote}

\end{fulllineitems}



\begin{fulllineitems}
\phantomsection\label{\detokenize{reference/javascript_api:delay}}\pysiglinewithargsret{\sphinxbfcode{\sphinxupquote{function }}\sphinxbfcode{\sphinxupquote{delay}}}{\sphinxoptional{\emph{wait}}}{{ $\rightarrow$ Deferred}}
Returns a deferred resolved after ‘wait’ milliseconds
\begin{quote}\begin{description}
\item[{Parameters}] \leavevmode\begin{itemize}

\sphinxstylestrong{wait}=\sphinxstyleemphasis{0} (\sphinxstyleliteralemphasis{\sphinxupquote{int}}) \textendash{} the delay in ms
\end{itemize}

\item[{Return Type}] \leavevmode
\sphinxstyleliteralemphasis{\sphinxupquote{Deferred}}

\end{description}\end{quote}

\end{fulllineitems}



\begin{fulllineitems}
\phantomsection\label{\detokenize{reference/javascript_api:DropMisordered}}\pysiglinewithargsret{\sphinxbfcode{\sphinxupquote{class }}\sphinxbfcode{\sphinxupquote{DropMisordered}}}{\sphinxoptional{\emph{failMisordered}}}{}~\begin{quote}\begin{description}
\item[{Extends}] \leavevmode{\hyperref[\detokenize{reference/javascript_api:web.Class.Class}]{\sphinxcrossref{
Class
}}}
\item[{Parameters}] \leavevmode\begin{itemize}

\sphinxstylestrong{failMisordered}=\sphinxstyleemphasis{false} (\sphinxstyleliteralemphasis{\sphinxupquote{boolean}}) \textendash{} whether mis-ordered responses
  should be failed or just ignored
\end{itemize}

\end{description}\end{quote}

The DropMisordered abstraction is useful for situations where you have
a sequence of operations that you want to do, but if one of them
completes after a subsequent operation, then its result is obsolete and
should be ignored.

Note that is is kind of similar to the DropPrevious abstraction, but
subtly different.  The DropMisordered operations will all resolves if
they complete in the correct order.


\begin{fulllineitems}
\phantomsection\label{\detokenize{reference/javascript_api:add}}\pysiglinewithargsret{\sphinxbfcode{\sphinxupquote{method }}\sphinxbfcode{\sphinxupquote{add}}}{\emph{deferred}}{{ $\rightarrow$ Deferred}}
Adds a deferred (usually an async request) to the sequencer
\begin{quote}\begin{description}
\item[{Parameters}] \leavevmode\begin{itemize}

\sphinxstylestrong{deferred} (\sphinxstyleliteralemphasis{\sphinxupquote{Deferred}}) \textendash{} to ensure add
\end{itemize}

\item[{Return Type}] \leavevmode
\sphinxstyleliteralemphasis{\sphinxupquote{Deferred}}

\end{description}\end{quote}

\end{fulllineitems}


\end{fulllineitems}



\begin{fulllineitems}
\phantomsection\label{\detokenize{reference/javascript_api:DropPrevious}}\pysiglinewithargsret{\sphinxbfcode{\sphinxupquote{class }}\sphinxbfcode{\sphinxupquote{DropPrevious}}}{}{}~\begin{quote}\begin{description}
\item[{Extends}] \leavevmode{\hyperref[\detokenize{reference/javascript_api:web.Class.Class}]{\sphinxcrossref{
Class
}}}
\end{description}\end{quote}

The DropPrevious abstraction is useful when you have a sequence of
operations that you want to execute, but you only care of the result of
the last operation.

For example, let us say that we have a \_fetch method on a widget which
fetches data.  We want to rerender the widget after.  We could do this:

\fvset{hllines={, ,}}%
\begin{sphinxVerbatim}[commandchars=\\\{\}]
\PYG{k}{this}\PYG{p}{.}\PYG{n+nx}{\PYGZus{}fetch}\PYG{p}{(}\PYG{p}{)}\PYG{p}{.}\PYG{n+nx}{then}\PYG{p}{(}\PYG{k+kd}{function} \PYG{p}{(}\PYG{n+nx}{result}\PYG{p}{)} \PYG{p}{\PYGZob{}}
    \PYG{n+nx}{self}\PYG{p}{.}\PYG{n+nx}{state} \PYG{o}{=} \PYG{n+nx}{result}\PYG{p}{;}
    \PYG{n+nx}{self}\PYG{p}{.}\PYG{n+nx}{render}\PYG{p}{(}\PYG{p}{)}\PYG{p}{;}
\PYG{p}{\PYGZcb{}}\PYG{p}{)}\PYG{p}{;}
\end{sphinxVerbatim}

Now, we have at least two problems:
\begin{itemize}
\item {} 
if this code is called twice and the second \_fetch completes before the
first, the end state will be the result of the first \_fetch, which is
not what we expect

\item {} 
in any cases, the user interface will rerender twice, which is bad.

\end{itemize}

Now, if we have a DropPrevious:

\fvset{hllines={, ,}}%
\begin{sphinxVerbatim}[commandchars=\\\{\}]
\PYG{k}{this}\PYG{p}{.}\PYG{n+nx}{dropPrevious} \PYG{o}{=} \PYG{k}{new} \PYG{n+nx}{DropPrevious}\PYG{p}{(}\PYG{p}{)}\PYG{p}{;}
\end{sphinxVerbatim}

Then we can wrap the \_fetch in a DropPrevious and have the expected
result:

\fvset{hllines={, ,}}%
\begin{sphinxVerbatim}[commandchars=\\\{\}]
\PYG{k}{this}\PYG{p}{.}\PYG{n+nx}{dropPrevious}
    \PYG{p}{.}\PYG{n+nx}{add}\PYG{p}{(}\PYG{k}{this}\PYG{p}{.}\PYG{n+nx}{\PYGZus{}fetch}\PYG{p}{(}\PYG{p}{)}\PYG{p}{)}
    \PYG{p}{.}\PYG{n+nx}{then}\PYG{p}{(}\PYG{k+kd}{function} \PYG{p}{(}\PYG{n+nx}{result}\PYG{p}{)} \PYG{p}{\PYGZob{}}
        \PYG{n+nx}{self}\PYG{p}{.}\PYG{n+nx}{state} \PYG{o}{=} \PYG{n+nx}{result}\PYG{p}{;}
        \PYG{n+nx}{self}\PYG{p}{.}\PYG{n+nx}{render}\PYG{p}{(}\PYG{p}{)}\PYG{p}{;}
    \PYG{p}{\PYGZcb{}}\PYG{p}{)}\PYG{p}{;}
\end{sphinxVerbatim}


\begin{fulllineitems}
\phantomsection\label{\detokenize{reference/javascript_api:add}}\pysiglinewithargsret{\sphinxbfcode{\sphinxupquote{method }}\sphinxbfcode{\sphinxupquote{add}}}{\emph{deferred}}{{ $\rightarrow$ Promise}}
Registers a new deferred and rejects the previous one
\begin{quote}\begin{description}
\item[{Parameters}] \leavevmode\begin{itemize}

\sphinxstylestrong{deferred} (\sphinxstyleliteralemphasis{\sphinxupquote{Deferred}}) \textendash{} the new deferred
\end{itemize}

\item[{Return Type}] \leavevmode
\sphinxstyleliteralemphasis{\sphinxupquote{Promise}}

\end{description}\end{quote}

\end{fulllineitems}


\end{fulllineitems}



\begin{fulllineitems}
\phantomsection\label{\detokenize{reference/javascript_api:Mutex}}\pysiglinewithargsret{\sphinxbfcode{\sphinxupquote{class }}\sphinxbfcode{\sphinxupquote{Mutex}}}{}{}~\begin{quote}\begin{description}
\item[{Extends}] \leavevmode{\hyperref[\detokenize{reference/javascript_api:web.Class.Class}]{\sphinxcrossref{
Class
}}}
\end{description}\end{quote}

A (Odoo) mutex is a primitive for serializing computations.  This is
useful to avoid a situation where two computations modify some shared
state and cause some corrupted state.

Imagine that we have a function to fetch some data \_load(), which returns
a deferred which resolves to something useful. Now, we have some code
looking like this:

\fvset{hllines={, ,}}%
\begin{sphinxVerbatim}[commandchars=\\\{\}]
\PYG{k}{return} \PYG{k}{this}\PYG{p}{.}\PYG{n+nx}{\PYGZus{}load}\PYG{p}{(}\PYG{p}{)}\PYG{p}{.}\PYG{n+nx}{then}\PYG{p}{(}\PYG{k+kd}{function} \PYG{p}{(}\PYG{n+nx}{result}\PYG{p}{)} \PYG{p}{\PYGZob{}}
    \PYG{k}{this}\PYG{p}{.}\PYG{n+nx}{state} \PYG{o}{=} \PYG{n+nx}{result}\PYG{p}{;}
\PYG{p}{\PYGZcb{}}\PYG{p}{)}\PYG{p}{;}
\end{sphinxVerbatim}

If this code is run twice, but the second execution ends before the
first, then the final state will be the result of the first call to
\_load.  However, if we have a mutex:

\fvset{hllines={, ,}}%
\begin{sphinxVerbatim}[commandchars=\\\{\}]
\PYG{k}{this}\PYG{p}{.}\PYG{n+nx}{mutex} \PYG{o}{=} \PYG{k}{new} \PYG{n+nx}{Mutex}\PYG{p}{(}\PYG{p}{)}\PYG{p}{;}
\end{sphinxVerbatim}

and if we wrap the calls to \_load in a mutex:

\fvset{hllines={, ,}}%
\begin{sphinxVerbatim}[commandchars=\\\{\}]
\PYG{k}{return} \PYG{k}{this}\PYG{p}{.}\PYG{n+nx}{mutex}\PYG{p}{.}\PYG{n+nx}{exec}\PYG{p}{(}\PYG{k+kd}{function}\PYG{p}{(}\PYG{p}{)} \PYG{p}{\PYGZob{}}
    \PYG{k}{return} \PYG{k}{this}\PYG{p}{.}\PYG{n+nx}{\PYGZus{}load}\PYG{p}{(}\PYG{p}{)}\PYG{p}{.}\PYG{n+nx}{then}\PYG{p}{(}\PYG{k+kd}{function} \PYG{p}{(}\PYG{n+nx}{result}\PYG{p}{)} \PYG{p}{\PYGZob{}}
        \PYG{k}{this}\PYG{p}{.}\PYG{n+nx}{state} \PYG{o}{=} \PYG{n+nx}{result}\PYG{p}{;}
    \PYG{p}{\PYGZcb{}}\PYG{p}{)}\PYG{p}{;}
\PYG{p}{\PYGZcb{}}\PYG{p}{)}\PYG{p}{;}
\end{sphinxVerbatim}

Then, it is guaranteed that the final state will be the result of the
second execution.

A Mutex has to be a class, and not a function, because we have to keep
track of some internal state.


\begin{fulllineitems}
\phantomsection\label{\detokenize{reference/javascript_api:exec}}\pysiglinewithargsret{\sphinxbfcode{\sphinxupquote{method }}\sphinxbfcode{\sphinxupquote{exec}}}{\emph{action}}{{ $\rightarrow$ Deferred}}
Add a computation to the queue, it will be executed as soon as the
previous computations are completed.
\begin{quote}\begin{description}
\item[{Parameters}] \leavevmode\begin{itemize}

\sphinxstylestrong{action} (\sphinxstyleliteralemphasis{\sphinxupquote{function}}) \textendash{} a function which may return a deferred
\end{itemize}

\item[{Return Type}] \leavevmode
\sphinxstyleliteralemphasis{\sphinxupquote{Deferred}}

\end{description}\end{quote}

\end{fulllineitems}


\end{fulllineitems}



\begin{fulllineitems}
\phantomsection\label{\detokenize{reference/javascript_api:rejectAfter}}\pysiglinewithargsret{\sphinxbfcode{\sphinxupquote{function }}\sphinxbfcode{\sphinxupquote{rejectAfter}}}{\sphinxoptional{\emph{target\_def}}\sphinxoptional{, \emph{reference\_def}}}{{ $\rightarrow$ Deferred}}
Rejects a deferred as soon as a reference deferred is either resolved or
rejected
\begin{quote}\begin{description}
\item[{Parameters}] \leavevmode\begin{itemize}

\sphinxstylestrong{target\_def} (\sphinxstyleliteralemphasis{\sphinxupquote{Deferred}}) \textendash{} the deferred to potentially reject

\sphinxstylestrong{reference\_def} (\sphinxstyleliteralemphasis{\sphinxupquote{Deferred}}) \textendash{} the reference target
\end{itemize}

\item[{Return Type}] \leavevmode
\sphinxstyleliteralemphasis{\sphinxupquote{Deferred}}

\end{description}\end{quote}

\end{fulllineitems}


\end{fulllineitems}


\end{fulllineitems}

\phantomsection\label{\detokenize{reference/javascript_api:module-web.IFrameWidget}}

\begin{fulllineitems}
\phantomsection\label{\detokenize{reference/javascript_api:web.IFrameWidget}}\pysigline{\sphinxbfcode{\sphinxupquote{module }}\sphinxbfcode{\sphinxupquote{web.IFrameWidget}}}~~\begin{quote}\begin{description}
\item[{Exports}] \leavevmode{\hyperref[\detokenize{reference/javascript_api:web.IFrameWidget.IFrameWidget}]{\sphinxcrossref{
IFrameWidget
}}}
\item[{Depends On}] \leavevmode\begin{itemize}
\item {} {\hyperref[\detokenize{reference/javascript_api:web.Widget}]{\sphinxcrossref{
web.Widget
}}}
\end{itemize}

\end{description}\end{quote}


\begin{fulllineitems}
\phantomsection\label{\detokenize{reference/javascript_api:IFrameWidget}}\pysiglinewithargsret{\sphinxbfcode{\sphinxupquote{class }}\sphinxbfcode{\sphinxupquote{IFrameWidget}}}{\emph{parent}, \emph{url}}{}~\begin{quote}\begin{description}
\item[{Extends}] \leavevmode{\hyperref[\detokenize{reference/javascript_api:web.Widget.Widget}]{\sphinxcrossref{
Widget
}}}
\item[{Parameters}] \leavevmode\begin{itemize}

\sphinxstylestrong{parent} ({\hyperref[\detokenize{reference/javascript_api:Widget}]{\sphinxcrossref{\sphinxstyleliteralemphasis{\sphinxupquote{Widget}}}}})

\sphinxstylestrong{url} (\sphinxstyleliteralemphasis{\sphinxupquote{string}})
\end{itemize}

\end{description}\end{quote}

Generic widget to create an iframe that listens for clicks

It should be extended by overwriting the methods:

\fvset{hllines={, ,}}%
\begin{sphinxVerbatim}[commandchars=\\\{\}]
\PYG{n+nx}{init}\PYG{o}{:} \PYG{k+kd}{function}\PYG{p}{(}\PYG{n+nx}{parent}\PYG{p}{)} \PYG{p}{\PYGZob{}}
    \PYG{k}{this}\PYG{p}{.}\PYG{n+nx}{\PYGZus{}super}\PYG{p}{(}\PYG{n+nx}{parent}\PYG{p}{,} \PYG{o}{\PYGZlt{}}\PYG{n+nx}{url\PYGZus{}of\PYGZus{}iframe}\PYG{o}{\PYGZgt{}}\PYG{p}{)}\PYG{p}{;}
\PYG{p}{\PYGZcb{}}\PYG{p}{,}
\PYG{n+nx}{\PYGZus{}onIFrameClicked}\PYG{o}{:} \PYG{k+kd}{function}\PYG{p}{(}\PYG{n+nx}{e}\PYG{p}{)}\PYG{p}{\PYGZob{}}
    \PYG{n+nx}{filter} \PYG{n+nx}{the} \PYG{n+nx}{clicks} \PYG{n+nx}{you} \PYG{n+nx}{want} \PYG{n+nx}{to} \PYG{n+nx}{use} \PYG{n+nx}{and} \PYG{n+nx}{apply}
    \PYG{n+nx}{an} \PYG{n+nx}{action} \PYG{n+nx}{on} \PYG{n+nx}{it}
\PYG{p}{\PYGZcb{}}
\end{sphinxVerbatim}

\end{fulllineitems}



\begin{fulllineitems}
\phantomsection\label{\detokenize{reference/javascript_api:IFrameWidget}}\pysiglinewithargsret{\sphinxbfcode{\sphinxupquote{class }}\sphinxbfcode{\sphinxupquote{IFrameWidget}}}{\emph{parent}, \emph{url}}{}~\begin{quote}\begin{description}
\item[{Extends}] \leavevmode{\hyperref[\detokenize{reference/javascript_api:web.Widget.Widget}]{\sphinxcrossref{
Widget
}}}
\item[{Parameters}] \leavevmode\begin{itemize}

\sphinxstylestrong{parent} ({\hyperref[\detokenize{reference/javascript_api:Widget}]{\sphinxcrossref{\sphinxstyleliteralemphasis{\sphinxupquote{Widget}}}}})

\sphinxstylestrong{url} (\sphinxstyleliteralemphasis{\sphinxupquote{string}})
\end{itemize}

\end{description}\end{quote}

Generic widget to create an iframe that listens for clicks

It should be extended by overwriting the methods:

\fvset{hllines={, ,}}%
\begin{sphinxVerbatim}[commandchars=\\\{\}]
\PYG{n+nx}{init}\PYG{o}{:} \PYG{k+kd}{function}\PYG{p}{(}\PYG{n+nx}{parent}\PYG{p}{)} \PYG{p}{\PYGZob{}}
    \PYG{k}{this}\PYG{p}{.}\PYG{n+nx}{\PYGZus{}super}\PYG{p}{(}\PYG{n+nx}{parent}\PYG{p}{,} \PYG{o}{\PYGZlt{}}\PYG{n+nx}{url\PYGZus{}of\PYGZus{}iframe}\PYG{o}{\PYGZgt{}}\PYG{p}{)}\PYG{p}{;}
\PYG{p}{\PYGZcb{}}\PYG{p}{,}
\PYG{n+nx}{\PYGZus{}onIFrameClicked}\PYG{o}{:} \PYG{k+kd}{function}\PYG{p}{(}\PYG{n+nx}{e}\PYG{p}{)}\PYG{p}{\PYGZob{}}
    \PYG{n+nx}{filter} \PYG{n+nx}{the} \PYG{n+nx}{clicks} \PYG{n+nx}{you} \PYG{n+nx}{want} \PYG{n+nx}{to} \PYG{n+nx}{use} \PYG{n+nx}{and} \PYG{n+nx}{apply}
    \PYG{n+nx}{an} \PYG{n+nx}{action} \PYG{n+nx}{on} \PYG{n+nx}{it}
\PYG{p}{\PYGZcb{}}
\end{sphinxVerbatim}

\end{fulllineitems}


\end{fulllineitems}

\phantomsection\label{\detokenize{reference/javascript_api:module-point_of_sale.devices}}

\begin{fulllineitems}
\phantomsection\label{\detokenize{reference/javascript_api:point_of_sale.devices}}\pysigline{\sphinxbfcode{\sphinxupquote{module }}\sphinxbfcode{\sphinxupquote{point\_of\_sale.devices}}}~~\begin{quote}\begin{description}
\item[{Exports}] \leavevmode{\hyperref[\detokenize{reference/javascript_api:point_of_sale.devices.}]{\sphinxcrossref{
\textless{}anonymous\textgreater{}
}}}
\item[{Depends On}] \leavevmode\begin{itemize}
\item {} {\hyperref[\detokenize{reference/javascript_api:point_of_sale.BaseWidget}]{\sphinxcrossref{
point\_of\_sale.BaseWidget
}}}
\item {} {\hyperref[\detokenize{reference/javascript_api:web.Session}]{\sphinxcrossref{
web.Session
}}}
\item {} {\hyperref[\detokenize{reference/javascript_api:web.core}]{\sphinxcrossref{
web.core
}}}
\item {} {\hyperref[\detokenize{reference/javascript_api:web.mixins}]{\sphinxcrossref{
web.mixins
}}}
\item {} {\hyperref[\detokenize{reference/javascript_api:web.rpc}]{\sphinxcrossref{
web.rpc
}}}
\end{itemize}

\end{description}\end{quote}


\begin{fulllineitems}
\phantomsection\label{\detokenize{reference/javascript_api:point_of_sale.devices.}}\pysigline{\sphinxbfcode{\sphinxupquote{namespace }}\sphinxbfcode{\sphinxupquote{}}}
\end{fulllineitems}


\end{fulllineitems}

\phantomsection\label{\detokenize{reference/javascript_api:module-web.KanbanView}}

\begin{fulllineitems}
\phantomsection\label{\detokenize{reference/javascript_api:web.KanbanView}}\pysigline{\sphinxbfcode{\sphinxupquote{module }}\sphinxbfcode{\sphinxupquote{web.KanbanView}}}~~\begin{quote}\begin{description}
\item[{Exports}] \leavevmode{\hyperref[\detokenize{reference/javascript_api:web.KanbanView.KanbanView}]{\sphinxcrossref{
KanbanView
}}}
\item[{Depends On}] \leavevmode\begin{itemize}
\item {} {\hyperref[\detokenize{reference/javascript_api:web.BasicView}]{\sphinxcrossref{
web.BasicView
}}}
\item {} {\hyperref[\detokenize{reference/javascript_api:web.KanbanController}]{\sphinxcrossref{
web.KanbanController
}}}
\item {} {\hyperref[\detokenize{reference/javascript_api:web.KanbanModel}]{\sphinxcrossref{
web.KanbanModel
}}}
\item {} {\hyperref[\detokenize{reference/javascript_api:web.KanbanRenderer}]{\sphinxcrossref{
web.KanbanRenderer
}}}
\item {} {\hyperref[\detokenize{reference/javascript_api:web.config}]{\sphinxcrossref{
web.config
}}}
\item {} {\hyperref[\detokenize{reference/javascript_api:web.core}]{\sphinxcrossref{
web.core
}}}
\item {} {\hyperref[\detokenize{reference/javascript_api:web.utils}]{\sphinxcrossref{
web.utils
}}}
\end{itemize}

\end{description}\end{quote}


\begin{fulllineitems}
\phantomsection\label{\detokenize{reference/javascript_api:KanbanView}}\pysiglinewithargsret{\sphinxbfcode{\sphinxupquote{class }}\sphinxbfcode{\sphinxupquote{KanbanView}}}{}{}~\begin{quote}\begin{description}
\item[{Extends}] \leavevmode{\hyperref[\detokenize{reference/javascript_api:web.BasicView.BasicView}]{\sphinxcrossref{
BasicView
}}}
\end{description}\end{quote}

\end{fulllineitems}


\end{fulllineitems}

\phantomsection\label{\detokenize{reference/javascript_api:module-web_editor.snippet.editor}}

\begin{fulllineitems}
\phantomsection\label{\detokenize{reference/javascript_api:web_editor.snippet.editor}}\pysigline{\sphinxbfcode{\sphinxupquote{module }}\sphinxbfcode{\sphinxupquote{web\_editor.snippet.editor}}}~~\begin{quote}\begin{description}
\item[{Exports}] \leavevmode{\hyperref[\detokenize{reference/javascript_api:web_editor.snippet.editor.}]{\sphinxcrossref{
\textless{}anonymous\textgreater{}
}}}
\item[{Depends On}] \leavevmode\begin{itemize}
\item {} {\hyperref[\detokenize{reference/javascript_api:web.Widget}]{\sphinxcrossref{
web.Widget
}}}
\item {} {\hyperref[\detokenize{reference/javascript_api:web.core}]{\sphinxcrossref{
web.core
}}}
\item {} {\hyperref[\detokenize{reference/javascript_api:web.dom}]{\sphinxcrossref{
web.dom
}}}
\item {} {\hyperref[\detokenize{reference/javascript_api:web_editor.snippets.options}]{\sphinxcrossref{
web\_editor.snippets.options
}}}
\end{itemize}

\end{description}\end{quote}


\begin{fulllineitems}
\phantomsection\label{\detokenize{reference/javascript_api:SnippetsMenu}}\pysiglinewithargsret{\sphinxbfcode{\sphinxupquote{class }}\sphinxbfcode{\sphinxupquote{SnippetsMenu}}}{\emph{parent}, \emph{\$editable}}{}~\begin{quote}\begin{description}
\item[{Extends}] \leavevmode{\hyperref[\detokenize{reference/javascript_api:web.Widget.Widget}]{\sphinxcrossref{
Widget
}}}
\item[{Parameters}] \leavevmode\begin{itemize}

\sphinxstylestrong{parent}

\sphinxstylestrong{\$editable}
\end{itemize}

\end{description}\end{quote}

Management of drag\&drop menu and snippet related behaviors in the page.

\end{fulllineitems}



\begin{fulllineitems}
\phantomsection\label{\detokenize{reference/javascript_api:SnippetEditor}}\pysiglinewithargsret{\sphinxbfcode{\sphinxupquote{class }}\sphinxbfcode{\sphinxupquote{SnippetEditor}}}{\emph{parent}, \emph{target}, \emph{templateOptions}}{}~\begin{quote}\begin{description}
\item[{Extends}] \leavevmode{\hyperref[\detokenize{reference/javascript_api:web.Widget.Widget}]{\sphinxcrossref{
Widget
}}}
\item[{Parameters}] \leavevmode\begin{itemize}

\sphinxstylestrong{parent} ({\hyperref[\detokenize{reference/javascript_api:Widget}]{\sphinxcrossref{\sphinxstyleliteralemphasis{\sphinxupquote{Widget}}}}})

\sphinxstylestrong{target} (\sphinxstyleliteralemphasis{\sphinxupquote{Element}})

\sphinxstylestrong{templateOptions}
\end{itemize}

\end{description}\end{quote}

Management of the overlay and option list for a snippet.


\begin{fulllineitems}
\phantomsection\label{\detokenize{reference/javascript_api:buildSnippet}}\pysiglinewithargsret{\sphinxbfcode{\sphinxupquote{method }}\sphinxbfcode{\sphinxupquote{buildSnippet}}}{}{}
Notifies all the associated snippet options that the snippet has just
been dropped in the page.

\end{fulllineitems}



\begin{fulllineitems}
\phantomsection\label{\detokenize{reference/javascript_api:cleanForSave}}\pysiglinewithargsret{\sphinxbfcode{\sphinxupquote{method }}\sphinxbfcode{\sphinxupquote{cleanForSave}}}{}{}
Notifies all the associated snippet options that the template which
contains the snippet is about to be saved.

\end{fulllineitems}



\begin{fulllineitems}
\phantomsection\label{\detokenize{reference/javascript_api:cover}}\pysiglinewithargsret{\sphinxbfcode{\sphinxupquote{method }}\sphinxbfcode{\sphinxupquote{cover}}}{}{}
Makes the editor overlay cover the associated snippet.

\end{fulllineitems}



\begin{fulllineitems}
\phantomsection\label{\detokenize{reference/javascript_api:toggleFocus}}\pysiglinewithargsret{\sphinxbfcode{\sphinxupquote{method }}\sphinxbfcode{\sphinxupquote{toggleFocus}}}{\emph{focus}}{}
Displays/Hides the editor overlay and notifies the associated snippet
options. Note: when it is displayed, this is here that the parent
snippet options are moved to the editor overlay.
\begin{quote}\begin{description}
\item[{Parameters}] \leavevmode\begin{itemize}

\sphinxstylestrong{focus} (\sphinxstyleliteralemphasis{\sphinxupquote{boolean}}) \textendash{} true to display, false to hide
\end{itemize}

\end{description}\end{quote}

\end{fulllineitems}


\end{fulllineitems}



\begin{fulllineitems}
\phantomsection\label{\detokenize{reference/javascript_api:web_editor.snippet.editor.}}\pysigline{\sphinxbfcode{\sphinxupquote{namespace }}\sphinxbfcode{\sphinxupquote{}}}~

\begin{fulllineitems}
\phantomsection\label{\detokenize{reference/javascript_api:SnippetsMenu}}\pysiglinewithargsret{\sphinxbfcode{\sphinxupquote{class }}\sphinxbfcode{\sphinxupquote{SnippetsMenu}}}{\emph{parent}, \emph{\$editable}}{}~\begin{quote}\begin{description}
\item[{Extends}] \leavevmode{\hyperref[\detokenize{reference/javascript_api:web.Widget.Widget}]{\sphinxcrossref{
Widget
}}}
\item[{Parameters}] \leavevmode\begin{itemize}

\sphinxstylestrong{parent}

\sphinxstylestrong{\$editable}
\end{itemize}

\end{description}\end{quote}

Management of drag\&drop menu and snippet related behaviors in the page.

\end{fulllineitems}



\begin{fulllineitems}
\phantomsection\label{\detokenize{reference/javascript_api:SnippetEditor}}\pysiglinewithargsret{\sphinxbfcode{\sphinxupquote{class }}\sphinxbfcode{\sphinxupquote{SnippetEditor}}}{\emph{parent}, \emph{target}, \emph{templateOptions}}{}~\begin{quote}\begin{description}
\item[{Extends}] \leavevmode{\hyperref[\detokenize{reference/javascript_api:web.Widget.Widget}]{\sphinxcrossref{
Widget
}}}
\item[{Parameters}] \leavevmode\begin{itemize}

\sphinxstylestrong{parent} ({\hyperref[\detokenize{reference/javascript_api:Widget}]{\sphinxcrossref{\sphinxstyleliteralemphasis{\sphinxupquote{Widget}}}}})

\sphinxstylestrong{target} (\sphinxstyleliteralemphasis{\sphinxupquote{Element}})

\sphinxstylestrong{templateOptions}
\end{itemize}

\end{description}\end{quote}

Management of the overlay and option list for a snippet.


\begin{fulllineitems}
\phantomsection\label{\detokenize{reference/javascript_api:buildSnippet}}\pysiglinewithargsret{\sphinxbfcode{\sphinxupquote{method }}\sphinxbfcode{\sphinxupquote{buildSnippet}}}{}{}
Notifies all the associated snippet options that the snippet has just
been dropped in the page.

\end{fulllineitems}



\begin{fulllineitems}
\phantomsection\label{\detokenize{reference/javascript_api:cleanForSave}}\pysiglinewithargsret{\sphinxbfcode{\sphinxupquote{method }}\sphinxbfcode{\sphinxupquote{cleanForSave}}}{}{}
Notifies all the associated snippet options that the template which
contains the snippet is about to be saved.

\end{fulllineitems}



\begin{fulllineitems}
\phantomsection\label{\detokenize{reference/javascript_api:cover}}\pysiglinewithargsret{\sphinxbfcode{\sphinxupquote{method }}\sphinxbfcode{\sphinxupquote{cover}}}{}{}
Makes the editor overlay cover the associated snippet.

\end{fulllineitems}



\begin{fulllineitems}
\phantomsection\label{\detokenize{reference/javascript_api:toggleFocus}}\pysiglinewithargsret{\sphinxbfcode{\sphinxupquote{method }}\sphinxbfcode{\sphinxupquote{toggleFocus}}}{\emph{focus}}{}
Displays/Hides the editor overlay and notifies the associated snippet
options. Note: when it is displayed, this is here that the parent
snippet options are moved to the editor overlay.
\begin{quote}\begin{description}
\item[{Parameters}] \leavevmode\begin{itemize}

\sphinxstylestrong{focus} (\sphinxstyleliteralemphasis{\sphinxupquote{boolean}}) \textendash{} true to display, false to hide
\end{itemize}

\end{description}\end{quote}

\end{fulllineitems}


\end{fulllineitems}


\end{fulllineitems}


\end{fulllineitems}

\phantomsection\label{\detokenize{reference/javascript_api:module-web.Apps}}

\begin{fulllineitems}
\phantomsection\label{\detokenize{reference/javascript_api:web.Apps}}\pysigline{\sphinxbfcode{\sphinxupquote{module }}\sphinxbfcode{\sphinxupquote{web.Apps}}}~~\begin{quote}\begin{description}
\item[{Exports}] \leavevmode{\hyperref[\detokenize{reference/javascript_api:web.Apps.Apps}]{\sphinxcrossref{
Apps
}}}
\item[{Depends On}] \leavevmode\begin{itemize}
\item {} {\hyperref[\detokenize{reference/javascript_api:web.Widget}]{\sphinxcrossref{
web.Widget
}}}
\item {} {\hyperref[\detokenize{reference/javascript_api:web.core}]{\sphinxcrossref{
web.core
}}}
\item {} {\hyperref[\detokenize{reference/javascript_api:web.framework}]{\sphinxcrossref{
web.framework
}}}
\item {} {\hyperref[\detokenize{reference/javascript_api:web.session}]{\sphinxcrossref{
web.session
}}}
\end{itemize}

\end{description}\end{quote}


\begin{fulllineitems}
\phantomsection\label{\detokenize{reference/javascript_api:Apps}}\pysiglinewithargsret{\sphinxbfcode{\sphinxupquote{class }}\sphinxbfcode{\sphinxupquote{Apps}}}{\emph{parent}, \emph{action}}{}~\begin{quote}\begin{description}
\item[{Extends}] \leavevmode{\hyperref[\detokenize{reference/javascript_api:web.Widget.Widget}]{\sphinxcrossref{
Widget
}}}
\item[{Parameters}] \leavevmode\begin{itemize}

\sphinxstylestrong{parent}

\sphinxstylestrong{action}
\end{itemize}

\end{description}\end{quote}

\end{fulllineitems}


\end{fulllineitems}

\phantomsection\label{\detokenize{reference/javascript_api:module-account.ReconciliationClientAction}}

\begin{fulllineitems}
\phantomsection\label{\detokenize{reference/javascript_api:account.ReconciliationClientAction}}\pysigline{\sphinxbfcode{\sphinxupquote{module }}\sphinxbfcode{\sphinxupquote{account.ReconciliationClientAction}}}~~\begin{quote}\begin{description}
\item[{Exports}] \leavevmode{\hyperref[\detokenize{reference/javascript_api:account.ReconciliationClientAction.}]{\sphinxcrossref{
\textless{}anonymous\textgreater{}
}}}
\item[{Depends On}] \leavevmode\begin{itemize}
\item {} {\hyperref[\detokenize{reference/javascript_api:account.ReconciliationModel}]{\sphinxcrossref{
account.ReconciliationModel
}}}
\item {} {\hyperref[\detokenize{reference/javascript_api:account.ReconciliationRenderer}]{\sphinxcrossref{
account.ReconciliationRenderer
}}}
\item {} {\hyperref[\detokenize{reference/javascript_api:web.ControlPanelMixin}]{\sphinxcrossref{
web.ControlPanelMixin
}}}
\item {} {\hyperref[\detokenize{reference/javascript_api:web.Widget}]{\sphinxcrossref{
web.Widget
}}}
\item {} {\hyperref[\detokenize{reference/javascript_api:web.core}]{\sphinxcrossref{
web.core
}}}
\end{itemize}

\end{description}\end{quote}


\begin{fulllineitems}
\phantomsection\label{\detokenize{reference/javascript_api:StatementAction}}\pysiglinewithargsret{\sphinxbfcode{\sphinxupquote{class }}\sphinxbfcode{\sphinxupquote{StatementAction}}}{\emph{parent}, \emph{params}}{}~\begin{quote}\begin{description}
\item[{Extends}] \leavevmode{\hyperref[\detokenize{reference/javascript_api:web.Widget.Widget}]{\sphinxcrossref{
Widget
}}}
\item[{Mixes}] \leavevmode\begin{itemize}
\item {} {\hyperref[\detokenize{reference/javascript_api:web.ControlPanelMixin.ControlPanelMixin}]{\sphinxcrossref{
ControlPanelMixin
}}}
\end{itemize}

\item[{Parameters}] \leavevmode\begin{itemize}

\sphinxstylestrong{parent}

\sphinxstylestrong{params} ({\hyperref[\detokenize{reference/javascript_api:account.ReconciliationClientAction.StatementActionParams}]{\sphinxcrossref{\sphinxstyleliteralemphasis{\sphinxupquote{StatementActionParams}}}}})
\end{itemize}

\end{description}\end{quote}

Widget used as action for ‘account.bank.statement’ reconciliation


\begin{fulllineitems}
\phantomsection\label{\detokenize{reference/javascript_api:willStart}}\pysiglinewithargsret{\sphinxbfcode{\sphinxupquote{method }}\sphinxbfcode{\sphinxupquote{willStart}}}{}{}
instantiate the action renderer

\end{fulllineitems}



\begin{fulllineitems}
\phantomsection\label{\detokenize{reference/javascript_api:start}}\pysiglinewithargsret{\sphinxbfcode{\sphinxupquote{method }}\sphinxbfcode{\sphinxupquote{start}}}{}{}
append the renderer and instantiate the line renderers

\end{fulllineitems}



\begin{fulllineitems}
\phantomsection\label{\detokenize{reference/javascript_api:do_show}}\pysiglinewithargsret{\sphinxbfcode{\sphinxupquote{method }}\sphinxbfcode{\sphinxupquote{do\_show}}}{}{}
update the control panel and breadcrumbs

\end{fulllineitems}



\begin{fulllineitems}
\phantomsection\label{\detokenize{reference/javascript_api:StatementActionParams}}\pysiglinewithargsret{\sphinxbfcode{\sphinxupquote{class }}\sphinxbfcode{\sphinxupquote{StatementActionParams}}}{}{}~

\begin{fulllineitems}
\phantomsection\label{\detokenize{reference/javascript_api:context}}\pysigline{\sphinxbfcode{\sphinxupquote{attribute }}\sphinxbfcode{\sphinxupquote{context}} Object}
\end{fulllineitems}


\end{fulllineitems}


\end{fulllineitems}



\begin{fulllineitems}
\phantomsection\label{\detokenize{reference/javascript_api:ManualAction}}\pysiglinewithargsret{\sphinxbfcode{\sphinxupquote{class }}\sphinxbfcode{\sphinxupquote{ManualAction}}}{}{}~\begin{quote}\begin{description}
\item[{Extends}] \leavevmode{\hyperref[\detokenize{reference/javascript_api:account.ReconciliationClientAction.StatementAction}]{\sphinxcrossref{
StatementAction
}}}
\end{description}\end{quote}

Widget used as action for ‘account.move.line’ and ‘res.partner’ for the
manual reconciliation and mark data as reconciliate

\end{fulllineitems}



\begin{fulllineitems}
\phantomsection\label{\detokenize{reference/javascript_api:account.ReconciliationClientAction.}}\pysigline{\sphinxbfcode{\sphinxupquote{namespace }}\sphinxbfcode{\sphinxupquote{}}}~

\begin{fulllineitems}
\phantomsection\label{\detokenize{reference/javascript_api:StatementAction}}\pysiglinewithargsret{\sphinxbfcode{\sphinxupquote{class }}\sphinxbfcode{\sphinxupquote{StatementAction}}}{\emph{parent}, \emph{params}}{}~\begin{quote}\begin{description}
\item[{Extends}] \leavevmode{\hyperref[\detokenize{reference/javascript_api:web.Widget.Widget}]{\sphinxcrossref{
Widget
}}}
\item[{Mixes}] \leavevmode\begin{itemize}
\item {} {\hyperref[\detokenize{reference/javascript_api:web.ControlPanelMixin.ControlPanelMixin}]{\sphinxcrossref{
ControlPanelMixin
}}}
\end{itemize}

\item[{Parameters}] \leavevmode\begin{itemize}

\sphinxstylestrong{parent}

\sphinxstylestrong{params} ({\hyperref[\detokenize{reference/javascript_api:account.ReconciliationClientAction.StatementActionParams}]{\sphinxcrossref{\sphinxstyleliteralemphasis{\sphinxupquote{StatementActionParams}}}}})
\end{itemize}

\end{description}\end{quote}

Widget used as action for ‘account.bank.statement’ reconciliation


\begin{fulllineitems}
\phantomsection\label{\detokenize{reference/javascript_api:willStart}}\pysiglinewithargsret{\sphinxbfcode{\sphinxupquote{method }}\sphinxbfcode{\sphinxupquote{willStart}}}{}{}
instantiate the action renderer

\end{fulllineitems}



\begin{fulllineitems}
\phantomsection\label{\detokenize{reference/javascript_api:start}}\pysiglinewithargsret{\sphinxbfcode{\sphinxupquote{method }}\sphinxbfcode{\sphinxupquote{start}}}{}{}
append the renderer and instantiate the line renderers

\end{fulllineitems}



\begin{fulllineitems}
\phantomsection\label{\detokenize{reference/javascript_api:do_show}}\pysiglinewithargsret{\sphinxbfcode{\sphinxupquote{method }}\sphinxbfcode{\sphinxupquote{do\_show}}}{}{}
update the control panel and breadcrumbs

\end{fulllineitems}



\begin{fulllineitems}
\phantomsection\label{\detokenize{reference/javascript_api:StatementActionParams}}\pysiglinewithargsret{\sphinxbfcode{\sphinxupquote{class }}\sphinxbfcode{\sphinxupquote{StatementActionParams}}}{}{}~

\begin{fulllineitems}
\phantomsection\label{\detokenize{reference/javascript_api:context}}\pysigline{\sphinxbfcode{\sphinxupquote{attribute }}\sphinxbfcode{\sphinxupquote{context}} Object}
\end{fulllineitems}


\end{fulllineitems}


\end{fulllineitems}



\begin{fulllineitems}
\phantomsection\label{\detokenize{reference/javascript_api:ManualAction}}\pysiglinewithargsret{\sphinxbfcode{\sphinxupquote{class }}\sphinxbfcode{\sphinxupquote{ManualAction}}}{}{}~\begin{quote}\begin{description}
\item[{Extends}] \leavevmode{\hyperref[\detokenize{reference/javascript_api:account.ReconciliationClientAction.StatementAction}]{\sphinxcrossref{
StatementAction
}}}
\end{description}\end{quote}

Widget used as action for ‘account.move.line’ and ‘res.partner’ for the
manual reconciliation and mark data as reconciliate

\end{fulllineitems}


\end{fulllineitems}


\end{fulllineitems}

\phantomsection\label{\detokenize{reference/javascript_api:module-website.theme}}

\begin{fulllineitems}
\phantomsection\label{\detokenize{reference/javascript_api:website.theme}}\pysigline{\sphinxbfcode{\sphinxupquote{module }}\sphinxbfcode{\sphinxupquote{website.theme}}}~~\begin{quote}\begin{description}
\item[{Exports}] \leavevmode{\hyperref[\detokenize{reference/javascript_api:website.theme.ThemeCustomizeDialog}]{\sphinxcrossref{
ThemeCustomizeDialog
}}}
\item[{Depends On}] \leavevmode\begin{itemize}
\item {} {\hyperref[\detokenize{reference/javascript_api:web.Widget}]{\sphinxcrossref{
web.Widget
}}}
\item {} {\hyperref[\detokenize{reference/javascript_api:web.ajax}]{\sphinxcrossref{
web.ajax
}}}
\item {} {\hyperref[\detokenize{reference/javascript_api:web.core}]{\sphinxcrossref{
web.core
}}}
\item {} {\hyperref[\detokenize{reference/javascript_api:web.session}]{\sphinxcrossref{
web.session
}}}
\item {} {\hyperref[\detokenize{reference/javascript_api:web_editor.context}]{\sphinxcrossref{
web\_editor.context
}}}
\item {} {\hyperref[\detokenize{reference/javascript_api:website.navbar}]{\sphinxcrossref{
website.navbar
}}}
\end{itemize}

\end{description}\end{quote}


\begin{fulllineitems}
\phantomsection\label{\detokenize{reference/javascript_api:ThemeCustomizeDialog}}\pysiglinewithargsret{\sphinxbfcode{\sphinxupquote{class }}\sphinxbfcode{\sphinxupquote{ThemeCustomizeDialog}}}{}{}~\begin{quote}\begin{description}
\item[{Extends}] \leavevmode{\hyperref[\detokenize{reference/javascript_api:web.Widget.Widget}]{\sphinxcrossref{
Widget
}}}
\end{description}\end{quote}

\end{fulllineitems}


\end{fulllineitems}

\phantomsection\label{\detokenize{reference/javascript_api:module-web.DomainSelectorDialog}}

\begin{fulllineitems}
\phantomsection\label{\detokenize{reference/javascript_api:web.DomainSelectorDialog}}\pysigline{\sphinxbfcode{\sphinxupquote{module }}\sphinxbfcode{\sphinxupquote{web.DomainSelectorDialog}}}~~\begin{quote}\begin{description}
\item[{Exports}] \leavevmode
\textless{}anonymous\textgreater{}

\item[{Depends On}] \leavevmode\begin{itemize}
\item {} {\hyperref[\detokenize{reference/javascript_api:web.Dialog}]{\sphinxcrossref{
web.Dialog
}}}
\item {} {\hyperref[\detokenize{reference/javascript_api:web.DomainSelector}]{\sphinxcrossref{
web.DomainSelector
}}}
\item {} {\hyperref[\detokenize{reference/javascript_api:web.core}]{\sphinxcrossref{
web.core
}}}
\end{itemize}

\end{description}\end{quote}


\begin{fulllineitems}
\phantomsection\label{\detokenize{reference/javascript_api:DomainSelectorDialog}}\pysiglinewithargsret{\sphinxbfcode{\sphinxupquote{class }}\sphinxbfcode{\sphinxupquote{DomainSelectorDialog}}}{\emph{parent}, \emph{model}, \emph{domain}, \emph{options}}{}~\begin{quote}\begin{description}
\item[{Extends}] \leavevmode{\hyperref[\detokenize{reference/javascript_api:web.Dialog.Dialog}]{\sphinxcrossref{
Dialog
}}}
\item[{Parameters}] \leavevmode\begin{itemize}

\sphinxstylestrong{parent}

\sphinxstylestrong{model}

\sphinxstylestrong{domain}

\sphinxstylestrong{options}
\end{itemize}

\end{description}\end{quote}

\end{fulllineitems}


\end{fulllineitems}

\phantomsection\label{\detokenize{reference/javascript_api:module-account.dashboard_setup_bar}}

\begin{fulllineitems}
\phantomsection\label{\detokenize{reference/javascript_api:account.dashboard_setup_bar}}\pysigline{\sphinxbfcode{\sphinxupquote{module }}\sphinxbfcode{\sphinxupquote{account.dashboard\_setup\_bar}}}~~\begin{quote}\begin{description}
\item[{Exports}] \leavevmode{\hyperref[\detokenize{reference/javascript_api:account.dashboard_setup_bar.}]{\sphinxcrossref{
\textless{}anonymous\textgreater{}
}}}
\item[{Depends On}] \leavevmode\begin{itemize}
\item {} {\hyperref[\detokenize{reference/javascript_api:web.KanbanController}]{\sphinxcrossref{
web.KanbanController
}}}
\item {} {\hyperref[\detokenize{reference/javascript_api:web.KanbanModel}]{\sphinxcrossref{
web.KanbanModel
}}}
\item {} {\hyperref[\detokenize{reference/javascript_api:web.KanbanRenderer}]{\sphinxcrossref{
web.KanbanRenderer
}}}
\item {} {\hyperref[\detokenize{reference/javascript_api:web.KanbanView}]{\sphinxcrossref{
web.KanbanView
}}}
\item {} {\hyperref[\detokenize{reference/javascript_api:web.core}]{\sphinxcrossref{
web.core
}}}
\item {} {\hyperref[\detokenize{reference/javascript_api:web.field_utils}]{\sphinxcrossref{
web.field\_utils
}}}
\item {} {\hyperref[\detokenize{reference/javascript_api:web.session}]{\sphinxcrossref{
web.session
}}}
\item {} {\hyperref[\detokenize{reference/javascript_api:web.view_registry}]{\sphinxcrossref{
web.view\_registry
}}}
\end{itemize}

\end{description}\end{quote}


\begin{fulllineitems}
\phantomsection\label{\detokenize{reference/javascript_api:account.dashboard_setup_bar.}}\pysigline{\sphinxbfcode{\sphinxupquote{namespace }}\sphinxbfcode{\sphinxupquote{}}}
\end{fulllineitems}


\end{fulllineitems}

\phantomsection\label{\detokenize{reference/javascript_api:module-point_of_sale.gui}}

\begin{fulllineitems}
\phantomsection\label{\detokenize{reference/javascript_api:point_of_sale.gui}}\pysigline{\sphinxbfcode{\sphinxupquote{module }}\sphinxbfcode{\sphinxupquote{point\_of\_sale.gui}}}~~\begin{quote}\begin{description}
\item[{Exports}] \leavevmode{\hyperref[\detokenize{reference/javascript_api:point_of_sale.gui.}]{\sphinxcrossref{
\textless{}anonymous\textgreater{}
}}}
\item[{Depends On}] \leavevmode\begin{itemize}
\item {} {\hyperref[\detokenize{reference/javascript_api:web.core}]{\sphinxcrossref{
web.core
}}}
\item {} {\hyperref[\detokenize{reference/javascript_api:web.field_utils}]{\sphinxcrossref{
web.field\_utils
}}}
\item {} {\hyperref[\detokenize{reference/javascript_api:web.session}]{\sphinxcrossref{
web.session
}}}
\end{itemize}

\end{description}\end{quote}


\begin{fulllineitems}
\phantomsection\label{\detokenize{reference/javascript_api:point_of_sale.gui.}}\pysigline{\sphinxbfcode{\sphinxupquote{namespace }}\sphinxbfcode{\sphinxupquote{}}}
\end{fulllineitems}


\end{fulllineitems}

\phantomsection\label{\detokenize{reference/javascript_api:module-website_hr_recruitment.tour}}

\begin{fulllineitems}
\phantomsection\label{\detokenize{reference/javascript_api:website_hr_recruitment.tour}}\pysigline{\sphinxbfcode{\sphinxupquote{module }}\sphinxbfcode{\sphinxupquote{website\_hr\_recruitment.tour}}}~~\begin{quote}\begin{description}
\item[{Exports}] \leavevmode{\hyperref[\detokenize{reference/javascript_api:website_hr_recruitment.tour.}]{\sphinxcrossref{
\textless{}anonymous\textgreater{}
}}}
\item[{Depends On}] \leavevmode\begin{itemize}
\item {} {\hyperref[\detokenize{reference/javascript_api:web_editor.base}]{\sphinxcrossref{
web\_editor.base
}}}
\item {} {\hyperref[\detokenize{reference/javascript_api:web_tour.tour}]{\sphinxcrossref{
web\_tour.tour
}}}
\end{itemize}

\end{description}\end{quote}


\begin{fulllineitems}
\phantomsection\label{\detokenize{reference/javascript_api:website_hr_recruitment.tour.}}\pysigline{\sphinxbfcode{\sphinxupquote{namespace }}\sphinxbfcode{\sphinxupquote{}}}
\end{fulllineitems}


\end{fulllineitems}

\phantomsection\label{\detokenize{reference/javascript_api:module-google_drive.sidebar}}

\begin{fulllineitems}
\phantomsection\label{\detokenize{reference/javascript_api:google_drive.sidebar}}\pysigline{\sphinxbfcode{\sphinxupquote{module }}\sphinxbfcode{\sphinxupquote{google\_drive.sidebar}}}~~\begin{quote}\begin{description}
\item[{Exports}] \leavevmode{\hyperref[\detokenize{reference/javascript_api:Sidebar}]{\sphinxcrossref{
Sidebar
}}}
\item[{Depends On}] \leavevmode\begin{itemize}
\item {} {\hyperref[\detokenize{reference/javascript_api:web.Sidebar}]{\sphinxcrossref{
web.Sidebar
}}}
\end{itemize}

\end{description}\end{quote}


\begin{fulllineitems}
\phantomsection\label{\detokenize{reference/javascript_api:Sidebar}}\pysiglinewithargsret{\sphinxbfcode{\sphinxupquote{class }}\sphinxbfcode{\sphinxupquote{Sidebar}}}{\emph{parent}, \emph{options}}{}~\begin{quote}\begin{description}
\item[{Extends}] \leavevmode{\hyperref[\detokenize{reference/javascript_api:web.Widget.Widget}]{\sphinxcrossref{
Widget
}}}
\item[{Parameters}] \leavevmode\begin{itemize}

\sphinxstylestrong{parent}

\sphinxstylestrong{options}
\end{itemize}

\end{description}\end{quote}


\begin{fulllineitems}
\phantomsection\label{\detokenize{reference/javascript_api:start}}\pysiglinewithargsret{\sphinxbfcode{\sphinxupquote{method }}\sphinxbfcode{\sphinxupquote{start}}}{}{}
Get the attachment linked to the record when the toolbar started

\end{fulllineitems}


\end{fulllineitems}


\end{fulllineitems}

\phantomsection\label{\detokenize{reference/javascript_api:module-account.ReconciliationModel}}

\begin{fulllineitems}
\phantomsection\label{\detokenize{reference/javascript_api:account.ReconciliationModel}}\pysigline{\sphinxbfcode{\sphinxupquote{module }}\sphinxbfcode{\sphinxupquote{account.ReconciliationModel}}}~~\begin{quote}\begin{description}
\item[{Exports}] \leavevmode{\hyperref[\detokenize{reference/javascript_api:account.ReconciliationModel.}]{\sphinxcrossref{
\textless{}anonymous\textgreater{}
}}}
\item[{Depends On}] \leavevmode\begin{itemize}
\item {} {\hyperref[\detokenize{reference/javascript_api:web.BasicModel}]{\sphinxcrossref{
web.BasicModel
}}}
\item {} {\hyperref[\detokenize{reference/javascript_api:web.CrashManager}]{\sphinxcrossref{
web.CrashManager
}}}
\item {} {\hyperref[\detokenize{reference/javascript_api:web.core}]{\sphinxcrossref{
web.core
}}}
\item {} {\hyperref[\detokenize{reference/javascript_api:web.field_utils}]{\sphinxcrossref{
web.field\_utils
}}}
\item {} {\hyperref[\detokenize{reference/javascript_api:web.session}]{\sphinxcrossref{
web.session
}}}
\item {} {\hyperref[\detokenize{reference/javascript_api:web.utils}]{\sphinxcrossref{
web.utils
}}}
\end{itemize}

\end{description}\end{quote}


\begin{fulllineitems}
\phantomsection\label{\detokenize{reference/javascript_api:ManualModel}}\pysiglinewithargsret{\sphinxbfcode{\sphinxupquote{class }}\sphinxbfcode{\sphinxupquote{ManualModel}}}{}{}~\begin{quote}\begin{description}
\item[{Extends}] \leavevmode{\hyperref[\detokenize{reference/javascript_api:account.ReconciliationModel.StatementModel}]{\sphinxcrossref{
StatementModel
}}}
\end{description}\end{quote}

Model use to fetch, format and update ‘account.move.line’ and ‘res.partner’
datas allowing manual reconciliation


\begin{fulllineitems}
\phantomsection\label{\detokenize{reference/javascript_api:load}}\pysiglinewithargsret{\sphinxbfcode{\sphinxupquote{method }}\sphinxbfcode{\sphinxupquote{load}}}{\emph{context}}{{ $\rightarrow$ Deferred}}
load data from
- ‘account.move.line’ fetch the lines to reconciliate
- ‘account.account’ fetch all account code
\begin{quote}\begin{description}
\item[{Parameters}] \leavevmode\begin{itemize}

\sphinxstylestrong{context} ({\hyperref[\detokenize{reference/javascript_api:account.ReconciliationModel.LoadContext}]{\sphinxcrossref{\sphinxstyleliteralemphasis{\sphinxupquote{LoadContext}}}}})
\end{itemize}

\item[{Return Type}] \leavevmode
\sphinxstyleliteralemphasis{\sphinxupquote{Deferred}}

\end{description}\end{quote}


\begin{fulllineitems}
\phantomsection\label{\detokenize{reference/javascript_api:LoadContext}}\pysiglinewithargsret{\sphinxbfcode{\sphinxupquote{class }}\sphinxbfcode{\sphinxupquote{LoadContext}}}{}{}~

\begin{fulllineitems}
\phantomsection\label{\detokenize{reference/javascript_api:mode}}\pysigline{\sphinxbfcode{\sphinxupquote{attribute }}\sphinxbfcode{\sphinxupquote{mode}} string}
‘customers’, ‘suppliers’ or ‘accounts’

\end{fulllineitems}



\begin{fulllineitems}
\phantomsection\label{\detokenize{reference/javascript_api:company_ids}}\pysigline{\sphinxbfcode{\sphinxupquote{attribute }}\sphinxbfcode{\sphinxupquote{company\_ids}} integer{[}{]}}
\end{fulllineitems}



\begin{fulllineitems}
\phantomsection\label{\detokenize{reference/javascript_api:partner_ids}}\pysigline{\sphinxbfcode{\sphinxupquote{attribute }}\sphinxbfcode{\sphinxupquote{partner\_ids}} integer{[}{]}}~\begin{description}
\item[{used for ‘customers’ and}] \leavevmode
‘suppliers’ mode

\end{description}

\end{fulllineitems}


\end{fulllineitems}


\end{fulllineitems}



\begin{fulllineitems}
\phantomsection\label{\detokenize{reference/javascript_api:validate}}\pysiglinewithargsret{\sphinxbfcode{\sphinxupquote{function }}\sphinxbfcode{\sphinxupquote{validate}}}{\emph{handle}}{{ $\rightarrow$ Deferred\textless{}Array\textgreater{}}}
Mark the account or the partner as reconciled
\begin{quote}\begin{description}
\item[{Parameters}] \leavevmode\begin{itemize}

\sphinxstylestrong{handle} (\sphinxstyleliteralemphasis{\sphinxupquote{string}}\sphinxstyleemphasis{ or }\sphinxstyleliteralemphasis{\sphinxupquote{Array}}\textless{}\sphinxstyleliteralemphasis{\sphinxupquote{string}}\textgreater{})
\end{itemize}

\item[{Returns}] \leavevmode
resolved with the handle array

\item[{Return Type}] \leavevmode
\sphinxstyleliteralemphasis{\sphinxupquote{Deferred}}\textless{}\sphinxstyleliteralemphasis{\sphinxupquote{Array}}\textgreater{}

\end{description}\end{quote}

\end{fulllineitems}


\end{fulllineitems}



\begin{fulllineitems}
\phantomsection\label{\detokenize{reference/javascript_api:StatementModel}}\pysiglinewithargsret{\sphinxbfcode{\sphinxupquote{class }}\sphinxbfcode{\sphinxupquote{StatementModel}}}{\emph{parent}, \emph{options}}{}~\begin{quote}\begin{description}
\item[{Extends}] \leavevmode{\hyperref[\detokenize{reference/javascript_api:web.BasicModel.BasicModel}]{\sphinxcrossref{
BasicModel
}}}
\item[{Parameters}] \leavevmode\begin{itemize}

\sphinxstylestrong{parent} ({\hyperref[\detokenize{reference/javascript_api:Widget}]{\sphinxcrossref{\sphinxstyleliteralemphasis{\sphinxupquote{Widget}}}}})

\sphinxstylestrong{options} (\sphinxstyleliteralemphasis{\sphinxupquote{object}})
\end{itemize}

\end{description}\end{quote}

Model use to fetch, format and update ‘account.bank.statement’ and
‘account.bank.statement.line’ datas allowing reconciliation

The statement internal structure:

\fvset{hllines={, ,}}%
\begin{sphinxVerbatim}[commandchars=\\\{\}]
\PYG{p}{\PYGZob{}}
    \PYG{n+nx}{valuenow}\PYG{o}{:} \PYG{n+nx}{integer}
    \PYG{n+nx}{valuenow}\PYG{o}{:} \PYG{n+nx}{valuemax}
    \PYG{p}{[}\PYG{n+nx}{bank\PYGZus{}statement\PYGZus{}id}\PYG{p}{]}\PYG{o}{:} \PYG{p}{\PYGZob{}}
        \PYG{n+nx}{id}\PYG{o}{:} \PYG{n+nx}{integer}
        \PYG{n+nx}{display\PYGZus{}name}\PYG{o}{:} \PYG{n+nx}{string}
    \PYG{p}{\PYGZcb{}}
    \PYG{n+nx}{reconcileModels}\PYG{o}{:} \PYG{p}{[}\PYG{n+nx}{object}\PYG{p}{]}
    \PYG{n+nx}{accounts}\PYG{o}{:} \PYG{p}{\PYGZob{}}\PYG{n+nx}{id}\PYG{o}{:} \PYG{n+nx}{code}\PYG{p}{\PYGZcb{}}
\PYG{p}{\PYGZcb{}}
\end{sphinxVerbatim}

The internal structure of each line is:

\fvset{hllines={, ,}}%
\begin{sphinxVerbatim}[commandchars=\\\{\}]
\PYG{p}{\PYGZob{}}
   \PYG{n+nx}{balance}\PYG{o}{:} \PYG{p}{\PYGZob{}}
       \PYG{n+nx}{type}\PYG{o}{:} \PYG{n+nx}{number} \PYG{o}{\PYGZhy{}} \PYG{n+nx}{show}\PYG{o}{/}\PYG{n+nx}{hide} \PYG{n+nx}{action} \PYG{n+nx}{button}
       \PYG{n+nx}{amount}\PYG{o}{:} \PYG{n+nx}{number} \PYG{o}{\PYGZhy{}} \PYG{n+nx}{real} \PYG{n+nx}{amount}
       \PYG{n+nx}{amount\PYGZus{}str}\PYG{o}{:} \PYG{n+nx}{string} \PYG{o}{\PYGZhy{}} \PYG{n+nx}{formated} \PYG{n+nx}{amount}
       \PYG{n+nx}{account\PYGZus{}code}\PYG{o}{:} \PYG{n+nx}{string}
   \PYG{p}{\PYGZcb{}}\PYG{p}{,}
   \PYG{n+nx}{st\PYGZus{}line}\PYG{o}{:} \PYG{p}{\PYGZob{}}
       \PYG{n+nx}{partner\PYGZus{}id}\PYG{o}{:} \PYG{n+nx}{integer}
       \PYG{n+nx}{partner\PYGZus{}name}\PYG{o}{:} \PYG{n+nx}{string}
   \PYG{p}{\PYGZcb{}}
   \PYG{n+nx}{mode}\PYG{o}{:} \PYG{n+nx}{string} \PYG{p}{(}\PYG{l+s+s1}{\PYGZsq{}inactive\PYGZsq{}}\PYG{p}{,} \PYG{l+s+s1}{\PYGZsq{}match\PYGZsq{}}\PYG{p}{,} \PYG{l+s+s1}{\PYGZsq{}create\PYGZsq{}}\PYG{p}{)}
   \PYG{n+nx}{reconciliation\PYGZus{}proposition}\PYG{o}{:} \PYG{p}{\PYGZob{}}
       \PYG{n+nx}{id}\PYG{o}{:} \PYG{n+nx}{number}\PYG{o}{\textbar{}}\PYG{n+nx}{string}
       \PYG{n+nx}{partial\PYGZus{}reconcile}\PYG{o}{:} \PYG{k+kr}{boolean}
       \PYG{n+nx}{invalid}\PYG{o}{:} \PYG{k+kr}{boolean} \PYG{o}{\PYGZhy{}} \PYG{n+nx}{through} \PYG{n+nx}{the} \PYG{n+nx}{invalid} \PYG{n+nx}{line} \PYG{p}{(}\PYG{n+nx}{without} \PYG{n+nx}{account}\PYG{p}{,} \PYG{n+nx}{label}\PYG{p}{...}\PYG{p}{)}
       \PYG{n+nx}{is\PYGZus{}tax}\PYG{o}{:} \PYG{k+kr}{boolean}
       \PYG{n+nx}{account\PYGZus{}code}\PYG{o}{:} \PYG{n+nx}{string}
       \PYG{n+nx}{date}\PYG{o}{:} \PYG{n+nx}{string}
       \PYG{n+nx}{date\PYGZus{}maturity}\PYG{o}{:} \PYG{n+nx}{string}
       \PYG{n+nx}{label}\PYG{o}{:} \PYG{n+nx}{string}
       \PYG{n+nx}{amount}\PYG{o}{:} \PYG{n+nx}{number} \PYG{o}{\PYGZhy{}} \PYG{n+nx}{real} \PYG{n+nx}{amount}
       \PYG{n+nx}{amount\PYGZus{}str}\PYG{o}{:} \PYG{n+nx}{string} \PYG{o}{\PYGZhy{}} \PYG{n+nx}{formated} \PYG{n+nx}{amount}
       \PYG{p}{[}\PYG{n+nx}{already\PYGZus{}paid}\PYG{p}{]}\PYG{o}{:} \PYG{k+kr}{boolean}
       \PYG{p}{[}\PYG{n+nx}{partner\PYGZus{}id}\PYG{p}{]}\PYG{o}{:} \PYG{n+nx}{integer}
       \PYG{p}{[}\PYG{n+nx}{partner\PYGZus{}name}\PYG{p}{]}\PYG{o}{:} \PYG{n+nx}{string}
       \PYG{p}{[}\PYG{n+nx}{account\PYGZus{}code}\PYG{p}{]}\PYG{o}{:} \PYG{n+nx}{string}
       \PYG{p}{[}\PYG{n+nx}{journal\PYGZus{}id}\PYG{p}{]}\PYG{o}{:} \PYG{p}{\PYGZob{}}
           \PYG{n+nx}{id}\PYG{o}{:} \PYG{n+nx}{integer}
           \PYG{n+nx}{display\PYGZus{}name}\PYG{o}{:} \PYG{n+nx}{string}
       \PYG{p}{\PYGZcb{}}
       \PYG{p}{[}\PYG{n+nx}{ref}\PYG{p}{]}\PYG{o}{:} \PYG{n+nx}{string}
       \PYG{p}{[}\PYG{n+nx}{is\PYGZus{}partially\PYGZus{}reconciled}\PYG{p}{]}\PYG{o}{:} \PYG{k+kr}{boolean}
       \PYG{p}{[}\PYG{n+nx}{amount\PYGZus{}currency\PYGZus{}str}\PYG{p}{]}\PYG{o}{:} \PYG{n+nx}{string}\PYG{o}{\textbar{}}\PYG{k+kc}{false} \PYG{p}{(}\PYG{n+nx}{amount} \PYG{k}{in} \PYG{n+nx}{record} \PYG{n+nx}{currency}\PYG{p}{)}
   \PYG{p}{\PYGZcb{}}
   \PYG{n+nx}{mv\PYGZus{}lines}\PYG{o}{:} \PYG{n+nx}{object} \PYG{o}{\PYGZhy{}} \PYG{n+nx}{idem} \PYG{n+nx}{than} \PYG{n+nx}{reconciliation\PYGZus{}proposition}
   \PYG{n+nx}{offset}\PYG{o}{:} \PYG{n+nx}{integer}
   \PYG{n+nx}{limitMoveLines}\PYG{o}{:} \PYG{n+nx}{integer}
   \PYG{n+nx}{filter}\PYG{o}{:} \PYG{n+nx}{string}
   \PYG{p}{[}\PYG{n+nx}{createForm}\PYG{p}{]}\PYG{o}{:} \PYG{p}{\PYGZob{}}
       \PYG{n+nx}{account\PYGZus{}id}\PYG{o}{:} \PYG{p}{\PYGZob{}}
           \PYG{n+nx}{id}\PYG{o}{:} \PYG{n+nx}{integer}
           \PYG{n+nx}{display\PYGZus{}name}\PYG{o}{:} \PYG{n+nx}{string}
       \PYG{p}{\PYGZcb{}}
       \PYG{n+nx}{tax\PYGZus{}id}\PYG{o}{:} \PYG{p}{\PYGZob{}}
           \PYG{n+nx}{id}\PYG{o}{:} \PYG{n+nx}{integer}
           \PYG{n+nx}{display\PYGZus{}name}\PYG{o}{:} \PYG{n+nx}{string}
       \PYG{p}{\PYGZcb{}}
       \PYG{n+nx}{analytic\PYGZus{}account\PYGZus{}id}\PYG{o}{:} \PYG{p}{\PYGZob{}}
           \PYG{n+nx}{id}\PYG{o}{:} \PYG{n+nx}{integer}
           \PYG{n+nx}{display\PYGZus{}name}\PYG{o}{:} \PYG{n+nx}{string}
       \PYG{p}{\PYGZcb{}}
       \PYG{n+nx}{label}\PYG{o}{:} \PYG{n+nx}{string}
       \PYG{n+nx}{amount}\PYG{o}{:} \PYG{n+nx}{number}\PYG{p}{,}
       \PYG{p}{[}\PYG{n+nx}{journal\PYGZus{}id}\PYG{p}{]}\PYG{o}{:} \PYG{p}{\PYGZob{}}
           \PYG{n+nx}{id}\PYG{o}{:} \PYG{n+nx}{integer}
           \PYG{n+nx}{display\PYGZus{}name}\PYG{o}{:} \PYG{n+nx}{string}
       \PYG{p}{\PYGZcb{}}
   \PYG{p}{\PYGZcb{}}
\PYG{p}{\PYGZcb{}}
\end{sphinxVerbatim}


\begin{fulllineitems}
\phantomsection\label{\detokenize{reference/javascript_api:addProposition}}\pysiglinewithargsret{\sphinxbfcode{\sphinxupquote{method }}\sphinxbfcode{\sphinxupquote{addProposition}}}{\emph{handle}, \emph{mv\_line\_id}}{{ $\rightarrow$ Deferred}}
add a reconciliation proposition from the matched lines
We also display a warning if the user tries to add 2 line with different
account type
\begin{quote}\begin{description}
\item[{Parameters}] \leavevmode\begin{itemize}

\sphinxstylestrong{handle} (\sphinxstyleliteralemphasis{\sphinxupquote{string}})

\sphinxstylestrong{mv\_line\_id} (\sphinxstyleliteralemphasis{\sphinxupquote{number}})
\end{itemize}

\item[{Return Type}] \leavevmode
\sphinxstyleliteralemphasis{\sphinxupquote{Deferred}}

\end{description}\end{quote}

\end{fulllineitems}



\begin{fulllineitems}
\phantomsection\label{\detokenize{reference/javascript_api:autoReconciliation}}\pysiglinewithargsret{\sphinxbfcode{\sphinxupquote{method }}\sphinxbfcode{\sphinxupquote{autoReconciliation}}}{}{{ $\rightarrow$ Deferred\textless{}Object\textgreater{}}}
send information ‘account.bank.statement.line’ model to reconciliate
lines, call rpc to ‘reconciliation\_widget\_auto\_reconcile’
Update the number of validated line
\begin{quote}\begin{description}
\item[{Returns}] \leavevmode
resolved with an object who contains
  ‘handles’ key and ‘notifications’

\item[{Return Type}] \leavevmode
\sphinxstyleliteralemphasis{\sphinxupquote{Deferred}}\textless{}\sphinxstyleliteralemphasis{\sphinxupquote{Object}}\textgreater{}

\end{description}\end{quote}

\end{fulllineitems}



\begin{fulllineitems}
\phantomsection\label{\detokenize{reference/javascript_api:changeFilter}}\pysiglinewithargsret{\sphinxbfcode{\sphinxupquote{method }}\sphinxbfcode{\sphinxupquote{changeFilter}}}{\emph{handle}, \emph{filter}}{{ $\rightarrow$ Deferred}}
change the filter for the target line and fetch the new matched lines
\begin{quote}\begin{description}
\item[{Parameters}] \leavevmode\begin{itemize}

\sphinxstylestrong{handle} (\sphinxstyleliteralemphasis{\sphinxupquote{string}})

\sphinxstylestrong{filter} (\sphinxstyleliteralemphasis{\sphinxupquote{string}})
\end{itemize}

\item[{Return Type}] \leavevmode
\sphinxstyleliteralemphasis{\sphinxupquote{Deferred}}

\end{description}\end{quote}

\end{fulllineitems}



\begin{fulllineitems}
\phantomsection\label{\detokenize{reference/javascript_api:changeMode}}\pysiglinewithargsret{\sphinxbfcode{\sphinxupquote{method }}\sphinxbfcode{\sphinxupquote{changeMode}}}{\emph{handle}, \emph{mode}}{{ $\rightarrow$ Deferred}}
change the mode line (‘inactive’, ‘match’, ‘create’), and fetch the new
matched lines or prepare to create a new line
\begin{description}
\item[{\sphinxcode{\sphinxupquote{match}}}] \leavevmode
display the matched lines, the user can select the lines to apply
there as proposition

\item[{\sphinxcode{\sphinxupquote{create}}}] \leavevmode
display fields and quick create button to create a new proposition
for the reconciliation

\end{description}
\begin{quote}\begin{description}
\item[{Parameters}] \leavevmode\begin{itemize}

\sphinxstylestrong{handle} (\sphinxstyleliteralemphasis{\sphinxupquote{string}})

\sphinxstylestrong{mode} (inactive\sphinxstyleemphasis{ or }match\sphinxstyleemphasis{ or }create)
\end{itemize}

\item[{Return Type}] \leavevmode
\sphinxstyleliteralemphasis{\sphinxupquote{Deferred}}

\end{description}\end{quote}

\end{fulllineitems}



\begin{fulllineitems}
\phantomsection\label{\detokenize{reference/javascript_api:changeName}}\pysiglinewithargsret{\sphinxbfcode{\sphinxupquote{method }}\sphinxbfcode{\sphinxupquote{changeName}}}{\emph{name}}{{ $\rightarrow$ Deferred}}
call ‘write’ method on the ‘account.bank.statement’
\begin{quote}\begin{description}
\item[{Parameters}] \leavevmode\begin{itemize}

\sphinxstylestrong{name} (\sphinxstyleliteralemphasis{\sphinxupquote{string}})
\end{itemize}

\item[{Return Type}] \leavevmode
\sphinxstyleliteralemphasis{\sphinxupquote{Deferred}}

\end{description}\end{quote}

\end{fulllineitems}



\begin{fulllineitems}
\phantomsection\label{\detokenize{reference/javascript_api:changeOffset}}\pysiglinewithargsret{\sphinxbfcode{\sphinxupquote{method }}\sphinxbfcode{\sphinxupquote{changeOffset}}}{\emph{handle}, \emph{offset}}{{ $\rightarrow$ Deferred}}
change the offset for the matched lines, and fetch the new matched lines
\begin{quote}\begin{description}
\item[{Parameters}] \leavevmode\begin{itemize}

\sphinxstylestrong{handle} (\sphinxstyleliteralemphasis{\sphinxupquote{string}})

\sphinxstylestrong{offset} (\sphinxstyleliteralemphasis{\sphinxupquote{number}})
\end{itemize}

\item[{Return Type}] \leavevmode
\sphinxstyleliteralemphasis{\sphinxupquote{Deferred}}

\end{description}\end{quote}

\end{fulllineitems}



\begin{fulllineitems}
\phantomsection\label{\detokenize{reference/javascript_api:changePartner}}\pysiglinewithargsret{\sphinxbfcode{\sphinxupquote{method }}\sphinxbfcode{\sphinxupquote{changePartner}}}{\emph{handle}, \emph{partner}}{{ $\rightarrow$ Deferred}}
change the partner on the line and fetch the new matched lines
\begin{quote}\begin{description}
\item[{Parameters}] \leavevmode\begin{itemize}

\sphinxstylestrong{handle} (\sphinxstyleliteralemphasis{\sphinxupquote{string}})

\sphinxstylestrong{partner} ({\hyperref[\detokenize{reference/javascript_api:account.ReconciliationModel.ChangePartnerPartner}]{\sphinxcrossref{\sphinxstyleliteralemphasis{\sphinxupquote{ChangePartnerPartner}}}}})
\end{itemize}

\item[{Return Type}] \leavevmode
\sphinxstyleliteralemphasis{\sphinxupquote{Deferred}}

\end{description}\end{quote}


\begin{fulllineitems}
\phantomsection\label{\detokenize{reference/javascript_api:ChangePartnerPartner}}\pysiglinewithargsret{\sphinxbfcode{\sphinxupquote{class }}\sphinxbfcode{\sphinxupquote{ChangePartnerPartner}}}{}{}~

\begin{fulllineitems}
\phantomsection\label{\detokenize{reference/javascript_api:display_name}}\pysigline{\sphinxbfcode{\sphinxupquote{attribute }}\sphinxbfcode{\sphinxupquote{display\_name}} string}
\end{fulllineitems}



\begin{fulllineitems}
\phantomsection\label{\detokenize{reference/javascript_api:id}}\pysigline{\sphinxbfcode{\sphinxupquote{attribute }}\sphinxbfcode{\sphinxupquote{id}} number}
\end{fulllineitems}


\end{fulllineitems}


\end{fulllineitems}



\begin{fulllineitems}
\phantomsection\label{\detokenize{reference/javascript_api:closeStatement}}\pysiglinewithargsret{\sphinxbfcode{\sphinxupquote{function }}\sphinxbfcode{\sphinxupquote{closeStatement}}}{}{{ $\rightarrow$ Deferred\textless{}number\textgreater{}}}
close the statement
\begin{quote}\begin{description}
\item[{Returns}] \leavevmode
resolves to the res\_id of the closed statements

\item[{Return Type}] \leavevmode
\sphinxstyleliteralemphasis{\sphinxupquote{Deferred}}\textless{}\sphinxstyleliteralemphasis{\sphinxupquote{number}}\textgreater{}

\end{description}\end{quote}

\end{fulllineitems}



\begin{fulllineitems}
\phantomsection\label{\detokenize{reference/javascript_api:createProposition}}\pysiglinewithargsret{\sphinxbfcode{\sphinxupquote{function }}\sphinxbfcode{\sphinxupquote{createProposition}}}{\emph{handle}}{{ $\rightarrow$ Deferred}}
then open the first available line
\begin{quote}\begin{description}
\item[{Parameters}] \leavevmode\begin{itemize}

\sphinxstylestrong{handle} (\sphinxstyleliteralemphasis{\sphinxupquote{string}})
\end{itemize}

\item[{Return Type}] \leavevmode
\sphinxstyleliteralemphasis{\sphinxupquote{Deferred}}

\end{description}\end{quote}

\end{fulllineitems}



\begin{fulllineitems}
\phantomsection\label{\detokenize{reference/javascript_api:getContext}}\pysiglinewithargsret{\sphinxbfcode{\sphinxupquote{function }}\sphinxbfcode{\sphinxupquote{getContext}}}{}{{ $\rightarrow$ Object}}
Return context information and journal\_id
\begin{quote}\begin{description}
\item[{Returns}] \leavevmode
context

\item[{Return Type}] \leavevmode
\sphinxstyleliteralemphasis{\sphinxupquote{Object}}

\end{description}\end{quote}

\end{fulllineitems}



\begin{fulllineitems}
\phantomsection\label{\detokenize{reference/javascript_api:getStatementLines}}\pysiglinewithargsret{\sphinxbfcode{\sphinxupquote{function }}\sphinxbfcode{\sphinxupquote{getStatementLines}}}{}{{ $\rightarrow$ Object}}
Return the lines that needs to be displayed by the widget
\begin{quote}\begin{description}
\item[{Returns}] \leavevmode
lines that are loaded and not yet displayed

\item[{Return Type}] \leavevmode
\sphinxstyleliteralemphasis{\sphinxupquote{Object}}

\end{description}\end{quote}

\end{fulllineitems}



\begin{fulllineitems}
\phantomsection\label{\detokenize{reference/javascript_api:hasMoreLines}}\pysiglinewithargsret{\sphinxbfcode{\sphinxupquote{function }}\sphinxbfcode{\sphinxupquote{hasMoreLines}}}{}{{ $\rightarrow$ boolean}}
Return a boolean telling if load button needs to be displayed or not
\begin{quote}\begin{description}
\item[{Returns}] \leavevmode
true if load more button needs to be displayed

\item[{Return Type}] \leavevmode
\sphinxstyleliteralemphasis{\sphinxupquote{boolean}}

\end{description}\end{quote}

\end{fulllineitems}



\begin{fulllineitems}
\phantomsection\label{\detokenize{reference/javascript_api:getLine}}\pysiglinewithargsret{\sphinxbfcode{\sphinxupquote{function }}\sphinxbfcode{\sphinxupquote{getLine}}}{\emph{handle}}{{ $\rightarrow$ Object}}
get the line data for this handle
\begin{quote}\begin{description}
\item[{Parameters}] \leavevmode\begin{itemize}

\sphinxstylestrong{handle} (\sphinxstyleliteralemphasis{\sphinxupquote{Object}})
\end{itemize}

\item[{Return Type}] \leavevmode
\sphinxstyleliteralemphasis{\sphinxupquote{Object}}

\end{description}\end{quote}

\end{fulllineitems}



\begin{fulllineitems}
\phantomsection\label{\detokenize{reference/javascript_api:load}}\pysiglinewithargsret{\sphinxbfcode{\sphinxupquote{function }}\sphinxbfcode{\sphinxupquote{load}}}{\emph{context}}{{ $\rightarrow$ Deferred}}
load data from
\begin{itemize}
\item {} 
‘account.bank.statement’ fetch the line id and bank\_statement\_id info

\item {} 
‘account.reconcile.model’  fetch all reconcile model (for quick add)

\item {} 
‘account.account’ fetch all account code

\item {} 
‘account.bank.statement.line’ fetch each line data

\end{itemize}
\begin{quote}\begin{description}
\item[{Parameters}] \leavevmode\begin{itemize}

\sphinxstylestrong{context} ({\hyperref[\detokenize{reference/javascript_api:account.ReconciliationModel.LoadContext}]{\sphinxcrossref{\sphinxstyleliteralemphasis{\sphinxupquote{LoadContext}}}}})
\end{itemize}

\item[{Return Type}] \leavevmode
\sphinxstyleliteralemphasis{\sphinxupquote{Deferred}}

\end{description}\end{quote}


\begin{fulllineitems}
\phantomsection\label{\detokenize{reference/javascript_api:LoadContext}}\pysiglinewithargsret{\sphinxbfcode{\sphinxupquote{class }}\sphinxbfcode{\sphinxupquote{LoadContext}}}{}{}~

\begin{fulllineitems}
\phantomsection\label{\detokenize{reference/javascript_api:statement_ids}}\pysigline{\sphinxbfcode{\sphinxupquote{attribute }}\sphinxbfcode{\sphinxupquote{statement\_ids}} number{[}{]}}
\end{fulllineitems}


\end{fulllineitems}


\end{fulllineitems}



\begin{fulllineitems}
\phantomsection\label{\detokenize{reference/javascript_api:loadMore}}\pysiglinewithargsret{\sphinxbfcode{\sphinxupquote{function }}\sphinxbfcode{\sphinxupquote{loadMore}}}{\emph{qty}}{{ $\rightarrow$ Deferred}}
Load more bank statement line
\begin{quote}\begin{description}
\item[{Parameters}] \leavevmode\begin{itemize}

\sphinxstylestrong{qty} (\sphinxstyleliteralemphasis{\sphinxupquote{integer}}) \textendash{} quantity to load
\end{itemize}

\item[{Return Type}] \leavevmode
\sphinxstyleliteralemphasis{\sphinxupquote{Deferred}}

\end{description}\end{quote}

\end{fulllineitems}



\begin{fulllineitems}
\phantomsection\label{\detokenize{reference/javascript_api:loadData}}\pysiglinewithargsret{\sphinxbfcode{\sphinxupquote{function }}\sphinxbfcode{\sphinxupquote{loadData}}}{\emph{ids}, \emph{excluded\_ids}}{{ $\rightarrow$ Deferred}}
RPC method to load informations on lines
\begin{quote}\begin{description}
\item[{Parameters}] \leavevmode\begin{itemize}

\sphinxstylestrong{ids} (\sphinxstyleliteralemphasis{\sphinxupquote{Array}}) \textendash{} ids of bank statement line passed to rpc call

\sphinxstylestrong{excluded\_ids} (\sphinxstyleliteralemphasis{\sphinxupquote{Array}}) \textendash{} list of move\_line ids that needs to be excluded from search
\end{itemize}

\item[{Return Type}] \leavevmode
\sphinxstyleliteralemphasis{\sphinxupquote{Deferred}}

\end{description}\end{quote}

\end{fulllineitems}



\begin{fulllineitems}
\phantomsection\label{\detokenize{reference/javascript_api:quickCreateProposition}}\pysiglinewithargsret{\sphinxbfcode{\sphinxupquote{function }}\sphinxbfcode{\sphinxupquote{quickCreateProposition}}}{\emph{handle}, \emph{reconcileModelId}}{{ $\rightarrow$ Deferred}}
Add lines into the propositions from the reconcile model
Can add 2 lines, and each with its taxes. The second line become editable
in the create mode.
\begin{quote}\begin{description}
\item[{Parameters}] \leavevmode\begin{itemize}

\sphinxstylestrong{handle} (\sphinxstyleliteralemphasis{\sphinxupquote{string}})

\sphinxstylestrong{reconcileModelId} (\sphinxstyleliteralemphasis{\sphinxupquote{integer}})
\end{itemize}

\item[{Return Type}] \leavevmode
\sphinxstyleliteralemphasis{\sphinxupquote{Deferred}}

\end{description}\end{quote}

\end{fulllineitems}



\begin{fulllineitems}
\phantomsection\label{\detokenize{reference/javascript_api:removeProposition}}\pysiglinewithargsret{\sphinxbfcode{\sphinxupquote{function }}\sphinxbfcode{\sphinxupquote{removeProposition}}}{\emph{handle}, \emph{id}}{{ $\rightarrow$ Deferred}}
Remove a proposition and switch to an active mode (‘create’ or ‘match’)
\begin{quote}\begin{description}
\item[{Parameters}] \leavevmode\begin{itemize}

\sphinxstylestrong{handle} (\sphinxstyleliteralemphasis{\sphinxupquote{string}})

\sphinxstylestrong{id} (\sphinxstyleliteralemphasis{\sphinxupquote{number}}) \textendash{} (move line id)
\end{itemize}

\item[{Return Type}] \leavevmode
\sphinxstyleliteralemphasis{\sphinxupquote{Deferred}}

\end{description}\end{quote}

\end{fulllineitems}



\begin{fulllineitems}
\phantomsection\label{\detokenize{reference/javascript_api:togglePartialReconcile}}\pysiglinewithargsret{\sphinxbfcode{\sphinxupquote{function }}\sphinxbfcode{\sphinxupquote{togglePartialReconcile}}}{\emph{handle}}{{ $\rightarrow$ Deferred}}
Force the partial reconciliation to display the reconciliate button.
This method should only  be called when there is onely one proposition.
\begin{quote}\begin{description}
\item[{Parameters}] \leavevmode\begin{itemize}

\sphinxstylestrong{handle} (\sphinxstyleliteralemphasis{\sphinxupquote{string}})
\end{itemize}

\item[{Return Type}] \leavevmode
\sphinxstyleliteralemphasis{\sphinxupquote{Deferred}}

\end{description}\end{quote}

\end{fulllineitems}



\begin{fulllineitems}
\phantomsection\label{\detokenize{reference/javascript_api:updateProposition}}\pysiglinewithargsret{\sphinxbfcode{\sphinxupquote{function }}\sphinxbfcode{\sphinxupquote{updateProposition}}}{\emph{handle}, \emph{values}}{{ $\rightarrow$ Deferred}}
Change the value of the editable proposition line or create a new one.

If the editable line comes from a reconcile model with 2 lines
and their ‘amount\_type’ is “percent” 
and their total equals 100\% (this doesn’t take into account the taxes
who can be included or not)
Then the total is recomputed to have 100\%.
\begin{quote}\begin{description}
\item[{Parameters}] \leavevmode\begin{itemize}

\sphinxstylestrong{handle} (\sphinxstyleliteralemphasis{\sphinxupquote{string}})

\sphinxstylestrong{values} (\sphinxstyleliteralemphasis{\sphinxupquote{any}})
\end{itemize}

\item[{Return Type}] \leavevmode
\sphinxstyleliteralemphasis{\sphinxupquote{Deferred}}

\end{description}\end{quote}

\end{fulllineitems}



\begin{fulllineitems}
\phantomsection\label{\detokenize{reference/javascript_api:validate}}\pysiglinewithargsret{\sphinxbfcode{\sphinxupquote{function }}\sphinxbfcode{\sphinxupquote{validate}}}{\emph{handle}}{{ $\rightarrow$ Deferred\textless{}Object\textgreater{}}}
Format the value and send it to ‘account.bank.statement.line’ model
Update the number of validated lines
\begin{quote}\begin{description}
\item[{Parameters}] \leavevmode\begin{itemize}

\sphinxstylestrong{handle} (\sphinxstyleliteralemphasis{\sphinxupquote{string}}\sphinxstyleemphasis{ or }\sphinxstyleliteralemphasis{\sphinxupquote{Array}}\textless{}\sphinxstyleliteralemphasis{\sphinxupquote{string}}\textgreater{})
\end{itemize}

\item[{Returns}] \leavevmode
resolved with an object who contains
  ‘handles’ key

\item[{Return Type}] \leavevmode
\sphinxstyleliteralemphasis{\sphinxupquote{Deferred}}\textless{}\sphinxstyleliteralemphasis{\sphinxupquote{Object}}\textgreater{}

\end{description}\end{quote}

\end{fulllineitems}


\end{fulllineitems}



\begin{fulllineitems}
\phantomsection\label{\detokenize{reference/javascript_api:account.ReconciliationModel.}}\pysigline{\sphinxbfcode{\sphinxupquote{namespace }}\sphinxbfcode{\sphinxupquote{}}}~

\begin{fulllineitems}
\phantomsection\label{\detokenize{reference/javascript_api:StatementModel}}\pysiglinewithargsret{\sphinxbfcode{\sphinxupquote{class }}\sphinxbfcode{\sphinxupquote{StatementModel}}}{\emph{parent}, \emph{options}}{}~\begin{quote}\begin{description}
\item[{Extends}] \leavevmode{\hyperref[\detokenize{reference/javascript_api:web.BasicModel.BasicModel}]{\sphinxcrossref{
BasicModel
}}}
\item[{Parameters}] \leavevmode\begin{itemize}

\sphinxstylestrong{parent} ({\hyperref[\detokenize{reference/javascript_api:Widget}]{\sphinxcrossref{\sphinxstyleliteralemphasis{\sphinxupquote{Widget}}}}})

\sphinxstylestrong{options} (\sphinxstyleliteralemphasis{\sphinxupquote{object}})
\end{itemize}

\end{description}\end{quote}

Model use to fetch, format and update ‘account.bank.statement’ and
‘account.bank.statement.line’ datas allowing reconciliation

The statement internal structure:

\fvset{hllines={, ,}}%
\begin{sphinxVerbatim}[commandchars=\\\{\}]
\PYG{p}{\PYGZob{}}
    \PYG{n+nx}{valuenow}\PYG{o}{:} \PYG{n+nx}{integer}
    \PYG{n+nx}{valuenow}\PYG{o}{:} \PYG{n+nx}{valuemax}
    \PYG{p}{[}\PYG{n+nx}{bank\PYGZus{}statement\PYGZus{}id}\PYG{p}{]}\PYG{o}{:} \PYG{p}{\PYGZob{}}
        \PYG{n+nx}{id}\PYG{o}{:} \PYG{n+nx}{integer}
        \PYG{n+nx}{display\PYGZus{}name}\PYG{o}{:} \PYG{n+nx}{string}
    \PYG{p}{\PYGZcb{}}
    \PYG{n+nx}{reconcileModels}\PYG{o}{:} \PYG{p}{[}\PYG{n+nx}{object}\PYG{p}{]}
    \PYG{n+nx}{accounts}\PYG{o}{:} \PYG{p}{\PYGZob{}}\PYG{n+nx}{id}\PYG{o}{:} \PYG{n+nx}{code}\PYG{p}{\PYGZcb{}}
\PYG{p}{\PYGZcb{}}
\end{sphinxVerbatim}

The internal structure of each line is:

\fvset{hllines={, ,}}%
\begin{sphinxVerbatim}[commandchars=\\\{\}]
\PYG{p}{\PYGZob{}}
   \PYG{n+nx}{balance}\PYG{o}{:} \PYG{p}{\PYGZob{}}
       \PYG{n+nx}{type}\PYG{o}{:} \PYG{n+nx}{number} \PYG{o}{\PYGZhy{}} \PYG{n+nx}{show}\PYG{o}{/}\PYG{n+nx}{hide} \PYG{n+nx}{action} \PYG{n+nx}{button}
       \PYG{n+nx}{amount}\PYG{o}{:} \PYG{n+nx}{number} \PYG{o}{\PYGZhy{}} \PYG{n+nx}{real} \PYG{n+nx}{amount}
       \PYG{n+nx}{amount\PYGZus{}str}\PYG{o}{:} \PYG{n+nx}{string} \PYG{o}{\PYGZhy{}} \PYG{n+nx}{formated} \PYG{n+nx}{amount}
       \PYG{n+nx}{account\PYGZus{}code}\PYG{o}{:} \PYG{n+nx}{string}
   \PYG{p}{\PYGZcb{}}\PYG{p}{,}
   \PYG{n+nx}{st\PYGZus{}line}\PYG{o}{:} \PYG{p}{\PYGZob{}}
       \PYG{n+nx}{partner\PYGZus{}id}\PYG{o}{:} \PYG{n+nx}{integer}
       \PYG{n+nx}{partner\PYGZus{}name}\PYG{o}{:} \PYG{n+nx}{string}
   \PYG{p}{\PYGZcb{}}
   \PYG{n+nx}{mode}\PYG{o}{:} \PYG{n+nx}{string} \PYG{p}{(}\PYG{l+s+s1}{\PYGZsq{}inactive\PYGZsq{}}\PYG{p}{,} \PYG{l+s+s1}{\PYGZsq{}match\PYGZsq{}}\PYG{p}{,} \PYG{l+s+s1}{\PYGZsq{}create\PYGZsq{}}\PYG{p}{)}
   \PYG{n+nx}{reconciliation\PYGZus{}proposition}\PYG{o}{:} \PYG{p}{\PYGZob{}}
       \PYG{n+nx}{id}\PYG{o}{:} \PYG{n+nx}{number}\PYG{o}{\textbar{}}\PYG{n+nx}{string}
       \PYG{n+nx}{partial\PYGZus{}reconcile}\PYG{o}{:} \PYG{k+kr}{boolean}
       \PYG{n+nx}{invalid}\PYG{o}{:} \PYG{k+kr}{boolean} \PYG{o}{\PYGZhy{}} \PYG{n+nx}{through} \PYG{n+nx}{the} \PYG{n+nx}{invalid} \PYG{n+nx}{line} \PYG{p}{(}\PYG{n+nx}{without} \PYG{n+nx}{account}\PYG{p}{,} \PYG{n+nx}{label}\PYG{p}{...}\PYG{p}{)}
       \PYG{n+nx}{is\PYGZus{}tax}\PYG{o}{:} \PYG{k+kr}{boolean}
       \PYG{n+nx}{account\PYGZus{}code}\PYG{o}{:} \PYG{n+nx}{string}
       \PYG{n+nx}{date}\PYG{o}{:} \PYG{n+nx}{string}
       \PYG{n+nx}{date\PYGZus{}maturity}\PYG{o}{:} \PYG{n+nx}{string}
       \PYG{n+nx}{label}\PYG{o}{:} \PYG{n+nx}{string}
       \PYG{n+nx}{amount}\PYG{o}{:} \PYG{n+nx}{number} \PYG{o}{\PYGZhy{}} \PYG{n+nx}{real} \PYG{n+nx}{amount}
       \PYG{n+nx}{amount\PYGZus{}str}\PYG{o}{:} \PYG{n+nx}{string} \PYG{o}{\PYGZhy{}} \PYG{n+nx}{formated} \PYG{n+nx}{amount}
       \PYG{p}{[}\PYG{n+nx}{already\PYGZus{}paid}\PYG{p}{]}\PYG{o}{:} \PYG{k+kr}{boolean}
       \PYG{p}{[}\PYG{n+nx}{partner\PYGZus{}id}\PYG{p}{]}\PYG{o}{:} \PYG{n+nx}{integer}
       \PYG{p}{[}\PYG{n+nx}{partner\PYGZus{}name}\PYG{p}{]}\PYG{o}{:} \PYG{n+nx}{string}
       \PYG{p}{[}\PYG{n+nx}{account\PYGZus{}code}\PYG{p}{]}\PYG{o}{:} \PYG{n+nx}{string}
       \PYG{p}{[}\PYG{n+nx}{journal\PYGZus{}id}\PYG{p}{]}\PYG{o}{:} \PYG{p}{\PYGZob{}}
           \PYG{n+nx}{id}\PYG{o}{:} \PYG{n+nx}{integer}
           \PYG{n+nx}{display\PYGZus{}name}\PYG{o}{:} \PYG{n+nx}{string}
       \PYG{p}{\PYGZcb{}}
       \PYG{p}{[}\PYG{n+nx}{ref}\PYG{p}{]}\PYG{o}{:} \PYG{n+nx}{string}
       \PYG{p}{[}\PYG{n+nx}{is\PYGZus{}partially\PYGZus{}reconciled}\PYG{p}{]}\PYG{o}{:} \PYG{k+kr}{boolean}
       \PYG{p}{[}\PYG{n+nx}{amount\PYGZus{}currency\PYGZus{}str}\PYG{p}{]}\PYG{o}{:} \PYG{n+nx}{string}\PYG{o}{\textbar{}}\PYG{k+kc}{false} \PYG{p}{(}\PYG{n+nx}{amount} \PYG{k}{in} \PYG{n+nx}{record} \PYG{n+nx}{currency}\PYG{p}{)}
   \PYG{p}{\PYGZcb{}}
   \PYG{n+nx}{mv\PYGZus{}lines}\PYG{o}{:} \PYG{n+nx}{object} \PYG{o}{\PYGZhy{}} \PYG{n+nx}{idem} \PYG{n+nx}{than} \PYG{n+nx}{reconciliation\PYGZus{}proposition}
   \PYG{n+nx}{offset}\PYG{o}{:} \PYG{n+nx}{integer}
   \PYG{n+nx}{limitMoveLines}\PYG{o}{:} \PYG{n+nx}{integer}
   \PYG{n+nx}{filter}\PYG{o}{:} \PYG{n+nx}{string}
   \PYG{p}{[}\PYG{n+nx}{createForm}\PYG{p}{]}\PYG{o}{:} \PYG{p}{\PYGZob{}}
       \PYG{n+nx}{account\PYGZus{}id}\PYG{o}{:} \PYG{p}{\PYGZob{}}
           \PYG{n+nx}{id}\PYG{o}{:} \PYG{n+nx}{integer}
           \PYG{n+nx}{display\PYGZus{}name}\PYG{o}{:} \PYG{n+nx}{string}
       \PYG{p}{\PYGZcb{}}
       \PYG{n+nx}{tax\PYGZus{}id}\PYG{o}{:} \PYG{p}{\PYGZob{}}
           \PYG{n+nx}{id}\PYG{o}{:} \PYG{n+nx}{integer}
           \PYG{n+nx}{display\PYGZus{}name}\PYG{o}{:} \PYG{n+nx}{string}
       \PYG{p}{\PYGZcb{}}
       \PYG{n+nx}{analytic\PYGZus{}account\PYGZus{}id}\PYG{o}{:} \PYG{p}{\PYGZob{}}
           \PYG{n+nx}{id}\PYG{o}{:} \PYG{n+nx}{integer}
           \PYG{n+nx}{display\PYGZus{}name}\PYG{o}{:} \PYG{n+nx}{string}
       \PYG{p}{\PYGZcb{}}
       \PYG{n+nx}{label}\PYG{o}{:} \PYG{n+nx}{string}
       \PYG{n+nx}{amount}\PYG{o}{:} \PYG{n+nx}{number}\PYG{p}{,}
       \PYG{p}{[}\PYG{n+nx}{journal\PYGZus{}id}\PYG{p}{]}\PYG{o}{:} \PYG{p}{\PYGZob{}}
           \PYG{n+nx}{id}\PYG{o}{:} \PYG{n+nx}{integer}
           \PYG{n+nx}{display\PYGZus{}name}\PYG{o}{:} \PYG{n+nx}{string}
       \PYG{p}{\PYGZcb{}}
   \PYG{p}{\PYGZcb{}}
\PYG{p}{\PYGZcb{}}
\end{sphinxVerbatim}


\begin{fulllineitems}
\phantomsection\label{\detokenize{reference/javascript_api:addProposition}}\pysiglinewithargsret{\sphinxbfcode{\sphinxupquote{method }}\sphinxbfcode{\sphinxupquote{addProposition}}}{\emph{handle}, \emph{mv\_line\_id}}{{ $\rightarrow$ Deferred}}
add a reconciliation proposition from the matched lines
We also display a warning if the user tries to add 2 line with different
account type
\begin{quote}\begin{description}
\item[{Parameters}] \leavevmode\begin{itemize}

\sphinxstylestrong{handle} (\sphinxstyleliteralemphasis{\sphinxupquote{string}})

\sphinxstylestrong{mv\_line\_id} (\sphinxstyleliteralemphasis{\sphinxupquote{number}})
\end{itemize}

\item[{Return Type}] \leavevmode
\sphinxstyleliteralemphasis{\sphinxupquote{Deferred}}

\end{description}\end{quote}

\end{fulllineitems}



\begin{fulllineitems}
\phantomsection\label{\detokenize{reference/javascript_api:autoReconciliation}}\pysiglinewithargsret{\sphinxbfcode{\sphinxupquote{method }}\sphinxbfcode{\sphinxupquote{autoReconciliation}}}{}{{ $\rightarrow$ Deferred\textless{}Object\textgreater{}}}
send information ‘account.bank.statement.line’ model to reconciliate
lines, call rpc to ‘reconciliation\_widget\_auto\_reconcile’
Update the number of validated line
\begin{quote}\begin{description}
\item[{Returns}] \leavevmode
resolved with an object who contains
  ‘handles’ key and ‘notifications’

\item[{Return Type}] \leavevmode
\sphinxstyleliteralemphasis{\sphinxupquote{Deferred}}\textless{}\sphinxstyleliteralemphasis{\sphinxupquote{Object}}\textgreater{}

\end{description}\end{quote}

\end{fulllineitems}



\begin{fulllineitems}
\phantomsection\label{\detokenize{reference/javascript_api:changeFilter}}\pysiglinewithargsret{\sphinxbfcode{\sphinxupquote{method }}\sphinxbfcode{\sphinxupquote{changeFilter}}}{\emph{handle}, \emph{filter}}{{ $\rightarrow$ Deferred}}
change the filter for the target line and fetch the new matched lines
\begin{quote}\begin{description}
\item[{Parameters}] \leavevmode\begin{itemize}

\sphinxstylestrong{handle} (\sphinxstyleliteralemphasis{\sphinxupquote{string}})

\sphinxstylestrong{filter} (\sphinxstyleliteralemphasis{\sphinxupquote{string}})
\end{itemize}

\item[{Return Type}] \leavevmode
\sphinxstyleliteralemphasis{\sphinxupquote{Deferred}}

\end{description}\end{quote}

\end{fulllineitems}



\begin{fulllineitems}
\phantomsection\label{\detokenize{reference/javascript_api:changeMode}}\pysiglinewithargsret{\sphinxbfcode{\sphinxupquote{method }}\sphinxbfcode{\sphinxupquote{changeMode}}}{\emph{handle}, \emph{mode}}{{ $\rightarrow$ Deferred}}
change the mode line (‘inactive’, ‘match’, ‘create’), and fetch the new
matched lines or prepare to create a new line
\begin{description}
\item[{\sphinxcode{\sphinxupquote{match}}}] \leavevmode
display the matched lines, the user can select the lines to apply
there as proposition

\item[{\sphinxcode{\sphinxupquote{create}}}] \leavevmode
display fields and quick create button to create a new proposition
for the reconciliation

\end{description}
\begin{quote}\begin{description}
\item[{Parameters}] \leavevmode\begin{itemize}

\sphinxstylestrong{handle} (\sphinxstyleliteralemphasis{\sphinxupquote{string}})

\sphinxstylestrong{mode} (inactive\sphinxstyleemphasis{ or }match\sphinxstyleemphasis{ or }create)
\end{itemize}

\item[{Return Type}] \leavevmode
\sphinxstyleliteralemphasis{\sphinxupquote{Deferred}}

\end{description}\end{quote}

\end{fulllineitems}



\begin{fulllineitems}
\phantomsection\label{\detokenize{reference/javascript_api:changeName}}\pysiglinewithargsret{\sphinxbfcode{\sphinxupquote{method }}\sphinxbfcode{\sphinxupquote{changeName}}}{\emph{name}}{{ $\rightarrow$ Deferred}}
call ‘write’ method on the ‘account.bank.statement’
\begin{quote}\begin{description}
\item[{Parameters}] \leavevmode\begin{itemize}

\sphinxstylestrong{name} (\sphinxstyleliteralemphasis{\sphinxupquote{string}})
\end{itemize}

\item[{Return Type}] \leavevmode
\sphinxstyleliteralemphasis{\sphinxupquote{Deferred}}

\end{description}\end{quote}

\end{fulllineitems}



\begin{fulllineitems}
\phantomsection\label{\detokenize{reference/javascript_api:changeOffset}}\pysiglinewithargsret{\sphinxbfcode{\sphinxupquote{method }}\sphinxbfcode{\sphinxupquote{changeOffset}}}{\emph{handle}, \emph{offset}}{{ $\rightarrow$ Deferred}}
change the offset for the matched lines, and fetch the new matched lines
\begin{quote}\begin{description}
\item[{Parameters}] \leavevmode\begin{itemize}

\sphinxstylestrong{handle} (\sphinxstyleliteralemphasis{\sphinxupquote{string}})

\sphinxstylestrong{offset} (\sphinxstyleliteralemphasis{\sphinxupquote{number}})
\end{itemize}

\item[{Return Type}] \leavevmode
\sphinxstyleliteralemphasis{\sphinxupquote{Deferred}}

\end{description}\end{quote}

\end{fulllineitems}



\begin{fulllineitems}
\phantomsection\label{\detokenize{reference/javascript_api:changePartner}}\pysiglinewithargsret{\sphinxbfcode{\sphinxupquote{method }}\sphinxbfcode{\sphinxupquote{changePartner}}}{\emph{handle}, \emph{partner}}{{ $\rightarrow$ Deferred}}
change the partner on the line and fetch the new matched lines
\begin{quote}\begin{description}
\item[{Parameters}] \leavevmode\begin{itemize}

\sphinxstylestrong{handle} (\sphinxstyleliteralemphasis{\sphinxupquote{string}})

\sphinxstylestrong{partner} ({\hyperref[\detokenize{reference/javascript_api:account.ReconciliationModel.ChangePartnerPartner}]{\sphinxcrossref{\sphinxstyleliteralemphasis{\sphinxupquote{ChangePartnerPartner}}}}})
\end{itemize}

\item[{Return Type}] \leavevmode
\sphinxstyleliteralemphasis{\sphinxupquote{Deferred}}

\end{description}\end{quote}


\begin{fulllineitems}
\phantomsection\label{\detokenize{reference/javascript_api:ChangePartnerPartner}}\pysiglinewithargsret{\sphinxbfcode{\sphinxupquote{class }}\sphinxbfcode{\sphinxupquote{ChangePartnerPartner}}}{}{}~

\begin{fulllineitems}
\phantomsection\label{\detokenize{reference/javascript_api:display_name}}\pysigline{\sphinxbfcode{\sphinxupquote{attribute }}\sphinxbfcode{\sphinxupquote{display\_name}} string}
\end{fulllineitems}



\begin{fulllineitems}
\phantomsection\label{\detokenize{reference/javascript_api:id}}\pysigline{\sphinxbfcode{\sphinxupquote{attribute }}\sphinxbfcode{\sphinxupquote{id}} number}
\end{fulllineitems}


\end{fulllineitems}


\end{fulllineitems}



\begin{fulllineitems}
\phantomsection\label{\detokenize{reference/javascript_api:closeStatement}}\pysiglinewithargsret{\sphinxbfcode{\sphinxupquote{function }}\sphinxbfcode{\sphinxupquote{closeStatement}}}{}{{ $\rightarrow$ Deferred\textless{}number\textgreater{}}}
close the statement
\begin{quote}\begin{description}
\item[{Returns}] \leavevmode
resolves to the res\_id of the closed statements

\item[{Return Type}] \leavevmode
\sphinxstyleliteralemphasis{\sphinxupquote{Deferred}}\textless{}\sphinxstyleliteralemphasis{\sphinxupquote{number}}\textgreater{}

\end{description}\end{quote}

\end{fulllineitems}



\begin{fulllineitems}
\phantomsection\label{\detokenize{reference/javascript_api:createProposition}}\pysiglinewithargsret{\sphinxbfcode{\sphinxupquote{function }}\sphinxbfcode{\sphinxupquote{createProposition}}}{\emph{handle}}{{ $\rightarrow$ Deferred}}
then open the first available line
\begin{quote}\begin{description}
\item[{Parameters}] \leavevmode\begin{itemize}

\sphinxstylestrong{handle} (\sphinxstyleliteralemphasis{\sphinxupquote{string}})
\end{itemize}

\item[{Return Type}] \leavevmode
\sphinxstyleliteralemphasis{\sphinxupquote{Deferred}}

\end{description}\end{quote}

\end{fulllineitems}



\begin{fulllineitems}
\phantomsection\label{\detokenize{reference/javascript_api:getContext}}\pysiglinewithargsret{\sphinxbfcode{\sphinxupquote{function }}\sphinxbfcode{\sphinxupquote{getContext}}}{}{{ $\rightarrow$ Object}}
Return context information and journal\_id
\begin{quote}\begin{description}
\item[{Returns}] \leavevmode
context

\item[{Return Type}] \leavevmode
\sphinxstyleliteralemphasis{\sphinxupquote{Object}}

\end{description}\end{quote}

\end{fulllineitems}



\begin{fulllineitems}
\phantomsection\label{\detokenize{reference/javascript_api:getStatementLines}}\pysiglinewithargsret{\sphinxbfcode{\sphinxupquote{function }}\sphinxbfcode{\sphinxupquote{getStatementLines}}}{}{{ $\rightarrow$ Object}}
Return the lines that needs to be displayed by the widget
\begin{quote}\begin{description}
\item[{Returns}] \leavevmode
lines that are loaded and not yet displayed

\item[{Return Type}] \leavevmode
\sphinxstyleliteralemphasis{\sphinxupquote{Object}}

\end{description}\end{quote}

\end{fulllineitems}



\begin{fulllineitems}
\phantomsection\label{\detokenize{reference/javascript_api:hasMoreLines}}\pysiglinewithargsret{\sphinxbfcode{\sphinxupquote{function }}\sphinxbfcode{\sphinxupquote{hasMoreLines}}}{}{{ $\rightarrow$ boolean}}
Return a boolean telling if load button needs to be displayed or not
\begin{quote}\begin{description}
\item[{Returns}] \leavevmode
true if load more button needs to be displayed

\item[{Return Type}] \leavevmode
\sphinxstyleliteralemphasis{\sphinxupquote{boolean}}

\end{description}\end{quote}

\end{fulllineitems}



\begin{fulllineitems}
\phantomsection\label{\detokenize{reference/javascript_api:getLine}}\pysiglinewithargsret{\sphinxbfcode{\sphinxupquote{function }}\sphinxbfcode{\sphinxupquote{getLine}}}{\emph{handle}}{{ $\rightarrow$ Object}}
get the line data for this handle
\begin{quote}\begin{description}
\item[{Parameters}] \leavevmode\begin{itemize}

\sphinxstylestrong{handle} (\sphinxstyleliteralemphasis{\sphinxupquote{Object}})
\end{itemize}

\item[{Return Type}] \leavevmode
\sphinxstyleliteralemphasis{\sphinxupquote{Object}}

\end{description}\end{quote}

\end{fulllineitems}



\begin{fulllineitems}
\phantomsection\label{\detokenize{reference/javascript_api:load}}\pysiglinewithargsret{\sphinxbfcode{\sphinxupquote{function }}\sphinxbfcode{\sphinxupquote{load}}}{\emph{context}}{{ $\rightarrow$ Deferred}}
load data from
\begin{itemize}
\item {} 
‘account.bank.statement’ fetch the line id and bank\_statement\_id info

\item {} 
‘account.reconcile.model’  fetch all reconcile model (for quick add)

\item {} 
‘account.account’ fetch all account code

\item {} 
‘account.bank.statement.line’ fetch each line data

\end{itemize}
\begin{quote}\begin{description}
\item[{Parameters}] \leavevmode\begin{itemize}

\sphinxstylestrong{context} ({\hyperref[\detokenize{reference/javascript_api:account.ReconciliationModel.LoadContext}]{\sphinxcrossref{\sphinxstyleliteralemphasis{\sphinxupquote{LoadContext}}}}})
\end{itemize}

\item[{Return Type}] \leavevmode
\sphinxstyleliteralemphasis{\sphinxupquote{Deferred}}

\end{description}\end{quote}


\begin{fulllineitems}
\phantomsection\label{\detokenize{reference/javascript_api:LoadContext}}\pysiglinewithargsret{\sphinxbfcode{\sphinxupquote{class }}\sphinxbfcode{\sphinxupquote{LoadContext}}}{}{}~

\begin{fulllineitems}
\phantomsection\label{\detokenize{reference/javascript_api:statement_ids}}\pysigline{\sphinxbfcode{\sphinxupquote{attribute }}\sphinxbfcode{\sphinxupquote{statement\_ids}} number{[}{]}}
\end{fulllineitems}


\end{fulllineitems}


\end{fulllineitems}



\begin{fulllineitems}
\phantomsection\label{\detokenize{reference/javascript_api:loadMore}}\pysiglinewithargsret{\sphinxbfcode{\sphinxupquote{function }}\sphinxbfcode{\sphinxupquote{loadMore}}}{\emph{qty}}{{ $\rightarrow$ Deferred}}
Load more bank statement line
\begin{quote}\begin{description}
\item[{Parameters}] \leavevmode\begin{itemize}

\sphinxstylestrong{qty} (\sphinxstyleliteralemphasis{\sphinxupquote{integer}}) \textendash{} quantity to load
\end{itemize}

\item[{Return Type}] \leavevmode
\sphinxstyleliteralemphasis{\sphinxupquote{Deferred}}

\end{description}\end{quote}

\end{fulllineitems}



\begin{fulllineitems}
\phantomsection\label{\detokenize{reference/javascript_api:loadData}}\pysiglinewithargsret{\sphinxbfcode{\sphinxupquote{function }}\sphinxbfcode{\sphinxupquote{loadData}}}{\emph{ids}, \emph{excluded\_ids}}{{ $\rightarrow$ Deferred}}
RPC method to load informations on lines
\begin{quote}\begin{description}
\item[{Parameters}] \leavevmode\begin{itemize}

\sphinxstylestrong{ids} (\sphinxstyleliteralemphasis{\sphinxupquote{Array}}) \textendash{} ids of bank statement line passed to rpc call

\sphinxstylestrong{excluded\_ids} (\sphinxstyleliteralemphasis{\sphinxupquote{Array}}) \textendash{} list of move\_line ids that needs to be excluded from search
\end{itemize}

\item[{Return Type}] \leavevmode
\sphinxstyleliteralemphasis{\sphinxupquote{Deferred}}

\end{description}\end{quote}

\end{fulllineitems}



\begin{fulllineitems}
\phantomsection\label{\detokenize{reference/javascript_api:quickCreateProposition}}\pysiglinewithargsret{\sphinxbfcode{\sphinxupquote{function }}\sphinxbfcode{\sphinxupquote{quickCreateProposition}}}{\emph{handle}, \emph{reconcileModelId}}{{ $\rightarrow$ Deferred}}
Add lines into the propositions from the reconcile model
Can add 2 lines, and each with its taxes. The second line become editable
in the create mode.
\begin{quote}\begin{description}
\item[{Parameters}] \leavevmode\begin{itemize}

\sphinxstylestrong{handle} (\sphinxstyleliteralemphasis{\sphinxupquote{string}})

\sphinxstylestrong{reconcileModelId} (\sphinxstyleliteralemphasis{\sphinxupquote{integer}})
\end{itemize}

\item[{Return Type}] \leavevmode
\sphinxstyleliteralemphasis{\sphinxupquote{Deferred}}

\end{description}\end{quote}

\end{fulllineitems}



\begin{fulllineitems}
\phantomsection\label{\detokenize{reference/javascript_api:removeProposition}}\pysiglinewithargsret{\sphinxbfcode{\sphinxupquote{function }}\sphinxbfcode{\sphinxupquote{removeProposition}}}{\emph{handle}, \emph{id}}{{ $\rightarrow$ Deferred}}
Remove a proposition and switch to an active mode (‘create’ or ‘match’)
\begin{quote}\begin{description}
\item[{Parameters}] \leavevmode\begin{itemize}

\sphinxstylestrong{handle} (\sphinxstyleliteralemphasis{\sphinxupquote{string}})

\sphinxstylestrong{id} (\sphinxstyleliteralemphasis{\sphinxupquote{number}}) \textendash{} (move line id)
\end{itemize}

\item[{Return Type}] \leavevmode
\sphinxstyleliteralemphasis{\sphinxupquote{Deferred}}

\end{description}\end{quote}

\end{fulllineitems}



\begin{fulllineitems}
\phantomsection\label{\detokenize{reference/javascript_api:togglePartialReconcile}}\pysiglinewithargsret{\sphinxbfcode{\sphinxupquote{function }}\sphinxbfcode{\sphinxupquote{togglePartialReconcile}}}{\emph{handle}}{{ $\rightarrow$ Deferred}}
Force the partial reconciliation to display the reconciliate button.
This method should only  be called when there is onely one proposition.
\begin{quote}\begin{description}
\item[{Parameters}] \leavevmode\begin{itemize}

\sphinxstylestrong{handle} (\sphinxstyleliteralemphasis{\sphinxupquote{string}})
\end{itemize}

\item[{Return Type}] \leavevmode
\sphinxstyleliteralemphasis{\sphinxupquote{Deferred}}

\end{description}\end{quote}

\end{fulllineitems}



\begin{fulllineitems}
\phantomsection\label{\detokenize{reference/javascript_api:updateProposition}}\pysiglinewithargsret{\sphinxbfcode{\sphinxupquote{function }}\sphinxbfcode{\sphinxupquote{updateProposition}}}{\emph{handle}, \emph{values}}{{ $\rightarrow$ Deferred}}
Change the value of the editable proposition line or create a new one.

If the editable line comes from a reconcile model with 2 lines
and their ‘amount\_type’ is “percent” 
and their total equals 100\% (this doesn’t take into account the taxes
who can be included or not)
Then the total is recomputed to have 100\%.
\begin{quote}\begin{description}
\item[{Parameters}] \leavevmode\begin{itemize}

\sphinxstylestrong{handle} (\sphinxstyleliteralemphasis{\sphinxupquote{string}})

\sphinxstylestrong{values} (\sphinxstyleliteralemphasis{\sphinxupquote{any}})
\end{itemize}

\item[{Return Type}] \leavevmode
\sphinxstyleliteralemphasis{\sphinxupquote{Deferred}}

\end{description}\end{quote}

\end{fulllineitems}



\begin{fulllineitems}
\phantomsection\label{\detokenize{reference/javascript_api:validate}}\pysiglinewithargsret{\sphinxbfcode{\sphinxupquote{function }}\sphinxbfcode{\sphinxupquote{validate}}}{\emph{handle}}{{ $\rightarrow$ Deferred\textless{}Object\textgreater{}}}
Format the value and send it to ‘account.bank.statement.line’ model
Update the number of validated lines
\begin{quote}\begin{description}
\item[{Parameters}] \leavevmode\begin{itemize}

\sphinxstylestrong{handle} (\sphinxstyleliteralemphasis{\sphinxupquote{string}}\sphinxstyleemphasis{ or }\sphinxstyleliteralemphasis{\sphinxupquote{Array}}\textless{}\sphinxstyleliteralemphasis{\sphinxupquote{string}}\textgreater{})
\end{itemize}

\item[{Returns}] \leavevmode
resolved with an object who contains
  ‘handles’ key

\item[{Return Type}] \leavevmode
\sphinxstyleliteralemphasis{\sphinxupquote{Deferred}}\textless{}\sphinxstyleliteralemphasis{\sphinxupquote{Object}}\textgreater{}

\end{description}\end{quote}

\end{fulllineitems}


\end{fulllineitems}



\begin{fulllineitems}
\phantomsection\label{\detokenize{reference/javascript_api:ManualModel}}\pysiglinewithargsret{\sphinxbfcode{\sphinxupquote{class }}\sphinxbfcode{\sphinxupquote{ManualModel}}}{}{}~\begin{quote}\begin{description}
\item[{Extends}] \leavevmode{\hyperref[\detokenize{reference/javascript_api:account.ReconciliationModel.StatementModel}]{\sphinxcrossref{
StatementModel
}}}
\end{description}\end{quote}

Model use to fetch, format and update ‘account.move.line’ and ‘res.partner’
datas allowing manual reconciliation


\begin{fulllineitems}
\phantomsection\label{\detokenize{reference/javascript_api:load}}\pysiglinewithargsret{\sphinxbfcode{\sphinxupquote{method }}\sphinxbfcode{\sphinxupquote{load}}}{\emph{context}}{{ $\rightarrow$ Deferred}}
load data from
- ‘account.move.line’ fetch the lines to reconciliate
- ‘account.account’ fetch all account code
\begin{quote}\begin{description}
\item[{Parameters}] \leavevmode\begin{itemize}

\sphinxstylestrong{context} ({\hyperref[\detokenize{reference/javascript_api:account.ReconciliationModel.LoadContext}]{\sphinxcrossref{\sphinxstyleliteralemphasis{\sphinxupquote{LoadContext}}}}})
\end{itemize}

\item[{Return Type}] \leavevmode
\sphinxstyleliteralemphasis{\sphinxupquote{Deferred}}

\end{description}\end{quote}


\begin{fulllineitems}
\phantomsection\label{\detokenize{reference/javascript_api:LoadContext}}\pysiglinewithargsret{\sphinxbfcode{\sphinxupquote{class }}\sphinxbfcode{\sphinxupquote{LoadContext}}}{}{}~

\begin{fulllineitems}
\phantomsection\label{\detokenize{reference/javascript_api:mode}}\pysigline{\sphinxbfcode{\sphinxupquote{attribute }}\sphinxbfcode{\sphinxupquote{mode}} string}
‘customers’, ‘suppliers’ or ‘accounts’

\end{fulllineitems}



\begin{fulllineitems}
\phantomsection\label{\detokenize{reference/javascript_api:company_ids}}\pysigline{\sphinxbfcode{\sphinxupquote{attribute }}\sphinxbfcode{\sphinxupquote{company\_ids}} integer{[}{]}}
\end{fulllineitems}



\begin{fulllineitems}
\phantomsection\label{\detokenize{reference/javascript_api:partner_ids}}\pysigline{\sphinxbfcode{\sphinxupquote{attribute }}\sphinxbfcode{\sphinxupquote{partner\_ids}} integer{[}{]}}~\begin{description}
\item[{used for ‘customers’ and}] \leavevmode
‘suppliers’ mode

\end{description}

\end{fulllineitems}


\end{fulllineitems}


\end{fulllineitems}



\begin{fulllineitems}
\phantomsection\label{\detokenize{reference/javascript_api:validate}}\pysiglinewithargsret{\sphinxbfcode{\sphinxupquote{function }}\sphinxbfcode{\sphinxupquote{validate}}}{\emph{handle}}{{ $\rightarrow$ Deferred\textless{}Array\textgreater{}}}
Mark the account or the partner as reconciled
\begin{quote}\begin{description}
\item[{Parameters}] \leavevmode\begin{itemize}

\sphinxstylestrong{handle} (\sphinxstyleliteralemphasis{\sphinxupquote{string}}\sphinxstyleemphasis{ or }\sphinxstyleliteralemphasis{\sphinxupquote{Array}}\textless{}\sphinxstyleliteralemphasis{\sphinxupquote{string}}\textgreater{})
\end{itemize}

\item[{Returns}] \leavevmode
resolved with the handle array

\item[{Return Type}] \leavevmode
\sphinxstyleliteralemphasis{\sphinxupquote{Deferred}}\textless{}\sphinxstyleliteralemphasis{\sphinxupquote{Array}}\textgreater{}

\end{description}\end{quote}

\end{fulllineitems}


\end{fulllineitems}


\end{fulllineitems}


\end{fulllineitems}

\phantomsection\label{\detokenize{reference/javascript_api:module-website_links.website_links}}

\begin{fulllineitems}
\phantomsection\label{\detokenize{reference/javascript_api:website_links.website_links}}\pysigline{\sphinxbfcode{\sphinxupquote{module }}\sphinxbfcode{\sphinxupquote{website\_links.website\_links}}}~~\begin{quote}\begin{description}
\item[{Exports}] \leavevmode{\hyperref[\detokenize{reference/javascript_api:website_links.website_links.exports}]{\sphinxcrossref{
exports
}}}
\item[{Depends On}] \leavevmode\begin{itemize}
\item {} {\hyperref[\detokenize{reference/javascript_api:web.Widget}]{\sphinxcrossref{
web.Widget
}}}
\item {} {\hyperref[\detokenize{reference/javascript_api:web.ajax}]{\sphinxcrossref{
web.ajax
}}}
\item {} {\hyperref[\detokenize{reference/javascript_api:web.core}]{\sphinxcrossref{
web.core
}}}
\item {} {\hyperref[\detokenize{reference/javascript_api:web.rpc}]{\sphinxcrossref{
web.rpc
}}}
\item {} {\hyperref[\detokenize{reference/javascript_api:web_editor.base}]{\sphinxcrossref{
web\_editor.base
}}}
\end{itemize}

\end{description}\end{quote}


\begin{fulllineitems}
\phantomsection\label{\detokenize{reference/javascript_api:exports}}\pysigline{\sphinxbfcode{\sphinxupquote{namespace }}\sphinxbfcode{\sphinxupquote{exports}}}
\end{fulllineitems}


\end{fulllineitems}

\phantomsection\label{\detokenize{reference/javascript_api:module-web.GraphView}}

\begin{fulllineitems}
\phantomsection\label{\detokenize{reference/javascript_api:web.GraphView}}\pysigline{\sphinxbfcode{\sphinxupquote{module }}\sphinxbfcode{\sphinxupquote{web.GraphView}}}~~\begin{quote}\begin{description}
\item[{Exports}] \leavevmode{\hyperref[\detokenize{reference/javascript_api:web.GraphView.GraphView}]{\sphinxcrossref{
GraphView
}}}
\item[{Depends On}] \leavevmode\begin{itemize}
\item {} {\hyperref[\detokenize{reference/javascript_api:web.AbstractView}]{\sphinxcrossref{
web.AbstractView
}}}
\item {} {\hyperref[\detokenize{reference/javascript_api:web.GraphController}]{\sphinxcrossref{
web.GraphController
}}}
\item {} {\hyperref[\detokenize{reference/javascript_api:web.GraphModel}]{\sphinxcrossref{
web.GraphModel
}}}
\item {} {\hyperref[\detokenize{reference/javascript_api:web.GraphRenderer}]{\sphinxcrossref{
web.GraphRenderer
}}}
\item {} {\hyperref[\detokenize{reference/javascript_api:web.core}]{\sphinxcrossref{
web.core
}}}
\end{itemize}

\end{description}\end{quote}


\begin{fulllineitems}
\phantomsection\label{\detokenize{reference/javascript_api:GraphView}}\pysiglinewithargsret{\sphinxbfcode{\sphinxupquote{class }}\sphinxbfcode{\sphinxupquote{GraphView}}}{\emph{viewInfo}}{}~\begin{quote}\begin{description}
\item[{Extends}] \leavevmode{\hyperref[\detokenize{reference/javascript_api:web.AbstractView.AbstractView}]{\sphinxcrossref{
AbstractView
}}}
\item[{Parameters}] \leavevmode\begin{itemize}

\sphinxstylestrong{viewInfo}
\end{itemize}

\end{description}\end{quote}

\end{fulllineitems}


\end{fulllineitems}

\phantomsection\label{\detokenize{reference/javascript_api:module-mail.chat_client_action}}

\begin{fulllineitems}
\phantomsection\label{\detokenize{reference/javascript_api:mail.chat_client_action}}\pysigline{\sphinxbfcode{\sphinxupquote{module }}\sphinxbfcode{\sphinxupquote{mail.chat\_client\_action}}}~~\begin{quote}\begin{description}
\item[{Exports}] \leavevmode{\hyperref[\detokenize{reference/javascript_api:mail.chat_client_action.ChatAction}]{\sphinxcrossref{
ChatAction
}}}
\item[{Depends On}] \leavevmode\begin{itemize}
\item {} {\hyperref[\detokenize{reference/javascript_api:mail.ChatThread}]{\sphinxcrossref{
mail.ChatThread
}}}
\item {} {\hyperref[\detokenize{reference/javascript_api:mail.chat_manager}]{\sphinxcrossref{
mail.chat\_manager
}}}
\item {} {\hyperref[\detokenize{reference/javascript_api:mail.composer}]{\sphinxcrossref{
mail.composer
}}}
\item {} {\hyperref[\detokenize{reference/javascript_api:mail.utils}]{\sphinxcrossref{
mail.utils
}}}
\item {} {\hyperref[\detokenize{reference/javascript_api:web.ControlPanelMixin}]{\sphinxcrossref{
web.ControlPanelMixin
}}}
\item {} {\hyperref[\detokenize{reference/javascript_api:web.Dialog}]{\sphinxcrossref{
web.Dialog
}}}
\item {} {\hyperref[\detokenize{reference/javascript_api:web.SearchView}]{\sphinxcrossref{
web.SearchView
}}}
\item {} {\hyperref[\detokenize{reference/javascript_api:web.Widget}]{\sphinxcrossref{
web.Widget
}}}
\item {} {\hyperref[\detokenize{reference/javascript_api:web.config}]{\sphinxcrossref{
web.config
}}}
\item {} {\hyperref[\detokenize{reference/javascript_api:web.core}]{\sphinxcrossref{
web.core
}}}
\item {} {\hyperref[\detokenize{reference/javascript_api:web.data}]{\sphinxcrossref{
web.data
}}}
\item {} {\hyperref[\detokenize{reference/javascript_api:web.dom}]{\sphinxcrossref{
web.dom
}}}
\item {} {\hyperref[\detokenize{reference/javascript_api:web.pyeval}]{\sphinxcrossref{
web.pyeval
}}}
\item {} {\hyperref[\detokenize{reference/javascript_api:web.session}]{\sphinxcrossref{
web.session
}}}
\end{itemize}

\end{description}\end{quote}


\begin{fulllineitems}
\phantomsection\label{\detokenize{reference/javascript_api:ChatAction}}\pysiglinewithargsret{\sphinxbfcode{\sphinxupquote{class }}\sphinxbfcode{\sphinxupquote{ChatAction}}}{}{}~\begin{quote}\begin{description}
\item[{Extends}] \leavevmode{\hyperref[\detokenize{reference/javascript_api:web.Widget.Widget}]{\sphinxcrossref{
Widget
}}}
\item[{Mixes}] \leavevmode\begin{itemize}
\item {} {\hyperref[\detokenize{reference/javascript_api:web.ControlPanelMixin.ControlPanelMixin}]{\sphinxcrossref{
ControlPanelMixin
}}}
\end{itemize}

\end{description}\end{quote}

\end{fulllineitems}



\begin{fulllineitems}
\phantomsection\label{\detokenize{reference/javascript_api:PartnerInviteDialog}}\pysiglinewithargsret{\sphinxbfcode{\sphinxupquote{class }}\sphinxbfcode{\sphinxupquote{PartnerInviteDialog}}}{\emph{parent}, \emph{title}, \emph{channel\_id}}{}~\begin{quote}\begin{description}
\item[{Extends}] \leavevmode{\hyperref[\detokenize{reference/javascript_api:web.Dialog.Dialog}]{\sphinxcrossref{
Dialog
}}}
\item[{Parameters}] \leavevmode\begin{itemize}

\sphinxstylestrong{parent}

\sphinxstylestrong{title}

\sphinxstylestrong{channel\_id}
\end{itemize}

\end{description}\end{quote}

Widget : Invite People to Channel Dialog

Popup containing a ‘many2many\_tags’ custom input to select multiple partners.
Searches user according to the input, and triggers event when selection is
validated.

\end{fulllineitems}


\end{fulllineitems}

\phantomsection\label{\detokenize{reference/javascript_api:module-website_quote.website_quote}}

\begin{fulllineitems}
\phantomsection\label{\detokenize{reference/javascript_api:website_quote.website_quote}}\pysigline{\sphinxbfcode{\sphinxupquote{module }}\sphinxbfcode{\sphinxupquote{website\_quote.website\_quote}}}~~\begin{quote}\begin{description}
\item[{Exports}] \leavevmode{\hyperref[\detokenize{reference/javascript_api:website_quote.website_quote.}]{\sphinxcrossref{
\textless{}anonymous\textgreater{}
}}}
\item[{Depends On}] \leavevmode\begin{itemize}
\item {} {\hyperref[\detokenize{reference/javascript_api:web.Widget}]{\sphinxcrossref{
web.Widget
}}}
\item {} {\hyperref[\detokenize{reference/javascript_api:web.ajax}]{\sphinxcrossref{
web.ajax
}}}
\item {} {\hyperref[\detokenize{reference/javascript_api:web.config}]{\sphinxcrossref{
web.config
}}}
\end{itemize}

\end{description}\end{quote}


\begin{fulllineitems}
\phantomsection\label{\detokenize{reference/javascript_api:website_quote.website_quote.}}\pysigline{\sphinxbfcode{\sphinxupquote{namespace }}\sphinxbfcode{\sphinxupquote{}}}
\end{fulllineitems}


\end{fulllineitems}

\phantomsection\label{\detokenize{reference/javascript_api:module-mail.Activity}}

\begin{fulllineitems}
\phantomsection\label{\detokenize{reference/javascript_api:mail.Activity}}\pysigline{\sphinxbfcode{\sphinxupquote{module }}\sphinxbfcode{\sphinxupquote{mail.Activity}}}~~\begin{quote}\begin{description}
\item[{Exports}] \leavevmode{\hyperref[\detokenize{reference/javascript_api:mail.Activity.Activity}]{\sphinxcrossref{
Activity
}}}
\item[{Depends On}] \leavevmode\begin{itemize}
\item {} {\hyperref[\detokenize{reference/javascript_api:mail.utils}]{\sphinxcrossref{
mail.utils
}}}
\item {} {\hyperref[\detokenize{reference/javascript_api:web.AbstractField}]{\sphinxcrossref{
web.AbstractField
}}}
\item {} {\hyperref[\detokenize{reference/javascript_api:web.BasicModel}]{\sphinxcrossref{
web.BasicModel
}}}
\item {} {\hyperref[\detokenize{reference/javascript_api:web.core}]{\sphinxcrossref{
web.core
}}}
\item {} {\hyperref[\detokenize{reference/javascript_api:web.field_registry}]{\sphinxcrossref{
web.field\_registry
}}}
\item {} {\hyperref[\detokenize{reference/javascript_api:web.time}]{\sphinxcrossref{
web.time
}}}
\end{itemize}

\end{description}\end{quote}


\begin{fulllineitems}
\phantomsection\label{\detokenize{reference/javascript_api:setDelayLabel}}\pysiglinewithargsret{\sphinxbfcode{\sphinxupquote{function }}\sphinxbfcode{\sphinxupquote{setDelayLabel}}}{\emph{activities}}{{ $\rightarrow$ Array}}
Set the ‘label\_delay’ entry in activity data according to the deadline date
\begin{quote}\begin{description}
\item[{Parameters}] \leavevmode\begin{itemize}

\sphinxstylestrong{activities} (\sphinxstyleliteralemphasis{\sphinxupquote{Array}}) \textendash{} list of activity Object
\end{itemize}

\item[{Returns}] \leavevmode
list of modified activity Object

\item[{Return Type}] \leavevmode
\sphinxstyleliteralemphasis{\sphinxupquote{Array}}

\end{description}\end{quote}

\end{fulllineitems}



\begin{fulllineitems}
\phantomsection\label{\detokenize{reference/javascript_api:_readActivities}}\pysiglinewithargsret{\sphinxbfcode{\sphinxupquote{function }}\sphinxbfcode{\sphinxupquote{\_readActivities}}}{\emph{self}, \emph{ids}}{{ $\rightarrow$ Deferred\textless{}Array\textgreater{}}}
Fetches activities and postprocesses them.

This standalone function performs an RPC, but to do so, it needs an instance
of a widget that implements the \_rpc() function.
\begin{quote}\begin{description}
\item[{Parameters}] \leavevmode\begin{itemize}

\sphinxstylestrong{self} ({\hyperref[\detokenize{reference/javascript_api:Widget}]{\sphinxcrossref{\sphinxstyleliteralemphasis{\sphinxupquote{Widget}}}}}) \textendash{} a widget instance that can perform RPCs

\sphinxstylestrong{ids} (\sphinxstyleliteralemphasis{\sphinxupquote{Array}}) \textendash{} the ids of activities to read
\end{itemize}

\item[{Returns}] \leavevmode
resolved with the activities

\item[{Return Type}] \leavevmode
\sphinxstyleliteralemphasis{\sphinxupquote{Deferred}}\textless{}\sphinxstyleliteralemphasis{\sphinxupquote{Array}}\textgreater{}

\end{description}\end{quote}

\end{fulllineitems}



\begin{fulllineitems}
\phantomsection\label{\detokenize{reference/javascript_api:Activity}}\pysiglinewithargsret{\sphinxbfcode{\sphinxupquote{class }}\sphinxbfcode{\sphinxupquote{Activity}}}{}{}~\begin{quote}\begin{description}
\item[{Extends}] \leavevmode
AbstractActivityField

\end{description}\end{quote}

\end{fulllineitems}


\end{fulllineitems}

\phantomsection\label{\detokenize{reference/javascript_api:module-web_kanban_gauge.widget}}

\begin{fulllineitems}
\phantomsection\label{\detokenize{reference/javascript_api:web_kanban_gauge.widget}}\pysigline{\sphinxbfcode{\sphinxupquote{module }}\sphinxbfcode{\sphinxupquote{web\_kanban\_gauge.widget}}}~~\begin{quote}\begin{description}
\item[{Exports}] \leavevmode{\hyperref[\detokenize{reference/javascript_api:web_kanban_gauge.widget.GaugeWidget}]{\sphinxcrossref{
GaugeWidget
}}}
\item[{Depends On}] \leavevmode\begin{itemize}
\item {} {\hyperref[\detokenize{reference/javascript_api:web.AbstractField}]{\sphinxcrossref{
web.AbstractField
}}}
\item {} {\hyperref[\detokenize{reference/javascript_api:web.field_registry}]{\sphinxcrossref{
web.field\_registry
}}}
\item {} {\hyperref[\detokenize{reference/javascript_api:web.utils}]{\sphinxcrossref{
web.utils
}}}
\end{itemize}

\end{description}\end{quote}


\begin{fulllineitems}
\phantomsection\label{\detokenize{reference/javascript_api:GaugeWidget}}\pysiglinewithargsret{\sphinxbfcode{\sphinxupquote{class }}\sphinxbfcode{\sphinxupquote{GaugeWidget}}}{}{}~\begin{quote}\begin{description}
\item[{Extends}] \leavevmode{\hyperref[\detokenize{reference/javascript_api:web.AbstractField.AbstractField}]{\sphinxcrossref{
AbstractField
}}}
\end{description}\end{quote}

options
\begin{itemize}
\item {} 
max\_value: maximum value of the gauge {[}default: 100{]}

\item {} 
max\_field: get the max\_value from the field that must be present in the
view; takes over max\_value

\item {} 
gauge\_value\_field: if set, the value displayed below the gauge is taken
from this field instead of the base field used for
the gauge. This allows to display a number different
from the gauge.

\item {} 
label: lable of the gauge, displayed below the gauge value

\item {} 
label\_field: get the label from the field that must be present in the
view; takes over label

\item {} 
title: title of the gauge, displayed on top of the gauge

\item {} 
style: custom style

\end{itemize}

\end{fulllineitems}



\begin{fulllineitems}
\phantomsection\label{\detokenize{reference/javascript_api:GaugeWidget}}\pysiglinewithargsret{\sphinxbfcode{\sphinxupquote{class }}\sphinxbfcode{\sphinxupquote{GaugeWidget}}}{}{}~\begin{quote}\begin{description}
\item[{Extends}] \leavevmode{\hyperref[\detokenize{reference/javascript_api:web.AbstractField.AbstractField}]{\sphinxcrossref{
AbstractField
}}}
\end{description}\end{quote}

options
\begin{itemize}
\item {} 
max\_value: maximum value of the gauge {[}default: 100{]}

\item {} 
max\_field: get the max\_value from the field that must be present in the
view; takes over max\_value

\item {} 
gauge\_value\_field: if set, the value displayed below the gauge is taken
from this field instead of the base field used for
the gauge. This allows to display a number different
from the gauge.

\item {} 
label: lable of the gauge, displayed below the gauge value

\item {} 
label\_field: get the label from the field that must be present in the
view; takes over label

\item {} 
title: title of the gauge, displayed on top of the gauge

\item {} 
style: custom style

\end{itemize}

\end{fulllineitems}


\end{fulllineitems}

\phantomsection\label{\detokenize{reference/javascript_api:module-web.UserMenu}}

\begin{fulllineitems}
\phantomsection\label{\detokenize{reference/javascript_api:web.UserMenu}}\pysigline{\sphinxbfcode{\sphinxupquote{module }}\sphinxbfcode{\sphinxupquote{web.UserMenu}}}~~\begin{quote}\begin{description}
\item[{Exports}] \leavevmode{\hyperref[\detokenize{reference/javascript_api:web.UserMenu.UserMenu}]{\sphinxcrossref{
UserMenu
}}}
\item[{Depends On}] \leavevmode\begin{itemize}
\item {} {\hyperref[\detokenize{reference/javascript_api:web.Widget}]{\sphinxcrossref{
web.Widget
}}}
\item {} {\hyperref[\detokenize{reference/javascript_api:web.framework}]{\sphinxcrossref{
web.framework
}}}
\end{itemize}

\end{description}\end{quote}


\begin{fulllineitems}
\phantomsection\label{\detokenize{reference/javascript_api:UserMenu}}\pysiglinewithargsret{\sphinxbfcode{\sphinxupquote{class }}\sphinxbfcode{\sphinxupquote{UserMenu}}}{}{}~\begin{quote}\begin{description}
\item[{Extends}] \leavevmode{\hyperref[\detokenize{reference/javascript_api:web.Widget.Widget}]{\sphinxcrossref{
Widget
}}}
\end{description}\end{quote}

\end{fulllineitems}


\end{fulllineitems}

\phantomsection\label{\detokenize{reference/javascript_api:module-im_livechat.im_livechat}}

\begin{fulllineitems}
\phantomsection\label{\detokenize{reference/javascript_api:im_livechat.im_livechat}}\pysigline{\sphinxbfcode{\sphinxupquote{module }}\sphinxbfcode{\sphinxupquote{im\_livechat.im\_livechat}}}~~\begin{quote}\begin{description}
\item[{Exports}] \leavevmode{\hyperref[\detokenize{reference/javascript_api:im_livechat.im_livechat.}]{\sphinxcrossref{
\textless{}anonymous\textgreater{}
}}}
\item[{Depends On}] \leavevmode\begin{itemize}
\item {} {\hyperref[\detokenize{reference/javascript_api:bus.bus}]{\sphinxcrossref{
bus.bus
}}}
\item {} {\hyperref[\detokenize{reference/javascript_api:mail.ChatWindow}]{\sphinxcrossref{
mail.ChatWindow
}}}
\item {} {\hyperref[\detokenize{reference/javascript_api:web.Widget}]{\sphinxcrossref{
web.Widget
}}}
\item {} {\hyperref[\detokenize{reference/javascript_api:web.config}]{\sphinxcrossref{
web.config
}}}
\item {} {\hyperref[\detokenize{reference/javascript_api:web.core}]{\sphinxcrossref{
web.core
}}}
\item {} {\hyperref[\detokenize{reference/javascript_api:web.local_storage}]{\sphinxcrossref{
web.local\_storage
}}}
\item {} {\hyperref[\detokenize{reference/javascript_api:web.session}]{\sphinxcrossref{
web.session
}}}
\item {} {\hyperref[\detokenize{reference/javascript_api:web.time}]{\sphinxcrossref{
web.time
}}}
\item {} {\hyperref[\detokenize{reference/javascript_api:web.utils}]{\sphinxcrossref{
web.utils
}}}
\end{itemize}

\end{description}\end{quote}


\begin{fulllineitems}
\phantomsection\label{\detokenize{reference/javascript_api:im_livechat.im_livechat.}}\pysigline{\sphinxbfcode{\sphinxupquote{namespace }}\sphinxbfcode{\sphinxupquote{}}}
\end{fulllineitems}


\end{fulllineitems}

\phantomsection\label{\detokenize{reference/javascript_api:module-web.CalendarModel}}

\begin{fulllineitems}
\phantomsection\label{\detokenize{reference/javascript_api:web.CalendarModel}}\pysigline{\sphinxbfcode{\sphinxupquote{module }}\sphinxbfcode{\sphinxupquote{web.CalendarModel}}}~~\begin{quote}\begin{description}
\item[{Exports}] \leavevmode{\hyperref[\detokenize{reference/javascript_api:web.CalendarModel.}]{\sphinxcrossref{
\textless{}anonymous\textgreater{}
}}}
\item[{Depends On}] \leavevmode\begin{itemize}
\item {} {\hyperref[\detokenize{reference/javascript_api:web.AbstractModel}]{\sphinxcrossref{
web.AbstractModel
}}}
\item {} {\hyperref[\detokenize{reference/javascript_api:web.Context}]{\sphinxcrossref{
web.Context
}}}
\item {} {\hyperref[\detokenize{reference/javascript_api:web.core}]{\sphinxcrossref{
web.core
}}}
\item {} {\hyperref[\detokenize{reference/javascript_api:web.field_utils}]{\sphinxcrossref{
web.field\_utils
}}}
\item {} {\hyperref[\detokenize{reference/javascript_api:web.session}]{\sphinxcrossref{
web.session
}}}
\item {} {\hyperref[\detokenize{reference/javascript_api:web.time}]{\sphinxcrossref{
web.time
}}}
\end{itemize}

\end{description}\end{quote}


\begin{fulllineitems}
\phantomsection\label{\detokenize{reference/javascript_api:web.CalendarModel.}}\pysiglinewithargsret{\sphinxbfcode{\sphinxupquote{class }}\sphinxbfcode{\sphinxupquote{}}}{}{}~\begin{quote}\begin{description}
\item[{Extends}] \leavevmode{\hyperref[\detokenize{reference/javascript_api:web.AbstractModel.AbstractModel}]{\sphinxcrossref{
AbstractModel
}}}
\end{description}\end{quote}


\begin{fulllineitems}
\phantomsection\label{\detokenize{reference/javascript_api:calendarEventToRecord}}\pysiglinewithargsret{\sphinxbfcode{\sphinxupquote{function }}\sphinxbfcode{\sphinxupquote{calendarEventToRecord}}}{\emph{event}}{}
Transform fullcalendar event object to OpenERP Data object
\begin{quote}\begin{description}
\item[{Parameters}] \leavevmode\begin{itemize}

\sphinxstylestrong{event}
\end{itemize}

\end{description}\end{quote}

\end{fulllineitems}


\end{fulllineitems}


\end{fulllineitems}

\phantomsection\label{\detokenize{reference/javascript_api:module-website.seo}}

\begin{fulllineitems}
\phantomsection\label{\detokenize{reference/javascript_api:website.seo}}\pysigline{\sphinxbfcode{\sphinxupquote{module }}\sphinxbfcode{\sphinxupquote{website.seo}}}~~\begin{quote}\begin{description}
\item[{Exports}] \leavevmode{\hyperref[\detokenize{reference/javascript_api:website.seo.}]{\sphinxcrossref{
\textless{}anonymous\textgreater{}
}}}
\item[{Depends On}] \leavevmode\begin{itemize}
\item {} {\hyperref[\detokenize{reference/javascript_api:web.Class}]{\sphinxcrossref{
web.Class
}}}
\item {} {\hyperref[\detokenize{reference/javascript_api:web.Dialog}]{\sphinxcrossref{
web.Dialog
}}}
\item {} {\hyperref[\detokenize{reference/javascript_api:web.Widget}]{\sphinxcrossref{
web.Widget
}}}
\item {} {\hyperref[\detokenize{reference/javascript_api:web.core}]{\sphinxcrossref{
web.core
}}}
\item {} {\hyperref[\detokenize{reference/javascript_api:web.mixins}]{\sphinxcrossref{
web.mixins
}}}
\item {} {\hyperref[\detokenize{reference/javascript_api:web.rpc}]{\sphinxcrossref{
web.rpc
}}}
\item {} {\hyperref[\detokenize{reference/javascript_api:web_editor.context}]{\sphinxcrossref{
web\_editor.context
}}}
\item {} {\hyperref[\detokenize{reference/javascript_api:website.navbar}]{\sphinxcrossref{
website.navbar
}}}
\end{itemize}

\end{description}\end{quote}


\begin{fulllineitems}
\phantomsection\label{\detokenize{reference/javascript_api:website.seo.}}\pysigline{\sphinxbfcode{\sphinxupquote{namespace }}\sphinxbfcode{\sphinxupquote{}}}
\end{fulllineitems}


\end{fulllineitems}

\phantomsection\label{\detokenize{reference/javascript_api:module-pos_restaurant.floors}}

\begin{fulllineitems}
\phantomsection\label{\detokenize{reference/javascript_api:pos_restaurant.floors}}\pysigline{\sphinxbfcode{\sphinxupquote{module }}\sphinxbfcode{\sphinxupquote{pos\_restaurant.floors}}}~~\begin{quote}\begin{description}
\item[{Exports}] \leavevmode{\hyperref[\detokenize{reference/javascript_api:pos_restaurant.floors.}]{\sphinxcrossref{
\textless{}anonymous\textgreater{}
}}}
\item[{Depends On}] \leavevmode\begin{itemize}
\item {} {\hyperref[\detokenize{reference/javascript_api:point_of_sale.BaseWidget}]{\sphinxcrossref{
point\_of\_sale.BaseWidget
}}}
\item {} {\hyperref[\detokenize{reference/javascript_api:point_of_sale.chrome}]{\sphinxcrossref{
point\_of\_sale.chrome
}}}
\item {} {\hyperref[\detokenize{reference/javascript_api:point_of_sale.gui}]{\sphinxcrossref{
point\_of\_sale.gui
}}}
\item {} {\hyperref[\detokenize{reference/javascript_api:point_of_sale.models}]{\sphinxcrossref{
point\_of\_sale.models
}}}
\item {} {\hyperref[\detokenize{reference/javascript_api:point_of_sale.screens}]{\sphinxcrossref{
point\_of\_sale.screens
}}}
\item {} {\hyperref[\detokenize{reference/javascript_api:web.core}]{\sphinxcrossref{
web.core
}}}
\item {} {\hyperref[\detokenize{reference/javascript_api:web.rpc}]{\sphinxcrossref{
web.rpc
}}}
\end{itemize}

\end{description}\end{quote}


\begin{fulllineitems}
\phantomsection\label{\detokenize{reference/javascript_api:pos_restaurant.floors.}}\pysigline{\sphinxbfcode{\sphinxupquote{namespace }}\sphinxbfcode{\sphinxupquote{}}}
\end{fulllineitems}


\end{fulllineitems}

\phantomsection\label{\detokenize{reference/javascript_api:module-web.KanbanColumn}}

\begin{fulllineitems}
\phantomsection\label{\detokenize{reference/javascript_api:web.KanbanColumn}}\pysigline{\sphinxbfcode{\sphinxupquote{module }}\sphinxbfcode{\sphinxupquote{web.KanbanColumn}}}~~\begin{quote}\begin{description}
\item[{Exports}] \leavevmode{\hyperref[\detokenize{reference/javascript_api:web.KanbanColumn.KanbanColumn}]{\sphinxcrossref{
KanbanColumn
}}}
\item[{Depends On}] \leavevmode\begin{itemize}
\item {} {\hyperref[\detokenize{reference/javascript_api:web.Dialog}]{\sphinxcrossref{
web.Dialog
}}}
\item {} {\hyperref[\detokenize{reference/javascript_api:web.KanbanColumnProgressBar}]{\sphinxcrossref{
web.KanbanColumnProgressBar
}}}
\item {} {\hyperref[\detokenize{reference/javascript_api:web.KanbanRecord}]{\sphinxcrossref{
web.KanbanRecord
}}}
\item {} {\hyperref[\detokenize{reference/javascript_api:web.Widget}]{\sphinxcrossref{
web.Widget
}}}
\item {} {\hyperref[\detokenize{reference/javascript_api:web.config}]{\sphinxcrossref{
web.config
}}}
\item {} {\hyperref[\detokenize{reference/javascript_api:web.core}]{\sphinxcrossref{
web.core
}}}
\item {} {\hyperref[\detokenize{reference/javascript_api:web.kanban_quick_create}]{\sphinxcrossref{
web.kanban\_quick\_create
}}}
\item {} {\hyperref[\detokenize{reference/javascript_api:web.view_dialogs}]{\sphinxcrossref{
web.view\_dialogs
}}}
\end{itemize}

\end{description}\end{quote}


\begin{fulllineitems}
\phantomsection\label{\detokenize{reference/javascript_api:KanbanColumn}}\pysiglinewithargsret{\sphinxbfcode{\sphinxupquote{class }}\sphinxbfcode{\sphinxupquote{KanbanColumn}}}{}{}~\begin{quote}\begin{description}
\item[{Extends}] \leavevmode{\hyperref[\detokenize{reference/javascript_api:web.Widget.Widget}]{\sphinxcrossref{
Widget
}}}
\end{description}\end{quote}


\begin{fulllineitems}
\phantomsection\label{\detokenize{reference/javascript_api:addQuickCreate}}\pysiglinewithargsret{\sphinxbfcode{\sphinxupquote{method }}\sphinxbfcode{\sphinxupquote{addQuickCreate}}}{}{}
Adds the quick create record to the top of the column.

\end{fulllineitems}


\end{fulllineitems}


\end{fulllineitems}

\phantomsection\label{\detokenize{reference/javascript_api:module-mail.ChatterComposer}}

\begin{fulllineitems}
\phantomsection\label{\detokenize{reference/javascript_api:mail.ChatterComposer}}\pysigline{\sphinxbfcode{\sphinxupquote{module }}\sphinxbfcode{\sphinxupquote{mail.ChatterComposer}}}~~\begin{quote}\begin{description}
\item[{Exports}] \leavevmode{\hyperref[\detokenize{reference/javascript_api:mail.ChatterComposer.ChatterComposer}]{\sphinxcrossref{
ChatterComposer
}}}
\item[{Depends On}] \leavevmode\begin{itemize}
\item {} {\hyperref[\detokenize{reference/javascript_api:mail.composer}]{\sphinxcrossref{
mail.composer
}}}
\item {} {\hyperref[\detokenize{reference/javascript_api:mail.utils}]{\sphinxcrossref{
mail.utils
}}}
\item {} {\hyperref[\detokenize{reference/javascript_api:web.core}]{\sphinxcrossref{
web.core
}}}
\item {} {\hyperref[\detokenize{reference/javascript_api:web.view_dialogs}]{\sphinxcrossref{
web.view\_dialogs
}}}
\end{itemize}

\end{description}\end{quote}


\begin{fulllineitems}
\phantomsection\label{\detokenize{reference/javascript_api:ChatterComposer}}\pysiglinewithargsret{\sphinxbfcode{\sphinxupquote{class }}\sphinxbfcode{\sphinxupquote{ChatterComposer}}}{\emph{parent}, \emph{model}, \emph{suggested\_partners}, \emph{options}}{}~\begin{quote}\begin{description}
\item[{Extends}] \leavevmode
BasicComposer

\item[{Parameters}] \leavevmode\begin{itemize}

\sphinxstylestrong{parent}

\sphinxstylestrong{model}

\sphinxstylestrong{suggested\_partners}

\sphinxstylestrong{options}
\end{itemize}

\end{description}\end{quote}


\begin{fulllineitems}
\phantomsection\label{\detokenize{reference/javascript_api:prevent_send}}\pysiglinewithargsret{\sphinxbfcode{\sphinxupquote{method }}\sphinxbfcode{\sphinxupquote{prevent\_send}}}{\emph{event}}{}
Send the message on SHIFT+ENTER, but go to new line on ENTER
\begin{quote}\begin{description}
\item[{Parameters}] \leavevmode\begin{itemize}

\sphinxstylestrong{event}
\end{itemize}

\end{description}\end{quote}

\end{fulllineitems}



\begin{fulllineitems}
\phantomsection\label{\detokenize{reference/javascript_api:get_checked_suggested_partners}}\pysiglinewithargsret{\sphinxbfcode{\sphinxupquote{method }}\sphinxbfcode{\sphinxupquote{get\_checked\_suggested\_partners}}}{}{{ $\rightarrow$ Array}}
Get the list of selected suggested partners
\begin{quote}\begin{description}
\item[{Returns}] \leavevmode
list of ‘recipient’ selected partners (may not be created in db)

\item[{Return Type}] \leavevmode
\sphinxstyleliteralemphasis{\sphinxupquote{Array}}

\end{description}\end{quote}

\end{fulllineitems}



\begin{fulllineitems}
\phantomsection\label{\detokenize{reference/javascript_api:check_suggested_partners}}\pysiglinewithargsret{\sphinxbfcode{\sphinxupquote{method }}\sphinxbfcode{\sphinxupquote{check\_suggested\_partners}}}{\emph{checked\_suggested\_partners}}{{ $\rightarrow$ Deferred}}
Check the additional partners (not necessary registered partners), and open a popup form view
for the ones who informations is missing.
\begin{quote}\begin{description}
\item[{Parameters}] \leavevmode\begin{itemize}

\sphinxstylestrong{checked\_suggested\_partners} (\sphinxstyleliteralemphasis{\sphinxupquote{Array}}) \textendash{} list of ‘recipient’ partners to complete informations or validate
\end{itemize}

\item[{Returns}] \leavevmode
resolved with the list of checked suggested partners (real partner)

\item[{Return Type}] \leavevmode
\sphinxstyleliteralemphasis{\sphinxupquote{Deferred}}

\end{description}\end{quote}

\end{fulllineitems}


\end{fulllineitems}


\end{fulllineitems}

\phantomsection\label{\detokenize{reference/javascript_api:module-web.notification}}

\begin{fulllineitems}
\phantomsection\label{\detokenize{reference/javascript_api:web.notification}}\pysigline{\sphinxbfcode{\sphinxupquote{module }}\sphinxbfcode{\sphinxupquote{web.notification}}}~~\begin{quote}\begin{description}
\item[{Exports}] \leavevmode{\hyperref[\detokenize{reference/javascript_api:web.notification.}]{\sphinxcrossref{
\textless{}anonymous\textgreater{}
}}}
\item[{Depends On}] \leavevmode\begin{itemize}
\item {} {\hyperref[\detokenize{reference/javascript_api:web.Widget}]{\sphinxcrossref{
web.Widget
}}}
\end{itemize}

\end{description}\end{quote}


\begin{fulllineitems}
\phantomsection\label{\detokenize{reference/javascript_api:web.notification.}}\pysigline{\sphinxbfcode{\sphinxupquote{namespace }}\sphinxbfcode{\sphinxupquote{}}}
\end{fulllineitems}


\end{fulllineitems}

\phantomsection\label{\detokenize{reference/javascript_api:module-website_sale.website_sale}}

\begin{fulllineitems}
\phantomsection\label{\detokenize{reference/javascript_api:website_sale.website_sale}}\pysigline{\sphinxbfcode{\sphinxupquote{module }}\sphinxbfcode{\sphinxupquote{website\_sale.website\_sale}}}~~\begin{quote}\begin{description}
\item[{Exports}] \leavevmode{\hyperref[\detokenize{reference/javascript_api:website_sale.website_sale.}]{\sphinxcrossref{
\textless{}anonymous\textgreater{}
}}}
\item[{Depends On}] \leavevmode\begin{itemize}
\item {} {\hyperref[\detokenize{reference/javascript_api:web.ajax}]{\sphinxcrossref{
web.ajax
}}}
\item {} {\hyperref[\detokenize{reference/javascript_api:web.config}]{\sphinxcrossref{
web.config
}}}
\item {} {\hyperref[\detokenize{reference/javascript_api:web.core}]{\sphinxcrossref{
web.core
}}}
\item {} {\hyperref[\detokenize{reference/javascript_api:web.utils}]{\sphinxcrossref{
web.utils
}}}
\item {} {\hyperref[\detokenize{reference/javascript_api:web_editor.base}]{\sphinxcrossref{
web\_editor.base
}}}
\end{itemize}

\end{description}\end{quote}


\begin{fulllineitems}
\phantomsection\label{\detokenize{reference/javascript_api:website_sale.website_sale.}}\pysigline{\sphinxbfcode{\sphinxupquote{namespace }}\sphinxbfcode{\sphinxupquote{}}}
\end{fulllineitems}


\end{fulllineitems}

\phantomsection\label{\detokenize{reference/javascript_api:module-web.ListController}}

\begin{fulllineitems}
\phantomsection\label{\detokenize{reference/javascript_api:web.ListController}}\pysigline{\sphinxbfcode{\sphinxupquote{module }}\sphinxbfcode{\sphinxupquote{web.ListController}}}~~\begin{quote}\begin{description}
\item[{Exports}] \leavevmode{\hyperref[\detokenize{reference/javascript_api:web.ListController.ListController}]{\sphinxcrossref{
ListController
}}}
\item[{Depends On}] \leavevmode\begin{itemize}
\item {} {\hyperref[\detokenize{reference/javascript_api:web.BasicController}]{\sphinxcrossref{
web.BasicController
}}}
\item {} {\hyperref[\detokenize{reference/javascript_api:web.DataExport}]{\sphinxcrossref{
web.DataExport
}}}
\item {} {\hyperref[\detokenize{reference/javascript_api:web.Sidebar}]{\sphinxcrossref{
web.Sidebar
}}}
\item {} {\hyperref[\detokenize{reference/javascript_api:web.core}]{\sphinxcrossref{
web.core
}}}
\item {} {\hyperref[\detokenize{reference/javascript_api:web.pyeval}]{\sphinxcrossref{
web.pyeval
}}}
\end{itemize}

\end{description}\end{quote}


\begin{fulllineitems}
\phantomsection\label{\detokenize{reference/javascript_api:ListController}}\pysiglinewithargsret{\sphinxbfcode{\sphinxupquote{class }}\sphinxbfcode{\sphinxupquote{ListController}}}{}{}~\begin{quote}\begin{description}
\item[{Extends}] \leavevmode{\hyperref[\detokenize{reference/javascript_api:web.BasicController.BasicController}]{\sphinxcrossref{
BasicController
}}}
\end{description}\end{quote}


\begin{fulllineitems}
\phantomsection\label{\detokenize{reference/javascript_api:getActiveDomain}}\pysiglinewithargsret{\sphinxbfcode{\sphinxupquote{method }}\sphinxbfcode{\sphinxupquote{getActiveDomain}}}{}{{ $\rightarrow$ Deferred\textless{}array{[}{]}\textgreater{}}}
Calculate the active domain of the list view. This should be done only
if the header checkbox has been checked. This is done by evaluating the
search results, and then adding the dataset domain (i.e. action domain).
\begin{quote}\begin{description}
\item[{Returns}] \leavevmode
a deferred that resolve to the active domain

\item[{Return Type}] \leavevmode
\sphinxstyleliteralemphasis{\sphinxupquote{Deferred}}\textless{}\sphinxstyleliteralemphasis{\sphinxupquote{Array}}\textless{}\sphinxstyleliteralemphasis{\sphinxupquote{array}}\textgreater{}\textgreater{}

\end{description}\end{quote}

\end{fulllineitems}



\begin{fulllineitems}
\phantomsection\label{\detokenize{reference/javascript_api:getSelectedIds}}\pysiglinewithargsret{\sphinxbfcode{\sphinxupquote{method }}\sphinxbfcode{\sphinxupquote{getSelectedIds}}}{}{{ $\rightarrow$ number{[}{]}}}
Returns the list of currently selected res\_ids (with the check boxes on
the left)
\begin{quote}\begin{description}
\item[{Returns}] \leavevmode
list of res\_ids

\item[{Return Type}] \leavevmode
\sphinxstyleliteralemphasis{\sphinxupquote{Array}}\textless{}\sphinxstyleliteralemphasis{\sphinxupquote{number}}\textgreater{}

\end{description}\end{quote}

\end{fulllineitems}



\begin{fulllineitems}
\phantomsection\label{\detokenize{reference/javascript_api:getSelectedRecords}}\pysiglinewithargsret{\sphinxbfcode{\sphinxupquote{method }}\sphinxbfcode{\sphinxupquote{getSelectedRecords}}}{}{{ $\rightarrow$ Object{[}{]}}}
Returns the list of currently selected records (with the check boxes on
the left)
\begin{quote}\begin{description}
\item[{Returns}] \leavevmode
list of records

\item[{Return Type}] \leavevmode
\sphinxstyleliteralemphasis{\sphinxupquote{Array}}\textless{}\sphinxstyleliteralemphasis{\sphinxupquote{Object}}\textgreater{}

\end{description}\end{quote}

\end{fulllineitems}



\begin{fulllineitems}
\phantomsection\label{\detokenize{reference/javascript_api:renderButtons}}\pysiglinewithargsret{\sphinxbfcode{\sphinxupquote{method }}\sphinxbfcode{\sphinxupquote{renderButtons}}}{}{}
Extends the renderButtons function of ListView by adding an event listener
on the import button.

\end{fulllineitems}



\begin{fulllineitems}
\phantomsection\label{\detokenize{reference/javascript_api:renderSidebar}}\pysiglinewithargsret{\sphinxbfcode{\sphinxupquote{method }}\sphinxbfcode{\sphinxupquote{renderSidebar}}}{\emph{\$node}}{}
Render the sidebar (the ‘action’ menu in the control panel, right of the
main buttons)
\begin{quote}\begin{description}
\item[{Parameters}] \leavevmode\begin{itemize}

\sphinxstylestrong{\$node} (\sphinxstyleliteralemphasis{\sphinxupquote{jQuery}})
\end{itemize}

\end{description}\end{quote}

\end{fulllineitems}


\end{fulllineitems}


\end{fulllineitems}

\phantomsection\label{\detokenize{reference/javascript_api:module-web.time}}

\begin{fulllineitems}
\phantomsection\label{\detokenize{reference/javascript_api:web.time}}\pysigline{\sphinxbfcode{\sphinxupquote{module }}\sphinxbfcode{\sphinxupquote{web.time}}}~~\begin{quote}\begin{description}
\item[{Exports}] \leavevmode{\hyperref[\detokenize{reference/javascript_api:web.time.}]{\sphinxcrossref{
\textless{}anonymous\textgreater{}
}}}
\item[{Depends On}] \leavevmode\begin{itemize}
\item {} {\hyperref[\detokenize{reference/javascript_api:web.translation}]{\sphinxcrossref{
web.translation
}}}
\item {} {\hyperref[\detokenize{reference/javascript_api:web.utils}]{\sphinxcrossref{
web.utils
}}}
\end{itemize}

\end{description}\end{quote}


\begin{fulllineitems}
\phantomsection\label{\detokenize{reference/javascript_api:datetime_to_str}}\pysiglinewithargsret{\sphinxbfcode{\sphinxupquote{function }}\sphinxbfcode{\sphinxupquote{datetime\_to\_str}}}{\emph{obj}}{{ $\rightarrow$ String}}
Converts a Date javascript object to a string using OpenERP’s
datetime string format (exemple: ‘2011-12-01 15:12:35’).

The time zone of the Date object is assumed to be the one of the
browser and it will be converted to UTC (standard for OpenERP 6.1).
\begin{quote}\begin{description}
\item[{Parameters}] \leavevmode\begin{itemize}

\sphinxstylestrong{obj} (\sphinxstyleliteralemphasis{\sphinxupquote{Date}})
\end{itemize}

\item[{Returns}] \leavevmode
A string representing a datetime.

\item[{Return Type}] \leavevmode
\sphinxstyleliteralemphasis{\sphinxupquote{String}}

\end{description}\end{quote}

\end{fulllineitems}



\begin{fulllineitems}
\phantomsection\label{\detokenize{reference/javascript_api:date_to_str}}\pysiglinewithargsret{\sphinxbfcode{\sphinxupquote{function }}\sphinxbfcode{\sphinxupquote{date\_to\_str}}}{\emph{obj}}{{ $\rightarrow$ String}}
Converts a Date javascript object to a string using OpenERP’s
date string format (exemple: ‘2011-12-01’).

As a date is not subject to time zones, we assume it should be
represented as a Date javascript object at 00:00:00 in the
time zone of the browser.
\begin{quote}\begin{description}
\item[{Parameters}] \leavevmode\begin{itemize}

\sphinxstylestrong{obj} (\sphinxstyleliteralemphasis{\sphinxupquote{Date}})
\end{itemize}

\item[{Returns}] \leavevmode
A string representing a date.

\item[{Return Type}] \leavevmode
\sphinxstyleliteralemphasis{\sphinxupquote{String}}

\end{description}\end{quote}

\end{fulllineitems}



\begin{fulllineitems}
\phantomsection\label{\detokenize{reference/javascript_api:web.time.}}\pysigline{\sphinxbfcode{\sphinxupquote{namespace }}\sphinxbfcode{\sphinxupquote{}}}~

\begin{fulllineitems}
\phantomsection\label{\detokenize{reference/javascript_api:date_to_utc}}\pysiglinewithargsret{\sphinxbfcode{\sphinxupquote{function }}\sphinxbfcode{\sphinxupquote{date\_to\_utc}}}{\emph{k}, \emph{v}}{{ $\rightarrow$ Object}}
Replacer function for JSON.stringify, serializes Date objects to UTC
datetime in the OpenERP Server format.

However, if a serialized value has a toJSON method that method is called
\sphinxstyleemphasis{before} the replacer is invoked. Date\#toJSON exists, and thus the value
passed to the replacer is a string, the original Date has to be fetched
on the parent object (which is provided as the replacer’s context).
\begin{quote}\begin{description}
\item[{Parameters}] \leavevmode\begin{itemize}

\sphinxstylestrong{k} (\sphinxstyleliteralemphasis{\sphinxupquote{String}})

\sphinxstylestrong{v} (\sphinxstyleliteralemphasis{\sphinxupquote{Object}})
\end{itemize}

\item[{Return Type}] \leavevmode
\sphinxstyleliteralemphasis{\sphinxupquote{Object}}

\end{description}\end{quote}

\end{fulllineitems}



\begin{fulllineitems}
\phantomsection\label{\detokenize{reference/javascript_api:str_to_datetime}}\pysiglinewithargsret{\sphinxbfcode{\sphinxupquote{function }}\sphinxbfcode{\sphinxupquote{str\_to\_datetime}}}{\emph{str}}{{ $\rightarrow$ Date}}
Converts a string to a Date javascript object using OpenERP’s
datetime string format (exemple: ‘2011-12-01 15:12:35.832’).

The time zone is assumed to be UTC (standard for OpenERP 6.1)
and will be converted to the browser’s time zone.
\begin{quote}\begin{description}
\item[{Parameters}] \leavevmode\begin{itemize}

\sphinxstylestrong{str} (\sphinxstyleliteralemphasis{\sphinxupquote{String}}) \textendash{} A string representing a datetime.
\end{itemize}

\item[{Return Type}] \leavevmode
\sphinxstyleliteralemphasis{\sphinxupquote{Date}}

\end{description}\end{quote}

\end{fulllineitems}



\begin{fulllineitems}
\phantomsection\label{\detokenize{reference/javascript_api:str_to_date}}\pysiglinewithargsret{\sphinxbfcode{\sphinxupquote{function }}\sphinxbfcode{\sphinxupquote{str\_to\_date}}}{\emph{str}}{{ $\rightarrow$ Date}}
Converts a string to a Date javascript object using OpenERP’s
date string format (exemple: ‘2011-12-01’).

As a date is not subject to time zones, we assume it should be
represented as a Date javascript object at 00:00:00 in the
time zone of the browser.
\begin{quote}\begin{description}
\item[{Parameters}] \leavevmode\begin{itemize}

\sphinxstylestrong{str} (\sphinxstyleliteralemphasis{\sphinxupquote{String}}) \textendash{} A string representing a date.
\end{itemize}

\item[{Return Type}] \leavevmode
\sphinxstyleliteralemphasis{\sphinxupquote{Date}}

\end{description}\end{quote}

\end{fulllineitems}



\begin{fulllineitems}
\phantomsection\label{\detokenize{reference/javascript_api:str_to_time}}\pysiglinewithargsret{\sphinxbfcode{\sphinxupquote{function }}\sphinxbfcode{\sphinxupquote{str\_to\_time}}}{\emph{str}}{{ $\rightarrow$ Date}}
Converts a string to a Date javascript object using OpenERP’s
time string format (exemple: ‘15:12:35’).

The OpenERP times are supposed to always be naive times. We assume it is
represented using a javascript Date with a date 1 of January 1970 and a
time corresponding to the meant time in the browser’s time zone.
\begin{quote}\begin{description}
\item[{Parameters}] \leavevmode\begin{itemize}

\sphinxstylestrong{str} (\sphinxstyleliteralemphasis{\sphinxupquote{String}}) \textendash{} A string representing a time.
\end{itemize}

\item[{Return Type}] \leavevmode
\sphinxstyleliteralemphasis{\sphinxupquote{Date}}

\end{description}\end{quote}

\end{fulllineitems}



\begin{fulllineitems}
\phantomsection\label{\detokenize{reference/javascript_api:datetime_to_str}}\pysiglinewithargsret{\sphinxbfcode{\sphinxupquote{function }}\sphinxbfcode{\sphinxupquote{datetime\_to\_str}}}{\emph{obj}}{{ $\rightarrow$ String}}
Converts a Date javascript object to a string using OpenERP’s
datetime string format (exemple: ‘2011-12-01 15:12:35’).

The time zone of the Date object is assumed to be the one of the
browser and it will be converted to UTC (standard for OpenERP 6.1).
\begin{quote}\begin{description}
\item[{Parameters}] \leavevmode\begin{itemize}

\sphinxstylestrong{obj} (\sphinxstyleliteralemphasis{\sphinxupquote{Date}})
\end{itemize}

\item[{Returns}] \leavevmode
A string representing a datetime.

\item[{Return Type}] \leavevmode
\sphinxstyleliteralemphasis{\sphinxupquote{String}}

\end{description}\end{quote}

\end{fulllineitems}



\begin{fulllineitems}
\phantomsection\label{\detokenize{reference/javascript_api:date_to_str}}\pysiglinewithargsret{\sphinxbfcode{\sphinxupquote{function }}\sphinxbfcode{\sphinxupquote{date\_to\_str}}}{\emph{obj}}{{ $\rightarrow$ String}}
Converts a Date javascript object to a string using OpenERP’s
date string format (exemple: ‘2011-12-01’).

As a date is not subject to time zones, we assume it should be
represented as a Date javascript object at 00:00:00 in the
time zone of the browser.
\begin{quote}\begin{description}
\item[{Parameters}] \leavevmode\begin{itemize}

\sphinxstylestrong{obj} (\sphinxstyleliteralemphasis{\sphinxupquote{Date}})
\end{itemize}

\item[{Returns}] \leavevmode
A string representing a date.

\item[{Return Type}] \leavevmode
\sphinxstyleliteralemphasis{\sphinxupquote{String}}

\end{description}\end{quote}

\end{fulllineitems}



\begin{fulllineitems}
\phantomsection\label{\detokenize{reference/javascript_api:time_to_str}}\pysiglinewithargsret{\sphinxbfcode{\sphinxupquote{function }}\sphinxbfcode{\sphinxupquote{time\_to\_str}}}{\emph{obj}}{{ $\rightarrow$ String}}
Converts a Date javascript object to a string using OpenERP’s
time string format (exemple: ‘15:12:35’).

The OpenERP times are supposed to always be naive times. We assume it is
represented using a javascript Date with a date 1 of January 1970 and a
time corresponding to the meant time in the browser’s time zone.
\begin{quote}\begin{description}
\item[{Parameters}] \leavevmode\begin{itemize}

\sphinxstylestrong{obj} (\sphinxstyleliteralemphasis{\sphinxupquote{Date}})
\end{itemize}

\item[{Returns}] \leavevmode
A string representing a time.

\item[{Return Type}] \leavevmode
\sphinxstyleliteralemphasis{\sphinxupquote{String}}

\end{description}\end{quote}

\end{fulllineitems}



\begin{fulllineitems}
\phantomsection\label{\detokenize{reference/javascript_api:strftime_to_moment_format}}\pysiglinewithargsret{\sphinxbfcode{\sphinxupquote{function }}\sphinxbfcode{\sphinxupquote{strftime\_to\_moment\_format}}}{\emph{value}}{}
Convert Python strftime to escaped moment.js format.
\begin{quote}\begin{description}
\item[{Parameters}] \leavevmode\begin{itemize}

\sphinxstylestrong{value} (\sphinxstyleliteralemphasis{\sphinxupquote{String}}) \textendash{} original format
\end{itemize}

\end{description}\end{quote}

\end{fulllineitems}



\begin{fulllineitems}
\phantomsection\label{\detokenize{reference/javascript_api:moment_to_strftime_format}}\pysiglinewithargsret{\sphinxbfcode{\sphinxupquote{function }}\sphinxbfcode{\sphinxupquote{moment\_to\_strftime\_format}}}{\emph{value}}{}
Convert moment.js format to python strftime
\begin{quote}\begin{description}
\item[{Parameters}] \leavevmode\begin{itemize}

\sphinxstylestrong{value} (\sphinxstyleliteralemphasis{\sphinxupquote{String}}) \textendash{} original format
\end{itemize}

\end{description}\end{quote}

\end{fulllineitems}



\begin{fulllineitems}
\phantomsection\label{\detokenize{reference/javascript_api:getLangDateFormat}}\pysiglinewithargsret{\sphinxbfcode{\sphinxupquote{function }}\sphinxbfcode{\sphinxupquote{getLangDateFormat}}}{}{}
Get date format of the user’s language

\end{fulllineitems}



\begin{fulllineitems}
\phantomsection\label{\detokenize{reference/javascript_api:getLangTimeFormat}}\pysiglinewithargsret{\sphinxbfcode{\sphinxupquote{function }}\sphinxbfcode{\sphinxupquote{getLangTimeFormat}}}{}{}
Get time format of the user’s language

\end{fulllineitems}



\begin{fulllineitems}
\phantomsection\label{\detokenize{reference/javascript_api:getLangDatetimeFormat}}\pysiglinewithargsret{\sphinxbfcode{\sphinxupquote{function }}\sphinxbfcode{\sphinxupquote{getLangDatetimeFormat}}}{}{}
Get date time format of the user’s language

\end{fulllineitems}


\end{fulllineitems}



\begin{fulllineitems}
\phantomsection\label{\detokenize{reference/javascript_api:str_to_time}}\pysiglinewithargsret{\sphinxbfcode{\sphinxupquote{function }}\sphinxbfcode{\sphinxupquote{str\_to\_time}}}{\emph{str}}{{ $\rightarrow$ Date}}
Converts a string to a Date javascript object using OpenERP’s
time string format (exemple: ‘15:12:35’).

The OpenERP times are supposed to always be naive times. We assume it is
represented using a javascript Date with a date 1 of January 1970 and a
time corresponding to the meant time in the browser’s time zone.
\begin{quote}\begin{description}
\item[{Parameters}] \leavevmode\begin{itemize}

\sphinxstylestrong{str} (\sphinxstyleliteralemphasis{\sphinxupquote{String}}) \textendash{} A string representing a time.
\end{itemize}

\item[{Return Type}] \leavevmode
\sphinxstyleliteralemphasis{\sphinxupquote{Date}}

\end{description}\end{quote}

\end{fulllineitems}



\begin{fulllineitems}
\phantomsection\label{\detokenize{reference/javascript_api:strftime_to_moment_format}}\pysiglinewithargsret{\sphinxbfcode{\sphinxupquote{function }}\sphinxbfcode{\sphinxupquote{strftime\_to\_moment\_format}}}{\emph{value}}{}
Convert Python strftime to escaped moment.js format.
\begin{quote}\begin{description}
\item[{Parameters}] \leavevmode\begin{itemize}

\sphinxstylestrong{value} (\sphinxstyleliteralemphasis{\sphinxupquote{String}}) \textendash{} original format
\end{itemize}

\end{description}\end{quote}

\end{fulllineitems}



\begin{fulllineitems}
\phantomsection\label{\detokenize{reference/javascript_api:time_to_str}}\pysiglinewithargsret{\sphinxbfcode{\sphinxupquote{function }}\sphinxbfcode{\sphinxupquote{time\_to\_str}}}{\emph{obj}}{{ $\rightarrow$ String}}
Converts a Date javascript object to a string using OpenERP’s
time string format (exemple: ‘15:12:35’).

The OpenERP times are supposed to always be naive times. We assume it is
represented using a javascript Date with a date 1 of January 1970 and a
time corresponding to the meant time in the browser’s time zone.
\begin{quote}\begin{description}
\item[{Parameters}] \leavevmode\begin{itemize}

\sphinxstylestrong{obj} (\sphinxstyleliteralemphasis{\sphinxupquote{Date}})
\end{itemize}

\item[{Returns}] \leavevmode
A string representing a time.

\item[{Return Type}] \leavevmode
\sphinxstyleliteralemphasis{\sphinxupquote{String}}

\end{description}\end{quote}

\end{fulllineitems}



\begin{fulllineitems}
\phantomsection\label{\detokenize{reference/javascript_api:str_to_datetime}}\pysiglinewithargsret{\sphinxbfcode{\sphinxupquote{function }}\sphinxbfcode{\sphinxupquote{str\_to\_datetime}}}{\emph{str}}{{ $\rightarrow$ Date}}
Converts a string to a Date javascript object using OpenERP’s
datetime string format (exemple: ‘2011-12-01 15:12:35.832’).

The time zone is assumed to be UTC (standard for OpenERP 6.1)
and will be converted to the browser’s time zone.
\begin{quote}\begin{description}
\item[{Parameters}] \leavevmode\begin{itemize}

\sphinxstylestrong{str} (\sphinxstyleliteralemphasis{\sphinxupquote{String}}) \textendash{} A string representing a datetime.
\end{itemize}

\item[{Return Type}] \leavevmode
\sphinxstyleliteralemphasis{\sphinxupquote{Date}}

\end{description}\end{quote}

\end{fulllineitems}



\begin{fulllineitems}
\phantomsection\label{\detokenize{reference/javascript_api:moment_to_strftime_format}}\pysiglinewithargsret{\sphinxbfcode{\sphinxupquote{function }}\sphinxbfcode{\sphinxupquote{moment\_to\_strftime\_format}}}{\emph{value}}{}
Convert moment.js format to python strftime
\begin{quote}\begin{description}
\item[{Parameters}] \leavevmode\begin{itemize}

\sphinxstylestrong{value} (\sphinxstyleliteralemphasis{\sphinxupquote{String}}) \textendash{} original format
\end{itemize}

\end{description}\end{quote}

\end{fulllineitems}



\begin{fulllineitems}
\phantomsection\label{\detokenize{reference/javascript_api:str_to_date}}\pysiglinewithargsret{\sphinxbfcode{\sphinxupquote{function }}\sphinxbfcode{\sphinxupquote{str\_to\_date}}}{\emph{str}}{{ $\rightarrow$ Date}}
Converts a string to a Date javascript object using OpenERP’s
date string format (exemple: ‘2011-12-01’).

As a date is not subject to time zones, we assume it should be
represented as a Date javascript object at 00:00:00 in the
time zone of the browser.
\begin{quote}\begin{description}
\item[{Parameters}] \leavevmode\begin{itemize}

\sphinxstylestrong{str} (\sphinxstyleliteralemphasis{\sphinxupquote{String}}) \textendash{} A string representing a date.
\end{itemize}

\item[{Return Type}] \leavevmode
\sphinxstyleliteralemphasis{\sphinxupquote{Date}}

\end{description}\end{quote}

\end{fulllineitems}



\begin{fulllineitems}
\phantomsection\label{\detokenize{reference/javascript_api:getLangDateFormat}}\pysiglinewithargsret{\sphinxbfcode{\sphinxupquote{function }}\sphinxbfcode{\sphinxupquote{getLangDateFormat}}}{}{}
Get date format of the user’s language

\end{fulllineitems}



\begin{fulllineitems}
\phantomsection\label{\detokenize{reference/javascript_api:getLangTimeFormat}}\pysiglinewithargsret{\sphinxbfcode{\sphinxupquote{function }}\sphinxbfcode{\sphinxupquote{getLangTimeFormat}}}{}{}
Get time format of the user’s language

\end{fulllineitems}



\begin{fulllineitems}
\phantomsection\label{\detokenize{reference/javascript_api:getLangDatetimeFormat}}\pysiglinewithargsret{\sphinxbfcode{\sphinxupquote{function }}\sphinxbfcode{\sphinxupquote{getLangDatetimeFormat}}}{}{}
Get date time format of the user’s language

\end{fulllineitems}



\begin{fulllineitems}
\phantomsection\label{\detokenize{reference/javascript_api:date_to_utc}}\pysiglinewithargsret{\sphinxbfcode{\sphinxupquote{function }}\sphinxbfcode{\sphinxupquote{date\_to\_utc}}}{\emph{k}, \emph{v}}{{ $\rightarrow$ Object}}
Replacer function for JSON.stringify, serializes Date objects to UTC
datetime in the OpenERP Server format.

However, if a serialized value has a toJSON method that method is called
\sphinxstyleemphasis{before} the replacer is invoked. Date\#toJSON exists, and thus the value
passed to the replacer is a string, the original Date has to be fetched
on the parent object (which is provided as the replacer’s context).
\begin{quote}\begin{description}
\item[{Parameters}] \leavevmode\begin{itemize}

\sphinxstylestrong{k} (\sphinxstyleliteralemphasis{\sphinxupquote{String}})

\sphinxstylestrong{v} (\sphinxstyleliteralemphasis{\sphinxupquote{Object}})
\end{itemize}

\item[{Return Type}] \leavevmode
\sphinxstyleliteralemphasis{\sphinxupquote{Object}}

\end{description}\end{quote}

\end{fulllineitems}


\end{fulllineitems}

\phantomsection\label{\detokenize{reference/javascript_api:module-website.customizeMenu}}

\begin{fulllineitems}
\phantomsection\label{\detokenize{reference/javascript_api:website.customizeMenu}}\pysigline{\sphinxbfcode{\sphinxupquote{module }}\sphinxbfcode{\sphinxupquote{website.customizeMenu}}}~~\begin{quote}\begin{description}
\item[{Exports}] \leavevmode{\hyperref[\detokenize{reference/javascript_api:website.customizeMenu.CustomizeMenu}]{\sphinxcrossref{
CustomizeMenu
}}}
\item[{Depends On}] \leavevmode\begin{itemize}
\item {} {\hyperref[\detokenize{reference/javascript_api:web.Widget}]{\sphinxcrossref{
web.Widget
}}}
\item {} {\hyperref[\detokenize{reference/javascript_api:web.core}]{\sphinxcrossref{
web.core
}}}
\item {} {\hyperref[\detokenize{reference/javascript_api:web_editor.context}]{\sphinxcrossref{
web\_editor.context
}}}
\item {} {\hyperref[\detokenize{reference/javascript_api:website.ace}]{\sphinxcrossref{
website.ace
}}}
\item {} {\hyperref[\detokenize{reference/javascript_api:website.navbar}]{\sphinxcrossref{
website.navbar
}}}
\end{itemize}

\end{description}\end{quote}


\begin{fulllineitems}
\phantomsection\label{\detokenize{reference/javascript_api:CustomizeMenu}}\pysiglinewithargsret{\sphinxbfcode{\sphinxupquote{class }}\sphinxbfcode{\sphinxupquote{CustomizeMenu}}}{}{}~\begin{quote}\begin{description}
\item[{Extends}] \leavevmode{\hyperref[\detokenize{reference/javascript_api:web.Widget.Widget}]{\sphinxcrossref{
Widget
}}}
\end{description}\end{quote}

\end{fulllineitems}


\end{fulllineitems}

\phantomsection\label{\detokenize{reference/javascript_api:module-web.ListRenderer}}

\begin{fulllineitems}
\phantomsection\label{\detokenize{reference/javascript_api:web.ListRenderer}}\pysigline{\sphinxbfcode{\sphinxupquote{module }}\sphinxbfcode{\sphinxupquote{web.ListRenderer}}}~~\begin{quote}\begin{description}
\item[{Exports}] \leavevmode{\hyperref[\detokenize{reference/javascript_api:web.ListRenderer.ListRenderer}]{\sphinxcrossref{
ListRenderer
}}}
\item[{Depends On}] \leavevmode\begin{itemize}
\item {} {\hyperref[\detokenize{reference/javascript_api:web.BasicRenderer}]{\sphinxcrossref{
web.BasicRenderer
}}}
\item {} {\hyperref[\detokenize{reference/javascript_api:web.Dialog}]{\sphinxcrossref{
web.Dialog
}}}
\item {} {\hyperref[\detokenize{reference/javascript_api:web.Pager}]{\sphinxcrossref{
web.Pager
}}}
\item {} {\hyperref[\detokenize{reference/javascript_api:web.config}]{\sphinxcrossref{
web.config
}}}
\item {} {\hyperref[\detokenize{reference/javascript_api:web.core}]{\sphinxcrossref{
web.core
}}}
\item {} {\hyperref[\detokenize{reference/javascript_api:web.dom}]{\sphinxcrossref{
web.dom
}}}
\item {} {\hyperref[\detokenize{reference/javascript_api:web.field_utils}]{\sphinxcrossref{
web.field\_utils
}}}
\item {} {\hyperref[\detokenize{reference/javascript_api:web.utils}]{\sphinxcrossref{
web.utils
}}}
\end{itemize}

\end{description}\end{quote}


\begin{fulllineitems}
\phantomsection\label{\detokenize{reference/javascript_api:ListRenderer}}\pysiglinewithargsret{\sphinxbfcode{\sphinxupquote{class }}\sphinxbfcode{\sphinxupquote{ListRenderer}}}{\emph{parent}, \emph{state}, \emph{params}}{}~\begin{quote}\begin{description}
\item[{Extends}] \leavevmode{\hyperref[\detokenize{reference/javascript_api:web.BasicRenderer.BasicRenderer}]{\sphinxcrossref{
BasicRenderer
}}}
\item[{Parameters}] \leavevmode\begin{itemize}

\sphinxstylestrong{parent}

\sphinxstylestrong{state}

\sphinxstylestrong{params} ({\hyperref[\detokenize{reference/javascript_api:ListRendererParams}]{\sphinxcrossref{\sphinxstyleliteralemphasis{\sphinxupquote{ListRendererParams}}}}})
\end{itemize}

\end{description}\end{quote}


\begin{fulllineitems}
\phantomsection\label{\detokenize{reference/javascript_api:canBeSaved}}\pysiglinewithargsret{\sphinxbfcode{\sphinxupquote{method }}\sphinxbfcode{\sphinxupquote{canBeSaved}}}{\sphinxoptional{\emph{recordID}}}{{ $\rightarrow$ string{[}{]}}}
If the given recordID is the list main one (or that no recordID is
given), then the whole view can be saved if one of the two following
conditions is true:
- There is no line in edition (all lines are saved so they are all valid)
- The line in edition can be saved
\begin{quote}\begin{description}
\item[{Parameters}] \leavevmode\begin{itemize}

\sphinxstylestrong{recordID} (\sphinxstyleliteralemphasis{\sphinxupquote{string}})
\end{itemize}

\item[{Return Type}] \leavevmode
\sphinxstyleliteralemphasis{\sphinxupquote{Array}}\textless{}\sphinxstyleliteralemphasis{\sphinxupquote{string}}\textgreater{}

\end{description}\end{quote}

\end{fulllineitems}



\begin{fulllineitems}
\phantomsection\label{\detokenize{reference/javascript_api:confirmChange}}\pysiglinewithargsret{\sphinxbfcode{\sphinxupquote{method }}\sphinxbfcode{\sphinxupquote{confirmChange}}}{\emph{state}, \emph{id}}{}
We need to override the confirmChange method from BasicRenderer to
reevaluate the row decorations.  Since they depends on the current value
of the row, they might have changed between each edit.
\begin{quote}\begin{description}
\item[{Parameters}] \leavevmode\begin{itemize}

\sphinxstylestrong{state}

\sphinxstylestrong{id}
\end{itemize}

\end{description}\end{quote}

\end{fulllineitems}



\begin{fulllineitems}
\phantomsection\label{\detokenize{reference/javascript_api:confirmUpdate}}\pysiglinewithargsret{\sphinxbfcode{\sphinxupquote{method }}\sphinxbfcode{\sphinxupquote{confirmUpdate}}}{\emph{state}, \emph{id}, \emph{fields}, \emph{ev}}{{ $\rightarrow$ Deferred\textless{}AbstractField{[}{]}\textgreater{}}}
This is a specialized version of confirmChange, meant to be called when
the change may have affected more than one line (so, for example, an
onchange which add/remove a few lines in a x2many.  This does not occur
in a normal list view)

The update is more difficult when other rows could have been changed. We
need to potentially remove some lines, add some other lines, update some
other lines and maybe reorder a few of them.  This problem would neatly
be solved by using a virtual dom, but we do not have this luxury yet.
So, in the meantime, what we do is basically remove every current row
except the ‘main’ one (the row which caused the update), then rerender
every new row and add them before/after the main one.
\begin{quote}\begin{description}
\item[{Parameters}] \leavevmode\begin{itemize}

\sphinxstylestrong{state} (\sphinxstyleliteralemphasis{\sphinxupquote{Object}})

\sphinxstylestrong{id} (\sphinxstyleliteralemphasis{\sphinxupquote{string}})

\sphinxstylestrong{fields} (\sphinxstyleliteralemphasis{\sphinxupquote{Array}}\textless{}\sphinxstyleliteralemphasis{\sphinxupquote{string}}\textgreater{})

\sphinxstylestrong{ev} (\sphinxstyleliteralemphasis{\sphinxupquote{OdooEvent}})
\end{itemize}

\item[{Returns}] \leavevmode
resolved with the list of widgets
                                     that have been reset

\item[{Return Type}] \leavevmode
\sphinxstyleliteralemphasis{\sphinxupquote{Deferred}}\textless{}\sphinxstyleliteralemphasis{\sphinxupquote{Array}}\textless{}{\hyperref[\detokenize{reference/javascript_api:AbstractField}]{\sphinxcrossref{\sphinxstyleliteralemphasis{\sphinxupquote{AbstractField}}}}}\textgreater{}\textgreater{}

\end{description}\end{quote}

\end{fulllineitems}



\begin{fulllineitems}
\phantomsection\label{\detokenize{reference/javascript_api:editRecord}}\pysiglinewithargsret{\sphinxbfcode{\sphinxupquote{method }}\sphinxbfcode{\sphinxupquote{editRecord}}}{\emph{recordID}}{}
Edit a given record in the list
\begin{quote}\begin{description}
\item[{Parameters}] \leavevmode\begin{itemize}

\sphinxstylestrong{recordID} (\sphinxstyleliteralemphasis{\sphinxupquote{string}})
\end{itemize}

\end{description}\end{quote}

\end{fulllineitems}



\begin{fulllineitems}
\phantomsection\label{\detokenize{reference/javascript_api:getEditableRecordID}}\pysiglinewithargsret{\sphinxbfcode{\sphinxupquote{method }}\sphinxbfcode{\sphinxupquote{getEditableRecordID}}}{}{{ $\rightarrow$ string\textbar{}null}}
Returns the recordID associated to the line which is currently in edition
or null if there is no line in edition.
\begin{quote}\begin{description}
\item[{Return Type}] \leavevmode
\sphinxstyleliteralemphasis{\sphinxupquote{string}}\sphinxstyleemphasis{ or }null

\end{description}\end{quote}

\end{fulllineitems}



\begin{fulllineitems}
\phantomsection\label{\detokenize{reference/javascript_api:removeLine}}\pysiglinewithargsret{\sphinxbfcode{\sphinxupquote{method }}\sphinxbfcode{\sphinxupquote{removeLine}}}{\emph{state}, \emph{recordID}}{}
Removes the line associated to the given recordID (the index of the row
is found thanks to the old state), then updates the state.
\begin{quote}\begin{description}
\item[{Parameters}] \leavevmode\begin{itemize}

\sphinxstylestrong{state} (\sphinxstyleliteralemphasis{\sphinxupquote{Object}})

\sphinxstylestrong{recordID} (\sphinxstyleliteralemphasis{\sphinxupquote{string}})
\end{itemize}

\end{description}\end{quote}

\end{fulllineitems}



\begin{fulllineitems}
\phantomsection\label{\detokenize{reference/javascript_api:setRowMode}}\pysiglinewithargsret{\sphinxbfcode{\sphinxupquote{method }}\sphinxbfcode{\sphinxupquote{setRowMode}}}{\emph{recordID}, \emph{mode}}{{ $\rightarrow$ Deferred}}
Updates the already rendered row associated to the given recordID so that
it fits the given mode.
\begin{quote}\begin{description}
\item[{Parameters}] \leavevmode\begin{itemize}

\sphinxstylestrong{recordID} (\sphinxstyleliteralemphasis{\sphinxupquote{string}})

\sphinxstylestrong{mode} (\sphinxstyleliteralemphasis{\sphinxupquote{string}})
\end{itemize}

\item[{Return Type}] \leavevmode
\sphinxstyleliteralemphasis{\sphinxupquote{Deferred}}

\end{description}\end{quote}

\end{fulllineitems}



\begin{fulllineitems}
\phantomsection\label{\detokenize{reference/javascript_api:unselectRow}}\pysiglinewithargsret{\sphinxbfcode{\sphinxupquote{method }}\sphinxbfcode{\sphinxupquote{unselectRow}}}{}{{ $\rightarrow$ Deferred}}
This method is called whenever we click/move outside of a row that was
in edit mode. This is the moment we save all accumulated changes on that
row, if needed (@see BasicController.saveRecord).

Note that we have to disable the focusable elements (inputs, …) to
prevent subsequent editions. These edits would be lost, because the list
view only saves records when unselecting a row.
\begin{quote}\begin{description}
\item[{Returns}] \leavevmode
The deferred resolves if the row was unselected (and
  possibly removed). If may be rejected, when the row is dirty and the
  user refuses to discard its changes.

\item[{Return Type}] \leavevmode
\sphinxstyleliteralemphasis{\sphinxupquote{Deferred}}

\end{description}\end{quote}

\end{fulllineitems}



\begin{fulllineitems}
\phantomsection\label{\detokenize{reference/javascript_api:ListRendererParams}}\pysiglinewithargsret{\sphinxbfcode{\sphinxupquote{class }}\sphinxbfcode{\sphinxupquote{ListRendererParams}}}{}{}~

\begin{fulllineitems}
\phantomsection\label{\detokenize{reference/javascript_api:addCreateLine}}\pysigline{\sphinxbfcode{\sphinxupquote{attribute }}\sphinxbfcode{\sphinxupquote{addCreateLine}} boolean}
\end{fulllineitems}



\begin{fulllineitems}
\phantomsection\label{\detokenize{reference/javascript_api:addTrashIcon}}\pysigline{\sphinxbfcode{\sphinxupquote{attribute }}\sphinxbfcode{\sphinxupquote{addTrashIcon}} boolean}
\end{fulllineitems}


\end{fulllineitems}


\end{fulllineitems}


\end{fulllineitems}

\phantomsection\label{\detokenize{reference/javascript_api:module-web.search_filters}}

\begin{fulllineitems}
\phantomsection\label{\detokenize{reference/javascript_api:web.search_filters}}\pysigline{\sphinxbfcode{\sphinxupquote{module }}\sphinxbfcode{\sphinxupquote{web.search\_filters}}}~~\begin{quote}\begin{description}
\item[{Exports}] \leavevmode{\hyperref[\detokenize{reference/javascript_api:web.search_filters.}]{\sphinxcrossref{
\textless{}anonymous\textgreater{}
}}}
\item[{Depends On}] \leavevmode\begin{itemize}
\item {} {\hyperref[\detokenize{reference/javascript_api:web.Widget}]{\sphinxcrossref{
web.Widget
}}}
\item {} {\hyperref[\detokenize{reference/javascript_api:web.core}]{\sphinxcrossref{
web.core
}}}
\item {} {\hyperref[\detokenize{reference/javascript_api:web.datepicker}]{\sphinxcrossref{
web.datepicker
}}}
\item {} {\hyperref[\detokenize{reference/javascript_api:web.field_utils}]{\sphinxcrossref{
web.field\_utils
}}}
\end{itemize}

\end{description}\end{quote}


\begin{fulllineitems}
\phantomsection\label{\detokenize{reference/javascript_api:web.search_filters.}}\pysigline{\sphinxbfcode{\sphinxupquote{namespace }}\sphinxbfcode{\sphinxupquote{}}}
\end{fulllineitems}


\end{fulllineitems}

\phantomsection\label{\detokenize{reference/javascript_api:module-web.Context}}

\begin{fulllineitems}
\phantomsection\label{\detokenize{reference/javascript_api:web.Context}}\pysigline{\sphinxbfcode{\sphinxupquote{module }}\sphinxbfcode{\sphinxupquote{web.Context}}}~~\begin{quote}\begin{description}
\item[{Exports}] \leavevmode{\hyperref[\detokenize{reference/javascript_api:web.Context.Context}]{\sphinxcrossref{
Context
}}}
\item[{Depends On}] \leavevmode\begin{itemize}
\item {} {\hyperref[\detokenize{reference/javascript_api:web.Class}]{\sphinxcrossref{
web.Class
}}}
\item {} {\hyperref[\detokenize{reference/javascript_api:web.pyeval}]{\sphinxcrossref{
web.pyeval
}}}
\end{itemize}

\end{description}\end{quote}


\begin{fulllineitems}
\phantomsection\label{\detokenize{reference/javascript_api:Context}}\pysiglinewithargsret{\sphinxbfcode{\sphinxupquote{class }}\sphinxbfcode{\sphinxupquote{Context}}}{}{}~\begin{quote}\begin{description}
\item[{Extends}] \leavevmode{\hyperref[\detokenize{reference/javascript_api:web.Class.Class}]{\sphinxcrossref{
Class
}}}
\end{description}\end{quote}


\begin{fulllineitems}
\phantomsection\label{\detokenize{reference/javascript_api:set_eval_context}}\pysiglinewithargsret{\sphinxbfcode{\sphinxupquote{method }}\sphinxbfcode{\sphinxupquote{set\_eval\_context}}}{\emph{evalContext}}{{ $\rightarrow$ Context}}
Set the evaluation context to be used when we actually eval.
\begin{quote}\begin{description}
\item[{Parameters}] \leavevmode\begin{itemize}

\sphinxstylestrong{evalContext} (\sphinxstyleliteralemphasis{\sphinxupquote{Object}})
\end{itemize}

\item[{Return Type}] \leavevmode
{\hyperref[\detokenize{reference/javascript_api:web.Context.Context}]{\sphinxcrossref{\sphinxstyleliteralemphasis{\sphinxupquote{Context}}}}}

\end{description}\end{quote}

\end{fulllineitems}


\end{fulllineitems}


\end{fulllineitems}

\phantomsection\label{\detokenize{reference/javascript_api:module-pad.pad}}

\begin{fulllineitems}
\phantomsection\label{\detokenize{reference/javascript_api:pad.pad}}\pysigline{\sphinxbfcode{\sphinxupquote{module }}\sphinxbfcode{\sphinxupquote{pad.pad}}}~~\begin{quote}\begin{description}
\item[{Exports}] \leavevmode{\hyperref[\detokenize{reference/javascript_api:pad.pad.FieldPad}]{\sphinxcrossref{
FieldPad
}}}
\item[{Depends On}] \leavevmode\begin{itemize}
\item {} {\hyperref[\detokenize{reference/javascript_api:web.AbstractField}]{\sphinxcrossref{
web.AbstractField
}}}
\item {} {\hyperref[\detokenize{reference/javascript_api:web.core}]{\sphinxcrossref{
web.core
}}}
\item {} {\hyperref[\detokenize{reference/javascript_api:web.field_registry}]{\sphinxcrossref{
web.field\_registry
}}}
\end{itemize}

\end{description}\end{quote}


\begin{fulllineitems}
\phantomsection\label{\detokenize{reference/javascript_api:FieldPad}}\pysiglinewithargsret{\sphinxbfcode{\sphinxupquote{class }}\sphinxbfcode{\sphinxupquote{FieldPad}}}{}{}~\begin{quote}\begin{description}
\item[{Extends}] \leavevmode{\hyperref[\detokenize{reference/javascript_api:web.AbstractField.AbstractField}]{\sphinxcrossref{
AbstractField
}}}
\end{description}\end{quote}


\begin{fulllineitems}
\phantomsection\label{\detokenize{reference/javascript_api:commitChanges}}\pysiglinewithargsret{\sphinxbfcode{\sphinxupquote{method }}\sphinxbfcode{\sphinxupquote{commitChanges}}}{}{}
If we had to generate an url, we wait for the generation to be completed,
so the current record will be associated with the correct pad url.

\end{fulllineitems}


\end{fulllineitems}


\end{fulllineitems}

\phantomsection\label{\detokenize{reference/javascript_api:module-web.AbstractWebClient}}

\begin{fulllineitems}
\phantomsection\label{\detokenize{reference/javascript_api:web.AbstractWebClient}}\pysigline{\sphinxbfcode{\sphinxupquote{module }}\sphinxbfcode{\sphinxupquote{web.AbstractWebClient}}}~~\begin{quote}\begin{description}
\item[{Exports}] \leavevmode{\hyperref[\detokenize{reference/javascript_api:web.AbstractWebClient.AbstractWebClient}]{\sphinxcrossref{
AbstractWebClient
}}}
\item[{Depends On}] \leavevmode\begin{itemize}
\item {} {\hyperref[\detokenize{reference/javascript_api:web.ActionManager}]{\sphinxcrossref{
web.ActionManager
}}}
\item {} {\hyperref[\detokenize{reference/javascript_api:web.Dialog}]{\sphinxcrossref{
web.Dialog
}}}
\item {} {\hyperref[\detokenize{reference/javascript_api:web.Loading}]{\sphinxcrossref{
web.Loading
}}}
\item {} {\hyperref[\detokenize{reference/javascript_api:web.Widget}]{\sphinxcrossref{
web.Widget
}}}
\item {} {\hyperref[\detokenize{reference/javascript_api:web.concurrency}]{\sphinxcrossref{
web.concurrency
}}}
\item {} {\hyperref[\detokenize{reference/javascript_api:web.config}]{\sphinxcrossref{
web.config
}}}
\item {} {\hyperref[\detokenize{reference/javascript_api:web.core}]{\sphinxcrossref{
web.core
}}}
\item {} {\hyperref[\detokenize{reference/javascript_api:web.crash_manager}]{\sphinxcrossref{
web.crash\_manager
}}}
\item {} {\hyperref[\detokenize{reference/javascript_api:web.data_manager}]{\sphinxcrossref{
web.data\_manager
}}}
\item {} {\hyperref[\detokenize{reference/javascript_api:web.mixins}]{\sphinxcrossref{
web.mixins
}}}
\item {} {\hyperref[\detokenize{reference/javascript_api:web.notification}]{\sphinxcrossref{
web.notification
}}}
\item {} {\hyperref[\detokenize{reference/javascript_api:web.rainbow_man}]{\sphinxcrossref{
web.rainbow\_man
}}}
\item {} {\hyperref[\detokenize{reference/javascript_api:web.session}]{\sphinxcrossref{
web.session
}}}
\end{itemize}

\end{description}\end{quote}


\begin{fulllineitems}
\phantomsection\label{\detokenize{reference/javascript_api:AbstractWebClient}}\pysiglinewithargsret{\sphinxbfcode{\sphinxupquote{class }}\sphinxbfcode{\sphinxupquote{AbstractWebClient}}}{\emph{parent}}{}~\begin{quote}\begin{description}
\item[{Extends}] \leavevmode{\hyperref[\detokenize{reference/javascript_api:web.Widget.Widget}]{\sphinxcrossref{
Widget
}}}
\item[{Mixes}] \leavevmode\begin{itemize}
\item {} 
ServiceProvider

\end{itemize}

\item[{Parameters}] \leavevmode\begin{itemize}

\sphinxstylestrong{parent}
\end{itemize}

\end{description}\end{quote}


\begin{fulllineitems}
\phantomsection\label{\detokenize{reference/javascript_api:set_title}}\pysiglinewithargsret{\sphinxbfcode{\sphinxupquote{method }}\sphinxbfcode{\sphinxupquote{set\_title}}}{\emph{title}}{}
Sets the first part of the title of the window, dedicated to the current action.
\begin{quote}\begin{description}
\item[{Parameters}] \leavevmode\begin{itemize}

\sphinxstylestrong{title}
\end{itemize}

\end{description}\end{quote}

\end{fulllineitems}



\begin{fulllineitems}
\phantomsection\label{\detokenize{reference/javascript_api:set_title_part}}\pysiglinewithargsret{\sphinxbfcode{\sphinxupquote{method }}\sphinxbfcode{\sphinxupquote{set\_title\_part}}}{\emph{part}, \emph{title}}{}
Sets an arbitrary part of the title of the window. Title parts are
identified by strings. Each time a title part is changed, all parts
are gathered, ordered by alphabetical order and displayed in the title
of the window separated by \sphinxcode{\sphinxupquote{-}}.
\begin{quote}\begin{description}
\item[{Parameters}] \leavevmode\begin{itemize}

\sphinxstylestrong{part}

\sphinxstylestrong{title}
\end{itemize}

\end{description}\end{quote}

\end{fulllineitems}



\begin{fulllineitems}
\phantomsection\label{\detokenize{reference/javascript_api:do_action}}\pysiglinewithargsret{\sphinxbfcode{\sphinxupquote{method }}\sphinxbfcode{\sphinxupquote{do\_action}}}{}{}
When do\_action is performed on the WebClient, forward it to the main ActionManager
This allows to widgets that are not inside the ActionManager to perform do\_action

\end{fulllineitems}


\end{fulllineitems}


\end{fulllineitems}

\phantomsection\label{\detokenize{reference/javascript_api:module-web_editor.snippets.options}}

\begin{fulllineitems}
\phantomsection\label{\detokenize{reference/javascript_api:web_editor.snippets.options}}\pysigline{\sphinxbfcode{\sphinxupquote{module }}\sphinxbfcode{\sphinxupquote{web\_editor.snippets.options}}}~~\begin{quote}\begin{description}
\item[{Exports}] \leavevmode{\hyperref[\detokenize{reference/javascript_api:web_editor.snippets.options.}]{\sphinxcrossref{
\textless{}anonymous\textgreater{}
}}}
\item[{Depends On}] \leavevmode\begin{itemize}
\item {} {\hyperref[\detokenize{reference/javascript_api:web.Dialog}]{\sphinxcrossref{
web.Dialog
}}}
\item {} {\hyperref[\detokenize{reference/javascript_api:web.Widget}]{\sphinxcrossref{
web.Widget
}}}
\item {} {\hyperref[\detokenize{reference/javascript_api:web.core}]{\sphinxcrossref{
web.core
}}}
\item {} {\hyperref[\detokenize{reference/javascript_api:web_editor.context}]{\sphinxcrossref{
web\_editor.context
}}}
\item {} {\hyperref[\detokenize{reference/javascript_api:web_editor.widget}]{\sphinxcrossref{
web\_editor.widget
}}}
\end{itemize}

\end{description}\end{quote}


\begin{fulllineitems}
\phantomsection\label{\detokenize{reference/javascript_api:SnippetOption}}\pysiglinewithargsret{\sphinxbfcode{\sphinxupquote{class }}\sphinxbfcode{\sphinxupquote{SnippetOption}}}{\emph{parent}, \emph{\$target}, \emph{\$overlay}, \emph{data}}{}~\begin{quote}\begin{description}
\item[{Extends}] \leavevmode{\hyperref[\detokenize{reference/javascript_api:web.Widget.Widget}]{\sphinxcrossref{
Widget
}}}
\item[{Parameters}] \leavevmode\begin{itemize}

\sphinxstylestrong{parent}

\sphinxstylestrong{\$target}

\sphinxstylestrong{\$overlay}

\sphinxstylestrong{data}
\end{itemize}

\end{description}\end{quote}

Handles a set of options for one snippet. The registry returned by this
module contains the names of the specialized SnippetOption which can be
referenced thanks to the data-js key in the web\_editor options template.


\begin{fulllineitems}
\phantomsection\label{\detokenize{reference/javascript_api:preventChildPropagation}}\pysigline{\sphinxbfcode{\sphinxupquote{attribute }}\sphinxbfcode{\sphinxupquote{preventChildPropagation}} Boolean}
When editing a snippet, its options are shown alongside the ones of its
parent snippets. The parent options are only shown if the following flag
is set to false (default).

\end{fulllineitems}



\begin{fulllineitems}
\phantomsection\label{\detokenize{reference/javascript_api:start}}\pysiglinewithargsret{\sphinxbfcode{\sphinxupquote{method }}\sphinxbfcode{\sphinxupquote{start}}}{}{}
Called when the option is initialized (i.e. the parent edition overlay is
shown for the first time).

\end{fulllineitems}



\begin{fulllineitems}
\phantomsection\label{\detokenize{reference/javascript_api:onFocus}}\pysiglinewithargsret{\sphinxbfcode{\sphinxupquote{method }}\sphinxbfcode{\sphinxupquote{onFocus}}}{}{}
Called when the parent edition overlay is covering the associated snippet
(the first time, this follows the call to the @see start method).

\end{fulllineitems}



\begin{fulllineitems}
\phantomsection\label{\detokenize{reference/javascript_api:onBuilt}}\pysiglinewithargsret{\sphinxbfcode{\sphinxupquote{method }}\sphinxbfcode{\sphinxupquote{onBuilt}}}{}{}
Called when the parent edition overlay is covering the associated snippet
for the first time, when it is a new snippet dropped from the d\&d snippet
menu. Note: this is called after the start and onFocus methods.

\end{fulllineitems}



\begin{fulllineitems}
\phantomsection\label{\detokenize{reference/javascript_api:onBlur}}\pysiglinewithargsret{\sphinxbfcode{\sphinxupquote{method }}\sphinxbfcode{\sphinxupquote{onBlur}}}{}{}
Called when the parent edition overlay is removed from the associated
snippet (another snippet enters edition for example).

\end{fulllineitems}



\begin{fulllineitems}
\phantomsection\label{\detokenize{reference/javascript_api:onClone}}\pysiglinewithargsret{\sphinxbfcode{\sphinxupquote{method }}\sphinxbfcode{\sphinxupquote{onClone}}}{\emph{options}}{}
Called when the associated snippet is the result of the cloning of
another snippet (so \sphinxcode{\sphinxupquote{this.\$target}} is a cloned element).
\begin{quote}\begin{description}
\item[{Parameters}] \leavevmode\begin{itemize}

\sphinxstylestrong{options} ({\hyperref[\detokenize{reference/javascript_api:web_editor.snippets.options.OnCloneOptions}]{\sphinxcrossref{\sphinxstyleliteralemphasis{\sphinxupquote{OnCloneOptions}}}}})
\end{itemize}

\end{description}\end{quote}


\begin{fulllineitems}
\phantomsection\label{\detokenize{reference/javascript_api:OnCloneOptions}}\pysiglinewithargsret{\sphinxbfcode{\sphinxupquote{class }}\sphinxbfcode{\sphinxupquote{OnCloneOptions}}}{}{}~

\begin{fulllineitems}
\phantomsection\label{\detokenize{reference/javascript_api:isCurrent}}\pysigline{\sphinxbfcode{\sphinxupquote{attribute }}\sphinxbfcode{\sphinxupquote{isCurrent}} boolean}~\begin{description}
\item[{true if the associated snippet is a clone of the main element that}] \leavevmode
was cloned (so not a clone of a child of this main element that
was cloned)

\end{description}

\end{fulllineitems}


\end{fulllineitems}


\end{fulllineitems}



\begin{fulllineitems}
\phantomsection\label{\detokenize{reference/javascript_api:onMove}}\pysiglinewithargsret{\sphinxbfcode{\sphinxupquote{function }}\sphinxbfcode{\sphinxupquote{onMove}}}{}{}
Called when the associated snippet is moved to another DOM location.

\end{fulllineitems}



\begin{fulllineitems}
\phantomsection\label{\detokenize{reference/javascript_api:onRemove}}\pysiglinewithargsret{\sphinxbfcode{\sphinxupquote{function }}\sphinxbfcode{\sphinxupquote{onRemove}}}{}{}
Called when the associated snippet is about to be removed from the DOM.

\end{fulllineitems}



\begin{fulllineitems}
\phantomsection\label{\detokenize{reference/javascript_api:cleanForSave}}\pysiglinewithargsret{\sphinxbfcode{\sphinxupquote{function }}\sphinxbfcode{\sphinxupquote{cleanForSave}}}{}{}
Called when the template which contains the associated snippet is about
to be saved.

\end{fulllineitems}



\begin{fulllineitems}
\phantomsection\label{\detokenize{reference/javascript_api:selectClass}}\pysiglinewithargsret{\sphinxbfcode{\sphinxupquote{function }}\sphinxbfcode{\sphinxupquote{selectClass}}}{\emph{previewMode}, \emph{value}, \emph{\$li}}{}
Default option method which allows to select one and only one class in
the option classes set and set it on the associated snippet. The common
case is having a subdropdown with each \textless{}li/\textgreater{} having a \sphinxcode{\sphinxupquote{data-select-class}}
value allowing to choose the associated class.
\begin{quote}\begin{description}
\item[{Parameters}] \leavevmode\begin{itemize}

\sphinxstylestrong{previewMode} (\sphinxstyleliteralemphasis{\sphinxupquote{boolean}}\sphinxstyleemphasis{ or }\sphinxstyleliteralemphasis{\sphinxupquote{string}}) \textendash{} truthy if the option is enabled for preview or if leaving it (in
         that second case, the value is ‘reset’)
       - false if the option should be activated for good

\sphinxstylestrong{value} (\sphinxstyleliteralemphasis{\sphinxupquote{any}}) \textendash{} the class to activate (\$li.data(‘selectClass’))

\sphinxstylestrong{\$li} (\sphinxstyleliteralemphasis{\sphinxupquote{jQuery}}) \textendash{} the related DOMElement option
\end{itemize}

\end{description}\end{quote}

\end{fulllineitems}



\begin{fulllineitems}
\phantomsection\label{\detokenize{reference/javascript_api:toggleClass}}\pysiglinewithargsret{\sphinxbfcode{\sphinxupquote{function }}\sphinxbfcode{\sphinxupquote{toggleClass}}}{\emph{previewMode}, \emph{value}, \emph{\$li}}{}
Default option method which allows to select one or multiple classes in
the option classes set and set it on the associated snippet. The common
case is having a subdropdown with each \textless{}li/\textgreater{} having a \sphinxcode{\sphinxupquote{data-toggle-class}}
value allowing to toggle the associated class.
\begin{quote}\begin{description}
\item[{Parameters}] \leavevmode\begin{itemize}

\sphinxstylestrong{previewMode}

\sphinxstylestrong{value}

\sphinxstylestrong{\$li}
\end{itemize}

\end{description}\end{quote}

\end{fulllineitems}



\begin{fulllineitems}
\phantomsection\label{\detokenize{reference/javascript_api:_}}\pysiglinewithargsret{\sphinxbfcode{\sphinxupquote{function }}\sphinxbfcode{\sphinxupquote{\$}}}{}{}
Override the helper method to search inside the \$target element instead
of the dropdown \textless{}li/\textgreater{} element.

\end{fulllineitems}



\begin{fulllineitems}
\phantomsection\label{\detokenize{reference/javascript_api:notify}}\pysiglinewithargsret{\sphinxbfcode{\sphinxupquote{function }}\sphinxbfcode{\sphinxupquote{notify}}}{\emph{name}, \emph{data}}{}
Sometimes, options may need to notify other options, even in parent
editors. This can be done thanks to the ‘option\_update’ event, which
will then be handled by this function.
\begin{quote}\begin{description}
\item[{Parameters}] \leavevmode\begin{itemize}

\sphinxstylestrong{name} (\sphinxstyleliteralemphasis{\sphinxupquote{string}}) \textendash{} an identifier for a type of update

\sphinxstylestrong{data} (\sphinxstyleliteralemphasis{\sphinxupquote{any}})
\end{itemize}

\end{description}\end{quote}

\end{fulllineitems}



\begin{fulllineitems}
\phantomsection\label{\detokenize{reference/javascript_api:setTarget}}\pysiglinewithargsret{\sphinxbfcode{\sphinxupquote{function }}\sphinxbfcode{\sphinxupquote{setTarget}}}{\emph{\$target}}{}
Sometimes, an option is binded on an element but should in fact apply on
another one. For example, elements which contain slides: we want all the
per-slide options to be in the main menu of the whole snippet. This
function allows to set the option’s target.
\begin{quote}\begin{description}
\item[{Parameters}] \leavevmode\begin{itemize}

\sphinxstylestrong{\$target} (\sphinxstyleliteralemphasis{\sphinxupquote{jQuery}}) \textendash{} the new target element
\end{itemize}

\end{description}\end{quote}

\end{fulllineitems}


\end{fulllineitems}



\begin{fulllineitems}
\phantomsection\label{\detokenize{reference/javascript_api:web_editor.snippets.options.}}\pysigline{\sphinxbfcode{\sphinxupquote{namespace }}\sphinxbfcode{\sphinxupquote{}}}~

\begin{fulllineitems}
\phantomsection\label{\detokenize{reference/javascript_api:SnippetOption}}\pysiglinewithargsret{\sphinxbfcode{\sphinxupquote{class }}\sphinxbfcode{\sphinxupquote{SnippetOption}}}{\emph{parent}, \emph{\$target}, \emph{\$overlay}, \emph{data}}{}~\begin{quote}\begin{description}
\item[{Extends}] \leavevmode{\hyperref[\detokenize{reference/javascript_api:web.Widget.Widget}]{\sphinxcrossref{
Widget
}}}
\item[{Parameters}] \leavevmode\begin{itemize}

\sphinxstylestrong{parent}

\sphinxstylestrong{\$target}

\sphinxstylestrong{\$overlay}

\sphinxstylestrong{data}
\end{itemize}

\end{description}\end{quote}

Handles a set of options for one snippet. The registry returned by this
module contains the names of the specialized SnippetOption which can be
referenced thanks to the data-js key in the web\_editor options template.


\begin{fulllineitems}
\phantomsection\label{\detokenize{reference/javascript_api:preventChildPropagation}}\pysigline{\sphinxbfcode{\sphinxupquote{attribute }}\sphinxbfcode{\sphinxupquote{preventChildPropagation}} Boolean}
When editing a snippet, its options are shown alongside the ones of its
parent snippets. The parent options are only shown if the following flag
is set to false (default).

\end{fulllineitems}



\begin{fulllineitems}
\phantomsection\label{\detokenize{reference/javascript_api:start}}\pysiglinewithargsret{\sphinxbfcode{\sphinxupquote{method }}\sphinxbfcode{\sphinxupquote{start}}}{}{}
Called when the option is initialized (i.e. the parent edition overlay is
shown for the first time).

\end{fulllineitems}



\begin{fulllineitems}
\phantomsection\label{\detokenize{reference/javascript_api:onFocus}}\pysiglinewithargsret{\sphinxbfcode{\sphinxupquote{method }}\sphinxbfcode{\sphinxupquote{onFocus}}}{}{}
Called when the parent edition overlay is covering the associated snippet
(the first time, this follows the call to the @see start method).

\end{fulllineitems}



\begin{fulllineitems}
\phantomsection\label{\detokenize{reference/javascript_api:onBuilt}}\pysiglinewithargsret{\sphinxbfcode{\sphinxupquote{method }}\sphinxbfcode{\sphinxupquote{onBuilt}}}{}{}
Called when the parent edition overlay is covering the associated snippet
for the first time, when it is a new snippet dropped from the d\&d snippet
menu. Note: this is called after the start and onFocus methods.

\end{fulllineitems}



\begin{fulllineitems}
\phantomsection\label{\detokenize{reference/javascript_api:onBlur}}\pysiglinewithargsret{\sphinxbfcode{\sphinxupquote{method }}\sphinxbfcode{\sphinxupquote{onBlur}}}{}{}
Called when the parent edition overlay is removed from the associated
snippet (another snippet enters edition for example).

\end{fulllineitems}



\begin{fulllineitems}
\phantomsection\label{\detokenize{reference/javascript_api:onClone}}\pysiglinewithargsret{\sphinxbfcode{\sphinxupquote{method }}\sphinxbfcode{\sphinxupquote{onClone}}}{\emph{options}}{}
Called when the associated snippet is the result of the cloning of
another snippet (so \sphinxcode{\sphinxupquote{this.\$target}} is a cloned element).
\begin{quote}\begin{description}
\item[{Parameters}] \leavevmode\begin{itemize}

\sphinxstylestrong{options} ({\hyperref[\detokenize{reference/javascript_api:web_editor.snippets.options.OnCloneOptions}]{\sphinxcrossref{\sphinxstyleliteralemphasis{\sphinxupquote{OnCloneOptions}}}}})
\end{itemize}

\end{description}\end{quote}


\begin{fulllineitems}
\phantomsection\label{\detokenize{reference/javascript_api:OnCloneOptions}}\pysiglinewithargsret{\sphinxbfcode{\sphinxupquote{class }}\sphinxbfcode{\sphinxupquote{OnCloneOptions}}}{}{}~

\begin{fulllineitems}
\phantomsection\label{\detokenize{reference/javascript_api:isCurrent}}\pysigline{\sphinxbfcode{\sphinxupquote{attribute }}\sphinxbfcode{\sphinxupquote{isCurrent}} boolean}~\begin{description}
\item[{true if the associated snippet is a clone of the main element that}] \leavevmode
was cloned (so not a clone of a child of this main element that
was cloned)

\end{description}

\end{fulllineitems}


\end{fulllineitems}


\end{fulllineitems}



\begin{fulllineitems}
\phantomsection\label{\detokenize{reference/javascript_api:onMove}}\pysiglinewithargsret{\sphinxbfcode{\sphinxupquote{function }}\sphinxbfcode{\sphinxupquote{onMove}}}{}{}
Called when the associated snippet is moved to another DOM location.

\end{fulllineitems}



\begin{fulllineitems}
\phantomsection\label{\detokenize{reference/javascript_api:onRemove}}\pysiglinewithargsret{\sphinxbfcode{\sphinxupquote{function }}\sphinxbfcode{\sphinxupquote{onRemove}}}{}{}
Called when the associated snippet is about to be removed from the DOM.

\end{fulllineitems}



\begin{fulllineitems}
\phantomsection\label{\detokenize{reference/javascript_api:cleanForSave}}\pysiglinewithargsret{\sphinxbfcode{\sphinxupquote{function }}\sphinxbfcode{\sphinxupquote{cleanForSave}}}{}{}
Called when the template which contains the associated snippet is about
to be saved.

\end{fulllineitems}



\begin{fulllineitems}
\phantomsection\label{\detokenize{reference/javascript_api:selectClass}}\pysiglinewithargsret{\sphinxbfcode{\sphinxupquote{function }}\sphinxbfcode{\sphinxupquote{selectClass}}}{\emph{previewMode}, \emph{value}, \emph{\$li}}{}
Default option method which allows to select one and only one class in
the option classes set and set it on the associated snippet. The common
case is having a subdropdown with each \textless{}li/\textgreater{} having a \sphinxcode{\sphinxupquote{data-select-class}}
value allowing to choose the associated class.
\begin{quote}\begin{description}
\item[{Parameters}] \leavevmode\begin{itemize}

\sphinxstylestrong{previewMode} (\sphinxstyleliteralemphasis{\sphinxupquote{boolean}}\sphinxstyleemphasis{ or }\sphinxstyleliteralemphasis{\sphinxupquote{string}}) \textendash{} truthy if the option is enabled for preview or if leaving it (in
         that second case, the value is ‘reset’)
       - false if the option should be activated for good

\sphinxstylestrong{value} (\sphinxstyleliteralemphasis{\sphinxupquote{any}}) \textendash{} the class to activate (\$li.data(‘selectClass’))

\sphinxstylestrong{\$li} (\sphinxstyleliteralemphasis{\sphinxupquote{jQuery}}) \textendash{} the related DOMElement option
\end{itemize}

\end{description}\end{quote}

\end{fulllineitems}



\begin{fulllineitems}
\phantomsection\label{\detokenize{reference/javascript_api:toggleClass}}\pysiglinewithargsret{\sphinxbfcode{\sphinxupquote{function }}\sphinxbfcode{\sphinxupquote{toggleClass}}}{\emph{previewMode}, \emph{value}, \emph{\$li}}{}
Default option method which allows to select one or multiple classes in
the option classes set and set it on the associated snippet. The common
case is having a subdropdown with each \textless{}li/\textgreater{} having a \sphinxcode{\sphinxupquote{data-toggle-class}}
value allowing to toggle the associated class.
\begin{quote}\begin{description}
\item[{Parameters}] \leavevmode\begin{itemize}

\sphinxstylestrong{previewMode}

\sphinxstylestrong{value}

\sphinxstylestrong{\$li}
\end{itemize}

\end{description}\end{quote}

\end{fulllineitems}



\begin{fulllineitems}
\phantomsection\label{\detokenize{reference/javascript_api:_}}\pysiglinewithargsret{\sphinxbfcode{\sphinxupquote{function }}\sphinxbfcode{\sphinxupquote{\$}}}{}{}
Override the helper method to search inside the \$target element instead
of the dropdown \textless{}li/\textgreater{} element.

\end{fulllineitems}



\begin{fulllineitems}
\phantomsection\label{\detokenize{reference/javascript_api:notify}}\pysiglinewithargsret{\sphinxbfcode{\sphinxupquote{function }}\sphinxbfcode{\sphinxupquote{notify}}}{\emph{name}, \emph{data}}{}
Sometimes, options may need to notify other options, even in parent
editors. This can be done thanks to the ‘option\_update’ event, which
will then be handled by this function.
\begin{quote}\begin{description}
\item[{Parameters}] \leavevmode\begin{itemize}

\sphinxstylestrong{name} (\sphinxstyleliteralemphasis{\sphinxupquote{string}}) \textendash{} an identifier for a type of update

\sphinxstylestrong{data} (\sphinxstyleliteralemphasis{\sphinxupquote{any}})
\end{itemize}

\end{description}\end{quote}

\end{fulllineitems}



\begin{fulllineitems}
\phantomsection\label{\detokenize{reference/javascript_api:setTarget}}\pysiglinewithargsret{\sphinxbfcode{\sphinxupquote{function }}\sphinxbfcode{\sphinxupquote{setTarget}}}{\emph{\$target}}{}
Sometimes, an option is binded on an element but should in fact apply on
another one. For example, elements which contain slides: we want all the
per-slide options to be in the main menu of the whole snippet. This
function allows to set the option’s target.
\begin{quote}\begin{description}
\item[{Parameters}] \leavevmode\begin{itemize}

\sphinxstylestrong{\$target} (\sphinxstyleliteralemphasis{\sphinxupquote{jQuery}}) \textendash{} the new target element
\end{itemize}

\end{description}\end{quote}

\end{fulllineitems}


\end{fulllineitems}



\begin{fulllineitems}
\phantomsection\label{\detokenize{reference/javascript_api:registry}}\pysigline{\sphinxbfcode{\sphinxupquote{namespace }}\sphinxbfcode{\sphinxupquote{registry}}}
The registry object contains the list of available options.


\begin{fulllineitems}
\phantomsection\label{\detokenize{reference/javascript_api:margin-y}}\pysiglinewithargsret{\sphinxbfcode{\sphinxupquote{class }}\sphinxbfcode{\sphinxupquote{margin-y}}}{}{}~\begin{quote}\begin{description}
\item[{Extends}] \leavevmode
marginAndResize

\end{description}\end{quote}

Handles the edition of margin-top and margin-bottom.

\end{fulllineitems}



\begin{fulllineitems}
\phantomsection\label{\detokenize{reference/javascript_api:resize}}\pysiglinewithargsret{\sphinxbfcode{\sphinxupquote{class }}\sphinxbfcode{\sphinxupquote{resize}}}{}{}~\begin{quote}\begin{description}
\item[{Extends}] \leavevmode
marginAndResize

\end{description}\end{quote}

Handles the edition of snippet’s height.

\end{fulllineitems}



\begin{fulllineitems}
\phantomsection\label{\detokenize{reference/javascript_api:colorpicker}}\pysiglinewithargsret{\sphinxbfcode{\sphinxupquote{class }}\sphinxbfcode{\sphinxupquote{colorpicker}}}{}{}~\begin{quote}\begin{description}
\item[{Extends}] \leavevmode{\hyperref[\detokenize{reference/javascript_api:web_editor.snippets.options.SnippetOption}]{\sphinxcrossref{
SnippetOption
}}}
\end{description}\end{quote}

Handles the edition of snippet’s background color classes.

\end{fulllineitems}



\begin{fulllineitems}
\phantomsection\label{\detokenize{reference/javascript_api:pos_background}}\pysiglinewithargsret{\sphinxbfcode{\sphinxupquote{class }}\sphinxbfcode{\sphinxupquote{pos\_background}}}{}{}~\begin{quote}\begin{description}
\item[{Extends}] \leavevmode{\hyperref[\detokenize{reference/javascript_api:web_editor.snippets.options.SnippetOption}]{\sphinxcrossref{
SnippetOption
}}}
\end{description}\end{quote}

Handles the edition of snippet’s background image.


\begin{fulllineitems}
\phantomsection\label{\detokenize{reference/javascript_api:background}}\pysiglinewithargsret{\sphinxbfcode{\sphinxupquote{method }}\sphinxbfcode{\sphinxupquote{background}}}{\emph{previewMode}, \emph{value}, \emph{\$li}}{}
Handles a background change.
\begin{quote}\begin{description}
\item[{Parameters}] \leavevmode\begin{itemize}

\sphinxstylestrong{previewMode}

\sphinxstylestrong{value}

\sphinxstylestrong{\$li}
\end{itemize}

\end{description}\end{quote}

\end{fulllineitems}



\begin{fulllineitems}
\phantomsection\label{\detokenize{reference/javascript_api:chooseImage}}\pysiglinewithargsret{\sphinxbfcode{\sphinxupquote{method }}\sphinxbfcode{\sphinxupquote{chooseImage}}}{\emph{previewMode}, \emph{value}, \emph{\$li}}{}
Opens a media dialog to add a custom background image.
\begin{quote}\begin{description}
\item[{Parameters}] \leavevmode\begin{itemize}

\sphinxstylestrong{previewMode}

\sphinxstylestrong{value}

\sphinxstylestrong{\$li}
\end{itemize}

\end{description}\end{quote}

\end{fulllineitems}



\begin{fulllineitems}
\phantomsection\label{\detokenize{reference/javascript_api:bindBackgroundEvents}}\pysiglinewithargsret{\sphinxbfcode{\sphinxupquote{method }}\sphinxbfcode{\sphinxupquote{bindBackgroundEvents}}}{}{}
Attaches events so that when a background-color is set, the background
image is removed.

\end{fulllineitems}


\end{fulllineitems}



\begin{fulllineitems}
\phantomsection\label{\detokenize{reference/javascript_api:background_position}}\pysiglinewithargsret{\sphinxbfcode{\sphinxupquote{class }}\sphinxbfcode{\sphinxupquote{background\_position}}}{}{}~\begin{quote}\begin{description}
\item[{Extends}] \leavevmode{\hyperref[\detokenize{reference/javascript_api:web_editor.snippets.options.SnippetOption}]{\sphinxcrossref{
SnippetOption
}}}
\end{description}\end{quote}

Handles the edition of snippet’s background image position.


\begin{fulllineitems}
\phantomsection\label{\detokenize{reference/javascript_api:backgroundPosition}}\pysiglinewithargsret{\sphinxbfcode{\sphinxupquote{method }}\sphinxbfcode{\sphinxupquote{backgroundPosition}}}{\emph{previewMode}, \emph{value}, \emph{\$li}}{}
Opens a Dialog to edit the snippet’s backgroung image position.
\begin{quote}\begin{description}
\item[{Parameters}] \leavevmode\begin{itemize}

\sphinxstylestrong{previewMode}

\sphinxstylestrong{value}

\sphinxstylestrong{\$li}
\end{itemize}

\end{description}\end{quote}

\end{fulllineitems}


\end{fulllineitems}



\begin{fulllineitems}
\phantomsection\label{\detokenize{reference/javascript_api:many2one}}\pysiglinewithargsret{\sphinxbfcode{\sphinxupquote{class }}\sphinxbfcode{\sphinxupquote{many2one}}}{}{}~\begin{quote}\begin{description}
\item[{Extends}] \leavevmode{\hyperref[\detokenize{reference/javascript_api:web_editor.snippets.options.SnippetOption}]{\sphinxcrossref{
SnippetOption
}}}
\end{description}\end{quote}

Allows to replace a text value with the name of a database record.

\end{fulllineitems}



\begin{fulllineitems}
\phantomsection\label{\detokenize{reference/javascript_api:pos_background}}\pysiglinewithargsret{\sphinxbfcode{\sphinxupquote{class }}\sphinxbfcode{\sphinxupquote{pos\_background}}}{}{}~\begin{quote}\begin{description}
\item[{Extends}] \leavevmode{\hyperref[\detokenize{reference/javascript_api:web_editor.snippets.options.SnippetOption}]{\sphinxcrossref{
SnippetOption
}}}
\end{description}\end{quote}

Handles the edition of snippet’s background image.


\begin{fulllineitems}
\phantomsection\label{\detokenize{reference/javascript_api:background}}\pysiglinewithargsret{\sphinxbfcode{\sphinxupquote{method }}\sphinxbfcode{\sphinxupquote{background}}}{\emph{previewMode}, \emph{value}, \emph{\$li}}{}
Handles a background change.
\begin{quote}\begin{description}
\item[{Parameters}] \leavevmode\begin{itemize}

\sphinxstylestrong{previewMode}

\sphinxstylestrong{value}

\sphinxstylestrong{\$li}
\end{itemize}

\end{description}\end{quote}

\end{fulllineitems}



\begin{fulllineitems}
\phantomsection\label{\detokenize{reference/javascript_api:chooseImage}}\pysiglinewithargsret{\sphinxbfcode{\sphinxupquote{method }}\sphinxbfcode{\sphinxupquote{chooseImage}}}{\emph{previewMode}, \emph{value}, \emph{\$li}}{}
Opens a media dialog to add a custom background image.
\begin{quote}\begin{description}
\item[{Parameters}] \leavevmode\begin{itemize}

\sphinxstylestrong{previewMode}

\sphinxstylestrong{value}

\sphinxstylestrong{\$li}
\end{itemize}

\end{description}\end{quote}

\end{fulllineitems}



\begin{fulllineitems}
\phantomsection\label{\detokenize{reference/javascript_api:bindBackgroundEvents}}\pysiglinewithargsret{\sphinxbfcode{\sphinxupquote{method }}\sphinxbfcode{\sphinxupquote{bindBackgroundEvents}}}{}{}
Attaches events so that when a background-color is set, the background
image is removed.

\end{fulllineitems}


\end{fulllineitems}


\end{fulllineitems}


\end{fulllineitems}



\begin{fulllineitems}
\phantomsection\label{\detokenize{reference/javascript_api:registry}}\pysigline{\sphinxbfcode{\sphinxupquote{namespace }}\sphinxbfcode{\sphinxupquote{registry}}}
The registry object contains the list of available options.


\begin{fulllineitems}
\phantomsection\label{\detokenize{reference/javascript_api:margin-y}}\pysiglinewithargsret{\sphinxbfcode{\sphinxupquote{class }}\sphinxbfcode{\sphinxupquote{margin-y}}}{}{}~\begin{quote}\begin{description}
\item[{Extends}] \leavevmode
marginAndResize

\end{description}\end{quote}

Handles the edition of margin-top and margin-bottom.

\end{fulllineitems}



\begin{fulllineitems}
\phantomsection\label{\detokenize{reference/javascript_api:resize}}\pysiglinewithargsret{\sphinxbfcode{\sphinxupquote{class }}\sphinxbfcode{\sphinxupquote{resize}}}{}{}~\begin{quote}\begin{description}
\item[{Extends}] \leavevmode
marginAndResize

\end{description}\end{quote}

Handles the edition of snippet’s height.

\end{fulllineitems}



\begin{fulllineitems}
\phantomsection\label{\detokenize{reference/javascript_api:colorpicker}}\pysiglinewithargsret{\sphinxbfcode{\sphinxupquote{class }}\sphinxbfcode{\sphinxupquote{colorpicker}}}{}{}~\begin{quote}\begin{description}
\item[{Extends}] \leavevmode{\hyperref[\detokenize{reference/javascript_api:web_editor.snippets.options.SnippetOption}]{\sphinxcrossref{
SnippetOption
}}}
\end{description}\end{quote}

Handles the edition of snippet’s background color classes.

\end{fulllineitems}



\begin{fulllineitems}
\phantomsection\label{\detokenize{reference/javascript_api:pos_background}}\pysiglinewithargsret{\sphinxbfcode{\sphinxupquote{class }}\sphinxbfcode{\sphinxupquote{pos\_background}}}{}{}~\begin{quote}\begin{description}
\item[{Extends}] \leavevmode{\hyperref[\detokenize{reference/javascript_api:web_editor.snippets.options.SnippetOption}]{\sphinxcrossref{
SnippetOption
}}}
\end{description}\end{quote}

Handles the edition of snippet’s background image.


\begin{fulllineitems}
\phantomsection\label{\detokenize{reference/javascript_api:background}}\pysiglinewithargsret{\sphinxbfcode{\sphinxupquote{method }}\sphinxbfcode{\sphinxupquote{background}}}{\emph{previewMode}, \emph{value}, \emph{\$li}}{}
Handles a background change.
\begin{quote}\begin{description}
\item[{Parameters}] \leavevmode\begin{itemize}

\sphinxstylestrong{previewMode}

\sphinxstylestrong{value}

\sphinxstylestrong{\$li}
\end{itemize}

\end{description}\end{quote}

\end{fulllineitems}



\begin{fulllineitems}
\phantomsection\label{\detokenize{reference/javascript_api:chooseImage}}\pysiglinewithargsret{\sphinxbfcode{\sphinxupquote{method }}\sphinxbfcode{\sphinxupquote{chooseImage}}}{\emph{previewMode}, \emph{value}, \emph{\$li}}{}
Opens a media dialog to add a custom background image.
\begin{quote}\begin{description}
\item[{Parameters}] \leavevmode\begin{itemize}

\sphinxstylestrong{previewMode}

\sphinxstylestrong{value}

\sphinxstylestrong{\$li}
\end{itemize}

\end{description}\end{quote}

\end{fulllineitems}



\begin{fulllineitems}
\phantomsection\label{\detokenize{reference/javascript_api:bindBackgroundEvents}}\pysiglinewithargsret{\sphinxbfcode{\sphinxupquote{method }}\sphinxbfcode{\sphinxupquote{bindBackgroundEvents}}}{}{}
Attaches events so that when a background-color is set, the background
image is removed.

\end{fulllineitems}


\end{fulllineitems}



\begin{fulllineitems}
\phantomsection\label{\detokenize{reference/javascript_api:background_position}}\pysiglinewithargsret{\sphinxbfcode{\sphinxupquote{class }}\sphinxbfcode{\sphinxupquote{background\_position}}}{}{}~\begin{quote}\begin{description}
\item[{Extends}] \leavevmode{\hyperref[\detokenize{reference/javascript_api:web_editor.snippets.options.SnippetOption}]{\sphinxcrossref{
SnippetOption
}}}
\end{description}\end{quote}

Handles the edition of snippet’s background image position.


\begin{fulllineitems}
\phantomsection\label{\detokenize{reference/javascript_api:backgroundPosition}}\pysiglinewithargsret{\sphinxbfcode{\sphinxupquote{method }}\sphinxbfcode{\sphinxupquote{backgroundPosition}}}{\emph{previewMode}, \emph{value}, \emph{\$li}}{}
Opens a Dialog to edit the snippet’s backgroung image position.
\begin{quote}\begin{description}
\item[{Parameters}] \leavevmode\begin{itemize}

\sphinxstylestrong{previewMode}

\sphinxstylestrong{value}

\sphinxstylestrong{\$li}
\end{itemize}

\end{description}\end{quote}

\end{fulllineitems}


\end{fulllineitems}



\begin{fulllineitems}
\phantomsection\label{\detokenize{reference/javascript_api:many2one}}\pysiglinewithargsret{\sphinxbfcode{\sphinxupquote{class }}\sphinxbfcode{\sphinxupquote{many2one}}}{}{}~\begin{quote}\begin{description}
\item[{Extends}] \leavevmode{\hyperref[\detokenize{reference/javascript_api:web_editor.snippets.options.SnippetOption}]{\sphinxcrossref{
SnippetOption
}}}
\end{description}\end{quote}

Allows to replace a text value with the name of a database record.

\end{fulllineitems}



\begin{fulllineitems}
\phantomsection\label{\detokenize{reference/javascript_api:pos_background}}\pysiglinewithargsret{\sphinxbfcode{\sphinxupquote{class }}\sphinxbfcode{\sphinxupquote{pos\_background}}}{}{}~\begin{quote}\begin{description}
\item[{Extends}] \leavevmode{\hyperref[\detokenize{reference/javascript_api:web_editor.snippets.options.SnippetOption}]{\sphinxcrossref{
SnippetOption
}}}
\end{description}\end{quote}

Handles the edition of snippet’s background image.


\begin{fulllineitems}
\phantomsection\label{\detokenize{reference/javascript_api:background}}\pysiglinewithargsret{\sphinxbfcode{\sphinxupquote{method }}\sphinxbfcode{\sphinxupquote{background}}}{\emph{previewMode}, \emph{value}, \emph{\$li}}{}
Handles a background change.
\begin{quote}\begin{description}
\item[{Parameters}] \leavevmode\begin{itemize}

\sphinxstylestrong{previewMode}

\sphinxstylestrong{value}

\sphinxstylestrong{\$li}
\end{itemize}

\end{description}\end{quote}

\end{fulllineitems}



\begin{fulllineitems}
\phantomsection\label{\detokenize{reference/javascript_api:chooseImage}}\pysiglinewithargsret{\sphinxbfcode{\sphinxupquote{method }}\sphinxbfcode{\sphinxupquote{chooseImage}}}{\emph{previewMode}, \emph{value}, \emph{\$li}}{}
Opens a media dialog to add a custom background image.
\begin{quote}\begin{description}
\item[{Parameters}] \leavevmode\begin{itemize}

\sphinxstylestrong{previewMode}

\sphinxstylestrong{value}

\sphinxstylestrong{\$li}
\end{itemize}

\end{description}\end{quote}

\end{fulllineitems}



\begin{fulllineitems}
\phantomsection\label{\detokenize{reference/javascript_api:bindBackgroundEvents}}\pysiglinewithargsret{\sphinxbfcode{\sphinxupquote{method }}\sphinxbfcode{\sphinxupquote{bindBackgroundEvents}}}{}{}
Attaches events so that when a background-color is set, the background
image is removed.

\end{fulllineitems}


\end{fulllineitems}


\end{fulllineitems}


\end{fulllineitems}

\phantomsection\label{\detokenize{reference/javascript_api:module-mail.ChatWindow}}

\begin{fulllineitems}
\phantomsection\label{\detokenize{reference/javascript_api:mail.ChatWindow}}\pysigline{\sphinxbfcode{\sphinxupquote{module }}\sphinxbfcode{\sphinxupquote{mail.ChatWindow}}}~~\begin{quote}\begin{description}
\item[{Exports}] \leavevmode{\hyperref[\detokenize{reference/javascript_api:mail.ChatWindow.}]{\sphinxcrossref{
\textless{}anonymous\textgreater{}
}}}
\item[{Depends On}] \leavevmode\begin{itemize}
\item {} {\hyperref[\detokenize{reference/javascript_api:mail.ChatThread}]{\sphinxcrossref{
mail.ChatThread
}}}
\item {} {\hyperref[\detokenize{reference/javascript_api:web.Widget}]{\sphinxcrossref{
web.Widget
}}}
\item {} {\hyperref[\detokenize{reference/javascript_api:web.config}]{\sphinxcrossref{
web.config
}}}
\item {} {\hyperref[\detokenize{reference/javascript_api:web.core}]{\sphinxcrossref{
web.core
}}}
\end{itemize}

\end{description}\end{quote}


\begin{fulllineitems}
\phantomsection\label{\detokenize{reference/javascript_api:mail.ChatWindow.}}\pysiglinewithargsret{\sphinxbfcode{\sphinxupquote{class }}\sphinxbfcode{\sphinxupquote{}}}{\emph{parent}, \emph{channel\_id}, \emph{title}, \emph{is\_folded}, \emph{unread\_msgs}, \emph{options}}{}~\begin{quote}\begin{description}
\item[{Extends}] \leavevmode{\hyperref[\detokenize{reference/javascript_api:web.Widget.Widget}]{\sphinxcrossref{
Widget
}}}
\item[{Parameters}] \leavevmode\begin{itemize}

\sphinxstylestrong{parent}

\sphinxstylestrong{channel\_id}

\sphinxstylestrong{title}

\sphinxstylestrong{is\_folded}

\sphinxstylestrong{unread\_msgs}

\sphinxstylestrong{options}
\end{itemize}

\end{description}\end{quote}

\end{fulllineitems}


\end{fulllineitems}

\phantomsection\label{\detokenize{reference/javascript_api:module-barcodes.BarcodeEvents}}

\begin{fulllineitems}
\phantomsection\label{\detokenize{reference/javascript_api:barcodes.BarcodeEvents}}\pysigline{\sphinxbfcode{\sphinxupquote{module }}\sphinxbfcode{\sphinxupquote{barcodes.BarcodeEvents}}}~~\begin{quote}\begin{description}
\item[{Exports}] \leavevmode{\hyperref[\detokenize{reference/javascript_api:barcodes.BarcodeEvents.}]{\sphinxcrossref{
\textless{}anonymous\textgreater{}
}}}
\item[{Depends On}] \leavevmode\begin{itemize}
\item {} {\hyperref[\detokenize{reference/javascript_api:web.core}]{\sphinxcrossref{
web.core
}}}
\item {} {\hyperref[\detokenize{reference/javascript_api:web.mixins}]{\sphinxcrossref{
web.mixins
}}}
\item {} {\hyperref[\detokenize{reference/javascript_api:web.session}]{\sphinxcrossref{
web.session
}}}
\end{itemize}

\end{description}\end{quote}


\begin{fulllineitems}
\phantomsection\label{\detokenize{reference/javascript_api:barcodes.BarcodeEvents.}}\pysigline{\sphinxbfcode{\sphinxupquote{namespace }}\sphinxbfcode{\sphinxupquote{}}}~

\begin{fulllineitems}
\phantomsection\label{\detokenize{reference/javascript_api:BarcodeEvents}}\pysigline{\sphinxbfcode{\sphinxupquote{object }}\sphinxbfcode{\sphinxupquote{BarcodeEvents}}\sphinxbfcode{\sphinxupquote{ instance of }}{\hyperref[\detokenize{reference/javascript_api:barcodes.BarcodeEvents.BarcodeEvents}]{\sphinxcrossref{BarcodeEvents}}}}
Singleton that emits barcode\_scanned events on core.bus

\end{fulllineitems}



\begin{fulllineitems}
\phantomsection\label{\detokenize{reference/javascript_api:ReservedBarcodePrefixes}}\pysigline{\sphinxbfcode{\sphinxupquote{attribute }}\sphinxbfcode{\sphinxupquote{ReservedBarcodePrefixes}} Array}
List of barcode prefixes that are reserved for internal purposes

\end{fulllineitems}


\end{fulllineitems}


\end{fulllineitems}

\phantomsection\label{\detokenize{reference/javascript_api:module-web.ControlPanel}}

\begin{fulllineitems}
\phantomsection\label{\detokenize{reference/javascript_api:web.ControlPanel}}\pysigline{\sphinxbfcode{\sphinxupquote{module }}\sphinxbfcode{\sphinxupquote{web.ControlPanel}}}~~\begin{quote}\begin{description}
\item[{Exports}] \leavevmode{\hyperref[\detokenize{reference/javascript_api:web.ControlPanel.ControlPanel}]{\sphinxcrossref{
ControlPanel
}}}
\item[{Depends On}] \leavevmode\begin{itemize}
\item {} {\hyperref[\detokenize{reference/javascript_api:web.Bus}]{\sphinxcrossref{
web.Bus
}}}
\item {} {\hyperref[\detokenize{reference/javascript_api:web.Widget}]{\sphinxcrossref{
web.Widget
}}}
\item {} {\hyperref[\detokenize{reference/javascript_api:web.data}]{\sphinxcrossref{
web.data
}}}
\end{itemize}

\end{description}\end{quote}


\begin{fulllineitems}
\phantomsection\label{\detokenize{reference/javascript_api:ControlPanel}}\pysiglinewithargsret{\sphinxbfcode{\sphinxupquote{class }}\sphinxbfcode{\sphinxupquote{ControlPanel}}}{\emph{parent}\sphinxoptional{, \emph{template}}}{}~\begin{quote}\begin{description}
\item[{Extends}] \leavevmode{\hyperref[\detokenize{reference/javascript_api:web.Widget.Widget}]{\sphinxcrossref{
Widget
}}}
\item[{Parameters}] \leavevmode\begin{itemize}

\sphinxstylestrong{parent}

\sphinxstylestrong{template} (\sphinxstyleliteralemphasis{\sphinxupquote{String}}) \textendash{} the QWeb template to render the ControlPanel.
By default, the template ‘ControlPanel’ will be used
\end{itemize}

\end{description}\end{quote}


\begin{fulllineitems}
\phantomsection\label{\detokenize{reference/javascript_api:start}}\pysiglinewithargsret{\sphinxbfcode{\sphinxupquote{method }}\sphinxbfcode{\sphinxupquote{start}}}{}{{ $\rightarrow$ jQuery.Deferred}}
Renders the control panel and creates a dictionnary of its exposed elements
\begin{quote}\begin{description}
\item[{Return Type}] \leavevmode
\sphinxstyleliteralemphasis{\sphinxupquote{jQuery.Deferred}}

\end{description}\end{quote}

\end{fulllineitems}



\begin{fulllineitems}
\phantomsection\label{\detokenize{reference/javascript_api:update}}\pysiglinewithargsret{\sphinxbfcode{\sphinxupquote{method }}\sphinxbfcode{\sphinxupquote{update}}}{\emph{status}, \emph{options}}{}
Updates the content and displays the ControlPanel
\begin{quote}\begin{description}
\item[{Parameters}] \leavevmode\begin{itemize}

\sphinxstylestrong{status} ({\hyperref[\detokenize{reference/javascript_api:web.ControlPanel.UpdateStatus}]{\sphinxcrossref{\sphinxstyleliteralemphasis{\sphinxupquote{UpdateStatus}}}}})

\sphinxstylestrong{options} ({\hyperref[\detokenize{reference/javascript_api:web.ControlPanel.UpdateOptions}]{\sphinxcrossref{\sphinxstyleliteralemphasis{\sphinxupquote{UpdateOptions}}}}})
\end{itemize}

\end{description}\end{quote}


\begin{fulllineitems}
\phantomsection\label{\detokenize{reference/javascript_api:UpdateStatus}}\pysiglinewithargsret{\sphinxbfcode{\sphinxupquote{class }}\sphinxbfcode{\sphinxupquote{UpdateStatus}}}{}{}~

\begin{fulllineitems}
\phantomsection\label{\detokenize{reference/javascript_api:active_view}}\pysigline{\sphinxbfcode{\sphinxupquote{attribute }}\sphinxbfcode{\sphinxupquote{active\_view}} Object}
the current active view

\end{fulllineitems}



\begin{fulllineitems}
\phantomsection\label{\detokenize{reference/javascript_api:breadcrumbs}}\pysigline{\sphinxbfcode{\sphinxupquote{attribute }}\sphinxbfcode{\sphinxupquote{breadcrumbs}} Array}
the breadcrumbs to display (see \_render\_breadcrumbs() for
precise description)

\end{fulllineitems}



\begin{fulllineitems}
\phantomsection\label{\detokenize{reference/javascript_api:cp_content}}\pysigline{\sphinxbfcode{\sphinxupquote{attribute }}\sphinxbfcode{\sphinxupquote{cp\_content}} Object}
dictionnary containing the new ControlPanel jQuery elements

\end{fulllineitems}



\begin{fulllineitems}
\phantomsection\label{\detokenize{reference/javascript_api:hidden}}\pysigline{\sphinxbfcode{\sphinxupquote{attribute }}\sphinxbfcode{\sphinxupquote{hidden}} Boolean}
true if the ControlPanel should be hidden

\end{fulllineitems}



\begin{fulllineitems}
\phantomsection\label{\detokenize{reference/javascript_api:searchview}}\pysigline{\sphinxbfcode{\sphinxupquote{attribute }}\sphinxbfcode{\sphinxupquote{searchview}} openerp.web.SearchView}
the searchview widget

\end{fulllineitems}



\begin{fulllineitems}
\phantomsection\label{\detokenize{reference/javascript_api:search_view_hidden}}\pysigline{\sphinxbfcode{\sphinxupquote{attribute }}\sphinxbfcode{\sphinxupquote{search\_view\_hidden}} Boolean}
true if the searchview is hidden, false otherwise

\end{fulllineitems}


\end{fulllineitems}



\begin{fulllineitems}
\phantomsection\label{\detokenize{reference/javascript_api:UpdateOptions}}\pysiglinewithargsret{\sphinxbfcode{\sphinxupquote{class }}\sphinxbfcode{\sphinxupquote{UpdateOptions}}}{}{}~

\begin{fulllineitems}
\phantomsection\label{\detokenize{reference/javascript_api:clear}}\pysigline{\sphinxbfcode{\sphinxupquote{attribute }}\sphinxbfcode{\sphinxupquote{clear}} Boolean}
set to true to clear from control panel
elements that are not in status.cp\_content

\end{fulllineitems}


\end{fulllineitems}


\end{fulllineitems}


\end{fulllineitems}


\end{fulllineitems}

\phantomsection\label{\detokenize{reference/javascript_api:module-web_editor.IframeRoot.instance}}

\begin{fulllineitems}
\phantomsection\label{\detokenize{reference/javascript_api:web_editor.IframeRoot.instance}}\pysigline{\sphinxbfcode{\sphinxupquote{module }}\sphinxbfcode{\sphinxupquote{web\_editor.IframeRoot.instance}}}~~\begin{quote}\begin{description}
\item[{Exports}] \leavevmode{\hyperref[\detokenize{reference/javascript_api:web_editor.IframeRoot.instance.}]{\sphinxcrossref{
\textless{}anonymous\textgreater{}
}}}
\item[{Depends On}] \leavevmode\begin{itemize}
\item {} {\hyperref[\detokenize{reference/javascript_api:web_editor.IframeRoot}]{\sphinxcrossref{
web\_editor.IframeRoot
}}}
\end{itemize}

\end{description}\end{quote}


\begin{fulllineitems}
\phantomsection\label{\detokenize{reference/javascript_api:web_editor.IframeRoot.instance.}}\pysigline{\sphinxbfcode{\sphinxupquote{namespace }}\sphinxbfcode{\sphinxupquote{}}}
\end{fulllineitems}


\end{fulllineitems}

\phantomsection\label{\detokenize{reference/javascript_api:module-website_sale.utils}}

\begin{fulllineitems}
\phantomsection\label{\detokenize{reference/javascript_api:website_sale.utils}}\pysigline{\sphinxbfcode{\sphinxupquote{module }}\sphinxbfcode{\sphinxupquote{website\_sale.utils}}}~~\begin{quote}\begin{description}
\item[{Exports}] \leavevmode{\hyperref[\detokenize{reference/javascript_api:website_sale.utils.}]{\sphinxcrossref{
\textless{}anonymous\textgreater{}
}}}
\end{description}\end{quote}


\begin{fulllineitems}
\phantomsection\label{\detokenize{reference/javascript_api:website_sale.utils.}}\pysigline{\sphinxbfcode{\sphinxupquote{namespace }}\sphinxbfcode{\sphinxupquote{}}}
\end{fulllineitems}


\end{fulllineitems}

\phantomsection\label{\detokenize{reference/javascript_api:module-crm.partner_assign}}

\begin{fulllineitems}
\phantomsection\label{\detokenize{reference/javascript_api:crm.partner_assign}}\pysigline{\sphinxbfcode{\sphinxupquote{module }}\sphinxbfcode{\sphinxupquote{crm.partner\_assign}}}~~\begin{quote}\begin{description}
\item[{Exports}] \leavevmode{\hyperref[\detokenize{reference/javascript_api:crm.partner_assign.}]{\sphinxcrossref{
\textless{}anonymous\textgreater{}
}}}
\item[{Depends On}] \leavevmode\begin{itemize}
\item {} {\hyperref[\detokenize{reference/javascript_api:web.rpc}]{\sphinxcrossref{
web.rpc
}}}
\end{itemize}

\end{description}\end{quote}


\begin{fulllineitems}
\phantomsection\label{\detokenize{reference/javascript_api:crm.partner_assign.}}\pysigline{\sphinxbfcode{\sphinxupquote{namespace }}\sphinxbfcode{\sphinxupquote{}}}
\end{fulllineitems}


\end{fulllineitems}

\phantomsection\label{\detokenize{reference/javascript_api:module-website.navbar}}

\begin{fulllineitems}
\phantomsection\label{\detokenize{reference/javascript_api:website.navbar}}\pysigline{\sphinxbfcode{\sphinxupquote{module }}\sphinxbfcode{\sphinxupquote{website.navbar}}}~~\begin{quote}\begin{description}
\item[{Exports}] \leavevmode{\hyperref[\detokenize{reference/javascript_api:website.navbar.}]{\sphinxcrossref{
\textless{}anonymous\textgreater{}
}}}
\item[{Depends On}] \leavevmode\begin{itemize}
\item {} {\hyperref[\detokenize{reference/javascript_api:web.Widget}]{\sphinxcrossref{
web.Widget
}}}
\item {} {\hyperref[\detokenize{reference/javascript_api:web.concurrency}]{\sphinxcrossref{
web.concurrency
}}}
\item {} {\hyperref[\detokenize{reference/javascript_api:web_editor.root_widget}]{\sphinxcrossref{
web\_editor.root\_widget
}}}
\item {} {\hyperref[\detokenize{reference/javascript_api:website.WebsiteRoot}]{\sphinxcrossref{
website.WebsiteRoot
}}}
\end{itemize}

\end{description}\end{quote}


\begin{fulllineitems}
\phantomsection\label{\detokenize{reference/javascript_api:website.navbar.}}\pysigline{\sphinxbfcode{\sphinxupquote{namespace }}\sphinxbfcode{\sphinxupquote{}}}
\end{fulllineitems}


\end{fulllineitems}

\phantomsection\label{\detokenize{reference/javascript_api:module-web_editor.root_widget}}

\begin{fulllineitems}
\phantomsection\label{\detokenize{reference/javascript_api:web_editor.root_widget}}\pysigline{\sphinxbfcode{\sphinxupquote{module }}\sphinxbfcode{\sphinxupquote{web\_editor.root\_widget}}}~~\begin{quote}\begin{description}
\item[{Exports}] \leavevmode{\hyperref[\detokenize{reference/javascript_api:web_editor.root_widget.}]{\sphinxcrossref{
\textless{}anonymous\textgreater{}
}}}
\item[{Depends On}] \leavevmode\begin{itemize}
\item {} {\hyperref[\detokenize{reference/javascript_api:web.Class}]{\sphinxcrossref{
web.Class
}}}
\item {} {\hyperref[\detokenize{reference/javascript_api:web.Widget}]{\sphinxcrossref{
web.Widget
}}}
\item {} {\hyperref[\detokenize{reference/javascript_api:web.dom}]{\sphinxcrossref{
web.dom
}}}
\item {} {\hyperref[\detokenize{reference/javascript_api:web.mixins}]{\sphinxcrossref{
web.mixins
}}}
\end{itemize}

\end{description}\end{quote}


\begin{fulllineitems}
\phantomsection\label{\detokenize{reference/javascript_api:RootWidget}}\pysiglinewithargsret{\sphinxbfcode{\sphinxupquote{class }}\sphinxbfcode{\sphinxupquote{RootWidget}}}{}{}~\begin{quote}\begin{description}
\item[{Extends}] \leavevmode{\hyperref[\detokenize{reference/javascript_api:web.Widget.Widget}]{\sphinxcrossref{
Widget
}}}
\end{description}\end{quote}

Specialized Widget which automatically instantiates child widgets to attach
to internal DOM elements once it is started. The widgets to instantiate are
known thanks to a linked registry which contains info about the widget
classes and jQuery selectors to use to find the elements to attach them to.

\end{fulllineitems}



\begin{fulllineitems}
\phantomsection\label{\detokenize{reference/javascript_api:web_editor.root_widget.}}\pysigline{\sphinxbfcode{\sphinxupquote{namespace }}\sphinxbfcode{\sphinxupquote{}}}~

\begin{fulllineitems}
\phantomsection\label{\detokenize{reference/javascript_api:RootWidget}}\pysiglinewithargsret{\sphinxbfcode{\sphinxupquote{class }}\sphinxbfcode{\sphinxupquote{RootWidget}}}{}{}~\begin{quote}\begin{description}
\item[{Extends}] \leavevmode{\hyperref[\detokenize{reference/javascript_api:web.Widget.Widget}]{\sphinxcrossref{
Widget
}}}
\end{description}\end{quote}

Specialized Widget which automatically instantiates child widgets to attach
to internal DOM elements once it is started. The widgets to instantiate are
known thanks to a linked registry which contains info about the widget
classes and jQuery selectors to use to find the elements to attach them to.

\end{fulllineitems}


\end{fulllineitems}


\end{fulllineitems}

\phantomsection\label{\detokenize{reference/javascript_api:module-web.view_registry}}

\begin{fulllineitems}
\phantomsection\label{\detokenize{reference/javascript_api:web.view_registry}}\pysigline{\sphinxbfcode{\sphinxupquote{module }}\sphinxbfcode{\sphinxupquote{web.view\_registry}}}~~\begin{quote}\begin{description}
\item[{Exports}] \leavevmode{\hyperref[\detokenize{reference/javascript_api:web.view_registry.}]{\sphinxcrossref{
\textless{}anonymous\textgreater{}
}}}
\item[{Depends On}] \leavevmode\begin{itemize}
\item {} {\hyperref[\detokenize{reference/javascript_api:web.Registry}]{\sphinxcrossref{
web.Registry
}}}
\end{itemize}

\end{description}\end{quote}


\begin{fulllineitems}
\phantomsection\label{\detokenize{reference/javascript_api:web.view_registry.}}\pysigline{\sphinxbfcode{\sphinxupquote{object }}\sphinxbfcode{\sphinxupquote{}}\sphinxbfcode{\sphinxupquote{ instance of }}{\hyperref[\detokenize{reference/javascript_api:web.Registry.Registry}]{\sphinxcrossref{Registry}}}}
\end{fulllineitems}


\end{fulllineitems}

\phantomsection\label{\detokenize{reference/javascript_api:module-web.planner.common}}

\begin{fulllineitems}
\phantomsection\label{\detokenize{reference/javascript_api:web.planner.common}}\pysigline{\sphinxbfcode{\sphinxupquote{module }}\sphinxbfcode{\sphinxupquote{web.planner.common}}}~~\begin{quote}\begin{description}
\item[{Exports}] \leavevmode{\hyperref[\detokenize{reference/javascript_api:web.planner.common.}]{\sphinxcrossref{
\textless{}anonymous\textgreater{}
}}}
\item[{Depends On}] \leavevmode\begin{itemize}
\item {} {\hyperref[\detokenize{reference/javascript_api:web.Dialog}]{\sphinxcrossref{
web.Dialog
}}}
\item {} {\hyperref[\detokenize{reference/javascript_api:web.Widget}]{\sphinxcrossref{
web.Widget
}}}
\item {} {\hyperref[\detokenize{reference/javascript_api:web.core}]{\sphinxcrossref{
web.core
}}}
\item {} {\hyperref[\detokenize{reference/javascript_api:web.dom}]{\sphinxcrossref{
web.dom
}}}
\item {} {\hyperref[\detokenize{reference/javascript_api:web.rpc}]{\sphinxcrossref{
web.rpc
}}}
\item {} {\hyperref[\detokenize{reference/javascript_api:web.session}]{\sphinxcrossref{
web.session
}}}
\item {} {\hyperref[\detokenize{reference/javascript_api:web.utils}]{\sphinxcrossref{
web.utils
}}}
\item {} {\hyperref[\detokenize{reference/javascript_api:web_editor.context}]{\sphinxcrossref{
web\_editor.context
}}}
\end{itemize}

\end{description}\end{quote}


\begin{fulllineitems}
\phantomsection\label{\detokenize{reference/javascript_api:web.planner.common.}}\pysigline{\sphinxbfcode{\sphinxupquote{namespace }}\sphinxbfcode{\sphinxupquote{}}}
\end{fulllineitems}


\end{fulllineitems}

\phantomsection\label{\detokenize{reference/javascript_api:module-stock.stock_report_generic}}

\begin{fulllineitems}
\phantomsection\label{\detokenize{reference/javascript_api:stock.stock_report_generic}}\pysigline{\sphinxbfcode{\sphinxupquote{module }}\sphinxbfcode{\sphinxupquote{stock.stock\_report\_generic}}}~~\begin{quote}\begin{description}
\item[{Exports}] \leavevmode{\hyperref[\detokenize{reference/javascript_api:stock.stock_report_generic.stock_report_generic}]{\sphinxcrossref{
stock\_report\_generic
}}}
\item[{Depends On}] \leavevmode\begin{itemize}
\item {} {\hyperref[\detokenize{reference/javascript_api:stock.ReportWidget}]{\sphinxcrossref{
stock.ReportWidget
}}}
\item {} {\hyperref[\detokenize{reference/javascript_api:web.ControlPanelMixin}]{\sphinxcrossref{
web.ControlPanelMixin
}}}
\item {} {\hyperref[\detokenize{reference/javascript_api:web.Widget}]{\sphinxcrossref{
web.Widget
}}}
\item {} {\hyperref[\detokenize{reference/javascript_api:web.core}]{\sphinxcrossref{
web.core
}}}
\item {} {\hyperref[\detokenize{reference/javascript_api:web.crash_manager}]{\sphinxcrossref{
web.crash\_manager
}}}
\item {} {\hyperref[\detokenize{reference/javascript_api:web.framework}]{\sphinxcrossref{
web.framework
}}}
\item {} {\hyperref[\detokenize{reference/javascript_api:web.session}]{\sphinxcrossref{
web.session
}}}
\end{itemize}

\end{description}\end{quote}


\begin{fulllineitems}
\phantomsection\label{\detokenize{reference/javascript_api:stock_report_generic}}\pysiglinewithargsret{\sphinxbfcode{\sphinxupquote{class }}\sphinxbfcode{\sphinxupquote{stock\_report\_generic}}}{\emph{parent}, \emph{action}}{}~\begin{quote}\begin{description}
\item[{Extends}] \leavevmode{\hyperref[\detokenize{reference/javascript_api:web.Widget.Widget}]{\sphinxcrossref{
Widget
}}}
\item[{Mixes}] \leavevmode\begin{itemize}
\item {} {\hyperref[\detokenize{reference/javascript_api:web.ControlPanelMixin.ControlPanelMixin}]{\sphinxcrossref{
ControlPanelMixin
}}}
\end{itemize}

\item[{Parameters}] \leavevmode\begin{itemize}

\sphinxstylestrong{parent}

\sphinxstylestrong{action}
\end{itemize}

\end{description}\end{quote}

\end{fulllineitems}


\end{fulllineitems}

\phantomsection\label{\detokenize{reference/javascript_api:module-web.GroupByMenu}}

\begin{fulllineitems}
\phantomsection\label{\detokenize{reference/javascript_api:web.GroupByMenu}}\pysigline{\sphinxbfcode{\sphinxupquote{module }}\sphinxbfcode{\sphinxupquote{web.GroupByMenu}}}~~\begin{quote}\begin{description}
\item[{Exports}] \leavevmode{\hyperref[\detokenize{reference/javascript_api:web.GroupByMenu.}]{\sphinxcrossref{
\textless{}anonymous\textgreater{}
}}}
\item[{Depends On}] \leavevmode\begin{itemize}
\item {} {\hyperref[\detokenize{reference/javascript_api:web.Widget}]{\sphinxcrossref{
web.Widget
}}}
\item {} {\hyperref[\detokenize{reference/javascript_api:web.core}]{\sphinxcrossref{
web.core
}}}
\item {} {\hyperref[\detokenize{reference/javascript_api:web.search_inputs}]{\sphinxcrossref{
web.search\_inputs
}}}
\end{itemize}

\end{description}\end{quote}


\begin{fulllineitems}
\phantomsection\label{\detokenize{reference/javascript_api:web.GroupByMenu.}}\pysiglinewithargsret{\sphinxbfcode{\sphinxupquote{class }}\sphinxbfcode{\sphinxupquote{}}}{\emph{parent}, \emph{groups}, \emph{fields}}{}~\begin{quote}\begin{description}
\item[{Extends}] \leavevmode{\hyperref[\detokenize{reference/javascript_api:web.Widget.Widget}]{\sphinxcrossref{
Widget
}}}
\item[{Parameters}] \leavevmode\begin{itemize}

\sphinxstylestrong{parent}

\sphinxstylestrong{groups}

\sphinxstylestrong{fields}
\end{itemize}

\end{description}\end{quote}

\end{fulllineitems}


\end{fulllineitems}

\phantomsection\label{\detokenize{reference/javascript_api:module-website.utils}}

\begin{fulllineitems}
\phantomsection\label{\detokenize{reference/javascript_api:website.utils}}\pysigline{\sphinxbfcode{\sphinxupquote{module }}\sphinxbfcode{\sphinxupquote{website.utils}}}~~\begin{quote}\begin{description}
\item[{Exports}] \leavevmode{\hyperref[\detokenize{reference/javascript_api:website.utils.}]{\sphinxcrossref{
\textless{}anonymous\textgreater{}
}}}
\item[{Depends On}] \leavevmode\begin{itemize}
\item {} {\hyperref[\detokenize{reference/javascript_api:web.ajax}]{\sphinxcrossref{
web.ajax
}}}
\item {} {\hyperref[\detokenize{reference/javascript_api:web.core}]{\sphinxcrossref{
web.core
}}}
\item {} {\hyperref[\detokenize{reference/javascript_api:web_editor.context}]{\sphinxcrossref{
web\_editor.context
}}}
\end{itemize}

\end{description}\end{quote}


\begin{fulllineitems}
\phantomsection\label{\detokenize{reference/javascript_api:autocompleteWithPages}}\pysiglinewithargsret{\sphinxbfcode{\sphinxupquote{function }}\sphinxbfcode{\sphinxupquote{autocompleteWithPages}}}{\emph{self}, \emph{\$input}}{}
Allows the given input to propose existing website URLs.
\begin{quote}\begin{description}
\item[{Parameters}] \leavevmode\begin{itemize}

\sphinxstylestrong{self} ({\hyperref[\detokenize{reference/javascript_api:ServicesMixin}]{\sphinxcrossref{\sphinxstyleliteralemphasis{\sphinxupquote{ServicesMixin}}}}}\sphinxstyleemphasis{ or }{\hyperref[\detokenize{reference/javascript_api:Widget}]{\sphinxcrossref{\sphinxstyleliteralemphasis{\sphinxupquote{Widget}}}}}) \textendash{} an element capable to trigger an RPC

\sphinxstylestrong{\$input} (\sphinxstyleliteralemphasis{\sphinxupquote{jQuery}})
\end{itemize}

\end{description}\end{quote}

\end{fulllineitems}



\begin{fulllineitems}
\phantomsection\label{\detokenize{reference/javascript_api:website.utils.}}\pysigline{\sphinxbfcode{\sphinxupquote{namespace }}\sphinxbfcode{\sphinxupquote{}}}~

\begin{fulllineitems}
\phantomsection\label{\detokenize{reference/javascript_api:autocompleteWithPages}}\pysiglinewithargsret{\sphinxbfcode{\sphinxupquote{function }}\sphinxbfcode{\sphinxupquote{autocompleteWithPages}}}{\emph{self}, \emph{\$input}}{}
Allows the given input to propose existing website URLs.
\begin{quote}\begin{description}
\item[{Parameters}] \leavevmode\begin{itemize}

\sphinxstylestrong{self} ({\hyperref[\detokenize{reference/javascript_api:ServicesMixin}]{\sphinxcrossref{\sphinxstyleliteralemphasis{\sphinxupquote{ServicesMixin}}}}}\sphinxstyleemphasis{ or }{\hyperref[\detokenize{reference/javascript_api:Widget}]{\sphinxcrossref{\sphinxstyleliteralemphasis{\sphinxupquote{Widget}}}}}) \textendash{} an element capable to trigger an RPC

\sphinxstylestrong{\$input} (\sphinxstyleliteralemphasis{\sphinxupquote{jQuery}})
\end{itemize}

\end{description}\end{quote}

\end{fulllineitems}


\end{fulllineitems}


\end{fulllineitems}

\phantomsection\label{\detokenize{reference/javascript_api:module-mail.ChatThread}}

\begin{fulllineitems}
\phantomsection\label{\detokenize{reference/javascript_api:mail.ChatThread}}\pysigline{\sphinxbfcode{\sphinxupquote{module }}\sphinxbfcode{\sphinxupquote{mail.ChatThread}}}~~\begin{quote}\begin{description}
\item[{Exports}] \leavevmode{\hyperref[\detokenize{reference/javascript_api:mail.ChatThread.Thread}]{\sphinxcrossref{
Thread
}}}
\item[{Depends On}] \leavevmode\begin{itemize}
\item {} {\hyperref[\detokenize{reference/javascript_api:mail.DocumentViewer}]{\sphinxcrossref{
mail.DocumentViewer
}}}
\item {} {\hyperref[\detokenize{reference/javascript_api:web.Widget}]{\sphinxcrossref{
web.Widget
}}}
\item {} {\hyperref[\detokenize{reference/javascript_api:web.core}]{\sphinxcrossref{
web.core
}}}
\item {} {\hyperref[\detokenize{reference/javascript_api:web.time}]{\sphinxcrossref{
web.time
}}}
\end{itemize}

\end{description}\end{quote}


\begin{fulllineitems}
\phantomsection\label{\detokenize{reference/javascript_api:Thread}}\pysiglinewithargsret{\sphinxbfcode{\sphinxupquote{class }}\sphinxbfcode{\sphinxupquote{Thread}}}{\emph{parent}, \emph{options}}{}~\begin{quote}\begin{description}
\item[{Extends}] \leavevmode{\hyperref[\detokenize{reference/javascript_api:web.Widget.Widget}]{\sphinxcrossref{
Widget
}}}
\item[{Parameters}] \leavevmode\begin{itemize}

\sphinxstylestrong{parent}

\sphinxstylestrong{options}
\end{itemize}

\end{description}\end{quote}


\begin{fulllineitems}
\phantomsection\label{\detokenize{reference/javascript_api:insert_read_more}}\pysiglinewithargsret{\sphinxbfcode{\sphinxupquote{method }}\sphinxbfcode{\sphinxupquote{insert\_read\_more}}}{\emph{\$element}}{}
Modifies \$element to add the ‘read more/read less’ functionality
All element nodes with “data-o-mail-quote” attribute are concerned.
All text nodes after a \sphinxcode{\sphinxupquote{\#stopSpelling}} element are concerned.
Those text nodes need to be wrapped in a span (toggle functionality).
All consecutive elements are joined in one ‘read more/read less’.
\begin{quote}\begin{description}
\item[{Parameters}] \leavevmode\begin{itemize}

\sphinxstylestrong{\$element}
\end{itemize}

\end{description}\end{quote}

\end{fulllineitems}



\begin{fulllineitems}
\phantomsection\label{\detokenize{reference/javascript_api:remove_message_and_render}}\pysiglinewithargsret{\sphinxbfcode{\sphinxupquote{method }}\sphinxbfcode{\sphinxupquote{remove\_message\_and\_render}}}{\sphinxoptional{\emph{message\_id}}\sphinxoptional{, \emph{messages}}\sphinxoptional{, \emph{options}}}{}
Removes a message and re-renders the thread
\begin{quote}\begin{description}
\item[{Parameters}] \leavevmode\begin{itemize}

\sphinxstylestrong{message\_id} (\sphinxstyleliteralemphasis{\sphinxupquote{int}}) \textendash{} the id of the removed message

\sphinxstylestrong{messages} (\sphinxstyleliteralemphasis{\sphinxupquote{array}}) \textendash{} the list of messages to display, without the removed one

\sphinxstylestrong{options} (\sphinxstyleliteralemphasis{\sphinxupquote{object}}) \textendash{} options for the thread rendering
\end{itemize}

\end{description}\end{quote}

\end{fulllineitems}



\begin{fulllineitems}
\phantomsection\label{\detokenize{reference/javascript_api:scroll_to}}\pysiglinewithargsret{\sphinxbfcode{\sphinxupquote{method }}\sphinxbfcode{\sphinxupquote{scroll\_to}}}{\emph{options}}{}
Scrolls the thread to a given message or offset if any, to bottom otherwise
\begin{quote}\begin{description}
\item[{Parameters}] \leavevmode\begin{itemize}

\sphinxstylestrong{options} ({\hyperref[\detokenize{reference/javascript_api:mail.ChatThread.ScrollToOptions}]{\sphinxcrossref{\sphinxstyleliteralemphasis{\sphinxupquote{ScrollToOptions}}}}})
\end{itemize}

\end{description}\end{quote}


\begin{fulllineitems}
\phantomsection\label{\detokenize{reference/javascript_api:ScrollToOptions}}\pysiglinewithargsret{\sphinxbfcode{\sphinxupquote{class }}\sphinxbfcode{\sphinxupquote{ScrollToOptions}}}{}{}~

\begin{fulllineitems}
\phantomsection\label{\detokenize{reference/javascript_api:id}}\pysigline{\sphinxbfcode{\sphinxupquote{attribute }}\sphinxbfcode{\sphinxupquote{id}} int}
optional: the id of the message to scroll to

\end{fulllineitems}



\begin{fulllineitems}
\phantomsection\label{\detokenize{reference/javascript_api:offset}}\pysigline{\sphinxbfcode{\sphinxupquote{attribute }}\sphinxbfcode{\sphinxupquote{offset}} int}
optional: the number of pixels to scroll

\end{fulllineitems}


\end{fulllineitems}


\end{fulllineitems}


\end{fulllineitems}


\end{fulllineitems}

\phantomsection\label{\detokenize{reference/javascript_api:module-website_sale.editor}}

\begin{fulllineitems}
\phantomsection\label{\detokenize{reference/javascript_api:website_sale.editor}}\pysigline{\sphinxbfcode{\sphinxupquote{module }}\sphinxbfcode{\sphinxupquote{website\_sale.editor}}}~~\begin{quote}\begin{description}
\item[{Exports}] \leavevmode{\hyperref[\detokenize{reference/javascript_api:website_sale.editor.}]{\sphinxcrossref{
\textless{}anonymous\textgreater{}
}}}
\item[{Depends On}] \leavevmode\begin{itemize}
\item {} {\hyperref[\detokenize{reference/javascript_api:web_editor.snippets.options}]{\sphinxcrossref{
web\_editor.snippets.options
}}}
\end{itemize}

\end{description}\end{quote}


\begin{fulllineitems}
\phantomsection\label{\detokenize{reference/javascript_api:website_sale.editor.}}\pysigline{\sphinxbfcode{\sphinxupquote{namespace }}\sphinxbfcode{\sphinxupquote{}}}
\end{fulllineitems}


\end{fulllineitems}

\phantomsection\label{\detokenize{reference/javascript_api:module-website.WebsiteRoot.instance}}

\begin{fulllineitems}
\phantomsection\label{\detokenize{reference/javascript_api:website.WebsiteRoot.instance}}\pysigline{\sphinxbfcode{\sphinxupquote{module }}\sphinxbfcode{\sphinxupquote{website.WebsiteRoot.instance}}}~~\begin{quote}\begin{description}
\item[{Exports}] \leavevmode{\hyperref[\detokenize{reference/javascript_api:website.WebsiteRoot.instance.}]{\sphinxcrossref{
\textless{}anonymous\textgreater{}
}}}
\item[{Depends On}] \leavevmode\begin{itemize}
\item {} {\hyperref[\detokenize{reference/javascript_api:website.WebsiteRoot}]{\sphinxcrossref{
website.WebsiteRoot
}}}
\end{itemize}

\end{description}\end{quote}


\begin{fulllineitems}
\phantomsection\label{\detokenize{reference/javascript_api:website.WebsiteRoot.instance.}}\pysigline{\sphinxbfcode{\sphinxupquote{namespace }}\sphinxbfcode{\sphinxupquote{}}}
\end{fulllineitems}


\end{fulllineitems}

\phantomsection\label{\detokenize{reference/javascript_api:module-mail.Chatter}}

\begin{fulllineitems}
\phantomsection\label{\detokenize{reference/javascript_api:mail.Chatter}}\pysigline{\sphinxbfcode{\sphinxupquote{module }}\sphinxbfcode{\sphinxupquote{mail.Chatter}}}~~\begin{quote}\begin{description}
\item[{Exports}] \leavevmode{\hyperref[\detokenize{reference/javascript_api:mail.Chatter.Chatter}]{\sphinxcrossref{
Chatter
}}}
\item[{Depends On}] \leavevmode\begin{itemize}
\item {} {\hyperref[\detokenize{reference/javascript_api:mail.Activity}]{\sphinxcrossref{
mail.Activity
}}}
\item {} {\hyperref[\detokenize{reference/javascript_api:mail.ChatterComposer}]{\sphinxcrossref{
mail.ChatterComposer
}}}
\item {} {\hyperref[\detokenize{reference/javascript_api:mail.Followers}]{\sphinxcrossref{
mail.Followers
}}}
\item {} {\hyperref[\detokenize{reference/javascript_api:mail.ThreadField}]{\sphinxcrossref{
mail.ThreadField
}}}
\item {} {\hyperref[\detokenize{reference/javascript_api:mail.chat_mixin}]{\sphinxcrossref{
mail.chat\_mixin
}}}
\item {} {\hyperref[\detokenize{reference/javascript_api:mail.utils}]{\sphinxcrossref{
mail.utils
}}}
\item {} {\hyperref[\detokenize{reference/javascript_api:web.Widget}]{\sphinxcrossref{
web.Widget
}}}
\item {} {\hyperref[\detokenize{reference/javascript_api:web.concurrency}]{\sphinxcrossref{
web.concurrency
}}}
\item {} {\hyperref[\detokenize{reference/javascript_api:web.config}]{\sphinxcrossref{
web.config
}}}
\item {} {\hyperref[\detokenize{reference/javascript_api:web.core}]{\sphinxcrossref{
web.core
}}}
\end{itemize}

\end{description}\end{quote}


\begin{fulllineitems}
\phantomsection\label{\detokenize{reference/javascript_api:Chatter}}\pysiglinewithargsret{\sphinxbfcode{\sphinxupquote{class }}\sphinxbfcode{\sphinxupquote{Chatter}}}{\emph{parent}, \emph{record}, \emph{mailFields}, \emph{options}}{}~\begin{quote}\begin{description}
\item[{Extends}] \leavevmode{\hyperref[\detokenize{reference/javascript_api:web.Widget.Widget}]{\sphinxcrossref{
Widget
}}}
\item[{Mixes}] \leavevmode\begin{itemize}
\item {} {\hyperref[\detokenize{reference/javascript_api:mail.chat_mixin.ChatMixin}]{\sphinxcrossref{
ChatMixin
}}}
\end{itemize}

\item[{Parameters}] \leavevmode\begin{itemize}

\sphinxstylestrong{parent}

\sphinxstylestrong{record}

\sphinxstylestrong{mailFields}

\sphinxstylestrong{options}
\end{itemize}

\end{description}\end{quote}

\end{fulllineitems}


\end{fulllineitems}

\phantomsection\label{\detokenize{reference/javascript_api:module-mail.chat_mixin}}

\begin{fulllineitems}
\phantomsection\label{\detokenize{reference/javascript_api:mail.chat_mixin}}\pysigline{\sphinxbfcode{\sphinxupquote{module }}\sphinxbfcode{\sphinxupquote{mail.chat\_mixin}}}~~\begin{quote}\begin{description}
\item[{Exports}] \leavevmode{\hyperref[\detokenize{reference/javascript_api:mail.chat_mixin.ChatMixin}]{\sphinxcrossref{
ChatMixin
}}}
\end{description}\end{quote}


\begin{fulllineitems}
\phantomsection\label{\detokenize{reference/javascript_api:ChatMixin}}\pysigline{\sphinxbfcode{\sphinxupquote{namespace }}\sphinxbfcode{\sphinxupquote{ChatMixin}}}
\end{fulllineitems}


\end{fulllineitems}

\phantomsection\label{\detokenize{reference/javascript_api:module-web.AutoComplete}}

\begin{fulllineitems}
\phantomsection\label{\detokenize{reference/javascript_api:web.AutoComplete}}\pysigline{\sphinxbfcode{\sphinxupquote{module }}\sphinxbfcode{\sphinxupquote{web.AutoComplete}}}~~\begin{quote}\begin{description}
\item[{Exports}] \leavevmode{\hyperref[\detokenize{reference/javascript_api:web.AutoComplete.}]{\sphinxcrossref{
\textless{}anonymous\textgreater{}
}}}
\item[{Depends On}] \leavevmode\begin{itemize}
\item {} {\hyperref[\detokenize{reference/javascript_api:web.Widget}]{\sphinxcrossref{
web.Widget
}}}
\end{itemize}

\end{description}\end{quote}


\begin{fulllineitems}
\phantomsection\label{\detokenize{reference/javascript_api:web.AutoComplete.}}\pysiglinewithargsret{\sphinxbfcode{\sphinxupquote{class }}\sphinxbfcode{\sphinxupquote{}}}{\emph{parent}, \emph{options}}{}~\begin{quote}\begin{description}
\item[{Extends}] \leavevmode{\hyperref[\detokenize{reference/javascript_api:web.Widget.Widget}]{\sphinxcrossref{
Widget
}}}
\item[{Parameters}] \leavevmode\begin{itemize}

\sphinxstylestrong{parent}

\sphinxstylestrong{options}
\end{itemize}

\end{description}\end{quote}

\end{fulllineitems}


\end{fulllineitems}

\phantomsection\label{\detokenize{reference/javascript_api:module-web.DataManager}}

\begin{fulllineitems}
\phantomsection\label{\detokenize{reference/javascript_api:web.DataManager}}\pysigline{\sphinxbfcode{\sphinxupquote{module }}\sphinxbfcode{\sphinxupquote{web.DataManager}}}~~\begin{quote}\begin{description}
\item[{Exports}] \leavevmode{\hyperref[\detokenize{reference/javascript_api:web.DataManager.}]{\sphinxcrossref{
\textless{}anonymous\textgreater{}
}}}
\item[{Depends On}] \leavevmode\begin{itemize}
\item {} {\hyperref[\detokenize{reference/javascript_api:web.config}]{\sphinxcrossref{
web.config
}}}
\item {} {\hyperref[\detokenize{reference/javascript_api:web.core}]{\sphinxcrossref{
web.core
}}}
\item {} {\hyperref[\detokenize{reference/javascript_api:web.field_registry}]{\sphinxcrossref{
web.field\_registry
}}}
\item {} {\hyperref[\detokenize{reference/javascript_api:web.pyeval}]{\sphinxcrossref{
web.pyeval
}}}
\item {} {\hyperref[\detokenize{reference/javascript_api:web.rpc}]{\sphinxcrossref{
web.rpc
}}}
\item {} {\hyperref[\detokenize{reference/javascript_api:web.utils}]{\sphinxcrossref{
web.utils
}}}
\end{itemize}

\end{description}\end{quote}


\begin{fulllineitems}
\phantomsection\label{\detokenize{reference/javascript_api:web.DataManager.}}\pysiglinewithargsret{\sphinxbfcode{\sphinxupquote{class }}\sphinxbfcode{\sphinxupquote{}}}{}{}~\begin{quote}\begin{description}
\item[{Extends}] \leavevmode{\hyperref[\detokenize{reference/javascript_api:web.Class.Class}]{\sphinxcrossref{
Class
}}}
\end{description}\end{quote}


\begin{fulllineitems}
\phantomsection\label{\detokenize{reference/javascript_api:invalidate}}\pysiglinewithargsret{\sphinxbfcode{\sphinxupquote{function }}\sphinxbfcode{\sphinxupquote{invalidate}}}{}{}
Invalidates the whole cache
Suggestion: could be refined to invalidate some part of the cache

\end{fulllineitems}



\begin{fulllineitems}
\phantomsection\label{\detokenize{reference/javascript_api:load_action}}\pysiglinewithargsret{\sphinxbfcode{\sphinxupquote{function }}\sphinxbfcode{\sphinxupquote{load\_action}}}{\sphinxoptional{\emph{action\_id}}\sphinxoptional{, \emph{additional\_context}}}{{ $\rightarrow$ Deferred}}
Loads an action from its id or xmlid.
\begin{quote}\begin{description}
\item[{Parameters}] \leavevmode\begin{itemize}

\sphinxstylestrong{action\_id} (\sphinxstyleliteralemphasis{\sphinxupquote{int}}\sphinxstyleemphasis{ or }\sphinxstyleliteralemphasis{\sphinxupquote{string}}) \textendash{} the action id or xmlid

\sphinxstylestrong{additional\_context} (\sphinxstyleliteralemphasis{\sphinxupquote{Object}}) \textendash{} used to load the action
\end{itemize}

\item[{Returns}] \leavevmode
resolved with the action whose id or xmlid is action\_id

\item[{Return Type}] \leavevmode
\sphinxstyleliteralemphasis{\sphinxupquote{Deferred}}

\end{description}\end{quote}

\end{fulllineitems}



\begin{fulllineitems}
\phantomsection\label{\detokenize{reference/javascript_api:load_views}}\pysiglinewithargsret{\sphinxbfcode{\sphinxupquote{function }}\sphinxbfcode{\sphinxupquote{load\_views}}}{\emph{params}\sphinxoptional{, \emph{options}}}{{ $\rightarrow$ Deferred}}
Loads various information concerning views: fields\_view for each view,
the fields of the corresponding model, and optionally the filters.
\begin{quote}\begin{description}
\item[{Parameters}] \leavevmode\begin{itemize}

\sphinxstylestrong{params} ({\hyperref[\detokenize{reference/javascript_api:web.DataManager.LoadViewsParams}]{\sphinxcrossref{\sphinxstyleliteralemphasis{\sphinxupquote{LoadViewsParams}}}}})

\sphinxstylestrong{options} (\sphinxstyleliteralemphasis{\sphinxupquote{Object}}) \textendash{} dictionary of various options:
    - options.load\_filters: whether or not to load the filters,
    - options.action\_id: the action\_id (required to load filters),
    - options.toolbar: whether or not a toolbar will be displayed,
\end{itemize}

\item[{Returns}] \leavevmode
resolved with the requested views information

\item[{Return Type}] \leavevmode
\sphinxstyleliteralemphasis{\sphinxupquote{Deferred}}

\end{description}\end{quote}


\begin{fulllineitems}
\phantomsection\label{\detokenize{reference/javascript_api:LoadViewsParams}}\pysiglinewithargsret{\sphinxbfcode{\sphinxupquote{class }}\sphinxbfcode{\sphinxupquote{LoadViewsParams}}}{}{}~

\begin{fulllineitems}
\phantomsection\label{\detokenize{reference/javascript_api:model}}\pysigline{\sphinxbfcode{\sphinxupquote{attribute }}\sphinxbfcode{\sphinxupquote{model}} String}
\end{fulllineitems}



\begin{fulllineitems}
\phantomsection\label{\detokenize{reference/javascript_api:context}}\pysigline{\sphinxbfcode{\sphinxupquote{attribute }}\sphinxbfcode{\sphinxupquote{context}} Object}
\end{fulllineitems}



\begin{fulllineitems}
\phantomsection\label{\detokenize{reference/javascript_api:views_descr}}\pysigline{\sphinxbfcode{\sphinxupquote{attribute }}\sphinxbfcode{\sphinxupquote{views\_descr}} Array}
array of {[}view\_id, view\_type{]}

\end{fulllineitems}


\end{fulllineitems}


\end{fulllineitems}



\begin{fulllineitems}
\phantomsection\label{\detokenize{reference/javascript_api:load_filters}}\pysiglinewithargsret{\sphinxbfcode{\sphinxupquote{function }}\sphinxbfcode{\sphinxupquote{load\_filters}}}{\sphinxoptional{\emph{dataset}}\sphinxoptional{, \emph{action\_id}}}{{ $\rightarrow$ Deferred}}
Loads the filters of a given model and optional action id.
\begin{quote}\begin{description}
\item[{Parameters}] \leavevmode\begin{itemize}

\sphinxstylestrong{dataset} (\sphinxstyleliteralemphasis{\sphinxupquote{Object}}) \textendash{} the dataset for which the filters are loaded

\sphinxstylestrong{action\_id} (\sphinxstyleliteralemphasis{\sphinxupquote{int}}) \textendash{} the id of the action (optional)
\end{itemize}

\item[{Returns}] \leavevmode
resolved with the requested filters

\item[{Return Type}] \leavevmode
\sphinxstyleliteralemphasis{\sphinxupquote{Deferred}}

\end{description}\end{quote}

\end{fulllineitems}



\begin{fulllineitems}
\phantomsection\label{\detokenize{reference/javascript_api:create_filter}}\pysiglinewithargsret{\sphinxbfcode{\sphinxupquote{function }}\sphinxbfcode{\sphinxupquote{create\_filter}}}{\sphinxoptional{\emph{filter}}}{{ $\rightarrow$ Deferred}}
Calls ‘create\_or\_replace’ on ‘ir\_filters’.
\begin{quote}\begin{description}
\item[{Parameters}] \leavevmode\begin{itemize}

\sphinxstylestrong{filter} (\sphinxstyleliteralemphasis{\sphinxupquote{Object}}) \textendash{} the filter description
\end{itemize}

\item[{Returns}] \leavevmode
resolved with the id of the created or replaced filter

\item[{Return Type}] \leavevmode
\sphinxstyleliteralemphasis{\sphinxupquote{Deferred}}

\end{description}\end{quote}

\end{fulllineitems}



\begin{fulllineitems}
\phantomsection\label{\detokenize{reference/javascript_api:delete_filter}}\pysiglinewithargsret{\sphinxbfcode{\sphinxupquote{function }}\sphinxbfcode{\sphinxupquote{delete\_filter}}}{\sphinxoptional{\emph{filter}}}{{ $\rightarrow$ Deferred}}
Calls ‘unlink’ on ‘ir\_filters’.
\begin{quote}\begin{description}
\item[{Parameters}] \leavevmode\begin{itemize}

\sphinxstylestrong{filter} (\sphinxstyleliteralemphasis{\sphinxupquote{Object}}) \textendash{} the description of the filter to remove
\end{itemize}

\item[{Return Type}] \leavevmode
\sphinxstyleliteralemphasis{\sphinxupquote{Deferred}}

\end{description}\end{quote}

\end{fulllineitems}



\begin{fulllineitems}
\phantomsection\label{\detokenize{reference/javascript_api:processViews}}\pysiglinewithargsret{\sphinxbfcode{\sphinxupquote{function }}\sphinxbfcode{\sphinxupquote{processViews}}}{\emph{fieldsViews}, \emph{fields}}{}
Processes fields and fields\_views. For each field, writes its name inside
the field description to make it self-contained. For each fields\_view,
completes its fields with the missing ones.
\begin{quote}\begin{description}
\item[{Parameters}] \leavevmode\begin{itemize}

\sphinxstylestrong{fieldsViews} (\sphinxstyleliteralemphasis{\sphinxupquote{Object}}) \textendash{} object of fields\_views (keys are view types)

\sphinxstylestrong{fields} (\sphinxstyleliteralemphasis{\sphinxupquote{Object}}) \textendash{} all the fields of the model
\end{itemize}

\end{description}\end{quote}

\end{fulllineitems}


\end{fulllineitems}


\end{fulllineitems}

\phantomsection\label{\detokenize{reference/javascript_api:module-web_editor.context}}

\begin{fulllineitems}
\phantomsection\label{\detokenize{reference/javascript_api:web_editor.context}}\pysigline{\sphinxbfcode{\sphinxupquote{module }}\sphinxbfcode{\sphinxupquote{web\_editor.context}}}~~\begin{quote}\begin{description}
\item[{Exports}] \leavevmode{\hyperref[\detokenize{reference/javascript_api:web_editor.context.}]{\sphinxcrossref{
\textless{}anonymous\textgreater{}
}}}
\end{description}\end{quote}


\begin{fulllineitems}
\phantomsection\label{\detokenize{reference/javascript_api:web_editor.context.}}\pysigline{\sphinxbfcode{\sphinxupquote{namespace }}\sphinxbfcode{\sphinxupquote{}}}
\end{fulllineitems}


\end{fulllineitems}

\phantomsection\label{\detokenize{reference/javascript_api:module-web.mixins}}

\begin{fulllineitems}
\phantomsection\label{\detokenize{reference/javascript_api:web.mixins}}\pysigline{\sphinxbfcode{\sphinxupquote{module }}\sphinxbfcode{\sphinxupquote{web.mixins}}}~~\begin{quote}\begin{description}
\item[{Exports}] \leavevmode{\hyperref[\detokenize{reference/javascript_api:web.mixins.}]{\sphinxcrossref{
\textless{}anonymous\textgreater{}
}}}
\item[{Depends On}] \leavevmode\begin{itemize}
\item {} {\hyperref[\detokenize{reference/javascript_api:web.AbstractService}]{\sphinxcrossref{
web.AbstractService
}}}
\item {} {\hyperref[\detokenize{reference/javascript_api:web.Class}]{\sphinxcrossref{
web.Class
}}}
\item {} {\hyperref[\detokenize{reference/javascript_api:web.utils}]{\sphinxcrossref{
web.utils
}}}
\end{itemize}

\end{description}\end{quote}


\begin{fulllineitems}
\phantomsection\label{\detokenize{reference/javascript_api:Events}}\pysiglinewithargsret{\sphinxbfcode{\sphinxupquote{class }}\sphinxbfcode{\sphinxupquote{Events}}}{}{}~\begin{quote}\begin{description}
\item[{Extends}] \leavevmode{\hyperref[\detokenize{reference/javascript_api:web.Class.Class}]{\sphinxcrossref{
Class
}}}
\end{description}\end{quote}

Backbone’s events. Do not ever use it directly, use EventDispatcherMixin instead.

This class just handle the dispatching of events, it is not meant to be extended,
nor used directly. All integration with parenting and automatic unregistration of
events is done in EventDispatcherMixin.

Copyright notice for the following Class:

(c) 2010-2012 Jeremy Ashkenas, DocumentCloud Inc.
Backbone may be freely distributed under the MIT license.
For all details and documentation:
\sphinxurl{http://backbonejs.org}

\end{fulllineitems}



\begin{fulllineitems}
\phantomsection\label{\detokenize{reference/javascript_api:ParentedMixin}}\pysigline{\sphinxbfcode{\sphinxupquote{mixin }}\sphinxbfcode{\sphinxupquote{ParentedMixin}}}
Mixin to structure objects’ life-cycles folowing a parent-children
relationship. Each object can a have a parent and multiple children.
When an object is destroyed, all its children are destroyed too releasing
any resource they could have reserved before.


\begin{fulllineitems}
\phantomsection\label{\detokenize{reference/javascript_api:setParent}}\pysiglinewithargsret{\sphinxbfcode{\sphinxupquote{function }}\sphinxbfcode{\sphinxupquote{setParent}}}{\emph{parent}}{}
Set the parent of the current object. When calling this method, the
parent will also be informed and will return the current object
when its getChildren() method is called. If the current object did
already have a parent, it is unregistered before, which means the
previous parent will not return the current object anymore when its
getChildren() method is called.
\begin{quote}\begin{description}
\item[{Parameters}] \leavevmode\begin{itemize}

\sphinxstylestrong{parent}
\end{itemize}

\end{description}\end{quote}

\end{fulllineitems}



\begin{fulllineitems}
\phantomsection\label{\detokenize{reference/javascript_api:getParent}}\pysiglinewithargsret{\sphinxbfcode{\sphinxupquote{function }}\sphinxbfcode{\sphinxupquote{getParent}}}{}{}
Return the current parent of the object (or null).

\end{fulllineitems}



\begin{fulllineitems}
\phantomsection\label{\detokenize{reference/javascript_api:getChildren}}\pysiglinewithargsret{\sphinxbfcode{\sphinxupquote{function }}\sphinxbfcode{\sphinxupquote{getChildren}}}{}{}
Return a list of the children of the current object.

\end{fulllineitems}



\begin{fulllineitems}
\phantomsection\label{\detokenize{reference/javascript_api:isDestroyed}}\pysiglinewithargsret{\sphinxbfcode{\sphinxupquote{function }}\sphinxbfcode{\sphinxupquote{isDestroyed}}}{}{}
Returns true if destroy() was called on the current object.

\end{fulllineitems}



\begin{fulllineitems}
\phantomsection\label{\detokenize{reference/javascript_api:alive}}\pysiglinewithargsret{\sphinxbfcode{\sphinxupquote{function }}\sphinxbfcode{\sphinxupquote{alive}}}{\emph{promise}\sphinxoptional{, \emph{reject}}}{{ $\rightarrow$ \$.Deferred}}
Utility method to only execute asynchronous actions if the current
object has not been destroyed.
\begin{quote}\begin{description}
\item[{Parameters}] \leavevmode\begin{itemize}

\sphinxstylestrong{promise} (\sphinxstyleliteralemphasis{\sphinxupquote{jQuery.Deferred}}) \textendash{} The promise representing the asynchronous
                            action.

\sphinxstylestrong{reject}=\sphinxstyleemphasis{false} (\sphinxstyleliteralemphasis{\sphinxupquote{bool}}) \textendash{} If true, the returned promise will be
                             rejected with no arguments if the current
                             object is destroyed. If false, the
                             returned promise will never be resolved
                             or rejected.
\end{itemize}

\item[{Returns}] \leavevmode
A promise that will mirror the given promise if
                      everything goes fine but will either be rejected
                      with no arguments or never resolved if the
                      current object is destroyed.

\item[{Return Type}] \leavevmode
\sphinxstyleliteralemphasis{\sphinxupquote{jQuery.Deferred}}

\end{description}\end{quote}

\end{fulllineitems}



\begin{fulllineitems}
\phantomsection\label{\detokenize{reference/javascript_api:destroy}}\pysiglinewithargsret{\sphinxbfcode{\sphinxupquote{function }}\sphinxbfcode{\sphinxupquote{destroy}}}{}{}
Inform the object it should destroy itself, releasing any
resource it could have reserved.

\end{fulllineitems}



\begin{fulllineitems}
\phantomsection\label{\detokenize{reference/javascript_api:findAncestor}}\pysiglinewithargsret{\sphinxbfcode{\sphinxupquote{function }}\sphinxbfcode{\sphinxupquote{findAncestor}}}{\emph{predicate}}{}
Find the closest ancestor matching predicate
\begin{quote}\begin{description}
\item[{Parameters}] \leavevmode\begin{itemize}

\sphinxstylestrong{predicate}
\end{itemize}

\end{description}\end{quote}

\end{fulllineitems}


\end{fulllineitems}



\begin{fulllineitems}
\phantomsection\label{\detokenize{reference/javascript_api:web.mixins.}}\pysigline{\sphinxbfcode{\sphinxupquote{namespace }}\sphinxbfcode{\sphinxupquote{}}}~

\begin{fulllineitems}
\phantomsection\label{\detokenize{reference/javascript_api:ParentedMixin}}\pysigline{\sphinxbfcode{\sphinxupquote{mixin }}\sphinxbfcode{\sphinxupquote{ParentedMixin}}}
Mixin to structure objects’ life-cycles folowing a parent-children
relationship. Each object can a have a parent and multiple children.
When an object is destroyed, all its children are destroyed too releasing
any resource they could have reserved before.


\begin{fulllineitems}
\phantomsection\label{\detokenize{reference/javascript_api:setParent}}\pysiglinewithargsret{\sphinxbfcode{\sphinxupquote{function }}\sphinxbfcode{\sphinxupquote{setParent}}}{\emph{parent}}{}
Set the parent of the current object. When calling this method, the
parent will also be informed and will return the current object
when its getChildren() method is called. If the current object did
already have a parent, it is unregistered before, which means the
previous parent will not return the current object anymore when its
getChildren() method is called.
\begin{quote}\begin{description}
\item[{Parameters}] \leavevmode\begin{itemize}

\sphinxstylestrong{parent}
\end{itemize}

\end{description}\end{quote}

\end{fulllineitems}



\begin{fulllineitems}
\phantomsection\label{\detokenize{reference/javascript_api:getParent}}\pysiglinewithargsret{\sphinxbfcode{\sphinxupquote{function }}\sphinxbfcode{\sphinxupquote{getParent}}}{}{}
Return the current parent of the object (or null).

\end{fulllineitems}



\begin{fulllineitems}
\phantomsection\label{\detokenize{reference/javascript_api:getChildren}}\pysiglinewithargsret{\sphinxbfcode{\sphinxupquote{function }}\sphinxbfcode{\sphinxupquote{getChildren}}}{}{}
Return a list of the children of the current object.

\end{fulllineitems}



\begin{fulllineitems}
\phantomsection\label{\detokenize{reference/javascript_api:isDestroyed}}\pysiglinewithargsret{\sphinxbfcode{\sphinxupquote{function }}\sphinxbfcode{\sphinxupquote{isDestroyed}}}{}{}
Returns true if destroy() was called on the current object.

\end{fulllineitems}



\begin{fulllineitems}
\phantomsection\label{\detokenize{reference/javascript_api:alive}}\pysiglinewithargsret{\sphinxbfcode{\sphinxupquote{function }}\sphinxbfcode{\sphinxupquote{alive}}}{\emph{promise}\sphinxoptional{, \emph{reject}}}{{ $\rightarrow$ \$.Deferred}}
Utility method to only execute asynchronous actions if the current
object has not been destroyed.
\begin{quote}\begin{description}
\item[{Parameters}] \leavevmode\begin{itemize}

\sphinxstylestrong{promise} (\sphinxstyleliteralemphasis{\sphinxupquote{jQuery.Deferred}}) \textendash{} The promise representing the asynchronous
                            action.

\sphinxstylestrong{reject}=\sphinxstyleemphasis{false} (\sphinxstyleliteralemphasis{\sphinxupquote{bool}}) \textendash{} If true, the returned promise will be
                             rejected with no arguments if the current
                             object is destroyed. If false, the
                             returned promise will never be resolved
                             or rejected.
\end{itemize}

\item[{Returns}] \leavevmode
A promise that will mirror the given promise if
                      everything goes fine but will either be rejected
                      with no arguments or never resolved if the
                      current object is destroyed.

\item[{Return Type}] \leavevmode
\sphinxstyleliteralemphasis{\sphinxupquote{jQuery.Deferred}}

\end{description}\end{quote}

\end{fulllineitems}



\begin{fulllineitems}
\phantomsection\label{\detokenize{reference/javascript_api:destroy}}\pysiglinewithargsret{\sphinxbfcode{\sphinxupquote{function }}\sphinxbfcode{\sphinxupquote{destroy}}}{}{}
Inform the object it should destroy itself, releasing any
resource it could have reserved.

\end{fulllineitems}



\begin{fulllineitems}
\phantomsection\label{\detokenize{reference/javascript_api:findAncestor}}\pysiglinewithargsret{\sphinxbfcode{\sphinxupquote{function }}\sphinxbfcode{\sphinxupquote{findAncestor}}}{\emph{predicate}}{}
Find the closest ancestor matching predicate
\begin{quote}\begin{description}
\item[{Parameters}] \leavevmode\begin{itemize}

\sphinxstylestrong{predicate}
\end{itemize}

\end{description}\end{quote}

\end{fulllineitems}


\end{fulllineitems}



\begin{fulllineitems}
\phantomsection\label{\detokenize{reference/javascript_api:EventDispatcherMixin}}\pysigline{\sphinxbfcode{\sphinxupquote{mixin }}\sphinxbfcode{\sphinxupquote{EventDispatcherMixin}}}
Mixin containing an event system. Events are also registered by specifying the target object
(the object which will receive the event when it is raised). Both the event-emitting object
and the target object store or reference to each other. This is used to correctly remove all
reference to the event handler when any of the object is destroyed (when the destroy() method
from ParentedMixin is called). Removing those references is necessary to avoid memory leak
and phantom events (events which are raised and sent to a previously destroyed object).

\end{fulllineitems}


\end{fulllineitems}



\begin{fulllineitems}
\phantomsection\label{\detokenize{reference/javascript_api:EventDispatcherMixin}}\pysigline{\sphinxbfcode{\sphinxupquote{mixin }}\sphinxbfcode{\sphinxupquote{EventDispatcherMixin}}}
Mixin containing an event system. Events are also registered by specifying the target object
(the object which will receive the event when it is raised). Both the event-emitting object
and the target object store or reference to each other. This is used to correctly remove all
reference to the event handler when any of the object is destroyed (when the destroy() method
from ParentedMixin is called). Removing those references is necessary to avoid memory leak
and phantom events (events which are raised and sent to a previously destroyed object).

\end{fulllineitems}


\end{fulllineitems}

\phantomsection\label{\detokenize{reference/javascript_api:module-website.planner}}

\begin{fulllineitems}
\phantomsection\label{\detokenize{reference/javascript_api:website.planner}}\pysigline{\sphinxbfcode{\sphinxupquote{module }}\sphinxbfcode{\sphinxupquote{website.planner}}}~~\begin{quote}\begin{description}
\item[{Exports}] \leavevmode{\hyperref[\detokenize{reference/javascript_api:website.planner.WebsitePlannerLauncher}]{\sphinxcrossref{
WebsitePlannerLauncher
}}}
\item[{Depends On}] \leavevmode\begin{itemize}
\item {} {\hyperref[\detokenize{reference/javascript_api:web.Widget}]{\sphinxcrossref{
web.Widget
}}}
\item {} {\hyperref[\detokenize{reference/javascript_api:web.planner.common}]{\sphinxcrossref{
web.planner.common
}}}
\item {} {\hyperref[\detokenize{reference/javascript_api:web.session}]{\sphinxcrossref{
web.session
}}}
\item {} {\hyperref[\detokenize{reference/javascript_api:website.navbar}]{\sphinxcrossref{
website.navbar
}}}
\end{itemize}

\end{description}\end{quote}


\begin{fulllineitems}
\phantomsection\label{\detokenize{reference/javascript_api:WebsitePlannerLauncher}}\pysiglinewithargsret{\sphinxbfcode{\sphinxupquote{class }}\sphinxbfcode{\sphinxupquote{WebsitePlannerLauncher}}}{}{}~\begin{quote}\begin{description}
\item[{Extends}] \leavevmode
PlannerLauncher

\end{description}\end{quote}

\end{fulllineitems}


\end{fulllineitems}

\phantomsection\label{\detokenize{reference/javascript_api:module-website.newMenu}}

\begin{fulllineitems}
\phantomsection\label{\detokenize{reference/javascript_api:website.newMenu}}\pysigline{\sphinxbfcode{\sphinxupquote{module }}\sphinxbfcode{\sphinxupquote{website.newMenu}}}~~\begin{quote}\begin{description}
\item[{Exports}] \leavevmode{\hyperref[\detokenize{reference/javascript_api:website.newMenu.NewContentMenu}]{\sphinxcrossref{
NewContentMenu
}}}
\item[{Depends On}] \leavevmode\begin{itemize}
\item {} {\hyperref[\detokenize{reference/javascript_api:web.core}]{\sphinxcrossref{
web.core
}}}
\item {} {\hyperref[\detokenize{reference/javascript_api:website.navbar}]{\sphinxcrossref{
website.navbar
}}}
\item {} {\hyperref[\detokenize{reference/javascript_api:website.utils}]{\sphinxcrossref{
website.utils
}}}
\end{itemize}

\end{description}\end{quote}


\begin{fulllineitems}
\phantomsection\label{\detokenize{reference/javascript_api:NewContentMenu}}\pysiglinewithargsret{\sphinxbfcode{\sphinxupquote{class }}\sphinxbfcode{\sphinxupquote{NewContentMenu}}}{}{}~\begin{quote}\begin{description}
\item[{Extends}] \leavevmode
WebsiteNavbarActionWidget

\end{description}\end{quote}

\end{fulllineitems}


\end{fulllineitems}

\phantomsection\label{\detokenize{reference/javascript_api:module-web.KanbanController}}

\begin{fulllineitems}
\phantomsection\label{\detokenize{reference/javascript_api:web.KanbanController}}\pysigline{\sphinxbfcode{\sphinxupquote{module }}\sphinxbfcode{\sphinxupquote{web.KanbanController}}}~~\begin{quote}\begin{description}
\item[{Exports}] \leavevmode{\hyperref[\detokenize{reference/javascript_api:web.KanbanController.KanbanController}]{\sphinxcrossref{
KanbanController
}}}
\item[{Depends On}] \leavevmode\begin{itemize}
\item {} {\hyperref[\detokenize{reference/javascript_api:web.BasicController}]{\sphinxcrossref{
web.BasicController
}}}
\item {} {\hyperref[\detokenize{reference/javascript_api:web.Context}]{\sphinxcrossref{
web.Context
}}}
\item {} {\hyperref[\detokenize{reference/javascript_api:web.Domain}]{\sphinxcrossref{
web.Domain
}}}
\item {} {\hyperref[\detokenize{reference/javascript_api:web.core}]{\sphinxcrossref{
web.core
}}}
\item {} {\hyperref[\detokenize{reference/javascript_api:web.view_dialogs}]{\sphinxcrossref{
web.view\_dialogs
}}}
\end{itemize}

\end{description}\end{quote}


\begin{fulllineitems}
\phantomsection\label{\detokenize{reference/javascript_api:KanbanController}}\pysiglinewithargsret{\sphinxbfcode{\sphinxupquote{class }}\sphinxbfcode{\sphinxupquote{KanbanController}}}{}{}~\begin{quote}\begin{description}
\item[{Extends}] \leavevmode{\hyperref[\detokenize{reference/javascript_api:web.BasicController.BasicController}]{\sphinxcrossref{
BasicController
}}}
\end{description}\end{quote}


\begin{fulllineitems}
\phantomsection\label{\detokenize{reference/javascript_api:renderButtons}}\pysiglinewithargsret{\sphinxbfcode{\sphinxupquote{method }}\sphinxbfcode{\sphinxupquote{renderButtons}}}{}{}
Extends the renderButtons function of ListView by adding an event listener
on the import button.

\end{fulllineitems}



\begin{fulllineitems}
\phantomsection\label{\detokenize{reference/javascript_api:update}}\pysiglinewithargsret{\sphinxbfcode{\sphinxupquote{method }}\sphinxbfcode{\sphinxupquote{update}}}{}{{ $\rightarrow$ Deferred}}
Override update method to recompute createColumnEnabled.
\begin{quote}\begin{description}
\item[{Return Type}] \leavevmode
\sphinxstyleliteralemphasis{\sphinxupquote{Deferred}}

\end{description}\end{quote}

\end{fulllineitems}


\end{fulllineitems}


\end{fulllineitems}

\phantomsection\label{\detokenize{reference/javascript_api:module-website_sale_stock.website_sale}}

\begin{fulllineitems}
\phantomsection\label{\detokenize{reference/javascript_api:website_sale_stock.website_sale}}\pysigline{\sphinxbfcode{\sphinxupquote{module }}\sphinxbfcode{\sphinxupquote{website\_sale\_stock.website\_sale}}}~~\begin{quote}\begin{description}
\item[{Exports}] \leavevmode{\hyperref[\detokenize{reference/javascript_api:website_sale_stock.website_sale.}]{\sphinxcrossref{
\textless{}anonymous\textgreater{}
}}}
\item[{Depends On}] \leavevmode\begin{itemize}
\item {} {\hyperref[\detokenize{reference/javascript_api:web.ajax}]{\sphinxcrossref{
web.ajax
}}}
\item {} {\hyperref[\detokenize{reference/javascript_api:web.core}]{\sphinxcrossref{
web.core
}}}
\item {} {\hyperref[\detokenize{reference/javascript_api:web_editor.base}]{\sphinxcrossref{
web\_editor.base
}}}
\end{itemize}

\end{description}\end{quote}


\begin{fulllineitems}
\phantomsection\label{\detokenize{reference/javascript_api:website_sale_stock.website_sale.}}\pysigline{\sphinxbfcode{\sphinxupquote{namespace }}\sphinxbfcode{\sphinxupquote{}}}
\end{fulllineitems}


\end{fulllineitems}

\phantomsection\label{\detokenize{reference/javascript_api:module-web.Pager}}

\begin{fulllineitems}
\phantomsection\label{\detokenize{reference/javascript_api:web.Pager}}\pysigline{\sphinxbfcode{\sphinxupquote{module }}\sphinxbfcode{\sphinxupquote{web.Pager}}}~~\begin{quote}\begin{description}
\item[{Exports}] \leavevmode{\hyperref[\detokenize{reference/javascript_api:web.Pager.Pager}]{\sphinxcrossref{
Pager
}}}
\item[{Depends On}] \leavevmode\begin{itemize}
\item {} {\hyperref[\detokenize{reference/javascript_api:web.Widget}]{\sphinxcrossref{
web.Widget
}}}
\item {} {\hyperref[\detokenize{reference/javascript_api:web.utils}]{\sphinxcrossref{
web.utils
}}}
\end{itemize}

\end{description}\end{quote}


\begin{fulllineitems}
\phantomsection\label{\detokenize{reference/javascript_api:Pager}}\pysiglinewithargsret{\sphinxbfcode{\sphinxupquote{class }}\sphinxbfcode{\sphinxupquote{Pager}}}{\sphinxoptional{\emph{parent}}\sphinxoptional{, \emph{size}}\sphinxoptional{, \emph{current\_min}}\sphinxoptional{, \emph{limit}}, \emph{options}}{}~\begin{quote}\begin{description}
\item[{Extends}] \leavevmode{\hyperref[\detokenize{reference/javascript_api:web.Widget.Widget}]{\sphinxcrossref{
Widget
}}}
\item[{Parameters}] \leavevmode\begin{itemize}

\sphinxstylestrong{parent} ({\hyperref[\detokenize{reference/javascript_api:Widget}]{\sphinxcrossref{\sphinxstyleliteralemphasis{\sphinxupquote{Widget}}}}}) \textendash{} the parent widget

\sphinxstylestrong{size} (\sphinxstyleliteralemphasis{\sphinxupquote{int}}) \textendash{} the total number of elements

\sphinxstylestrong{current\_min} (\sphinxstyleliteralemphasis{\sphinxupquote{int}}) \textendash{} the first element of the current\_page

\sphinxstylestrong{limit} (\sphinxstyleliteralemphasis{\sphinxupquote{int}}) \textendash{} the number of elements per page

\sphinxstylestrong{options} ({\hyperref[\detokenize{reference/javascript_api:web.Pager.PagerOptions}]{\sphinxcrossref{\sphinxstyleliteralemphasis{\sphinxupquote{PagerOptions}}}}})
\end{itemize}

\end{description}\end{quote}


\begin{fulllineitems}
\phantomsection\label{\detokenize{reference/javascript_api:start}}\pysiglinewithargsret{\sphinxbfcode{\sphinxupquote{method }}\sphinxbfcode{\sphinxupquote{start}}}{}{{ $\rightarrow$ jQuery.Deferred}}
Renders the pager
\begin{quote}\begin{description}
\item[{Return Type}] \leavevmode
\sphinxstyleliteralemphasis{\sphinxupquote{jQuery.Deferred}}

\end{description}\end{quote}

\end{fulllineitems}



\begin{fulllineitems}
\phantomsection\label{\detokenize{reference/javascript_api:disable}}\pysiglinewithargsret{\sphinxbfcode{\sphinxupquote{method }}\sphinxbfcode{\sphinxupquote{disable}}}{}{}
Disables the pager’s arrows and the edition

\end{fulllineitems}



\begin{fulllineitems}
\phantomsection\label{\detokenize{reference/javascript_api:enable}}\pysiglinewithargsret{\sphinxbfcode{\sphinxupquote{method }}\sphinxbfcode{\sphinxupquote{enable}}}{}{}
Enables the pager’s arrows and the edition

\end{fulllineitems}



\begin{fulllineitems}
\phantomsection\label{\detokenize{reference/javascript_api:next}}\pysiglinewithargsret{\sphinxbfcode{\sphinxupquote{method }}\sphinxbfcode{\sphinxupquote{next}}}{}{}
Executes the next action on the pager

\end{fulllineitems}



\begin{fulllineitems}
\phantomsection\label{\detokenize{reference/javascript_api:previous}}\pysiglinewithargsret{\sphinxbfcode{\sphinxupquote{method }}\sphinxbfcode{\sphinxupquote{previous}}}{}{}
Executes the previous action on the pager

\end{fulllineitems}



\begin{fulllineitems}
\phantomsection\label{\detokenize{reference/javascript_api:updateState}}\pysiglinewithargsret{\sphinxbfcode{\sphinxupquote{method }}\sphinxbfcode{\sphinxupquote{updateState}}}{\sphinxoptional{\emph{state}}\sphinxoptional{, \emph{options}}}{}
Sets the state of the pager and renders it
\begin{quote}\begin{description}
\item[{Parameters}] \leavevmode\begin{itemize}

\sphinxstylestrong{state} (\sphinxstyleliteralemphasis{\sphinxupquote{Object}}) \textendash{} the values to update (size, current\_min and limit)

\sphinxstylestrong{options} ({\hyperref[\detokenize{reference/javascript_api:web.Pager.UpdateStateOptions}]{\sphinxcrossref{\sphinxstyleliteralemphasis{\sphinxupquote{UpdateStateOptions}}}}})
\end{itemize}

\end{description}\end{quote}


\begin{fulllineitems}
\phantomsection\label{\detokenize{reference/javascript_api:UpdateStateOptions}}\pysiglinewithargsret{\sphinxbfcode{\sphinxupquote{class }}\sphinxbfcode{\sphinxupquote{UpdateStateOptions}}}{}{}~

\begin{fulllineitems}
\phantomsection\label{\detokenize{reference/javascript_api:notifyChange}}\pysigline{\sphinxbfcode{\sphinxupquote{attribute }}\sphinxbfcode{\sphinxupquote{notifyChange}} boolean}~\begin{description}
\item[{set to true to make the pager}] \leavevmode
notify the environment that its state changed

\end{description}

\end{fulllineitems}


\end{fulllineitems}


\end{fulllineitems}



\begin{fulllineitems}
\phantomsection\label{\detokenize{reference/javascript_api:PagerOptions}}\pysiglinewithargsret{\sphinxbfcode{\sphinxupquote{class }}\sphinxbfcode{\sphinxupquote{PagerOptions}}}{}{}~

\begin{fulllineitems}
\phantomsection\label{\detokenize{reference/javascript_api:can_edit}}\pysigline{\sphinxbfcode{\sphinxupquote{attribute }}\sphinxbfcode{\sphinxupquote{can\_edit}} boolean}
editable feature of the pager

\end{fulllineitems}



\begin{fulllineitems}
\phantomsection\label{\detokenize{reference/javascript_api:single_page_hidden}}\pysigline{\sphinxbfcode{\sphinxupquote{attribute }}\sphinxbfcode{\sphinxupquote{single\_page\_hidden}} boolean}~\begin{description}
\item[{(not) to display the pager}] \leavevmode
if only one page

\end{description}

\end{fulllineitems}



\begin{fulllineitems}
\phantomsection\label{\detokenize{reference/javascript_api:validate}}\pysigline{\sphinxbfcode{\sphinxupquote{attribute }}\sphinxbfcode{\sphinxupquote{validate}} function}~\begin{description}
\item[{callback returning a Deferred to}] \leavevmode
validate changes

\end{description}

\end{fulllineitems}


\end{fulllineitems}


\end{fulllineitems}


\end{fulllineitems}

\phantomsection\label{\detokenize{reference/javascript_api:module-stock.ReportWidget}}

\begin{fulllineitems}
\phantomsection\label{\detokenize{reference/javascript_api:stock.ReportWidget}}\pysigline{\sphinxbfcode{\sphinxupquote{module }}\sphinxbfcode{\sphinxupquote{stock.ReportWidget}}}~~\begin{quote}\begin{description}
\item[{Exports}] \leavevmode{\hyperref[\detokenize{reference/javascript_api:stock.ReportWidget.ReportWidget}]{\sphinxcrossref{
ReportWidget
}}}
\item[{Depends On}] \leavevmode\begin{itemize}
\item {} {\hyperref[\detokenize{reference/javascript_api:web.Widget}]{\sphinxcrossref{
web.Widget
}}}
\item {} {\hyperref[\detokenize{reference/javascript_api:web.core}]{\sphinxcrossref{
web.core
}}}
\end{itemize}

\end{description}\end{quote}


\begin{fulllineitems}
\phantomsection\label{\detokenize{reference/javascript_api:ReportWidget}}\pysiglinewithargsret{\sphinxbfcode{\sphinxupquote{class }}\sphinxbfcode{\sphinxupquote{ReportWidget}}}{\emph{parent}}{}~\begin{quote}\begin{description}
\item[{Extends}] \leavevmode{\hyperref[\detokenize{reference/javascript_api:web.Widget.Widget}]{\sphinxcrossref{
Widget
}}}
\item[{Parameters}] \leavevmode\begin{itemize}

\sphinxstylestrong{parent}
\end{itemize}

\end{description}\end{quote}

\end{fulllineitems}


\end{fulllineitems}

\phantomsection\label{\detokenize{reference/javascript_api:module-web.FormController}}

\begin{fulllineitems}
\phantomsection\label{\detokenize{reference/javascript_api:web.FormController}}\pysigline{\sphinxbfcode{\sphinxupquote{module }}\sphinxbfcode{\sphinxupquote{web.FormController}}}~~\begin{quote}\begin{description}
\item[{Exports}] \leavevmode{\hyperref[\detokenize{reference/javascript_api:web.FormController.FormController}]{\sphinxcrossref{
FormController
}}}
\item[{Depends On}] \leavevmode\begin{itemize}
\item {} {\hyperref[\detokenize{reference/javascript_api:web.BasicController}]{\sphinxcrossref{
web.BasicController
}}}
\item {} {\hyperref[\detokenize{reference/javascript_api:web.Dialog}]{\sphinxcrossref{
web.Dialog
}}}
\item {} {\hyperref[\detokenize{reference/javascript_api:web.Sidebar}]{\sphinxcrossref{
web.Sidebar
}}}
\item {} {\hyperref[\detokenize{reference/javascript_api:web.core}]{\sphinxcrossref{
web.core
}}}
\item {} {\hyperref[\detokenize{reference/javascript_api:web.view_dialogs}]{\sphinxcrossref{
web.view\_dialogs
}}}
\end{itemize}

\end{description}\end{quote}


\begin{fulllineitems}
\phantomsection\label{\detokenize{reference/javascript_api:FormController}}\pysiglinewithargsret{\sphinxbfcode{\sphinxupquote{class }}\sphinxbfcode{\sphinxupquote{FormController}}}{\emph{parent}, \emph{model}, \emph{renderer}, \emph{params}}{}~\begin{quote}\begin{description}
\item[{Extends}] \leavevmode{\hyperref[\detokenize{reference/javascript_api:web.BasicController.BasicController}]{\sphinxcrossref{
BasicController
}}}
\item[{Parameters}] \leavevmode\begin{itemize}

\sphinxstylestrong{parent}

\sphinxstylestrong{model}

\sphinxstylestrong{renderer}

\sphinxstylestrong{params}
\end{itemize}

\end{description}\end{quote}


\begin{fulllineitems}
\phantomsection\label{\detokenize{reference/javascript_api:autofocus}}\pysiglinewithargsret{\sphinxbfcode{\sphinxupquote{method }}\sphinxbfcode{\sphinxupquote{autofocus}}}{}{}
Calls autofocus on the renderer

\end{fulllineitems}



\begin{fulllineitems}
\phantomsection\label{\detokenize{reference/javascript_api:createRecord}}\pysiglinewithargsret{\sphinxbfcode{\sphinxupquote{method }}\sphinxbfcode{\sphinxupquote{createRecord}}}{\sphinxoptional{\emph{parentID}}}{{ $\rightarrow$ Deferred}}
This method switches the form view in edit mode, with a new record.
\begin{quote}\begin{description}
\item[{Parameters}] \leavevmode\begin{itemize}

\sphinxstylestrong{parentID} (\sphinxstyleliteralemphasis{\sphinxupquote{string}}) \textendash{} if given, the parentID will be used as parent
                           for the new record.
\end{itemize}

\item[{Return Type}] \leavevmode
\sphinxstyleliteralemphasis{\sphinxupquote{Deferred}}

\end{description}\end{quote}

\end{fulllineitems}



\begin{fulllineitems}
\phantomsection\label{\detokenize{reference/javascript_api:getSelectedIds}}\pysiglinewithargsret{\sphinxbfcode{\sphinxupquote{method }}\sphinxbfcode{\sphinxupquote{getSelectedIds}}}{}{{ $\rightarrow$ number{[}{]}}}
Returns the current res\_id, wrapped in a list. This is only used by the
sidebar (and the debugmanager)
\begin{quote}\begin{description}
\item[{Returns}] \leavevmode
either {[}current res\_id{]} or {[}{]}

\item[{Return Type}] \leavevmode
\sphinxstyleliteralemphasis{\sphinxupquote{Array}}\textless{}\sphinxstyleliteralemphasis{\sphinxupquote{number}}\textgreater{}

\end{description}\end{quote}

\end{fulllineitems}



\begin{fulllineitems}
\phantomsection\label{\detokenize{reference/javascript_api:on_attach_callback}}\pysiglinewithargsret{\sphinxbfcode{\sphinxupquote{method }}\sphinxbfcode{\sphinxupquote{on\_attach\_callback}}}{}{}
Called each time the form view is attached into the DOM

\end{fulllineitems}



\begin{fulllineitems}
\phantomsection\label{\detokenize{reference/javascript_api:renderButtons}}\pysiglinewithargsret{\sphinxbfcode{\sphinxupquote{method }}\sphinxbfcode{\sphinxupquote{renderButtons}}}{\emph{\$node}}{}
Render buttons for the control panel.  The form view can be rendered in
a dialog, and in that case, if we have buttons defined in the footer, we
have to use them instead of the standard buttons.
\begin{quote}\begin{description}
\item[{Parameters}] \leavevmode\begin{itemize}

\sphinxstylestrong{\$node} (\sphinxstyleliteralemphasis{\sphinxupquote{jQueryElement}})
\end{itemize}

\end{description}\end{quote}

\end{fulllineitems}



\begin{fulllineitems}
\phantomsection\label{\detokenize{reference/javascript_api:renderPager}}\pysiglinewithargsret{\sphinxbfcode{\sphinxupquote{method }}\sphinxbfcode{\sphinxupquote{renderPager}}}{\emph{\$node}, \emph{options}}{}
The form view has to prevent a click on the pager if the form is dirty
\begin{quote}\begin{description}
\item[{Parameters}] \leavevmode\begin{itemize}

\sphinxstylestrong{\$node} (\sphinxstyleliteralemphasis{\sphinxupquote{jQueryElement}})

\sphinxstylestrong{options} (\sphinxstyleliteralemphasis{\sphinxupquote{Object}})
\end{itemize}

\end{description}\end{quote}

\end{fulllineitems}



\begin{fulllineitems}
\phantomsection\label{\detokenize{reference/javascript_api:renderSidebar}}\pysiglinewithargsret{\sphinxbfcode{\sphinxupquote{method }}\sphinxbfcode{\sphinxupquote{renderSidebar}}}{\sphinxoptional{\emph{\$node}}}{}
Instantiate and render the sidebar if a sidebar is requested
Sets this.sidebar
\begin{quote}\begin{description}
\item[{Parameters}] \leavevmode\begin{itemize}

\sphinxstylestrong{\$node} (\sphinxstyleliteralemphasis{\sphinxupquote{jQuery}}) \textendash{} a jQuery node where the sidebar should be
  inserted
\end{itemize}

\end{description}\end{quote}

\end{fulllineitems}



\begin{fulllineitems}
\phantomsection\label{\detokenize{reference/javascript_api:saveRecord}}\pysiglinewithargsret{\sphinxbfcode{\sphinxupquote{method }}\sphinxbfcode{\sphinxupquote{saveRecord}}}{}{}
Show a warning message if the user modified a translated field.  For each
field, the notification provides a link to edit the field’s translations.

\end{fulllineitems}



\begin{fulllineitems}
\phantomsection\label{\detokenize{reference/javascript_api:update}}\pysiglinewithargsret{\sphinxbfcode{\sphinxupquote{method }}\sphinxbfcode{\sphinxupquote{update}}}{\emph{params}, \emph{options}}{}
Overrides to force the viewType to ‘form’, so that we ensure that the
correct fields are reloaded (this is only useful for one2many form views).
\begin{quote}\begin{description}
\item[{Parameters}] \leavevmode\begin{itemize}

\sphinxstylestrong{params}

\sphinxstylestrong{options}
\end{itemize}

\end{description}\end{quote}

\end{fulllineitems}


\end{fulllineitems}


\end{fulllineitems}

\phantomsection\label{\detokenize{reference/javascript_api:module-web.KanbanColumnProgressBar}}

\begin{fulllineitems}
\phantomsection\label{\detokenize{reference/javascript_api:web.KanbanColumnProgressBar}}\pysigline{\sphinxbfcode{\sphinxupquote{module }}\sphinxbfcode{\sphinxupquote{web.KanbanColumnProgressBar}}}~~\begin{quote}\begin{description}
\item[{Exports}] \leavevmode{\hyperref[\detokenize{reference/javascript_api:web.KanbanColumnProgressBar.KanbanColumnProgressBar}]{\sphinxcrossref{
KanbanColumnProgressBar
}}}
\item[{Depends On}] \leavevmode\begin{itemize}
\item {} {\hyperref[\detokenize{reference/javascript_api:web.Widget}]{\sphinxcrossref{
web.Widget
}}}
\item {} {\hyperref[\detokenize{reference/javascript_api:web.session}]{\sphinxcrossref{
web.session
}}}
\item {} {\hyperref[\detokenize{reference/javascript_api:web.utils}]{\sphinxcrossref{
web.utils
}}}
\end{itemize}

\end{description}\end{quote}


\begin{fulllineitems}
\phantomsection\label{\detokenize{reference/javascript_api:KanbanColumnProgressBar}}\pysiglinewithargsret{\sphinxbfcode{\sphinxupquote{class }}\sphinxbfcode{\sphinxupquote{KanbanColumnProgressBar}}}{\emph{parent}, \emph{options}, \emph{columnState}}{}~\begin{quote}\begin{description}
\item[{Extends}] \leavevmode{\hyperref[\detokenize{reference/javascript_api:web.Widget.Widget}]{\sphinxcrossref{
Widget
}}}
\item[{Parameters}] \leavevmode\begin{itemize}

\sphinxstylestrong{parent}

\sphinxstylestrong{options}

\sphinxstylestrong{columnState}
\end{itemize}

\end{description}\end{quote}


\begin{fulllineitems}
\phantomsection\label{\detokenize{reference/javascript_api:ANIMATE}}\pysigline{\sphinxbfcode{\sphinxupquote{attribute }}\sphinxbfcode{\sphinxupquote{ANIMATE}} boolean}
Allows to disable animations for tests.

\end{fulllineitems}


\end{fulllineitems}


\end{fulllineitems}

\phantomsection\label{\detokenize{reference/javascript_api:module-website.contentMenu}}

\begin{fulllineitems}
\phantomsection\label{\detokenize{reference/javascript_api:website.contentMenu}}\pysigline{\sphinxbfcode{\sphinxupquote{module }}\sphinxbfcode{\sphinxupquote{website.contentMenu}}}~~\begin{quote}\begin{description}
\item[{Exports}] \leavevmode{\hyperref[\detokenize{reference/javascript_api:website.contentMenu.}]{\sphinxcrossref{
\textless{}anonymous\textgreater{}
}}}
\item[{Depends On}] \leavevmode\begin{itemize}
\item {} {\hyperref[\detokenize{reference/javascript_api:web.Dialog}]{\sphinxcrossref{
web.Dialog
}}}
\item {} {\hyperref[\detokenize{reference/javascript_api:web.Widget}]{\sphinxcrossref{
web.Widget
}}}
\item {} {\hyperref[\detokenize{reference/javascript_api:web.core}]{\sphinxcrossref{
web.core
}}}
\item {} {\hyperref[\detokenize{reference/javascript_api:web.time}]{\sphinxcrossref{
web.time
}}}
\item {} {\hyperref[\detokenize{reference/javascript_api:web_editor.context}]{\sphinxcrossref{
web\_editor.context
}}}
\item {} {\hyperref[\detokenize{reference/javascript_api:web_editor.widget}]{\sphinxcrossref{
web\_editor.widget
}}}
\item {} {\hyperref[\detokenize{reference/javascript_api:website.WebsiteRoot}]{\sphinxcrossref{
website.WebsiteRoot
}}}
\item {} {\hyperref[\detokenize{reference/javascript_api:website.navbar}]{\sphinxcrossref{
website.navbar
}}}
\end{itemize}

\end{description}\end{quote}


\begin{fulllineitems}
\phantomsection\label{\detokenize{reference/javascript_api:website.contentMenu.}}\pysigline{\sphinxbfcode{\sphinxupquote{namespace }}\sphinxbfcode{\sphinxupquote{}}}
\end{fulllineitems}


\end{fulllineitems}

\phantomsection\label{\detokenize{reference/javascript_api:module-web.BasicController}}

\begin{fulllineitems}
\phantomsection\label{\detokenize{reference/javascript_api:web.BasicController}}\pysigline{\sphinxbfcode{\sphinxupquote{module }}\sphinxbfcode{\sphinxupquote{web.BasicController}}}~~\begin{quote}\begin{description}
\item[{Exports}] \leavevmode{\hyperref[\detokenize{reference/javascript_api:web.BasicController.BasicController}]{\sphinxcrossref{
BasicController
}}}
\item[{Depends On}] \leavevmode\begin{itemize}
\item {} {\hyperref[\detokenize{reference/javascript_api:web.AbstractController}]{\sphinxcrossref{
web.AbstractController
}}}
\item {} {\hyperref[\detokenize{reference/javascript_api:web.Dialog}]{\sphinxcrossref{
web.Dialog
}}}
\item {} {\hyperref[\detokenize{reference/javascript_api:web.FieldManagerMixin}]{\sphinxcrossref{
web.FieldManagerMixin
}}}
\item {} {\hyperref[\detokenize{reference/javascript_api:web.Pager}]{\sphinxcrossref{
web.Pager
}}}
\item {} {\hyperref[\detokenize{reference/javascript_api:web.core}]{\sphinxcrossref{
web.core
}}}
\end{itemize}

\end{description}\end{quote}


\begin{fulllineitems}
\phantomsection\label{\detokenize{reference/javascript_api:BasicController}}\pysiglinewithargsret{\sphinxbfcode{\sphinxupquote{class }}\sphinxbfcode{\sphinxupquote{BasicController}}}{\emph{parent}, \emph{model}, \emph{renderer}, \emph{params}}{}~\begin{quote}\begin{description}
\item[{Extends}] \leavevmode{\hyperref[\detokenize{reference/javascript_api:web.AbstractController.AbstractController}]{\sphinxcrossref{
AbstractController
}}}
\item[{Mixes}] \leavevmode\begin{itemize}
\item {} {\hyperref[\detokenize{reference/javascript_api:web.FieldManagerMixin.FieldManagerMixin}]{\sphinxcrossref{
FieldManagerMixin
}}}
\end{itemize}

\item[{Parameters}] \leavevmode\begin{itemize}

\sphinxstylestrong{parent}

\sphinxstylestrong{model}

\sphinxstylestrong{renderer}

\sphinxstylestrong{params} ({\hyperref[\detokenize{reference/javascript_api:web.BasicController.BasicControllerParams}]{\sphinxcrossref{\sphinxstyleliteralemphasis{\sphinxupquote{BasicControllerParams}}}}})
\end{itemize}

\end{description}\end{quote}


\begin{fulllineitems}
\phantomsection\label{\detokenize{reference/javascript_api:canBeDiscarded}}\pysiglinewithargsret{\sphinxbfcode{\sphinxupquote{method }}\sphinxbfcode{\sphinxupquote{canBeDiscarded}}}{\sphinxoptional{\emph{recordID}}}{{ $\rightarrow$ Deferred\textless{}boolean\textgreater{}}}
Determines if we can discard the current changes. If the model is not
dirty, that is not a problem. However, if it is dirty, we have to ask
the user for confirmation.
\begin{quote}\begin{description}
\item[{Parameters}] \leavevmode\begin{itemize}

\sphinxstylestrong{recordID} (\sphinxstyleliteralemphasis{\sphinxupquote{string}}) \textendash{} default to main recordID
\end{itemize}

\item[{Returns}] \leavevmode
resolved if can be discarded, a boolean value is given to tells
         if there is something to discard or not
         rejected otherwise

\item[{Return Type}] \leavevmode
\sphinxstyleliteralemphasis{\sphinxupquote{Deferred}}\textless{}\sphinxstyleliteralemphasis{\sphinxupquote{boolean}}\textgreater{}

\end{description}\end{quote}

\end{fulllineitems}



\begin{fulllineitems}
\phantomsection\label{\detokenize{reference/javascript_api:canBeSaved}}\pysiglinewithargsret{\sphinxbfcode{\sphinxupquote{method }}\sphinxbfcode{\sphinxupquote{canBeSaved}}}{\sphinxoptional{\emph{recordID}}}{{ $\rightarrow$ boolean}}
Ask the renderer if all associated field widget are in a valid state for
saving (valid value and non-empty value for required fields). If this is
not the case, this notifies the user with a warning containing the names
of the invalid fields.

Note: changing the style of invalid fields is the renderer’s job.
\begin{quote}\begin{description}
\item[{Parameters}] \leavevmode\begin{itemize}

\sphinxstylestrong{recordID} (\sphinxstyleliteralemphasis{\sphinxupquote{string}}) \textendash{} default to main recordID
\end{itemize}

\item[{Return Type}] \leavevmode
\sphinxstyleliteralemphasis{\sphinxupquote{boolean}}

\end{description}\end{quote}

\end{fulllineitems}



\begin{fulllineitems}
\phantomsection\label{\detokenize{reference/javascript_api:discardChanges}}\pysiglinewithargsret{\sphinxbfcode{\sphinxupquote{method }}\sphinxbfcode{\sphinxupquote{discardChanges}}}{\emph{recordID}, \emph{options}}{}
Waits for the mutex to be unlocked and then calls \_.discardChanges.
This ensures that the confirm dialog isn’t displayed directly if there is
a pending ‘write’ rpc.
\begin{quote}\begin{description}
\item[{Parameters}] \leavevmode\begin{itemize}

\sphinxstylestrong{recordID}

\sphinxstylestrong{options}
\end{itemize}

\end{description}\end{quote}

\end{fulllineitems}



\begin{fulllineitems}
\phantomsection\label{\detokenize{reference/javascript_api:getSelectedIds}}\pysiglinewithargsret{\sphinxbfcode{\sphinxupquote{method }}\sphinxbfcode{\sphinxupquote{getSelectedIds}}}{}{{ $\rightarrow$ Array}}
Method that will be overriden by the views with the ability to have selected ids
\begin{quote}\begin{description}
\item[{Return Type}] \leavevmode
\sphinxstyleliteralemphasis{\sphinxupquote{Array}}

\end{description}\end{quote}

\end{fulllineitems}



\begin{fulllineitems}
\phantomsection\label{\detokenize{reference/javascript_api:saveRecord}}\pysiglinewithargsret{\sphinxbfcode{\sphinxupquote{method }}\sphinxbfcode{\sphinxupquote{saveRecord}}}{\sphinxoptional{\emph{recordID}}\sphinxoptional{, \emph{options}}}{{ $\rightarrow$ Deferred}}
Saves the record whose ID is given if necessary (@see \_saveRecord).
\begin{quote}\begin{description}
\item[{Parameters}] \leavevmode\begin{itemize}

\sphinxstylestrong{recordID} (\sphinxstyleliteralemphasis{\sphinxupquote{string}}) \textendash{} default to main recordID

\sphinxstylestrong{options} (\sphinxstyleliteralemphasis{\sphinxupquote{Object}})
\end{itemize}

\item[{Returns}] \leavevmode
Resolved with the list of field names (whose value has been modified)
       Rejected if the record can’t be saved

\item[{Return Type}] \leavevmode
\sphinxstyleliteralemphasis{\sphinxupquote{Deferred}}

\end{description}\end{quote}

\end{fulllineitems}



\begin{fulllineitems}
\phantomsection\label{\detokenize{reference/javascript_api:BasicControllerParams}}\pysiglinewithargsret{\sphinxbfcode{\sphinxupquote{class }}\sphinxbfcode{\sphinxupquote{BasicControllerParams}}}{}{}~

\begin{fulllineitems}
\phantomsection\label{\detokenize{reference/javascript_api:archiveEnabled}}\pysigline{\sphinxbfcode{\sphinxupquote{attribute }}\sphinxbfcode{\sphinxupquote{archiveEnabled}} boolean}
\end{fulllineitems}



\begin{fulllineitems}
\phantomsection\label{\detokenize{reference/javascript_api:confirmOnDelete}}\pysigline{\sphinxbfcode{\sphinxupquote{attribute }}\sphinxbfcode{\sphinxupquote{confirmOnDelete}} boolean}
\end{fulllineitems}



\begin{fulllineitems}
\phantomsection\label{\detokenize{reference/javascript_api:hasButtons}}\pysigline{\sphinxbfcode{\sphinxupquote{attribute }}\sphinxbfcode{\sphinxupquote{hasButtons}} boolean}
\end{fulllineitems}


\end{fulllineitems}


\end{fulllineitems}


\end{fulllineitems}

\phantomsection\label{\detokenize{reference/javascript_api:module-pos_reprint.pos_reprint}}

\begin{fulllineitems}
\phantomsection\label{\detokenize{reference/javascript_api:pos_reprint.pos_reprint}}\pysigline{\sphinxbfcode{\sphinxupquote{module }}\sphinxbfcode{\sphinxupquote{pos\_reprint.pos\_reprint}}}~~\begin{quote}\begin{description}
\item[{Exports}] \leavevmode{\hyperref[\detokenize{reference/javascript_api:pos_reprint.pos_reprint.}]{\sphinxcrossref{
\textless{}anonymous\textgreater{}
}}}
\item[{Depends On}] \leavevmode\begin{itemize}
\item {} {\hyperref[\detokenize{reference/javascript_api:point_of_sale.gui}]{\sphinxcrossref{
point\_of\_sale.gui
}}}
\item {} {\hyperref[\detokenize{reference/javascript_api:point_of_sale.screens}]{\sphinxcrossref{
point\_of\_sale.screens
}}}
\item {} {\hyperref[\detokenize{reference/javascript_api:web.core}]{\sphinxcrossref{
web.core
}}}
\end{itemize}

\end{description}\end{quote}


\begin{fulllineitems}
\phantomsection\label{\detokenize{reference/javascript_api:pos_reprint.pos_reprint.}}\pysigline{\sphinxbfcode{\sphinxupquote{namespace }}\sphinxbfcode{\sphinxupquote{}}}
\end{fulllineitems}


\end{fulllineitems}

\phantomsection\label{\detokenize{reference/javascript_api:module-web_editor.transcoder}}

\begin{fulllineitems}
\phantomsection\label{\detokenize{reference/javascript_api:web_editor.transcoder}}\pysigline{\sphinxbfcode{\sphinxupquote{module }}\sphinxbfcode{\sphinxupquote{web\_editor.transcoder}}}~~\begin{quote}\begin{description}
\item[{Exports}] \leavevmode{\hyperref[\detokenize{reference/javascript_api:web_editor.transcoder.}]{\sphinxcrossref{
\textless{}anonymous\textgreater{}
}}}
\item[{Depends On}] \leavevmode\begin{itemize}
\item {} {\hyperref[\detokenize{reference/javascript_api:web_editor.widget}]{\sphinxcrossref{
web\_editor.widget
}}}
\end{itemize}

\end{description}\end{quote}


\begin{fulllineitems}
\phantomsection\label{\detokenize{reference/javascript_api:imgToFont}}\pysiglinewithargsret{\sphinxbfcode{\sphinxupquote{function }}\sphinxbfcode{\sphinxupquote{imgToFont}}}{\emph{\$editable}}{}
Converts images which were the result of a font icon convertion to a font
icon again.
\begin{quote}\begin{description}
\item[{Parameters}] \leavevmode\begin{itemize}

\sphinxstylestrong{\$editable} (\sphinxstyleliteralemphasis{\sphinxupquote{jQuery}}) \textendash{} the element in which the images will be converted
                          back to font icons
\end{itemize}

\end{description}\end{quote}

\end{fulllineitems}



\begin{fulllineitems}
\phantomsection\label{\detokenize{reference/javascript_api:getMatchedCSSRules}}\pysiglinewithargsret{\sphinxbfcode{\sphinxupquote{function }}\sphinxbfcode{\sphinxupquote{getMatchedCSSRules}}}{\emph{a}}{{ $\rightarrow$ Object}}
Returns the css rules which applies on an element, tweaked so that they are
browser/mail client ok.
\begin{quote}\begin{description}
\item[{Parameters}] \leavevmode\begin{itemize}

\sphinxstylestrong{a} (\sphinxstyleliteralemphasis{\sphinxupquote{DOMElement}})
\end{itemize}

\item[{Returns}] \leavevmode
css property name -\textgreater{} css property value

\item[{Return Type}] \leavevmode
\sphinxstyleliteralemphasis{\sphinxupquote{Object}}

\end{description}\end{quote}

\end{fulllineitems}



\begin{fulllineitems}
\phantomsection\label{\detokenize{reference/javascript_api:classToStyle}}\pysiglinewithargsret{\sphinxbfcode{\sphinxupquote{function }}\sphinxbfcode{\sphinxupquote{classToStyle}}}{\emph{\$editable}}{}
Converts css style to inline style (leave the classes on elements but forces
the style they give as inline style).
\begin{quote}\begin{description}
\item[{Parameters}] \leavevmode\begin{itemize}

\sphinxstylestrong{\$editable} (\sphinxstyleliteralemphasis{\sphinxupquote{jQuery}})
\end{itemize}

\end{description}\end{quote}

\end{fulllineitems}



\begin{fulllineitems}
\phantomsection\label{\detokenize{reference/javascript_api:attachmentThumbnailToLinkImg}}\pysiglinewithargsret{\sphinxbfcode{\sphinxupquote{function }}\sphinxbfcode{\sphinxupquote{attachmentThumbnailToLinkImg}}}{\emph{\$editable}}{}
Converts css display for attachment link to real image.
Without this post process, the display depends on the css and the picture
does not appear when we use the html without css (to send by email for e.g.)
\begin{quote}\begin{description}
\item[{Parameters}] \leavevmode\begin{itemize}

\sphinxstylestrong{\$editable} (\sphinxstyleliteralemphasis{\sphinxupquote{jQuery}})
\end{itemize}

\end{description}\end{quote}

\end{fulllineitems}



\begin{fulllineitems}
\phantomsection\label{\detokenize{reference/javascript_api:fontToImg}}\pysiglinewithargsret{\sphinxbfcode{\sphinxupquote{function }}\sphinxbfcode{\sphinxupquote{fontToImg}}}{\emph{\$editable}}{}
Converts font icons to images.
\begin{quote}\begin{description}
\item[{Parameters}] \leavevmode\begin{itemize}

\sphinxstylestrong{\$editable} (\sphinxstyleliteralemphasis{\sphinxupquote{jQuery}}) \textendash{} the element in which the font icons have to be
                          converted to images
\end{itemize}

\end{description}\end{quote}

\end{fulllineitems}



\begin{fulllineitems}
\phantomsection\label{\detokenize{reference/javascript_api:linkImgToAttachmentThumbnail}}\pysiglinewithargsret{\sphinxbfcode{\sphinxupquote{function }}\sphinxbfcode{\sphinxupquote{linkImgToAttachmentThumbnail}}}{\emph{\$editable}}{}
Revert attachmentThumbnailToLinkImg changes
\begin{quote}\begin{description}
\item[{Parameters}] \leavevmode\begin{itemize}

\sphinxstylestrong{\$editable} (\sphinxstyleliteralemphasis{\sphinxupquote{jQuery}})
\end{itemize}

\end{description}\end{quote}

\end{fulllineitems}



\begin{fulllineitems}
\phantomsection\label{\detokenize{reference/javascript_api:web_editor.transcoder.}}\pysigline{\sphinxbfcode{\sphinxupquote{namespace }}\sphinxbfcode{\sphinxupquote{}}}~

\begin{fulllineitems}
\phantomsection\label{\detokenize{reference/javascript_api:fontToImg}}\pysiglinewithargsret{\sphinxbfcode{\sphinxupquote{function }}\sphinxbfcode{\sphinxupquote{fontToImg}}}{\emph{\$editable}}{}
Converts font icons to images.
\begin{quote}\begin{description}
\item[{Parameters}] \leavevmode\begin{itemize}

\sphinxstylestrong{\$editable} (\sphinxstyleliteralemphasis{\sphinxupquote{jQuery}}) \textendash{} the element in which the font icons have to be
                          converted to images
\end{itemize}

\end{description}\end{quote}

\end{fulllineitems}



\begin{fulllineitems}
\phantomsection\label{\detokenize{reference/javascript_api:imgToFont}}\pysiglinewithargsret{\sphinxbfcode{\sphinxupquote{function }}\sphinxbfcode{\sphinxupquote{imgToFont}}}{\emph{\$editable}}{}
Converts images which were the result of a font icon convertion to a font
icon again.
\begin{quote}\begin{description}
\item[{Parameters}] \leavevmode\begin{itemize}

\sphinxstylestrong{\$editable} (\sphinxstyleliteralemphasis{\sphinxupquote{jQuery}}) \textendash{} the element in which the images will be converted
                          back to font icons
\end{itemize}

\end{description}\end{quote}

\end{fulllineitems}



\begin{fulllineitems}
\phantomsection\label{\detokenize{reference/javascript_api:classToStyle}}\pysiglinewithargsret{\sphinxbfcode{\sphinxupquote{function }}\sphinxbfcode{\sphinxupquote{classToStyle}}}{\emph{\$editable}}{}
Converts css style to inline style (leave the classes on elements but forces
the style they give as inline style).
\begin{quote}\begin{description}
\item[{Parameters}] \leavevmode\begin{itemize}

\sphinxstylestrong{\$editable} (\sphinxstyleliteralemphasis{\sphinxupquote{jQuery}})
\end{itemize}

\end{description}\end{quote}

\end{fulllineitems}



\begin{fulllineitems}
\phantomsection\label{\detokenize{reference/javascript_api:styleToClass}}\pysiglinewithargsret{\sphinxbfcode{\sphinxupquote{function }}\sphinxbfcode{\sphinxupquote{styleToClass}}}{\emph{\$editable}}{}
Removes the inline style which is not necessary (because, for example, a
class on an element will induce the same style).
\begin{quote}\begin{description}
\item[{Parameters}] \leavevmode\begin{itemize}

\sphinxstylestrong{\$editable} (\sphinxstyleliteralemphasis{\sphinxupquote{jQuery}})
\end{itemize}

\end{description}\end{quote}

\end{fulllineitems}



\begin{fulllineitems}
\phantomsection\label{\detokenize{reference/javascript_api:attachmentThumbnailToLinkImg}}\pysiglinewithargsret{\sphinxbfcode{\sphinxupquote{function }}\sphinxbfcode{\sphinxupquote{attachmentThumbnailToLinkImg}}}{\emph{\$editable}}{}
Converts css display for attachment link to real image.
Without this post process, the display depends on the css and the picture
does not appear when we use the html without css (to send by email for e.g.)
\begin{quote}\begin{description}
\item[{Parameters}] \leavevmode\begin{itemize}

\sphinxstylestrong{\$editable} (\sphinxstyleliteralemphasis{\sphinxupquote{jQuery}})
\end{itemize}

\end{description}\end{quote}

\end{fulllineitems}



\begin{fulllineitems}
\phantomsection\label{\detokenize{reference/javascript_api:linkImgToAttachmentThumbnail}}\pysiglinewithargsret{\sphinxbfcode{\sphinxupquote{function }}\sphinxbfcode{\sphinxupquote{linkImgToAttachmentThumbnail}}}{\emph{\$editable}}{}
Revert attachmentThumbnailToLinkImg changes
\begin{quote}\begin{description}
\item[{Parameters}] \leavevmode\begin{itemize}

\sphinxstylestrong{\$editable} (\sphinxstyleliteralemphasis{\sphinxupquote{jQuery}})
\end{itemize}

\end{description}\end{quote}

\end{fulllineitems}


\end{fulllineitems}



\begin{fulllineitems}
\phantomsection\label{\detokenize{reference/javascript_api:styleToClass}}\pysiglinewithargsret{\sphinxbfcode{\sphinxupquote{function }}\sphinxbfcode{\sphinxupquote{styleToClass}}}{\emph{\$editable}}{}
Removes the inline style which is not necessary (because, for example, a
class on an element will induce the same style).
\begin{quote}\begin{description}
\item[{Parameters}] \leavevmode\begin{itemize}

\sphinxstylestrong{\$editable} (\sphinxstyleliteralemphasis{\sphinxupquote{jQuery}})
\end{itemize}

\end{description}\end{quote}

\end{fulllineitems}


\end{fulllineitems}

\phantomsection\label{\detokenize{reference/javascript_api:module-web.Session}}

\begin{fulllineitems}
\phantomsection\label{\detokenize{reference/javascript_api:web.Session}}\pysigline{\sphinxbfcode{\sphinxupquote{module }}\sphinxbfcode{\sphinxupquote{web.Session}}}~~\begin{quote}\begin{description}
\item[{Exports}] \leavevmode{\hyperref[\detokenize{reference/javascript_api:web.Session.Session}]{\sphinxcrossref{
Session
}}}
\item[{Depends On}] \leavevmode\begin{itemize}
\item {} {\hyperref[\detokenize{reference/javascript_api:web.ajax}]{\sphinxcrossref{
web.ajax
}}}
\item {} {\hyperref[\detokenize{reference/javascript_api:web.concurrency}]{\sphinxcrossref{
web.concurrency
}}}
\item {} {\hyperref[\detokenize{reference/javascript_api:web.core}]{\sphinxcrossref{
web.core
}}}
\item {} {\hyperref[\detokenize{reference/javascript_api:web.mixins}]{\sphinxcrossref{
web.mixins
}}}
\item {} {\hyperref[\detokenize{reference/javascript_api:web.utils}]{\sphinxcrossref{
web.utils
}}}
\end{itemize}

\end{description}\end{quote}


\begin{fulllineitems}
\phantomsection\label{\detokenize{reference/javascript_api:Session}}\pysiglinewithargsret{\sphinxbfcode{\sphinxupquote{class }}\sphinxbfcode{\sphinxupquote{Session}}}{\emph{parent}, \emph{origin}, \emph{options}}{}~\begin{quote}\begin{description}
\item[{Extends}] \leavevmode{\hyperref[\detokenize{reference/javascript_api:web.Class.Class}]{\sphinxcrossref{
Class
}}}
\item[{Mixes}] \leavevmode\begin{itemize}
\item {} {\hyperref[\detokenize{reference/javascript_api:web.mixins.EventDispatcherMixin}]{\sphinxcrossref{
EventDispatcherMixin
}}}
\end{itemize}

\item[{Parameters}] \leavevmode\begin{itemize}

\sphinxstylestrong{parent} \textendash{} The parent of the newly created object.
    or {}`null{}` if the server to contact is the origin server.

\sphinxstylestrong{origin}

\sphinxstylestrong{options} (\sphinxstyleliteralemphasis{\sphinxupquote{Dict}}) \textendash{} A dictionary that can contain the following options:
“override\_session”: Default to false. If true, the current session object will
          not try to re-use a previously created session id stored in a cookie.
“session\_id”: Default to null. If specified, the specified session\_id will be used
          by this session object. Specifying this option automatically implies that the option
          “override\_session” is set to true.
\end{itemize}

\end{description}\end{quote}


\begin{fulllineitems}
\phantomsection\label{\detokenize{reference/javascript_api:session_bind}}\pysiglinewithargsret{\sphinxbfcode{\sphinxupquote{method }}\sphinxbfcode{\sphinxupquote{session\_bind}}}{\emph{origin}}{}
Setup a session
\begin{quote}\begin{description}
\item[{Parameters}] \leavevmode\begin{itemize}

\sphinxstylestrong{origin}
\end{itemize}

\end{description}\end{quote}

\end{fulllineitems}



\begin{fulllineitems}
\phantomsection\label{\detokenize{reference/javascript_api:session_init}}\pysiglinewithargsret{\sphinxbfcode{\sphinxupquote{method }}\sphinxbfcode{\sphinxupquote{session\_init}}}{}{}
Init a session, reloads from cookie, if it exists

\end{fulllineitems}



\begin{fulllineitems}
\phantomsection\label{\detokenize{reference/javascript_api:session_authenticate}}\pysiglinewithargsret{\sphinxbfcode{\sphinxupquote{method }}\sphinxbfcode{\sphinxupquote{session\_authenticate}}}{}{}
The session is validated by restoration of a previous session

\end{fulllineitems}



\begin{fulllineitems}
\phantomsection\label{\detokenize{reference/javascript_api:load_modules}}\pysiglinewithargsret{\sphinxbfcode{\sphinxupquote{method }}\sphinxbfcode{\sphinxupquote{load\_modules}}}{}{}
Load additional web addons of that instance and init them

\end{fulllineitems}



\begin{fulllineitems}
\phantomsection\label{\detokenize{reference/javascript_api:session_reload}}\pysiglinewithargsret{\sphinxbfcode{\sphinxupquote{method }}\sphinxbfcode{\sphinxupquote{session\_reload}}}{}{{ $\rightarrow$ \$.Deferred}}
(re)loads the content of a session: db name, username, user id, session
context and status of the support contract
\begin{quote}\begin{description}
\item[{Returns}] \leavevmode
deferred indicating the session is done reloading

\item[{Return Type}] \leavevmode
\sphinxstyleliteralemphasis{\sphinxupquote{jQuery.Deferred}}

\end{description}\end{quote}

\end{fulllineitems}



\begin{fulllineitems}
\phantomsection\label{\detokenize{reference/javascript_api:rpc}}\pysiglinewithargsret{\sphinxbfcode{\sphinxupquote{method }}\sphinxbfcode{\sphinxupquote{rpc}}}{\emph{url}, \emph{params}, \emph{options}}{{ $\rightarrow$ jQuery.Deferred}}
Executes an RPC call, registering the provided callbacks.

Registers a default error callback if none is provided, and handles
setting the correct session id and session context in the parameter
objects
\begin{quote}\begin{description}
\item[{Parameters}] \leavevmode\begin{itemize}

\sphinxstylestrong{url} (\sphinxstyleliteralemphasis{\sphinxupquote{String}}) \textendash{} RPC endpoint

\sphinxstylestrong{params} (\sphinxstyleliteralemphasis{\sphinxupquote{Object}}) \textendash{} call parameters

\sphinxstylestrong{options} (\sphinxstyleliteralemphasis{\sphinxupquote{Object}}) \textendash{} additional options for rpc call
\end{itemize}

\item[{Returns}] \leavevmode
jquery-provided ajax deferred

\item[{Return Type}] \leavevmode
\sphinxstyleliteralemphasis{\sphinxupquote{jQuery.Deferred}}

\end{description}\end{quote}

\end{fulllineitems}



\begin{fulllineitems}
\phantomsection\label{\detokenize{reference/javascript_api:getTZOffset}}\pysiglinewithargsret{\sphinxbfcode{\sphinxupquote{method }}\sphinxbfcode{\sphinxupquote{getTZOffset}}}{\emph{date}}{{ $\rightarrow$ integer}}
Returns the time zone difference (in minutes) from the current locale
(host system settings) to UTC, for a given date. The offset is positive
if the local timezone is behind UTC, and negative if it is ahead.
\begin{quote}\begin{description}
\item[{Parameters}] \leavevmode\begin{itemize}

\sphinxstylestrong{date} (\sphinxstyleliteralemphasis{\sphinxupquote{string}}\sphinxstyleemphasis{ or }\sphinxstyleliteralemphasis{\sphinxupquote{moment}}) \textendash{} a valid string date or moment instance
\end{itemize}

\item[{Return Type}] \leavevmode
\sphinxstyleliteralemphasis{\sphinxupquote{integer}}

\end{description}\end{quote}

\end{fulllineitems}


\end{fulllineitems}


\end{fulllineitems}

\phantomsection\label{\detokenize{reference/javascript_api:module-website.content.compatibility}}

\begin{fulllineitems}
\phantomsection\label{\detokenize{reference/javascript_api:website.content.compatibility}}\pysigline{\sphinxbfcode{\sphinxupquote{module }}\sphinxbfcode{\sphinxupquote{website.content.compatibility}}}~~\begin{quote}\begin{description}
\item[{Exports}] \leavevmode{\hyperref[\detokenize{reference/javascript_api:website.content.compatibility.}]{\sphinxcrossref{
\textless{}anonymous\textgreater{}
}}}
\end{description}\end{quote}


\begin{fulllineitems}
\phantomsection\label{\detokenize{reference/javascript_api:website.content.compatibility.}}\pysigline{\sphinxbfcode{\sphinxupquote{namespace }}\sphinxbfcode{\sphinxupquote{}}}
\end{fulllineitems}


\end{fulllineitems}

\phantomsection\label{\detokenize{reference/javascript_api:module-web_editor.translate}}

\begin{fulllineitems}
\phantomsection\label{\detokenize{reference/javascript_api:web_editor.translate}}\pysigline{\sphinxbfcode{\sphinxupquote{module }}\sphinxbfcode{\sphinxupquote{web\_editor.translate}}}~~\begin{quote}\begin{description}
\item[{Exports}] \leavevmode{\hyperref[\detokenize{reference/javascript_api:web_editor.translate.}]{\sphinxcrossref{
\textless{}anonymous\textgreater{}
}}}
\item[{Depends On}] \leavevmode\begin{itemize}
\item {} {\hyperref[\detokenize{reference/javascript_api:web.Dialog}]{\sphinxcrossref{
web.Dialog
}}}
\item {} {\hyperref[\detokenize{reference/javascript_api:web.Widget}]{\sphinxcrossref{
web.Widget
}}}
\item {} {\hyperref[\detokenize{reference/javascript_api:web.core}]{\sphinxcrossref{
web.core
}}}
\item {} {\hyperref[\detokenize{reference/javascript_api:web.local_storage}]{\sphinxcrossref{
web.local\_storage
}}}
\item {} {\hyperref[\detokenize{reference/javascript_api:web_editor.context}]{\sphinxcrossref{
web\_editor.context
}}}
\item {} {\hyperref[\detokenize{reference/javascript_api:web_editor.rte}]{\sphinxcrossref{
web\_editor.rte
}}}
\item {} {\hyperref[\detokenize{reference/javascript_api:web_editor.widget}]{\sphinxcrossref{
web\_editor.widget
}}}
\end{itemize}

\end{description}\end{quote}


\begin{fulllineitems}
\phantomsection\label{\detokenize{reference/javascript_api:web_editor.translate.}}\pysigline{\sphinxbfcode{\sphinxupquote{namespace }}\sphinxbfcode{\sphinxupquote{}}}
\end{fulllineitems}


\end{fulllineitems}

\phantomsection\label{\detokenize{reference/javascript_api:module-point_of_sale.models}}

\begin{fulllineitems}
\phantomsection\label{\detokenize{reference/javascript_api:point_of_sale.models}}\pysigline{\sphinxbfcode{\sphinxupquote{module }}\sphinxbfcode{\sphinxupquote{point\_of\_sale.models}}}~~\begin{quote}\begin{description}
\item[{Exports}] \leavevmode{\hyperref[\detokenize{reference/javascript_api:point_of_sale.models.exports}]{\sphinxcrossref{
exports
}}}
\item[{Depends On}] \leavevmode\begin{itemize}
\item {} {\hyperref[\detokenize{reference/javascript_api:barcodes.BarcodeParser}]{\sphinxcrossref{
barcodes.BarcodeParser
}}}
\item {} {\hyperref[\detokenize{reference/javascript_api:point_of_sale.DB}]{\sphinxcrossref{
point\_of\_sale.DB
}}}
\item {} {\hyperref[\detokenize{reference/javascript_api:point_of_sale.devices}]{\sphinxcrossref{
point\_of\_sale.devices
}}}
\item {} {\hyperref[\detokenize{reference/javascript_api:web.ajax}]{\sphinxcrossref{
web.ajax
}}}
\item {} {\hyperref[\detokenize{reference/javascript_api:web.concurrency}]{\sphinxcrossref{
web.concurrency
}}}
\item {} {\hyperref[\detokenize{reference/javascript_api:web.core}]{\sphinxcrossref{
web.core
}}}
\item {} {\hyperref[\detokenize{reference/javascript_api:web.field_utils}]{\sphinxcrossref{
web.field\_utils
}}}
\item {} {\hyperref[\detokenize{reference/javascript_api:web.rpc}]{\sphinxcrossref{
web.rpc
}}}
\item {} {\hyperref[\detokenize{reference/javascript_api:web.session}]{\sphinxcrossref{
web.session
}}}
\item {} {\hyperref[\detokenize{reference/javascript_api:web.time}]{\sphinxcrossref{
web.time
}}}
\item {} {\hyperref[\detokenize{reference/javascript_api:web.utils}]{\sphinxcrossref{
web.utils
}}}
\end{itemize}

\end{description}\end{quote}


\begin{fulllineitems}
\phantomsection\label{\detokenize{reference/javascript_api:exports}}\pysigline{\sphinxbfcode{\sphinxupquote{namespace }}\sphinxbfcode{\sphinxupquote{exports}}}
\end{fulllineitems}


\end{fulllineitems}

\phantomsection\label{\detokenize{reference/javascript_api:module-account.ReconciliationRenderer}}

\begin{fulllineitems}
\phantomsection\label{\detokenize{reference/javascript_api:account.ReconciliationRenderer}}\pysigline{\sphinxbfcode{\sphinxupquote{module }}\sphinxbfcode{\sphinxupquote{account.ReconciliationRenderer}}}~~\begin{quote}\begin{description}
\item[{Exports}] \leavevmode{\hyperref[\detokenize{reference/javascript_api:account.ReconciliationRenderer.}]{\sphinxcrossref{
\textless{}anonymous\textgreater{}
}}}
\item[{Depends On}] \leavevmode\begin{itemize}
\item {} {\hyperref[\detokenize{reference/javascript_api:web.FieldManagerMixin}]{\sphinxcrossref{
web.FieldManagerMixin
}}}
\item {} {\hyperref[\detokenize{reference/javascript_api:web.Widget}]{\sphinxcrossref{
web.Widget
}}}
\item {} {\hyperref[\detokenize{reference/javascript_api:web.basic_fields}]{\sphinxcrossref{
web.basic\_fields
}}}
\item {} {\hyperref[\detokenize{reference/javascript_api:web.core}]{\sphinxcrossref{
web.core
}}}
\item {} {\hyperref[\detokenize{reference/javascript_api:web.relational_fields}]{\sphinxcrossref{
web.relational\_fields
}}}
\item {} {\hyperref[\detokenize{reference/javascript_api:web.time}]{\sphinxcrossref{
web.time
}}}
\end{itemize}

\end{description}\end{quote}


\begin{fulllineitems}
\phantomsection\label{\detokenize{reference/javascript_api:ManualRenderer}}\pysiglinewithargsret{\sphinxbfcode{\sphinxupquote{class }}\sphinxbfcode{\sphinxupquote{ManualRenderer}}}{}{}~\begin{quote}\begin{description}
\item[{Extends}] \leavevmode{\hyperref[\detokenize{reference/javascript_api:account.ReconciliationRenderer.StatementRenderer}]{\sphinxcrossref{
StatementRenderer
}}}
\end{description}\end{quote}

rendering of the manual reconciliation action contains progress bar, title
and auto reconciliation button

\end{fulllineitems}



\begin{fulllineitems}
\phantomsection\label{\detokenize{reference/javascript_api:ManualLineRenderer}}\pysiglinewithargsret{\sphinxbfcode{\sphinxupquote{class }}\sphinxbfcode{\sphinxupquote{ManualLineRenderer}}}{}{}~\begin{quote}\begin{description}
\item[{Extends}] \leavevmode{\hyperref[\detokenize{reference/javascript_api:account.ReconciliationRenderer.LineRenderer}]{\sphinxcrossref{
LineRenderer
}}}
\end{description}\end{quote}

rendering of the manual reconciliation, contains line data, proposition and
view for ‘match’ mode


\begin{fulllineitems}
\phantomsection\label{\detokenize{reference/javascript_api:start}}\pysiglinewithargsret{\sphinxbfcode{\sphinxupquote{method }}\sphinxbfcode{\sphinxupquote{start}}}{}{}
move the partner field

\end{fulllineitems}


\end{fulllineitems}



\begin{fulllineitems}
\phantomsection\label{\detokenize{reference/javascript_api:account.ReconciliationRenderer.}}\pysigline{\sphinxbfcode{\sphinxupquote{namespace }}\sphinxbfcode{\sphinxupquote{}}}~

\begin{fulllineitems}
\phantomsection\label{\detokenize{reference/javascript_api:StatementRenderer}}\pysiglinewithargsret{\sphinxbfcode{\sphinxupquote{class }}\sphinxbfcode{\sphinxupquote{StatementRenderer}}}{\emph{parent}, \emph{model}, \emph{state}}{}~\begin{quote}\begin{description}
\item[{Extends}] \leavevmode{\hyperref[\detokenize{reference/javascript_api:web.Widget.Widget}]{\sphinxcrossref{
Widget
}}}
\item[{Mixes}] \leavevmode\begin{itemize}
\item {} {\hyperref[\detokenize{reference/javascript_api:web.FieldManagerMixin.FieldManagerMixin}]{\sphinxcrossref{
FieldManagerMixin
}}}
\end{itemize}

\item[{Parameters}] \leavevmode\begin{itemize}

\sphinxstylestrong{parent}

\sphinxstylestrong{model}

\sphinxstylestrong{state}
\end{itemize}

\end{description}\end{quote}

rendering of the bank statement action contains progress bar, title and
auto reconciliation button


\begin{fulllineitems}
\phantomsection\label{\detokenize{reference/javascript_api:start}}\pysiglinewithargsret{\sphinxbfcode{\sphinxupquote{method }}\sphinxbfcode{\sphinxupquote{start}}}{}{}
display iniial state and create the name statement field

\end{fulllineitems}



\begin{fulllineitems}
\phantomsection\label{\detokenize{reference/javascript_api:update}}\pysiglinewithargsret{\sphinxbfcode{\sphinxupquote{method }}\sphinxbfcode{\sphinxupquote{update}}}{\emph{state}}{}
update the statement rendering
\begin{quote}\begin{description}
\item[{Parameters}] \leavevmode\begin{itemize}

\sphinxstylestrong{state} ({\hyperref[\detokenize{reference/javascript_api:account.ReconciliationRenderer.UpdateState}]{\sphinxcrossref{\sphinxstyleliteralemphasis{\sphinxupquote{UpdateState}}}}}) \textendash{} statement data
\end{itemize}

\end{description}\end{quote}


\begin{fulllineitems}
\phantomsection\label{\detokenize{reference/javascript_api:UpdateState}}\pysiglinewithargsret{\sphinxbfcode{\sphinxupquote{class }}\sphinxbfcode{\sphinxupquote{UpdateState}}}{}{}
statement data


\begin{fulllineitems}
\phantomsection\label{\detokenize{reference/javascript_api:valuenow}}\pysigline{\sphinxbfcode{\sphinxupquote{attribute }}\sphinxbfcode{\sphinxupquote{valuenow}} integer}
for the progress bar

\end{fulllineitems}



\begin{fulllineitems}
\phantomsection\label{\detokenize{reference/javascript_api:valuemax}}\pysigline{\sphinxbfcode{\sphinxupquote{attribute }}\sphinxbfcode{\sphinxupquote{valuemax}} integer}
for the progress bar

\end{fulllineitems}



\begin{fulllineitems}
\phantomsection\label{\detokenize{reference/javascript_api:title}}\pysigline{\sphinxbfcode{\sphinxupquote{attribute }}\sphinxbfcode{\sphinxupquote{title}} string}
for the progress bar

\end{fulllineitems}



\begin{fulllineitems}
\phantomsection\label{\detokenize{reference/javascript_api:notifications}}\pysigline{\sphinxbfcode{\sphinxupquote{attribute }}\sphinxbfcode{\sphinxupquote{notifications}} {[}object{]}}
\end{fulllineitems}


\end{fulllineitems}


\end{fulllineitems}


\end{fulllineitems}



\begin{fulllineitems}
\phantomsection\label{\detokenize{reference/javascript_api:ManualRenderer}}\pysiglinewithargsret{\sphinxbfcode{\sphinxupquote{class }}\sphinxbfcode{\sphinxupquote{ManualRenderer}}}{}{}~\begin{quote}\begin{description}
\item[{Extends}] \leavevmode{\hyperref[\detokenize{reference/javascript_api:account.ReconciliationRenderer.StatementRenderer}]{\sphinxcrossref{
StatementRenderer
}}}
\end{description}\end{quote}

rendering of the manual reconciliation action contains progress bar, title
and auto reconciliation button

\end{fulllineitems}



\begin{fulllineitems}
\phantomsection\label{\detokenize{reference/javascript_api:LineRenderer}}\pysiglinewithargsret{\sphinxbfcode{\sphinxupquote{class }}\sphinxbfcode{\sphinxupquote{LineRenderer}}}{\emph{parent}, \emph{model}, \emph{state}}{}~\begin{quote}\begin{description}
\item[{Extends}] \leavevmode{\hyperref[\detokenize{reference/javascript_api:web.Widget.Widget}]{\sphinxcrossref{
Widget
}}}
\item[{Mixes}] \leavevmode\begin{itemize}
\item {} {\hyperref[\detokenize{reference/javascript_api:web.FieldManagerMixin.FieldManagerMixin}]{\sphinxcrossref{
FieldManagerMixin
}}}
\end{itemize}

\item[{Parameters}] \leavevmode\begin{itemize}

\sphinxstylestrong{parent}

\sphinxstylestrong{model}

\sphinxstylestrong{state}
\end{itemize}

\end{description}\end{quote}

rendering of the bank statement line, contains line data, proposition and
view for ‘match’ and ‘create’ mode


\begin{fulllineitems}
\phantomsection\label{\detokenize{reference/javascript_api:update}}\pysiglinewithargsret{\sphinxbfcode{\sphinxupquote{method }}\sphinxbfcode{\sphinxupquote{update}}}{\emph{state}}{}
update the statement line rendering
\begin{quote}\begin{description}
\item[{Parameters}] \leavevmode\begin{itemize}

\sphinxstylestrong{state} (\sphinxstyleliteralemphasis{\sphinxupquote{object}}) \textendash{} statement line
\end{itemize}

\end{description}\end{quote}

\end{fulllineitems}


\end{fulllineitems}



\begin{fulllineitems}
\phantomsection\label{\detokenize{reference/javascript_api:ManualLineRenderer}}\pysiglinewithargsret{\sphinxbfcode{\sphinxupquote{class }}\sphinxbfcode{\sphinxupquote{ManualLineRenderer}}}{}{}~\begin{quote}\begin{description}
\item[{Extends}] \leavevmode{\hyperref[\detokenize{reference/javascript_api:account.ReconciliationRenderer.LineRenderer}]{\sphinxcrossref{
LineRenderer
}}}
\end{description}\end{quote}

rendering of the manual reconciliation, contains line data, proposition and
view for ‘match’ mode


\begin{fulllineitems}
\phantomsection\label{\detokenize{reference/javascript_api:start}}\pysiglinewithargsret{\sphinxbfcode{\sphinxupquote{method }}\sphinxbfcode{\sphinxupquote{start}}}{}{}
move the partner field

\end{fulllineitems}


\end{fulllineitems}


\end{fulllineitems}



\begin{fulllineitems}
\phantomsection\label{\detokenize{reference/javascript_api:LineRenderer}}\pysiglinewithargsret{\sphinxbfcode{\sphinxupquote{class }}\sphinxbfcode{\sphinxupquote{LineRenderer}}}{\emph{parent}, \emph{model}, \emph{state}}{}~\begin{quote}\begin{description}
\item[{Extends}] \leavevmode{\hyperref[\detokenize{reference/javascript_api:web.Widget.Widget}]{\sphinxcrossref{
Widget
}}}
\item[{Mixes}] \leavevmode\begin{itemize}
\item {} {\hyperref[\detokenize{reference/javascript_api:web.FieldManagerMixin.FieldManagerMixin}]{\sphinxcrossref{
FieldManagerMixin
}}}
\end{itemize}

\item[{Parameters}] \leavevmode\begin{itemize}

\sphinxstylestrong{parent}

\sphinxstylestrong{model}

\sphinxstylestrong{state}
\end{itemize}

\end{description}\end{quote}

rendering of the bank statement line, contains line data, proposition and
view for ‘match’ and ‘create’ mode


\begin{fulllineitems}
\phantomsection\label{\detokenize{reference/javascript_api:update}}\pysiglinewithargsret{\sphinxbfcode{\sphinxupquote{method }}\sphinxbfcode{\sphinxupquote{update}}}{\emph{state}}{}
update the statement line rendering
\begin{quote}\begin{description}
\item[{Parameters}] \leavevmode\begin{itemize}

\sphinxstylestrong{state} (\sphinxstyleliteralemphasis{\sphinxupquote{object}}) \textendash{} statement line
\end{itemize}

\end{description}\end{quote}

\end{fulllineitems}


\end{fulllineitems}



\begin{fulllineitems}
\phantomsection\label{\detokenize{reference/javascript_api:StatementRenderer}}\pysiglinewithargsret{\sphinxbfcode{\sphinxupquote{class }}\sphinxbfcode{\sphinxupquote{StatementRenderer}}}{\emph{parent}, \emph{model}, \emph{state}}{}~\begin{quote}\begin{description}
\item[{Extends}] \leavevmode{\hyperref[\detokenize{reference/javascript_api:web.Widget.Widget}]{\sphinxcrossref{
Widget
}}}
\item[{Mixes}] \leavevmode\begin{itemize}
\item {} {\hyperref[\detokenize{reference/javascript_api:web.FieldManagerMixin.FieldManagerMixin}]{\sphinxcrossref{
FieldManagerMixin
}}}
\end{itemize}

\item[{Parameters}] \leavevmode\begin{itemize}

\sphinxstylestrong{parent}

\sphinxstylestrong{model}

\sphinxstylestrong{state}
\end{itemize}

\end{description}\end{quote}

rendering of the bank statement action contains progress bar, title and
auto reconciliation button


\begin{fulllineitems}
\phantomsection\label{\detokenize{reference/javascript_api:start}}\pysiglinewithargsret{\sphinxbfcode{\sphinxupquote{method }}\sphinxbfcode{\sphinxupquote{start}}}{}{}
display iniial state and create the name statement field

\end{fulllineitems}



\begin{fulllineitems}
\phantomsection\label{\detokenize{reference/javascript_api:update}}\pysiglinewithargsret{\sphinxbfcode{\sphinxupquote{method }}\sphinxbfcode{\sphinxupquote{update}}}{\emph{state}}{}
update the statement rendering
\begin{quote}\begin{description}
\item[{Parameters}] \leavevmode\begin{itemize}

\sphinxstylestrong{state} ({\hyperref[\detokenize{reference/javascript_api:account.ReconciliationRenderer.UpdateState}]{\sphinxcrossref{\sphinxstyleliteralemphasis{\sphinxupquote{UpdateState}}}}}) \textendash{} statement data
\end{itemize}

\end{description}\end{quote}


\begin{fulllineitems}
\phantomsection\label{\detokenize{reference/javascript_api:UpdateState}}\pysiglinewithargsret{\sphinxbfcode{\sphinxupquote{class }}\sphinxbfcode{\sphinxupquote{UpdateState}}}{}{}
statement data


\begin{fulllineitems}
\phantomsection\label{\detokenize{reference/javascript_api:valuenow}}\pysigline{\sphinxbfcode{\sphinxupquote{attribute }}\sphinxbfcode{\sphinxupquote{valuenow}} integer}
for the progress bar

\end{fulllineitems}



\begin{fulllineitems}
\phantomsection\label{\detokenize{reference/javascript_api:valuemax}}\pysigline{\sphinxbfcode{\sphinxupquote{attribute }}\sphinxbfcode{\sphinxupquote{valuemax}} integer}
for the progress bar

\end{fulllineitems}



\begin{fulllineitems}
\phantomsection\label{\detokenize{reference/javascript_api:title}}\pysigline{\sphinxbfcode{\sphinxupquote{attribute }}\sphinxbfcode{\sphinxupquote{title}} string}
for the progress bar

\end{fulllineitems}



\begin{fulllineitems}
\phantomsection\label{\detokenize{reference/javascript_api:notifications}}\pysigline{\sphinxbfcode{\sphinxupquote{attribute }}\sphinxbfcode{\sphinxupquote{notifications}} {[}object{]}}
\end{fulllineitems}


\end{fulllineitems}


\end{fulllineitems}


\end{fulllineitems}


\end{fulllineitems}

\phantomsection\label{\detokenize{reference/javascript_api:module-website_crm.tour}}

\begin{fulllineitems}
\phantomsection\label{\detokenize{reference/javascript_api:website_crm.tour}}\pysigline{\sphinxbfcode{\sphinxupquote{module }}\sphinxbfcode{\sphinxupquote{website\_crm.tour}}}~~\begin{quote}\begin{description}
\item[{Exports}] \leavevmode{\hyperref[\detokenize{reference/javascript_api:website_crm.tour.}]{\sphinxcrossref{
\textless{}anonymous\textgreater{}
}}}
\item[{Depends On}] \leavevmode\begin{itemize}
\item {} {\hyperref[\detokenize{reference/javascript_api:web_editor.base}]{\sphinxcrossref{
web\_editor.base
}}}
\item {} {\hyperref[\detokenize{reference/javascript_api:web_tour.tour}]{\sphinxcrossref{
web\_tour.tour
}}}
\end{itemize}

\end{description}\end{quote}


\begin{fulllineitems}
\phantomsection\label{\detokenize{reference/javascript_api:website_crm.tour.}}\pysigline{\sphinxbfcode{\sphinxupquote{namespace }}\sphinxbfcode{\sphinxupquote{}}}
\end{fulllineitems}


\end{fulllineitems}

\phantomsection\label{\detokenize{reference/javascript_api:module-point_of_sale.popups}}

\begin{fulllineitems}
\phantomsection\label{\detokenize{reference/javascript_api:point_of_sale.popups}}\pysigline{\sphinxbfcode{\sphinxupquote{module }}\sphinxbfcode{\sphinxupquote{point\_of\_sale.popups}}}~~\begin{quote}\begin{description}
\item[{Exports}] \leavevmode{\hyperref[\detokenize{reference/javascript_api:point_of_sale.popups.PopupWidget}]{\sphinxcrossref{
PopupWidget
}}}
\item[{Depends On}] \leavevmode\begin{itemize}
\item {} {\hyperref[\detokenize{reference/javascript_api:point_of_sale.BaseWidget}]{\sphinxcrossref{
point\_of\_sale.BaseWidget
}}}
\item {} {\hyperref[\detokenize{reference/javascript_api:point_of_sale.gui}]{\sphinxcrossref{
point\_of\_sale.gui
}}}
\item {} {\hyperref[\detokenize{reference/javascript_api:web.core}]{\sphinxcrossref{
web.core
}}}
\end{itemize}

\end{description}\end{quote}


\begin{fulllineitems}
\phantomsection\label{\detokenize{reference/javascript_api:SelectionPopupWidget}}\pysiglinewithargsret{\sphinxbfcode{\sphinxupquote{class }}\sphinxbfcode{\sphinxupquote{SelectionPopupWidget}}}{}{}~\begin{quote}\begin{description}
\item[{Extends}] \leavevmode{\hyperref[\detokenize{reference/javascript_api:point_of_sale.popups.PopupWidget}]{\sphinxcrossref{
PopupWidget
}}}
\end{description}\end{quote}

A popup that allows the user to select one item from a list.

Example:

\fvset{hllines={, ,}}%
\begin{sphinxVerbatim}[commandchars=\\\{\}]
\PYG{n+nx}{show\PYGZus{}popup}\PYG{p}{(}\PYG{l+s+s1}{\PYGZsq{}selection\PYGZsq{}}\PYG{p}{,}\PYG{p}{\PYGZob{}}
    \PYG{n+nx}{title}\PYG{o}{:} \PYG{l+s+s2}{\PYGZdq{}Popup Title\PYGZdq{}}\PYG{p}{,}
    \PYG{n+nx}{list}\PYG{o}{:} \PYG{p}{[}
        \PYG{p}{\PYGZob{}} \PYG{n+nx}{label}\PYG{o}{:} \PYG{l+s+s1}{\PYGZsq{}foobar\PYGZsq{}}\PYG{p}{,}  \PYG{n+nx}{item}\PYG{o}{:} \PYG{l+m+mi}{45} \PYG{p}{\PYGZcb{}}\PYG{p}{,}
        \PYG{p}{\PYGZob{}} \PYG{n+nx}{label}\PYG{o}{:} \PYG{l+s+s1}{\PYGZsq{}bar foo\PYGZsq{}}\PYG{p}{,} \PYG{n+nx}{item}\PYG{o}{:} \PYG{l+s+s1}{\PYGZsq{}stuff\PYGZsq{}} \PYG{p}{\PYGZcb{}}\PYG{p}{,}
    \PYG{p}{]}\PYG{p}{,}
    \PYG{n+nx}{confirm}\PYG{o}{:} \PYG{k+kd}{function}\PYG{p}{(}\PYG{n+nx}{item}\PYG{p}{)} \PYG{p}{\PYGZob{}}
        \PYG{c+c1}{// get the item selected by the user.}
    \PYG{p}{\PYGZcb{}}\PYG{p}{,}
    \PYG{n+nx}{cancel}\PYG{o}{:} \PYG{k+kd}{function}\PYG{p}{(}\PYG{p}{)}\PYG{p}{\PYGZob{}}
        \PYG{c+c1}{// user chose nothing}
    \PYG{p}{\PYGZcb{}}
\PYG{p}{\PYGZcb{}}\PYG{p}{)}\PYG{p}{;}
\end{sphinxVerbatim}

\end{fulllineitems}



\begin{fulllineitems}
\phantomsection\label{\detokenize{reference/javascript_api:PopupWidget}}\pysiglinewithargsret{\sphinxbfcode{\sphinxupquote{class }}\sphinxbfcode{\sphinxupquote{PopupWidget}}}{\emph{parent}, \emph{args}}{}~\begin{quote}\begin{description}
\item[{Extends}] \leavevmode{\hyperref[\detokenize{reference/javascript_api:point_of_sale.BaseWidget.PosBaseWidget}]{\sphinxcrossref{
PosBaseWidget
}}}
\item[{Parameters}] \leavevmode\begin{itemize}

\sphinxstylestrong{parent}

\sphinxstylestrong{args}
\end{itemize}

\end{description}\end{quote}

\end{fulllineitems}


\end{fulllineitems}

\phantomsection\label{\detokenize{reference/javascript_api:module-base_import.import}}

\begin{fulllineitems}
\phantomsection\label{\detokenize{reference/javascript_api:base_import.import}}\pysigline{\sphinxbfcode{\sphinxupquote{module }}\sphinxbfcode{\sphinxupquote{base\_import.import}}}~~\begin{quote}\begin{description}
\item[{Exports}] \leavevmode{\hyperref[\detokenize{reference/javascript_api:base_import.import.}]{\sphinxcrossref{
\textless{}anonymous\textgreater{}
}}}
\item[{Depends On}] \leavevmode\begin{itemize}
\item {} {\hyperref[\detokenize{reference/javascript_api:web.ControlPanelMixin}]{\sphinxcrossref{
web.ControlPanelMixin
}}}
\item {} {\hyperref[\detokenize{reference/javascript_api:web.Widget}]{\sphinxcrossref{
web.Widget
}}}
\item {} {\hyperref[\detokenize{reference/javascript_api:web.core}]{\sphinxcrossref{
web.core
}}}
\item {} {\hyperref[\detokenize{reference/javascript_api:web.session}]{\sphinxcrossref{
web.session
}}}
\item {} {\hyperref[\detokenize{reference/javascript_api:web.time}]{\sphinxcrossref{
web.time
}}}
\end{itemize}

\end{description}\end{quote}


\begin{fulllineitems}
\phantomsection\label{\detokenize{reference/javascript_api:jsonp}}\pysiglinewithargsret{\sphinxbfcode{\sphinxupquote{function }}\sphinxbfcode{\sphinxupquote{jsonp}}}{\emph{form}, \emph{attributes}, \emph{callback}}{}
Safari does not deal well at all with raw JSON data being
returned. As a result, we’re going to cheat by using a
pseudo-jsonp: instead of getting JSON data in the iframe, we’re
getting a \sphinxcode{\sphinxupquote{script}} tag which consists of a function call and
the returned data (the json dump).

The function is an auto-generated name bound to \sphinxcode{\sphinxupquote{window}},
which calls back into the callback provided here.
\begin{quote}\begin{description}
\item[{Parameters}] \leavevmode\begin{itemize}

\sphinxstylestrong{form} (\sphinxstyleliteralemphasis{\sphinxupquote{Object}}) \textendash{} the form element (DOM or jQuery) to use in the call

\sphinxstylestrong{attributes} (\sphinxstyleliteralemphasis{\sphinxupquote{Object}}) \textendash{} jquery.form attributes object

\sphinxstylestrong{callback} (\sphinxstyleliteralemphasis{\sphinxupquote{Function}}) \textendash{} function to call with the returned data
\end{itemize}

\end{description}\end{quote}

\end{fulllineitems}



\begin{fulllineitems}
\phantomsection\label{\detokenize{reference/javascript_api:base_import.import.}}\pysigline{\sphinxbfcode{\sphinxupquote{namespace }}\sphinxbfcode{\sphinxupquote{}}}
\end{fulllineitems}


\end{fulllineitems}

\phantomsection\label{\detokenize{reference/javascript_api:module-web.Loading}}

\begin{fulllineitems}
\phantomsection\label{\detokenize{reference/javascript_api:web.Loading}}\pysigline{\sphinxbfcode{\sphinxupquote{module }}\sphinxbfcode{\sphinxupquote{web.Loading}}}~~\begin{quote}\begin{description}
\item[{Exports}] \leavevmode{\hyperref[\detokenize{reference/javascript_api:web.Loading.Loading}]{\sphinxcrossref{
Loading
}}}
\item[{Depends On}] \leavevmode\begin{itemize}
\item {} {\hyperref[\detokenize{reference/javascript_api:web.Widget}]{\sphinxcrossref{
web.Widget
}}}
\item {} {\hyperref[\detokenize{reference/javascript_api:web.core}]{\sphinxcrossref{
web.core
}}}
\item {} {\hyperref[\detokenize{reference/javascript_api:web.framework}]{\sphinxcrossref{
web.framework
}}}
\item {} {\hyperref[\detokenize{reference/javascript_api:web.session}]{\sphinxcrossref{
web.session
}}}
\end{itemize}

\end{description}\end{quote}


\begin{fulllineitems}
\phantomsection\label{\detokenize{reference/javascript_api:Loading}}\pysiglinewithargsret{\sphinxbfcode{\sphinxupquote{class }}\sphinxbfcode{\sphinxupquote{Loading}}}{\emph{parent}}{}~\begin{quote}\begin{description}
\item[{Extends}] \leavevmode{\hyperref[\detokenize{reference/javascript_api:web.Widget.Widget}]{\sphinxcrossref{
Widget
}}}
\item[{Parameters}] \leavevmode\begin{itemize}

\sphinxstylestrong{parent}
\end{itemize}

\end{description}\end{quote}

\end{fulllineitems}


\end{fulllineitems}

\phantomsection\label{\detokenize{reference/javascript_api:module-payment.payment_form}}

\begin{fulllineitems}
\phantomsection\label{\detokenize{reference/javascript_api:payment.payment_form}}\pysigline{\sphinxbfcode{\sphinxupquote{module }}\sphinxbfcode{\sphinxupquote{payment.payment\_form}}}~~\begin{quote}\begin{description}
\item[{Exports}] \leavevmode{\hyperref[\detokenize{reference/javascript_api:payment.payment_form.PaymentForm}]{\sphinxcrossref{
PaymentForm
}}}
\item[{Depends On}] \leavevmode\begin{itemize}
\item {} {\hyperref[\detokenize{reference/javascript_api:web.Dialog}]{\sphinxcrossref{
web.Dialog
}}}
\item {} {\hyperref[\detokenize{reference/javascript_api:web.Widget}]{\sphinxcrossref{
web.Widget
}}}
\item {} {\hyperref[\detokenize{reference/javascript_api:web.ajax}]{\sphinxcrossref{
web.ajax
}}}
\item {} {\hyperref[\detokenize{reference/javascript_api:web.core}]{\sphinxcrossref{
web.core
}}}
\item {} {\hyperref[\detokenize{reference/javascript_api:web.rpc}]{\sphinxcrossref{
web.rpc
}}}
\end{itemize}

\end{description}\end{quote}


\begin{fulllineitems}
\phantomsection\label{\detokenize{reference/javascript_api:PaymentForm}}\pysiglinewithargsret{\sphinxbfcode{\sphinxupquote{class }}\sphinxbfcode{\sphinxupquote{PaymentForm}}}{\emph{parent}, \emph{options}}{}~\begin{quote}\begin{description}
\item[{Extends}] \leavevmode{\hyperref[\detokenize{reference/javascript_api:web.Widget.Widget}]{\sphinxcrossref{
Widget
}}}
\item[{Parameters}] \leavevmode\begin{itemize}

\sphinxstylestrong{parent}

\sphinxstylestrong{options}
\end{itemize}

\end{description}\end{quote}

\end{fulllineitems}


\end{fulllineitems}

\phantomsection\label{\detokenize{reference/javascript_api:module-web.ChangePassword}}

\begin{fulllineitems}
\phantomsection\label{\detokenize{reference/javascript_api:web.ChangePassword}}\pysigline{\sphinxbfcode{\sphinxupquote{module }}\sphinxbfcode{\sphinxupquote{web.ChangePassword}}}~~\begin{quote}\begin{description}
\item[{Exports}] \leavevmode{\hyperref[\detokenize{reference/javascript_api:web.ChangePassword.ChangePassword}]{\sphinxcrossref{
ChangePassword
}}}
\item[{Depends On}] \leavevmode\begin{itemize}
\item {} {\hyperref[\detokenize{reference/javascript_api:web.Dialog}]{\sphinxcrossref{
web.Dialog
}}}
\item {} {\hyperref[\detokenize{reference/javascript_api:web.Widget}]{\sphinxcrossref{
web.Widget
}}}
\item {} {\hyperref[\detokenize{reference/javascript_api:web.core}]{\sphinxcrossref{
web.core
}}}
\item {} 
web.web\_client

\end{itemize}

\end{description}\end{quote}


\begin{fulllineitems}
\phantomsection\label{\detokenize{reference/javascript_api:ChangePassword}}\pysiglinewithargsret{\sphinxbfcode{\sphinxupquote{class }}\sphinxbfcode{\sphinxupquote{ChangePassword}}}{}{}~\begin{quote}\begin{description}
\item[{Extends}] \leavevmode{\hyperref[\detokenize{reference/javascript_api:web.Widget.Widget}]{\sphinxcrossref{
Widget
}}}
\end{description}\end{quote}

\end{fulllineitems}


\end{fulllineitems}

\phantomsection\label{\detokenize{reference/javascript_api:module-web.rainbow_man}}

\begin{fulllineitems}
\phantomsection\label{\detokenize{reference/javascript_api:web.rainbow_man}}\pysigline{\sphinxbfcode{\sphinxupquote{module }}\sphinxbfcode{\sphinxupquote{web.rainbow\_man}}}~~\begin{quote}\begin{description}
\item[{Exports}] \leavevmode{\hyperref[\detokenize{reference/javascript_api:web.rainbow_man.RainbowMan}]{\sphinxcrossref{
RainbowMan
}}}
\item[{Depends On}] \leavevmode\begin{itemize}
\item {} {\hyperref[\detokenize{reference/javascript_api:web.Widget}]{\sphinxcrossref{
web.Widget
}}}
\item {} {\hyperref[\detokenize{reference/javascript_api:web.ajax}]{\sphinxcrossref{
web.ajax
}}}
\item {} {\hyperref[\detokenize{reference/javascript_api:web.core}]{\sphinxcrossref{
web.core
}}}
\end{itemize}

\end{description}\end{quote}


\begin{fulllineitems}
\phantomsection\label{\detokenize{reference/javascript_api:RainbowMan}}\pysiglinewithargsret{\sphinxbfcode{\sphinxupquote{class }}\sphinxbfcode{\sphinxupquote{RainbowMan}}}{\sphinxoptional{\emph{options}}}{}~\begin{quote}\begin{description}
\item[{Extends}] \leavevmode{\hyperref[\detokenize{reference/javascript_api:web.Widget.Widget}]{\sphinxcrossref{
Widget
}}}
\item[{Parameters}] \leavevmode\begin{itemize}

\sphinxstylestrong{options} ({\hyperref[\detokenize{reference/javascript_api:web.rainbow_man.RainbowManOptions}]{\sphinxcrossref{\sphinxstyleliteralemphasis{\sphinxupquote{RainbowManOptions}}}}}) \textendash{} key-value options to decide rainbowman behavior / appearance
\end{itemize}

\end{description}\end{quote}


\begin{fulllineitems}
\phantomsection\label{\detokenize{reference/javascript_api:RainbowManOptions}}\pysiglinewithargsret{\sphinxbfcode{\sphinxupquote{class }}\sphinxbfcode{\sphinxupquote{RainbowManOptions}}}{}{}
key-value options to decide rainbowman behavior / appearance


\begin{fulllineitems}
\phantomsection\label{\detokenize{reference/javascript_api:message}}\pysigline{\sphinxbfcode{\sphinxupquote{attribute }}\sphinxbfcode{\sphinxupquote{message}} string}
Message to be displayed on rainbowman card

\end{fulllineitems}



\begin{fulllineitems}
\phantomsection\label{\detokenize{reference/javascript_api:fadeout}}\pysigline{\sphinxbfcode{\sphinxupquote{attribute }}\sphinxbfcode{\sphinxupquote{fadeout}} string}~\begin{description}
\item[{Delay for rainbowman to disappear - {[}options.fadeout=’fast’{]} will make rainbowman}] \leavevmode
dissapear quickly, {[}options.fadeout=’medium’{]} and {[}options.fadeout=’slow’{]} will
wait little longer before disappearing (can be used when {[}options.message{]}
is longer), {[}options.fadeout=’no’{]} will keep rainbowman on screen until
user clicks anywhere outside rainbowman

\end{description}

\end{fulllineitems}



\begin{fulllineitems}
\phantomsection\label{\detokenize{reference/javascript_api:img_url}}\pysigline{\sphinxbfcode{\sphinxupquote{attribute }}\sphinxbfcode{\sphinxupquote{img\_url}} string}
URL of the image to be displayed

\end{fulllineitems}



\begin{fulllineitems}
\phantomsection\label{\detokenize{reference/javascript_api:click_close}}\pysigline{\sphinxbfcode{\sphinxupquote{attribute }}\sphinxbfcode{\sphinxupquote{click\_close}} Boolean}
If true, destroys rainbowman on click outside

\end{fulllineitems}


\end{fulllineitems}


\end{fulllineitems}


\end{fulllineitems}

\phantomsection\label{\detokenize{reference/javascript_api:module-base.settings}}

\begin{fulllineitems}
\phantomsection\label{\detokenize{reference/javascript_api:base.settings}}\pysigline{\sphinxbfcode{\sphinxupquote{module }}\sphinxbfcode{\sphinxupquote{base.settings}}}~~\begin{quote}\begin{description}
\item[{Exports}] \leavevmode{\hyperref[\detokenize{reference/javascript_api:base.settings.}]{\sphinxcrossref{
\textless{}anonymous\textgreater{}
}}}
\item[{Depends On}] \leavevmode\begin{itemize}
\item {} {\hyperref[\detokenize{reference/javascript_api:web.FormController}]{\sphinxcrossref{
web.FormController
}}}
\item {} {\hyperref[\detokenize{reference/javascript_api:web.FormRenderer}]{\sphinxcrossref{
web.FormRenderer
}}}
\item {} {\hyperref[\detokenize{reference/javascript_api:web.FormView}]{\sphinxcrossref{
web.FormView
}}}
\item {} {\hyperref[\detokenize{reference/javascript_api:web.config}]{\sphinxcrossref{
web.config
}}}
\item {} {\hyperref[\detokenize{reference/javascript_api:web.core}]{\sphinxcrossref{
web.core
}}}
\item {} {\hyperref[\detokenize{reference/javascript_api:web.view_registry}]{\sphinxcrossref{
web.view\_registry
}}}
\end{itemize}

\end{description}\end{quote}


\begin{fulllineitems}
\phantomsection\label{\detokenize{reference/javascript_api:base.settings.}}\pysigline{\sphinxbfcode{\sphinxupquote{namespace }}\sphinxbfcode{\sphinxupquote{}}}
\end{fulllineitems}


\end{fulllineitems}

\phantomsection\label{\detokenize{reference/javascript_api:module-website_blog.editor}}

\begin{fulllineitems}
\phantomsection\label{\detokenize{reference/javascript_api:website_blog.editor}}\pysigline{\sphinxbfcode{\sphinxupquote{module }}\sphinxbfcode{\sphinxupquote{website\_blog.editor}}}~~\begin{quote}\begin{description}
\item[{Exports}] \leavevmode{\hyperref[\detokenize{reference/javascript_api:website_blog.editor.}]{\sphinxcrossref{
\textless{}anonymous\textgreater{}
}}}
\item[{Depends On}] \leavevmode\begin{itemize}
\item {} {\hyperref[\detokenize{reference/javascript_api:web_editor.rte}]{\sphinxcrossref{
web\_editor.rte
}}}
\item {} {\hyperref[\detokenize{reference/javascript_api:web_editor.snippets.options}]{\sphinxcrossref{
web\_editor.snippets.options
}}}
\item {} {\hyperref[\detokenize{reference/javascript_api:web_editor.widget}]{\sphinxcrossref{
web\_editor.widget
}}}
\end{itemize}

\end{description}\end{quote}


\begin{fulllineitems}
\phantomsection\label{\detokenize{reference/javascript_api:website_blog.editor.}}\pysigline{\sphinxbfcode{\sphinxupquote{namespace }}\sphinxbfcode{\sphinxupquote{}}}
\end{fulllineitems}


\end{fulllineitems}

\phantomsection\label{\detokenize{reference/javascript_api:module-website_links.charts}}

\begin{fulllineitems}
\phantomsection\label{\detokenize{reference/javascript_api:website_links.charts}}\pysigline{\sphinxbfcode{\sphinxupquote{module }}\sphinxbfcode{\sphinxupquote{website\_links.charts}}}~~\begin{quote}\begin{description}
\item[{Exports}] \leavevmode{\hyperref[\detokenize{reference/javascript_api:website_links.charts.exports}]{\sphinxcrossref{
exports
}}}
\item[{Depends On}] \leavevmode\begin{itemize}
\item {} {\hyperref[\detokenize{reference/javascript_api:web.Widget}]{\sphinxcrossref{
web.Widget
}}}
\item {} {\hyperref[\detokenize{reference/javascript_api:web.core}]{\sphinxcrossref{
web.core
}}}
\item {} {\hyperref[\detokenize{reference/javascript_api:web.rpc}]{\sphinxcrossref{
web.rpc
}}}
\item {} {\hyperref[\detokenize{reference/javascript_api:web_editor.base}]{\sphinxcrossref{
web\_editor.base
}}}
\end{itemize}

\end{description}\end{quote}


\begin{fulllineitems}
\phantomsection\label{\detokenize{reference/javascript_api:exports}}\pysigline{\sphinxbfcode{\sphinxupquote{namespace }}\sphinxbfcode{\sphinxupquote{exports}}}
\end{fulllineitems}


\end{fulllineitems}

\phantomsection\label{\detokenize{reference/javascript_api:module-mail.DocumentViewer}}

\begin{fulllineitems}
\phantomsection\label{\detokenize{reference/javascript_api:mail.DocumentViewer}}\pysigline{\sphinxbfcode{\sphinxupquote{module }}\sphinxbfcode{\sphinxupquote{mail.DocumentViewer}}}~~\begin{quote}\begin{description}
\item[{Exports}] \leavevmode{\hyperref[\detokenize{reference/javascript_api:mail.DocumentViewer.DocumentViewer}]{\sphinxcrossref{
DocumentViewer
}}}
\item[{Depends On}] \leavevmode\begin{itemize}
\item {} {\hyperref[\detokenize{reference/javascript_api:web.Widget}]{\sphinxcrossref{
web.Widget
}}}
\item {} {\hyperref[\detokenize{reference/javascript_api:web.core}]{\sphinxcrossref{
web.core
}}}
\end{itemize}

\end{description}\end{quote}


\begin{fulllineitems}
\phantomsection\label{\detokenize{reference/javascript_api:DocumentViewer}}\pysiglinewithargsret{\sphinxbfcode{\sphinxupquote{class }}\sphinxbfcode{\sphinxupquote{DocumentViewer}}}{\emph{parent}, \emph{attachments}, \emph{activeAttachmentID}}{}~\begin{quote}\begin{description}
\item[{Extends}] \leavevmode{\hyperref[\detokenize{reference/javascript_api:web.Widget.Widget}]{\sphinxcrossref{
Widget
}}}
\item[{Parameters}] \leavevmode\begin{itemize}

\sphinxstylestrong{parent}

\sphinxstylestrong{attachments} (\sphinxstyleliteralemphasis{\sphinxupquote{Array}}\textless{}\sphinxstyleliteralemphasis{\sphinxupquote{Object}}\textgreater{}) \textendash{} list of attachments

\sphinxstylestrong{activeAttachmentID} (\sphinxstyleliteralemphasis{\sphinxupquote{integer}})
\end{itemize}

\end{description}\end{quote}


\begin{fulllineitems}
\phantomsection\label{\detokenize{reference/javascript_api:start}}\pysiglinewithargsret{\sphinxbfcode{\sphinxupquote{method }}\sphinxbfcode{\sphinxupquote{start}}}{}{}
Open a modal displaying the active attachment

\end{fulllineitems}


\end{fulllineitems}


\end{fulllineitems}

\phantomsection\label{\detokenize{reference/javascript_api:module-web.CalendarQuickCreate}}

\begin{fulllineitems}
\phantomsection\label{\detokenize{reference/javascript_api:web.CalendarQuickCreate}}\pysigline{\sphinxbfcode{\sphinxupquote{module }}\sphinxbfcode{\sphinxupquote{web.CalendarQuickCreate}}}~~\begin{quote}\begin{description}
\item[{Exports}] \leavevmode{\hyperref[\detokenize{reference/javascript_api:web.CalendarQuickCreate.QuickCreate}]{\sphinxcrossref{
QuickCreate
}}}
\item[{Depends On}] \leavevmode\begin{itemize}
\item {} {\hyperref[\detokenize{reference/javascript_api:web.Dialog}]{\sphinxcrossref{
web.Dialog
}}}
\item {} {\hyperref[\detokenize{reference/javascript_api:web.core}]{\sphinxcrossref{
web.core
}}}
\end{itemize}

\end{description}\end{quote}


\begin{fulllineitems}
\phantomsection\label{\detokenize{reference/javascript_api:QuickCreate}}\pysiglinewithargsret{\sphinxbfcode{\sphinxupquote{class }}\sphinxbfcode{\sphinxupquote{QuickCreate}}}{\emph{parent}, \emph{buttons}, \emph{options}, \emph{dataTemplate}, \emph{dataCalendar}}{}~\begin{quote}\begin{description}
\item[{Extends}] \leavevmode{\hyperref[\detokenize{reference/javascript_api:web.Dialog.Dialog}]{\sphinxcrossref{
Dialog
}}}
\item[{Parameters}] \leavevmode\begin{itemize}

\sphinxstylestrong{parent} ({\hyperref[\detokenize{reference/javascript_api:Widget}]{\sphinxcrossref{\sphinxstyleliteralemphasis{\sphinxupquote{Widget}}}}})

\sphinxstylestrong{buttons} (\sphinxstyleliteralemphasis{\sphinxupquote{Object}})

\sphinxstylestrong{options} (\sphinxstyleliteralemphasis{\sphinxupquote{Object}})

\sphinxstylestrong{dataTemplate} (\sphinxstyleliteralemphasis{\sphinxupquote{Object}})

\sphinxstylestrong{dataCalendar} (\sphinxstyleliteralemphasis{\sphinxupquote{Object}})
\end{itemize}

\end{description}\end{quote}

Quick creation view.

Triggers a single event “added” with a single parameter “name”, which is the
name entered by the user

\end{fulllineitems}



\begin{fulllineitems}
\phantomsection\label{\detokenize{reference/javascript_api:QuickCreate}}\pysiglinewithargsret{\sphinxbfcode{\sphinxupquote{class }}\sphinxbfcode{\sphinxupquote{QuickCreate}}}{\emph{parent}, \emph{buttons}, \emph{options}, \emph{dataTemplate}, \emph{dataCalendar}}{}~\begin{quote}\begin{description}
\item[{Extends}] \leavevmode{\hyperref[\detokenize{reference/javascript_api:web.Dialog.Dialog}]{\sphinxcrossref{
Dialog
}}}
\item[{Parameters}] \leavevmode\begin{itemize}

\sphinxstylestrong{parent} ({\hyperref[\detokenize{reference/javascript_api:Widget}]{\sphinxcrossref{\sphinxstyleliteralemphasis{\sphinxupquote{Widget}}}}})

\sphinxstylestrong{buttons} (\sphinxstyleliteralemphasis{\sphinxupquote{Object}})

\sphinxstylestrong{options} (\sphinxstyleliteralemphasis{\sphinxupquote{Object}})

\sphinxstylestrong{dataTemplate} (\sphinxstyleliteralemphasis{\sphinxupquote{Object}})

\sphinxstylestrong{dataCalendar} (\sphinxstyleliteralemphasis{\sphinxupquote{Object}})
\end{itemize}

\end{description}\end{quote}

Quick creation view.

Triggers a single event “added” with a single parameter “name”, which is the
name entered by the user

\end{fulllineitems}


\end{fulllineitems}

\phantomsection\label{\detokenize{reference/javascript_api:module-web.StandaloneFieldManagerMixin}}

\begin{fulllineitems}
\phantomsection\label{\detokenize{reference/javascript_api:web.StandaloneFieldManagerMixin}}\pysigline{\sphinxbfcode{\sphinxupquote{module }}\sphinxbfcode{\sphinxupquote{web.StandaloneFieldManagerMixin}}}~~\begin{quote}\begin{description}
\item[{Exports}] \leavevmode{\hyperref[\detokenize{reference/javascript_api:web.StandaloneFieldManagerMixin.StandaloneFieldManagerMixin}]{\sphinxcrossref{
StandaloneFieldManagerMixin
}}}
\item[{Depends On}] \leavevmode\begin{itemize}
\item {} {\hyperref[\detokenize{reference/javascript_api:web.FieldManagerMixin}]{\sphinxcrossref{
web.FieldManagerMixin
}}}
\end{itemize}

\end{description}\end{quote}


\begin{fulllineitems}
\phantomsection\label{\detokenize{reference/javascript_api:StandaloneFieldManagerMixin}}\pysigline{\sphinxbfcode{\sphinxupquote{mixin }}\sphinxbfcode{\sphinxupquote{StandaloneFieldManagerMixin}}}
The StandaloneFieldManagerMixin is a mixin, designed to be used by a widget
that instanciates its own field widgets.

\end{fulllineitems}



\begin{fulllineitems}
\phantomsection\label{\detokenize{reference/javascript_api:StandaloneFieldManagerMixin}}\pysigline{\sphinxbfcode{\sphinxupquote{mixin }}\sphinxbfcode{\sphinxupquote{StandaloneFieldManagerMixin}}}
The StandaloneFieldManagerMixin is a mixin, designed to be used by a widget
that instanciates its own field widgets.

\end{fulllineitems}


\end{fulllineitems}

\phantomsection\label{\detokenize{reference/javascript_api:module-web.SystrayMenu}}

\begin{fulllineitems}
\phantomsection\label{\detokenize{reference/javascript_api:web.SystrayMenu}}\pysigline{\sphinxbfcode{\sphinxupquote{module }}\sphinxbfcode{\sphinxupquote{web.SystrayMenu}}}~~\begin{quote}\begin{description}
\item[{Exports}] \leavevmode{\hyperref[\detokenize{reference/javascript_api:web.SystrayMenu.SystrayMenu}]{\sphinxcrossref{
SystrayMenu
}}}
\item[{Depends On}] \leavevmode\begin{itemize}
\item {} {\hyperref[\detokenize{reference/javascript_api:web.Widget}]{\sphinxcrossref{
web.Widget
}}}
\end{itemize}

\end{description}\end{quote}


\begin{fulllineitems}
\phantomsection\label{\detokenize{reference/javascript_api:SystrayMenu}}\pysiglinewithargsret{\sphinxbfcode{\sphinxupquote{class }}\sphinxbfcode{\sphinxupquote{SystrayMenu}}}{\emph{parent}}{}~\begin{quote}\begin{description}
\item[{Extends}] \leavevmode{\hyperref[\detokenize{reference/javascript_api:web.Widget.Widget}]{\sphinxcrossref{
Widget
}}}
\item[{Parameters}] \leavevmode\begin{itemize}

\sphinxstylestrong{parent}
\end{itemize}

\end{description}\end{quote}

The SystrayMenu is the class that manage the list of icons in the top right
of the menu bar.

\end{fulllineitems}



\begin{fulllineitems}
\phantomsection\label{\detokenize{reference/javascript_api:SystrayMenu}}\pysiglinewithargsret{\sphinxbfcode{\sphinxupquote{class }}\sphinxbfcode{\sphinxupquote{SystrayMenu}}}{\emph{parent}}{}~\begin{quote}\begin{description}
\item[{Extends}] \leavevmode{\hyperref[\detokenize{reference/javascript_api:web.Widget.Widget}]{\sphinxcrossref{
Widget
}}}
\item[{Parameters}] \leavevmode\begin{itemize}

\sphinxstylestrong{parent}
\end{itemize}

\end{description}\end{quote}

The SystrayMenu is the class that manage the list of icons in the top right
of the menu bar.

\end{fulllineitems}


\end{fulllineitems}

\phantomsection\label{\detokenize{reference/javascript_api:module-web.LocalStorageService}}

\begin{fulllineitems}
\phantomsection\label{\detokenize{reference/javascript_api:web.LocalStorageService}}\pysigline{\sphinxbfcode{\sphinxupquote{module }}\sphinxbfcode{\sphinxupquote{web.LocalStorageService}}}~~\begin{quote}\begin{description}
\item[{Exports}] \leavevmode{\hyperref[\detokenize{reference/javascript_api:web.LocalStorageService.LocalStorageService}]{\sphinxcrossref{
LocalStorageService
}}}
\item[{Depends On}] \leavevmode\begin{itemize}
\item {} {\hyperref[\detokenize{reference/javascript_api:web.AbstractService}]{\sphinxcrossref{
web.AbstractService
}}}
\item {} {\hyperref[\detokenize{reference/javascript_api:web.local_storage}]{\sphinxcrossref{
web.local\_storage
}}}
\end{itemize}

\end{description}\end{quote}


\begin{fulllineitems}
\phantomsection\label{\detokenize{reference/javascript_api:LocalStorageService}}\pysiglinewithargsret{\sphinxbfcode{\sphinxupquote{class }}\sphinxbfcode{\sphinxupquote{LocalStorageService}}}{}{}~\begin{quote}\begin{description}
\item[{Extends}] \leavevmode{\hyperref[\detokenize{reference/javascript_api:web.AbstractService.AbstractService}]{\sphinxcrossref{
AbstractService
}}}
\end{description}\end{quote}

\end{fulllineitems}


\end{fulllineitems}

\phantomsection\label{\detokenize{reference/javascript_api:module-web.AbstractController}}

\begin{fulllineitems}
\phantomsection\label{\detokenize{reference/javascript_api:web.AbstractController}}\pysigline{\sphinxbfcode{\sphinxupquote{module }}\sphinxbfcode{\sphinxupquote{web.AbstractController}}}~~\begin{quote}\begin{description}
\item[{Exports}] \leavevmode{\hyperref[\detokenize{reference/javascript_api:web.AbstractController.AbstractController}]{\sphinxcrossref{
AbstractController
}}}
\item[{Depends On}] \leavevmode\begin{itemize}
\item {} {\hyperref[\detokenize{reference/javascript_api:web.Widget}]{\sphinxcrossref{
web.Widget
}}}
\end{itemize}

\end{description}\end{quote}


\begin{fulllineitems}
\phantomsection\label{\detokenize{reference/javascript_api:AbstractController}}\pysiglinewithargsret{\sphinxbfcode{\sphinxupquote{class }}\sphinxbfcode{\sphinxupquote{AbstractController}}}{\emph{parent}, \emph{model}, \emph{renderer}, \emph{params}}{}~\begin{quote}\begin{description}
\item[{Extends}] \leavevmode{\hyperref[\detokenize{reference/javascript_api:web.Widget.Widget}]{\sphinxcrossref{
Widget
}}}
\item[{Parameters}] \leavevmode\begin{itemize}

\sphinxstylestrong{parent} ({\hyperref[\detokenize{reference/javascript_api:Widget}]{\sphinxcrossref{\sphinxstyleliteralemphasis{\sphinxupquote{Widget}}}}})

\sphinxstylestrong{model} ({\hyperref[\detokenize{reference/javascript_api:AbstractModel}]{\sphinxcrossref{\sphinxstyleliteralemphasis{\sphinxupquote{AbstractModel}}}}})

\sphinxstylestrong{renderer} ({\hyperref[\detokenize{reference/javascript_api:AbstractRenderer}]{\sphinxcrossref{\sphinxstyleliteralemphasis{\sphinxupquote{AbstractRenderer}}}}})

\sphinxstylestrong{params} ({\hyperref[\detokenize{reference/javascript_api:web.AbstractController.AbstractControllerParams}]{\sphinxcrossref{\sphinxstyleliteralemphasis{\sphinxupquote{AbstractControllerParams}}}}})
\end{itemize}

\end{description}\end{quote}


\begin{fulllineitems}
\phantomsection\label{\detokenize{reference/javascript_api:start}}\pysiglinewithargsret{\sphinxbfcode{\sphinxupquote{method }}\sphinxbfcode{\sphinxupquote{start}}}{}{{ $\rightarrow$ Deferred}}
Simply renders and updates the url.
\begin{quote}\begin{description}
\item[{Return Type}] \leavevmode
\sphinxstyleliteralemphasis{\sphinxupquote{Deferred}}

\end{description}\end{quote}

\end{fulllineitems}



\begin{fulllineitems}
\phantomsection\label{\detokenize{reference/javascript_api:on_attach_callback}}\pysiglinewithargsret{\sphinxbfcode{\sphinxupquote{method }}\sphinxbfcode{\sphinxupquote{on\_attach\_callback}}}{}{}
Called each time the controller is attached into the DOM.

\end{fulllineitems}



\begin{fulllineitems}
\phantomsection\label{\detokenize{reference/javascript_api:on_detach_callback}}\pysiglinewithargsret{\sphinxbfcode{\sphinxupquote{method }}\sphinxbfcode{\sphinxupquote{on\_detach\_callback}}}{}{}
Called each time the controller is detached from the DOM.

\end{fulllineitems}



\begin{fulllineitems}
\phantomsection\label{\detokenize{reference/javascript_api:discardChanges}}\pysiglinewithargsret{\sphinxbfcode{\sphinxupquote{method }}\sphinxbfcode{\sphinxupquote{discardChanges}}}{\sphinxoptional{\emph{recordID}}}{{ $\rightarrow$ Deferred}}
Discards the changes made on the record associated to the given ID, or
all changes made by the current controller if no recordID is given. For
example, when the user open the ‘home’ screen, the view manager will call
this method on the active view to make sure it is ok to open the home
screen (and lose all current state).

Note that it returns a deferred, because the view could choose to ask the
user if he agrees to discard.
\begin{quote}\begin{description}
\item[{Parameters}] \leavevmode\begin{itemize}

\sphinxstylestrong{recordID} (\sphinxstyleliteralemphasis{\sphinxupquote{string}}) \textendash{} if not given, we consider all the changes made by the controller
\end{itemize}

\item[{Returns}] \leavevmode
resolved if properly discarded, rejected otherwise

\item[{Return Type}] \leavevmode
\sphinxstyleliteralemphasis{\sphinxupquote{Deferred}}

\end{description}\end{quote}

\end{fulllineitems}



\begin{fulllineitems}
\phantomsection\label{\detokenize{reference/javascript_api:getContext}}\pysiglinewithargsret{\sphinxbfcode{\sphinxupquote{method }}\sphinxbfcode{\sphinxupquote{getContext}}}{}{{ $\rightarrow$ Object}}
Returns any special keys that may be useful when reloading the view to
get the same effect.  This is necessary for saving the current view in
the favorites.  For example, a graph view might want to add a key to
save the current graph type.
\begin{quote}\begin{description}
\item[{Return Type}] \leavevmode
\sphinxstyleliteralemphasis{\sphinxupquote{Object}}

\end{description}\end{quote}

\end{fulllineitems}



\begin{fulllineitems}
\phantomsection\label{\detokenize{reference/javascript_api:getTitle}}\pysiglinewithargsret{\sphinxbfcode{\sphinxupquote{method }}\sphinxbfcode{\sphinxupquote{getTitle}}}{}{{ $\rightarrow$ string}}
Returns a title that may be displayed in the breadcrumb area.  For
example, the name of the record.

Note: this seems wrong right now, it should not be implemented, we have
no guarantee that there is a display\_name variable in a controller.
\begin{quote}\begin{description}
\item[{Return Type}] \leavevmode
\sphinxstyleliteralemphasis{\sphinxupquote{string}}

\end{description}\end{quote}

\end{fulllineitems}



\begin{fulllineitems}
\phantomsection\label{\detokenize{reference/javascript_api:is_action_enabled}}\pysiglinewithargsret{\sphinxbfcode{\sphinxupquote{method }}\sphinxbfcode{\sphinxupquote{is\_action\_enabled}}}{\emph{action}}{{ $\rightarrow$ boolean}}
The use of this method is discouraged.  It is still snakecased, because
it currently is used in many templates, but we will move to a simpler
mechanism as soon as we can.
\begin{quote}\begin{description}
\item[{Parameters}] \leavevmode\begin{itemize}

\sphinxstylestrong{action} (\sphinxstyleliteralemphasis{\sphinxupquote{string}}) \textendash{} type of action, such as ‘create’, ‘read’, …
\end{itemize}

\item[{Return Type}] \leavevmode
\sphinxstyleliteralemphasis{\sphinxupquote{boolean}}

\end{description}\end{quote}

\end{fulllineitems}



\begin{fulllineitems}
\phantomsection\label{\detokenize{reference/javascript_api:reload}}\pysiglinewithargsret{\sphinxbfcode{\sphinxupquote{method }}\sphinxbfcode{\sphinxupquote{reload}}}{\sphinxoptional{\emph{params}}}{{ $\rightarrow$ Deferred}}
Short helper method to reload the view
\begin{quote}\begin{description}
\item[{Parameters}] \leavevmode\begin{itemize}

\sphinxstylestrong{params} (\sphinxstyleliteralemphasis{\sphinxupquote{Object}}) \textendash{} This object will simply be given to the update
\end{itemize}

\item[{Return Type}] \leavevmode
\sphinxstyleliteralemphasis{\sphinxupquote{Deferred}}

\end{description}\end{quote}

\end{fulllineitems}



\begin{fulllineitems}
\phantomsection\label{\detokenize{reference/javascript_api:renderButtons}}\pysiglinewithargsret{\sphinxbfcode{\sphinxupquote{method }}\sphinxbfcode{\sphinxupquote{renderButtons}}}{\emph{\$node}}{}
Most likely called by the view manager, this method is responsible for
adding buttons in the control panel (buttons such as save/discard/…)

Note that there is no guarantee that this method will be called. The
controller is supposed to work even without a view manager, for example
in the frontend (odoo frontend = public website)
\begin{quote}\begin{description}
\item[{Parameters}] \leavevmode\begin{itemize}

\sphinxstylestrong{\$node} (\sphinxstyleliteralemphasis{\sphinxupquote{jQuery}})
\end{itemize}

\end{description}\end{quote}

\end{fulllineitems}



\begin{fulllineitems}
\phantomsection\label{\detokenize{reference/javascript_api:renderPager}}\pysiglinewithargsret{\sphinxbfcode{\sphinxupquote{method }}\sphinxbfcode{\sphinxupquote{renderPager}}}{\emph{\$node}}{}
For views that require a pager, this method will be called to allow the
controller to instantiate and render a pager. Note that in theory, the
controller can actually render whatever he wants in the pager zone.  If
your view does not want a pager, just let this method empty.
\begin{quote}\begin{description}
\item[{Parameters}] \leavevmode\begin{itemize}

\sphinxstylestrong{\$node} (\sphinxstyleliteralemphasis{\sphinxupquote{Query}})
\end{itemize}

\end{description}\end{quote}

\end{fulllineitems}



\begin{fulllineitems}
\phantomsection\label{\detokenize{reference/javascript_api:renderSidebar}}\pysiglinewithargsret{\sphinxbfcode{\sphinxupquote{method }}\sphinxbfcode{\sphinxupquote{renderSidebar}}}{\emph{\$node}}{}
Same as renderPager, but for the ‘sidebar’ zone (the zone with the menu
dropdown in the control panel next to the buttons)
\begin{quote}\begin{description}
\item[{Parameters}] \leavevmode\begin{itemize}

\sphinxstylestrong{\$node} (\sphinxstyleliteralemphasis{\sphinxupquote{Query}})
\end{itemize}

\end{description}\end{quote}

\end{fulllineitems}



\begin{fulllineitems}
\phantomsection\label{\detokenize{reference/javascript_api:setScrollTop}}\pysiglinewithargsret{\sphinxbfcode{\sphinxupquote{method }}\sphinxbfcode{\sphinxupquote{setScrollTop}}}{\emph{scrollTop}}{}
Not sure about this one, it probably needs to be reworked, maybe merged
in get/set local state methods.
\begin{quote}\begin{description}
\item[{Parameters}] \leavevmode\begin{itemize}

\sphinxstylestrong{scrollTop} (\sphinxstyleliteralemphasis{\sphinxupquote{number}})
\end{itemize}

\end{description}\end{quote}

\end{fulllineitems}



\begin{fulllineitems}
\phantomsection\label{\detokenize{reference/javascript_api:update}}\pysiglinewithargsret{\sphinxbfcode{\sphinxupquote{method }}\sphinxbfcode{\sphinxupquote{update}}}{\emph{params}\sphinxoptional{, \emph{options}}}{{ $\rightarrow$ Deferred}}
This is the main entry point for the controller.  Changes from the search
view arrive in this method, and internal changes can sometimes also call
this method.  It is basically the way everything notifies the controller
that something has changed.

The update method is responsible for fetching necessary data, then
updating the renderer and wait for the rendering to complete.
\begin{quote}\begin{description}
\item[{Parameters}] \leavevmode\begin{itemize}

\sphinxstylestrong{params} (\sphinxstyleliteralemphasis{\sphinxupquote{Object}}) \textendash{} will be given to the model and to the renderer

\sphinxstylestrong{options} ({\hyperref[\detokenize{reference/javascript_api:web.AbstractController.UpdateOptions}]{\sphinxcrossref{\sphinxstyleliteralemphasis{\sphinxupquote{UpdateOptions}}}}})
\end{itemize}

\item[{Return Type}] \leavevmode
\sphinxstyleliteralemphasis{\sphinxupquote{Deferred}}

\end{description}\end{quote}


\begin{fulllineitems}
\phantomsection\label{\detokenize{reference/javascript_api:UpdateOptions}}\pysiglinewithargsret{\sphinxbfcode{\sphinxupquote{class }}\sphinxbfcode{\sphinxupquote{UpdateOptions}}}{}{}~

\begin{fulllineitems}
\phantomsection\label{\detokenize{reference/javascript_api:reload}}\pysigline{\sphinxbfcode{\sphinxupquote{attribute }}\sphinxbfcode{\sphinxupquote{reload}} boolean}
if true, the model will reload data

\end{fulllineitems}


\end{fulllineitems}


\end{fulllineitems}



\begin{fulllineitems}
\phantomsection\label{\detokenize{reference/javascript_api:AbstractControllerParams}}\pysiglinewithargsret{\sphinxbfcode{\sphinxupquote{class }}\sphinxbfcode{\sphinxupquote{AbstractControllerParams}}}{}{}~

\begin{fulllineitems}
\phantomsection\label{\detokenize{reference/javascript_api:modelName}}\pysigline{\sphinxbfcode{\sphinxupquote{attribute }}\sphinxbfcode{\sphinxupquote{modelName}} string}
\end{fulllineitems}



\begin{fulllineitems}
\phantomsection\label{\detokenize{reference/javascript_api:handle}}\pysigline{\sphinxbfcode{\sphinxupquote{attribute }}\sphinxbfcode{\sphinxupquote{handle}} any}
a handle that will be given to the model (some id)

\end{fulllineitems}



\begin{fulllineitems}
\phantomsection\label{\detokenize{reference/javascript_api:initialState}}\pysigline{\sphinxbfcode{\sphinxupquote{attribute }}\sphinxbfcode{\sphinxupquote{initialState}} any}
the initialState

\end{fulllineitems}


\end{fulllineitems}


\end{fulllineitems}


\end{fulllineitems}

\phantomsection\label{\detokenize{reference/javascript_api:module-web.data}}

\begin{fulllineitems}
\phantomsection\label{\detokenize{reference/javascript_api:web.data}}\pysigline{\sphinxbfcode{\sphinxupquote{module }}\sphinxbfcode{\sphinxupquote{web.data}}}~~\begin{quote}\begin{description}
\item[{Exports}] \leavevmode{\hyperref[\detokenize{reference/javascript_api:web.data.data}]{\sphinxcrossref{
data
}}}
\item[{Depends On}] \leavevmode\begin{itemize}
\item {} {\hyperref[\detokenize{reference/javascript_api:web.Class}]{\sphinxcrossref{
web.Class
}}}
\item {} {\hyperref[\detokenize{reference/javascript_api:web.Context}]{\sphinxcrossref{
web.Context
}}}
\item {} {\hyperref[\detokenize{reference/javascript_api:web.concurrency}]{\sphinxcrossref{
web.concurrency
}}}
\item {} {\hyperref[\detokenize{reference/javascript_api:web.mixins}]{\sphinxcrossref{
web.mixins
}}}
\item {} {\hyperref[\detokenize{reference/javascript_api:web.pyeval}]{\sphinxcrossref{
web.pyeval
}}}
\item {} {\hyperref[\detokenize{reference/javascript_api:web.session}]{\sphinxcrossref{
web.session
}}}
\item {} {\hyperref[\detokenize{reference/javascript_api:web.translation}]{\sphinxcrossref{
web.translation
}}}
\end{itemize}

\end{description}\end{quote}


\begin{fulllineitems}
\phantomsection\label{\detokenize{reference/javascript_api:data}}\pysigline{\sphinxbfcode{\sphinxupquote{namespace }}\sphinxbfcode{\sphinxupquote{data}}}
\end{fulllineitems}



\begin{fulllineitems}
\phantomsection\label{\detokenize{reference/javascript_api:serialize_sort}}\pysiglinewithargsret{\sphinxbfcode{\sphinxupquote{function }}\sphinxbfcode{\sphinxupquote{serialize\_sort}}}{\emph{criterion}}{{ $\rightarrow$ String}}
Serializes the sort criterion array of a dataset into a form which can be
consumed by OpenERP’s RPC APIs.
\begin{quote}\begin{description}
\item[{Parameters}] \leavevmode\begin{itemize}

\sphinxstylestrong{criterion} (\sphinxstyleliteralemphasis{\sphinxupquote{Array}}) \textendash{} array of fields, from first to last criteria, prefixed with ‘-‘ for reverse sorting
\end{itemize}

\item[{Returns}] \leavevmode
SQL-like sorting string ({}`{}`ORDER BY{}`{}`) clause

\item[{Return Type}] \leavevmode
\sphinxstyleliteralemphasis{\sphinxupquote{String}}

\end{description}\end{quote}

\end{fulllineitems}



\begin{fulllineitems}
\phantomsection\label{\detokenize{reference/javascript_api:deserialize_sort}}\pysiglinewithargsret{\sphinxbfcode{\sphinxupquote{function }}\sphinxbfcode{\sphinxupquote{deserialize\_sort}}}{\emph{criterion}}{}
Reverse of the serialize\_sort function: convert an array of SQL-like sort
descriptors into a list of fields prefixed with ‘-‘ if necessary.
\begin{quote}\begin{description}
\item[{Parameters}] \leavevmode\begin{itemize}

\sphinxstylestrong{criterion}
\end{itemize}

\end{description}\end{quote}

\end{fulllineitems}


\end{fulllineitems}

\phantomsection\label{\detokenize{reference/javascript_api:module-web.Registry}}

\begin{fulllineitems}
\phantomsection\label{\detokenize{reference/javascript_api:web.Registry}}\pysigline{\sphinxbfcode{\sphinxupquote{module }}\sphinxbfcode{\sphinxupquote{web.Registry}}}~~\begin{quote}\begin{description}
\item[{Exports}] \leavevmode{\hyperref[\detokenize{reference/javascript_api:web.Registry.Registry}]{\sphinxcrossref{
Registry
}}}
\item[{Depends On}] \leavevmode\begin{itemize}
\item {} {\hyperref[\detokenize{reference/javascript_api:web.Class}]{\sphinxcrossref{
web.Class
}}}
\end{itemize}

\end{description}\end{quote}


\begin{fulllineitems}
\phantomsection\label{\detokenize{reference/javascript_api:Registry}}\pysiglinewithargsret{\sphinxbfcode{\sphinxupquote{class }}\sphinxbfcode{\sphinxupquote{Registry}}}{\sphinxoptional{\emph{mapping}}}{}~\begin{quote}\begin{description}
\item[{Extends}] \leavevmode{\hyperref[\detokenize{reference/javascript_api:web.Class.Class}]{\sphinxcrossref{
Class
}}}
\item[{Parameters}] \leavevmode\begin{itemize}

\sphinxstylestrong{mapping} (\sphinxstyleliteralemphasis{\sphinxupquote{Object}}) \textendash{} the initial data in the registry
\end{itemize}

\end{description}\end{quote}

The registry is really pretty much only a mapping from some keys to some
values. The Registry class only add a few simple methods around that to make
it nicer and slightly safer.

Note that registries have a fundamental problem: the value that you try to
get in a registry might not have been added yet, so of course, you need to
make sure that your dependencies are solid.  For this reason, it is a good
practice to avoid using the registry if you can simply import what you need
with the ‘require’ statement.

However, on the flip side, sometimes you cannot just simply import something
because we would have a dependency cycle.  In that case, registries might
help.


\begin{fulllineitems}
\phantomsection\label{\detokenize{reference/javascript_api:add}}\pysiglinewithargsret{\sphinxbfcode{\sphinxupquote{method }}\sphinxbfcode{\sphinxupquote{add}}}{\emph{key}, \emph{value}}{{ $\rightarrow$ Registry}}
Add a key (and a value) to the registry.
\begin{quote}\begin{description}
\item[{Parameters}] \leavevmode\begin{itemize}

\sphinxstylestrong{key} (\sphinxstyleliteralemphasis{\sphinxupquote{string}})

\sphinxstylestrong{value} (\sphinxstyleliteralemphasis{\sphinxupquote{any}})
\end{itemize}

\item[{Returns}] \leavevmode
can be used to chain add calls.

\item[{Return Type}] \leavevmode
{\hyperref[\detokenize{reference/javascript_api:web.Registry.Registry}]{\sphinxcrossref{\sphinxstyleliteralemphasis{\sphinxupquote{Registry}}}}}

\end{description}\end{quote}

\end{fulllineitems}



\begin{fulllineitems}
\phantomsection\label{\detokenize{reference/javascript_api:contains}}\pysiglinewithargsret{\sphinxbfcode{\sphinxupquote{method }}\sphinxbfcode{\sphinxupquote{contains}}}{\emph{key}}{{ $\rightarrow$ boolean}}
Check if the registry contains the value
\begin{quote}\begin{description}
\item[{Parameters}] \leavevmode\begin{itemize}

\sphinxstylestrong{key} (\sphinxstyleliteralemphasis{\sphinxupquote{string}})
\end{itemize}

\item[{Return Type}] \leavevmode
\sphinxstyleliteralemphasis{\sphinxupquote{boolean}}

\end{description}\end{quote}

\end{fulllineitems}



\begin{fulllineitems}
\phantomsection\label{\detokenize{reference/javascript_api:extend}}\pysiglinewithargsret{\sphinxbfcode{\sphinxupquote{method }}\sphinxbfcode{\sphinxupquote{extend}}}{\sphinxoptional{\emph{mapping}}}{}
Creates and returns a copy of the current mapping, with the provided
mapping argument added in (replacing existing keys if needed)

Parent and child remain linked, a new key in the parent (which is not
overwritten by the child) will appear in the child.
\begin{quote}\begin{description}
\item[{Parameters}] \leavevmode\begin{itemize}

\sphinxstylestrong{mapping}=\sphinxstyleemphasis{\{\}} (\sphinxstyleliteralemphasis{\sphinxupquote{Object}}) \textendash{} a mapping of keys to object-paths
\end{itemize}

\end{description}\end{quote}

\end{fulllineitems}



\begin{fulllineitems}
\phantomsection\label{\detokenize{reference/javascript_api:get}}\pysiglinewithargsret{\sphinxbfcode{\sphinxupquote{method }}\sphinxbfcode{\sphinxupquote{get}}}{\emph{key}}{{ $\rightarrow$ any}}
Returns the value associated to the given key.
\begin{quote}\begin{description}
\item[{Parameters}] \leavevmode\begin{itemize}

\sphinxstylestrong{key} (\sphinxstyleliteralemphasis{\sphinxupquote{string}})
\end{itemize}

\item[{Return Type}] \leavevmode
\sphinxstyleliteralemphasis{\sphinxupquote{any}}

\end{description}\end{quote}

\end{fulllineitems}



\begin{fulllineitems}
\phantomsection\label{\detokenize{reference/javascript_api:getAny}}\pysiglinewithargsret{\sphinxbfcode{\sphinxupquote{method }}\sphinxbfcode{\sphinxupquote{getAny}}}{\emph{keys}}{{ $\rightarrow$ any}}
Tries a number of keys, and returns the first object matching one of
the keys.
\begin{quote}\begin{description}
\item[{Parameters}] \leavevmode\begin{itemize}

\sphinxstylestrong{keys} (\sphinxstyleliteralemphasis{\sphinxupquote{Array}}\textless{}\sphinxstyleliteralemphasis{\sphinxupquote{string}}\textgreater{}) \textendash{} a sequence of keys to fetch the object for
\end{itemize}

\item[{Returns}] \leavevmode
the first result found matching an object

\item[{Return Type}] \leavevmode
\sphinxstyleliteralemphasis{\sphinxupquote{any}}

\end{description}\end{quote}

\end{fulllineitems}


\end{fulllineitems}



\begin{fulllineitems}
\phantomsection\label{\detokenize{reference/javascript_api:Registry}}\pysiglinewithargsret{\sphinxbfcode{\sphinxupquote{class }}\sphinxbfcode{\sphinxupquote{Registry}}}{\sphinxoptional{\emph{mapping}}}{}~\begin{quote}\begin{description}
\item[{Extends}] \leavevmode{\hyperref[\detokenize{reference/javascript_api:web.Class.Class}]{\sphinxcrossref{
Class
}}}
\item[{Parameters}] \leavevmode\begin{itemize}

\sphinxstylestrong{mapping} (\sphinxstyleliteralemphasis{\sphinxupquote{Object}}) \textendash{} the initial data in the registry
\end{itemize}

\end{description}\end{quote}

The registry is really pretty much only a mapping from some keys to some
values. The Registry class only add a few simple methods around that to make
it nicer and slightly safer.

Note that registries have a fundamental problem: the value that you try to
get in a registry might not have been added yet, so of course, you need to
make sure that your dependencies are solid.  For this reason, it is a good
practice to avoid using the registry if you can simply import what you need
with the ‘require’ statement.

However, on the flip side, sometimes you cannot just simply import something
because we would have a dependency cycle.  In that case, registries might
help.


\begin{fulllineitems}
\phantomsection\label{\detokenize{reference/javascript_api:add}}\pysiglinewithargsret{\sphinxbfcode{\sphinxupquote{method }}\sphinxbfcode{\sphinxupquote{add}}}{\emph{key}, \emph{value}}{{ $\rightarrow$ Registry}}
Add a key (and a value) to the registry.
\begin{quote}\begin{description}
\item[{Parameters}] \leavevmode\begin{itemize}

\sphinxstylestrong{key} (\sphinxstyleliteralemphasis{\sphinxupquote{string}})

\sphinxstylestrong{value} (\sphinxstyleliteralemphasis{\sphinxupquote{any}})
\end{itemize}

\item[{Returns}] \leavevmode
can be used to chain add calls.

\item[{Return Type}] \leavevmode
{\hyperref[\detokenize{reference/javascript_api:web.Registry.Registry}]{\sphinxcrossref{\sphinxstyleliteralemphasis{\sphinxupquote{Registry}}}}}

\end{description}\end{quote}

\end{fulllineitems}



\begin{fulllineitems}
\phantomsection\label{\detokenize{reference/javascript_api:contains}}\pysiglinewithargsret{\sphinxbfcode{\sphinxupquote{method }}\sphinxbfcode{\sphinxupquote{contains}}}{\emph{key}}{{ $\rightarrow$ boolean}}
Check if the registry contains the value
\begin{quote}\begin{description}
\item[{Parameters}] \leavevmode\begin{itemize}

\sphinxstylestrong{key} (\sphinxstyleliteralemphasis{\sphinxupquote{string}})
\end{itemize}

\item[{Return Type}] \leavevmode
\sphinxstyleliteralemphasis{\sphinxupquote{boolean}}

\end{description}\end{quote}

\end{fulllineitems}



\begin{fulllineitems}
\phantomsection\label{\detokenize{reference/javascript_api:extend}}\pysiglinewithargsret{\sphinxbfcode{\sphinxupquote{method }}\sphinxbfcode{\sphinxupquote{extend}}}{\sphinxoptional{\emph{mapping}}}{}
Creates and returns a copy of the current mapping, with the provided
mapping argument added in (replacing existing keys if needed)

Parent and child remain linked, a new key in the parent (which is not
overwritten by the child) will appear in the child.
\begin{quote}\begin{description}
\item[{Parameters}] \leavevmode\begin{itemize}

\sphinxstylestrong{mapping}=\sphinxstyleemphasis{\{\}} (\sphinxstyleliteralemphasis{\sphinxupquote{Object}}) \textendash{} a mapping of keys to object-paths
\end{itemize}

\end{description}\end{quote}

\end{fulllineitems}



\begin{fulllineitems}
\phantomsection\label{\detokenize{reference/javascript_api:get}}\pysiglinewithargsret{\sphinxbfcode{\sphinxupquote{method }}\sphinxbfcode{\sphinxupquote{get}}}{\emph{key}}{{ $\rightarrow$ any}}
Returns the value associated to the given key.
\begin{quote}\begin{description}
\item[{Parameters}] \leavevmode\begin{itemize}

\sphinxstylestrong{key} (\sphinxstyleliteralemphasis{\sphinxupquote{string}})
\end{itemize}

\item[{Return Type}] \leavevmode
\sphinxstyleliteralemphasis{\sphinxupquote{any}}

\end{description}\end{quote}

\end{fulllineitems}



\begin{fulllineitems}
\phantomsection\label{\detokenize{reference/javascript_api:getAny}}\pysiglinewithargsret{\sphinxbfcode{\sphinxupquote{method }}\sphinxbfcode{\sphinxupquote{getAny}}}{\emph{keys}}{{ $\rightarrow$ any}}
Tries a number of keys, and returns the first object matching one of
the keys.
\begin{quote}\begin{description}
\item[{Parameters}] \leavevmode\begin{itemize}

\sphinxstylestrong{keys} (\sphinxstyleliteralemphasis{\sphinxupquote{Array}}\textless{}\sphinxstyleliteralemphasis{\sphinxupquote{string}}\textgreater{}) \textendash{} a sequence of keys to fetch the object for
\end{itemize}

\item[{Returns}] \leavevmode
the first result found matching an object

\item[{Return Type}] \leavevmode
\sphinxstyleliteralemphasis{\sphinxupquote{any}}

\end{description}\end{quote}

\end{fulllineitems}


\end{fulllineitems}


\end{fulllineitems}

\phantomsection\label{\detokenize{reference/javascript_api:module-website.ace}}

\begin{fulllineitems}
\phantomsection\label{\detokenize{reference/javascript_api:website.ace}}\pysigline{\sphinxbfcode{\sphinxupquote{module }}\sphinxbfcode{\sphinxupquote{website.ace}}}~~\begin{quote}\begin{description}
\item[{Exports}] \leavevmode{\hyperref[\detokenize{reference/javascript_api:website.ace.WebsiteAceEditor}]{\sphinxcrossref{
WebsiteAceEditor
}}}
\item[{Depends On}] \leavevmode\begin{itemize}
\item {} {\hyperref[\detokenize{reference/javascript_api:web_editor.ace}]{\sphinxcrossref{
web\_editor.ace
}}}
\end{itemize}

\end{description}\end{quote}


\begin{fulllineitems}
\phantomsection\label{\detokenize{reference/javascript_api:WebsiteAceEditor}}\pysiglinewithargsret{\sphinxbfcode{\sphinxupquote{class }}\sphinxbfcode{\sphinxupquote{WebsiteAceEditor}}}{}{}~\begin{quote}\begin{description}
\item[{Extends}] \leavevmode{\hyperref[\detokenize{reference/javascript_api:web_editor.ace.ViewEditor}]{\sphinxcrossref{
ViewEditor
}}}
\end{description}\end{quote}

Extends the default view editor so that the URL hash is updated with view ID

\end{fulllineitems}



\begin{fulllineitems}
\phantomsection\label{\detokenize{reference/javascript_api:WebsiteAceEditor}}\pysiglinewithargsret{\sphinxbfcode{\sphinxupquote{class }}\sphinxbfcode{\sphinxupquote{WebsiteAceEditor}}}{}{}~\begin{quote}\begin{description}
\item[{Extends}] \leavevmode{\hyperref[\detokenize{reference/javascript_api:web_editor.ace.ViewEditor}]{\sphinxcrossref{
ViewEditor
}}}
\end{description}\end{quote}

Extends the default view editor so that the URL hash is updated with view ID

\end{fulllineitems}


\end{fulllineitems}

\phantomsection\label{\detokenize{reference/javascript_api:module-web.ListView}}

\begin{fulllineitems}
\phantomsection\label{\detokenize{reference/javascript_api:web.ListView}}\pysigline{\sphinxbfcode{\sphinxupquote{module }}\sphinxbfcode{\sphinxupquote{web.ListView}}}~~\begin{quote}\begin{description}
\item[{Exports}] \leavevmode{\hyperref[\detokenize{reference/javascript_api:web.ListView.ListView}]{\sphinxcrossref{
ListView
}}}
\item[{Depends On}] \leavevmode\begin{itemize}
\item {} {\hyperref[\detokenize{reference/javascript_api:web.BasicView}]{\sphinxcrossref{
web.BasicView
}}}
\item {} {\hyperref[\detokenize{reference/javascript_api:web.ListController}]{\sphinxcrossref{
web.ListController
}}}
\item {} {\hyperref[\detokenize{reference/javascript_api:web.ListRenderer}]{\sphinxcrossref{
web.ListRenderer
}}}
\item {} {\hyperref[\detokenize{reference/javascript_api:web.core}]{\sphinxcrossref{
web.core
}}}
\end{itemize}

\end{description}\end{quote}


\begin{fulllineitems}
\phantomsection\label{\detokenize{reference/javascript_api:ListView}}\pysiglinewithargsret{\sphinxbfcode{\sphinxupquote{class }}\sphinxbfcode{\sphinxupquote{ListView}}}{}{}~\begin{quote}\begin{description}
\item[{Extends}] \leavevmode{\hyperref[\detokenize{reference/javascript_api:web.BasicView.BasicView}]{\sphinxcrossref{
BasicView
}}}
\end{description}\end{quote}

\end{fulllineitems}


\end{fulllineitems}

\phantomsection\label{\detokenize{reference/javascript_api:module-web.BasicView}}

\begin{fulllineitems}
\phantomsection\label{\detokenize{reference/javascript_api:web.BasicView}}\pysigline{\sphinxbfcode{\sphinxupquote{module }}\sphinxbfcode{\sphinxupquote{web.BasicView}}}~~\begin{quote}\begin{description}
\item[{Exports}] \leavevmode{\hyperref[\detokenize{reference/javascript_api:web.BasicView.BasicView}]{\sphinxcrossref{
BasicView
}}}
\item[{Depends On}] \leavevmode\begin{itemize}
\item {} {\hyperref[\detokenize{reference/javascript_api:web.AbstractView}]{\sphinxcrossref{
web.AbstractView
}}}
\item {} {\hyperref[\detokenize{reference/javascript_api:web.BasicController}]{\sphinxcrossref{
web.BasicController
}}}
\item {} {\hyperref[\detokenize{reference/javascript_api:web.BasicModel}]{\sphinxcrossref{
web.BasicModel
}}}
\end{itemize}

\end{description}\end{quote}


\begin{fulllineitems}
\phantomsection\label{\detokenize{reference/javascript_api:BasicView}}\pysiglinewithargsret{\sphinxbfcode{\sphinxupquote{class }}\sphinxbfcode{\sphinxupquote{BasicView}}}{\emph{viewInfo}}{}~\begin{quote}\begin{description}
\item[{Extends}] \leavevmode{\hyperref[\detokenize{reference/javascript_api:web.AbstractView.AbstractView}]{\sphinxcrossref{
AbstractView
}}}
\item[{Parameters}] \leavevmode\begin{itemize}

\sphinxstylestrong{viewInfo}
\end{itemize}

\end{description}\end{quote}

\end{fulllineitems}


\end{fulllineitems}

\phantomsection\label{\detokenize{reference/javascript_api:module-web.AbstractModel}}

\begin{fulllineitems}
\phantomsection\label{\detokenize{reference/javascript_api:web.AbstractModel}}\pysigline{\sphinxbfcode{\sphinxupquote{module }}\sphinxbfcode{\sphinxupquote{web.AbstractModel}}}~~\begin{quote}\begin{description}
\item[{Exports}] \leavevmode{\hyperref[\detokenize{reference/javascript_api:web.AbstractModel.AbstractModel}]{\sphinxcrossref{
AbstractModel
}}}
\item[{Depends On}] \leavevmode\begin{itemize}
\item {} {\hyperref[\detokenize{reference/javascript_api:web.Class}]{\sphinxcrossref{
web.Class
}}}
\item {} {\hyperref[\detokenize{reference/javascript_api:web.ServicesMixin}]{\sphinxcrossref{
web.ServicesMixin
}}}
\item {} {\hyperref[\detokenize{reference/javascript_api:web.field_utils}]{\sphinxcrossref{
web.field\_utils
}}}
\item {} {\hyperref[\detokenize{reference/javascript_api:web.mixins}]{\sphinxcrossref{
web.mixins
}}}
\end{itemize}

\end{description}\end{quote}


\begin{fulllineitems}
\phantomsection\label{\detokenize{reference/javascript_api:AbstractModel}}\pysiglinewithargsret{\sphinxbfcode{\sphinxupquote{class }}\sphinxbfcode{\sphinxupquote{AbstractModel}}}{\emph{parent}}{}~\begin{quote}\begin{description}
\item[{Extends}] \leavevmode{\hyperref[\detokenize{reference/javascript_api:web.Class.Class}]{\sphinxcrossref{
Class
}}}
\item[{Mixes}] \leavevmode\begin{itemize}
\item {} {\hyperref[\detokenize{reference/javascript_api:web.mixins.EventDispatcherMixin}]{\sphinxcrossref{
EventDispatcherMixin
}}}
\item {} {\hyperref[\detokenize{reference/javascript_api:web.ServicesMixin.ServicesMixin}]{\sphinxcrossref{
ServicesMixin
}}}
\end{itemize}

\item[{Parameters}] \leavevmode\begin{itemize}

\sphinxstylestrong{parent} ({\hyperref[\detokenize{reference/javascript_api:Widget}]{\sphinxcrossref{\sphinxstyleliteralemphasis{\sphinxupquote{Widget}}}}})
\end{itemize}

\end{description}\end{quote}


\begin{fulllineitems}
\phantomsection\label{\detokenize{reference/javascript_api:get}}\pysiglinewithargsret{\sphinxbfcode{\sphinxupquote{method }}\sphinxbfcode{\sphinxupquote{get}}}{}{{ $\rightarrow$ *}}
This method should return the complete state necessary for the renderer
to display the currently viewed data.
\begin{quote}\begin{description}
\item[{Return Type}] \leavevmode
\sphinxstyleliteralemphasis{\sphinxupquote{any}}

\end{description}\end{quote}

\end{fulllineitems}



\begin{fulllineitems}
\phantomsection\label{\detokenize{reference/javascript_api:load}}\pysiglinewithargsret{\sphinxbfcode{\sphinxupquote{method }}\sphinxbfcode{\sphinxupquote{load}}}{\emph{params}}{{ $\rightarrow$ Deferred}}
The load method is called once in a model, when we load the data for the
first time.  The method returns (a deferred that resolves to) some kind
of token/handle.  The handle can then be used with the get method to
access a representation of the data.
\begin{quote}\begin{description}
\item[{Parameters}] \leavevmode\begin{itemize}

\sphinxstylestrong{params} ({\hyperref[\detokenize{reference/javascript_api:web.AbstractModel.LoadParams}]{\sphinxcrossref{\sphinxstyleliteralemphasis{\sphinxupquote{LoadParams}}}}})
\end{itemize}

\item[{Returns}] \leavevmode
The deferred resolves to some kind of handle

\item[{Return Type}] \leavevmode
\sphinxstyleliteralemphasis{\sphinxupquote{Deferred}}

\end{description}\end{quote}


\begin{fulllineitems}
\phantomsection\label{\detokenize{reference/javascript_api:LoadParams}}\pysiglinewithargsret{\sphinxbfcode{\sphinxupquote{class }}\sphinxbfcode{\sphinxupquote{LoadParams}}}{}{}~

\begin{fulllineitems}
\phantomsection\label{\detokenize{reference/javascript_api:modelName}}\pysigline{\sphinxbfcode{\sphinxupquote{attribute }}\sphinxbfcode{\sphinxupquote{modelName}} string}
the name of the model

\end{fulllineitems}


\end{fulllineitems}


\end{fulllineitems}



\begin{fulllineitems}
\phantomsection\label{\detokenize{reference/javascript_api:reload}}\pysiglinewithargsret{\sphinxbfcode{\sphinxupquote{function }}\sphinxbfcode{\sphinxupquote{reload}}}{\emph{params}}{{ $\rightarrow$ Deferred}}
When something changes, the data may need to be refetched.  This is the
job for this method: reloading (only if necessary) all the data and
making sure that they are ready to be redisplayed.
\begin{quote}\begin{description}
\item[{Parameters}] \leavevmode\begin{itemize}

\sphinxstylestrong{params} (\sphinxstyleliteralemphasis{\sphinxupquote{Object}})
\end{itemize}

\item[{Return Type}] \leavevmode
\sphinxstyleliteralemphasis{\sphinxupquote{Deferred}}

\end{description}\end{quote}

\end{fulllineitems}


\end{fulllineitems}


\end{fulllineitems}

\phantomsection\label{\detokenize{reference/javascript_api:module-web_tour.TourManager}}

\begin{fulllineitems}
\phantomsection\label{\detokenize{reference/javascript_api:web_tour.TourManager}}\pysigline{\sphinxbfcode{\sphinxupquote{module }}\sphinxbfcode{\sphinxupquote{web\_tour.TourManager}}}~~\begin{quote}\begin{description}
\item[{Exports}] \leavevmode{\hyperref[\detokenize{reference/javascript_api:web_tour.TourManager.}]{\sphinxcrossref{
\textless{}anonymous\textgreater{}
}}}
\item[{Depends On}] \leavevmode\begin{itemize}
\item {} {\hyperref[\detokenize{reference/javascript_api:web.ServicesMixin}]{\sphinxcrossref{
web.ServicesMixin
}}}
\item {} {\hyperref[\detokenize{reference/javascript_api:web.core}]{\sphinxcrossref{
web.core
}}}
\item {} {\hyperref[\detokenize{reference/javascript_api:web.local_storage}]{\sphinxcrossref{
web.local\_storage
}}}
\item {} {\hyperref[\detokenize{reference/javascript_api:web.mixins}]{\sphinxcrossref{
web.mixins
}}}
\item {} {\hyperref[\detokenize{reference/javascript_api:web.rainbow_man}]{\sphinxcrossref{
web.rainbow\_man
}}}
\item {} {\hyperref[\detokenize{reference/javascript_api:web.session}]{\sphinxcrossref{
web.session
}}}
\item {} {\hyperref[\detokenize{reference/javascript_api:web_tour.Tip}]{\sphinxcrossref{
web\_tour.Tip
}}}
\end{itemize}

\end{description}\end{quote}


\begin{fulllineitems}
\phantomsection\label{\detokenize{reference/javascript_api:web_tour.TourManager.}}\pysiglinewithargsret{\sphinxbfcode{\sphinxupquote{class }}\sphinxbfcode{\sphinxupquote{}}}{\emph{parent}, \emph{consumed\_tours}}{}~\begin{quote}\begin{description}
\item[{Extends}] \leavevmode{\hyperref[\detokenize{reference/javascript_api:web.Class.Class}]{\sphinxcrossref{
Class
}}}
\item[{Mixes}] \leavevmode\begin{itemize}
\item {} {\hyperref[\detokenize{reference/javascript_api:web.mixins.EventDispatcherMixin}]{\sphinxcrossref{
EventDispatcherMixin
}}}
\item {} {\hyperref[\detokenize{reference/javascript_api:web.ServicesMixin.ServicesMixin}]{\sphinxcrossref{
ServicesMixin
}}}
\end{itemize}

\item[{Parameters}] \leavevmode\begin{itemize}

\sphinxstylestrong{parent}

\sphinxstylestrong{consumed\_tours}
\end{itemize}

\end{description}\end{quote}


\begin{fulllineitems}
\phantomsection\label{\detokenize{reference/javascript_api:register}}\pysiglinewithargsret{\sphinxbfcode{\sphinxupquote{function }}\sphinxbfcode{\sphinxupquote{register}}}{\emph{name}\sphinxoptional{, \emph{options}}, \emph{steps}}{}
Registers a tour described by the following arguments \sphinxstyleemphasis{in order}
\begin{quote}\begin{description}
\item[{Parameters}] \leavevmode\begin{itemize}

\sphinxstylestrong{name} (\sphinxstyleliteralemphasis{\sphinxupquote{string}}) \textendash{} tour’s name

\sphinxstylestrong{options} ({\hyperref[\detokenize{reference/javascript_api:web_tour.TourManager.RegisterOptions}]{\sphinxcrossref{\sphinxstyleliteralemphasis{\sphinxupquote{RegisterOptions}}}}}) \textendash{} options (optional), available options are:

\sphinxstylestrong{steps} (\sphinxstyleliteralemphasis{\sphinxupquote{Array}}\textless{}\sphinxstyleliteralemphasis{\sphinxupquote{Object}}\textgreater{}) \textendash{} steps’ descriptions, each step being an object
                    containing a tip description
\end{itemize}

\end{description}\end{quote}


\begin{fulllineitems}
\phantomsection\label{\detokenize{reference/javascript_api:RegisterOptions}}\pysiglinewithargsret{\sphinxbfcode{\sphinxupquote{class }}\sphinxbfcode{\sphinxupquote{RegisterOptions}}}{}{}
options (optional), available options are:


\begin{fulllineitems}
\phantomsection\label{\detokenize{reference/javascript_api:test}}\pysigline{\sphinxbfcode{\sphinxupquote{attribute }}\sphinxbfcode{\sphinxupquote{test}} boolean}
true if this is only for tests

\end{fulllineitems}



\begin{fulllineitems}
\phantomsection\label{\detokenize{reference/javascript_api:skip_enabled}}\pysigline{\sphinxbfcode{\sphinxupquote{attribute }}\sphinxbfcode{\sphinxupquote{skip\_enabled}} boolean}
true to add a link in its tips to consume the whole tour

\end{fulllineitems}



\begin{fulllineitems}
\phantomsection\label{\detokenize{reference/javascript_api:url}}\pysigline{\sphinxbfcode{\sphinxupquote{attribute }}\sphinxbfcode{\sphinxupquote{url}} string}
the url to load when manually running the tour

\end{fulllineitems}



\begin{fulllineitems}
\phantomsection\label{\detokenize{reference/javascript_api:rainbowMan}}\pysigline{\sphinxbfcode{\sphinxupquote{attribute }}\sphinxbfcode{\sphinxupquote{rainbowMan}} boolean}
whether or not the rainbowman must be shown at the end of the tour

\end{fulllineitems}



\begin{fulllineitems}
\phantomsection\label{\detokenize{reference/javascript_api:wait_for}}\pysigline{\sphinxbfcode{\sphinxupquote{attribute }}\sphinxbfcode{\sphinxupquote{wait\_for}} Deferred}
indicates when the tour can be started

\end{fulllineitems}


\end{fulllineitems}


\end{fulllineitems}



\begin{fulllineitems}
\phantomsection\label{\detokenize{reference/javascript_api:update}}\pysiglinewithargsret{\sphinxbfcode{\sphinxupquote{function }}\sphinxbfcode{\sphinxupquote{update}}}{\emph{tour\_name}}{}
Checks for tooltips to activate (only from the running tour or specified tour if there
is one, from all active tours otherwise). Should be called each time the DOM changes.
\begin{quote}\begin{description}
\item[{Parameters}] \leavevmode\begin{itemize}

\sphinxstylestrong{tour\_name}
\end{itemize}

\end{description}\end{quote}

\end{fulllineitems}



\begin{fulllineitems}
\phantomsection\label{\detokenize{reference/javascript_api:STEPS}}\pysigline{\sphinxbfcode{\sphinxupquote{namespace }}\sphinxbfcode{\sphinxupquote{STEPS}}}
Tour predefined steps

\end{fulllineitems}


\end{fulllineitems}


\end{fulllineitems}

\phantomsection\label{\detokenize{reference/javascript_api:module-website_forum.website_forum}}

\begin{fulllineitems}
\phantomsection\label{\detokenize{reference/javascript_api:website_forum.website_forum}}\pysigline{\sphinxbfcode{\sphinxupquote{module }}\sphinxbfcode{\sphinxupquote{website\_forum.website\_forum}}}~~\begin{quote}\begin{description}
\item[{Exports}] \leavevmode{\hyperref[\detokenize{reference/javascript_api:website_forum.website_forum.}]{\sphinxcrossref{
\textless{}anonymous\textgreater{}
}}}
\item[{Depends On}] \leavevmode\begin{itemize}
\item {} {\hyperref[\detokenize{reference/javascript_api:web.ajax}]{\sphinxcrossref{
web.ajax
}}}
\item {} {\hyperref[\detokenize{reference/javascript_api:web.core}]{\sphinxcrossref{
web.core
}}}
\end{itemize}

\end{description}\end{quote}


\begin{fulllineitems}
\phantomsection\label{\detokenize{reference/javascript_api:website_forum.website_forum.}}\pysigline{\sphinxbfcode{\sphinxupquote{namespace }}\sphinxbfcode{\sphinxupquote{}}}
\end{fulllineitems}


\end{fulllineitems}

\phantomsection\label{\detokenize{reference/javascript_api:module-web.CalendarController}}

\begin{fulllineitems}
\phantomsection\label{\detokenize{reference/javascript_api:web.CalendarController}}\pysigline{\sphinxbfcode{\sphinxupquote{module }}\sphinxbfcode{\sphinxupquote{web.CalendarController}}}~~\begin{quote}\begin{description}
\item[{Exports}] \leavevmode{\hyperref[\detokenize{reference/javascript_api:web.CalendarController.CalendarController}]{\sphinxcrossref{
CalendarController
}}}
\item[{Depends On}] \leavevmode\begin{itemize}
\item {} {\hyperref[\detokenize{reference/javascript_api:web.AbstractController}]{\sphinxcrossref{
web.AbstractController
}}}
\item {} {\hyperref[\detokenize{reference/javascript_api:web.CalendarQuickCreate}]{\sphinxcrossref{
web.CalendarQuickCreate
}}}
\item {} {\hyperref[\detokenize{reference/javascript_api:web.Dialog}]{\sphinxcrossref{
web.Dialog
}}}
\item {} {\hyperref[\detokenize{reference/javascript_api:web.core}]{\sphinxcrossref{
web.core
}}}
\item {} {\hyperref[\detokenize{reference/javascript_api:web.view_dialogs}]{\sphinxcrossref{
web.view\_dialogs
}}}
\end{itemize}

\end{description}\end{quote}


\begin{fulllineitems}
\phantomsection\label{\detokenize{reference/javascript_api:CalendarController}}\pysiglinewithargsret{\sphinxbfcode{\sphinxupquote{class }}\sphinxbfcode{\sphinxupquote{CalendarController}}}{\emph{parent}, \emph{model}, \emph{renderer}, \emph{params}}{}~\begin{quote}\begin{description}
\item[{Extends}] \leavevmode{\hyperref[\detokenize{reference/javascript_api:web.AbstractController.AbstractController}]{\sphinxcrossref{
AbstractController
}}}
\item[{Parameters}] \leavevmode\begin{itemize}

\sphinxstylestrong{parent} ({\hyperref[\detokenize{reference/javascript_api:Widget}]{\sphinxcrossref{\sphinxstyleliteralemphasis{\sphinxupquote{Widget}}}}})

\sphinxstylestrong{model} ({\hyperref[\detokenize{reference/javascript_api:AbstractModel}]{\sphinxcrossref{\sphinxstyleliteralemphasis{\sphinxupquote{AbstractModel}}}}})

\sphinxstylestrong{renderer} ({\hyperref[\detokenize{reference/javascript_api:AbstractRenderer}]{\sphinxcrossref{\sphinxstyleliteralemphasis{\sphinxupquote{AbstractRenderer}}}}})

\sphinxstylestrong{params} (\sphinxstyleliteralemphasis{\sphinxupquote{Object}})
\end{itemize}

\end{description}\end{quote}


\begin{fulllineitems}
\phantomsection\label{\detokenize{reference/javascript_api:renderButtons}}\pysiglinewithargsret{\sphinxbfcode{\sphinxupquote{method }}\sphinxbfcode{\sphinxupquote{renderButtons}}}{\sphinxoptional{\emph{\$node}}}{}
Render the buttons according to the CalendarView.buttons template and
add listeners on it. Set this.\$buttons with the produced jQuery element
\begin{quote}\begin{description}
\item[{Parameters}] \leavevmode\begin{itemize}

\sphinxstylestrong{\$node} (\sphinxstyleliteralemphasis{\sphinxupquote{jQueryElement}}) \textendash{} a jQuery node where the rendered buttons
  should be inserted. \$node may be undefined, in which case the Calendar
  inserts them into this.options.\$buttons or into a div of its template
\end{itemize}

\end{description}\end{quote}

\end{fulllineitems}


\end{fulllineitems}


\end{fulllineitems}

\phantomsection\label{\detokenize{reference/javascript_api:module-mail.systray}}

\begin{fulllineitems}
\phantomsection\label{\detokenize{reference/javascript_api:mail.systray}}\pysigline{\sphinxbfcode{\sphinxupquote{module }}\sphinxbfcode{\sphinxupquote{mail.systray}}}~~\begin{quote}\begin{description}
\item[{Exports}] \leavevmode{\hyperref[\detokenize{reference/javascript_api:mail.systray.}]{\sphinxcrossref{
\textless{}anonymous\textgreater{}
}}}
\item[{Depends On}] \leavevmode\begin{itemize}
\item {} {\hyperref[\detokenize{reference/javascript_api:mail.chat_manager}]{\sphinxcrossref{
mail.chat\_manager
}}}
\item {} {\hyperref[\detokenize{reference/javascript_api:web.SystrayMenu}]{\sphinxcrossref{
web.SystrayMenu
}}}
\item {} {\hyperref[\detokenize{reference/javascript_api:web.Widget}]{\sphinxcrossref{
web.Widget
}}}
\item {} {\hyperref[\detokenize{reference/javascript_api:web.config}]{\sphinxcrossref{
web.config
}}}
\item {} {\hyperref[\detokenize{reference/javascript_api:web.core}]{\sphinxcrossref{
web.core
}}}
\item {} {\hyperref[\detokenize{reference/javascript_api:web.session}]{\sphinxcrossref{
web.session
}}}
\end{itemize}

\end{description}\end{quote}


\begin{fulllineitems}
\phantomsection\label{\detokenize{reference/javascript_api:ActivityMenu}}\pysiglinewithargsret{\sphinxbfcode{\sphinxupquote{class }}\sphinxbfcode{\sphinxupquote{ActivityMenu}}}{}{}~\begin{quote}\begin{description}
\item[{Extends}] \leavevmode{\hyperref[\detokenize{reference/javascript_api:web.Widget.Widget}]{\sphinxcrossref{
Widget
}}}
\end{description}\end{quote}

Menu item appended in the systray part of the navbar, redirects to the next activities of all app

\end{fulllineitems}



\begin{fulllineitems}
\phantomsection\label{\detokenize{reference/javascript_api:MessagingMenu}}\pysiglinewithargsret{\sphinxbfcode{\sphinxupquote{class }}\sphinxbfcode{\sphinxupquote{MessagingMenu}}}{}{}~\begin{quote}\begin{description}
\item[{Extends}] \leavevmode{\hyperref[\detokenize{reference/javascript_api:web.Widget.Widget}]{\sphinxcrossref{
Widget
}}}
\end{description}\end{quote}

Menu item appended in the systray part of the navbar

The menu item indicates the counter of needactions + unread messages in chat channels. When
clicking on it, it toggles a dropdown containing a preview of each pinned channels (except
static and mass mailing channels) with a quick link to open them in chat windows. It also
contains a direct link to the Inbox in Discuss.

\end{fulllineitems}



\begin{fulllineitems}
\phantomsection\label{\detokenize{reference/javascript_api:mail.systray.}}\pysigline{\sphinxbfcode{\sphinxupquote{namespace }}\sphinxbfcode{\sphinxupquote{}}}~

\begin{fulllineitems}
\phantomsection\label{\detokenize{reference/javascript_api:ActivityMenu}}\pysiglinewithargsret{\sphinxbfcode{\sphinxupquote{class }}\sphinxbfcode{\sphinxupquote{ActivityMenu}}}{}{}~\begin{quote}\begin{description}
\item[{Extends}] \leavevmode{\hyperref[\detokenize{reference/javascript_api:web.Widget.Widget}]{\sphinxcrossref{
Widget
}}}
\end{description}\end{quote}

Menu item appended in the systray part of the navbar, redirects to the next activities of all app

\end{fulllineitems}


\end{fulllineitems}


\end{fulllineitems}

\phantomsection\label{\detokenize{reference/javascript_api:module-web.PivotView}}

\begin{fulllineitems}
\phantomsection\label{\detokenize{reference/javascript_api:web.PivotView}}\pysigline{\sphinxbfcode{\sphinxupquote{module }}\sphinxbfcode{\sphinxupquote{web.PivotView}}}~~\begin{quote}\begin{description}
\item[{Exports}] \leavevmode{\hyperref[\detokenize{reference/javascript_api:web.PivotView.PivotView}]{\sphinxcrossref{
PivotView
}}}
\item[{Depends On}] \leavevmode\begin{itemize}
\item {} {\hyperref[\detokenize{reference/javascript_api:web.AbstractView}]{\sphinxcrossref{
web.AbstractView
}}}
\item {} {\hyperref[\detokenize{reference/javascript_api:web.PivotController}]{\sphinxcrossref{
web.PivotController
}}}
\item {} {\hyperref[\detokenize{reference/javascript_api:web.PivotModel}]{\sphinxcrossref{
web.PivotModel
}}}
\item {} {\hyperref[\detokenize{reference/javascript_api:web.PivotRenderer}]{\sphinxcrossref{
web.PivotRenderer
}}}
\item {} {\hyperref[\detokenize{reference/javascript_api:web.core}]{\sphinxcrossref{
web.core
}}}
\end{itemize}

\end{description}\end{quote}


\begin{fulllineitems}
\phantomsection\label{\detokenize{reference/javascript_api:PivotView}}\pysiglinewithargsret{\sphinxbfcode{\sphinxupquote{class }}\sphinxbfcode{\sphinxupquote{PivotView}}}{\emph{viewInfo}, \emph{params}}{}~\begin{quote}\begin{description}
\item[{Extends}] \leavevmode{\hyperref[\detokenize{reference/javascript_api:web.AbstractView.AbstractView}]{\sphinxcrossref{
AbstractView
}}}
\item[{Parameters}] \leavevmode\begin{itemize}

\sphinxstylestrong{viewInfo}

\sphinxstylestrong{params} (\sphinxstyleliteralemphasis{\sphinxupquote{Object}})
\end{itemize}

\end{description}\end{quote}

\end{fulllineitems}


\end{fulllineitems}

\phantomsection\label{\detokenize{reference/javascript_api:module-web.kanban_quick_create}}

\begin{fulllineitems}
\phantomsection\label{\detokenize{reference/javascript_api:web.kanban_quick_create}}\pysigline{\sphinxbfcode{\sphinxupquote{module }}\sphinxbfcode{\sphinxupquote{web.kanban\_quick\_create}}}~~\begin{quote}\begin{description}
\item[{Exports}] \leavevmode{\hyperref[\detokenize{reference/javascript_api:web.kanban_quick_create.}]{\sphinxcrossref{
\textless{}anonymous\textgreater{}
}}}
\item[{Depends On}] \leavevmode\begin{itemize}
\item {} {\hyperref[\detokenize{reference/javascript_api:web.Widget}]{\sphinxcrossref{
web.Widget
}}}
\end{itemize}

\end{description}\end{quote}


\begin{fulllineitems}
\phantomsection\label{\detokenize{reference/javascript_api:web.kanban_quick_create.}}\pysigline{\sphinxbfcode{\sphinxupquote{namespace }}\sphinxbfcode{\sphinxupquote{}}}
\end{fulllineitems}


\end{fulllineitems}

\phantomsection\label{\detokenize{reference/javascript_api:module-web.data_manager}}

\begin{fulllineitems}
\phantomsection\label{\detokenize{reference/javascript_api:web.data_manager}}\pysigline{\sphinxbfcode{\sphinxupquote{module }}\sphinxbfcode{\sphinxupquote{web.data\_manager}}}~~\begin{quote}\begin{description}
\item[{Exports}] \leavevmode{\hyperref[\detokenize{reference/javascript_api:web.data_manager.data_manager}]{\sphinxcrossref{
data\_manager
}}}
\item[{Depends On}] \leavevmode\begin{itemize}
\item {} {\hyperref[\detokenize{reference/javascript_api:web.DataManager}]{\sphinxcrossref{
web.DataManager
}}}
\end{itemize}

\end{description}\end{quote}


\begin{fulllineitems}
\phantomsection\label{\detokenize{reference/javascript_api:data_manager}}\pysigline{\sphinxbfcode{\sphinxupquote{object }}\sphinxbfcode{\sphinxupquote{data\_manager}}\sphinxbfcode{\sphinxupquote{ instance of }}{\hyperref[\detokenize{reference/javascript_api:web.DataManager.}]{\sphinxcrossref{}}}}
\end{fulllineitems}


\end{fulllineitems}

\phantomsection\label{\detokenize{reference/javascript_api:module-web.local_storage}}

\begin{fulllineitems}
\phantomsection\label{\detokenize{reference/javascript_api:web.local_storage}}\pysigline{\sphinxbfcode{\sphinxupquote{module }}\sphinxbfcode{\sphinxupquote{web.local\_storage}}}~~\begin{quote}\begin{description}
\item[{Exports}] \leavevmode{\hyperref[\detokenize{reference/javascript_api:web.local_storage.localStorage}]{\sphinxcrossref{
localStorage
}}}
\item[{Depends On}] \leavevmode\begin{itemize}
\item {} {\hyperref[\detokenize{reference/javascript_api:web.Class}]{\sphinxcrossref{
web.Class
}}}
\end{itemize}

\end{description}\end{quote}


\begin{fulllineitems}
\phantomsection\label{\detokenize{reference/javascript_api:localStorage}}\pysigline{\sphinxbfcode{\sphinxupquote{object }}\sphinxbfcode{\sphinxupquote{localStorage}}\sphinxbfcode{\sphinxupquote{ instance of }}RamStorage}
\end{fulllineitems}


\end{fulllineitems}

\phantomsection\label{\detokenize{reference/javascript_api:module-website_sale.website_sale_category}}

\begin{fulllineitems}
\phantomsection\label{\detokenize{reference/javascript_api:website_sale.website_sale_category}}\pysigline{\sphinxbfcode{\sphinxupquote{module }}\sphinxbfcode{\sphinxupquote{website\_sale.website\_sale\_category}}}~~\begin{quote}\begin{description}
\item[{Exports}] \leavevmode{\hyperref[\detokenize{reference/javascript_api:website_sale.website_sale_category.}]{\sphinxcrossref{
\textless{}anonymous\textgreater{}
}}}
\end{description}\end{quote}


\begin{fulllineitems}
\phantomsection\label{\detokenize{reference/javascript_api:website_sale.website_sale_category.}}\pysigline{\sphinxbfcode{\sphinxupquote{namespace }}\sphinxbfcode{\sphinxupquote{}}}
\end{fulllineitems}


\end{fulllineitems}

\phantomsection\label{\detokenize{reference/javascript_api:module-web.Domain}}

\begin{fulllineitems}
\phantomsection\label{\detokenize{reference/javascript_api:web.Domain}}\pysigline{\sphinxbfcode{\sphinxupquote{module }}\sphinxbfcode{\sphinxupquote{web.Domain}}}~~\begin{quote}\begin{description}
\item[{Exports}] \leavevmode{\hyperref[\detokenize{reference/javascript_api:web.Domain.Domain}]{\sphinxcrossref{
Domain
}}}
\item[{Depends On}] \leavevmode\begin{itemize}
\item {} {\hyperref[\detokenize{reference/javascript_api:web.collections}]{\sphinxcrossref{
web.collections
}}}
\item {} {\hyperref[\detokenize{reference/javascript_api:web.pyeval}]{\sphinxcrossref{
web.pyeval
}}}
\end{itemize}

\end{description}\end{quote}


\begin{fulllineitems}
\phantomsection\label{\detokenize{reference/javascript_api:Domain}}\pysiglinewithargsret{\sphinxbfcode{\sphinxupquote{class }}\sphinxbfcode{\sphinxupquote{Domain}}}{\emph{domain}\sphinxoptional{, \emph{evalContext}}}{}~\begin{quote}\begin{description}
\item[{Extends}] \leavevmode{\hyperref[\detokenize{reference/javascript_api:web.collections.Tree}]{\sphinxcrossref{
Tree
}}}
\item[{Parameters}] \leavevmode\begin{itemize}

\sphinxstylestrong{domain} (\sphinxstyleliteralemphasis{\sphinxupquote{string}}\sphinxstyleemphasis{ or }\sphinxstyleliteralemphasis{\sphinxupquote{Array}}\sphinxstyleemphasis{ or }\sphinxstyleliteralemphasis{\sphinxupquote{boolean}}\sphinxstyleemphasis{ or }undefined) \textendash{} The given domain can be:
           * a string representation of the Python prefix-array
             representation of the domain.
           * a JS prefix-array representation of the domain.
           * a boolean where the “true” domain match all records and the
             “false” domain does not match any records.
           * undefined, considered as the false boolean.
           * a number, considered as true except 0 considered as false.

\sphinxstylestrong{evalContext} (\sphinxstyleliteralemphasis{\sphinxupquote{Object}}) \textendash{} in case the given domain is a string, an
                              evaluation context might be needed
\end{itemize}

\end{description}\end{quote}

The Domain Class allows to work with a domain as a tree and provides tools
to manipulate array and string representations of domains.


\begin{fulllineitems}
\phantomsection\label{\detokenize{reference/javascript_api:compute}}\pysiglinewithargsret{\sphinxbfcode{\sphinxupquote{method }}\sphinxbfcode{\sphinxupquote{compute}}}{\emph{values}}{{ $\rightarrow$ boolean}}
Evaluates the domain with a set of values.
\begin{quote}\begin{description}
\item[{Parameters}] \leavevmode\begin{itemize}

\sphinxstylestrong{values} (\sphinxstyleliteralemphasis{\sphinxupquote{Object}}) \textendash{} a mapping \{fieldName -\textgreater{} fieldValue\} (note: all
                       the fields used in the domain should be given a
                       value otherwise the computation will break)
\end{itemize}

\item[{Return Type}] \leavevmode
\sphinxstyleliteralemphasis{\sphinxupquote{boolean}}

\end{description}\end{quote}

\end{fulllineitems}



\begin{fulllineitems}
\phantomsection\label{\detokenize{reference/javascript_api:toArray}}\pysiglinewithargsret{\sphinxbfcode{\sphinxupquote{method }}\sphinxbfcode{\sphinxupquote{toArray}}}{}{{ $\rightarrow$ Array}}
Return the JS prefix-array representation of this domain. Note that all
domains that use the “false” domain cannot be represented as such.
\begin{quote}\begin{description}
\item[{Returns}] \leavevmode
JS prefix-array representation of this domain

\item[{Return Type}] \leavevmode
\sphinxstyleliteralemphasis{\sphinxupquote{Array}}

\end{description}\end{quote}

\end{fulllineitems}



\begin{fulllineitems}
\phantomsection\label{\detokenize{reference/javascript_api:arrayToString}}\pysiglinewithargsret{\sphinxbfcode{\sphinxupquote{method }}\sphinxbfcode{\sphinxupquote{arrayToString}}}{\emph{domain}}{{ $\rightarrow$ string}}
Converts JS prefix-array representation of a domain to a string
representation of the Python prefix-array representation of this domain.
\begin{quote}\begin{description}
\item[{Parameters}] \leavevmode\begin{itemize}

\sphinxstylestrong{domain} (\sphinxstyleliteralemphasis{\sphinxupquote{Array}}\sphinxstyleemphasis{ or }\sphinxstyleliteralemphasis{\sphinxupquote{string}})
\end{itemize}

\item[{Return Type}] \leavevmode
\sphinxstyleliteralemphasis{\sphinxupquote{string}}

\end{description}\end{quote}

\end{fulllineitems}



\begin{fulllineitems}
\phantomsection\label{\detokenize{reference/javascript_api:stringToArray}}\pysiglinewithargsret{\sphinxbfcode{\sphinxupquote{method }}\sphinxbfcode{\sphinxupquote{stringToArray}}}{\emph{domain}\sphinxoptional{, \emph{evalContext}}}{{ $\rightarrow$ Array}}
Converts a string representation of the Python prefix-array
representation of a domain to a JS prefix-array representation of this
domain.
\begin{quote}\begin{description}
\item[{Parameters}] \leavevmode\begin{itemize}

\sphinxstylestrong{domain} (\sphinxstyleliteralemphasis{\sphinxupquote{string}}\sphinxstyleemphasis{ or }\sphinxstyleliteralemphasis{\sphinxupquote{Array}})

\sphinxstylestrong{evalContext} (\sphinxstyleliteralemphasis{\sphinxupquote{Object}})
\end{itemize}

\item[{Return Type}] \leavevmode
\sphinxstyleliteralemphasis{\sphinxupquote{Array}}

\end{description}\end{quote}

\end{fulllineitems}



\begin{fulllineitems}
\phantomsection\label{\detokenize{reference/javascript_api:normalizeArray}}\pysiglinewithargsret{\sphinxbfcode{\sphinxupquote{method }}\sphinxbfcode{\sphinxupquote{normalizeArray}}}{\emph{domain}}{{ $\rightarrow$ Array}}
Makes implicit “\&” operators explicit in the given JS prefix-array
representation of domain (e.g {[}A, B{]} -\textgreater{} {[}“\&”, A, B{]})
\begin{quote}\begin{description}
\item[{Parameters}] \leavevmode\begin{itemize}

\sphinxstylestrong{domain} (\sphinxstyleliteralemphasis{\sphinxupquote{Array}}) \textendash{} the JS prefix-array representation of the domain
                      to normalize (! will be normalized in-place)
\end{itemize}

\item[{Returns}] \leavevmode
the normalized JS prefix-array representation of the
                 given domain

\item[{Return Type}] \leavevmode
\sphinxstyleliteralemphasis{\sphinxupquote{Array}}

\end{description}\end{quote}

\end{fulllineitems}


\end{fulllineitems}



\begin{fulllineitems}
\phantomsection\label{\detokenize{reference/javascript_api:Domain}}\pysiglinewithargsret{\sphinxbfcode{\sphinxupquote{class }}\sphinxbfcode{\sphinxupquote{Domain}}}{\emph{domain}\sphinxoptional{, \emph{evalContext}}}{}~\begin{quote}\begin{description}
\item[{Extends}] \leavevmode{\hyperref[\detokenize{reference/javascript_api:web.collections.Tree}]{\sphinxcrossref{
Tree
}}}
\item[{Parameters}] \leavevmode\begin{itemize}

\sphinxstylestrong{domain} (\sphinxstyleliteralemphasis{\sphinxupquote{string}}\sphinxstyleemphasis{ or }\sphinxstyleliteralemphasis{\sphinxupquote{Array}}\sphinxstyleemphasis{ or }\sphinxstyleliteralemphasis{\sphinxupquote{boolean}}\sphinxstyleemphasis{ or }undefined) \textendash{} The given domain can be:
           * a string representation of the Python prefix-array
             representation of the domain.
           * a JS prefix-array representation of the domain.
           * a boolean where the “true” domain match all records and the
             “false” domain does not match any records.
           * undefined, considered as the false boolean.
           * a number, considered as true except 0 considered as false.

\sphinxstylestrong{evalContext} (\sphinxstyleliteralemphasis{\sphinxupquote{Object}}) \textendash{} in case the given domain is a string, an
                              evaluation context might be needed
\end{itemize}

\end{description}\end{quote}

The Domain Class allows to work with a domain as a tree and provides tools
to manipulate array and string representations of domains.


\begin{fulllineitems}
\phantomsection\label{\detokenize{reference/javascript_api:compute}}\pysiglinewithargsret{\sphinxbfcode{\sphinxupquote{method }}\sphinxbfcode{\sphinxupquote{compute}}}{\emph{values}}{{ $\rightarrow$ boolean}}
Evaluates the domain with a set of values.
\begin{quote}\begin{description}
\item[{Parameters}] \leavevmode\begin{itemize}

\sphinxstylestrong{values} (\sphinxstyleliteralemphasis{\sphinxupquote{Object}}) \textendash{} a mapping \{fieldName -\textgreater{} fieldValue\} (note: all
                       the fields used in the domain should be given a
                       value otherwise the computation will break)
\end{itemize}

\item[{Return Type}] \leavevmode
\sphinxstyleliteralemphasis{\sphinxupquote{boolean}}

\end{description}\end{quote}

\end{fulllineitems}



\begin{fulllineitems}
\phantomsection\label{\detokenize{reference/javascript_api:toArray}}\pysiglinewithargsret{\sphinxbfcode{\sphinxupquote{method }}\sphinxbfcode{\sphinxupquote{toArray}}}{}{{ $\rightarrow$ Array}}
Return the JS prefix-array representation of this domain. Note that all
domains that use the “false” domain cannot be represented as such.
\begin{quote}\begin{description}
\item[{Returns}] \leavevmode
JS prefix-array representation of this domain

\item[{Return Type}] \leavevmode
\sphinxstyleliteralemphasis{\sphinxupquote{Array}}

\end{description}\end{quote}

\end{fulllineitems}



\begin{fulllineitems}
\phantomsection\label{\detokenize{reference/javascript_api:arrayToString}}\pysiglinewithargsret{\sphinxbfcode{\sphinxupquote{method }}\sphinxbfcode{\sphinxupquote{arrayToString}}}{\emph{domain}}{{ $\rightarrow$ string}}
Converts JS prefix-array representation of a domain to a string
representation of the Python prefix-array representation of this domain.
\begin{quote}\begin{description}
\item[{Parameters}] \leavevmode\begin{itemize}

\sphinxstylestrong{domain} (\sphinxstyleliteralemphasis{\sphinxupquote{Array}}\sphinxstyleemphasis{ or }\sphinxstyleliteralemphasis{\sphinxupquote{string}})
\end{itemize}

\item[{Return Type}] \leavevmode
\sphinxstyleliteralemphasis{\sphinxupquote{string}}

\end{description}\end{quote}

\end{fulllineitems}



\begin{fulllineitems}
\phantomsection\label{\detokenize{reference/javascript_api:stringToArray}}\pysiglinewithargsret{\sphinxbfcode{\sphinxupquote{method }}\sphinxbfcode{\sphinxupquote{stringToArray}}}{\emph{domain}\sphinxoptional{, \emph{evalContext}}}{{ $\rightarrow$ Array}}
Converts a string representation of the Python prefix-array
representation of a domain to a JS prefix-array representation of this
domain.
\begin{quote}\begin{description}
\item[{Parameters}] \leavevmode\begin{itemize}

\sphinxstylestrong{domain} (\sphinxstyleliteralemphasis{\sphinxupquote{string}}\sphinxstyleemphasis{ or }\sphinxstyleliteralemphasis{\sphinxupquote{Array}})

\sphinxstylestrong{evalContext} (\sphinxstyleliteralemphasis{\sphinxupquote{Object}})
\end{itemize}

\item[{Return Type}] \leavevmode
\sphinxstyleliteralemphasis{\sphinxupquote{Array}}

\end{description}\end{quote}

\end{fulllineitems}



\begin{fulllineitems}
\phantomsection\label{\detokenize{reference/javascript_api:normalizeArray}}\pysiglinewithargsret{\sphinxbfcode{\sphinxupquote{method }}\sphinxbfcode{\sphinxupquote{normalizeArray}}}{\emph{domain}}{{ $\rightarrow$ Array}}
Makes implicit “\&” operators explicit in the given JS prefix-array
representation of domain (e.g {[}A, B{]} -\textgreater{} {[}“\&”, A, B{]})
\begin{quote}\begin{description}
\item[{Parameters}] \leavevmode\begin{itemize}

\sphinxstylestrong{domain} (\sphinxstyleliteralemphasis{\sphinxupquote{Array}}) \textendash{} the JS prefix-array representation of the domain
                      to normalize (! will be normalized in-place)
\end{itemize}

\item[{Returns}] \leavevmode
the normalized JS prefix-array representation of the
                 given domain

\item[{Return Type}] \leavevmode
\sphinxstyleliteralemphasis{\sphinxupquote{Array}}

\end{description}\end{quote}

\end{fulllineitems}


\end{fulllineitems}


\end{fulllineitems}

\phantomsection\label{\detokenize{reference/javascript_api:module-mail.composer}}

\begin{fulllineitems}
\phantomsection\label{\detokenize{reference/javascript_api:mail.composer}}\pysigline{\sphinxbfcode{\sphinxupquote{module }}\sphinxbfcode{\sphinxupquote{mail.composer}}}~~\begin{quote}\begin{description}
\item[{Exports}] \leavevmode{\hyperref[\detokenize{reference/javascript_api:mail.composer.}]{\sphinxcrossref{
\textless{}anonymous\textgreater{}
}}}
\item[{Depends On}] \leavevmode\begin{itemize}
\item {} {\hyperref[\detokenize{reference/javascript_api:mail.DocumentViewer}]{\sphinxcrossref{
mail.DocumentViewer
}}}
\item {} {\hyperref[\detokenize{reference/javascript_api:mail.chat_mixin}]{\sphinxcrossref{
mail.chat\_mixin
}}}
\item {} {\hyperref[\detokenize{reference/javascript_api:mail.utils}]{\sphinxcrossref{
mail.utils
}}}
\item {} {\hyperref[\detokenize{reference/javascript_api:web.Widget}]{\sphinxcrossref{
web.Widget
}}}
\item {} {\hyperref[\detokenize{reference/javascript_api:web.config}]{\sphinxcrossref{
web.config
}}}
\item {} {\hyperref[\detokenize{reference/javascript_api:web.core}]{\sphinxcrossref{
web.core
}}}
\item {} {\hyperref[\detokenize{reference/javascript_api:web.data}]{\sphinxcrossref{
web.data
}}}
\item {} {\hyperref[\detokenize{reference/javascript_api:web.dom}]{\sphinxcrossref{
web.dom
}}}
\item {} {\hyperref[\detokenize{reference/javascript_api:web.session}]{\sphinxcrossref{
web.session
}}}
\end{itemize}

\end{description}\end{quote}


\begin{fulllineitems}
\phantomsection\label{\detokenize{reference/javascript_api:mail.composer.}}\pysigline{\sphinxbfcode{\sphinxupquote{namespace }}\sphinxbfcode{\sphinxupquote{}}}
\end{fulllineitems}


\end{fulllineitems}

\phantomsection\label{\detokenize{reference/javascript_api:module-web.basic_fields}}

\begin{fulllineitems}
\phantomsection\label{\detokenize{reference/javascript_api:web.basic_fields}}\pysigline{\sphinxbfcode{\sphinxupquote{module }}\sphinxbfcode{\sphinxupquote{web.basic\_fields}}}~~\begin{quote}\begin{description}
\item[{Exports}] \leavevmode{\hyperref[\detokenize{reference/javascript_api:web.basic_fields.}]{\sphinxcrossref{
\textless{}anonymous\textgreater{}
}}}
\item[{Depends On}] \leavevmode\begin{itemize}
\item {} {\hyperref[\detokenize{reference/javascript_api:web.AbstractField}]{\sphinxcrossref{
web.AbstractField
}}}
\item {} {\hyperref[\detokenize{reference/javascript_api:web.Domain}]{\sphinxcrossref{
web.Domain
}}}
\item {} {\hyperref[\detokenize{reference/javascript_api:web.DomainSelector}]{\sphinxcrossref{
web.DomainSelector
}}}
\item {} {\hyperref[\detokenize{reference/javascript_api:web.DomainSelectorDialog}]{\sphinxcrossref{
web.DomainSelectorDialog
}}}
\item {} {\hyperref[\detokenize{reference/javascript_api:web.config}]{\sphinxcrossref{
web.config
}}}
\item {} {\hyperref[\detokenize{reference/javascript_api:web.core}]{\sphinxcrossref{
web.core
}}}
\item {} {\hyperref[\detokenize{reference/javascript_api:web.crash_manager}]{\sphinxcrossref{
web.crash\_manager
}}}
\item {} {\hyperref[\detokenize{reference/javascript_api:web.datepicker}]{\sphinxcrossref{
web.datepicker
}}}
\item {} {\hyperref[\detokenize{reference/javascript_api:web.dom}]{\sphinxcrossref{
web.dom
}}}
\item {} {\hyperref[\detokenize{reference/javascript_api:web.field_utils}]{\sphinxcrossref{
web.field\_utils
}}}
\item {} {\hyperref[\detokenize{reference/javascript_api:web.framework}]{\sphinxcrossref{
web.framework
}}}
\item {} {\hyperref[\detokenize{reference/javascript_api:web.session}]{\sphinxcrossref{
web.session
}}}
\item {} {\hyperref[\detokenize{reference/javascript_api:web.utils}]{\sphinxcrossref{
web.utils
}}}
\item {} {\hyperref[\detokenize{reference/javascript_api:web.view_dialogs}]{\sphinxcrossref{
web.view\_dialogs
}}}
\end{itemize}

\end{description}\end{quote}


\begin{fulllineitems}
\phantomsection\label{\detokenize{reference/javascript_api:FieldToggleBoolean}}\pysiglinewithargsret{\sphinxbfcode{\sphinxupquote{class }}\sphinxbfcode{\sphinxupquote{FieldToggleBoolean}}}{}{}~\begin{quote}\begin{description}
\item[{Extends}] \leavevmode{\hyperref[\detokenize{reference/javascript_api:web.AbstractField.AbstractField}]{\sphinxcrossref{
AbstractField
}}}
\end{description}\end{quote}

This widget is intended to be used on boolean fields. It toggles a button
switching between a green bullet / gray bullet.


\begin{fulllineitems}
\phantomsection\label{\detokenize{reference/javascript_api:isSet}}\pysiglinewithargsret{\sphinxbfcode{\sphinxupquote{method }}\sphinxbfcode{\sphinxupquote{isSet}}}{}{}
A boolean field is always set since false is a valid value.

\end{fulllineitems}


\end{fulllineitems}



\begin{fulllineitems}
\phantomsection\label{\detokenize{reference/javascript_api:FieldProgressBar}}\pysiglinewithargsret{\sphinxbfcode{\sphinxupquote{class }}\sphinxbfcode{\sphinxupquote{FieldProgressBar}}}{}{}~\begin{quote}\begin{description}
\item[{Extends}] \leavevmode{\hyperref[\detokenize{reference/javascript_api:web.AbstractField.AbstractField}]{\sphinxcrossref{
AbstractField
}}}
\end{description}\end{quote}

Node options:
\begin{itemize}
\item {} 
title: title of the bar, displayed on top of the bar options

\item {} 
editable: boolean if value is editable

\item {} 
current\_value: get the current\_value from the field that must be present in the view

\item {} 
max\_value: get the max\_value from the field that must be present in the view

\item {} 
edit\_max\_value: boolean if the max\_value is editable

\item {} 
title: title of the bar, displayed on top of the bar \textendash{}\textgreater{} not translated,  use parameter “title” instead

\end{itemize}

\end{fulllineitems}



\begin{fulllineitems}
\phantomsection\label{\detokenize{reference/javascript_api:HandleWidget}}\pysiglinewithargsret{\sphinxbfcode{\sphinxupquote{class }}\sphinxbfcode{\sphinxupquote{HandleWidget}}}{}{}~\begin{quote}\begin{description}
\item[{Extends}] \leavevmode{\hyperref[\detokenize{reference/javascript_api:web.AbstractField.AbstractField}]{\sphinxcrossref{
AbstractField
}}}
\end{description}\end{quote}

Displays a handle to modify the sequence.

\end{fulllineitems}



\begin{fulllineitems}
\phantomsection\label{\detokenize{reference/javascript_api:AceEditor}}\pysiglinewithargsret{\sphinxbfcode{\sphinxupquote{class }}\sphinxbfcode{\sphinxupquote{AceEditor}}}{}{}~\begin{quote}\begin{description}
\item[{Extends}] \leavevmode
DebouncedField

\end{description}\end{quote}

This widget is intended to be used on Text fields. It will provide Ace Editor
for editing XML and Python.

\end{fulllineitems}



\begin{fulllineitems}
\phantomsection\label{\detokenize{reference/javascript_api:FieldDomain}}\pysiglinewithargsret{\sphinxbfcode{\sphinxupquote{class }}\sphinxbfcode{\sphinxupquote{FieldDomain}}}{}{}~\begin{quote}\begin{description}
\item[{Extends}] \leavevmode{\hyperref[\detokenize{reference/javascript_api:web.AbstractField.AbstractField}]{\sphinxcrossref{
AbstractField
}}}
\end{description}\end{quote}

The “Domain” field allows the user to construct a technical-prefix domain
thanks to a tree-like interface and see the selected records in real time.
In debug mode, an input is also there to be able to enter the prefix char
domain directly (or to build advanced domains the tree-like interface does
not allow to).


\begin{fulllineitems}
\phantomsection\label{\detokenize{reference/javascript_api:specialData}}\pysigline{\sphinxbfcode{\sphinxupquote{attribute }}\sphinxbfcode{\sphinxupquote{specialData}} String}
Fetches the number of records which are matched by the domain (if the
domain is not server-valid, the value is false) and the model the
field must work with.

\end{fulllineitems}



\begin{fulllineitems}
\phantomsection\label{\detokenize{reference/javascript_api:isSet}}\pysiglinewithargsret{\sphinxbfcode{\sphinxupquote{method }}\sphinxbfcode{\sphinxupquote{isSet}}}{}{}
A domain field is always set since the false value is considered to be
equal to “{[}{]}” (match all records).

\end{fulllineitems}


\end{fulllineitems}



\begin{fulllineitems}
\phantomsection\label{\detokenize{reference/javascript_api:web.basic_fields.}}\pysigline{\sphinxbfcode{\sphinxupquote{namespace }}\sphinxbfcode{\sphinxupquote{}}}~

\begin{fulllineitems}
\phantomsection\label{\detokenize{reference/javascript_api:FieldDomain}}\pysiglinewithargsret{\sphinxbfcode{\sphinxupquote{class }}\sphinxbfcode{\sphinxupquote{FieldDomain}}}{}{}~\begin{quote}\begin{description}
\item[{Extends}] \leavevmode{\hyperref[\detokenize{reference/javascript_api:web.AbstractField.AbstractField}]{\sphinxcrossref{
AbstractField
}}}
\end{description}\end{quote}

The “Domain” field allows the user to construct a technical-prefix domain
thanks to a tree-like interface and see the selected records in real time.
In debug mode, an input is also there to be able to enter the prefix char
domain directly (or to build advanced domains the tree-like interface does
not allow to).


\begin{fulllineitems}
\phantomsection\label{\detokenize{reference/javascript_api:specialData}}\pysigline{\sphinxbfcode{\sphinxupquote{attribute }}\sphinxbfcode{\sphinxupquote{specialData}} String}
Fetches the number of records which are matched by the domain (if the
domain is not server-valid, the value is false) and the model the
field must work with.

\end{fulllineitems}



\begin{fulllineitems}
\phantomsection\label{\detokenize{reference/javascript_api:isSet}}\pysiglinewithargsret{\sphinxbfcode{\sphinxupquote{method }}\sphinxbfcode{\sphinxupquote{isSet}}}{}{}
A domain field is always set since the false value is considered to be
equal to “{[}{]}” (match all records).

\end{fulllineitems}


\end{fulllineitems}



\begin{fulllineitems}
\phantomsection\label{\detokenize{reference/javascript_api:FieldProgressBar}}\pysiglinewithargsret{\sphinxbfcode{\sphinxupquote{class }}\sphinxbfcode{\sphinxupquote{FieldProgressBar}}}{}{}~\begin{quote}\begin{description}
\item[{Extends}] \leavevmode{\hyperref[\detokenize{reference/javascript_api:web.AbstractField.AbstractField}]{\sphinxcrossref{
AbstractField
}}}
\end{description}\end{quote}

Node options:
\begin{itemize}
\item {} 
title: title of the bar, displayed on top of the bar options

\item {} 
editable: boolean if value is editable

\item {} 
current\_value: get the current\_value from the field that must be present in the view

\item {} 
max\_value: get the max\_value from the field that must be present in the view

\item {} 
edit\_max\_value: boolean if the max\_value is editable

\item {} 
title: title of the bar, displayed on top of the bar \textendash{}\textgreater{} not translated,  use parameter “title” instead

\end{itemize}

\end{fulllineitems}



\begin{fulllineitems}
\phantomsection\label{\detokenize{reference/javascript_api:FieldToggleBoolean}}\pysiglinewithargsret{\sphinxbfcode{\sphinxupquote{class }}\sphinxbfcode{\sphinxupquote{FieldToggleBoolean}}}{}{}~\begin{quote}\begin{description}
\item[{Extends}] \leavevmode{\hyperref[\detokenize{reference/javascript_api:web.AbstractField.AbstractField}]{\sphinxcrossref{
AbstractField
}}}
\end{description}\end{quote}

This widget is intended to be used on boolean fields. It toggles a button
switching between a green bullet / gray bullet.


\begin{fulllineitems}
\phantomsection\label{\detokenize{reference/javascript_api:isSet}}\pysiglinewithargsret{\sphinxbfcode{\sphinxupquote{method }}\sphinxbfcode{\sphinxupquote{isSet}}}{}{}
A boolean field is always set since false is a valid value.

\end{fulllineitems}


\end{fulllineitems}



\begin{fulllineitems}
\phantomsection\label{\detokenize{reference/javascript_api:HandleWidget}}\pysiglinewithargsret{\sphinxbfcode{\sphinxupquote{class }}\sphinxbfcode{\sphinxupquote{HandleWidget}}}{}{}~\begin{quote}\begin{description}
\item[{Extends}] \leavevmode{\hyperref[\detokenize{reference/javascript_api:web.AbstractField.AbstractField}]{\sphinxcrossref{
AbstractField
}}}
\end{description}\end{quote}

Displays a handle to modify the sequence.

\end{fulllineitems}



\begin{fulllineitems}
\phantomsection\label{\detokenize{reference/javascript_api:AceEditor}}\pysiglinewithargsret{\sphinxbfcode{\sphinxupquote{class }}\sphinxbfcode{\sphinxupquote{AceEditor}}}{}{}~\begin{quote}\begin{description}
\item[{Extends}] \leavevmode
DebouncedField

\end{description}\end{quote}

This widget is intended to be used on Text fields. It will provide Ace Editor
for editing XML and Python.

\end{fulllineitems}


\end{fulllineitems}


\end{fulllineitems}

\phantomsection\label{\detokenize{reference/javascript_api:module-portal.chatter}}

\begin{fulllineitems}
\phantomsection\label{\detokenize{reference/javascript_api:portal.chatter}}\pysigline{\sphinxbfcode{\sphinxupquote{module }}\sphinxbfcode{\sphinxupquote{portal.chatter}}}~~\begin{quote}\begin{description}
\item[{Exports}] \leavevmode{\hyperref[\detokenize{reference/javascript_api:portal.chatter.}]{\sphinxcrossref{
\textless{}anonymous\textgreater{}
}}}
\item[{Depends On}] \leavevmode\begin{itemize}
\item {} {\hyperref[\detokenize{reference/javascript_api:web.Widget}]{\sphinxcrossref{
web.Widget
}}}
\item {} {\hyperref[\detokenize{reference/javascript_api:web.ajax}]{\sphinxcrossref{
web.ajax
}}}
\item {} {\hyperref[\detokenize{reference/javascript_api:web.core}]{\sphinxcrossref{
web.core
}}}
\item {} {\hyperref[\detokenize{reference/javascript_api:web.rpc}]{\sphinxcrossref{
web.rpc
}}}
\item {} {\hyperref[\detokenize{reference/javascript_api:web.time}]{\sphinxcrossref{
web.time
}}}
\item {} {\hyperref[\detokenize{reference/javascript_api:web_editor.base}]{\sphinxcrossref{
web\_editor.base
}}}
\end{itemize}

\end{description}\end{quote}


\begin{fulllineitems}
\phantomsection\label{\detokenize{reference/javascript_api:PortalChatter}}\pysiglinewithargsret{\sphinxbfcode{\sphinxupquote{class }}\sphinxbfcode{\sphinxupquote{PortalChatter}}}{\emph{parent}, \emph{options}}{}~\begin{quote}\begin{description}
\item[{Extends}] \leavevmode{\hyperref[\detokenize{reference/javascript_api:web.Widget.Widget}]{\sphinxcrossref{
Widget
}}}
\item[{Parameters}] \leavevmode\begin{itemize}

\sphinxstylestrong{parent}

\sphinxstylestrong{options}
\end{itemize}

\end{description}\end{quote}

Widget PortalChatter
\begin{itemize}
\item {} 
Fetch message fron controller

\item {} 
Display chatter: pager, total message, composer (according to access right)

\item {} 
Provider API to filter displayed messages

\end{itemize}


\begin{fulllineitems}
\phantomsection\label{\detokenize{reference/javascript_api:messageFetch}}\pysiglinewithargsret{\sphinxbfcode{\sphinxupquote{method }}\sphinxbfcode{\sphinxupquote{messageFetch}}}{\emph{domain}}{{ $\rightarrow$ Deferred}}
Fetch the messages and the message count from the server for the
current page and current domain.
\begin{quote}\begin{description}
\item[{Parameters}] \leavevmode\begin{itemize}

\sphinxstylestrong{domain} (\sphinxstyleliteralemphasis{\sphinxupquote{Array}})
\end{itemize}

\item[{Return Type}] \leavevmode
\sphinxstyleliteralemphasis{\sphinxupquote{Deferred}}

\end{description}\end{quote}

\end{fulllineitems}



\begin{fulllineitems}
\phantomsection\label{\detokenize{reference/javascript_api:preprocessMessages}}\pysiglinewithargsret{\sphinxbfcode{\sphinxupquote{method }}\sphinxbfcode{\sphinxupquote{preprocessMessages}}}{\emph{messages}}{{ $\rightarrow$ Array}}
Update the messages format
\begin{quote}\begin{description}
\item[{Parameters}] \leavevmode\begin{itemize}

\sphinxstylestrong{messages} (\sphinxstyleliteralemphasis{\sphinxupquote{Array}}\textless{}\sphinxstyleliteralemphasis{\sphinxupquote{Object}}\textgreater{})
\end{itemize}

\item[{Return Type}] \leavevmode
\sphinxstyleliteralemphasis{\sphinxupquote{Array}}

\end{description}\end{quote}

\end{fulllineitems}



\begin{fulllineitems}
\phantomsection\label{\detokenize{reference/javascript_api:round_to_half}}\pysiglinewithargsret{\sphinxbfcode{\sphinxupquote{method }}\sphinxbfcode{\sphinxupquote{round\_to\_half}}}{\emph{value}}{}
Round the given value with a precision of 0.5.

Examples:
- 1.2 \textendash{}\textgreater{} 1.0
- 1.7 \textendash{}\textgreater{} 1.5
- 1.9 \textendash{}\textgreater{} 2.0
\begin{quote}\begin{description}
\item[{Parameters}] \leavevmode\begin{itemize}

\sphinxstylestrong{value} (\sphinxstyleliteralemphasis{\sphinxupquote{Number}})
\end{itemize}

\end{description}\end{quote}

\end{fulllineitems}


\end{fulllineitems}



\begin{fulllineitems}
\phantomsection\label{\detokenize{reference/javascript_api:portal.chatter.}}\pysigline{\sphinxbfcode{\sphinxupquote{namespace }}\sphinxbfcode{\sphinxupquote{}}}~

\begin{fulllineitems}
\phantomsection\label{\detokenize{reference/javascript_api:PortalChatter}}\pysiglinewithargsret{\sphinxbfcode{\sphinxupquote{class }}\sphinxbfcode{\sphinxupquote{PortalChatter}}}{\emph{parent}, \emph{options}}{}~\begin{quote}\begin{description}
\item[{Extends}] \leavevmode{\hyperref[\detokenize{reference/javascript_api:web.Widget.Widget}]{\sphinxcrossref{
Widget
}}}
\item[{Parameters}] \leavevmode\begin{itemize}

\sphinxstylestrong{parent}

\sphinxstylestrong{options}
\end{itemize}

\end{description}\end{quote}

Widget PortalChatter
\begin{itemize}
\item {} 
Fetch message fron controller

\item {} 
Display chatter: pager, total message, composer (according to access right)

\item {} 
Provider API to filter displayed messages

\end{itemize}


\begin{fulllineitems}
\phantomsection\label{\detokenize{reference/javascript_api:messageFetch}}\pysiglinewithargsret{\sphinxbfcode{\sphinxupquote{method }}\sphinxbfcode{\sphinxupquote{messageFetch}}}{\emph{domain}}{{ $\rightarrow$ Deferred}}
Fetch the messages and the message count from the server for the
current page and current domain.
\begin{quote}\begin{description}
\item[{Parameters}] \leavevmode\begin{itemize}

\sphinxstylestrong{domain} (\sphinxstyleliteralemphasis{\sphinxupquote{Array}})
\end{itemize}

\item[{Return Type}] \leavevmode
\sphinxstyleliteralemphasis{\sphinxupquote{Deferred}}

\end{description}\end{quote}

\end{fulllineitems}



\begin{fulllineitems}
\phantomsection\label{\detokenize{reference/javascript_api:preprocessMessages}}\pysiglinewithargsret{\sphinxbfcode{\sphinxupquote{method }}\sphinxbfcode{\sphinxupquote{preprocessMessages}}}{\emph{messages}}{{ $\rightarrow$ Array}}
Update the messages format
\begin{quote}\begin{description}
\item[{Parameters}] \leavevmode\begin{itemize}

\sphinxstylestrong{messages} (\sphinxstyleliteralemphasis{\sphinxupquote{Array}}\textless{}\sphinxstyleliteralemphasis{\sphinxupquote{Object}}\textgreater{})
\end{itemize}

\item[{Return Type}] \leavevmode
\sphinxstyleliteralemphasis{\sphinxupquote{Array}}

\end{description}\end{quote}

\end{fulllineitems}



\begin{fulllineitems}
\phantomsection\label{\detokenize{reference/javascript_api:round_to_half}}\pysiglinewithargsret{\sphinxbfcode{\sphinxupquote{method }}\sphinxbfcode{\sphinxupquote{round\_to\_half}}}{\emph{value}}{}
Round the given value with a precision of 0.5.

Examples:
- 1.2 \textendash{}\textgreater{} 1.0
- 1.7 \textendash{}\textgreater{} 1.5
- 1.9 \textendash{}\textgreater{} 2.0
\begin{quote}\begin{description}
\item[{Parameters}] \leavevmode\begin{itemize}

\sphinxstylestrong{value} (\sphinxstyleliteralemphasis{\sphinxupquote{Number}})
\end{itemize}

\end{description}\end{quote}

\end{fulllineitems}


\end{fulllineitems}


\end{fulllineitems}


\end{fulllineitems}

\phantomsection\label{\detokenize{reference/javascript_api:module-report.utils}}

\begin{fulllineitems}
\phantomsection\label{\detokenize{reference/javascript_api:report.utils}}\pysigline{\sphinxbfcode{\sphinxupquote{module }}\sphinxbfcode{\sphinxupquote{report.utils}}}~~\begin{quote}\begin{description}
\item[{Exports}] \leavevmode{\hyperref[\detokenize{reference/javascript_api:report.utils.}]{\sphinxcrossref{
\textless{}anonymous\textgreater{}
}}}
\end{description}\end{quote}


\begin{fulllineitems}
\phantomsection\label{\detokenize{reference/javascript_api:report.utils.}}\pysigline{\sphinxbfcode{\sphinxupquote{namespace }}\sphinxbfcode{\sphinxupquote{}}}
\end{fulllineitems}


\end{fulllineitems}

\phantomsection\label{\detokenize{reference/javascript_api:module-web.view_dialogs}}

\begin{fulllineitems}
\phantomsection\label{\detokenize{reference/javascript_api:web.view_dialogs}}\pysigline{\sphinxbfcode{\sphinxupquote{module }}\sphinxbfcode{\sphinxupquote{web.view\_dialogs}}}~~\begin{quote}\begin{description}
\item[{Exports}] \leavevmode{\hyperref[\detokenize{reference/javascript_api:web.view_dialogs.}]{\sphinxcrossref{
\textless{}anonymous\textgreater{}
}}}
\item[{Depends On}] \leavevmode\begin{itemize}
\item {} {\hyperref[\detokenize{reference/javascript_api:web.Dialog}]{\sphinxcrossref{
web.Dialog
}}}
\item {} {\hyperref[\detokenize{reference/javascript_api:web.ListController}]{\sphinxcrossref{
web.ListController
}}}
\item {} {\hyperref[\detokenize{reference/javascript_api:web.ListView}]{\sphinxcrossref{
web.ListView
}}}
\item {} {\hyperref[\detokenize{reference/javascript_api:web.SearchView}]{\sphinxcrossref{
web.SearchView
}}}
\item {} {\hyperref[\detokenize{reference/javascript_api:web.core}]{\sphinxcrossref{
web.core
}}}
\item {} {\hyperref[\detokenize{reference/javascript_api:web.data}]{\sphinxcrossref{
web.data
}}}
\item {} {\hyperref[\detokenize{reference/javascript_api:web.dom}]{\sphinxcrossref{
web.dom
}}}
\item {} {\hyperref[\detokenize{reference/javascript_api:web.pyeval}]{\sphinxcrossref{
web.pyeval
}}}
\item {} {\hyperref[\detokenize{reference/javascript_api:web.view_registry}]{\sphinxcrossref{
web.view\_registry
}}}
\end{itemize}

\end{description}\end{quote}


\begin{fulllineitems}
\phantomsection\label{\detokenize{reference/javascript_api:SelectCreateDialog}}\pysiglinewithargsret{\sphinxbfcode{\sphinxupquote{class }}\sphinxbfcode{\sphinxupquote{SelectCreateDialog}}}{}{}~\begin{quote}\begin{description}
\item[{Extends}] \leavevmode{\hyperref[\detokenize{reference/javascript_api:web.view_dialogs.ViewDialog}]{\sphinxcrossref{
ViewDialog
}}}
\end{description}\end{quote}

Search dialog (displays a list of records and permits to create a new one by switching to a form view)

\end{fulllineitems}



\begin{fulllineitems}
\phantomsection\label{\detokenize{reference/javascript_api:web.view_dialogs.}}\pysigline{\sphinxbfcode{\sphinxupquote{namespace }}\sphinxbfcode{\sphinxupquote{}}}~

\begin{fulllineitems}
\phantomsection\label{\detokenize{reference/javascript_api:FormViewDialog}}\pysiglinewithargsret{\sphinxbfcode{\sphinxupquote{class }}\sphinxbfcode{\sphinxupquote{FormViewDialog}}}{\emph{parent}\sphinxoptional{, \emph{options}}}{}~\begin{quote}\begin{description}
\item[{Extends}] \leavevmode{\hyperref[\detokenize{reference/javascript_api:web.view_dialogs.ViewDialog}]{\sphinxcrossref{
ViewDialog
}}}
\item[{Parameters}] \leavevmode\begin{itemize}

\sphinxstylestrong{parent} ({\hyperref[\detokenize{reference/javascript_api:Widget}]{\sphinxcrossref{\sphinxstyleliteralemphasis{\sphinxupquote{Widget}}}}})

\sphinxstylestrong{options} ({\hyperref[\detokenize{reference/javascript_api:web.view_dialogs.FormViewDialogOptions}]{\sphinxcrossref{\sphinxstyleliteralemphasis{\sphinxupquote{FormViewDialogOptions}}}}})
\end{itemize}

\end{description}\end{quote}

Create and edit dialog (displays a form view record and leave once saved)


\begin{fulllineitems}
\phantomsection\label{\detokenize{reference/javascript_api:open}}\pysiglinewithargsret{\sphinxbfcode{\sphinxupquote{method }}\sphinxbfcode{\sphinxupquote{open}}}{}{{ $\rightarrow$ FormViewDialog}}
Open the form view dialog.  It is necessarily asynchronous, but this
method returns immediately.
\begin{quote}\begin{description}
\item[{Returns}] \leavevmode
this instance

\item[{Return Type}] \leavevmode
{\hyperref[\detokenize{reference/javascript_api:web.view_dialogs.FormViewDialog}]{\sphinxcrossref{\sphinxstyleliteralemphasis{\sphinxupquote{FormViewDialog}}}}}

\end{description}\end{quote}

\end{fulllineitems}



\begin{fulllineitems}
\phantomsection\label{\detokenize{reference/javascript_api:FormViewDialogOptions}}\pysiglinewithargsret{\sphinxbfcode{\sphinxupquote{class }}\sphinxbfcode{\sphinxupquote{FormViewDialogOptions}}}{}{}~

\begin{fulllineitems}
\phantomsection\label{\detokenize{reference/javascript_api:parentID}}\pysigline{\sphinxbfcode{\sphinxupquote{attribute }}\sphinxbfcode{\sphinxupquote{parentID}} string}~\begin{description}
\item[{the id of the parent record. It is}] \leavevmode
useful for situations such as a one2many opened in a form view dialog.
In that case, we want to be able to properly evaluate domains with the
‘parent’ key.

\end{description}

\end{fulllineitems}



\begin{fulllineitems}
\phantomsection\label{\detokenize{reference/javascript_api:res_id}}\pysigline{\sphinxbfcode{\sphinxupquote{attribute }}\sphinxbfcode{\sphinxupquote{res\_id}} integer}
the id of the record to open

\end{fulllineitems}



\begin{fulllineitems}
\phantomsection\label{\detokenize{reference/javascript_api:form_view_options}}\pysigline{\sphinxbfcode{\sphinxupquote{attribute }}\sphinxbfcode{\sphinxupquote{form\_view\_options}} Object}~\begin{description}
\item[{dict of options to pass to}] \leavevmode
the Form View @todo: make it work

\end{description}

\end{fulllineitems}



\begin{fulllineitems}
\phantomsection\label{\detokenize{reference/javascript_api:fields_view}}\pysigline{\sphinxbfcode{\sphinxupquote{attribute }}\sphinxbfcode{\sphinxupquote{fields\_view}} Object}
optional form fields\_view

\end{fulllineitems}



\begin{fulllineitems}
\phantomsection\label{\detokenize{reference/javascript_api:readonly}}\pysigline{\sphinxbfcode{\sphinxupquote{attribute }}\sphinxbfcode{\sphinxupquote{readonly}} boolean}~\begin{description}
\item[{only applicable when not in}] \leavevmode
creation mode

\end{description}

\end{fulllineitems}



\begin{fulllineitems}
\phantomsection\label{\detokenize{reference/javascript_api:on_saved}}\pysigline{\sphinxbfcode{\sphinxupquote{attribute }}\sphinxbfcode{\sphinxupquote{on\_saved}} function}~\begin{description}
\item[{callback executed after saving a}] \leavevmode
record.  It will be called with the record data, and a boolean which
indicates if something was changed

\end{description}

\end{fulllineitems}



\begin{fulllineitems}
\phantomsection\label{\detokenize{reference/javascript_api:model}}\pysigline{\sphinxbfcode{\sphinxupquote{attribute }}\sphinxbfcode{\sphinxupquote{model}} {\hyperref[\detokenize{reference/javascript_api:BasicModel}]{\sphinxcrossref{BasicModel}}}}~\begin{description}
\item[{if given, it will be used instead of}] \leavevmode
a new form view model

\end{description}

\end{fulllineitems}



\begin{fulllineitems}
\phantomsection\label{\detokenize{reference/javascript_api:recordID}}\pysigline{\sphinxbfcode{\sphinxupquote{attribute }}\sphinxbfcode{\sphinxupquote{recordID}} string}~\begin{description}
\item[{if given, the model has to be given as}] \leavevmode
well, and in that case, it will be used without loading anything.

\end{description}

\end{fulllineitems}



\begin{fulllineitems}
\phantomsection\label{\detokenize{reference/javascript_api:shouldSaveLocally}}\pysigline{\sphinxbfcode{\sphinxupquote{attribute }}\sphinxbfcode{\sphinxupquote{shouldSaveLocally}} boolean}~\begin{description}
\item[{if true, the view dialog}] \leavevmode
will save locally instead of actually saving (useful for one2manys)

\end{description}

\end{fulllineitems}


\end{fulllineitems}


\end{fulllineitems}



\begin{fulllineitems}
\phantomsection\label{\detokenize{reference/javascript_api:SelectCreateDialog}}\pysiglinewithargsret{\sphinxbfcode{\sphinxupquote{class }}\sphinxbfcode{\sphinxupquote{SelectCreateDialog}}}{}{}~\begin{quote}\begin{description}
\item[{Extends}] \leavevmode{\hyperref[\detokenize{reference/javascript_api:web.view_dialogs.ViewDialog}]{\sphinxcrossref{
ViewDialog
}}}
\end{description}\end{quote}

Search dialog (displays a list of records and permits to create a new one by switching to a form view)

\end{fulllineitems}


\end{fulllineitems}



\begin{fulllineitems}
\phantomsection\label{\detokenize{reference/javascript_api:FormViewDialog}}\pysiglinewithargsret{\sphinxbfcode{\sphinxupquote{class }}\sphinxbfcode{\sphinxupquote{FormViewDialog}}}{\emph{parent}\sphinxoptional{, \emph{options}}}{}~\begin{quote}\begin{description}
\item[{Extends}] \leavevmode{\hyperref[\detokenize{reference/javascript_api:web.view_dialogs.ViewDialog}]{\sphinxcrossref{
ViewDialog
}}}
\item[{Parameters}] \leavevmode\begin{itemize}

\sphinxstylestrong{parent} ({\hyperref[\detokenize{reference/javascript_api:Widget}]{\sphinxcrossref{\sphinxstyleliteralemphasis{\sphinxupquote{Widget}}}}})

\sphinxstylestrong{options} ({\hyperref[\detokenize{reference/javascript_api:web.view_dialogs.FormViewDialogOptions}]{\sphinxcrossref{\sphinxstyleliteralemphasis{\sphinxupquote{FormViewDialogOptions}}}}})
\end{itemize}

\end{description}\end{quote}

Create and edit dialog (displays a form view record and leave once saved)


\begin{fulllineitems}
\phantomsection\label{\detokenize{reference/javascript_api:open}}\pysiglinewithargsret{\sphinxbfcode{\sphinxupquote{method }}\sphinxbfcode{\sphinxupquote{open}}}{}{{ $\rightarrow$ FormViewDialog}}
Open the form view dialog.  It is necessarily asynchronous, but this
method returns immediately.
\begin{quote}\begin{description}
\item[{Returns}] \leavevmode
this instance

\item[{Return Type}] \leavevmode
{\hyperref[\detokenize{reference/javascript_api:web.view_dialogs.FormViewDialog}]{\sphinxcrossref{\sphinxstyleliteralemphasis{\sphinxupquote{FormViewDialog}}}}}

\end{description}\end{quote}

\end{fulllineitems}



\begin{fulllineitems}
\phantomsection\label{\detokenize{reference/javascript_api:FormViewDialogOptions}}\pysiglinewithargsret{\sphinxbfcode{\sphinxupquote{class }}\sphinxbfcode{\sphinxupquote{FormViewDialogOptions}}}{}{}~

\begin{fulllineitems}
\phantomsection\label{\detokenize{reference/javascript_api:parentID}}\pysigline{\sphinxbfcode{\sphinxupquote{attribute }}\sphinxbfcode{\sphinxupquote{parentID}} string}~\begin{description}
\item[{the id of the parent record. It is}] \leavevmode
useful for situations such as a one2many opened in a form view dialog.
In that case, we want to be able to properly evaluate domains with the
‘parent’ key.

\end{description}

\end{fulllineitems}



\begin{fulllineitems}
\phantomsection\label{\detokenize{reference/javascript_api:res_id}}\pysigline{\sphinxbfcode{\sphinxupquote{attribute }}\sphinxbfcode{\sphinxupquote{res\_id}} integer}
the id of the record to open

\end{fulllineitems}



\begin{fulllineitems}
\phantomsection\label{\detokenize{reference/javascript_api:form_view_options}}\pysigline{\sphinxbfcode{\sphinxupquote{attribute }}\sphinxbfcode{\sphinxupquote{form\_view\_options}} Object}~\begin{description}
\item[{dict of options to pass to}] \leavevmode
the Form View @todo: make it work

\end{description}

\end{fulllineitems}



\begin{fulllineitems}
\phantomsection\label{\detokenize{reference/javascript_api:fields_view}}\pysigline{\sphinxbfcode{\sphinxupquote{attribute }}\sphinxbfcode{\sphinxupquote{fields\_view}} Object}
optional form fields\_view

\end{fulllineitems}



\begin{fulllineitems}
\phantomsection\label{\detokenize{reference/javascript_api:readonly}}\pysigline{\sphinxbfcode{\sphinxupquote{attribute }}\sphinxbfcode{\sphinxupquote{readonly}} boolean}~\begin{description}
\item[{only applicable when not in}] \leavevmode
creation mode

\end{description}

\end{fulllineitems}



\begin{fulllineitems}
\phantomsection\label{\detokenize{reference/javascript_api:on_saved}}\pysigline{\sphinxbfcode{\sphinxupquote{attribute }}\sphinxbfcode{\sphinxupquote{on\_saved}} function}~\begin{description}
\item[{callback executed after saving a}] \leavevmode
record.  It will be called with the record data, and a boolean which
indicates if something was changed

\end{description}

\end{fulllineitems}



\begin{fulllineitems}
\phantomsection\label{\detokenize{reference/javascript_api:model}}\pysigline{\sphinxbfcode{\sphinxupquote{attribute }}\sphinxbfcode{\sphinxupquote{model}} {\hyperref[\detokenize{reference/javascript_api:BasicModel}]{\sphinxcrossref{BasicModel}}}}~\begin{description}
\item[{if given, it will be used instead of}] \leavevmode
a new form view model

\end{description}

\end{fulllineitems}



\begin{fulllineitems}
\phantomsection\label{\detokenize{reference/javascript_api:recordID}}\pysigline{\sphinxbfcode{\sphinxupquote{attribute }}\sphinxbfcode{\sphinxupquote{recordID}} string}~\begin{description}
\item[{if given, the model has to be given as}] \leavevmode
well, and in that case, it will be used without loading anything.

\end{description}

\end{fulllineitems}



\begin{fulllineitems}
\phantomsection\label{\detokenize{reference/javascript_api:shouldSaveLocally}}\pysigline{\sphinxbfcode{\sphinxupquote{attribute }}\sphinxbfcode{\sphinxupquote{shouldSaveLocally}} boolean}~\begin{description}
\item[{if true, the view dialog}] \leavevmode
will save locally instead of actually saving (useful for one2manys)

\end{description}

\end{fulllineitems}


\end{fulllineitems}


\end{fulllineitems}



\begin{fulllineitems}
\phantomsection\label{\detokenize{reference/javascript_api:ViewDialog}}\pysiglinewithargsret{\sphinxbfcode{\sphinxupquote{class }}\sphinxbfcode{\sphinxupquote{ViewDialog}}}{\emph{parent}\sphinxoptional{, \emph{options}}}{}~\begin{quote}\begin{description}
\item[{Extends}] \leavevmode{\hyperref[\detokenize{reference/javascript_api:web.Dialog.Dialog}]{\sphinxcrossref{
Dialog
}}}
\item[{Parameters}] \leavevmode\begin{itemize}

\sphinxstylestrong{parent} ({\hyperref[\detokenize{reference/javascript_api:Widget}]{\sphinxcrossref{\sphinxstyleliteralemphasis{\sphinxupquote{Widget}}}}})

\sphinxstylestrong{options} ({\hyperref[\detokenize{reference/javascript_api:web.view_dialogs.ViewDialogOptions}]{\sphinxcrossref{\sphinxstyleliteralemphasis{\sphinxupquote{ViewDialogOptions}}}}})
\end{itemize}

\end{description}\end{quote}

Class with everything which is common between FormViewDialog and
SelectCreateDialog.


\begin{fulllineitems}
\phantomsection\label{\detokenize{reference/javascript_api:ViewDialogOptions}}\pysiglinewithargsret{\sphinxbfcode{\sphinxupquote{class }}\sphinxbfcode{\sphinxupquote{ViewDialogOptions}}}{}{}~

\begin{fulllineitems}
\phantomsection\label{\detokenize{reference/javascript_api:dialogClass}}\pysigline{\sphinxbfcode{\sphinxupquote{attribute }}\sphinxbfcode{\sphinxupquote{dialogClass}} string}
\end{fulllineitems}



\begin{fulllineitems}
\phantomsection\label{\detokenize{reference/javascript_api:res_model}}\pysigline{\sphinxbfcode{\sphinxupquote{attribute }}\sphinxbfcode{\sphinxupquote{res\_model}} string}
the model of the record(s) to open

\end{fulllineitems}



\begin{fulllineitems}
\phantomsection\label{\detokenize{reference/javascript_api:domain}}\pysigline{\sphinxbfcode{\sphinxupquote{attribute }}\sphinxbfcode{\sphinxupquote{domain}} any{[}{]}}
\end{fulllineitems}



\begin{fulllineitems}
\phantomsection\label{\detokenize{reference/javascript_api:context}}\pysigline{\sphinxbfcode{\sphinxupquote{attribute }}\sphinxbfcode{\sphinxupquote{context}} Object}
\end{fulllineitems}


\end{fulllineitems}


\end{fulllineitems}


\end{fulllineitems}

\phantomsection\label{\detokenize{reference/javascript_api:module-hr_attendance.kiosk_mode}}

\begin{fulllineitems}
\phantomsection\label{\detokenize{reference/javascript_api:hr_attendance.kiosk_mode}}\pysigline{\sphinxbfcode{\sphinxupquote{module }}\sphinxbfcode{\sphinxupquote{hr\_attendance.kiosk\_mode}}}~~\begin{quote}\begin{description}
\item[{Exports}] \leavevmode{\hyperref[\detokenize{reference/javascript_api:hr_attendance.kiosk_mode.KioskMode}]{\sphinxcrossref{
KioskMode
}}}
\item[{Depends On}] \leavevmode\begin{itemize}
\item {} {\hyperref[\detokenize{reference/javascript_api:web.Widget}]{\sphinxcrossref{
web.Widget
}}}
\item {} {\hyperref[\detokenize{reference/javascript_api:web.core}]{\sphinxcrossref{
web.core
}}}
\item {} {\hyperref[\detokenize{reference/javascript_api:web.session}]{\sphinxcrossref{
web.session
}}}
\end{itemize}

\end{description}\end{quote}


\begin{fulllineitems}
\phantomsection\label{\detokenize{reference/javascript_api:KioskMode}}\pysiglinewithargsret{\sphinxbfcode{\sphinxupquote{class }}\sphinxbfcode{\sphinxupquote{KioskMode}}}{}{}~\begin{quote}\begin{description}
\item[{Extends}] \leavevmode{\hyperref[\detokenize{reference/javascript_api:web.Widget.Widget}]{\sphinxcrossref{
Widget
}}}
\end{description}\end{quote}

\end{fulllineitems}


\end{fulllineitems}

\phantomsection\label{\detokenize{reference/javascript_api:module-web.PivotModel}}

\begin{fulllineitems}
\phantomsection\label{\detokenize{reference/javascript_api:web.PivotModel}}\pysigline{\sphinxbfcode{\sphinxupquote{module }}\sphinxbfcode{\sphinxupquote{web.PivotModel}}}~~\begin{quote}\begin{description}
\item[{Exports}] \leavevmode{\hyperref[\detokenize{reference/javascript_api:web.PivotModel.PivotModel}]{\sphinxcrossref{
PivotModel
}}}
\item[{Depends On}] \leavevmode\begin{itemize}
\item {} {\hyperref[\detokenize{reference/javascript_api:web.AbstractModel}]{\sphinxcrossref{
web.AbstractModel
}}}
\item {} {\hyperref[\detokenize{reference/javascript_api:web.concurrency}]{\sphinxcrossref{
web.concurrency
}}}
\item {} {\hyperref[\detokenize{reference/javascript_api:web.core}]{\sphinxcrossref{
web.core
}}}
\item {} {\hyperref[\detokenize{reference/javascript_api:web.session}]{\sphinxcrossref{
web.session
}}}
\item {} {\hyperref[\detokenize{reference/javascript_api:web.utils}]{\sphinxcrossref{
web.utils
}}}
\end{itemize}

\end{description}\end{quote}


\begin{fulllineitems}
\phantomsection\label{\detokenize{reference/javascript_api:PivotModel}}\pysiglinewithargsret{\sphinxbfcode{\sphinxupquote{class }}\sphinxbfcode{\sphinxupquote{PivotModel}}}{\emph{params}}{}~\begin{quote}\begin{description}
\item[{Extends}] \leavevmode{\hyperref[\detokenize{reference/javascript_api:web.AbstractModel.AbstractModel}]{\sphinxcrossref{
AbstractModel
}}}
\item[{Parameters}] \leavevmode\begin{itemize}

\sphinxstylestrong{params} (\sphinxstyleliteralemphasis{\sphinxupquote{Object}})
\end{itemize}

\end{description}\end{quote}


\begin{fulllineitems}
\phantomsection\label{\detokenize{reference/javascript_api:closeHeader}}\pysiglinewithargsret{\sphinxbfcode{\sphinxupquote{method }}\sphinxbfcode{\sphinxupquote{closeHeader}}}{\emph{headerID}}{{ $\rightarrow$ Deferred}}
Close a header. This method is actually synchronous, but returns a
deferred.
\begin{quote}\begin{description}
\item[{Parameters}] \leavevmode\begin{itemize}

\sphinxstylestrong{headerID} (\sphinxstyleliteralemphasis{\sphinxupquote{any}})
\end{itemize}

\item[{Return Type}] \leavevmode
\sphinxstyleliteralemphasis{\sphinxupquote{Deferred}}

\end{description}\end{quote}

\end{fulllineitems}



\begin{fulllineitems}
\phantomsection\label{\detokenize{reference/javascript_api:expandHeader}}\pysiglinewithargsret{\sphinxbfcode{\sphinxupquote{method }}\sphinxbfcode{\sphinxupquote{expandHeader}}}{\emph{header}, \emph{field}}{}
Expand (open up) a given header, be it a row or a column.
\begin{quote}\begin{description}
\item[{Parameters}] \leavevmode\begin{itemize}

\sphinxstylestrong{header} (\sphinxstyleliteralemphasis{\sphinxupquote{any}})

\sphinxstylestrong{field} (\sphinxstyleliteralemphasis{\sphinxupquote{any}})
\end{itemize}

\end{description}\end{quote}

\end{fulllineitems}



\begin{fulllineitems}
\phantomsection\label{\detokenize{reference/javascript_api:exportData}}\pysiglinewithargsret{\sphinxbfcode{\sphinxupquote{method }}\sphinxbfcode{\sphinxupquote{exportData}}}{}{{ $\rightarrow$ Object}}
Export the current pivot view in a simple JS object.
\begin{quote}\begin{description}
\item[{Return Type}] \leavevmode
\sphinxstyleliteralemphasis{\sphinxupquote{Object}}

\end{description}\end{quote}

\end{fulllineitems}



\begin{fulllineitems}
\phantomsection\label{\detokenize{reference/javascript_api:flip}}\pysiglinewithargsret{\sphinxbfcode{\sphinxupquote{method }}\sphinxbfcode{\sphinxupquote{flip}}}{}{}
Swap the columns and the rows.  It is a synchronous operation.

\end{fulllineitems}



\begin{fulllineitems}
\phantomsection\label{\detokenize{reference/javascript_api:sortRows}}\pysiglinewithargsret{\sphinxbfcode{\sphinxupquote{method }}\sphinxbfcode{\sphinxupquote{sortRows}}}{\emph{col\_id}, \emph{measure}, \emph{descending}}{}
Sort the rows, depending on the values of a given column.  This is an
in-memory sort.
\begin{quote}\begin{description}
\item[{Parameters}] \leavevmode\begin{itemize}

\sphinxstylestrong{col\_id} (\sphinxstyleliteralemphasis{\sphinxupquote{any}})

\sphinxstylestrong{measure} (\sphinxstyleliteralemphasis{\sphinxupquote{any}})

\sphinxstylestrong{descending} (\sphinxstyleliteralemphasis{\sphinxupquote{any}})
\end{itemize}

\end{description}\end{quote}

\end{fulllineitems}



\begin{fulllineitems}
\phantomsection\label{\detokenize{reference/javascript_api:toggleMeasure}}\pysiglinewithargsret{\sphinxbfcode{\sphinxupquote{method }}\sphinxbfcode{\sphinxupquote{toggleMeasure}}}{\emph{field}}{{ $\rightarrow$ Deferred}}
Toggle the active state for a given measure, then reload the data.
\begin{quote}\begin{description}
\item[{Parameters}] \leavevmode\begin{itemize}

\sphinxstylestrong{field} (\sphinxstyleliteralemphasis{\sphinxupquote{string}})
\end{itemize}

\item[{Return Type}] \leavevmode
\sphinxstyleliteralemphasis{\sphinxupquote{Deferred}}

\end{description}\end{quote}

\end{fulllineitems}


\end{fulllineitems}


\end{fulllineitems}

\end{sphinxadmonition}


\subsection{Overview}
\label{\detokenize{reference/javascript_reference:overview}}
The Javascript framework is designed to work with three main use cases:
\begin{itemize}
\item {} 
the \sphinxstyleemphasis{web client}: this is the private web application, where one can view and
edit business data. This is a single page application (the page is never
reloaded, only the new data is fetched from the server whenever it is needed)

\item {} 
the \sphinxstyleemphasis{website}: this is the public part of Odoo.  It allows an unidentified
user to browse some content, to shop or to perform many actions, as a client.
This is a classical website: various routes with controllers and some
javascript to make it work.

\item {} 
the \sphinxstyleemphasis{point of sale}: this is the interface for the point of sale. It is a
specialized single page application.

\end{itemize}

Some javascript code is common to these three use cases, and is bundled together
(see below in the assets section).  This document will focus mostly on the web
client design.


\subsection{Web client}
\label{\detokenize{reference/javascript_reference:web-client}}

\subsubsection{Single Page Application}
\label{\detokenize{reference/javascript_reference:single-page-application}}
In short, the \sphinxstyleemphasis{webClient}, instance of \sphinxstyleemphasis{WebClient} is the root component of the
whole user interface.  Its responsability is to orchestrate all various
subcomponents, and to provide services, such as rpcs, local storage and more.

In runtime, the web client is a single page application. It does not need to
request a full page from the server each time the user perform an action. Instead,
it only requests what it needs, and then replaces/updates the view. Also, it
manages the url: it is kept in sync with the web client state.

It means that while a user is working on Odoo, the web client class (and the
action manager) actually creates and destroys many sub components. The state is
highly dynamic, and each widget could be destroyed at any time.


\subsubsection{Overview of web client JS code}
\label{\detokenize{reference/javascript_reference:overview-of-web-client-js-code}}
Here, we give a very quick overview on the web client code, in
the \sphinxstyleemphasis{web/static/src/js} addon. Note that it is deliberately not exhaustive.
We only cover the most important files/folders.
\begin{itemize}
\item {} 
\sphinxstyleemphasis{boot.js}: this is the file that defines the module system.  It needs to be
loaded first.

\item {} 
\sphinxstyleemphasis{core/}: this is a collection of lower level building blocks. Notably, it
contains the class system, the widget system, concurrency utilities, and many
other class/functions.

\item {} 
\sphinxstyleemphasis{chrome/}: in this folder, we have most large widgets which make up most of
the user interface.

\item {} 
\sphinxstyleemphasis{chrome/abstract\_web\_client.js} and \sphinxstyleemphasis{chrome/web\_client.js}: together, these
files define the WebClient widget, which is the root widget for the web client.

\item {} 
\sphinxstyleemphasis{chrome/action\_manager.js}: this is the code that will convert an action into
a widget (for example a kanban or a form view)

\item {} 
\sphinxstyleemphasis{chrome/search\_X.js} all these files define the search view (it is not a view
in the point of view of the web client, only from the server point of view)

\item {} 
\sphinxstyleemphasis{fields}: all main view field widgets are defined here

\item {} 
\sphinxstyleemphasis{views}: this is where the views are located

\end{itemize}


\subsection{Assets Management}
\label{\detokenize{reference/javascript_reference:assets-management}}
Managing assets in Odoo is not as straightforward as it is in some other apps.
One of the reason is that we have a variety of situations where some, but not all
the assets are required.  For example, the needs of the web client, the point of
sale, the website or even the mobile application are different.  Also, some
assets may be large, but are seldom needed.  In that case, we sometimes want them
to be loaded lazily.

The main idea is that we define a set of \sphinxstyleemphasis{bundles} in xml.  A bundle is here defined as
a collection of files (javascript, css, less). In Odoo, the most important
bundles are defined in the file \sphinxstyleemphasis{addons/web/views/webclient\_templates.xml}. It looks
like this:

\fvset{hllines={, ,}}%
\begin{sphinxVerbatim}[commandchars=\\\{\}]
\PYG{n+nt}{\PYGZlt{}template} \PYG{n+na}{id=}\PYG{l+s}{\PYGZdq{}web.assets\PYGZus{}common\PYGZdq{}} \PYG{n+na}{name=}\PYG{l+s}{\PYGZdq{}Common Assets (used in backend interface and website)\PYGZdq{}}\PYG{n+nt}{\PYGZgt{}}
    \PYG{n+nt}{\PYGZlt{}link} \PYG{n+na}{rel=}\PYG{l+s}{\PYGZdq{}stylesheet\PYGZdq{}} \PYG{n+na}{type=}\PYG{l+s}{\PYGZdq{}text/css\PYGZdq{}} \PYG{n+na}{href=}\PYG{l+s}{\PYGZdq{}/web/static/lib/jquery.ui/jquery\PYGZhy{}ui.css\PYGZdq{}}\PYG{n+nt}{/\PYGZgt{}}
    ...
    \PYG{n+nt}{\PYGZlt{}script} \PYG{n+na}{type=}\PYG{l+s}{\PYGZdq{}text/javascript\PYGZdq{}} \PYG{n+na}{src=}\PYG{l+s}{\PYGZdq{}/web/static/src/js/boot.js\PYGZdq{}}\PYG{n+nt}{\PYGZgt{}}\PYG{n+nt}{\PYGZlt{}/script\PYGZgt{}}
    ...
\PYG{n+nt}{\PYGZlt{}/template\PYGZgt{}}
\end{sphinxVerbatim}

The files in a bundle can then be inserted into a template by using the \sphinxstyleemphasis{t-call-assets}
directive:

\fvset{hllines={, ,}}%
\begin{sphinxVerbatim}[commandchars=\\\{\}]
\PYG{n+nt}{\PYGZlt{}t} \PYG{n+na}{t\PYGZhy{}call\PYGZhy{}assets=}\PYG{l+s}{\PYGZdq{}web.assets\PYGZus{}common\PYGZdq{}} \PYG{n+na}{t\PYGZhy{}js=}\PYG{l+s}{\PYGZdq{}false\PYGZdq{}}\PYG{n+nt}{/\PYGZgt{}}
\PYG{n+nt}{\PYGZlt{}t} \PYG{n+na}{t\PYGZhy{}call\PYGZhy{}assets=}\PYG{l+s}{\PYGZdq{}web.assets\PYGZus{}common\PYGZdq{}} \PYG{n+na}{t\PYGZhy{}css=}\PYG{l+s}{\PYGZdq{}false\PYGZdq{}}\PYG{n+nt}{/\PYGZgt{}}
\end{sphinxVerbatim}

Here is what happens when a template is rendered by the server with these directives:
\begin{itemize}
\item {} 
all the \sphinxstyleemphasis{less} files described in the bundle are compiled into css files. A file
named \sphinxstyleemphasis{file.less} will be compiled in a file named \sphinxstyleemphasis{file.less.css}.

\item {} \begin{description}
\item[{if we are in \sphinxstyleemphasis{debug=assets} mode,}] \leavevmode\begin{itemize}
\item {} 
the \sphinxstyleemphasis{t-call-assets} directive with the \sphinxstyleemphasis{t-js} attribute set to false will
be replaced by a list of stylesheet tags pointing to the css files

\item {} 
the \sphinxstyleemphasis{t-call-assets} directive with the \sphinxstyleemphasis{t-css} attribute set to false will
be replaced by a list of script tags pointing to the js files

\end{itemize}

\end{description}

\item {} \begin{description}
\item[{if we are not in \sphinxstyleemphasis{debug=assets} mode,}] \leavevmode\begin{itemize}
\item {} 
the css files will be concatenated and minified, then splits into files
with no more than 4096 rules (to get around an old limitation of IE9). Then,
we generate as many stylesheet tags as necessary

\item {} 
the js files are concatenated and minified, then a script tag is generated

\end{itemize}

\end{description}

\end{itemize}

Note that the assets files are cached, so in theory, a browser should only load
them once.


\subsubsection{Main bundles}
\label{\detokenize{reference/javascript_reference:main-bundles}}
When the Odoo server is started, it checks the timestamp of each file in a bundle,
and if necessary, will create/recreate the corresponding bundles.

Here are some important bundles that most developers will need to know:
\begin{itemize}
\item {} 
\sphinxstyleemphasis{web.assets\_common}: this bundle contains most assets which are common to the
web client, the website, and also the point of sale. This is supposed to contain
lower level building blocks for the odoo framework.  Note that it contains the
\sphinxstyleemphasis{boot.js} file, which defines the odoo module system.

\item {} 
\sphinxstyleemphasis{web.assets\_backend}: this bundle contains the code specific to the web client
(notably the web client/action manager/views)

\item {} 
\sphinxstyleemphasis{web.assets\_frontend}: this bundle is about all that is specific to the public
website: ecommerce, forum, blog, event management, …

\end{itemize}


\subsubsection{Adding files in an asset bundle}
\label{\detokenize{reference/javascript_reference:adding-files-in-an-asset-bundle}}
The proper way to add a file located in \sphinxstyleemphasis{addons/web} to a bundle is simple:
it is just enough to add a \sphinxstyleemphasis{script} or a \sphinxstyleemphasis{stylesheet} tag to the bundle in the
file \sphinxstyleemphasis{webclient\_templates.xml}.  But when we work in a different addon, we need
to add a file from that addon.  In that case, it should be done in three steps:
\begin{enumerate}
\item {} 
add a \sphinxstyleemphasis{assets.xml} file in the \sphinxstyleemphasis{views/} folder

\item {} 
add the string ‘views/assets.xml’ in the ‘data’ key in the manifest file

\item {} 
create an inherited view of the desired bundle, and add the file(s) with an
xpath expression. For example,

\end{enumerate}

\fvset{hllines={, ,}}%
\begin{sphinxVerbatim}[commandchars=\\\{\}]
\PYG{n+nt}{\PYGZlt{}template} \PYG{n+na}{id=}\PYG{l+s}{\PYGZdq{}assets\PYGZus{}backend\PYGZdq{}} \PYG{n+na}{name=}\PYG{l+s}{\PYGZdq{}helpdesk assets\PYGZdq{}} \PYG{n+na}{inherit\PYGZus{}id=}\PYG{l+s}{\PYGZdq{}web.assets\PYGZus{}backend\PYGZdq{}}\PYG{n+nt}{\PYGZgt{}}
    \PYG{n+nt}{\PYGZlt{}xpath} \PYG{n+na}{expr=}\PYG{l+s}{\PYGZdq{}//script[last()]\PYGZdq{}} \PYG{n+na}{position=}\PYG{l+s}{\PYGZdq{}after\PYGZdq{}}\PYG{n+nt}{\PYGZgt{}}
        \PYG{n+nt}{\PYGZlt{}link} \PYG{n+na}{rel=}\PYG{l+s}{\PYGZdq{}stylesheet\PYGZdq{}} \PYG{n+na}{href=}\PYG{l+s}{\PYGZdq{}/helpdesk/static/src/less/helpdesk.less\PYGZdq{}}\PYG{n+nt}{/\PYGZgt{}}
        \PYG{n+nt}{\PYGZlt{}script} \PYG{n+na}{type=}\PYG{l+s}{\PYGZdq{}text/javascript\PYGZdq{}} \PYG{n+na}{src=}\PYG{l+s}{\PYGZdq{}/helpdesk/static/src/js/helpdesk\PYGZus{}dashboard.js\PYGZdq{}}\PYG{n+nt}{\PYGZgt{}}\PYG{n+nt}{\PYGZlt{}/script\PYGZgt{}}
    \PYG{n+nt}{\PYGZlt{}/xpath\PYGZgt{}}
\PYG{n+nt}{\PYGZlt{}/template\PYGZgt{}}
\end{sphinxVerbatim}

\begin{sphinxadmonition}{note}{Note:}
Note that the files in a bundle are all loaded immediately when the user loads the
odoo web client.  This means that the files are transferred through the network
everytime (except when the browser cache is active).  In some cases, it may be
better to lazyload some assets.  For example, if a widget requires a large
library, and that widget is not a core part of the experience, then it may be
a good idea to only load the library when the widget is actually created. The
widget class has actually builtin support just for this use case. (see section
{\hyperref[\detokenize{reference/javascript_reference:reference-javascript-reference-qweb}]{\sphinxcrossref{\DUrole{std,std-ref}{QWeb Template Engine}}}})
\end{sphinxadmonition}


\subsubsection{What to do if a file is not loaded/updated}
\label{\detokenize{reference/javascript_reference:what-to-do-if-a-file-is-not-loaded-updated}}
There are many different reasons why a file may not be properly loaded.  Here
are a few things you can try to solve the issue:
\begin{itemize}
\item {} 
once the server is started, it does not know if an asset file has been
modified.  So, you can simply restart the server to regenerate the assets.

\item {} 
check the console (in the dev tools, usually opened with F12) to make sure
there are no obvious errors

\item {} 
try to add a console.log at the beginning of your file (before any module
definition), so you can see if a file has been loaded or not

\item {} 
in the user interface, in debug mode (INSERT LINK HERE TO DEBUG MODE), there
is an option to force the server to update its assets files.

\item {} 
use the \sphinxstyleemphasis{debug=assets} mode.  This will actually bypass the asset bundles (note
that it does not actually solve the issue. The server still uses outdated bundles)

\item {} 
finally, the most convenient way to do it, for a developer, is to start the
server with the \sphinxstyleemphasis{\textendash{}dev=all} option. This activates the file watcher options,
which will automatically invalidate assets when necessary.  Note that it does
not work very well if the OS is Windows.

\item {} 
remember to refresh your page!

\item {} 
or maybe to save your code file…

\end{itemize}

\begin{sphinxadmonition}{note}{Note:}
Once an asset file has been recreated, you need to refresh the page, to reload
the proper files (if that does not work, the files may be cached).
\end{sphinxadmonition}


\subsection{Javascript Module System}
\label{\detokenize{reference/javascript_reference:javascript-module-system}}
Once we are able to load our javascript files into the browser, we need to make
sure they are loaded in the correct order.  In order to do that, Odoo has defined
a small module system (located in the file \sphinxstyleemphasis{addons/web/static/src/js/boot.js},
which needs to be loaded first).

The Odoo module system, inspired by AMD, works by defining the function \sphinxstyleemphasis{define}
on the global odoo object. We then define each javascript module by calling that
function.  In the Odoo framework, a module is a piece of code that will be executed
as soon as possible.  It has a name and potentially some dependencies.  When its
dependencies are loaded, a module will then be loaded as well.  The value of the
module is then the return value of the function defining the module.

As an example, it may look like this:

\fvset{hllines={, ,}}%
\begin{sphinxVerbatim}[commandchars=\\\{\}]
\PYG{c+c1}{// in file a.js}
\PYG{n+nx}{odoo}\PYG{p}{.}\PYG{n+nx}{define}\PYG{p}{(}\PYG{l+s+s1}{\PYGZsq{}module.A\PYGZsq{}}\PYG{p}{,} \PYG{k+kd}{function} \PYG{p}{(}\PYG{n+nx}{require}\PYG{p}{)} \PYG{p}{\PYGZob{}}
    \PYG{l+s+s2}{\PYGZdq{}use strict\PYGZdq{}}\PYG{p}{;}

    \PYG{k+kd}{var} \PYG{n+nx}{A} \PYG{o}{=} \PYG{p}{...}\PYG{p}{;}

    \PYG{k}{return} \PYG{n+nx}{A}\PYG{p}{;}
\PYG{p}{\PYGZcb{}}\PYG{p}{)}\PYG{p}{;}

\PYG{c+c1}{// in file b.js}
\PYG{n+nx}{odoo}\PYG{p}{.}\PYG{n+nx}{define}\PYG{p}{(}\PYG{l+s+s1}{\PYGZsq{}module.B\PYGZsq{}}\PYG{p}{,} \PYG{k+kd}{function} \PYG{p}{(}\PYG{n+nx}{require}\PYG{p}{)} \PYG{p}{\PYGZob{}}
    \PYG{l+s+s2}{\PYGZdq{}use strict\PYGZdq{}}\PYG{p}{;}

    \PYG{k+kd}{var} \PYG{n+nx}{A} \PYG{o}{=} \PYG{n+nx}{require}\PYG{p}{(}\PYG{l+s+s1}{\PYGZsq{}module.A\PYGZsq{}}\PYG{p}{)}\PYG{p}{;}

    \PYG{k+kd}{var} \PYG{n+nx}{B} \PYG{o}{=} \PYG{p}{...}\PYG{p}{;} \PYG{c+c1}{// something that involves A}

    \PYG{k}{return} \PYG{n+nx}{B}\PYG{p}{;}
\PYG{p}{\PYGZcb{}}\PYG{p}{)}\PYG{p}{;}
\end{sphinxVerbatim}

An alternative way to define a module is to give explicitely a list of dependencies
in the second argument.

\fvset{hllines={, ,}}%
\begin{sphinxVerbatim}[commandchars=\\\{\}]
\PYG{n+nx}{odoo}\PYG{p}{.}\PYG{n+nx}{define}\PYG{p}{(}\PYG{l+s+s1}{\PYGZsq{}module.Something\PYGZsq{}}\PYG{p}{,} \PYG{p}{[}\PYG{l+s+s1}{\PYGZsq{}module.A\PYGZsq{}}\PYG{p}{,} \PYG{l+s+s1}{\PYGZsq{}module.B\PYGZsq{}}\PYG{p}{]}\PYG{p}{,} \PYG{k+kd}{function} \PYG{p}{(}\PYG{n+nx}{require}\PYG{p}{)} \PYG{p}{\PYGZob{}}
    \PYG{l+s+s2}{\PYGZdq{}use strict\PYGZdq{}}\PYG{p}{;}

    \PYG{k+kd}{var} \PYG{n+nx}{A} \PYG{o}{=} \PYG{n+nx}{require}\PYG{p}{(}\PYG{l+s+s1}{\PYGZsq{}module.A\PYGZsq{}}\PYG{p}{)}\PYG{p}{;}
    \PYG{k+kd}{var} \PYG{n+nx}{B} \PYG{o}{=} \PYG{n+nx}{require}\PYG{p}{(}\PYG{l+s+s1}{\PYGZsq{}module.B\PYGZsq{}}\PYG{p}{)}\PYG{p}{;}

    \PYG{c+c1}{// some code}
\PYG{p}{\PYGZcb{}}\PYG{p}{)}\PYG{p}{;}
\end{sphinxVerbatim}

If some dependencies are missing/non ready, then the module will simply not be
loaded.  There will be a warning in the console after a few seconds.

Note that circular dependencies are not supported. It makes sense, but it means that one
needs to be careful.


\subsubsection{Defining a module}
\label{\detokenize{reference/javascript_reference:defining-a-module}}
The \sphinxstyleemphasis{odoo.define} method is given three arguments:
\begin{itemize}
\item {} 
\sphinxstyleemphasis{moduleName}: the name of the javascript module.  It should be a unique string.
The convention is to have the name of the odoo addon followed by a specific
description. For example, ‘web.Widget’ describes a module defined in the \sphinxstyleemphasis{web}
addon, which exports a \sphinxstyleemphasis{Widget} class (because the first letter is capitalized)

If the name is not unique, an exception will be thrown and displayed in the
console.

\item {} 
\sphinxstyleemphasis{dependencies}: the second argument is optional. If given, it should be a list
of strings, each corresponding to a javascript module.  This describes the
dependencies that are required to be loaded before the module is executed. If
the dependencies are not explicitely given here, then the module system will
extract them from the function by calling toString on it, then using a regexp
to find all \sphinxstyleemphasis{require} statements.

\item {} 
finally, the last argument is a function which defines the module. Its return
value is the value of the module, which may be passed to other modules requiring
it.  Note that there is a small exception for asynchronous modules, see the
next section.

\end{itemize}

If an error happens, it will be logged (in debug mode) in the console:
\begin{itemize}
\item {} 
\sphinxcode{\sphinxupquote{Missing dependencies}}:
These modules do not appear in the page. It is possible that the JavaScript
file is not in the page or that the module name is wrong

\item {} 
\sphinxcode{\sphinxupquote{Failed modules}}:
A javascript error is detected

\item {} 
\sphinxcode{\sphinxupquote{Rejected modules}}:
The module returns a rejected deferred. It (and its dependent modules) is not
loaded.

\item {} 
\sphinxcode{\sphinxupquote{Rejected linked modules}}:
Modules who depend on a rejected module

\item {} 
\sphinxcode{\sphinxupquote{Non loaded modules}}:
Modules who depend on a missing or a failed module

\end{itemize}


\subsubsection{Asynchronous modules}
\label{\detokenize{reference/javascript_reference:asynchronous-modules}}
It can happen that a module needs to perform some work before it is ready.  For
example, it could do a rpc to load some data.  In that case, the module can
simply return a deferred (promise).  In that case, the module system will simply
wait for the deferred to complete before registering the module.

\fvset{hllines={, ,}}%
\begin{sphinxVerbatim}[commandchars=\\\{\}]
\PYG{n+nx}{odoo}\PYG{p}{.}\PYG{n+nx}{define}\PYG{p}{(}\PYG{l+s+s1}{\PYGZsq{}module.Something\PYGZsq{}}\PYG{p}{,} \PYG{p}{[}\PYG{l+s+s1}{\PYGZsq{}web.ajax\PYGZsq{}}\PYG{p}{]}\PYG{p}{,} \PYG{k+kd}{function} \PYG{p}{(}\PYG{n+nx}{require}\PYG{p}{)} \PYG{p}{\PYGZob{}}
    \PYG{l+s+s2}{\PYGZdq{}use strict\PYGZdq{}}\PYG{p}{;}

    \PYG{k+kd}{var} \PYG{n+nx}{ajax} \PYG{o}{=} \PYG{n+nx}{require}\PYG{p}{(}\PYG{l+s+s1}{\PYGZsq{}web.ajax\PYGZsq{}}\PYG{p}{)}\PYG{p}{;}

    \PYG{k}{return} \PYG{n+nx}{ajax}\PYG{p}{.}\PYG{n+nx}{rpc}\PYG{p}{(}\PYG{p}{...}\PYG{p}{)}\PYG{p}{.}\PYG{n+nx}{then}\PYG{p}{(}\PYG{k+kd}{function} \PYG{p}{(}\PYG{n+nx}{result}\PYG{p}{)} \PYG{p}{\PYGZob{}}
        \PYG{c+c1}{// some code here}
        \PYG{k}{return} \PYG{n+nx}{something}\PYG{p}{;}
    \PYG{p}{\PYGZcb{}}\PYG{p}{)}\PYG{p}{;}
\PYG{p}{\PYGZcb{}}\PYG{p}{)}\PYG{p}{;}
\end{sphinxVerbatim}


\subsubsection{Best practices}
\label{\detokenize{reference/javascript_reference:best-practices}}\begin{itemize}
\item {} 
remember the convention for a module name: \sphinxstyleemphasis{addon name} suffixed with \sphinxstyleemphasis{module
name}.

\item {} 
declare all your dependencies at the top of the module. Also, they should be
sorted alphabetically by module name. This makes it easier to understand your module.

\item {} 
declare all exported values at the end

\item {} 
try to avoid exporting too much things from one module.  It is usually better
to simply export one thing in one (small/smallish) module.

\item {} 
asynchronous modules can be used to simplify some use cases.  For example,
the \sphinxstyleemphasis{web.dom\_ready} module returns a deferred which will be resolved when the
dom is actually ready.  So, another module that needs the DOM could simply have
a \sphinxstyleemphasis{require(‘web.dom\_ready’)} statement somewhere, and the code will only be
executed when the DOM is ready.

\item {} 
try to avoid defining more than one module in one file.  It may be convenient
in the short term, but this is actually harder to maintain.

\end{itemize}


\subsection{Class System}
\label{\detokenize{reference/javascript_reference:class-system}}
Odoo was developped before ECMAScript 6 classes were available.  In Ecmascript 5,
the standard way to define a class is to define a function and to add methods
on its prototype object.  This is fine, but it is slightly complex when we want
to use inheritance, mixins.

For these reasons, Odoo decided to use its own class system, inspired by John
Resig. The base Class is located in \sphinxstyleemphasis{web.Class}, in the file \sphinxstyleemphasis{class.js}.


\subsubsection{Creating a subclass}
\label{\detokenize{reference/javascript_reference:creating-a-subclass}}
Let us discuss how classes are created.  The main mechanism is to use the
\sphinxstyleemphasis{extend} method (this is more or less the equivalent of \sphinxstyleemphasis{extend} in ES6 classes).

\fvset{hllines={, ,}}%
\begin{sphinxVerbatim}[commandchars=\\\{\}]
\PYG{k+kd}{var} \PYG{n+nx}{Class} \PYG{o}{=} \PYG{n+nx}{require}\PYG{p}{(}\PYG{l+s+s1}{\PYGZsq{}web.Class\PYGZsq{}}\PYG{p}{)}\PYG{p}{;}

\PYG{k+kd}{var} \PYG{n+nx}{Animal} \PYG{o}{=} \PYG{n+nx}{Class}\PYG{p}{.}\PYG{n+nx}{extend}\PYG{p}{(}\PYG{p}{\PYGZob{}}
    \PYG{n+nx}{init}\PYG{o}{:} \PYG{k+kd}{function} \PYG{p}{(}\PYG{p}{)} \PYG{p}{\PYGZob{}}
        \PYG{k}{this}\PYG{p}{.}\PYG{n+nx}{x} \PYG{o}{=} \PYG{l+m+mi}{0}\PYG{p}{;}
        \PYG{k}{this}\PYG{p}{.}\PYG{n+nx}{hunger} \PYG{o}{=} \PYG{l+m+mi}{0}\PYG{p}{;}
    \PYG{p}{\PYGZcb{}}\PYG{p}{,}
    \PYG{n+nx}{move}\PYG{o}{:} \PYG{k+kd}{function} \PYG{p}{(}\PYG{p}{)} \PYG{p}{\PYGZob{}}
        \PYG{k}{this}\PYG{p}{.}\PYG{n+nx}{x} \PYG{o}{=} \PYG{k}{this}\PYG{p}{.}\PYG{n+nx}{x} \PYG{o}{+} \PYG{l+m+mi}{1}\PYG{p}{;}
        \PYG{k}{this}\PYG{p}{.}\PYG{n+nx}{hunger} \PYG{o}{=} \PYG{k}{this}\PYG{p}{.}\PYG{n+nx}{hunger} \PYG{o}{+} \PYG{l+m+mi}{1}\PYG{p}{;}
    \PYG{p}{\PYGZcb{}}\PYG{p}{,}
    \PYG{n+nx}{eat}\PYG{o}{:} \PYG{k+kd}{function} \PYG{p}{(}\PYG{p}{)} \PYG{p}{\PYGZob{}}
        \PYG{k}{this}\PYG{p}{.}\PYG{n+nx}{hunger} \PYG{o}{=} \PYG{l+m+mi}{0}\PYG{p}{;}
    \PYG{p}{\PYGZcb{}}\PYG{p}{,}
\PYG{p}{\PYGZcb{}}\PYG{p}{)}\PYG{p}{;}
\end{sphinxVerbatim}

In this example, the \sphinxstyleemphasis{init} function is the constructor.  It will be called when
an instance is created.  Making an instance is done by using the \sphinxstyleemphasis{new} keyword.


\subsubsection{Inheritance}
\label{\detokenize{reference/javascript_reference:inheritance}}
It is convenient to be able to inherit an existing class.  This is simply done
by using the \sphinxstyleemphasis{extend} method on the superclass.  When a method is called, the
framework will secretly rebind a special method: \sphinxstyleemphasis{\_super} to the currently
called method.  This allows us to use \sphinxstyleemphasis{this.\_super} whenever we need to call a
parent method.

\fvset{hllines={, ,}}%
\begin{sphinxVerbatim}[commandchars=\\\{\}]
\PYG{k+kd}{var} \PYG{n+nx}{Animal} \PYG{o}{=} \PYG{n+nx}{require}\PYG{p}{(}\PYG{l+s+s1}{\PYGZsq{}web.Animal\PYGZsq{}}\PYG{p}{)}\PYG{p}{;}

\PYG{k+kd}{var} \PYG{n+nx}{Dog} \PYG{o}{=} \PYG{n+nx}{Animal}\PYG{p}{.}\PYG{n+nx}{extend}\PYG{p}{(}\PYG{p}{\PYGZob{}}
    \PYG{n+nx}{move}\PYG{o}{:} \PYG{k+kd}{function} \PYG{p}{(}\PYG{p}{)} \PYG{p}{\PYGZob{}}
        \PYG{k}{this}\PYG{p}{.}\PYG{n+nx}{bark}\PYG{p}{(}\PYG{p}{)}\PYG{p}{;}
        \PYG{k}{this}\PYG{p}{.}\PYG{n+nx}{\PYGZus{}super}\PYG{p}{.}\PYG{n+nx}{apply}\PYG{p}{(}\PYG{k}{this}\PYG{p}{,} \PYG{n+nx}{arguments}\PYG{p}{)}\PYG{p}{;}
    \PYG{p}{\PYGZcb{}}\PYG{p}{,}
    \PYG{n+nx}{bark}\PYG{o}{:} \PYG{k+kd}{function} \PYG{p}{(}\PYG{p}{)} \PYG{p}{\PYGZob{}}
        \PYG{n+nx}{console}\PYG{p}{.}\PYG{n+nx}{log}\PYG{p}{(}\PYG{l+s+s1}{\PYGZsq{}woof\PYGZsq{}}\PYG{p}{)}\PYG{p}{;}
    \PYG{p}{\PYGZcb{}}\PYG{p}{,}
\PYG{p}{\PYGZcb{}}\PYG{p}{)}\PYG{p}{;}

\PYG{k+kd}{var} \PYG{n+nx}{dog} \PYG{o}{=} \PYG{k}{new} \PYG{n+nx}{Dog}\PYG{p}{(}\PYG{p}{)}\PYG{p}{;}
\PYG{n+nx}{dog}\PYG{p}{.}\PYG{n+nx}{move}\PYG{p}{(}\PYG{p}{)}
\end{sphinxVerbatim}


\subsubsection{Mixins}
\label{\detokenize{reference/javascript_reference:mixins}}
The odoo Class system does not support multiple inheritance, but for those cases
when we need to share some behaviour, we have a mixin system: the \sphinxstyleemphasis{extend}
method can actually take an arbitrary number of arguments, and will combine all
of them in the new class.

\fvset{hllines={, ,}}%
\begin{sphinxVerbatim}[commandchars=\\\{\}]
\PYG{k+kd}{var} \PYG{n+nx}{Animal} \PYG{o}{=} \PYG{n+nx}{require}\PYG{p}{(}\PYG{l+s+s1}{\PYGZsq{}web.Animal\PYGZsq{}}\PYG{p}{)}\PYG{p}{;}
\PYG{k+kd}{var} \PYG{n+nx}{DanceMixin} \PYG{o}{=} \PYG{p}{\PYGZob{}}
    \PYG{n+nx}{dance}\PYG{o}{:} \PYG{k+kd}{function} \PYG{p}{(}\PYG{p}{)} \PYG{p}{\PYGZob{}}
        \PYG{n+nx}{console}\PYG{p}{.}\PYG{n+nx}{log}\PYG{p}{(}\PYG{l+s+s1}{\PYGZsq{}dancing...\PYGZsq{}}\PYG{p}{)}\PYG{p}{;}
    \PYG{p}{\PYGZcb{}}\PYG{p}{,}
\PYG{p}{\PYGZcb{}}\PYG{p}{;}

\PYG{k+kd}{var} \PYG{n+nx}{Hamster} \PYG{o}{=} \PYG{n+nx}{Hamster}\PYG{p}{.}\PYG{n+nx}{extend}\PYG{p}{(}\PYG{n+nx}{DanceMixin}\PYG{p}{,} \PYG{p}{\PYGZob{}}
    \PYG{n+nx}{sleep}\PYG{o}{:} \PYG{k+kd}{function} \PYG{p}{(}\PYG{p}{)} \PYG{p}{\PYGZob{}}
        \PYG{n+nx}{console}\PYG{p}{.}\PYG{n+nx}{log}\PYG{p}{(}\PYG{l+s+s1}{\PYGZsq{}sleeping\PYGZsq{}}\PYG{p}{)}\PYG{p}{;}
    \PYG{p}{\PYGZcb{}}\PYG{p}{,}
\PYG{p}{\PYGZcb{}}\PYG{p}{)}\PYG{p}{;}
\end{sphinxVerbatim}

In this example, the \sphinxstyleemphasis{Hamster} class is a subclass of Animal, but it also mix
the DanceMixin in.


\subsubsection{Patching an existing class}
\label{\detokenize{reference/javascript_reference:patching-an-existing-class}}
It is not common, but we sometimes need to modify another class \sphinxstyleemphasis{in place}. The
goal is to have a mechanism to change a class and all future/present instances.
This is done by using the \sphinxstyleemphasis{include} method:

\fvset{hllines={, ,}}%
\begin{sphinxVerbatim}[commandchars=\\\{\}]
\PYG{k+kd}{var} \PYG{n+nx}{Hamster} \PYG{o}{=} \PYG{n+nx}{require}\PYG{p}{(}\PYG{l+s+s1}{\PYGZsq{}web.Hamster\PYGZsq{}}\PYG{p}{)}\PYG{p}{;}

\PYG{n+nx}{Hamster}\PYG{p}{.}\PYG{n+nx}{include}\PYG{p}{(}\PYG{p}{\PYGZob{}}
    \PYG{n+nx}{sleep}\PYG{o}{:} \PYG{k+kd}{function} \PYG{p}{(}\PYG{p}{)} \PYG{p}{\PYGZob{}}
        \PYG{k}{this}\PYG{p}{.}\PYG{n+nx}{\PYGZus{}super}\PYG{p}{.}\PYG{n+nx}{apply}\PYG{p}{(}\PYG{k}{this}\PYG{p}{,} \PYG{n+nx}{arguments}\PYG{p}{)}\PYG{p}{;}
        \PYG{n+nx}{console}\PYG{p}{.}\PYG{n+nx}{log}\PYG{p}{(}\PYG{l+s+s1}{\PYGZsq{}zzzz\PYGZsq{}}\PYG{p}{)}\PYG{p}{;}
    \PYG{p}{\PYGZcb{}}\PYG{p}{,}
\PYG{p}{\PYGZcb{}}\PYG{p}{)}\PYG{p}{;}
\end{sphinxVerbatim}

This is obviously a dangerous operation and should be done with care.  But with
the way Odoo is structured, it is sometimes necessary in one addon to modify
the behavior of a widget/class defined in another addon.  Note that it will
modify all instances of the class, even if they have already been created.


\subsection{Widgets}
\label{\detokenize{reference/javascript_reference:widgets}}
The \sphinxstyleemphasis{Widget} class is really an important building block of the user interface.
Pretty much everything in the user interface is under the control of a widget.
The Widget class is defined in the module \sphinxstyleemphasis{web.Widget}, in \sphinxstyleemphasis{widget.js}.

In short, the features provided by the Widget class include:
\begin{itemize}
\item {} 
parent/child relationships between widgets (\sphinxstyleemphasis{PropertiesMixin})

\item {} \begin{description}
\item[{extensive lifecycle management with safety features (e.g.}] \leavevmode
automatically destroying children widgets during the destruction of a
parent)

\end{description}

\item {} 
automatic rendering with {\hyperref[\detokenize{reference/qweb:reference-qweb}]{\sphinxcrossref{\DUrole{std,std-ref}{qweb}}}}

\item {} 
various utility functions to help interacting with the outside environment.

\end{itemize}

Here is an example of a basic counter widget:

\fvset{hllines={, ,}}%
\begin{sphinxVerbatim}[commandchars=\\\{\}]
\PYG{k+kd}{var} \PYG{n+nx}{Widget} \PYG{o}{=} \PYG{n+nx}{require}\PYG{p}{(}\PYG{l+s+s1}{\PYGZsq{}web.Widget\PYGZsq{}}\PYG{p}{)}\PYG{p}{;}

\PYG{k+kd}{var} \PYG{n+nx}{Counter} \PYG{o}{=} \PYG{n+nx}{Widget}\PYG{p}{.}\PYG{n+nx}{extend}\PYG{p}{(}\PYG{p}{\PYGZob{}}
    \PYG{n+nx}{template}\PYG{o}{:} \PYG{l+s+s1}{\PYGZsq{}some.template\PYGZsq{}}\PYG{p}{,}
    \PYG{n+nx}{events}\PYG{o}{:} \PYG{p}{\PYGZob{}}
        \PYG{l+s+s1}{\PYGZsq{}click button\PYGZsq{}}\PYG{o}{:} \PYG{l+s+s1}{\PYGZsq{}\PYGZus{}onClick\PYGZsq{}}\PYG{p}{,}
    \PYG{p}{\PYGZcb{}}\PYG{p}{,}
    \PYG{n+nx}{init}\PYG{o}{:} \PYG{k+kd}{function} \PYG{p}{(}\PYG{n+nx}{parent}\PYG{p}{,} \PYG{n+nx}{value}\PYG{p}{)} \PYG{p}{\PYGZob{}}
        \PYG{k}{this}\PYG{p}{.}\PYG{n+nx}{\PYGZus{}super}\PYG{p}{(}\PYG{n+nx}{parent}\PYG{p}{)}\PYG{p}{;}
        \PYG{k}{this}\PYG{p}{.}\PYG{n+nx}{count} \PYG{o}{=} \PYG{n+nx}{value}\PYG{p}{;}
    \PYG{p}{\PYGZcb{}}\PYG{p}{,}
    \PYG{n+nx}{\PYGZus{}onClick}\PYG{o}{:} \PYG{k+kd}{function} \PYG{p}{(}\PYG{p}{)} \PYG{p}{\PYGZob{}}
        \PYG{k}{this}\PYG{p}{.}\PYG{n+nx}{count}\PYG{o}{++}\PYG{p}{;}
        \PYG{k}{this}\PYG{p}{.}\PYG{n+nx}{\PYGZdl{}}\PYG{p}{(}\PYG{l+s+s1}{\PYGZsq{}.val\PYGZsq{}}\PYG{p}{)}\PYG{p}{.}\PYG{n+nx}{text}\PYG{p}{(}\PYG{k}{this}\PYG{p}{.}\PYG{n+nx}{count}\PYG{p}{)}\PYG{p}{;}
    \PYG{p}{\PYGZcb{}}\PYG{p}{,}
\PYG{p}{\PYGZcb{}}\PYG{p}{)}\PYG{p}{;}
\end{sphinxVerbatim}

For this example, assume that the template \sphinxstyleemphasis{some.template} (and is properly
loaded: the template is in a file, which is properly defined in the \sphinxstyleemphasis{qweb} key
in the module manifest) is given by:

\fvset{hllines={, ,}}%
\begin{sphinxVerbatim}[commandchars=\\\{\}]
\PYG{n+nt}{\PYGZlt{}div} \PYG{n+na}{t\PYGZhy{}name=}\PYG{l+s}{\PYGZdq{}some.template\PYGZdq{}}\PYG{n+nt}{\PYGZgt{}}
    \PYG{n+nt}{\PYGZlt{}span} \PYG{n+na}{class=}\PYG{l+s}{\PYGZdq{}val\PYGZdq{}}\PYG{n+nt}{\PYGZgt{}}\PYG{n+nt}{\PYGZlt{}t} \PYG{n+na}{t\PYGZhy{}esc=}\PYG{l+s}{\PYGZdq{}widget.count\PYGZdq{}}\PYG{n+nt}{/\PYGZgt{}}\PYG{n+nt}{\PYGZlt{}/span\PYGZgt{}}
    \PYG{n+nt}{\PYGZlt{}button}\PYG{n+nt}{\PYGZgt{}}Increment\PYG{n+nt}{\PYGZlt{}/button\PYGZgt{}}
\PYG{n+nt}{\PYGZlt{}/div\PYGZgt{}}
\end{sphinxVerbatim}

This example widget can be used in the following manner:

\fvset{hllines={, ,}}%
\begin{sphinxVerbatim}[commandchars=\\\{\}]
\PYG{c+c1}{// Create the instance}
\PYG{k+kd}{var} \PYG{n+nx}{counter} \PYG{o}{=} \PYG{k}{new} \PYG{n+nx}{Counter}\PYG{p}{(}\PYG{k}{this}\PYG{p}{,} \PYG{l+m+mi}{4}\PYG{p}{)}\PYG{p}{;}
\PYG{c+c1}{// Render and insert into DOM}
\PYG{n+nx}{counter}\PYG{p}{.}\PYG{n+nx}{appendTo}\PYG{p}{(}\PYG{l+s+s2}{\PYGZdq{}.some\PYGZhy{}div\PYGZdq{}}\PYG{p}{)}\PYG{p}{;}
\end{sphinxVerbatim}

This example illustrates a few of the features of the \sphinxstyleemphasis{Widget} class, including
the event system, the template system, the constructor with the initial \sphinxstyleemphasis{parent} argument.


\subsubsection{Widget Lifecycle}
\label{\detokenize{reference/javascript_reference:widget-lifecycle}}
Like many component systems, the widget class has a well defined lifecycle. The
usual lifecycle is the following: \sphinxstyleemphasis{init} is called, then \sphinxstyleemphasis{willStart}, then the
rendering takes place, then \sphinxstyleemphasis{start} and finally \sphinxstyleemphasis{destroy}.
\index{Widget.init() (Widget method)}

\begin{fulllineitems}
\phantomsection\label{\detokenize{reference/javascript_reference:Widget.init}}\pysiglinewithargsret{\sphinxcode{\sphinxupquote{Widget.}}\sphinxbfcode{\sphinxupquote{init}}}{\emph{parent}}{}
this is the constructor.  The init method is supposed to initialize the
base state of the widget. It is synchronous and can be overridden to
take more parameters from the widget’s creator/parent
\begin{quote}\begin{description}
\item[{Arguments}] \leavevmode\begin{itemize}
\item {} 
\sphinxstyleliteralstrong{\sphinxupquote{parent}} ({\hyperref[\detokenize{reference/javascript_api:Widget}]{\sphinxcrossref{\sphinxcode{\sphinxupquote{Widget()}}}}}) \textendash{} the new widget’s parent, used to handle automatic
destruction and event propagation. Can be \sphinxcode{\sphinxupquote{null}} for
the widget to have no parent.

\end{itemize}

\end{description}\end{quote}

\end{fulllineitems}

\index{Widget.willStart() (Widget method)}

\begin{fulllineitems}
\phantomsection\label{\detokenize{reference/javascript_reference:Widget.willStart}}\pysiglinewithargsret{\sphinxcode{\sphinxupquote{Widget.}}\sphinxbfcode{\sphinxupquote{willStart}}}{}{}
this method will be called once by the framework when a widget is created
and in the process of being appended to the DOM.  The \sphinxstyleemphasis{willStart} method is a
hook that should return a deferred.  The JS framework will wait for this deferred
to complete before moving on to the rendering step.  Note that at this point,
the widget does not have a DOM root element.  The \sphinxstyleemphasis{willStart} hook is mostly
useful to perfom some asynchronous work, such as fetching data from the server

\end{fulllineitems}

\index{{[}Rendering{]}() (built-in function)}

\begin{fulllineitems}
\phantomsection\label{\detokenize{reference/javascript_reference:_Rendering_}}\pysiglinewithargsret{\sphinxbfcode{\sphinxupquote{{[}Rendering{]}}}}{}{}
This step is automatically done by the framework.  What happens is
that the framework checks if a template key is defined on the widget.  If that is
the case, then it will render that template with the \sphinxstyleemphasis{widget} key bound to the
widget in the rendering context (see the example above: we use \sphinxstyleemphasis{widget.count}
in the QWeb template to read the value from the widget). If no template is
defined, we read the \sphinxstyleemphasis{tagName} key and create a corresponding DOM element.
When the rendering is done, we set the result as the \$el property of the widget.
After this, we automatically bind all events in the events and custom\_events
keys.

\end{fulllineitems}

\index{Widget.start() (Widget method)}

\begin{fulllineitems}
\phantomsection\label{\detokenize{reference/javascript_reference:Widget.start}}\pysiglinewithargsret{\sphinxcode{\sphinxupquote{Widget.}}\sphinxbfcode{\sphinxupquote{start}}}{}{}
when the rendering is complete, the framework will automatically call
the \sphinxstyleemphasis{start} method.  This is useful to perform some specialized post-rendering
work.  For example, setting up a library.

Must return a deferred to indicate when its work is done.
\begin{quote}\begin{description}
\item[{Returns}] \leavevmode
deferred object

\end{description}\end{quote}

\end{fulllineitems}

\index{Widget.destroy() (Widget method)}

\begin{fulllineitems}
\phantomsection\label{\detokenize{reference/javascript_reference:Widget.destroy}}\pysiglinewithargsret{\sphinxcode{\sphinxupquote{Widget.}}\sphinxbfcode{\sphinxupquote{destroy}}}{}{}
This is always the final step in the life of a widget.  When a
widget is destroyed, we basically perform all necessary cleanup operations:
removing the widget from the component tree, unbinding all events, …

Automatically called when the widget’s parent is destroyed,
must be called explicitly if the widget has no parent or if it is
removed but its parent remains.

\end{fulllineitems}


Note that the willStart and start method are not necessarily called.  A widget
can be created (the \sphinxstyleemphasis{init} method will be called) and then destroyed (\sphinxstyleemphasis{destroy}
method) without ever having been appended to the DOM.  If that is the case, the
willStart and start will not even be called.


\subsubsection{Widget API}
\label{\detokenize{reference/javascript_reference:widget-api}}\index{Widget.tagName (Widget attribute)}

\begin{fulllineitems}
\phantomsection\label{\detokenize{reference/javascript_reference:Widget.tagName}}\pysigline{\sphinxcode{\sphinxupquote{Widget.}}\sphinxbfcode{\sphinxupquote{tagName}}}
Used if the widget has no template defined. Defaults to \sphinxcode{\sphinxupquote{div}},
will be used as the tag name to create the DOM element to set as
the widget’s DOM root. It is possible to further customize this
generated DOM root with the following attributes:
\index{Widget.Widget.id (Widget.Widget attribute)}

\begin{fulllineitems}
\phantomsection\label{\detokenize{reference/javascript_reference:Widget.Widget.id}}\pysigline{\sphinxcode{\sphinxupquote{Widget.Widget.}}\sphinxbfcode{\sphinxupquote{id}}}
Used to generate an \sphinxcode{\sphinxupquote{id}} attribute on the generated DOM
root. Note that this is rarely needed, and is probably not a good idea
if a widget can be used more than once.

\end{fulllineitems}

\index{Widget.className (Widget attribute)}

\begin{fulllineitems}
\phantomsection\label{\detokenize{reference/javascript_reference:Widget.className}}\pysigline{\sphinxcode{\sphinxupquote{Widget.}}\sphinxbfcode{\sphinxupquote{className}}}
Used to generate a \sphinxcode{\sphinxupquote{class}} attribute on the generated DOM root. Note
that it can actually contain more than one css class:
\sphinxstyleemphasis{‘some-class other-class’}

\end{fulllineitems}

\index{Widget.attributes (Widget attribute)}

\begin{fulllineitems}
\phantomsection\label{\detokenize{reference/javascript_reference:Widget.attributes}}\pysigline{\sphinxcode{\sphinxupquote{Widget.}}\sphinxbfcode{\sphinxupquote{attributes}}}
Mapping (object literal) of attribute names to attribute
values. Each of these k:v pairs will be set as a DOM attribute
on the generated DOM root.

\end{fulllineitems}


\end{fulllineitems}

\index{Widget.el (Widget attribute)}

\begin{fulllineitems}
\phantomsection\label{\detokenize{reference/javascript_reference:Widget.el}}\pysigline{\sphinxcode{\sphinxupquote{Widget.}}\sphinxbfcode{\sphinxupquote{el}}}
raw DOM element set as root to the widget (only available after the start
lifecyle method)

\end{fulllineitems}

\index{Widget.\$el (Widget attribute)}

\begin{fulllineitems}
\phantomsection\label{\detokenize{reference/javascript_reference:Widget._S_el}}\pysigline{\sphinxcode{\sphinxupquote{Widget.}}\sphinxbfcode{\sphinxupquote{\$el}}}
jQuery wrapper around {\hyperref[\detokenize{reference/javascript_reference:Widget.el}]{\sphinxcrossref{\sphinxcode{\sphinxupquote{el}}}}}. (only available after the start
lifecyle method)

\end{fulllineitems}

\index{Widget.template (Widget attribute)}

\begin{fulllineitems}
\phantomsection\label{\detokenize{reference/javascript_reference:Widget.template}}\pysigline{\sphinxcode{\sphinxupquote{Widget.}}\sphinxbfcode{\sphinxupquote{template}}}
Should be set to the name of a {\hyperref[\detokenize{reference/qweb:reference-qweb}]{\sphinxcrossref{\DUrole{std,std-ref}{QWeb template}}}}.
If set, the template will be rendered after the widget has been
initialized but before it has been started. The root element generated by
the template will be set as the DOM root of the widget.

\end{fulllineitems}

\index{xmlDependencies (None attribute)}

\begin{fulllineitems}
\phantomsection\label{\detokenize{reference/javascript_reference:xmlDependencies}}\pysigline{\sphinxbfcode{\sphinxupquote{xmlDependencies}}}
List of paths to xml files that need to be loaded before the
widget can be rendered. This will not induce loading anything that has already
been loaded.

\end{fulllineitems}

\index{events (None attribute)}

\begin{fulllineitems}
\phantomsection\label{\detokenize{reference/javascript_reference:events}}\pysigline{\sphinxbfcode{\sphinxupquote{events}}}
Events are a mapping of an event selector (an event name and an optional
CSS selector separated by a space) to a callback. The callback can
be the name of a widget’s method or a function object. In either case, the
\sphinxcode{\sphinxupquote{this}} will be set to the widget:

\fvset{hllines={, ,}}%
\begin{sphinxVerbatim}[commandchars=\\\{\}]
\PYG{n+nx}{events}\PYG{o}{:} \PYG{p}{\PYGZob{}}
    \PYG{l+s+s1}{\PYGZsq{}click p.oe\PYGZus{}some\PYGZus{}class a\PYGZsq{}}\PYG{o}{:} \PYG{l+s+s1}{\PYGZsq{}some\PYGZus{}method\PYGZsq{}}\PYG{p}{,}
    \PYG{l+s+s1}{\PYGZsq{}change input\PYGZsq{}}\PYG{o}{:} \PYG{k+kd}{function} \PYG{p}{(}\PYG{n+nx}{e}\PYG{p}{)} \PYG{p}{\PYGZob{}}
        \PYG{n+nx}{e}\PYG{p}{.}\PYG{n+nx}{stopPropagation}\PYG{p}{(}\PYG{p}{)}\PYG{p}{;}
    \PYG{p}{\PYGZcb{}}
\PYG{p}{\PYGZcb{}}\PYG{p}{,}
\end{sphinxVerbatim}

The selector is used for jQuery’s event delegation, the
callback will only be triggered for descendants of the DOM root
matching the selector. If the selector is left out
(only an event name is specified), the event will be set directly on the
widget’s DOM root.

Note: the use of an inline function is discouraged, and will probably be
removed sometimes in the future.

\end{fulllineitems}

\index{custom\_events (None attribute)}

\begin{fulllineitems}
\phantomsection\label{\detokenize{reference/javascript_reference:custom_events}}\pysigline{\sphinxbfcode{\sphinxupquote{custom\_events}}}
this is almost the same as the \sphinxstyleemphasis{events} attribute, but the keys
are arbitrary strings.  They represent business events triggered by
some sub widgets.  When an event is triggered, it will ‘bubble up’ the widget
tree (see the section on component communication for more details).

\end{fulllineitems}

\index{Widget.isDestroyed() (Widget method)}

\begin{fulllineitems}
\phantomsection\label{\detokenize{reference/javascript_reference:Widget.isDestroyed}}\pysiglinewithargsret{\sphinxcode{\sphinxupquote{Widget.}}\sphinxbfcode{\sphinxupquote{isDestroyed}}}{}{}~\begin{quote}\begin{description}
\item[{Returns}] \leavevmode
\sphinxcode{\sphinxupquote{true}} if the widget is being or has been destroyed, \sphinxcode{\sphinxupquote{false}}
otherwise

\end{description}\end{quote}

\end{fulllineitems}

\index{Widget.\$() (Widget method)}

\begin{fulllineitems}
\phantomsection\label{\detokenize{reference/javascript_reference:Widget._S_}}\pysiglinewithargsret{\sphinxcode{\sphinxupquote{Widget.}}\sphinxbfcode{\sphinxupquote{\$}}}{\emph{selector}}{}
Applies the CSS selector specified as parameter to the widget’s
DOM root:

\fvset{hllines={, ,}}%
\begin{sphinxVerbatim}[commandchars=\\\{\}]
\PYG{k}{this}\PYG{p}{.}\PYG{n+nx}{\PYGZdl{}}\PYG{p}{(}\PYG{n+nx}{selector}\PYG{p}{)}\PYG{p}{;}
\end{sphinxVerbatim}

is functionally identical to:

\fvset{hllines={, ,}}%
\begin{sphinxVerbatim}[commandchars=\\\{\}]
\PYG{k}{this}\PYG{p}{.}\PYG{n+nx}{\PYGZdl{}el}\PYG{p}{.}\PYG{n+nx}{find}\PYG{p}{(}\PYG{n+nx}{selector}\PYG{p}{)}\PYG{p}{;}
\end{sphinxVerbatim}
\begin{quote}\begin{description}
\item[{Arguments}] \leavevmode\begin{itemize}
\item {} 
\sphinxstyleliteralstrong{\sphinxupquote{selector}} (\sphinxstyleliteralemphasis{\sphinxupquote{String}}) \textendash{} CSS selector

\end{itemize}

\item[{Returns}] \leavevmode
jQuery object

\end{description}\end{quote}

\begin{sphinxadmonition}{note}{Note:}
this helper method is similar to \sphinxcode{\sphinxupquote{Backbone.View.\$}}
\end{sphinxadmonition}

\end{fulllineitems}

\index{Widget.setElement() (Widget method)}

\begin{fulllineitems}
\phantomsection\label{\detokenize{reference/javascript_reference:Widget.setElement}}\pysiglinewithargsret{\sphinxcode{\sphinxupquote{Widget.}}\sphinxbfcode{\sphinxupquote{setElement}}}{\emph{element}}{}
Re-sets the widget’s DOM root to the provided element, also
handles re-setting the various aliases of the DOM root as well as
unsetting and re-setting delegated events.
\begin{quote}\begin{description}
\item[{Arguments}] \leavevmode\begin{itemize}
\item {} 
\sphinxstyleliteralstrong{\sphinxupquote{element}} (\sphinxstyleliteralemphasis{\sphinxupquote{Element}}) \textendash{} a DOM element or jQuery object to set as
the widget’s DOM root

\end{itemize}

\end{description}\end{quote}

\end{fulllineitems}



\subsubsection{Inserting a widget in the DOM}
\label{\detokenize{reference/javascript_reference:inserting-a-widget-in-the-dom}}\index{Widget.appendTo() (Widget method)}

\begin{fulllineitems}
\phantomsection\label{\detokenize{reference/javascript_reference:Widget.appendTo}}\pysiglinewithargsret{\sphinxcode{\sphinxupquote{Widget.}}\sphinxbfcode{\sphinxupquote{appendTo}}}{\emph{element}}{}
Renders the widget and inserts it as the last child of the target, uses
\sphinxhref{http://api.jquery.com/appendTo/}{.appendTo()}

\end{fulllineitems}

\index{Widget.prependTo() (Widget method)}

\begin{fulllineitems}
\phantomsection\label{\detokenize{reference/javascript_reference:Widget.prependTo}}\pysiglinewithargsret{\sphinxcode{\sphinxupquote{Widget.}}\sphinxbfcode{\sphinxupquote{prependTo}}}{\emph{element}}{}
Renders the widget and inserts it as the first child of the target, uses
\sphinxhref{http://api.jquery.com/prependTo/}{.prependTo()}

\end{fulllineitems}

\index{Widget.insertAfter() (Widget method)}

\begin{fulllineitems}
\phantomsection\label{\detokenize{reference/javascript_reference:Widget.insertAfter}}\pysiglinewithargsret{\sphinxcode{\sphinxupquote{Widget.}}\sphinxbfcode{\sphinxupquote{insertAfter}}}{\emph{element}}{}
Renders the widget and inserts it as the preceding sibling of the target,
uses \sphinxhref{http://api.jquery.com/insertAfter/}{.insertAfter()}

\end{fulllineitems}

\index{Widget.insertBefore() (Widget method)}

\begin{fulllineitems}
\phantomsection\label{\detokenize{reference/javascript_reference:Widget.insertBefore}}\pysiglinewithargsret{\sphinxcode{\sphinxupquote{Widget.}}\sphinxbfcode{\sphinxupquote{insertBefore}}}{\emph{element}}{}
Renders the widget and inserts it as the following sibling of the target,
uses \sphinxhref{http://api.jquery.com/insertBefore/}{.insertBefore()}

\end{fulllineitems}


All of these methods accept whatever the corresponding jQuery method accepts
(CSS selectors, DOM nodes or jQuery objects). They all return a \sphinxhref{http://api.jquery.com/category/deferred-object/}{deferred}
and are charged with three tasks:
\begin{itemize}
\item {} \begin{description}
\item[{rendering the widget’s root element via}] \leavevmode
\sphinxcode{\sphinxupquote{renderElement()}}

\end{description}

\item {} \begin{description}
\item[{inserting the widget’s root element in the DOM using whichever jQuery}] \leavevmode
method they match

\end{description}

\item {} 
starting the widget, and returning the result of starting it

\end{itemize}


\subsubsection{Widget Guidelines}
\label{\detokenize{reference/javascript_reference:widget-guidelines}}\begin{itemize}
\item {} \begin{description}
\item[{Identifiers (\sphinxcode{\sphinxupquote{id}} attribute) should be avoided. In generic applications}] \leavevmode
and modules, \sphinxcode{\sphinxupquote{id}} limits the re-usability of components and tends to make
code more brittle. Most of the time, they can be replaced with nothing,
classes or keeping a reference to a DOM node or jQuery element.

If an \sphinxcode{\sphinxupquote{id}} is absolutely necessary (because a third-party library requires
one), the id should be partially generated using \sphinxcode{\sphinxupquote{\_.uniqueId()}} e.g.:

\fvset{hllines={, ,}}%
\begin{sphinxVerbatim}[commandchars=\\\{\}]
\PYG{k}{this}\PYG{p}{.}\PYG{n+nx}{id} \PYG{o}{=} \PYG{n+nx}{\PYGZus{}}\PYG{p}{.}\PYG{n+nx}{uniqueId}\PYG{p}{(}\PYG{l+s+s1}{\PYGZsq{}my\PYGZhy{}widget\PYGZhy{}\PYGZsq{}}\PYG{p}{)}\PYG{p}{;}
\end{sphinxVerbatim}

\end{description}

\item {} 
Avoid predictable/common CSS class names. Class names such as “content” or
“navigation” might match the desired meaning/semantics, but it is likely an
other developer will have the same need, creating a naming conflict and
unintended behavior. Generic class names should be prefixed with e.g. the
name of the component they belong to (creating “informal” namespaces, much
as in C or Objective-C).

\item {} 
Global selectors should be avoided. Because a component may be used several
times in a single page (an example in Odoo is dashboards), queries should be
restricted to a given component’s scope. Unfiltered selections such as
\sphinxcode{\sphinxupquote{\$(selector)}} or \sphinxcode{\sphinxupquote{document.querySelectorAll(selector)}} will generally
lead to unintended or incorrect behavior.  Odoo Web’s
{\hyperref[\detokenize{reference/javascript_api:Widget}]{\sphinxcrossref{\sphinxcode{\sphinxupquote{Widget()}}}}} has an attribute providing its DOM root
({\hyperref[\detokenize{reference/javascript_reference:Widget._S_el}]{\sphinxcrossref{\sphinxcode{\sphinxupquote{\$el}}}}}), and a shortcut to select nodes directly
({\hyperref[\detokenize{reference/javascript_reference:Widget._S_}]{\sphinxcrossref{\sphinxcode{\sphinxupquote{\$()}}}}}).

\item {} 
More generally, never assume your components own or controls anything beyond
its own personal {\hyperref[\detokenize{reference/javascript_reference:Widget._S_el}]{\sphinxcrossref{\sphinxcode{\sphinxupquote{\$el}}}}} (so, avoid using a reference to the
parent widget)

\item {} 
Html templating/rendering should use QWeb unless absolutely trivial.

\item {} 
All interactive components (components displaying information to the screen
or intercepting DOM events) must inherit from {\hyperref[\detokenize{reference/javascript_api:Widget}]{\sphinxcrossref{\sphinxcode{\sphinxupquote{Widget()}}}}}
and correctly implement and use its API and life cycle.

\end{itemize}


\subsection{QWeb Template Engine}
\label{\detokenize{reference/javascript_reference:reference-javascript-reference-qweb}}\label{\detokenize{reference/javascript_reference:qweb-template-engine}}
The web client uses the {\hyperref[\detokenize{reference/qweb::doc}]{\sphinxcrossref{\DUrole{doc}{QWeb}}}} template engine to render widgets (unless they
override the \sphinxstyleemphasis{renderElement} method to do something else).
The Qweb JS template engine is based on XML, and is mostly compatible with the
python implementation.

Now, let us explain how the templates are loaded.  Whenever the web client
starts, a rpc is made to the \sphinxstyleemphasis{/web/webclient/qweb} route.  The server will then
return a list of all templates defined in data files for each installed modules.
The correct files are listed in the \sphinxstyleemphasis{qweb} entry in each module manifest.

The web client will wait for that list of template to be loaded, before starting
its first widget.

This mechanism works quite well for our needs, but sometimes, we want to lazy
load a template.  For example, imagine that we have a widget which is rarely
used.  In that case, maybe we prefer to not load its template in the main file,
in order to make the web client slightly lighter.  In that case, we can use the
\sphinxstyleemphasis{xmlDependencies} key of the Widget:

\fvset{hllines={, ,}}%
\begin{sphinxVerbatim}[commandchars=\\\{\}]
\PYG{k+kd}{var} \PYG{n+nx}{Widget} \PYG{o}{=} \PYG{n+nx}{require}\PYG{p}{(}\PYG{l+s+s1}{\PYGZsq{}web.Widget\PYGZsq{}}\PYG{p}{)}\PYG{p}{;}

\PYG{k+kd}{var} \PYG{n+nx}{Counter} \PYG{o}{=} \PYG{n+nx}{Widget}\PYG{p}{.}\PYG{n+nx}{extend}\PYG{p}{(}\PYG{p}{\PYGZob{}}
    \PYG{n+nx}{template}\PYG{o}{:} \PYG{l+s+s1}{\PYGZsq{}some.template\PYGZsq{}}\PYG{p}{,}
    \PYG{n+nx}{xmlDependencies}\PYG{o}{:} \PYG{p}{[}\PYG{l+s+s1}{\PYGZsq{}/myaddon/path/to/my/file.xml\PYGZsq{}}\PYG{p}{]}\PYG{p}{,}

    \PYG{p}{...}

\PYG{p}{\PYGZcb{}}\PYG{p}{)}\PYG{p}{;}
\end{sphinxVerbatim}

With this, the \sphinxstyleemphasis{Counter} widget will load the xmlDependencies files in its
\sphinxstyleemphasis{willStart} method, so the template will be ready when the rendering is performed.


\subsection{Event system}
\label{\detokenize{reference/javascript_reference:event-system}}
There are currently two event systems supported by Odoo: a simple system which
allows adding listeners and triggering events, and a more complete system that
also makes events ‘bubble up’.

Both of these event systems are implemented in the \sphinxstyleemphasis{EventDispatcherMixin}, in
the file \sphinxstyleemphasis{mixins.js}. This mixin is included in the \sphinxstyleemphasis{Widget} class.


\subsubsection{Base Event system}
\label{\detokenize{reference/javascript_reference:base-event-system}}
This event system was historically the first.  It implements a simple bus
pattern. We have 4 main methods:
\begin{itemize}
\item {} 
\sphinxstyleemphasis{on}: this is used to register a listener on an event.

\item {} 
\sphinxstyleemphasis{off}: useful to remove events listener.

\item {} 
\sphinxstyleemphasis{once}: this is used to register a listener that will only be called once.

\item {} 
\sphinxstyleemphasis{trigger}: trigger an event. This will cause each listeners to be called.

\end{itemize}

Here is an example on how this event system could be used:

\fvset{hllines={, ,}}%
\begin{sphinxVerbatim}[commandchars=\\\{\}]
\PYG{k+kd}{var} \PYG{n+nx}{Widget} \PYG{o}{=} \PYG{n+nx}{require}\PYG{p}{(}\PYG{l+s+s1}{\PYGZsq{}web.Widget\PYGZsq{}}\PYG{p}{)}\PYG{p}{;}
\PYG{k+kd}{var} \PYG{n+nx}{Counter} \PYG{o}{=} \PYG{n+nx}{require}\PYG{p}{(}\PYG{l+s+s1}{\PYGZsq{}myModule.Counter\PYGZsq{}}\PYG{p}{)}\PYG{p}{;}

\PYG{k+kd}{var} \PYG{n+nx}{MyWidget} \PYG{o}{=} \PYG{n+nx}{Widget}\PYG{p}{.}\PYG{n+nx}{extend}\PYG{p}{(}\PYG{p}{\PYGZob{}}
    \PYG{n+nx}{start}\PYG{o}{:} \PYG{k+kd}{function} \PYG{p}{(}\PYG{p}{)} \PYG{p}{\PYGZob{}}
        \PYG{k}{this}\PYG{p}{.}\PYG{n+nx}{counter} \PYG{o}{=} \PYG{k}{new} \PYG{n+nx}{Counter}\PYG{p}{(}\PYG{k}{this}\PYG{p}{)}\PYG{p}{;}
        \PYG{k}{this}\PYG{p}{.}\PYG{n+nx}{counter}\PYG{p}{.}\PYG{n+nx}{on}\PYG{p}{(}\PYG{l+s+s1}{\PYGZsq{}valuechange\PYGZsq{}}\PYG{p}{,} \PYG{k}{this}\PYG{p}{,} \PYG{k}{this}\PYG{p}{.}\PYG{n+nx}{\PYGZus{}onValueChange}\PYG{p}{)}\PYG{p}{;}
        \PYG{k+kd}{var} \PYG{n+nx}{def} \PYG{o}{=} \PYG{k}{this}\PYG{p}{.}\PYG{n+nx}{counter}\PYG{p}{.}\PYG{n+nx}{appendTo}\PYG{p}{(}\PYG{k}{this}\PYG{p}{.}\PYG{n+nx}{\PYGZdl{}el}\PYG{p}{)}\PYG{p}{;}
        \PYG{k}{return} \PYG{n+nx}{\PYGZdl{}}\PYG{p}{.}\PYG{n+nx}{when}\PYG{p}{(}\PYG{n+nx}{def}\PYG{p}{,} \PYG{k}{this}\PYG{p}{.}\PYG{n+nx}{\PYGZus{}super}\PYG{p}{.}\PYG{n+nx}{apply}\PYG{p}{(}\PYG{k}{this}\PYG{p}{,} \PYG{n+nx}{arguments}\PYG{p}{)}\PYG{p}{;}
    \PYG{p}{\PYGZcb{}}\PYG{p}{,}
    \PYG{n+nx}{\PYGZus{}onValueChange}\PYG{o}{:} \PYG{k+kd}{function} \PYG{p}{(}\PYG{n+nx}{val}\PYG{p}{)} \PYG{p}{\PYGZob{}}
        \PYG{c+c1}{// do something with val}
    \PYG{p}{\PYGZcb{}}\PYG{p}{,}
\PYG{p}{\PYGZcb{}}\PYG{p}{)}\PYG{p}{;}

\PYG{c+c1}{// in Counter widget, we need to call the trigger method:}

\PYG{p}{...} \PYG{k}{this}\PYG{p}{.}\PYG{n+nx}{trigger}\PYG{p}{(}\PYG{l+s+s1}{\PYGZsq{}valuechange\PYGZsq{}}\PYG{p}{,} \PYG{n+nx}{someValue}\PYG{p}{)}\PYG{p}{;}
\end{sphinxVerbatim}

\begin{sphinxadmonition}{warning}{Warning:}
the use of this event system is discouraged, we plan to replace each
\sphinxstyleemphasis{trigger} method by the \sphinxstyleemphasis{trigger\_up} method from the extended event system
\end{sphinxadmonition}


\subsubsection{Extended Event System}
\label{\detokenize{reference/javascript_reference:extended-event-system}}
The custom event widgets is a more advanced system, which mimic the DOM events
API.  Whenever an event is triggered, it will ‘bubble up’ the component tree,
until it reaches the root widget, or is stopped.
\begin{itemize}
\item {} 
\sphinxstyleemphasis{trigger\_up}: this is the method that will create a small \sphinxstyleemphasis{OdooEvent} and
dispatch it in the component tree.  Note that it will start with the component
that triggered the event

\item {} 
\sphinxstyleemphasis{custom\_events}: this is the equivalent of the \sphinxstyleemphasis{event} dictionary, but for
odoo events.

\end{itemize}

The OdooEvent class is very simple.  It has three public attributes: \sphinxstyleemphasis{target}
(the widget that triggered the event), \sphinxstyleemphasis{name} (the event name) and \sphinxstyleemphasis{data} (the
payload).  It also has 2 methods: \sphinxstyleemphasis{stopPropagation} and \sphinxstyleemphasis{is\_stopped}.

The previous example can be updated to use the custom event system:

\fvset{hllines={, ,}}%
\begin{sphinxVerbatim}[commandchars=\\\{\}]
\PYG{k+kd}{var} \PYG{n+nx}{Widget} \PYG{o}{=} \PYG{n+nx}{require}\PYG{p}{(}\PYG{l+s+s1}{\PYGZsq{}web.Widget\PYGZsq{}}\PYG{p}{)}\PYG{p}{;}
\PYG{k+kd}{var} \PYG{n+nx}{Counter} \PYG{o}{=} \PYG{n+nx}{require}\PYG{p}{(}\PYG{l+s+s1}{\PYGZsq{}myModule.Counter\PYGZsq{}}\PYG{p}{)}\PYG{p}{;}

\PYG{k+kd}{var} \PYG{n+nx}{MyWidget} \PYG{o}{=} \PYG{n+nx}{Widget}\PYG{p}{.}\PYG{n+nx}{extend}\PYG{p}{(}\PYG{p}{\PYGZob{}}
    \PYG{n+nx}{custom\PYGZus{}events}\PYG{o}{:} \PYG{p}{\PYGZob{}}
        \PYG{n+nx}{valuechange}\PYG{o}{:} \PYG{l+s+s1}{\PYGZsq{}\PYGZus{}onValueChange\PYGZsq{}}
    \PYG{p}{\PYGZcb{}}\PYG{p}{,}
    \PYG{n+nx}{start}\PYG{o}{:} \PYG{k+kd}{function} \PYG{p}{(}\PYG{p}{)} \PYG{p}{\PYGZob{}}
        \PYG{k}{this}\PYG{p}{.}\PYG{n+nx}{counter} \PYG{o}{=} \PYG{k}{new} \PYG{n+nx}{Counter}\PYG{p}{(}\PYG{k}{this}\PYG{p}{)}\PYG{p}{;}
        \PYG{k+kd}{var} \PYG{n+nx}{def} \PYG{o}{=} \PYG{k}{this}\PYG{p}{.}\PYG{n+nx}{counter}\PYG{p}{.}\PYG{n+nx}{appendTo}\PYG{p}{(}\PYG{k}{this}\PYG{p}{.}\PYG{n+nx}{\PYGZdl{}el}\PYG{p}{)}\PYG{p}{;}
        \PYG{k}{return} \PYG{n+nx}{\PYGZdl{}}\PYG{p}{.}\PYG{n+nx}{when}\PYG{p}{(}\PYG{n+nx}{def}\PYG{p}{,} \PYG{k}{this}\PYG{p}{.}\PYG{n+nx}{\PYGZus{}super}\PYG{p}{.}\PYG{n+nx}{apply}\PYG{p}{(}\PYG{k}{this}\PYG{p}{,} \PYG{n+nx}{arguments}\PYG{p}{)}\PYG{p}{;}
    \PYG{p}{\PYGZcb{}}\PYG{p}{,}
    \PYG{n+nx}{\PYGZus{}onValueChange}\PYG{o}{:} \PYG{k+kd}{function}\PYG{p}{(}\PYG{n+nx}{event}\PYG{p}{)} \PYG{p}{\PYGZob{}}
        \PYG{c+c1}{// do something with event.data.val}
    \PYG{p}{\PYGZcb{}}\PYG{p}{,}
\PYG{p}{\PYGZcb{}}\PYG{p}{)}\PYG{p}{;}

\PYG{c+c1}{// in Counter widget, we need to call the trigger\PYGZus{}up method:}

\PYG{p}{...} \PYG{k}{this}\PYG{p}{.}\PYG{n+nx}{trigger\PYGZus{}up}\PYG{p}{(}\PYG{l+s+s1}{\PYGZsq{}valuechange\PYGZsq{}}\PYG{p}{,} \PYG{p}{\PYGZob{}}\PYG{n+nx}{value}\PYG{o}{:} \PYG{n+nx}{someValue}\PYG{p}{\PYGZcb{}}\PYG{p}{)}\PYG{p}{;}
\end{sphinxVerbatim}


\subsection{Registries}
\label{\detokenize{reference/javascript_reference:registries}}
A common need in the Odoo ecosystem is to extend/change the behaviour of the
base system from the outside (by installing an application, i.e. a different
module).  For example, one may need to add a new widget type in some views.  In
that case, and many others, the usual process is to create the desired component,
then add it to a registry (registering step), to make the rest of the web client
aware of its existence.

There are a few registries available in the system:
\begin{itemize}
\item {} \begin{description}
\item[{field registry (exported by ‘web.field\_registry’). The field registry contains}] \leavevmode
all field widgets known to the web client.  Whenever a view (typically, form,
or list/kanban) needs a field widget, this is where it will look. A typical
use case look like this:

\fvset{hllines={, ,}}%
\begin{sphinxVerbatim}[commandchars=\\\{\}]
\PYG{k+kd}{var} \PYG{n+nx}{fieldRegistry} \PYG{o}{=} \PYG{n+nx}{require}\PYG{p}{(}\PYG{l+s+s1}{\PYGZsq{}web.field\PYGZus{}registry\PYGZsq{}}\PYG{p}{)}\PYG{p}{;}

\PYG{k+kd}{var} \PYG{n+nx}{FieldPad} \PYG{o}{=} \PYG{p}{...}\PYG{p}{;}

\PYG{n+nx}{fieldRegistry}\PYG{p}{.}\PYG{n+nx}{add}\PYG{p}{(}\PYG{l+s+s1}{\PYGZsq{}pad\PYGZsq{}}\PYG{p}{,} \PYG{n+nx}{FieldPad}\PYG{p}{)}\PYG{p}{;}
\end{sphinxVerbatim}

Note that each value should be a subclass of \sphinxstyleemphasis{AbstractField}

\end{description}

\item {} \begin{description}
\item[{view registry: this registry contains all JS views known to the web client}] \leavevmode
(and in particular, the view manager).  Each value of this registry should
be a subclass of \sphinxstyleemphasis{AbstractView}

\end{description}

\item {} \begin{description}
\item[{action registry: we keep track of all client actions in this registry.  This}] \leavevmode
is where the action manager looks up whenever it needs to create a client
action.  In version 11, each value should simply be a subclass of \sphinxstyleemphasis{Widget}.
However, in version 12, the values are required to be \sphinxstyleemphasis{AbstractAction}.

\end{description}

\end{itemize}


\subsection{Communication between widgets}
\label{\detokenize{reference/javascript_reference:communication-between-widgets}}
There are many ways to communicate between components.
\begin{itemize}
\item {} \begin{description}
\item[{From a parent to its child:}] \leavevmode
this is a simple case. The parent widget can simply call a method on its
child:

\fvset{hllines={, ,}}%
\begin{sphinxVerbatim}[commandchars=\\\{\}]
\PYG{k}{this}\PYG{p}{.}\PYG{n+nx}{someWidget}\PYG{p}{.}\PYG{n+nx}{update}\PYG{p}{(}\PYG{n+nx}{someInfo}\PYG{p}{)}\PYG{p}{;}
\end{sphinxVerbatim}

\end{description}

\item {} \begin{description}
\item[{From a widget to its parent/some ancestor:}] \leavevmode
in this case, the widget’s job is simply to notify its environment that
something happened.  Since we do not want the widget to have a reference to
its parent (this would couple the widget with its parent’s implementation),
the best way to proceed is usually to trigger an event, which will bubble up
the component tree, by using the \sphinxcode{\sphinxupquote{trigger\_up}} method:

\fvset{hllines={, ,}}%
\begin{sphinxVerbatim}[commandchars=\\\{\}]
\PYG{k}{this}\PYG{p}{.}\PYG{n+nx}{trigger\PYGZus{}up}\PYG{p}{(}\PYG{l+s+s1}{\PYGZsq{}open\PYGZus{}record\PYGZsq{}}\PYG{p}{,} \PYG{p}{\PYGZob{}} \PYG{n+nx}{record}\PYG{o}{:} \PYG{n+nx}{record}\PYG{p}{,} \PYG{n+nx}{id}\PYG{o}{:} \PYG{n+nx}{id}\PYG{p}{\PYGZcb{}}\PYG{p}{)}\PYG{p}{;}
\end{sphinxVerbatim}

This event will be triggered on the widget, then will bubble up and be
eventually caught by some upstream widget:

\fvset{hllines={, ,}}%
\begin{sphinxVerbatim}[commandchars=\\\{\}]
\PYG{k+kd}{var} \PYG{n+nx}{SomeAncestor} \PYG{o}{=} \PYG{n+nx}{Widget}\PYG{p}{.}\PYG{n+nx}{extend}\PYG{p}{(}\PYG{p}{\PYGZob{}}
    \PYG{n+nx}{custom\PYGZus{}events}\PYG{o}{:} \PYG{p}{\PYGZob{}}
        \PYG{l+s+s1}{\PYGZsq{}open\PYGZus{}record\PYGZsq{}}\PYG{o}{:} \PYG{l+s+s1}{\PYGZsq{}\PYGZus{}onOpenRecord\PYGZsq{}}\PYG{p}{,}
    \PYG{p}{\PYGZcb{}}\PYG{p}{,}
    \PYG{n+nx}{\PYGZus{}onOpenRecord}\PYG{o}{:} \PYG{k+kd}{function} \PYG{p}{(}\PYG{n+nx}{event}\PYG{p}{)} \PYG{p}{\PYGZob{}}
        \PYG{k+kd}{var} \PYG{n+nx}{record} \PYG{o}{=} \PYG{n+nx}{event}\PYG{p}{.}\PYG{n+nx}{data}\PYG{p}{.}\PYG{n+nx}{record}\PYG{p}{;}
        \PYG{k+kd}{var} \PYG{n+nx}{id} \PYG{o}{=} \PYG{n+nx}{event}\PYG{p}{.}\PYG{n+nx}{data}\PYG{p}{.}\PYG{n+nx}{id}\PYG{p}{;}
        \PYG{c+c1}{// do something with the event.}
    \PYG{p}{\PYGZcb{}}\PYG{p}{,}
\PYG{p}{\PYGZcb{}}\PYG{p}{)}\PYG{p}{;}
\end{sphinxVerbatim}

\end{description}

\item {} \begin{description}
\item[{Cross component:}] \leavevmode
Cross component communication can be achieved by using a bus.  This is not
the preferred form of communication, because it has the disadvantage of
making the code harder to maintain.  However, it has the advantage of
decoupling the components.  In that case, this is simply done by triggering
and listening to events on a bus.  For example:

\fvset{hllines={, ,}}%
\begin{sphinxVerbatim}[commandchars=\\\{\}]
\PYG{c+c1}{// in WidgetA}
\PYG{k+kd}{var} \PYG{n+nx}{core} \PYG{o}{=} \PYG{n+nx}{require}\PYG{p}{(}\PYG{l+s+s1}{\PYGZsq{}web.core\PYGZsq{}}\PYG{p}{)}\PYG{p}{;}

\PYG{k+kd}{var} \PYG{n+nx}{WidgetA} \PYG{o}{=} \PYG{n+nx}{Widget}\PYG{p}{.}\PYG{n+nx}{extend}\PYG{p}{(}\PYG{p}{\PYGZob{}}
    \PYG{p}{...}
    \PYG{n+nx}{start}\PYG{o}{:} \PYG{k+kd}{function} \PYG{p}{(}\PYG{p}{)} \PYG{p}{\PYGZob{}}
        \PYG{n+nx}{core}\PYG{p}{.}\PYG{n+nx}{bus}\PYG{p}{.}\PYG{n+nx}{on}\PYG{p}{(}\PYG{l+s+s1}{\PYGZsq{}barcode\PYGZus{}scanned\PYGZsq{}}\PYG{p}{,} \PYG{k}{this}\PYG{p}{,} \PYG{k}{this}\PYG{p}{.}\PYG{n+nx}{\PYGZus{}onBarcodeScanned}\PYG{p}{)}\PYG{p}{;}
    \PYG{p}{\PYGZcb{}}\PYG{p}{,}
\PYG{p}{\PYGZcb{}}\PYG{p}{)}\PYG{p}{;}

\PYG{c+c1}{// in WidgetB}
\PYG{k+kd}{var} \PYG{n+nx}{WidgetB} \PYG{o}{=} \PYG{n+nx}{Widget}\PYG{p}{.}\PYG{n+nx}{extend}\PYG{p}{(}\PYG{p}{\PYGZob{}}
    \PYG{p}{...}
    \PYG{n+nx}{someFunction}\PYG{o}{:} \PYG{k+kd}{function} \PYG{p}{(}\PYG{n+nx}{barcode}\PYG{p}{)} \PYG{p}{\PYGZob{}}
        \PYG{n+nx}{core}\PYG{p}{.}\PYG{n+nx}{bus}\PYG{p}{.}\PYG{n+nx}{trigger}\PYG{p}{(}\PYG{l+s+s1}{\PYGZsq{}barcode\PYGZus{}scanned\PYGZsq{}}\PYG{p}{,} \PYG{n+nx}{barcode}\PYG{p}{)}\PYG{p}{;}
    \PYG{p}{\PYGZcb{}}\PYG{p}{,}
\PYG{p}{\PYGZcb{}}\PYG{p}{)}\PYG{p}{;}
\end{sphinxVerbatim}

In this example, we use the bus exported by \sphinxstyleemphasis{web.core}, but this is not
required. A bus could be created for a specific purpose.

\end{description}

\end{itemize}


\subsection{Services}
\label{\detokenize{reference/javascript_reference:services}}
In version 11.0, we introduced the notion of \sphinxstyleemphasis{service}.  The main idea is to
give to sub components a controlled way to access their environment, in a way
that allow the framework enough control, and which is testable.

The service system is organized around three ideas: services, service providers
and widgets.  The way it works is that widgets trigger (with \sphinxstyleemphasis{trigger\_up})
events, these events bubble up to a service provider, which will ask a service
to perform a task, then maybe return an answer.


\subsubsection{Service}
\label{\detokenize{reference/javascript_reference:service}}
A service is an instance of the \sphinxstyleemphasis{AbstractService} class.  It basically only has
a name and a few methods.  Its job is to perform some work, typically something
depending on the environment.

For example, we have the \sphinxstyleemphasis{ajax} service (job is to perform a rpc), the
\sphinxstyleemphasis{localStorage} (interact with the browser local storage) and many others.

Here is a simplified example on how the ajax service is implemented:

\fvset{hllines={, ,}}%
\begin{sphinxVerbatim}[commandchars=\\\{\}]
\PYG{k+kd}{var} \PYG{n+nx}{AbstractService} \PYG{o}{=} \PYG{n+nx}{require}\PYG{p}{(}\PYG{l+s+s1}{\PYGZsq{}web.AbstractService\PYGZsq{}}\PYG{p}{)}\PYG{p}{;}

\PYG{k+kd}{var} \PYG{n+nx}{AjaxService} \PYG{o}{=} \PYG{n+nx}{AbstractService}\PYG{p}{.}\PYG{n+nx}{extend}\PYG{p}{(}\PYG{p}{\PYGZob{}}
    \PYG{n+nx}{name}\PYG{o}{:} \PYG{l+s+s1}{\PYGZsq{}ajax\PYGZsq{}}\PYG{p}{,}
    \PYG{n+nx}{rpc}\PYG{o}{:} \PYG{k+kd}{function} \PYG{p}{(}\PYG{p}{...}\PYG{p}{)} \PYG{p}{\PYGZob{}}
        \PYG{k}{return} \PYG{p}{...}\PYG{p}{;}
    \PYG{p}{\PYGZcb{}}\PYG{p}{,}
\PYG{p}{\PYGZcb{}}\PYG{p}{)}\PYG{p}{;}
\end{sphinxVerbatim}

This service is named ‘ajax’ and define one method, \sphinxstyleemphasis{rpc}.


\subsubsection{Service Provider}
\label{\detokenize{reference/javascript_reference:service-provider}}
For services to work, it is necessary that we have a service provider ready to
dispatch the custom events.  In the \sphinxstyleemphasis{backend} (web client), this is done by the
main web client instance. Note that the code for the service provider comes from
the \sphinxstyleemphasis{ServiceProviderMixin}.


\subsubsection{Widget}
\label{\detokenize{reference/javascript_reference:widget}}
The widget is the part that requests a service.  In order to do that, it simply
triggers an event \sphinxstyleemphasis{call\_service} (typically by using the helper function \sphinxstyleemphasis{call}).
This event will bubble up and communicate the intent to the rest of the system.

In practice, some functions are so frequently called that we have some helpers
functions to make them easier to use. For example, the \sphinxstyleemphasis{\_rpc} method is a helper
that helps making a rpc.

\fvset{hllines={, ,}}%
\begin{sphinxVerbatim}[commandchars=\\\{\}]
\PYG{k+kd}{var} \PYG{n+nx}{SomeWidget} \PYG{o}{=} \PYG{n+nx}{Widget}\PYG{p}{.}\PYG{n+nx}{extend}\PYG{p}{(}\PYG{p}{\PYGZob{}}
    \PYG{n+nx}{\PYGZus{}getActivityModelViewID}\PYG{o}{:} \PYG{k+kd}{function} \PYG{p}{(}\PYG{n+nx}{model}\PYG{p}{)} \PYG{p}{\PYGZob{}}
        \PYG{k}{return} \PYG{k}{this}\PYG{p}{.}\PYG{n+nx}{\PYGZus{}rpc}\PYG{p}{(}\PYG{p}{\PYGZob{}}
            \PYG{n+nx}{model}\PYG{o}{:} \PYG{n+nx}{model}\PYG{p}{,}
            \PYG{n+nx}{method}\PYG{o}{:} \PYG{l+s+s1}{\PYGZsq{}get\PYGZus{}activity\PYGZus{}view\PYGZus{}id\PYGZsq{}}
        \PYG{p}{\PYGZcb{}}\PYG{p}{)}\PYG{p}{;}
    \PYG{p}{\PYGZcb{}}\PYG{p}{,}
\PYG{p}{\PYGZcb{}}\PYG{p}{)}\PYG{p}{;}
\end{sphinxVerbatim}

\begin{sphinxadmonition}{warning}{Warning:}
If a widget is destroyed, it will be detached from the main component tree
and will not have a parent.  In that case, the events will not bubble up, which
means that the work will not be done.  This is usually exactly what we want from
a destroyed widget.
\end{sphinxadmonition}


\subsubsection{RPCs}
\label{\detokenize{reference/javascript_reference:rpcs}}
The rpc functionality is supplied by the ajax service.  But most people will
probably only interact with the \sphinxstyleemphasis{\_rpc} helpers.

There are typically two usecases when working on Odoo: one may need to call a
method on a (python) model (this goes through a controller \sphinxstyleemphasis{call\_kw}), or one
may need to directly call a controller (available on some route).
\begin{itemize}
\item {} 
Calling a method on a python model:

\end{itemize}

\fvset{hllines={, ,}}%
\begin{sphinxVerbatim}[commandchars=\\\{\}]
\PYG{k}{return} \PYG{k}{this}\PYG{p}{.}\PYG{n+nx}{\PYGZus{}rpc}\PYG{p}{(}\PYG{p}{\PYGZob{}}
    \PYG{n+nx}{model}\PYG{o}{:} \PYG{l+s+s1}{\PYGZsq{}some.model\PYGZsq{}}\PYG{p}{,}
    \PYG{n+nx}{method}\PYG{o}{:} \PYG{l+s+s1}{\PYGZsq{}some\PYGZus{}method\PYGZsq{}}\PYG{p}{,}
    \PYG{n+nx}{args}\PYG{o}{:} \PYG{p}{[}\PYG{n+nx}{some}\PYG{p}{,} \PYG{n+nx}{args}\PYG{p}{]}\PYG{p}{,}
\PYG{p}{\PYGZcb{}}\PYG{p}{)}\PYG{p}{;}
\end{sphinxVerbatim}
\begin{itemize}
\item {} 
Directly calling a controller:

\end{itemize}

\fvset{hllines={, ,}}%
\begin{sphinxVerbatim}[commandchars=\\\{\}]
\PYG{k}{return} \PYG{k}{this}\PYG{p}{.}\PYG{n+nx}{\PYGZus{}rpc}\PYG{p}{(}\PYG{p}{\PYGZob{}}
    \PYG{n+nx}{route}\PYG{o}{:} \PYG{l+s+s1}{\PYGZsq{}/some/route/\PYGZsq{}}\PYG{p}{,}
    \PYG{n+nx}{params}\PYG{o}{:} \PYG{p}{\PYGZob{}} \PYG{n+nx}{some}\PYG{o}{:} \PYG{n+nx}{kwargs}\PYG{p}{\PYGZcb{}}\PYG{p}{,}
\PYG{p}{\PYGZcb{}}\PYG{p}{)}\PYG{p}{;}
\end{sphinxVerbatim}


\subsection{Translation management}
\label{\detokenize{reference/javascript_reference:translation-management}}
Some translations are made on the server side (basically all text strings rendered or
processed by the server), but there are strings in the static files that need
to be translated.  The way it currently works is the following:
\begin{itemize}
\item {} 
each translatable string is tagged with the special function \sphinxstyleemphasis{\_t} (available in
the JS module \sphinxstyleemphasis{web.core}

\item {} 
these strings are used by the server to generate the proper PO files

\item {} 
whenever the web client is loaded, it will call the route \sphinxstyleemphasis{/web/webclient/translations},
which returns a list of all translatable terms

\item {} 
in runtime, whenever the function \sphinxstyleemphasis{\_t} is called, it will look up in this list
in order to find a translation, and return it or the original string if none
is found.

\end{itemize}

Note that translations are explained in more details, from the server point of
view, in the document {\hyperref[\detokenize{reference/translations::doc}]{\sphinxcrossref{\DUrole{doc}{Translating Modules}}}}.

There are two important functions for the translations in javascript: \sphinxstyleemphasis{\_t} and
\sphinxstyleemphasis{\_lt}.  The difference is that \sphinxstyleemphasis{\_lt} is lazily evaluated.

\fvset{hllines={, ,}}%
\begin{sphinxVerbatim}[commandchars=\\\{\}]
\PYG{k+kd}{var} \PYG{n+nx}{core} \PYG{o}{=} \PYG{n+nx}{require}\PYG{p}{(}\PYG{l+s+s1}{\PYGZsq{}web.core\PYGZsq{}}\PYG{p}{)}\PYG{p}{;}

\PYG{k+kd}{var} \PYG{n+nx}{\PYGZus{}t} \PYG{o}{=} \PYG{n+nx}{core}\PYG{p}{.}\PYG{n+nx}{\PYGZus{}t}\PYG{p}{;}
\PYG{k+kd}{var} \PYG{n+nx}{\PYGZus{}lt} \PYG{o}{=} \PYG{n+nx}{core}\PYG{p}{.}\PYG{n+nx}{\PYGZus{}lt}\PYG{p}{;}

\PYG{k+kd}{var} \PYG{n+nx}{SomeWidget} \PYG{o}{=} \PYG{n+nx}{Widget}\PYG{p}{.}\PYG{n+nx}{extend}\PYG{p}{(}\PYG{p}{\PYGZob{}}
    \PYG{n+nx}{exampleString}\PYG{o}{:} \PYG{n+nx}{\PYGZus{}lt}\PYG{p}{(}\PYG{l+s+s1}{\PYGZsq{}this should be translated\PYGZsq{}}\PYG{p}{)}\PYG{p}{,}
    \PYG{p}{...}
    \PYG{n+nx}{someMethod}\PYG{o}{:} \PYG{k+kd}{function} \PYG{p}{(}\PYG{p}{)} \PYG{p}{\PYGZob{}}
        \PYG{k+kd}{var} \PYG{n+nx}{str} \PYG{o}{=} \PYG{n+nx}{\PYGZus{}t}\PYG{p}{(}\PYG{l+s+s1}{\PYGZsq{}some text\PYGZsq{}}\PYG{p}{)}\PYG{p}{;}
        \PYG{p}{...}
    \PYG{p}{\PYGZcb{}}\PYG{p}{,}
\PYG{p}{\PYGZcb{}}\PYG{p}{)}\PYG{p}{;}
\end{sphinxVerbatim}

In this example, the \sphinxstyleemphasis{\_lt} is necessary because the translations are not ready
when the module is loaded.

Note that translation functions need some care.  The string given in argument
should not be dynamic.


\subsection{Views}
\label{\detokenize{reference/javascript_reference:views}}
The word ‘view’ has more than one meaning. This section is about the design of
the javascript code of the views, not the structure of the \sphinxstyleemphasis{arch} or anything
else.

In 2017, Odoo replaced the previous view code with a new architecture.  The
main need was to separate the rendering logic from the model logic.

Views (in a generic sense) are now described with  4 pieces: a View, a
Controller, a Renderer and a Model.  The API of these 4 pieces is described in
the AbstractView, AbstractController, AbstractRenderer and AbstractModel classes.


\begin{itemize}
\item {} 
the View is the factory. Its job is to get a set of fields, arch, context and
some other parameters, then to construct a Controller/Renderer/Model triplet.

The view’s role is to properly setup each piece of the MVC pattern, with the correct
information.  Usually, it has to process the arch string and extract the
data necessary for each other parts of the view.

Note that the view is a class, not a widget.  Once its job has been done, it
can be discarded.

\item {} 
the Renderer has one job: representing the data being viewed in a DOM element.
Each view can render the data in a different way.  Also, it should listen on
appropriate user actions and notify its parent (the Controller) if necessary.

The Renderer is the V in the MVC pattern.

\item {} 
the Model: its job is to fetch and hold the state of the view.  Usually, it
represents in some way a set of records in the database.  The Model is the
owner of the ‘business data’. It is the M in the MVC pattern.

\item {} 
the Controller: its job is to coordinate the renderer and the model.  Also, it
is the main entry point for the rest of the web client.  For example, when
the user changes something in the search view, the \sphinxstyleemphasis{update} method of the
controller will be called with the appropriate information.

It is the C in the MVC pattern.

\end{itemize}

\begin{sphinxadmonition}{note}{Note:}
The JS code for the views has been designed to be usable outside of the
context of a view manager/action manager.  They could be used in a client action,
or, they could be displayed in the public website (with some work on the assets).
\end{sphinxadmonition}


\subsection{Field Widgets}
\label{\detokenize{reference/javascript_reference:field-widgets}}
A good part of the web client experience is about editing and creating data. Most
of that work is done with the help of field widgets, which are aware of the field
type and of the specific details on how a value should be displayed and edited.


\subsubsection{AbstractField}
\label{\detokenize{reference/javascript_reference:abstractfield}}
The \sphinxstyleemphasis{AbstractField} class is the base class for all widgets in a view, for all
views that support them (currently: Form, List, Kanban).

There are many differences between the v11 field widgets and the previous versions.
Let us mention the most important ones:
\begin{itemize}
\item {} 
the widgets are shared between all views (well, Form/List/Kanban). No need to
duplicate the implementation anymore.  Note that it is possible to have a
specialized version of a widget for a view, by prefixing it with the view name
in the view registry: \sphinxstyleemphasis{list.many2one} will be chosen in priority over \sphinxstyleemphasis{many2one}.

\item {} 
the widgets are no longer the owner of the field value.  They only represent
the data and communicate with the rest of the view.

\item {} 
the widgets do no longer need to be able to switch between edit and readonly
mode.  Now, when such a change is necessary, the widget will be destroyed and
rerendered again.  It is not a problem, since they do not own their value
anyway

\item {} 
the field widgets can be used outside of a view.  Their API is slightly
awkward, but they are designed to be standalone.

\end{itemize}


\subsubsection{Non relational fields}
\label{\detokenize{reference/javascript_reference:non-relational-fields}}
We document here all non relational fields available by default, in no particular
order.
\begin{itemize}
\item {} \begin{description}
\item[{integer (FieldInteger)}] \leavevmode
This is the default field type for fields of type \sphinxstyleemphasis{integer}.
\begin{itemize}
\item {} 
Supported field types: \sphinxstyleemphasis{integer}

\end{itemize}

\end{description}

\item {} \begin{description}
\item[{float (FieldFloat)}] \leavevmode
This is the default field type for fields of type \sphinxstyleemphasis{float}.
\begin{itemize}
\item {} 
Supported field types: \sphinxstyleemphasis{float}

\end{itemize}

Attributes:
\begin{itemize}
\item {} 
digits: displayed precision

\end{itemize}

\fvset{hllines={, ,}}%
\begin{sphinxVerbatim}[commandchars=\\\{\}]
\PYG{n+nt}{\PYGZlt{}field} \PYG{n+na}{name=}\PYG{l+s}{\PYGZdq{}factor\PYGZdq{}} \PYG{n+na}{digits=}\PYG{l+s}{\PYGZdq{}[42,5]\PYGZdq{}}\PYG{n+nt}{/\PYGZgt{}}
\end{sphinxVerbatim}

\end{description}

\item {} \begin{description}
\item[{float\_time (FieldFloatTime)}] \leavevmode
The goal of this widget is to display properly a float value that represents
a time interval (in hours).  So, for example, 0.5 should be formatted as 0:30,
or 4.75 correspond to 4:45.
\begin{itemize}
\item {} 
Supported field types: \sphinxstyleemphasis{float}

\end{itemize}

\end{description}

\item {} \begin{description}
\item[{boolean (FieldBoolean)}] \leavevmode
This is the default field type for fields of type \sphinxstyleemphasis{boolean}.
\begin{itemize}
\item {} 
Supported field types: \sphinxstyleemphasis{boolean}

\end{itemize}

\end{description}

\item {} \begin{description}
\item[{char (FieldChar)}] \leavevmode
This is the default field type for fields of type \sphinxstyleemphasis{char}.
\begin{itemize}
\item {} 
Supported field types: \sphinxstyleemphasis{char}

\end{itemize}

\end{description}

\item {} \begin{description}
\item[{date (FieldDate)}] \leavevmode
This is the default field type for fields of type \sphinxstyleemphasis{date}. Note that it also
works with datetime fields.  It uses the session timezone when formatting
dates.
\begin{itemize}
\item {} 
Supported field types: \sphinxstyleemphasis{date}, \sphinxstyleemphasis{datetime}

\end{itemize}

\end{description}

\item {} \begin{description}
\item[{datetime (FieldDateTime)}] \leavevmode
This is the default field type for fields of type \sphinxstyleemphasis{datetime}.
\begin{itemize}
\item {} 
Supported field types: \sphinxstyleemphasis{date}, \sphinxstyleemphasis{datetime}

\end{itemize}

\end{description}

\item {} \begin{description}
\item[{monetary (FieldMonetary)}] \leavevmode
This is the default field type for fields of type ‘monetary’. It is used to
display a currency.  If there is a currency fields given in option, it will
use that, otherwise it will fall back to the default currency (in the session)
\begin{itemize}
\item {} 
Supported field types: \sphinxstyleemphasis{monetary}, \sphinxstyleemphasis{float}

\end{itemize}

Options:
\begin{itemize}
\item {} 
currency\_field: another field name which should be a many2one on currency.

\end{itemize}

\fvset{hllines={, ,}}%
\begin{sphinxVerbatim}[commandchars=\\\{\}]
\PYG{n+nt}{\PYGZlt{}field} \PYG{n+na}{name=}\PYG{l+s}{\PYGZdq{}value\PYGZdq{}} \PYG{n+na}{widget=}\PYG{l+s}{\PYGZdq{}monetary\PYGZdq{}} \PYG{n+na}{options=}\PYG{l+s}{\PYGZdq{}\PYGZob{}\PYGZsq{}currency\PYGZus{}field\PYGZsq{}: \PYGZsq{}currency\PYGZus{}id\PYGZsq{}\PYGZcb{}\PYGZdq{}}\PYG{n+nt}{/\PYGZgt{}}
\end{sphinxVerbatim}

\end{description}

\item {} \begin{description}
\item[{text (FieldText)}] \leavevmode
This is the default field type for fields of type \sphinxstyleemphasis{text}.
\begin{itemize}
\item {} 
Supported field types: \sphinxstyleemphasis{text}

\end{itemize}

\end{description}

\item {} \begin{description}
\item[{handle (HandleWidget)}] \leavevmode
This field’s job is to be displayed as a \sphinxstyleemphasis{handle} in a list view, and allow
reordering the various records by drag and dropping lines.
\begin{itemize}
\item {} 
Supported field types: \sphinxstyleemphasis{integer}

\end{itemize}

\end{description}

\item {} \begin{description}
\item[{email (FieldEmail)}] \leavevmode
This field displays email address.  The main reason to use it is that it
is rendered as an anchor tag with the proper href, in readonly mode.
\begin{itemize}
\item {} 
Supported field types: \sphinxstyleemphasis{char}

\end{itemize}

\end{description}

\item {} \begin{description}
\item[{phone (FieldPhone)}] \leavevmode
This field displays a phone number.  The main reason to use it is that it
is rendered as an anchor tag with the proper href, in readonly mode, but
only in some cases: we only want to make it clickable if the device can
call this particular number.
\begin{itemize}
\item {} 
Supported field types: \sphinxstyleemphasis{char}

\end{itemize}

\end{description}

\item {} \begin{description}
\item[{url (UrlWidget)}] \leavevmode
This field displays an url (in readonly mode). The main reason to use it is
that it is rendered as an anchor tag with the proper css classes and href.
\begin{itemize}
\item {} 
Supported field types: \sphinxstyleemphasis{char}

\end{itemize}

Also, the text of the anchor tag can be customized with the \sphinxstyleemphasis{text} attribute
(it won’t change the href value).

\fvset{hllines={, ,}}%
\begin{sphinxVerbatim}[commandchars=\\\{\}]
\PYG{n+nt}{\PYGZlt{}field} \PYG{n+na}{name=}\PYG{l+s}{\PYGZdq{}foo\PYGZdq{}} \PYG{n+na}{widget=}\PYG{l+s}{\PYGZdq{}url\PYGZdq{}} \PYG{n+na}{text=}\PYG{l+s}{\PYGZdq{}Some URL\PYGZdq{}}\PYG{n+nt}{/\PYGZgt{}}
\end{sphinxVerbatim}

\end{description}

\item {} \begin{description}
\item[{domain (FieldDomain)}] \leavevmode
The “Domain” field allows the user to construct a technical-prefix domain
thanks to a tree-like interface and see the selected records in real time.
In debug mode, an input is also there to be able to enter the prefix char
domain directly (or to build advanced domains the tree-like interface does
not allow to).

Note that this is limited to ‘static’ domain (no dynamic expression, or access
to context variable).
\begin{itemize}
\item {} 
Supported field types: \sphinxstyleemphasis{char}

\end{itemize}

\end{description}

\item {} \begin{description}
\item[{link\_button (LinkButton)}] \leavevmode
The LinkButton widget actually simply displays a span with an icon and the
text value as content. The link is clickable and will open a new browser
window with its value as url.
\begin{itemize}
\item {} 
Supported field types: \sphinxstyleemphasis{char}

\end{itemize}

\end{description}

\item {} \begin{description}
\item[{image (FieldBinaryImage)}] \leavevmode
This widget is used to represent a binary value as an image. In some cases,
the server returns a ‘bin\_size’ instead of the real image (a bin\_size is a
string representing a file size, such as 6.5kb).  In that case, the widget
will make an image with a source attribute corresponding to an image on the
server.
\begin{itemize}
\item {} 
Supported field types: \sphinxstyleemphasis{binary}

\end{itemize}

Options:
\begin{itemize}
\item {} 
preview\_image: if the image is only loaded as a ‘bin\_size’, then this
option is useful to inform the web client that the default field name is
not the name of the current field, but the name of another field.

\end{itemize}

\fvset{hllines={, ,}}%
\begin{sphinxVerbatim}[commandchars=\\\{\}]
\PYG{n+nt}{\PYGZlt{}field} \PYG{n+na}{name=}\PYG{l+s}{\PYGZdq{}image\PYGZdq{}} \PYG{n+na}{widget=}\PYG{l+s}{\PYGZsq{}image\PYGZsq{}} \PYG{n+na}{options=}\PYG{l+s}{\PYGZsq{}\PYGZob{}\PYGZdq{}preview\PYGZus{}image\PYGZdq{}:\PYGZdq{}image\PYGZus{}medium\PYGZdq{}\PYGZcb{}\PYGZsq{}}\PYG{n+nt}{/\PYGZgt{}}
\end{sphinxVerbatim}

\end{description}

\item {} \begin{description}
\item[{binary (FieldBinaryFile)}] \leavevmode
Generic widget to allow saving/downloading a binary file.
\begin{itemize}
\item {} 
Supported field types: \sphinxstyleemphasis{binary}

\end{itemize}

Attribute:
\begin{itemize}
\item {} 
filename: saving a binary file will lose its file name, since it only
saves the binary value. The filename can be saved in another field. To do
that, an attribute filename should be set to a field present in the view.

\end{itemize}

\fvset{hllines={, ,}}%
\begin{sphinxVerbatim}[commandchars=\\\{\}]
\PYG{n+nt}{\PYGZlt{}field} \PYG{n+na}{name=}\PYG{l+s}{\PYGZdq{}datas\PYGZdq{}} \PYG{n+na}{filename=}\PYG{l+s}{\PYGZdq{}datas\PYGZus{}fname\PYGZdq{}}\PYG{n+nt}{/\PYGZgt{}}
\end{sphinxVerbatim}

\end{description}

\item {} \begin{description}
\item[{priority (PriorityWidget)}] \leavevmode
This widget is rendered as a set of stars, allowing the user to click on it
to select a value or not. This is useful for example to mark a task as high
priority.

Note that this widget also works in ‘readonly’ mode, which is unusual.
\begin{itemize}
\item {} 
Supported field types: \sphinxstyleemphasis{selection}

\end{itemize}

\end{description}

\item {} \begin{description}
\item[{attachment\_image (AttachmentImage)}] \leavevmode
Image widget for many2one fields.  If the field is set, this widget will be
rendered as an image with the proper src url. This widget does not have a
different behaviour in edit or readonly mode, it is only useful to view an
image.
\begin{itemize}
\item {} 
Supported field types: \sphinxstyleemphasis{many2one}

\end{itemize}

\fvset{hllines={, ,}}%
\begin{sphinxVerbatim}[commandchars=\\\{\}]
\PYG{n+nt}{\PYGZlt{}field} \PYG{n+na}{name=}\PYG{l+s}{\PYGZdq{}displayed\PYGZus{}image\PYGZus{}id\PYGZdq{}} \PYG{n+na}{widget=}\PYG{l+s}{\PYGZdq{}attachment\PYGZus{}image\PYGZdq{}}\PYG{n+nt}{/\PYGZgt{}}
\end{sphinxVerbatim}

\end{description}

\item {} \begin{description}
\item[{image\_selection (ImageSelection)}] \leavevmode
Allow the user to select a value by clicking on an image.
\begin{itemize}
\item {} 
Supported field types: \sphinxstyleemphasis{selection}

\end{itemize}

Options: a dictionary with a mapping from a selection value to an object with
the url for an image (\sphinxstyleemphasis{image\_link}) and a preview image (\sphinxstyleemphasis{preview\_link}).

Note that this option is not optional!

\fvset{hllines={, ,}}%
\begin{sphinxVerbatim}[commandchars=\\\{\}]
\PYG{n+nt}{\PYGZlt{}field} \PYG{n+na}{name=}\PYG{l+s}{\PYGZdq{}external\PYGZus{}report\PYGZus{}layout\PYGZdq{}} \PYG{n+na}{widget=}\PYG{l+s}{\PYGZdq{}image\PYGZus{}selection\PYGZdq{}} \PYG{n+na}{options=}\PYG{l+s}{\PYGZdq{}\PYGZob{}}
\PYG{l+s}{    \PYGZsq{}background\PYGZsq{}: \PYGZob{}}
\PYG{l+s}{        \PYGZsq{}image\PYGZus{}link\PYGZsq{}: \PYGZsq{}/base/static/img/preview\PYGZus{}background.png\PYGZsq{},}
\PYG{l+s}{        \PYGZsq{}preview\PYGZus{}link\PYGZsq{}: \PYGZsq{}/base/static/pdf/preview\PYGZus{}background.pdf\PYGZsq{}}
\PYG{l+s}{    \PYGZcb{},}
\PYG{l+s}{    \PYGZsq{}standard\PYGZsq{}: \PYGZob{}}
\PYG{l+s}{        \PYGZsq{}image\PYGZus{}link\PYGZsq{}: \PYGZsq{}/base/static/img/preview\PYGZus{}standard.png\PYGZsq{},}
\PYG{l+s}{        \PYGZsq{}preview\PYGZus{}link\PYGZsq{}: \PYGZsq{}/base/static/pdf/preview\PYGZus{}standard.pdf\PYGZsq{}}
\PYG{l+s}{    \PYGZcb{}}
\PYG{l+s}{\PYGZcb{}\PYGZdq{}}\PYG{n+nt}{/\PYGZgt{}}
\end{sphinxVerbatim}

\end{description}

\item {} \begin{description}
\item[{label\_selection (LabelSelection)}] \leavevmode
This widget renders a simple non-editable label.  This is only useful to
display some information, not to edit it.
\begin{itemize}
\item {} 
Supported field types: \sphinxstyleemphasis{selection}

\end{itemize}

Options:
\begin{itemize}
\item {} 
classes: a mapping from a selection value to a css class

\end{itemize}

\fvset{hllines={, ,}}%
\begin{sphinxVerbatim}[commandchars=\\\{\}]
\PYG{n+nt}{\PYGZlt{}field} \PYG{n+na}{name=}\PYG{l+s}{\PYGZdq{}state\PYGZdq{}} \PYG{n+na}{widget=}\PYG{l+s}{\PYGZdq{}label\PYGZus{}selection\PYGZdq{}} \PYG{n+na}{options=}\PYG{l+s}{\PYGZdq{}\PYGZob{}}
\PYG{l+s}{    \PYGZsq{}classes\PYGZsq{}: \PYGZob{}\PYGZsq{}draft\PYGZsq{}: \PYGZsq{}default\PYGZsq{}, \PYGZsq{}cancel\PYGZsq{}: \PYGZsq{}default\PYGZsq{}, \PYGZsq{}none\PYGZsq{}: \PYGZsq{}danger\PYGZsq{}\PYGZcb{}}
\PYG{l+s}{\PYGZcb{}\PYGZdq{}}\PYG{n+nt}{/\PYGZgt{}}
\end{sphinxVerbatim}

\end{description}

\item {} \begin{description}
\item[{state\_selection (StateSelectionWidget)}] \leavevmode
This is a specialized selection widget. It assumes that the record has some
hardcoded fields, present in the view: \sphinxstyleemphasis{stage\_id}, \sphinxstyleemphasis{legend\_normal},
\sphinxstyleemphasis{legend\_blocked}, \sphinxstyleemphasis{legend\_done}.  This is mostly used to display and change
the state of a task in a project, with additional information displayed in
the dropdown.
\begin{itemize}
\item {} 
Supported field types: \sphinxstyleemphasis{selection}

\end{itemize}

\fvset{hllines={, ,}}%
\begin{sphinxVerbatim}[commandchars=\\\{\}]
\PYG{n+nt}{\PYGZlt{}field} \PYG{n+na}{name=}\PYG{l+s}{\PYGZdq{}kanban\PYGZus{}state\PYGZdq{}} \PYG{n+na}{widget=}\PYG{l+s}{\PYGZdq{}state\PYGZus{}selection\PYGZdq{}}\PYG{n+nt}{/\PYGZgt{}}
\end{sphinxVerbatim}

\end{description}

\item {} \begin{description}
\item[{kanban\_state\_selection (StateSelectionWidget)}] \leavevmode
This is exactly the same widget as state\_selection
\begin{itemize}
\item {} 
Supported field types: \sphinxstyleemphasis{selection}

\end{itemize}

\end{description}

\item {} \begin{description}
\item[{boolean\_favorite (FavoriteWidget)}] \leavevmode
This widget is displayed as an empty (or not) star, depending on a boolean
value. Note that it also can be edited in readonly mode.
\begin{itemize}
\item {} 
Supported field types: \sphinxstyleemphasis{boolean}

\end{itemize}

\end{description}

\item {} \begin{description}
\item[{boolean\_button (FieldBooleanButton)}] \leavevmode
The Boolean Button widget is meant to be used in a stat button in a form view.
The goal is to display a nice button with the current state of a boolean
field (for example, ‘Active’), and allow the user to change that field when
clicking on it.

Note that it also can be edited in readonly mode.
\begin{itemize}
\item {} 
Supported field types: \sphinxstyleemphasis{boolean}

\end{itemize}

Options:
\begin{itemize}
\item {} 
terminology: it can be either ‘active’, ‘archive’, ‘close’ or a customized
mapping with the keys \sphinxstyleemphasis{string\_true}, \sphinxstyleemphasis{string\_false}, \sphinxstyleemphasis{hover\_true}, \sphinxstyleemphasis{hover\_false}

\end{itemize}

\fvset{hllines={, ,}}%
\begin{sphinxVerbatim}[commandchars=\\\{\}]
\PYG{n+nt}{\PYGZlt{}field} \PYG{n+na}{name=}\PYG{l+s}{\PYGZdq{}active\PYGZdq{}} \PYG{n+na}{widget=}\PYG{l+s}{\PYGZdq{}boolean\PYGZus{}button\PYGZdq{}} \PYG{n+na}{options=}\PYG{l+s}{\PYGZsq{}\PYGZob{}\PYGZdq{}terminology\PYGZdq{}: \PYGZdq{}archive\PYGZdq{}\PYGZcb{}\PYGZsq{}}\PYG{n+nt}{/\PYGZgt{}}
\end{sphinxVerbatim}

\end{description}

\item {} \begin{description}
\item[{boolean\_toggle (BooleanToggle)}] \leavevmode
Displays a toggle switch to represent a boolean. This is a subfield of
FieldBoolean, mostly used to have a different look.

\end{description}

\item {} \begin{description}
\item[{statinfo (StatInfo)}] \leavevmode
This widget is meant to represent statistical information in a \sphinxstyleemphasis{stat button}.
It is basically just a label with a number.
\begin{itemize}
\item {} 
Supported field types: \sphinxstyleemphasis{integer, float}

\end{itemize}

Options:
\begin{itemize}
\item {} 
label\_field: if given, the widget will use the value of the label\_field as
text.

\end{itemize}

\fvset{hllines={, ,}}%
\begin{sphinxVerbatim}[commandchars=\\\{\}]
\PYG{n+nt}{\PYGZlt{}button} \PYG{n+na}{name=}\PYG{l+s}{\PYGZdq{}\PYGZpc{}(act\PYGZus{}payslip\PYGZus{}lines)d\PYGZdq{}}
    \PYG{n+na}{icon=}\PYG{l+s}{\PYGZdq{}fa\PYGZhy{}money\PYGZdq{}}
    \PYG{n+na}{type=}\PYG{l+s}{\PYGZdq{}action\PYGZdq{}}\PYG{n+nt}{\PYGZgt{}}
    \PYG{n+nt}{\PYGZlt{}field} \PYG{n+na}{name=}\PYG{l+s}{\PYGZdq{}payslip\PYGZus{}count\PYGZdq{}} \PYG{n+na}{widget=}\PYG{l+s}{\PYGZdq{}statinfo\PYGZdq{}}
        \PYG{n+na}{string=}\PYG{l+s}{\PYGZdq{}Payslip\PYGZdq{}}
        \PYG{n+na}{options=}\PYG{l+s}{\PYGZdq{}\PYGZob{}\PYGZsq{}label\PYGZus{}field\PYGZsq{}: \PYGZsq{}label\PYGZus{}tasks\PYGZsq{}\PYGZcb{}\PYGZdq{}}\PYG{n+nt}{/\PYGZgt{}}
\PYG{n+nt}{\PYGZlt{}/button\PYGZgt{}}
\end{sphinxVerbatim}

\end{description}

\item {} \begin{description}
\item[{percentpie (FieldPercentPie)}] \leavevmode
This widget is meant to represent statistical information in a \sphinxstyleemphasis{stat button}.
This is similar to a statinfo widget, but the information is represented in
a \sphinxstyleemphasis{pie} chart (empty to full).  Note that the value is interpreted as a
percentage (a number between 0 and 100).
\begin{itemize}
\item {} 
Supported field types: \sphinxstyleemphasis{integer, float}

\end{itemize}

\fvset{hllines={, ,}}%
\begin{sphinxVerbatim}[commandchars=\\\{\}]
\PYG{n+nt}{\PYGZlt{}field} \PYG{n+na}{name=}\PYG{l+s}{\PYGZdq{}replied\PYGZus{}ratio\PYGZdq{}} \PYG{n+na}{string=}\PYG{l+s}{\PYGZdq{}Replied\PYGZdq{}} \PYG{n+na}{widget=}\PYG{l+s}{\PYGZdq{}percentpie\PYGZdq{}}\PYG{n+nt}{/\PYGZgt{}}
\end{sphinxVerbatim}

\end{description}

\item {} \begin{description}
\item[{progressbar (FieldProgressBar)}] \leavevmode
Represent a value as a progress bar (from 0 to some value)
\begin{itemize}
\item {} 
Supported field types: \sphinxstyleemphasis{integer, float}

\end{itemize}

Options:
\begin{itemize}
\item {} 
title: title of the bar, displayed on top of the bar options

\item {} 
editable: boolean if value is editable

\item {} 
current\_value: get the current\_value from the field that must be present in the view

\item {} 
max\_value: get the max\_value from the field that must be present in the view

\item {} 
edit\_max\_value: boolean if the max\_value is editable

\item {} 
title: title of the bar, displayed on top of the bar \textendash{}\textgreater{} not translated,
use parameter (not option) “title” instead

\end{itemize}

\fvset{hllines={, ,}}%
\begin{sphinxVerbatim}[commandchars=\\\{\}]
\PYG{n+nt}{\PYGZlt{}field} \PYG{n+na}{name=}\PYG{l+s}{\PYGZdq{}absence\PYGZus{}of\PYGZus{}today\PYGZdq{}} \PYG{n+na}{widget=}\PYG{l+s}{\PYGZdq{}progressbar\PYGZdq{}}
    \PYG{n+na}{options=}\PYG{l+s}{\PYGZdq{}\PYGZob{}\PYGZsq{}current\PYGZus{}value\PYGZsq{}: \PYGZsq{}absence\PYGZus{}of\PYGZus{}today\PYGZsq{}, \PYGZsq{}max\PYGZus{}value\PYGZsq{}: \PYGZsq{}total\PYGZus{}employee\PYGZsq{}, \PYGZsq{}editable\PYGZsq{}: false\PYGZcb{}\PYGZdq{}}\PYG{n+nt}{/\PYGZgt{}}
\end{sphinxVerbatim}

\end{description}

\item {} \begin{description}
\item[{toggle\_button (FieldToggleBoolean)}] \leavevmode
This widget is intended to be used on boolean fields. It toggles a button
switching between a green bullet / gray bullet. It also set up a tooltip,
depending on the value and some options.
\begin{itemize}
\item {} 
Supported field types: \sphinxstyleemphasis{boolean}

\end{itemize}

Options:
\begin{itemize}
\item {} 
active: the string for the tooltip that should be set when boolean is true

\item {} 
inactive: the tooltip that should be set when boolean is false

\end{itemize}

\fvset{hllines={, ,}}%
\begin{sphinxVerbatim}[commandchars=\\\{\}]
\PYG{n+nt}{\PYGZlt{}field} \PYG{n+na}{name=}\PYG{l+s}{\PYGZdq{}payslip\PYGZus{}status\PYGZdq{}} \PYG{n+na}{widget=}\PYG{l+s}{\PYGZdq{}toggle\PYGZus{}button\PYGZdq{}}
    \PYG{n+na}{options=}\PYG{l+s}{\PYGZsq{}\PYGZob{}\PYGZdq{}active\PYGZdq{}: \PYGZdq{}Reported in last payslips\PYGZdq{}, \PYGZdq{}inactive\PYGZdq{}: \PYGZdq{}To Report in Payslip\PYGZdq{}\PYGZcb{}\PYGZsq{}}
\PYG{n+nt}{/\PYGZgt{}}
\end{sphinxVerbatim}

\end{description}

\item {} \begin{description}
\item[{dashboard\_graph (JournalDashboardGraph)}] \leavevmode
This is a more specialized widget, useful to display a graph representing a
set of data.  For example, it is used in the accounting dashboard kanban view.

It assumes that the field is a JSON serialization of a set of data.
\begin{itemize}
\item {} 
Supported field types: \sphinxstyleemphasis{char}

\end{itemize}

Attribute
\begin{itemize}
\item {} 
graph\_type: string, can be either ‘line’ or ‘bar’

\end{itemize}

\fvset{hllines={, ,}}%
\begin{sphinxVerbatim}[commandchars=\\\{\}]
\PYG{n+nt}{\PYGZlt{}field} \PYG{n+na}{name=}\PYG{l+s}{\PYGZdq{}dashboard\PYGZus{}graph\PYGZus{}data\PYGZdq{}}
    \PYG{n+na}{widget=}\PYG{l+s}{\PYGZdq{}dashboard\PYGZus{}graph\PYGZdq{}}
    \PYG{n+na}{graph\PYGZus{}type=}\PYG{l+s}{\PYGZdq{}line\PYGZdq{}}\PYG{n+nt}{/\PYGZgt{}}
\end{sphinxVerbatim}

\end{description}

\item {} \begin{description}
\item[{ace (AceEditor)}] \leavevmode
This widget is intended to be used on Text fields. It provides Ace Editor
for editing XML and Python.
\begin{itemize}
\item {} 
Supported field types: \sphinxstyleemphasis{char, text}

\end{itemize}

\end{description}

\end{itemize}


\subsubsection{Relational fields}
\label{\detokenize{reference/javascript_reference:relational-fields}}

\begin{fulllineitems}
\phantomsection\label{\detokenize{reference/javascript_reference:FieldSelection}}\pysiglinewithargsret{\sphinxbfcode{\sphinxupquote{class }}\sphinxbfcode{\sphinxupquote{FieldSelection}}}{}{}~\begin{quote}\begin{description}
\item[{Extends}] \leavevmode{\hyperref[\detokenize{reference/javascript_api:web.AbstractField.AbstractField}]{\sphinxcrossref{
AbstractField
}}}
\end{description}\end{quote}

The FieldSelection widget is a simple select tag with a dropdown menu to
allow the selection of a range of values.  It is designed to work with fields
of type ‘selection’ and ‘many2one’.

Supported field types: \sphinxstyleemphasis{selection}, \sphinxstyleemphasis{many2one}
\index{placeholder (None attribute)}

\begin{fulllineitems}
\phantomsection\label{\detokenize{reference/javascript_reference:placeholder}}\pysigline{\sphinxbfcode{\sphinxupquote{placeholder}}}
a string which is used to display some info when no value is selected

\end{fulllineitems}


\fvset{hllines={, ,}}%
\begin{sphinxVerbatim}[commandchars=\\\{\}]
\PYG{n+nt}{\PYGZlt{}field} \PYG{n+na}{name=}\PYG{l+s}{\PYGZdq{}tax\PYGZus{}id\PYGZdq{}} \PYG{n+na}{widget=}\PYG{l+s}{\PYGZdq{}selection\PYGZdq{}} \PYG{n+na}{placeholder=}\PYG{l+s}{\PYGZdq{}Select a tax\PYGZdq{}}\PYG{n+nt}{/\PYGZgt{}}
\end{sphinxVerbatim}

\end{fulllineitems}

\begin{itemize}
\item {} \begin{description}
\item[{radio (FieldRadio)}] \leavevmode
This is a subfield of FielSelection, but specialized to display all the
valid choices as radio buttons.

Note that if used on a many2one records, then more rpcs will be done to fetch
the name\_gets of the related records.
\begin{itemize}
\item {} 
Supported field types: \sphinxstyleemphasis{selection, many2one}

\end{itemize}

Options:
\begin{itemize}
\item {} 
horizontal: if true, radio buttons will be diplayed horizontally.

\end{itemize}

\fvset{hllines={, ,}}%
\begin{sphinxVerbatim}[commandchars=\\\{\}]
\PYG{n+nt}{\PYGZlt{}field} \PYG{n+na}{name=}\PYG{l+s}{\PYGZdq{}recommended\PYGZus{}activity\PYGZus{}type\PYGZus{}id\PYGZdq{}} \PYG{n+na}{widget=}\PYG{l+s}{\PYGZdq{}radio\PYGZdq{}}
    \PYG{n+na}{options=}\PYG{l+s}{\PYGZdq{}\PYGZob{}\PYGZsq{}horizontal\PYGZsq{}:true\PYGZcb{}\PYGZdq{}}\PYG{n+nt}{/\PYGZgt{}}
\end{sphinxVerbatim}

\end{description}

\item {} \begin{description}
\item[{many2one (FieldMany2One)}] \leavevmode
Default widget for many2one fields.
\begin{itemize}
\item {} 
Supported field types: \sphinxstyleemphasis{selection, many2one}

\end{itemize}

Attributes:
\begin{itemize}
\item {} 
can\_create: allow the creation of related records (take precedence over no\_create
option)

\item {} 
can\_write: allow the edition of related records (default: true)

\end{itemize}

Options:
\begin{itemize}
\item {} 
no\_create: prevent the creation of related records

\item {} 
quick\_create: allow the quick creation of related records (default: true)

\item {} 
no\_quick\_create: prevent the quick creation of related records (don’t ask me)

\item {} 
no\_create\_edit: same as no\_create, maybe…

\item {} 
create\_name\_field: when creating a related record, if this option is set, the value of the \sphinxstyleemphasis{create\_name\_field} will be filled with the value of the input (default: \sphinxstyleemphasis{name})

\item {} 
always\_reload: boolean, default to false.  If true, the widget will always
do an additional name\_get to fetch its name value.  This is used for the
situations where the name\_get method is overridden (please do not do that)

\item {} 
no\_open: boolean, default to false.  If set to true, the many2one will not
redirect on the record when clicking on it (in readonly mode)

\end{itemize}

\fvset{hllines={, ,}}%
\begin{sphinxVerbatim}[commandchars=\\\{\}]
\PYG{n+nt}{\PYGZlt{}field} \PYG{n+na}{name=}\PYG{l+s}{\PYGZdq{}currency\PYGZus{}id\PYGZdq{}} \PYG{n+na}{options=}\PYG{l+s}{\PYGZdq{}\PYGZob{}\PYGZsq{}no\PYGZus{}create\PYGZsq{}: True, \PYGZsq{}no\PYGZus{}open\PYGZsq{}: True\PYGZcb{}\PYGZdq{}}\PYG{n+nt}{/\PYGZgt{}}
\end{sphinxVerbatim}

\end{description}

\item {} \begin{description}
\item[{list.many2one (ListFieldMany2One)}] \leavevmode
Default widget for many2one fields (in list view).

Specialization of many2one field for list views.  The main reason is that we
need to render many2one fields (in readonly mode) as a text, which does not
allow opening the related records.
\begin{itemize}
\item {} 
Supported field types: \sphinxstyleemphasis{many2one}

\end{itemize}

\end{description}

\item {} \begin{description}
\item[{kanban.many2one (KanbanFieldMany2One)}] \leavevmode
Default widget for many2one fields (in kanban view). We need to disable all
edition in kanban views.
\begin{itemize}
\item {} 
Supported field types: \sphinxstyleemphasis{many2one}

\end{itemize}

\end{description}

\item {} \begin{description}
\item[{many2many (FieldMany2Many)}] \leavevmode
Defaut widget for many2many fields.
\begin{itemize}
\item {} 
Supported field types: \sphinxstyleemphasis{many2many}

\end{itemize}

Attributes:
\begin{itemize}
\item {} 
mode: string, default view to display

\item {} 
domain: restrict the data to a specific domain

\end{itemize}

Options:
\begin{itemize}
\item {} 
create\_text: allow the customization of the text displayed when adding a
new record

\end{itemize}

\end{description}

\item {} \begin{description}
\item[{many2many\_binary (FieldMany2ManyBinaryMultiFiles)}] \leavevmode
This widget helps the user to upload or delete one or more files at the same
time.

Note that this widget is specific to the model ‘ir.attachment’.
\begin{itemize}
\item {} 
Supported field types: \sphinxstyleemphasis{many2many}

\end{itemize}

\end{description}

\item {} \begin{description}
\item[{many2many\_tags (FieldMany2ManyTags)}] \leavevmode
Display many2many as a list of tags.
\begin{itemize}
\item {} 
Supported field types: \sphinxstyleemphasis{many2many}

\end{itemize}

Options:
\begin{itemize}
\item {} 
color\_field: the name of a numeric field, which should be present in the
view.  A color will be chosen depending on its value.

\end{itemize}

\fvset{hllines={, ,}}%
\begin{sphinxVerbatim}[commandchars=\\\{\}]
\PYG{n+nt}{\PYGZlt{}field} \PYG{n+na}{name=}\PYG{l+s}{\PYGZdq{}category\PYGZus{}id\PYGZdq{}} \PYG{n+na}{widget=}\PYG{l+s}{\PYGZdq{}many2many\PYGZus{}tags\PYGZdq{}} \PYG{n+na}{options=}\PYG{l+s}{\PYGZdq{}\PYGZob{}\PYGZsq{}color\PYGZus{}field\PYGZsq{}: \PYGZsq{}color\PYGZsq{}\PYGZcb{}\PYGZdq{}}\PYG{n+nt}{/\PYGZgt{}}
\end{sphinxVerbatim}

\end{description}

\item {} \begin{description}
\item[{form.many2many\_tags (FormFieldMany2ManyTags)}] \leavevmode
Specialization of many2many\_tags widget for form views. It has some extra
code to allow editing the color of a tag.
\begin{itemize}
\item {} 
Supported field types: \sphinxstyleemphasis{many2many}

\end{itemize}

\end{description}

\item {} \begin{description}
\item[{kanban.many2many\_tags (KanbanFieldMany2ManyTags)}] \leavevmode
Specialization of many2many\_tags widget for kanban views.
\begin{itemize}
\item {} 
Supported field types: \sphinxstyleemphasis{many2many}

\end{itemize}

\end{description}

\item {} \begin{description}
\item[{many2many\_checkboxes (FieldMany2ManyCheckBoxes)}] \leavevmode
This field displays a list of checkboxes and allow the user to select a
subset of the choices.
\begin{itemize}
\item {} 
Supported field types: \sphinxstyleemphasis{many2many}

\end{itemize}

\end{description}

\item {} \begin{description}
\item[{one2many (FieldOne2Many)}] \leavevmode
Defaut widget for one2many fields.

It usually displays data in a sub list view, or a sub kanban view.
\begin{itemize}
\item {} 
Supported field types: \sphinxstyleemphasis{one2many}

\end{itemize}

\end{description}

\item {} \begin{description}
\item[{statusbar (FieldStatus)}] \leavevmode
This is a really specialized widget for the form views. It is the bar on top
of many forms which represent a flow, and allow selecting a specific state.
\begin{itemize}
\item {} 
Supported field types: \sphinxstyleemphasis{selection, many2one}

\end{itemize}

\end{description}

\item {} \begin{description}
\item[{reference (FieldReference)}] \leavevmode
The FieldReference is a combination of a select (for the model) and a
FieldMany2One (for its value).  It allows the selection of a record on an
arbitrary model.
\begin{itemize}
\item {} 
Supported field types: \sphinxstyleemphasis{char, reference}

\end{itemize}

\end{description}

\item {} \begin{description}
\item[{one2many\_list (FieldOne2Many)}] \leavevmode
This widget is exactly the same as a FieldOne2Many.  It is registered at this
key only for backward compatibility reasons.  Please avoid using this.

\end{description}

\end{itemize}


\section{Translating Modules}
\label{\detokenize{reference/translations:translating-modules}}\label{\detokenize{reference/translations:deferred}}\label{\detokenize{reference/translations::doc}}\label{\detokenize{reference/translations:reference-translations}}

\subsection{Exporting translatable term}
\label{\detokenize{reference/translations:exporting-translatable-term}}
A number of terms in your modules are “implicitly translatable” as a result,
even if you haven’t done any specific work towards translation you can export
your module’s translatable terms and may find content to work with.

Translations export is performed via the administration interface by logging into
the backend interface and opening \sphinxmenuselection{Settings \(\rightarrow\) Translations
\(\rightarrow\) Import / Export \(\rightarrow\) Export Translations}
\begin{itemize}
\item {} 
leave the language to the default (new language/empty template)

\item {} 
select the \sphinxhref{http://en.wikipedia.org/wiki/Gettext\#Translating}{PO File} format

\item {} 
select your module

\item {} 
click \sphinxmenuselection{Export} and download the file

\end{itemize}

\noindent{\hspace*{\fill}\sphinxincludegraphics[width=0.750\linewidth]{{po-export}.png}\hspace*{\fill}}

This gives you a file called \sphinxcode{\sphinxupquote{\sphinxstyleemphasis{yourmodule}.pot}} which should be moved to
the \sphinxcode{\sphinxupquote{\sphinxstyleemphasis{yourmodule}/i18n/}} directory. The file is a \sphinxstyleemphasis{PO Template} which
simply lists translatable strings and from which actual translations (PO files)
can be created. PO files can be created using \sphinxhref{http://www.gnu.org/software/gettext/manual/gettext.html\#Creating}{msginit}, with a dedicated
translation tool like \sphinxhref{http://poedit.net/}{POEdit} or by simply copying the template to a new file
called \sphinxcode{\sphinxupquote{\sphinxstyleemphasis{language}.po}}. Translation files should be put in
\sphinxcode{\sphinxupquote{\sphinxstyleemphasis{yourmodule}/i18n/}}, next to \sphinxcode{\sphinxupquote{\sphinxstyleemphasis{yourmodule}.pot}}, and will be
automatically loaded by Odoo when the corresponding language is installed (via
\sphinxmenuselection{Settings \(\rightarrow\) Translations \(\rightarrow\) Load a Translation})

\begin{sphinxadmonition}{note}{Note:}
translations for all loaded languages are also installed or updated
when installing or updating a module
\end{sphinxadmonition}


\subsection{Implicit exports}
\label{\detokenize{reference/translations:implicit-exports}}
Odoo automatically exports translatable strings from “data”-type content:
\begin{itemize}
\item {} 
in non-QWeb views, all text nodes are exported as well as the content of
the \sphinxcode{\sphinxupquote{string}}, \sphinxcode{\sphinxupquote{help}}, \sphinxcode{\sphinxupquote{sum}}, \sphinxcode{\sphinxupquote{confirm}} and \sphinxcode{\sphinxupquote{placeholder}}
attributes

\item {} 
QWeb templates (both server-side and client-side), all text nodes are
exported except inside \sphinxcode{\sphinxupquote{t-translation="off"}} blocks, the content of the
\sphinxcode{\sphinxupquote{title}}, \sphinxcode{\sphinxupquote{alt}}, \sphinxcode{\sphinxupquote{label}} and \sphinxcode{\sphinxupquote{placeholder}} attributes are also
exported

\item {} 
for {\hyperref[\detokenize{reference/orm:odoo.fields.Field}]{\sphinxcrossref{\sphinxcode{\sphinxupquote{Field}}}}}, unless their model is marked with
\sphinxcode{\sphinxupquote{\_translate = False}}:
\begin{itemize}
\item {} 
their \sphinxcode{\sphinxupquote{string}} and \sphinxcode{\sphinxupquote{help}} attributes are exported

\item {} 
if \sphinxcode{\sphinxupquote{selection}} is present and a list (or tuple), it’s exported

\item {} 
if their \sphinxcode{\sphinxupquote{translate}} attribute is set to \sphinxcode{\sphinxupquote{True}}, all of their existing
values (across all records) are exported

\end{itemize}

\item {} 
help/error messages of {\hyperref[\detokenize{reference/orm:odoo.models.Model._constraints}]{\sphinxcrossref{\sphinxcode{\sphinxupquote{\_constraints}}}}} and
{\hyperref[\detokenize{reference/orm:odoo.models.Model._sql_constraints}]{\sphinxcrossref{\sphinxcode{\sphinxupquote{\_sql\_constraints}}}}} are exported

\end{itemize}


\subsection{Explicit exports}
\label{\detokenize{reference/translations:explicit-exports}}
When it comes to more “imperative” situations in Python code or Javascript
code, Odoo cannot automatically export translatable terms so they
must be marked explicitly for export. This is done by wrapping a literal
string in a function call.

In Python, the wrapping function is \sphinxcode{\sphinxupquote{odoo.\_()}}:

\fvset{hllines={, ,}}%
\begin{sphinxVerbatim}[commandchars=\\\{\}]
\PYG{n}{title} \PYG{o}{=} \PYG{n}{\PYGZus{}}\PYG{p}{(}\PYG{l+s+s2}{\PYGZdq{}}\PYG{l+s+s2}{Bank Accounts}\PYG{l+s+s2}{\PYGZdq{}}\PYG{p}{)}
\end{sphinxVerbatim}

In JavaScript, the wrapping function is generally \sphinxcode{\sphinxupquote{odoo.web.\_t()}}:

\fvset{hllines={, ,}}%
\begin{sphinxVerbatim}[commandchars=\\\{\}]
\PYG{n+nx}{title} \PYG{o}{=} \PYG{n+nx}{\PYGZus{}t}\PYG{p}{(}\PYG{l+s+s2}{\PYGZdq{}Bank Accounts\PYGZdq{}}\PYG{p}{)}\PYG{p}{;}
\end{sphinxVerbatim}

\begin{sphinxadmonition}{warning}{Warning:}
Only literal strings can be marked for exports, not expressions or
variables. For situations where strings are formatted, this means the
format string must be marked, not the formatted string
\end{sphinxadmonition}


\subsubsection{Variables}
\label{\detokenize{reference/translations:variables}}
\sphinxstylestrong{Don’t} the extract may work but it will not translate the text correctly:

\fvset{hllines={, ,}}%
\begin{sphinxVerbatim}[commandchars=\\\{\}]
\PYG{n}{\PYGZus{}}\PYG{p}{(}\PYG{l+s+s2}{\PYGZdq{}}\PYG{l+s+s2}{Scheduled meeting with }\PYG{l+s+si}{\PYGZpc{}s}\PYG{l+s+s2}{\PYGZdq{}} \PYG{o}{\PYGZpc{}} \PYG{n}{invitee}\PYG{o}{.}\PYG{n}{name}\PYG{p}{)}
\end{sphinxVerbatim}

\sphinxstylestrong{Do} set the dynamic variables outside of the translation lookup:

\fvset{hllines={, ,}}%
\begin{sphinxVerbatim}[commandchars=\\\{\}]
\PYG{n}{\PYGZus{}}\PYG{p}{(}\PYG{l+s+s2}{\PYGZdq{}}\PYG{l+s+s2}{Scheduled meeting with }\PYG{l+s+si}{\PYGZpc{}s}\PYG{l+s+s2}{\PYGZdq{}}\PYG{p}{)} \PYG{o}{\PYGZpc{}} \PYG{n}{invitee}\PYG{o}{.}\PYG{n}{name}
\end{sphinxVerbatim}


\subsubsection{Blocks}
\label{\detokenize{reference/translations:blocks}}
\sphinxstylestrong{Don’t} split your translation in several blocks or multiples lines:

\fvset{hllines={, ,}}%
\begin{sphinxVerbatim}[commandchars=\\\{\}]
\PYG{c+c1}{\PYGZsh{} bad, trailing spaces, blocks out of context}
\PYG{n}{\PYGZus{}}\PYG{p}{(}\PYG{l+s+s2}{\PYGZdq{}}\PYG{l+s+s2}{You have }\PYG{l+s+s2}{\PYGZdq{}}\PYG{p}{)} \PYG{o}{+} \PYG{n+nb}{len}\PYG{p}{(}\PYG{n}{invoices}\PYG{p}{)} \PYG{o}{+} \PYG{n}{\PYGZus{}}\PYG{p}{(}\PYG{l+s+s2}{\PYGZdq{}}\PYG{l+s+s2}{ invoices waiting}\PYG{l+s+s2}{\PYGZdq{}}\PYG{p}{)}

\PYG{c+c1}{\PYGZsh{} bad, multiple small translations}
\PYG{n}{\PYGZus{}}\PYG{p}{(}\PYG{l+s+s2}{\PYGZdq{}}\PYG{l+s+s2}{Reference of the document that generated }\PYG{l+s+s2}{\PYGZdq{}}\PYG{p}{)} \PYG{o}{+} \PYGZbs{}
\PYG{n}{\PYGZus{}}\PYG{p}{(}\PYG{l+s+s2}{\PYGZdq{}}\PYG{l+s+s2}{this sales order request.}\PYG{l+s+s2}{\PYGZdq{}}\PYG{p}{)}
\end{sphinxVerbatim}

\sphinxstylestrong{Do} keep in one block, giving the full context to translators:

\fvset{hllines={, ,}}%
\begin{sphinxVerbatim}[commandchars=\\\{\}]
\PYG{c+c1}{\PYGZsh{} good, allow to change position of the number in the translation}
\PYG{n}{\PYGZus{}}\PYG{p}{(}\PYG{l+s+s2}{\PYGZdq{}}\PYG{l+s+s2}{You have }\PYG{l+s+si}{\PYGZpc{}s}\PYG{l+s+s2}{ invoices wainting}\PYG{l+s+s2}{\PYGZdq{}}\PYG{p}{)} \PYG{o}{\PYGZpc{}} \PYG{n+nb}{len}\PYG{p}{(}\PYG{n}{invoices}\PYG{p}{)}

\PYG{c+c1}{\PYGZsh{} good, full sentence is understandable}
\PYG{n}{\PYGZus{}}\PYG{p}{(}\PYG{l+s+s2}{\PYGZdq{}}\PYG{l+s+s2}{Reference of the document that generated }\PYG{l+s+s2}{\PYGZdq{}} \PYG{o}{+} \PYGZbs{}
  \PYG{l+s+s2}{\PYGZdq{}}\PYG{l+s+s2}{this sales order request.}\PYG{l+s+s2}{\PYGZdq{}}\PYG{p}{)}
\end{sphinxVerbatim}


\subsubsection{Plural}
\label{\detokenize{reference/translations:plural}}
\sphinxstylestrong{Don’t} pluralize terms the English-way:

\fvset{hllines={, ,}}%
\begin{sphinxVerbatim}[commandchars=\\\{\}]
\PYG{n}{msg} \PYG{o}{=} \PYG{n}{\PYGZus{}}\PYG{p}{(}\PYG{l+s+s2}{\PYGZdq{}}\PYG{l+s+s2}{You have }\PYG{l+s+si}{\PYGZpc{}s}\PYG{l+s+s2}{ invoice}\PYG{l+s+s2}{\PYGZdq{}}\PYG{p}{)} \PYG{o}{\PYGZpc{}} \PYG{n}{invoice\PYGZus{}count}
\PYG{k}{if} \PYG{n}{invoice\PYGZus{}count} \PYG{o}{\PYGZgt{}} \PYG{l+m+mi}{1}\PYG{p}{:}
  \PYG{n}{msg} \PYG{o}{+}\PYG{o}{=} \PYG{n}{\PYGZus{}}\PYG{p}{(}\PYG{l+s+s2}{\PYGZdq{}}\PYG{l+s+s2}{s}\PYG{l+s+s2}{\PYGZdq{}}\PYG{p}{)}
\end{sphinxVerbatim}

\sphinxstylestrong{Do} keep in mind every language has different plural forms:

\fvset{hllines={, ,}}%
\begin{sphinxVerbatim}[commandchars=\\\{\}]
\PYG{k}{if} \PYG{n}{invoice\PYGZus{}count} \PYG{o}{\PYGZgt{}} \PYG{l+m+mi}{1}\PYG{p}{:}
  \PYG{n}{msg} \PYG{o}{=} \PYG{n}{\PYGZus{}}\PYG{p}{(}\PYG{l+s+s2}{\PYGZdq{}}\PYG{l+s+s2}{You have }\PYG{l+s+si}{\PYGZpc{}s}\PYG{l+s+s2}{ invoices}\PYG{l+s+s2}{\PYGZdq{}}\PYG{p}{)} \PYG{o}{\PYGZpc{}} \PYG{n}{invoice\PYGZus{}count}
\PYG{k}{else}\PYG{p}{:}
  \PYG{n}{msg} \PYG{o}{=} \PYG{n}{\PYGZus{}}\PYG{p}{(}\PYG{l+s+s2}{\PYGZdq{}}\PYG{l+s+s2}{You have }\PYG{l+s+si}{\PYGZpc{}s}\PYG{l+s+s2}{ invoice}\PYG{l+s+s2}{\PYGZdq{}}\PYG{p}{)} \PYG{o}{\PYGZpc{}} \PYG{n}{invoice\PYGZus{}count}
\end{sphinxVerbatim}


\subsubsection{Read vs Run Time}
\label{\detokenize{reference/translations:read-vs-run-time}}
\sphinxstylestrong{Don’t} invoke translation lookup at server launch:

\fvset{hllines={, ,}}%
\begin{sphinxVerbatim}[commandchars=\\\{\}]
\PYG{n}{ERROR\PYGZus{}MESSAGE} \PYG{o}{=} \PYG{p}{\PYGZob{}}
  \PYG{c+c1}{\PYGZsh{} bad, evaluated at server launch with no user language}
  \PYG{n}{access\PYGZus{}error}\PYG{p}{:} \PYG{n}{\PYGZus{}}\PYG{p}{(}\PYG{l+s+s1}{\PYGZsq{}}\PYG{l+s+s1}{Access Error}\PYG{l+s+s1}{\PYGZsq{}}\PYG{p}{)}\PYG{p}{,}
  \PYG{n}{missing\PYGZus{}error}\PYG{p}{:} \PYG{n}{\PYGZus{}}\PYG{p}{(}\PYG{l+s+s1}{\PYGZsq{}}\PYG{l+s+s1}{Missing Record}\PYG{l+s+s1}{\PYGZsq{}}\PYG{p}{)}\PYG{p}{,}
\PYG{p}{\PYGZcb{}}

\PYG{k}{class} \PYG{n+nc}{Record}\PYG{p}{(}\PYG{n}{models}\PYG{o}{.}\PYG{n}{Model}\PYG{p}{)}\PYG{p}{:}

  \PYG{k}{def} \PYG{n+nf}{\PYGZus{}raise\PYGZus{}error}\PYG{p}{(}\PYG{n+nb+bp}{self}\PYG{p}{,} \PYG{n}{code}\PYG{p}{)}\PYG{p}{:}
    \PYG{k}{raise} \PYG{n}{UserError}\PYG{p}{(}\PYG{n}{ERROR\PYGZus{}MESSAGE}\PYG{p}{[}\PYG{n}{code}\PYG{p}{]}\PYG{p}{)}
\end{sphinxVerbatim}

\sphinxstylestrong{Don’t} invoke translation lookup when the javascript file is read:

\fvset{hllines={, ,}}%
\begin{sphinxVerbatim}[commandchars=\\\{\}]
\PYG{c+c1}{\PYGZsh{} bad, js \PYGZus{}t is evaluated too early}
\PYG{n}{var} \PYG{n}{core} \PYG{o}{=} \PYG{n}{require}\PYG{p}{(}\PYG{l+s+s1}{\PYGZsq{}}\PYG{l+s+s1}{web.core}\PYG{l+s+s1}{\PYGZsq{}}\PYG{p}{)}\PYG{p}{;}
\PYG{n}{var} \PYG{n}{\PYGZus{}t} \PYG{o}{=} \PYG{n}{core}\PYG{o}{.}\PYG{n}{\PYGZus{}t}\PYG{p}{;}
\PYG{n}{var} \PYG{n}{map\PYGZus{}title} \PYG{o}{=} \PYG{p}{\PYGZob{}}
    \PYG{n}{access\PYGZus{}error}\PYG{p}{:} \PYG{n}{\PYGZus{}t}\PYG{p}{(}\PYG{l+s+s1}{\PYGZsq{}}\PYG{l+s+s1}{Access Error}\PYG{l+s+s1}{\PYGZsq{}}\PYG{p}{)}\PYG{p}{,}
    \PYG{n}{missing\PYGZus{}error}\PYG{p}{:} \PYG{n}{\PYGZus{}t}\PYG{p}{(}\PYG{l+s+s1}{\PYGZsq{}}\PYG{l+s+s1}{Missing Record}\PYG{l+s+s1}{\PYGZsq{}}\PYG{p}{)}\PYG{p}{,}
\PYG{p}{\PYGZcb{}}\PYG{p}{;}
\end{sphinxVerbatim}

\sphinxstylestrong{Do} evaluate dynamically the translatable content:

\fvset{hllines={, ,}}%
\begin{sphinxVerbatim}[commandchars=\\\{\}]
\PYG{c+c1}{\PYGZsh{} good, evaluated at run time}
\PYG{k}{def} \PYG{n+nf}{\PYGZus{}get\PYGZus{}error\PYGZus{}message}\PYG{p}{(}\PYG{p}{)}\PYG{p}{:}
  \PYG{k}{return} \PYG{p}{\PYGZob{}}
    \PYG{n}{access\PYGZus{}error}\PYG{p}{:} \PYG{n}{\PYGZus{}}\PYG{p}{(}\PYG{l+s+s1}{\PYGZsq{}}\PYG{l+s+s1}{Access Error}\PYG{l+s+s1}{\PYGZsq{}}\PYG{p}{)}\PYG{p}{,}
    \PYG{n}{missing\PYGZus{}error}\PYG{p}{:} \PYG{n}{\PYGZus{}}\PYG{p}{(}\PYG{l+s+s1}{\PYGZsq{}}\PYG{l+s+s1}{Missing Record}\PYG{l+s+s1}{\PYGZsq{}}\PYG{p}{)}\PYG{p}{,}
  \PYG{p}{\PYGZcb{}}
\end{sphinxVerbatim}

\sphinxstylestrong{Do} in the case where the translation lookup is done when the JS file is
\sphinxstyleemphasis{read}, use \sphinxcode{\sphinxupquote{\_lt}} instead of \sphinxcode{\sphinxupquote{\_t}} to translate the term when it is \sphinxstyleemphasis{used}:

\fvset{hllines={, ,}}%
\begin{sphinxVerbatim}[commandchars=\\\{\}]
\PYG{c+c1}{\PYGZsh{} good, js \PYGZus{}lt is evaluated lazily}
\PYG{n}{var} \PYG{n}{core} \PYG{o}{=} \PYG{n}{require}\PYG{p}{(}\PYG{l+s+s1}{\PYGZsq{}}\PYG{l+s+s1}{web.core}\PYG{l+s+s1}{\PYGZsq{}}\PYG{p}{)}\PYG{p}{;}
\PYG{n}{var} \PYG{n}{\PYGZus{}lt} \PYG{o}{=} \PYG{n}{core}\PYG{o}{.}\PYG{n}{\PYGZus{}lt}\PYG{p}{;}
\PYG{n}{var} \PYG{n}{map\PYGZus{}title} \PYG{o}{=} \PYG{p}{\PYGZob{}}
    \PYG{n}{access\PYGZus{}error}\PYG{p}{:} \PYG{n}{\PYGZus{}lt}\PYG{p}{(}\PYG{l+s+s1}{\PYGZsq{}}\PYG{l+s+s1}{Access Error}\PYG{l+s+s1}{\PYGZsq{}}\PYG{p}{)}\PYG{p}{,}
    \PYG{n}{missing\PYGZus{}error}\PYG{p}{:} \PYG{n}{\PYGZus{}lt}\PYG{p}{(}\PYG{l+s+s1}{\PYGZsq{}}\PYG{l+s+s1}{Missing Record}\PYG{l+s+s1}{\PYGZsq{}}\PYG{p}{)}\PYG{p}{,}
\PYG{p}{\PYGZcb{}}\PYG{p}{;}
\end{sphinxVerbatim}


\section{QWeb Reports}
\label{\detokenize{reference/reports::doc}}\label{\detokenize{reference/reports:poedit}}\label{\detokenize{reference/reports:qweb-reports}}
Reports are written in HTML/QWeb, like all regular views in Odoo. You can use
the usual {\hyperref[\detokenize{reference/qweb:reference-qweb}]{\sphinxcrossref{\DUrole{std,std-ref}{QWeb control flow tools}}}}. The PDF rendering
itself is performed by \sphinxhref{http://wkhtmltopdf.org}{wkhtmltopdf}.

If you want to create a report on a certain model, you will need to define
this {\hyperref[\detokenize{reference/reports:reference-reports-report}]{\sphinxcrossref{\DUrole{std,std-ref}{Report}}}} and the
{\hyperref[\detokenize{reference/reports:reference-reports-templates}]{\sphinxcrossref{\DUrole{std,std-ref}{Report template}}}} it will use. If you wish, you can also
specify a specific {\hyperref[\detokenize{reference/reports:reference-reports-paper-formats}]{\sphinxcrossref{\DUrole{std,std-ref}{Paper Format}}}} for this
report. Finally, if you need access to more than your model, you can define a
{\hyperref[\detokenize{reference/reports:reference-reports-custom-reports}]{\sphinxcrossref{\DUrole{std,std-ref}{Custom Reports}}}} class that gives you access to more
models and records in the template.


\subsection{Report}
\label{\detokenize{reference/reports:reference-reports-report}}\label{\detokenize{reference/reports:report}}
Every report must be declared by a {\hyperref[\detokenize{reference/actions:reference-actions-report}]{\sphinxcrossref{\DUrole{std,std-ref}{report action}}}}.

For simplicity, a shortcut \sphinxcode{\sphinxupquote{\textless{}report\textgreater{}}} element is available to define a
report, rather than have to set up {\hyperref[\detokenize{reference/actions:reference-actions-report}]{\sphinxcrossref{\DUrole{std,std-ref}{the action}}}} and its surroundings manually. That \sphinxcode{\sphinxupquote{\textless{}report\textgreater{}}}
can take the following attributes:
\begin{description}
\item[{\sphinxcode{\sphinxupquote{id}}}] \leavevmode
the generated record’s \DUrole{xref,std,std-term}{external id}

\item[{\sphinxcode{\sphinxupquote{name}} (mandatory)}] \leavevmode
only useful as a mnemonic/description of the report when looking for one
in a list of some sort

\item[{\sphinxcode{\sphinxupquote{model}} (mandatory)}] \leavevmode
the model your report will be about

\item[{\sphinxcode{\sphinxupquote{report\_type}} (mandatory)}] \leavevmode
either \sphinxcode{\sphinxupquote{qweb-pdf}} for PDF reports or \sphinxcode{\sphinxupquote{qweb-html}} for HTML

\item[{\sphinxcode{\sphinxupquote{report\_name}}}] \leavevmode
the name of your report (which will be the name of the PDF output)

\item[{\sphinxcode{\sphinxupquote{groups}}}] \leavevmode
{\hyperref[\detokenize{reference/orm:odoo.fields.Many2many}]{\sphinxcrossref{\sphinxcode{\sphinxupquote{Many2many}}}}} field to the groups allowed to view/use
the current report

\item[{\sphinxcode{\sphinxupquote{attachment\_use}}}] \leavevmode
if set to True, the report will be stored as an attachment of the record
using the name generated by the \sphinxcode{\sphinxupquote{attachment}} expression; you can use
this if you need your report to be generated only once (for legal reasons,
for example)

\item[{\sphinxcode{\sphinxupquote{attachment}}}] \leavevmode
python expression that defines the name of the report; the record is
acessible as the variable \sphinxcode{\sphinxupquote{object}}

\item[{\sphinxcode{\sphinxupquote{paperformat}}}] \leavevmode
external id of the paperformat you wish to use (defaults to the company’s
paperformat if not specified)

\end{description}

Example:

\fvset{hllines={, ,}}%
\begin{sphinxVerbatim}[commandchars=\\\{\}]
\PYG{n+nt}{\PYGZlt{}report}
    \PYG{n+na}{id=}\PYG{l+s}{\PYGZdq{}account\PYGZus{}invoices\PYGZdq{}}
    \PYG{n+na}{model=}\PYG{l+s}{\PYGZdq{}account.invoice\PYGZdq{}}
    \PYG{n+na}{string=}\PYG{l+s}{\PYGZdq{}Invoices\PYGZdq{}}
    \PYG{n+na}{report\PYGZus{}type=}\PYG{l+s}{\PYGZdq{}qweb\PYGZhy{}pdf\PYGZdq{}}
    \PYG{n+na}{name=}\PYG{l+s}{\PYGZdq{}account.report\PYGZus{}invoice\PYGZdq{}}
    \PYG{n+na}{file=}\PYG{l+s}{\PYGZdq{}account.report\PYGZus{}invoice\PYGZdq{}}
    \PYG{n+na}{attachment\PYGZus{}use=}\PYG{l+s}{\PYGZdq{}True\PYGZdq{}}
    \PYG{n+na}{attachment=}\PYG{l+s}{\PYGZdq{}(object.state in (\PYGZsq{}open\PYGZsq{},\PYGZsq{}paid\PYGZsq{})) and}
\PYG{l+s}{        (\PYGZsq{}INV\PYGZsq{}+(object.number or \PYGZsq{}\PYGZsq{}).replace(\PYGZsq{}/\PYGZsq{},\PYGZsq{}\PYGZsq{})+\PYGZsq{}.pdf\PYGZsq{})\PYGZdq{}}
\PYG{n+nt}{/\PYGZgt{}}
\end{sphinxVerbatim}


\subsection{Report template}
\label{\detokenize{reference/reports:reference-reports-templates}}\label{\detokenize{reference/reports:report-template}}

\subsubsection{Minimal viable template}
\label{\detokenize{reference/reports:minimal-viable-template}}
A minimal template would look like:

\fvset{hllines={, ,}}%
\begin{sphinxVerbatim}[commandchars=\\\{\}]
\PYG{n+nt}{\PYGZlt{}template} \PYG{n+na}{id=}\PYG{l+s}{\PYGZdq{}report\PYGZus{}invoice\PYGZdq{}}\PYG{n+nt}{\PYGZgt{}}
    \PYG{n+nt}{\PYGZlt{}t} \PYG{n+na}{t\PYGZhy{}call=}\PYG{l+s}{\PYGZdq{}web.html\PYGZus{}container\PYGZdq{}}\PYG{n+nt}{\PYGZgt{}}
        \PYG{n+nt}{\PYGZlt{}t} \PYG{n+na}{t\PYGZhy{}foreach=}\PYG{l+s}{\PYGZdq{}docs\PYGZdq{}} \PYG{n+na}{t\PYGZhy{}as=}\PYG{l+s}{\PYGZdq{}o\PYGZdq{}}\PYG{n+nt}{\PYGZgt{}}
            \PYG{n+nt}{\PYGZlt{}t} \PYG{n+na}{t\PYGZhy{}call=}\PYG{l+s}{\PYGZdq{}web.external\PYGZus{}layout\PYGZdq{}}\PYG{n+nt}{\PYGZgt{}}
                \PYG{n+nt}{\PYGZlt{}div} \PYG{n+na}{class=}\PYG{l+s}{\PYGZdq{}page\PYGZdq{}}\PYG{n+nt}{\PYGZgt{}}
                    \PYG{n+nt}{\PYGZlt{}h2}\PYG{n+nt}{\PYGZgt{}}Report title\PYG{n+nt}{\PYGZlt{}/h2\PYGZgt{}}
                    \PYG{n+nt}{\PYGZlt{}p}\PYG{n+nt}{\PYGZgt{}}This object\PYGZsq{}s name is \PYG{n+nt}{\PYGZlt{}span} \PYG{n+na}{t\PYGZhy{}field=}\PYG{l+s}{\PYGZdq{}o.name\PYGZdq{}}\PYG{n+nt}{/\PYGZgt{}}\PYG{n+nt}{\PYGZlt{}/p\PYGZgt{}}
                \PYG{n+nt}{\PYGZlt{}/div\PYGZgt{}}
            \PYG{n+nt}{\PYGZlt{}/t\PYGZgt{}}
        \PYG{n+nt}{\PYGZlt{}/t\PYGZgt{}}
    \PYG{n+nt}{\PYGZlt{}/t\PYGZgt{}}
\PYG{n+nt}{\PYGZlt{}/template\PYGZgt{}}
\end{sphinxVerbatim}

Calling \sphinxcode{\sphinxupquote{external\_layout}} will add the default header and footer on your
report. The PDF body will be the content inside the \sphinxcode{\sphinxupquote{\textless{}div
class="page"\textgreater{}}}. The template’s \sphinxcode{\sphinxupquote{id}} must be the name specified in the
report declaration; for example \sphinxcode{\sphinxupquote{account.report\_invoice}} for the above
report. Since this is a QWeb template, you can access all the fields of the
\sphinxcode{\sphinxupquote{docs}} objects received by the template.

There are some specific variables accessible in reports, mainly:
\begin{description}
\item[{\sphinxcode{\sphinxupquote{docs}}}] \leavevmode
records for the current report

\item[{\sphinxcode{\sphinxupquote{doc\_ids}}}] \leavevmode
list of ids for the \sphinxcode{\sphinxupquote{docs}} records

\item[{\sphinxcode{\sphinxupquote{doc\_model}}}] \leavevmode
model for the \sphinxcode{\sphinxupquote{docs}} records

\item[{\sphinxcode{\sphinxupquote{time}}}] \leavevmode
a reference to \sphinxhref{https://docs.python.org/3/library/time.html\#module-time}{\sphinxcode{\sphinxupquote{time}}} from the Python standard library

\item[{\sphinxcode{\sphinxupquote{user}}}] \leavevmode
\sphinxcode{\sphinxupquote{res.user}} record for the user printing the report

\item[{\sphinxcode{\sphinxupquote{res\_company}}}] \leavevmode
record for the current \sphinxcode{\sphinxupquote{user}}’s company

\end{description}

If you wish to access other records/models in the template, you will need
{\hyperref[\detokenize{reference/reports:reference-reports-custom-reports}]{\sphinxcrossref{\DUrole{std,std-ref}{a custom report}}}}.


\subsubsection{Translatable Templates}
\label{\detokenize{reference/reports:translatable-templates}}
If you wish to translate reports (to the language of a partner, for example),
you need to define two templates:
\begin{itemize}
\item {} 
The main report template

\item {} 
The translatable document

\end{itemize}

You can then call the translatable document from your main template with the attribute
\sphinxcode{\sphinxupquote{t-lang}} set to a language code (for example \sphinxcode{\sphinxupquote{fr}} or \sphinxcode{\sphinxupquote{en\_US}}) or to a record field.
You will also need to re-browse the related records with the proper context if you use
fields that are translatable (like country names, sales conditions, etc.)

\begin{sphinxadmonition}{warning}{Warning:}
If your report template does not use translatable record fields, re-browsing the record
in another language is \sphinxstyleemphasis{not} necessary and will impact performances.
\end{sphinxadmonition}

For example, let’s look at the Sale Order report from the Sale module:

\fvset{hllines={, ,}}%
\begin{sphinxVerbatim}[commandchars=\\\{\}]
\PYG{c}{\PYGZlt{}!\PYGZhy{}\PYGZhy{}}\PYG{c}{ Main template }\PYG{c}{\PYGZhy{}\PYGZhy{}\PYGZgt{}}
\PYG{n+nt}{\PYGZlt{}template} \PYG{n+na}{id=}\PYG{l+s}{\PYGZdq{}report\PYGZus{}saleorder\PYGZdq{}}\PYG{n+nt}{\PYGZgt{}}
    \PYG{n+nt}{\PYGZlt{}t} \PYG{n+na}{t\PYGZhy{}call=}\PYG{l+s}{\PYGZdq{}web.html\PYGZus{}container\PYGZdq{}}\PYG{n+nt}{\PYGZgt{}}
        \PYG{n+nt}{\PYGZlt{}t} \PYG{n+na}{t\PYGZhy{}foreach=}\PYG{l+s}{\PYGZdq{}docs\PYGZdq{}} \PYG{n+na}{t\PYGZhy{}as=}\PYG{l+s}{\PYGZdq{}doc\PYGZdq{}}\PYG{n+nt}{\PYGZgt{}}
            \PYG{n+nt}{\PYGZlt{}t} \PYG{n+na}{t\PYGZhy{}call=}\PYG{l+s}{\PYGZdq{}sale.report\PYGZus{}saleorder\PYGZus{}document\PYGZdq{}} \PYG{n+na}{t\PYGZhy{}lang=}\PYG{l+s}{\PYGZdq{}doc.partner\PYGZus{}id.lang\PYGZdq{}}\PYG{n+nt}{/\PYGZgt{}}
        \PYG{n+nt}{\PYGZlt{}/t\PYGZgt{}}
    \PYG{n+nt}{\PYGZlt{}/t\PYGZgt{}}
\PYG{n+nt}{\PYGZlt{}/template\PYGZgt{}}

\PYG{c}{\PYGZlt{}!\PYGZhy{}\PYGZhy{}}\PYG{c}{ Translatable template }\PYG{c}{\PYGZhy{}\PYGZhy{}\PYGZgt{}}
\PYG{n+nt}{\PYGZlt{}template} \PYG{n+na}{id=}\PYG{l+s}{\PYGZdq{}report\PYGZus{}saleorder\PYGZus{}document\PYGZdq{}}\PYG{n+nt}{\PYGZgt{}}
    \PYG{c}{\PYGZlt{}!\PYGZhy{}\PYGZhy{}}\PYG{c}{ Re}\PYG{c}{\PYGZhy{}}\PYG{c}{browse of the record with the partner lang }\PYG{c}{\PYGZhy{}\PYGZhy{}\PYGZgt{}}
    \PYG{n+nt}{\PYGZlt{}t} \PYG{n+na}{t\PYGZhy{}set=}\PYG{l+s}{\PYGZdq{}doc\PYGZdq{}} \PYG{n+na}{t\PYGZhy{}value=}\PYG{l+s}{\PYGZdq{}doc.with\PYGZus{}context(\PYGZob{}\PYGZsq{}lang\PYGZsq{}:doc.partner\PYGZus{}id.lang\PYGZcb{})\PYGZdq{}} \PYG{n+nt}{/\PYGZgt{}}
    \PYG{n+nt}{\PYGZlt{}t} \PYG{n+na}{t\PYGZhy{}call=}\PYG{l+s}{\PYGZdq{}web.external\PYGZus{}layout\PYGZdq{}}\PYG{n+nt}{\PYGZgt{}}
        \PYG{n+nt}{\PYGZlt{}div} \PYG{n+na}{class=}\PYG{l+s}{\PYGZdq{}page\PYGZdq{}}\PYG{n+nt}{\PYGZgt{}}
            \PYG{n+nt}{\PYGZlt{}div} \PYG{n+na}{class=}\PYG{l+s}{\PYGZdq{}oe\PYGZus{}structure\PYGZdq{}}\PYG{n+nt}{/\PYGZgt{}}
            \PYG{n+nt}{\PYGZlt{}div} \PYG{n+na}{class=}\PYG{l+s}{\PYGZdq{}row\PYGZdq{}}\PYG{n+nt}{\PYGZgt{}}
                \PYG{n+nt}{\PYGZlt{}div} \PYG{n+na}{class=}\PYG{l+s}{\PYGZdq{}col\PYGZhy{}xs\PYGZhy{}6\PYGZdq{}}\PYG{n+nt}{\PYGZgt{}}
                    \PYG{n+nt}{\PYGZlt{}strong} \PYG{n+na}{t\PYGZhy{}if=}\PYG{l+s}{\PYGZdq{}doc.partner\PYGZus{}shipping\PYGZus{}id == doc.partner\PYGZus{}invoice\PYGZus{}id\PYGZdq{}}\PYG{n+nt}{\PYGZgt{}}Invoice and shipping address:\PYG{n+nt}{\PYGZlt{}/strong\PYGZgt{}}
                    \PYG{n+nt}{\PYGZlt{}strong} \PYG{n+na}{t\PYGZhy{}if=}\PYG{l+s}{\PYGZdq{}doc.partner\PYGZus{}shipping\PYGZus{}id != doc.partner\PYGZus{}invoice\PYGZus{}id\PYGZdq{}}\PYG{n+nt}{\PYGZgt{}}Invoice address:\PYG{n+nt}{\PYGZlt{}/strong\PYGZgt{}}
                    \PYG{n+nt}{\PYGZlt{}div} \PYG{n+na}{t\PYGZhy{}field=}\PYG{l+s}{\PYGZdq{}doc.partner\PYGZus{}invoice\PYGZus{}id\PYGZdq{}} \PYG{n+na}{t\PYGZhy{}options=}\PYG{l+s}{\PYGZdq{}\PYGZob{}\PYGZam{}quot;no\PYGZus{}marker\PYGZam{}quot;: True\PYGZcb{}\PYGZdq{}}\PYG{n+nt}{/\PYGZgt{}}
                \PYG{n+nt}{\PYGZlt{}...}\PYG{n+nt}{\PYGZgt{}}
            \PYG{n+nt}{\PYGZlt{}div} \PYG{n+na}{class=}\PYG{l+s}{\PYGZdq{}oe\PYGZus{}structure\PYGZdq{}}\PYG{n+nt}{/\PYGZgt{}}
        \PYG{n+nt}{\PYGZlt{}/div\PYGZgt{}}
    \PYG{n+nt}{\PYGZlt{}/t\PYGZgt{}}
\PYG{n+nt}{\PYGZlt{}/template\PYGZgt{}}
\end{sphinxVerbatim}

The main template calls the translatable template with \sphinxcode{\sphinxupquote{doc.partner\_id.lang}} as a
\sphinxcode{\sphinxupquote{t-lang}} parameter, so it will be rendered in the language of the partner. This way,
each Sale Order will be printed in the language of the corresponding customer. If you wish
to translate only the body of the document, but keep the header and footer in a default
language, you could call the report’s external layout this way:

\fvset{hllines={, ,}}%
\begin{sphinxVerbatim}[commandchars=\\\{\}]
\PYG{n+nt}{\PYGZlt{}t} \PYG{n+na}{t\PYGZhy{}call=}\PYG{l+s}{\PYGZdq{}web.external\PYGZus{}layout\PYGZdq{}} \PYG{n+na}{t\PYGZhy{}lang=}\PYG{l+s}{\PYGZdq{}en\PYGZus{}US\PYGZdq{}}\PYG{n+nt}{\PYGZgt{}}
\end{sphinxVerbatim}

\begin{sphinxadmonition}{tip}{Tip:}
Please take note that this works only when calling external templates, you will not be
able to translate part of a document by setting a \sphinxcode{\sphinxupquote{t-lang}} attribute on an xml node other
than \sphinxcode{\sphinxupquote{t-call}}. If you wish to translate part of a template, you can create an external
template with this partial template and call it from the main one with the \sphinxcode{\sphinxupquote{t-lang}}
attribute.
\end{sphinxadmonition}


\subsubsection{Barcodes}
\label{\detokenize{reference/reports:barcodes}}
Barcodes are images returned by a controller and can easily be embedded in
reports thanks to the QWeb syntax (e.g. see {\hyperref[\detokenize{reference/qweb:reference-qweb-attributes}]{\sphinxcrossref{\DUrole{std,std-ref}{attributes}}}}):

\fvset{hllines={, ,}}%
\begin{sphinxVerbatim}[commandchars=\\\{\}]
\PYG{p}{\PYGZlt{}}\PYG{n+nt}{img} \PYG{n+na}{t\PYGZhy{}att\PYGZhy{}src}\PYG{o}{=}\PYG{l+s}{\PYGZdq{}\PYGZsq{}/report/barcode/QR/\PYGZpc{}s\PYGZsq{} \PYGZpc{} \PYGZsq{}My text in qr code\PYGZsq{}\PYGZdq{}}\PYG{p}{/}\PYG{p}{\PYGZgt{}}
\end{sphinxVerbatim}

More parameters can be passed as a query string

\fvset{hllines={, ,}}%
\begin{sphinxVerbatim}[commandchars=\\\{\}]
\PYG{p}{\PYGZlt{}}\PYG{n+nt}{img} \PYG{n+na}{t\PYGZhy{}att\PYGZhy{}src}\PYG{o}{=}\PYG{l+s}{\PYGZdq{}\PYGZsq{}/report/barcode/?}
\PYG{l+s}{    type=\PYGZpc{}s\PYGZam{}value=\PYGZpc{}s\PYGZam{}width=\PYGZpc{}s\PYGZam{}height=\PYGZpc{}s\PYGZsq{}\PYGZpc{}(\PYGZsq{}QR\PYGZsq{}, \PYGZsq{}text\PYGZsq{}, 200, 200)\PYGZdq{}}\PYG{p}{/}\PYG{p}{\PYGZgt{}}
\end{sphinxVerbatim}


\subsubsection{Useful Remarks}
\label{\detokenize{reference/reports:useful-remarks}}\begin{itemize}
\item {} 
Twitter Bootstrap and FontAwesome classes can be used in your report
template

\item {} 
Local CSS can be put directly in the template

\item {} 
Global CSS can be inserted in the main report layout by inheriting its
template and inserting your CSS:

\fvset{hllines={, ,}}%
\begin{sphinxVerbatim}[commandchars=\\\{\}]
\PYG{n+nt}{\PYGZlt{}template} \PYG{n+na}{id=}\PYG{l+s}{\PYGZdq{}report\PYGZus{}saleorder\PYGZus{}style\PYGZdq{}} \PYG{n+na}{inherit\PYGZus{}id=}\PYG{l+s}{\PYGZdq{}report.style\PYGZdq{}}\PYG{n+nt}{\PYGZgt{}}
  \PYG{n+nt}{\PYGZlt{}xpath} \PYG{n+na}{expr=}\PYG{l+s}{\PYGZdq{}.\PYGZdq{}}\PYG{n+nt}{\PYGZgt{}}
    \PYG{n+nt}{\PYGZlt{}t}\PYG{n+nt}{\PYGZgt{}}
      .example\PYGZhy{}css\PYGZhy{}class \PYGZob{}
        background\PYGZhy{}color: red;
      \PYGZcb{}
    \PYG{n+nt}{\PYGZlt{}/t\PYGZgt{}}
  \PYG{n+nt}{\PYGZlt{}/xpath\PYGZgt{}}
\PYG{n+nt}{\PYGZlt{}/template\PYGZgt{}}
\end{sphinxVerbatim}

\item {} 
If it appears that your PDF report is missing the styles, please check
{\hyperref[\detokenize{howtos/backend:reference-backend-reporting-printed-reports-pdf-without-styles}]{\sphinxcrossref{\DUrole{std,std-ref}{these instructions}}}}.

\end{itemize}


\subsection{Paper Format}
\label{\detokenize{reference/reports:reference-reports-paper-formats}}\label{\detokenize{reference/reports:paper-format}}
Paper formats are records of \sphinxcode{\sphinxupquote{report.paperformat}} and can contain the
following attributes:
\begin{description}
\item[{\sphinxcode{\sphinxupquote{name}} (mandatory)}] \leavevmode
only useful as a mnemonic/description of the report when looking for one
in a list of some sort

\item[{\sphinxcode{\sphinxupquote{description}}}] \leavevmode
a small description of your format

\item[{\sphinxcode{\sphinxupquote{format}}}] \leavevmode
either a predefined format (A0 to A9, B0 to B10, Legal, Letter,
Tabloid,…) or \sphinxcode{\sphinxupquote{custom}}; A4 by default. You cannot use a non-custom
format if you define the page dimensions.

\item[{\sphinxcode{\sphinxupquote{dpi}}}] \leavevmode
output DPI; 90 by default

\item[{\sphinxcode{\sphinxupquote{margin\_top}}, \sphinxcode{\sphinxupquote{margin\_bottom}}, \sphinxcode{\sphinxupquote{margin\_left}}, \sphinxcode{\sphinxupquote{margin\_right}}}] \leavevmode
margin sizes in mm

\item[{\sphinxcode{\sphinxupquote{page\_height}}, \sphinxcode{\sphinxupquote{page\_width}}}] \leavevmode
page dimensions in mm

\item[{\sphinxcode{\sphinxupquote{orientation}}}] \leavevmode
Landscape or Portrait

\item[{\sphinxcode{\sphinxupquote{header\_line}}}] \leavevmode
boolean to display a header line

\item[{\sphinxcode{\sphinxupquote{header\_spacing}}}] \leavevmode
header spacing in mm

\end{description}

Example:

\fvset{hllines={, ,}}%
\begin{sphinxVerbatim}[commandchars=\\\{\}]
\PYG{n+nt}{\PYGZlt{}record} \PYG{n+na}{id=}\PYG{l+s}{\PYGZdq{}paperformat\PYGZus{}frenchcheck\PYGZdq{}} \PYG{n+na}{model=}\PYG{l+s}{\PYGZdq{}report.paperformat\PYGZdq{}}\PYG{n+nt}{\PYGZgt{}}
    \PYG{n+nt}{\PYGZlt{}field} \PYG{n+na}{name=}\PYG{l+s}{\PYGZdq{}name\PYGZdq{}}\PYG{n+nt}{\PYGZgt{}}French Bank Check\PYG{n+nt}{\PYGZlt{}/field\PYGZgt{}}
    \PYG{n+nt}{\PYGZlt{}field} \PYG{n+na}{name=}\PYG{l+s}{\PYGZdq{}default\PYGZdq{}} \PYG{n+na}{eval=}\PYG{l+s}{\PYGZdq{}True\PYGZdq{}}\PYG{n+nt}{/\PYGZgt{}}
    \PYG{n+nt}{\PYGZlt{}field} \PYG{n+na}{name=}\PYG{l+s}{\PYGZdq{}format\PYGZdq{}}\PYG{n+nt}{\PYGZgt{}}custom\PYG{n+nt}{\PYGZlt{}/field\PYGZgt{}}
    \PYG{n+nt}{\PYGZlt{}field} \PYG{n+na}{name=}\PYG{l+s}{\PYGZdq{}page\PYGZus{}height\PYGZdq{}}\PYG{n+nt}{\PYGZgt{}}80\PYG{n+nt}{\PYGZlt{}/field\PYGZgt{}}
    \PYG{n+nt}{\PYGZlt{}field} \PYG{n+na}{name=}\PYG{l+s}{\PYGZdq{}page\PYGZus{}width\PYGZdq{}}\PYG{n+nt}{\PYGZgt{}}175\PYG{n+nt}{\PYGZlt{}/field\PYGZgt{}}
    \PYG{n+nt}{\PYGZlt{}field} \PYG{n+na}{name=}\PYG{l+s}{\PYGZdq{}orientation\PYGZdq{}}\PYG{n+nt}{\PYGZgt{}}Portrait\PYG{n+nt}{\PYGZlt{}/field\PYGZgt{}}
    \PYG{n+nt}{\PYGZlt{}field} \PYG{n+na}{name=}\PYG{l+s}{\PYGZdq{}margin\PYGZus{}top\PYGZdq{}}\PYG{n+nt}{\PYGZgt{}}3\PYG{n+nt}{\PYGZlt{}/field\PYGZgt{}}
    \PYG{n+nt}{\PYGZlt{}field} \PYG{n+na}{name=}\PYG{l+s}{\PYGZdq{}margin\PYGZus{}bottom\PYGZdq{}}\PYG{n+nt}{\PYGZgt{}}3\PYG{n+nt}{\PYGZlt{}/field\PYGZgt{}}
    \PYG{n+nt}{\PYGZlt{}field} \PYG{n+na}{name=}\PYG{l+s}{\PYGZdq{}margin\PYGZus{}left\PYGZdq{}}\PYG{n+nt}{\PYGZgt{}}3\PYG{n+nt}{\PYGZlt{}/field\PYGZgt{}}
    \PYG{n+nt}{\PYGZlt{}field} \PYG{n+na}{name=}\PYG{l+s}{\PYGZdq{}margin\PYGZus{}right\PYGZdq{}}\PYG{n+nt}{\PYGZgt{}}3\PYG{n+nt}{\PYGZlt{}/field\PYGZgt{}}
    \PYG{n+nt}{\PYGZlt{}field} \PYG{n+na}{name=}\PYG{l+s}{\PYGZdq{}header\PYGZus{}line\PYGZdq{}} \PYG{n+na}{eval=}\PYG{l+s}{\PYGZdq{}False\PYGZdq{}}\PYG{n+nt}{/\PYGZgt{}}
    \PYG{n+nt}{\PYGZlt{}field} \PYG{n+na}{name=}\PYG{l+s}{\PYGZdq{}header\PYGZus{}spacing\PYGZdq{}}\PYG{n+nt}{\PYGZgt{}}3\PYG{n+nt}{\PYGZlt{}/field\PYGZgt{}}
    \PYG{n+nt}{\PYGZlt{}field} \PYG{n+na}{name=}\PYG{l+s}{\PYGZdq{}dpi\PYGZdq{}}\PYG{n+nt}{\PYGZgt{}}80\PYG{n+nt}{\PYGZlt{}/field\PYGZgt{}}
\PYG{n+nt}{\PYGZlt{}/record\PYGZgt{}}
\end{sphinxVerbatim}


\subsection{Custom Reports}
\label{\detokenize{reference/reports:reference-reports-custom-reports}}\label{\detokenize{reference/reports:custom-reports}}
The report model has a default \sphinxcode{\sphinxupquote{get\_html}} function that looks for a model
named \sphinxcode{\sphinxupquote{report.\sphinxstyleemphasis{module.report\_name}}}. If it exists, it will use it to
call the QWeb engine; otherwise a generic function will be used. If you wish
to customize your reports by including more things in the template (like
records of others models, for example), you can define this model, overwrite
the function \sphinxcode{\sphinxupquote{render\_html}} and pass objects in the \sphinxcode{\sphinxupquote{docargs}} dictionary:

\fvset{hllines={, ,}}%
\begin{sphinxVerbatim}[commandchars=\\\{\}]
\PYG{k+kn}{from} \PYG{n+nn}{odoo} \PYG{k+kn}{import} \PYG{n}{api}\PYG{p}{,} \PYG{n}{models}

\PYG{k}{class} \PYG{n+nc}{ParticularReport}\PYG{p}{(}\PYG{n}{models}\PYG{o}{.}\PYG{n}{AbstractModel}\PYG{p}{)}\PYG{p}{:}
    \PYG{n}{\PYGZus{}name} \PYG{o}{=} \PYG{l+s+s1}{\PYGZsq{}}\PYG{l+s+s1}{report.module.report\PYGZus{}name}\PYG{l+s+s1}{\PYGZsq{}}
    \PYG{n+nd}{@api.model}
    \PYG{k}{def} \PYG{n+nf}{render\PYGZus{}html}\PYG{p}{(}\PYG{n+nb+bp}{self}\PYG{p}{,} \PYG{n}{docids}\PYG{p}{,} \PYG{n}{data}\PYG{o}{=}\PYG{n+nb+bp}{None}\PYG{p}{)}\PYG{p}{:}
        \PYG{n}{report\PYGZus{}obj} \PYG{o}{=} \PYG{n+nb+bp}{self}\PYG{o}{.}\PYG{n}{env}\PYG{p}{[}\PYG{l+s+s1}{\PYGZsq{}}\PYG{l+s+s1}{report}\PYG{l+s+s1}{\PYGZsq{}}\PYG{p}{]}
        \PYG{n}{report} \PYG{o}{=} \PYG{n}{report\PYGZus{}obj}\PYG{o}{.}\PYG{n}{\PYGZus{}get\PYGZus{}report\PYGZus{}from\PYGZus{}name}\PYG{p}{(}\PYG{l+s+s1}{\PYGZsq{}}\PYG{l+s+s1}{module.report\PYGZus{}name}\PYG{l+s+s1}{\PYGZsq{}}\PYG{p}{)}
        \PYG{n}{docargs} \PYG{o}{=} \PYG{p}{\PYGZob{}}
            \PYG{l+s+s1}{\PYGZsq{}}\PYG{l+s+s1}{doc\PYGZus{}ids}\PYG{l+s+s1}{\PYGZsq{}}\PYG{p}{:} \PYG{n}{docids}\PYG{p}{,}
            \PYG{l+s+s1}{\PYGZsq{}}\PYG{l+s+s1}{doc\PYGZus{}model}\PYG{l+s+s1}{\PYGZsq{}}\PYG{p}{:} \PYG{n}{report}\PYG{o}{.}\PYG{n}{model}\PYG{p}{,}
            \PYG{l+s+s1}{\PYGZsq{}}\PYG{l+s+s1}{docs}\PYG{l+s+s1}{\PYGZsq{}}\PYG{p}{:} \PYG{n+nb+bp}{self}\PYG{p}{,}
        \PYG{p}{\PYGZcb{}}
        \PYG{k}{return} \PYG{n}{report\PYGZus{}obj}\PYG{o}{.}\PYG{n}{render}\PYG{p}{(}\PYG{l+s+s1}{\PYGZsq{}}\PYG{l+s+s1}{module.report\PYGZus{}name}\PYG{l+s+s1}{\PYGZsq{}}\PYG{p}{,} \PYG{n}{docargs}\PYG{p}{)}
\end{sphinxVerbatim}


\subsection{Reports are web pages}
\label{\detokenize{reference/reports:reports-are-web-pages}}
Reports are dynamically generated by the report module and can be accessed
directly via URL:

For example, you can access a Sale Order report in html mode by going to
http://\textless{}server-address\textgreater{}/report/html/sale.report\_saleorder/38

Or you can access the pdf version at
http://\textless{}server-address\textgreater{}/report/pdf/sale.report\_saleorder/38


\section{Mixins and Useful Classes}
\label{\detokenize{reference/mixins:reference-mixins}}\label{\detokenize{reference/mixins:wkhtmltopdf}}\label{\detokenize{reference/mixins::doc}}\label{\detokenize{reference/mixins:mixins-and-useful-classes}}
Odoo implements some useful classes and mixins that make it easy for you to add
often-used behaviours on your objects. This guide will details most of them, with
examples and use cases.


\subsection{Messaging features}
\label{\detokenize{reference/mixins:messaging-features}}\label{\detokenize{reference/mixins:reference-mixins-mail}}

\subsubsection{Messaging integration}
\label{\detokenize{reference/mixins:messaging-integration}}\label{\detokenize{reference/mixins:reference-mixins-mail-chatter}}

\paragraph{Basic messaging system}
\label{\detokenize{reference/mixins:basic-messaging-system}}
Integrating messaging features to your model is extremely easy. Simply inheriting
the \sphinxcode{\sphinxupquote{mail.thread}} model and adding the messaging fields (and their appropriate
widgets) to your form view will get you up and running in no time.

\begin{sphinxadmonition}{note}{Example}

Let’s create a simplistic model representing a business trip. Since organizing
this kind of trip usually involves a lot of people and a lot of discussion, let’s
add support for message exchange on the model.

\fvset{hllines={, ,}}%
\begin{sphinxVerbatim}[commandchars=\\\{\}]
\PYG{k}{class} \PYG{n+nc}{BusinessTrip}\PYG{p}{(}\PYG{n}{models}\PYG{o}{.}\PYG{n}{Model}\PYG{p}{)}\PYG{p}{:}
    \PYG{n}{\PYGZus{}name} \PYG{o}{=} \PYG{l+s+s1}{\PYGZsq{}}\PYG{l+s+s1}{business.trip}\PYG{l+s+s1}{\PYGZsq{}}
    \PYG{n}{\PYGZus{}inherit} \PYG{o}{=} \PYG{p}{[}\PYG{l+s+s1}{\PYGZsq{}}\PYG{l+s+s1}{mail.thread}\PYG{l+s+s1}{\PYGZsq{}}\PYG{p}{]}
    \PYG{n}{\PYGZus{}description} \PYG{o}{=} \PYG{l+s+s1}{\PYGZsq{}}\PYG{l+s+s1}{Business Trip}\PYG{l+s+s1}{\PYGZsq{}}

    \PYG{n}{name} \PYG{o}{=} \PYG{n}{fields}\PYG{o}{.}\PYG{n}{Char}\PYG{p}{(}\PYG{p}{)}
    \PYG{n}{partner\PYGZus{}id} \PYG{o}{=} \PYG{n}{fields}\PYG{o}{.}\PYG{n}{Many2one}\PYG{p}{(}\PYG{l+s+s1}{\PYGZsq{}}\PYG{l+s+s1}{res.partner}\PYG{l+s+s1}{\PYGZsq{}}\PYG{p}{,} \PYG{l+s+s1}{\PYGZsq{}}\PYG{l+s+s1}{Responsible}\PYG{l+s+s1}{\PYGZsq{}}\PYG{p}{)}
    \PYG{n}{guest\PYGZus{}ids} \PYG{o}{=} \PYG{n}{fields}\PYG{o}{.}\PYG{n}{Many2many}\PYG{p}{(}\PYG{l+s+s1}{\PYGZsq{}}\PYG{l+s+s1}{res.partner}\PYG{l+s+s1}{\PYGZsq{}}\PYG{p}{,} \PYG{l+s+s1}{\PYGZsq{}}\PYG{l+s+s1}{Participants}\PYG{l+s+s1}{\PYGZsq{}}\PYG{p}{)}
\end{sphinxVerbatim}

In the form view:

\fvset{hllines={, ,}}%
\begin{sphinxVerbatim}[commandchars=\\\{\}]
\PYG{n+nt}{\PYGZlt{}record} \PYG{n+na}{id=}\PYG{l+s}{\PYGZdq{}businness\PYGZus{}trip\PYGZus{}form\PYGZdq{}} \PYG{n+na}{model=}\PYG{l+s}{\PYGZdq{}ir.ui.view\PYGZdq{}}\PYG{n+nt}{\PYGZgt{}}
    \PYG{n+nt}{\PYGZlt{}field} \PYG{n+na}{name=}\PYG{l+s}{\PYGZdq{}name\PYGZdq{}}\PYG{n+nt}{\PYGZgt{}}business.trip.form\PYG{n+nt}{\PYGZlt{}/field\PYGZgt{}}
    \PYG{n+nt}{\PYGZlt{}field} \PYG{n+na}{name=}\PYG{l+s}{\PYGZdq{}model\PYGZdq{}}\PYG{n+nt}{\PYGZgt{}}business.trip\PYG{n+nt}{\PYGZlt{}/field\PYGZgt{}}
    \PYG{n+nt}{\PYGZlt{}field} \PYG{n+na}{name=}\PYG{l+s}{\PYGZdq{}arch\PYGZdq{}} \PYG{n+na}{type=}\PYG{l+s}{\PYGZdq{}xml\PYGZdq{}}\PYG{n+nt}{\PYGZgt{}}
        \PYG{n+nt}{\PYGZlt{}form} \PYG{n+na}{string=}\PYG{l+s}{\PYGZdq{}Business Trip\PYGZdq{}}\PYG{n+nt}{\PYGZgt{}}
            \PYG{c}{\PYGZlt{}!\PYGZhy{}\PYGZhy{}}\PYG{c}{ Your usual form view goes here}
\PYG{c}{            ...}
\PYG{c}{            Then comes chatter integration }\PYG{c}{\PYGZhy{}\PYGZhy{}\PYGZgt{}}
            \PYG{n+nt}{\PYGZlt{}div} \PYG{n+na}{class=}\PYG{l+s}{\PYGZdq{}oe\PYGZus{}chatter\PYGZdq{}}\PYG{n+nt}{\PYGZgt{}}
                \PYG{n+nt}{\PYGZlt{}field} \PYG{n+na}{name=}\PYG{l+s}{\PYGZdq{}message\PYGZus{}follower\PYGZus{}ids\PYGZdq{}} \PYG{n+na}{widget=}\PYG{l+s}{\PYGZdq{}mail\PYGZus{}followers\PYGZdq{}}\PYG{n+nt}{/\PYGZgt{}}
                \PYG{n+nt}{\PYGZlt{}field} \PYG{n+na}{name=}\PYG{l+s}{\PYGZdq{}message\PYGZus{}ids\PYGZdq{}} \PYG{n+na}{widget=}\PYG{l+s}{\PYGZdq{}mail\PYGZus{}thread\PYGZdq{}}\PYG{n+nt}{/\PYGZgt{}}
            \PYG{n+nt}{\PYGZlt{}/div\PYGZgt{}}
        \PYG{n+nt}{\PYGZlt{}/form\PYGZgt{}}
    \PYG{n+nt}{\PYGZlt{}/field\PYGZgt{}}
\PYG{n+nt}{\PYGZlt{}/record\PYGZgt{}}
\end{sphinxVerbatim}
\end{sphinxadmonition}

Once you’ve added chatter support on your model, users can easily add messages
or internal notes on any record of your model; every one of those will send a
notification (to all followers for messages, to employee (\sphinxstyleemphasis{base.group\_user})
users for internal notes). If your mail gateway and catchall address are correctly
configured, these notifications will be sent by e-mail and can be replied-to directly
from your mail client; the automatic routing system will route the answer to the
correct thread.

Server-side, some helper functions are there to help you easily send messages and
to manage followers on your record:
\paragraph{Posting messages}
\index{message\_post()}

\begin{fulllineitems}
\phantomsection\label{\detokenize{reference/mixins:message_post}}\pysiglinewithargsret{\sphinxbfcode{\sphinxupquote{message\_post}}}{\emph{self}, \emph{body=''}, \emph{subject=None}, \emph{message\_type='notification'}, \emph{subtype=None}, \emph{parent\_id=False}, \emph{attachments=None}, \emph{content\_subtype='html'}, \emph{**kwargs}}{}
Post a new message in an existing thread, returning the new
mail.message ID.
\begin{quote}\begin{description}
\item[{Parameters}] \leavevmode\begin{itemize}
\item {} 
\sphinxstyleliteralstrong{\sphinxupquote{body}} (\sphinxhref{https://docs.python.org/3/library/stdtypes.html\#str}{\sphinxstyleliteralemphasis{\sphinxupquote{str}}}) \textendash{} body of the message, usually raw HTML that will
be sanitized

\item {} 
\sphinxstyleliteralstrong{\sphinxupquote{message\_type}} (\sphinxhref{https://docs.python.org/3/library/stdtypes.html\#str}{\sphinxstyleliteralemphasis{\sphinxupquote{str}}}) \textendash{} see mail\_message.type field

\item {} 
\sphinxstyleliteralstrong{\sphinxupquote{content\_subtype}} (\sphinxhref{https://docs.python.org/3/library/stdtypes.html\#str}{\sphinxstyleliteralemphasis{\sphinxupquote{str}}}) \textendash{} if plaintext: convert body into html

\item {} 
\sphinxstyleliteralstrong{\sphinxupquote{parent\_id}} (\sphinxhref{https://docs.python.org/3/library/functions.html\#int}{\sphinxstyleliteralemphasis{\sphinxupquote{int}}}) \textendash{} handle reply to a previous message by adding the
parent partners to the message in case of private discussion

\item {} 
\sphinxstyleliteralstrong{\sphinxupquote{attachments}} (\sphinxhref{https://docs.python.org/3/library/stdtypes.html\#list}{\sphinxstyleliteralemphasis{\sphinxupquote{list}}}\sphinxstyleliteralemphasis{\sphinxupquote{(}}\sphinxhref{https://docs.python.org/3/library/stdtypes.html\#tuple}{\sphinxstyleliteralemphasis{\sphinxupquote{tuple}}}\sphinxstyleliteralemphasis{\sphinxupquote{(}}\sphinxhref{https://docs.python.org/3/library/stdtypes.html\#str}{\sphinxstyleliteralemphasis{\sphinxupquote{str}}}\sphinxstyleliteralemphasis{\sphinxupquote{,}}\sphinxhref{https://docs.python.org/3/library/stdtypes.html\#str}{\sphinxstyleliteralemphasis{\sphinxupquote{str}}}\sphinxstyleliteralemphasis{\sphinxupquote{)}}\sphinxstyleliteralemphasis{\sphinxupquote{)}}) \textendash{} list of attachment tuples in the form
\sphinxcode{\sphinxupquote{(name,content)}}, where content is NOT base64 encoded

\item {} 
\sphinxstyleliteralstrong{\sphinxupquote{**kwargs}} \textendash{} extra keyword arguments will be used as default column values for the
new mail.message record

\end{itemize}

\item[{Returns}] \leavevmode
ID of newly created mail.message

\item[{Return type}] \leavevmode
\sphinxhref{https://docs.python.org/3/library/functions.html\#int}{int}

\end{description}\end{quote}

\end{fulllineitems}



\begin{fulllineitems}
\pysigline{\sphinxbfcode{\sphinxupquote{message\_post\_with\_view(views\_or\_xmlid,~**kwargs):}}}
Helper method to send a mail / post a message using a view\_id to
render using the ir.qweb engine. This method is stand alone, because
there is nothing in template and composer that allows to handle
views in batch. This method will probably disappear when templates
handle ir ui views.
\begin{quote}\begin{description}
\item[{Parameters}] \leavevmode
\sphinxstyleliteralstrong{\sphinxupquote{or ir.ui.view record}} (\sphinxhref{https://docs.python.org/3/library/stdtypes.html\#str}{\sphinxstyleliteralemphasis{\sphinxupquote{str}}}) \textendash{} external id or record of the view that
should be sent

\end{description}\end{quote}

\end{fulllineitems}

\index{message\_post\_with\_template()}

\begin{fulllineitems}
\phantomsection\label{\detokenize{reference/mixins:message_post_with_template}}\pysiglinewithargsret{\sphinxbfcode{\sphinxupquote{message\_post\_with\_template}}}{\emph{template\_id}, \emph{**kwargs}}{}
Helper method to send a mail with a template
\begin{quote}\begin{description}
\item[{Parameters}] \leavevmode\begin{itemize}
\item {} 
\sphinxstyleliteralstrong{\sphinxupquote{template\_id}} \textendash{} the id of the template to render to create the body of the message

\item {} 
\sphinxstyleliteralstrong{\sphinxupquote{**kwargs}} \textendash{} parameter to create a mail.compose.message wizzard (which inherit from mail.message)

\end{itemize}

\end{description}\end{quote}

\end{fulllineitems}

\paragraph{Receiving messages}

These methods are called when a new e-mail is processed by the mail gateway. These
e-mails can either be new thread (if they arrive via an {\hyperref[\detokenize{reference/mixins:reference-mixins-mail-alias}]{\sphinxcrossref{\DUrole{std,std-ref}{alias}}}})
or simply replies from an existing thread. Overriding them allows you to set values
on the thread’s record depending on some values from the email itself (i.e. update
a date or an e-mail address, add CC’s addresses as followers, etc.).
\index{message\_new()}

\begin{fulllineitems}
\phantomsection\label{\detokenize{reference/mixins:message_new}}\pysiglinewithargsret{\sphinxbfcode{\sphinxupquote{message\_new}}}{\emph{msg\_dict}, \emph{custom\_values=None}}{}
Called by \sphinxcode{\sphinxupquote{message\_process}} when a new message is received
for a given thread model, if the message did not belong to
an existing thread.

The default behavior is to create a new record of the corresponding
model (based on some very basic info extracted from the message).
Additional behavior may be implemented by overriding this method.
\begin{quote}\begin{description}
\item[{Parameters}] \leavevmode\begin{itemize}
\item {} 
\sphinxstyleliteralstrong{\sphinxupquote{msg\_dict}} (\sphinxhref{https://docs.python.org/3/library/stdtypes.html\#dict}{\sphinxstyleliteralemphasis{\sphinxupquote{dict}}}) \textendash{} a map containing the email details and
attachments. See \sphinxcode{\sphinxupquote{message\_process}} and \sphinxcode{\sphinxupquote{mail.message.parse}} for details

\item {} 
\sphinxstyleliteralstrong{\sphinxupquote{custom\_values}} (\sphinxhref{https://docs.python.org/3/library/stdtypes.html\#dict}{\sphinxstyleliteralemphasis{\sphinxupquote{dict}}}) \textendash{} optional dictionary of additional
field values to pass to create() when creating the new thread record;
be careful, these values may override any other values coming from
the message

\end{itemize}

\item[{Return type}] \leavevmode
\sphinxhref{https://docs.python.org/3/library/functions.html\#int}{int}

\item[{Returns}] \leavevmode
the id of the newly created thread object

\end{description}\end{quote}

\end{fulllineitems}

\index{message\_update()}

\begin{fulllineitems}
\phantomsection\label{\detokenize{reference/mixins:message_update}}\pysiglinewithargsret{\sphinxbfcode{\sphinxupquote{message\_update}}}{\emph{msg\_dict}, \emph{update\_vals=None}}{}
Called by \sphinxcode{\sphinxupquote{message\_process}} when a new message is received
for an existing thread. The default behavior is to update the record
with \sphinxcode{\sphinxupquote{update\_vals}} taken from the incoming email.

Additional behavior may be implemented by overriding this
method.
\begin{quote}\begin{description}
\item[{Parameters}] \leavevmode\begin{itemize}
\item {} 
\sphinxstyleliteralstrong{\sphinxupquote{msg\_dict}} (\sphinxhref{https://docs.python.org/3/library/stdtypes.html\#dict}{\sphinxstyleliteralemphasis{\sphinxupquote{dict}}}) \textendash{} a map containing the email details and attachments;
see \sphinxcode{\sphinxupquote{message\_process}} and \sphinxcode{\sphinxupquote{mail.message.parse()}} for details.

\item {} 
\sphinxstyleliteralstrong{\sphinxupquote{update\_vals}} (\sphinxhref{https://docs.python.org/3/library/stdtypes.html\#dict}{\sphinxstyleliteralemphasis{\sphinxupquote{dict}}}) \textendash{} a dict containing values to update records given
their ids; if the dict is None or is void, no write operation is performed.

\end{itemize}

\item[{Returns}] \leavevmode
True

\end{description}\end{quote}

\end{fulllineitems}

\paragraph{Followers management}
\index{message\_subscribe()}

\begin{fulllineitems}
\phantomsection\label{\detokenize{reference/mixins:message_subscribe}}\pysiglinewithargsret{\sphinxbfcode{\sphinxupquote{message\_subscribe}}}{\emph{partner\_ids=None}, \emph{channel\_ids=None}, \emph{subtype\_ids=None}, \emph{force=True}}{}
Add partners to the records followers.
\begin{quote}\begin{description}
\item[{Parameters}] \leavevmode\begin{itemize}
\item {} 
\sphinxstyleliteralstrong{\sphinxupquote{partner\_ids}} (\sphinxhref{https://docs.python.org/3/library/stdtypes.html\#list}{\sphinxstyleliteralemphasis{\sphinxupquote{list}}}\sphinxstyleliteralemphasis{\sphinxupquote{(}}\sphinxhref{https://docs.python.org/3/library/functions.html\#int}{\sphinxstyleliteralemphasis{\sphinxupquote{int}}}\sphinxstyleliteralemphasis{\sphinxupquote{)}}) \textendash{} IDs of the partners that will be subscribed
to the record

\item {} 
\sphinxstyleliteralstrong{\sphinxupquote{channel\_ids}} (\sphinxhref{https://docs.python.org/3/library/stdtypes.html\#list}{\sphinxstyleliteralemphasis{\sphinxupquote{list}}}\sphinxstyleliteralemphasis{\sphinxupquote{(}}\sphinxhref{https://docs.python.org/3/library/functions.html\#int}{\sphinxstyleliteralemphasis{\sphinxupquote{int}}}\sphinxstyleliteralemphasis{\sphinxupquote{)}}) \textendash{} IDs of the channels that will be subscribed
to the record

\item {} 
\sphinxstyleliteralstrong{\sphinxupquote{subtype\_ids}} (\sphinxhref{https://docs.python.org/3/library/stdtypes.html\#list}{\sphinxstyleliteralemphasis{\sphinxupquote{list}}}\sphinxstyleliteralemphasis{\sphinxupquote{(}}\sphinxhref{https://docs.python.org/3/library/functions.html\#int}{\sphinxstyleliteralemphasis{\sphinxupquote{int}}}\sphinxstyleliteralemphasis{\sphinxupquote{)}}) \textendash{} IDs of the subtypes that the channels/partners
will be subscribed to (defaults to the default subtypes if \sphinxcode{\sphinxupquote{None}})

\item {} 
\sphinxstyleliteralstrong{\sphinxupquote{force}} \textendash{} if True, delete existing followers before creating new one
using the subtypes given in the parameters

\end{itemize}

\item[{Returns}] \leavevmode
Success/Failure

\item[{Return type}] \leavevmode
\sphinxhref{https://docs.python.org/3/library/functions.html\#bool}{bool}

\end{description}\end{quote}

\end{fulllineitems}

\index{message\_subscribe\_users()}

\begin{fulllineitems}
\phantomsection\label{\detokenize{reference/mixins:message_subscribe_users}}\pysiglinewithargsret{\sphinxbfcode{\sphinxupquote{message\_subscribe\_users}}}{\emph{user\_ids=None}, \emph{subtype\_ids=None}}{}
Wrapper on message\_subscribe, using users instead of partners.
\begin{quote}\begin{description}
\item[{Parameters}] \leavevmode\begin{itemize}
\item {} 
\sphinxstyleliteralstrong{\sphinxupquote{user\_ids}} (\sphinxhref{https://docs.python.org/3/library/stdtypes.html\#list}{\sphinxstyleliteralemphasis{\sphinxupquote{list}}}\sphinxstyleliteralemphasis{\sphinxupquote{(}}\sphinxhref{https://docs.python.org/3/library/functions.html\#int}{\sphinxstyleliteralemphasis{\sphinxupquote{int}}}\sphinxstyleliteralemphasis{\sphinxupquote{)}}) \textendash{} IDs of the users that will be subscribed
to the record; if \sphinxcode{\sphinxupquote{None}}, subscribe the current user instead.

\item {} 
\sphinxstyleliteralstrong{\sphinxupquote{subtype\_ids}} (\sphinxhref{https://docs.python.org/3/library/stdtypes.html\#list}{\sphinxstyleliteralemphasis{\sphinxupquote{list}}}\sphinxstyleliteralemphasis{\sphinxupquote{(}}\sphinxhref{https://docs.python.org/3/library/functions.html\#int}{\sphinxstyleliteralemphasis{\sphinxupquote{int}}}\sphinxstyleliteralemphasis{\sphinxupquote{)}}) \textendash{} IDs of the subtypes that the channels/partners
will be subscribed to

\end{itemize}

\item[{Returns}] \leavevmode
Success

\item[{Return type}] \leavevmode
\sphinxhref{https://docs.python.org/3/library/functions.html\#bool}{bool}

\end{description}\end{quote}

\end{fulllineitems}

\index{message\_unsubscribe()}

\begin{fulllineitems}
\phantomsection\label{\detokenize{reference/mixins:message_unsubscribe}}\pysiglinewithargsret{\sphinxbfcode{\sphinxupquote{message\_unsubscribe}}}{\emph{partner\_ids=None}, \emph{channel\_ids=None}}{}
Remove partners from the record’s followers.
\begin{quote}\begin{description}
\item[{Parameters}] \leavevmode\begin{itemize}
\item {} 
\sphinxstyleliteralstrong{\sphinxupquote{partner\_ids}} (\sphinxhref{https://docs.python.org/3/library/stdtypes.html\#list}{\sphinxstyleliteralemphasis{\sphinxupquote{list}}}\sphinxstyleliteralemphasis{\sphinxupquote{(}}\sphinxhref{https://docs.python.org/3/library/functions.html\#int}{\sphinxstyleliteralemphasis{\sphinxupquote{int}}}\sphinxstyleliteralemphasis{\sphinxupquote{)}}) \textendash{} IDs of the partners that will be subscribed
to the record

\item {} 
\sphinxstyleliteralstrong{\sphinxupquote{channel\_ids}} (\sphinxhref{https://docs.python.org/3/library/stdtypes.html\#list}{\sphinxstyleliteralemphasis{\sphinxupquote{list}}}\sphinxstyleliteralemphasis{\sphinxupquote{(}}\sphinxhref{https://docs.python.org/3/library/functions.html\#int}{\sphinxstyleliteralemphasis{\sphinxupquote{int}}}\sphinxstyleliteralemphasis{\sphinxupquote{)}}) \textendash{} IDs of the channels that will be subscribed
to the record

\end{itemize}

\item[{Returns}] \leavevmode
True

\item[{Return type}] \leavevmode
\sphinxhref{https://docs.python.org/3/library/functions.html\#bool}{bool}

\end{description}\end{quote}

\end{fulllineitems}

\index{message\_unsubscribe\_users()}

\begin{fulllineitems}
\phantomsection\label{\detokenize{reference/mixins:message_unsubscribe_users}}\pysiglinewithargsret{\sphinxbfcode{\sphinxupquote{message\_unsubscribe\_users}}}{\emph{user\_ids=None}}{}
Wrapper on message\_subscribe, using users.
\begin{quote}\begin{description}
\item[{Parameters}] \leavevmode
\sphinxstyleliteralstrong{\sphinxupquote{user\_ids}} (\sphinxhref{https://docs.python.org/3/library/stdtypes.html\#list}{\sphinxstyleliteralemphasis{\sphinxupquote{list}}}\sphinxstyleliteralemphasis{\sphinxupquote{(}}\sphinxhref{https://docs.python.org/3/library/functions.html\#int}{\sphinxstyleliteralemphasis{\sphinxupquote{int}}}\sphinxstyleliteralemphasis{\sphinxupquote{)}}) \textendash{} IDs of the users that will be unsubscribed
to the record; if None, unsubscribe the current user instead.

\item[{Returns}] \leavevmode
True

\item[{Return type}] \leavevmode
\sphinxhref{https://docs.python.org/3/library/functions.html\#bool}{bool}

\end{description}\end{quote}

\end{fulllineitems}



\paragraph{Logging changes}
\label{\detokenize{reference/mixins:logging-changes}}
The \sphinxcode{\sphinxupquote{mail}} module adds a powerful tracking system on fields, allowing you
to log changes to specific fields in the record’s chatter.

To add tracking to a field, simple add the track\_visibility attribute with the
value \sphinxcode{\sphinxupquote{onchange}} (if it should be displayed in the notification only if the
field changed) or \sphinxcode{\sphinxupquote{always}} (if the value should always be displayed in change
notifications even if this particular field did not change - useful to make
notification more explanatory by always adding the name field, for example).

\begin{sphinxadmonition}{note}{Example}

Let’s track changes on the name and responsible of our business trips:

\fvset{hllines={, ,}}%
\begin{sphinxVerbatim}[commandchars=\\\{\}]
\PYG{k}{class} \PYG{n+nc}{BusinessTrip}\PYG{p}{(}\PYG{n}{models}\PYG{o}{.}\PYG{n}{Model}\PYG{p}{)}\PYG{p}{:}
    \PYG{n}{\PYGZus{}name} \PYG{o}{=} \PYG{l+s+s1}{\PYGZsq{}}\PYG{l+s+s1}{business.trip}\PYG{l+s+s1}{\PYGZsq{}}
    \PYG{n}{\PYGZus{}inherit} \PYG{o}{=} \PYG{p}{[}\PYG{l+s+s1}{\PYGZsq{}}\PYG{l+s+s1}{mail.thread}\PYG{l+s+s1}{\PYGZsq{}}\PYG{p}{]}
    \PYG{n}{\PYGZus{}description} \PYG{o}{=} \PYG{l+s+s1}{\PYGZsq{}}\PYG{l+s+s1}{Business Trip}\PYG{l+s+s1}{\PYGZsq{}}

    \PYG{n}{name} \PYG{o}{=} \PYG{n}{fields}\PYG{o}{.}\PYG{n}{Char}\PYG{p}{(}\PYG{n}{track\PYGZus{}visibility}\PYG{o}{=}\PYG{l+s+s1}{\PYGZsq{}}\PYG{l+s+s1}{always}\PYG{l+s+s1}{\PYGZsq{}}\PYG{p}{)}
    \PYG{n}{partner\PYGZus{}id} \PYG{o}{=} \PYG{n}{fields}\PYG{o}{.}\PYG{n}{Many2one}\PYG{p}{(}\PYG{l+s+s1}{\PYGZsq{}}\PYG{l+s+s1}{res.partner}\PYG{l+s+s1}{\PYGZsq{}}\PYG{p}{,} \PYG{l+s+s1}{\PYGZsq{}}\PYG{l+s+s1}{Responsible}\PYG{l+s+s1}{\PYGZsq{}}\PYG{p}{,}
                                 \PYG{n}{track\PYGZus{}visibility}\PYG{o}{=}\PYG{l+s+s1}{\PYGZsq{}}\PYG{l+s+s1}{onchange}\PYG{l+s+s1}{\PYGZsq{}}\PYG{p}{)}
    \PYG{n}{guest\PYGZus{}ids} \PYG{o}{=} \PYG{n}{fields}\PYG{o}{.}\PYG{n}{Many2many}\PYG{p}{(}\PYG{l+s+s1}{\PYGZsq{}}\PYG{l+s+s1}{res.partner}\PYG{l+s+s1}{\PYGZsq{}}\PYG{p}{,} \PYG{l+s+s1}{\PYGZsq{}}\PYG{l+s+s1}{Participants}\PYG{l+s+s1}{\PYGZsq{}}\PYG{p}{)}
\end{sphinxVerbatim}

From now on, every change to a trip’s name or responsible will log a note
on the record. The \sphinxcode{\sphinxupquote{name}} field will be displayed in the notification as
well to give more context about the notification (even if the name did not
change).
\end{sphinxadmonition}


\paragraph{Subtypes}
\label{\detokenize{reference/mixins:subtypes}}
Subtypes give you more granular control over messages. Subtypes act as a classification
system for notifications, allowing subscribers to a document to customize the
subtype of notifications they wish to receive.

Subtypes are created as data in your module; the model has the following fields:
\begin{description}
\item[{\sphinxcode{\sphinxupquote{name}} (mandatory) - {\hyperref[\detokenize{reference/orm:odoo.fields.Char}]{\sphinxcrossref{\sphinxcode{\sphinxupquote{Char}}}}}}] \leavevmode
name of the subtype, will be displayed in the notification customization
popup

\item[{\sphinxcode{\sphinxupquote{description}} - {\hyperref[\detokenize{reference/orm:odoo.fields.Char}]{\sphinxcrossref{\sphinxcode{\sphinxupquote{Char}}}}}}] \leavevmode
description that will be added in the message posted for this
subtype. If void, the name will be added instead

\item[{\sphinxcode{\sphinxupquote{internal}} - {\hyperref[\detokenize{reference/orm:odoo.fields.Boolean}]{\sphinxcrossref{\sphinxcode{\sphinxupquote{Boolean}}}}}}] \leavevmode
messages with internal subtypes will be visible only by employees,
aka members of the \sphinxcode{\sphinxupquote{base.group\_user}} group

\item[{\sphinxcode{\sphinxupquote{parent\_id}} - {\hyperref[\detokenize{reference/orm:odoo.fields.Many2one}]{\sphinxcrossref{\sphinxcode{\sphinxupquote{Many2one}}}}}}] \leavevmode
link subtypes for automatic subscription; for example project subtypes are
linked to task subtypes through this link. When someone is subscribed to
a project, he will be subscribed to all tasks of this project with
subtypes found using the parent subtype

\item[{\sphinxcode{\sphinxupquote{relation\_field}} - {\hyperref[\detokenize{reference/orm:odoo.fields.Char}]{\sphinxcrossref{\sphinxcode{\sphinxupquote{Char}}}}}}] \leavevmode
as an example, when linking project and tasks subtypes, the relation
field is the project\_id field of tasks

\item[{\sphinxcode{\sphinxupquote{res\_model}} - {\hyperref[\detokenize{reference/orm:odoo.fields.Char}]{\sphinxcrossref{\sphinxcode{\sphinxupquote{Char}}}}}}] \leavevmode
model the subtype applies to; if False, this subtype applies to all models

\item[{\sphinxcode{\sphinxupquote{default}} - {\hyperref[\detokenize{reference/orm:odoo.fields.Boolean}]{\sphinxcrossref{\sphinxcode{\sphinxupquote{Boolean}}}}}}] \leavevmode
wether the subtype is activated by default when subscribing

\item[{\sphinxcode{\sphinxupquote{sequence}} - {\hyperref[\detokenize{reference/orm:odoo.fields.Integer}]{\sphinxcrossref{\sphinxcode{\sphinxupquote{Integer}}}}}}] \leavevmode
used to order subtypes in the notification customization popup

\item[{\sphinxcode{\sphinxupquote{hidden}} - {\hyperref[\detokenize{reference/orm:odoo.fields.Boolean}]{\sphinxcrossref{\sphinxcode{\sphinxupquote{Boolean}}}}}}] \leavevmode
wether the subtype is hidden in the notification customization popup

\end{description}

Interfacing subtypes with field tracking allows to subscribe to different kind
of notifications depending on what might interest users. To do this, you
can override the \sphinxcode{\sphinxupquote{\_track\_subtype()}} function:
\index{\_track\_subtype()}

\begin{fulllineitems}
\phantomsection\label{\detokenize{reference/mixins:_track_subtype}}\pysiglinewithargsret{\sphinxbfcode{\sphinxupquote{\_track\_subtype}}}{\emph{init\_values}}{}
Give the subtype triggered by the changes on the record according
to values that have been updated.
\begin{quote}\begin{description}
\item[{Parameters}] \leavevmode
\sphinxstyleliteralstrong{\sphinxupquote{init\_values}} (\sphinxhref{https://docs.python.org/3/library/stdtypes.html\#dict}{\sphinxstyleliteralemphasis{\sphinxupquote{dict}}}) \textendash{} the original values of the record; only modified fields
are present in the dict

\item[{Returns}] \leavevmode
a subtype’s full external id or False if no subtype is triggered

\end{description}\end{quote}

\end{fulllineitems}


\begin{sphinxadmonition}{note}{Example}

Let’s add a \sphinxcode{\sphinxupquote{state}} field on our example class and trigger a notification
with a specific subtype when this field change values.

First, let’s define our subtype:

\fvset{hllines={, ,}}%
\begin{sphinxVerbatim}[commandchars=\\\{\}]
\PYG{n+nt}{\PYGZlt{}record} \PYG{n+na}{id=}\PYG{l+s}{\PYGZdq{}mt\PYGZus{}state\PYGZus{}change\PYGZdq{}} \PYG{n+na}{model=}\PYG{l+s}{\PYGZdq{}mail.message.subtype\PYGZdq{}}\PYG{n+nt}{\PYGZgt{}}
    \PYG{n+nt}{\PYGZlt{}field} \PYG{n+na}{name=}\PYG{l+s}{\PYGZdq{}name\PYGZdq{}}\PYG{n+nt}{\PYGZgt{}}Trip confirmed\PYG{n+nt}{\PYGZlt{}/field\PYGZgt{}}
    \PYG{n+nt}{\PYGZlt{}field} \PYG{n+na}{name=}\PYG{l+s}{\PYGZdq{}res\PYGZus{}model\PYGZdq{}}\PYG{n+nt}{\PYGZgt{}}business.trip\PYG{n+nt}{\PYGZlt{}/field\PYGZgt{}}
    \PYG{n+nt}{\PYGZlt{}field} \PYG{n+na}{name=}\PYG{l+s}{\PYGZdq{}default\PYGZdq{}} \PYG{n+na}{eval=}\PYG{l+s}{\PYGZdq{}True\PYGZdq{}}\PYG{n+nt}{/\PYGZgt{}}
    \PYG{n+nt}{\PYGZlt{}field} \PYG{n+na}{name=}\PYG{l+s}{\PYGZdq{}description\PYGZdq{}}\PYG{n+nt}{\PYGZgt{}}Business Trip confirmed!\PYG{n+nt}{\PYGZlt{}/field\PYGZgt{}}
\PYG{n+nt}{\PYGZlt{}/record\PYGZgt{}}
\end{sphinxVerbatim}

Then, we need to override the \sphinxcode{\sphinxupquote{track\_subtype()}} function. This function
is called by the tracking system to know which subtype should be used depending
on the change currently being applied. In our case, we want to use our shiny new
subtype when the \sphinxcode{\sphinxupquote{state}} field changes from \sphinxstyleemphasis{draft} to \sphinxstyleemphasis{confirmed}:

\fvset{hllines={, ,}}%
\begin{sphinxVerbatim}[commandchars=\\\{\}]
\PYG{k}{class} \PYG{n+nc}{BusinessTrip}\PYG{p}{(}\PYG{n}{models}\PYG{o}{.}\PYG{n}{Model}\PYG{p}{)}\PYG{p}{:}
    \PYG{n}{\PYGZus{}name} \PYG{o}{=} \PYG{l+s+s1}{\PYGZsq{}}\PYG{l+s+s1}{business.trip}\PYG{l+s+s1}{\PYGZsq{}}
    \PYG{n}{\PYGZus{}inherit} \PYG{o}{=} \PYG{p}{[}\PYG{l+s+s1}{\PYGZsq{}}\PYG{l+s+s1}{mail.thread}\PYG{l+s+s1}{\PYGZsq{}}\PYG{p}{]}
    \PYG{n}{\PYGZus{}description} \PYG{o}{=} \PYG{l+s+s1}{\PYGZsq{}}\PYG{l+s+s1}{Business Trip}\PYG{l+s+s1}{\PYGZsq{}}

    \PYG{n}{name} \PYG{o}{=} \PYG{n}{fields}\PYG{o}{.}\PYG{n}{Char}\PYG{p}{(}\PYG{n}{track\PYGZus{}visibility}\PYG{o}{=}\PYG{l+s+s1}{\PYGZsq{}}\PYG{l+s+s1}{onchange}\PYG{l+s+s1}{\PYGZsq{}}\PYG{p}{)}
    \PYG{n}{partner\PYGZus{}id} \PYG{o}{=} \PYG{n}{fields}\PYG{o}{.}\PYG{n}{Many2one}\PYG{p}{(}\PYG{l+s+s1}{\PYGZsq{}}\PYG{l+s+s1}{res.partner}\PYG{l+s+s1}{\PYGZsq{}}\PYG{p}{,} \PYG{l+s+s1}{\PYGZsq{}}\PYG{l+s+s1}{Responsible}\PYG{l+s+s1}{\PYGZsq{}}\PYG{p}{,}
                                 \PYG{n}{track\PYGZus{}visibility}\PYG{o}{=}\PYG{l+s+s1}{\PYGZsq{}}\PYG{l+s+s1}{onchange}\PYG{l+s+s1}{\PYGZsq{}}\PYG{p}{)}
    \PYG{n}{guest\PYGZus{}ids} \PYG{o}{=} \PYG{n}{fields}\PYG{o}{.}\PYG{n}{Many2many}\PYG{p}{(}\PYG{l+s+s1}{\PYGZsq{}}\PYG{l+s+s1}{res.partner}\PYG{l+s+s1}{\PYGZsq{}}\PYG{p}{,} \PYG{l+s+s1}{\PYGZsq{}}\PYG{l+s+s1}{Participants}\PYG{l+s+s1}{\PYGZsq{}}\PYG{p}{)}
    \PYG{n}{state} \PYG{o}{=} \PYG{n}{fields}\PYG{o}{.}\PYG{n}{Selection}\PYG{p}{(}\PYG{p}{[}\PYG{p}{(}\PYG{l+s+s1}{\PYGZsq{}}\PYG{l+s+s1}{draft}\PYG{l+s+s1}{\PYGZsq{}}\PYG{p}{,} \PYG{l+s+s1}{\PYGZsq{}}\PYG{l+s+s1}{New}\PYG{l+s+s1}{\PYGZsq{}}\PYG{p}{)}\PYG{p}{,} \PYG{p}{(}\PYG{l+s+s1}{\PYGZsq{}}\PYG{l+s+s1}{confirmed}\PYG{l+s+s1}{\PYGZsq{}}\PYG{p}{,} \PYG{l+s+s1}{\PYGZsq{}}\PYG{l+s+s1}{Confirmed}\PYG{l+s+s1}{\PYGZsq{}}\PYG{p}{)}\PYG{p}{]}\PYG{p}{,}
                             \PYG{n}{track\PYGZus{}visibility}\PYG{o}{=}\PYG{l+s+s1}{\PYGZsq{}}\PYG{l+s+s1}{onchange}\PYG{l+s+s1}{\PYGZsq{}}\PYG{p}{)}

    \PYG{k}{def} \PYG{n+nf}{\PYGZus{}track\PYGZus{}subtype}\PYG{p}{(}\PYG{n+nb+bp}{self}\PYG{p}{,} \PYG{n}{init\PYGZus{}values}\PYG{p}{)}\PYG{p}{:}
        \PYG{c+c1}{\PYGZsh{} init\PYGZus{}values contains the modified fields\PYGZsq{} values before the changes}
        \PYG{c+c1}{\PYGZsh{}}
        \PYG{c+c1}{\PYGZsh{} the applied values can be accessed on the record as they are already}
        \PYG{c+c1}{\PYGZsh{} in cache}
        \PYG{n+nb+bp}{self}\PYG{o}{.}\PYG{n}{ensure\PYGZus{}one}\PYG{p}{(}\PYG{p}{)}
        \PYG{k}{if} \PYG{l+s+s1}{\PYGZsq{}}\PYG{l+s+s1}{state}\PYG{l+s+s1}{\PYGZsq{}} \PYG{o+ow}{in} \PYG{n}{init\PYGZus{}values} \PYG{o+ow}{and} \PYG{n+nb+bp}{self}\PYG{o}{.}\PYG{n}{state} \PYG{o}{==} \PYG{l+s+s1}{\PYGZsq{}}\PYG{l+s+s1}{confirmed}\PYG{l+s+s1}{\PYGZsq{}}\PYG{p}{:}
            \PYG{k}{return} \PYG{l+s+s1}{\PYGZsq{}}\PYG{l+s+s1}{my\PYGZus{}module.mt\PYGZus{}state\PYGZus{}change}\PYG{l+s+s1}{\PYGZsq{}}  \PYG{c+c1}{\PYGZsh{} Full external id}
        \PYG{k}{return} \PYG{n+nb}{super}\PYG{p}{(}\PYG{n}{BusinessTrip}\PYG{p}{,} \PYG{n+nb+bp}{self}\PYG{p}{)}\PYG{o}{.}\PYG{n}{\PYGZus{}track\PYGZus{}subtype}\PYG{p}{(}\PYG{n}{init\PYGZus{}values}\PYG{p}{)}
\end{sphinxVerbatim}
\end{sphinxadmonition}


\paragraph{Customizing notifications}
\label{\detokenize{reference/mixins:customizing-notifications}}
When sending notifications to followers, it can be quite useful to add buttons in
the template to allow quick actions directly from the e-mail. Even a simple button
to link directly to the record’s form view can be useful; however in most cases
you don’t want to display these buttons to portal users.

The notification system allows customizing notification templates in the following
ways:
\begin{itemize}
\item {} 
Display \sphinxstyleemphasis{Access Buttons}: these buttons are visible at the top of the notification
e-mail and allow the recipient to directly access the form view of the record

\item {} 
Display \sphinxstyleemphasis{Follow Buttons}: these buttons allow the recipient to
directly quickly subscribe from the record

\item {} 
Display \sphinxstyleemphasis{Unfollow Buttons}: these buttons allow the recipient to
directly quickly unsubscribe from the record

\item {} 
Display \sphinxstyleemphasis{Custom Action Buttons}: these buttons are calls to specific routes
and allow you to make some useful actions directly available from the e-mail (i.e.
converting a lead to an opportunity, validating an expense sheet for an
Expense Manager, etc.)

\end{itemize}

These buttons settings can be applied to different groups that you can define
yourself by overriding the function \sphinxcode{\sphinxupquote{\_notification\_recipients}}.
\index{\_notification\_recipients()}

\begin{fulllineitems}
\phantomsection\label{\detokenize{reference/mixins:_notification_recipients}}\pysiglinewithargsret{\sphinxbfcode{\sphinxupquote{\_notification\_recipients}}}{\emph{message}, \emph{groups}}{}
Give the subtype triggered by the changes on the record according
to values that have been updated.
\begin{quote}\begin{description}
\item[{Parameters}] \leavevmode\begin{itemize}
\item {} 
\sphinxstyleliteralstrong{\sphinxupquote{message}} (\sphinxstyleliteralemphasis{\sphinxupquote{record}}) \textendash{} \sphinxcode{\sphinxupquote{mail.message}} record currently being sent

\item {} 
\sphinxstyleliteralstrong{\sphinxupquote{groups}} (\sphinxhref{https://docs.python.org/3/library/stdtypes.html\#list}{\sphinxstyleliteralemphasis{\sphinxupquote{list}}}\sphinxstyleliteralemphasis{\sphinxupquote{(}}\sphinxhref{https://docs.python.org/3/library/stdtypes.html\#tuple}{\sphinxstyleliteralemphasis{\sphinxupquote{tuple}}}\sphinxstyleliteralemphasis{\sphinxupquote{)}}) \textendash{} 
list of tuple of the form (group\_name, group\_func,group\_data) where:
\begin{description}
\item[{group\_name}] \leavevmode
is an identifier used only to be able to override and manipulate
groups. Default groups are \sphinxcode{\sphinxupquote{user}} (recipients linked to an employee user),
\sphinxcode{\sphinxupquote{portal}} (recipients linked to a portal user) and \sphinxcode{\sphinxupquote{customer}} (recipients not
linked to any user). An example of override use would be to add a group
linked to a res.groups like Hr Officers to set specific action buttons to
them.

\item[{group\_func}] \leavevmode
is a function pointer taking a partner record as parameter. This
method will be applied on recipients to know whether they belong to a given
group or not. Only first matching group is kept. Evaluation order is the
list order.

\item[{group\_data}] \leavevmode
is a dict containing parameters for the notification email with the following
possible keys - values:
\begin{itemize}
\item {} \begin{description}
\item[{has\_button\_access}] \leavevmode
whether to display Access \textless{}Document\textgreater{} in email. True by default for
new groups, False for portal / customer.

\end{description}

\item {} \begin{description}
\item[{button\_access}] \leavevmode
dict with url and title of the button

\end{description}

\item {} \begin{description}
\item[{has\_button\_follow}] \leavevmode
whether to display Follow in email (if recipient is not currently
following the thread). True by default for new groups, False for
portal / customer.

\end{description}

\item {} \begin{description}
\item[{button\_follow}] \leavevmode
dict with url adn title of the button

\end{description}

\item {} \begin{description}
\item[{has\_button\_unfollow}] \leavevmode
whether to display Unfollow in email (if recipient is currently following the thread).
True by default for new groups, False for portal / customer.

\end{description}

\item {} \begin{description}
\item[{button\_unfollow}] \leavevmode
dict with url and title of the button

\end{description}

\item {} \begin{description}
\item[{actions}] \leavevmode
list of action buttons to display in the notification email.
Each action is a dict containing url and title of the button.

\end{description}

\end{itemize}

\end{description}


\end{itemize}

\item[{Returns}] \leavevmode
a subtype’s full external id or False if no subtype is triggered

\end{description}\end{quote}

\end{fulllineitems}


The urls in the actions list can be generated automatically by calling the
\sphinxcode{\sphinxupquote{\_notification\_link\_helper()}} function:
\index{\_notification\_link\_helper()}

\begin{fulllineitems}
\phantomsection\label{\detokenize{reference/mixins:_notification_link_helper}}\pysiglinewithargsret{\sphinxbfcode{\sphinxupquote{\_notification\_link\_helper}}}{\emph{self}, \emph{link\_type}, \emph{**kwargs}}{}
Generate a link for the given type on the current record (or on a specific
record if the kwargs \sphinxcode{\sphinxupquote{model}} and \sphinxcode{\sphinxupquote{res\_id}} are set).
\begin{quote}\begin{description}
\item[{Parameters}] \leavevmode
\sphinxstyleliteralstrong{\sphinxupquote{link\_type}} (\sphinxhref{https://docs.python.org/3/library/stdtypes.html\#str}{\sphinxstyleliteralemphasis{\sphinxupquote{str}}}) \textendash{} 
link type to be generated; can be any of these values:
\begin{description}
\item[{\sphinxcode{\sphinxupquote{view}}}] \leavevmode
link to form view of the record

\item[{\sphinxcode{\sphinxupquote{assign}}}] \leavevmode
assign the logged user to the \sphinxcode{\sphinxupquote{user\_id}} field of
the record (if it exists)

\item[{\sphinxcode{\sphinxupquote{follow}}}] \leavevmode
self-explanatory

\item[{\sphinxcode{\sphinxupquote{unfollow}}}] \leavevmode
self-explanatory

\item[{\sphinxcode{\sphinxupquote{method}}}] \leavevmode
call a method on the record; the method’s name should be
provided as the kwarg \sphinxcode{\sphinxupquote{method}}

\item[{\sphinxcode{\sphinxupquote{new}}}] \leavevmode
open an empty form view for a new record; you can specify
a specific action by providing its id (database id or fully resolved
external id) in the kwarg \sphinxcode{\sphinxupquote{action\_id}}

\end{description}


\item[{Returns}] \leavevmode
link of the type selected for the record

\item[{Return type}] \leavevmode
\sphinxhref{https://docs.python.org/3/library/stdtypes.html\#str}{str}

\end{description}\end{quote}

\end{fulllineitems}


\begin{sphinxadmonition}{note}{Example}

Let’s add a custom button to the Business Trip state change notification;
this button will reset the state to Draft and will be only visible to a member
of the (imaginary) group Travel Manager (\sphinxcode{\sphinxupquote{business.group\_trip\_manager}})

\fvset{hllines={, ,}}%
\begin{sphinxVerbatim}[commandchars=\\\{\}]
\PYG{k}{class} \PYG{n+nc}{BusinessTrip}\PYG{p}{(}\PYG{n}{models}\PYG{o}{.}\PYG{n}{Model}\PYG{p}{)}\PYG{p}{:}
    \PYG{n}{\PYGZus{}name} \PYG{o}{=} \PYG{l+s+s1}{\PYGZsq{}}\PYG{l+s+s1}{business.trip}\PYG{l+s+s1}{\PYGZsq{}}
    \PYG{n}{\PYGZus{}inherit} \PYG{o}{=} \PYG{p}{[}\PYG{l+s+s1}{\PYGZsq{}}\PYG{l+s+s1}{mail.thread}\PYG{l+s+s1}{\PYGZsq{}}\PYG{p}{,} \PYG{l+s+s1}{\PYGZsq{}}\PYG{l+s+s1}{mail.alias.mixin}\PYG{l+s+s1}{\PYGZsq{}}\PYG{p}{]}
    \PYG{n}{\PYGZus{}description} \PYG{o}{=} \PYG{l+s+s1}{\PYGZsq{}}\PYG{l+s+s1}{Business Trip}\PYG{l+s+s1}{\PYGZsq{}}

    \PYG{c+c1}{\PYGZsh{} Pevious code goes here}

    \PYG{k}{def} \PYG{n+nf}{action\PYGZus{}cancel}\PYG{p}{(}\PYG{n+nb+bp}{self}\PYG{p}{)}\PYG{p}{:}
        \PYG{n+nb+bp}{self}\PYG{o}{.}\PYG{n}{write}\PYG{p}{(}\PYG{p}{\PYGZob{}}\PYG{l+s+s1}{\PYGZsq{}}\PYG{l+s+s1}{state}\PYG{l+s+s1}{\PYGZsq{}}\PYG{p}{:} \PYG{l+s+s1}{\PYGZsq{}}\PYG{l+s+s1}{draft}\PYG{l+s+s1}{\PYGZsq{}}\PYG{p}{\PYGZcb{}}\PYG{p}{)}

    \PYG{k}{def} \PYG{n+nf}{\PYGZus{}notification\PYGZus{}recipients}\PYG{p}{(}\PYG{n+nb+bp}{self}\PYG{p}{,} \PYG{n}{message}\PYG{p}{,} \PYG{n}{groups}\PYG{p}{)}\PYG{p}{:}
        \PYG{l+s+sd}{\PYGZdq{}\PYGZdq{}\PYGZdq{} Handle Trip Manager recipients that can cancel the trip at the last}
\PYG{l+s+sd}{        minute and kill all the fun. \PYGZdq{}\PYGZdq{}\PYGZdq{}}
        \PYG{n}{groups} \PYG{o}{=} \PYG{n+nb}{super}\PYG{p}{(}\PYG{n}{BusinessTrip}\PYG{p}{,} \PYG{n+nb+bp}{self}\PYG{p}{)}\PYG{o}{.}\PYG{n}{\PYGZus{}notification\PYGZus{}recipients}\PYG{p}{(}\PYG{n}{message}\PYG{p}{,} \PYG{n}{groups}\PYG{p}{)}

        \PYG{n+nb+bp}{self}\PYG{o}{.}\PYG{n}{ensure\PYGZus{}one}\PYG{p}{(}\PYG{p}{)}
        \PYG{k}{if} \PYG{n+nb+bp}{self}\PYG{o}{.}\PYG{n}{state} \PYG{o}{==} \PYG{l+s+s1}{\PYGZsq{}}\PYG{l+s+s1}{confirmed}\PYG{l+s+s1}{\PYGZsq{}}\PYG{p}{:}
            \PYG{n}{app\PYGZus{}action} \PYG{o}{=} \PYG{n+nb+bp}{self}\PYG{o}{.}\PYG{n}{\PYGZus{}notification\PYGZus{}link\PYGZus{}helper}\PYG{p}{(}\PYG{l+s+s1}{\PYGZsq{}}\PYG{l+s+s1}{method}\PYG{l+s+s1}{\PYGZsq{}}\PYG{p}{,}
                                \PYG{n}{method}\PYG{o}{=}\PYG{l+s+s1}{\PYGZsq{}}\PYG{l+s+s1}{action\PYGZus{}cancel}\PYG{l+s+s1}{\PYGZsq{}}\PYG{p}{)}
            \PYG{n}{trip\PYGZus{}actions} \PYG{o}{=} \PYG{p}{[}\PYG{p}{\PYGZob{}}\PYG{l+s+s1}{\PYGZsq{}}\PYG{l+s+s1}{url}\PYG{l+s+s1}{\PYGZsq{}}\PYG{p}{:} \PYG{n}{app\PYGZus{}action}\PYG{p}{,} \PYG{l+s+s1}{\PYGZsq{}}\PYG{l+s+s1}{title}\PYG{l+s+s1}{\PYGZsq{}}\PYG{p}{:} \PYG{n}{\PYGZus{}}\PYG{p}{(}\PYG{l+s+s1}{\PYGZsq{}}\PYG{l+s+s1}{Cancel}\PYG{l+s+s1}{\PYGZsq{}}\PYG{p}{)}\PYG{p}{\PYGZcb{}}\PYG{p}{]}

        \PYG{n}{new\PYGZus{}group} \PYG{o}{=} \PYG{p}{(}
            \PYG{l+s+s1}{\PYGZsq{}}\PYG{l+s+s1}{group\PYGZus{}trip\PYGZus{}manager}\PYG{l+s+s1}{\PYGZsq{}}\PYG{p}{,}
            \PYG{k}{lambda} \PYG{n}{partner}\PYG{p}{:} \PYG{n+nb}{bool}\PYG{p}{(}\PYG{n}{partner}\PYG{o}{.}\PYG{n}{user\PYGZus{}ids}\PYG{p}{)} \PYG{o+ow}{and}
            \PYG{n+nb}{any}\PYG{p}{(}\PYG{n}{user}\PYG{o}{.}\PYG{n}{has\PYGZus{}group}\PYG{p}{(}\PYG{l+s+s1}{\PYGZsq{}}\PYG{l+s+s1}{business.group\PYGZus{}trip\PYGZus{}manager}\PYG{l+s+s1}{\PYGZsq{}}\PYG{p}{)}
            \PYG{k}{for} \PYG{n}{user} \PYG{o+ow}{in} \PYG{n}{partner}\PYG{o}{.}\PYG{n}{user\PYGZus{}ids}\PYG{p}{)}\PYG{p}{,}
            \PYG{p}{\PYGZob{}}
                \PYG{l+s+s1}{\PYGZsq{}}\PYG{l+s+s1}{actions}\PYG{l+s+s1}{\PYGZsq{}}\PYG{p}{:} \PYG{n}{trip\PYGZus{}actions}\PYG{p}{,}
            \PYG{p}{\PYGZcb{}}\PYG{p}{)}

        \PYG{k}{return} \PYG{p}{[}\PYG{n}{new\PYGZus{}group}\PYG{p}{]} \PYG{o}{+} \PYG{n}{groups}
\end{sphinxVerbatim}

Note that that I could have defined my evaluation function outside of this
method and define a global function to do it instead of a lambda, but for
the sake of being more brief and less verbose in these documentation files
that can sometimes be boring, I choose the former instead of the latter.
\end{sphinxadmonition}


\paragraph{Overriding defaults}
\label{\detokenize{reference/mixins:overriding-defaults}}
There are several ways you can customize the behaviour of \sphinxcode{\sphinxupquote{mail.thread}} models,
including (but not limited to):
\begin{description}
\item[{\sphinxcode{\sphinxupquote{\_mail\_post\_access}} - {\hyperref[\detokenize{reference/orm:odoo.models.Model}]{\sphinxcrossref{\sphinxcode{\sphinxupquote{Model}}}}}  attribute}] \leavevmode
the required access rights to be able to post a message on the model; by
default a \sphinxcode{\sphinxupquote{write}} access is needed, can be set to \sphinxcode{\sphinxupquote{read}} as well

\item[{Context keys:}] \leavevmode
These context keys can be used to somewhat control \sphinxcode{\sphinxupquote{mail.thread}} features
like auto-subscription or field tracking during calls to \sphinxcode{\sphinxupquote{create()}} or
\sphinxcode{\sphinxupquote{write()}} (or any other method where it may be useful).
\begin{itemize}
\item {} 
\sphinxcode{\sphinxupquote{mail\_create\_nosubscribe}}: at create or message\_post, do not subscribe
the current user to the record thread

\item {} 
\sphinxcode{\sphinxupquote{mail\_create\_nolog}}: at create, do not log the automatic ‘\textless{}Document\textgreater{}
created’ message

\item {} 
\sphinxcode{\sphinxupquote{mail\_notrack}}: at create and write, do not perform the value tracking
creating messages

\item {} 
\sphinxcode{\sphinxupquote{tracking\_disable}}: at create and write, perform no MailThread features
(auto subscription, tracking, post, …)

\item {} 
\sphinxcode{\sphinxupquote{mail\_auto\_delete}}: auto delete mail notifications; True by default

\item {} 
\sphinxcode{\sphinxupquote{mail\_notify\_force\_send}}: if less than 50 email notifications to send,
send them directly instead of using the queue; True by default

\item {} 
\sphinxcode{\sphinxupquote{mail\_notify\_user\_signature}}: add the current user signature in
email notifications; True by default

\end{itemize}

\end{description}


\subsubsection{Mail alias}
\label{\detokenize{reference/mixins:mail-alias}}\label{\detokenize{reference/mixins:reference-mixins-mail-alias}}
Aliases are configurable email addresses that are linked to a specific record
(which usually inherits the \sphinxcode{\sphinxupquote{mail.alias.mixin}} model) that will create new records when
contacted via e-mail. They are an easy way to make your system accessible from
the outside, allowing users or customers to quickly create records in your
database without needing to connect to Odoo directly.


\paragraph{Aliases vs. Incoming Mail Gateway}
\label{\detokenize{reference/mixins:aliases-vs-incoming-mail-gateway}}
Some people use the Incoming Mail Gateway for this same purpose. You still need
a correctly configured mail gateway to use aliases, however a single
catchall domain will be sufficient since all routing will be done inside Odoo.
Aliases have several advantages over Mail Gateways:
\begin{itemize}
\item {} \begin{description}
\item[{Easier to configure}] \leavevmode\begin{itemize}
\item {} 
A single incoming gateway can be used by many aliases; this avoids having
to configure multiple emails on your domain name (all configuration is done
inside Odoo)

\item {} 
No need for System access rights to configure aliases

\end{itemize}

\end{description}

\item {} \begin{description}
\item[{More coherent}] \leavevmode\begin{itemize}
\item {} 
Configurable on the related record, not in a Settings submenu

\end{itemize}

\end{description}

\item {} \begin{description}
\item[{Easier to override server-side}] \leavevmode\begin{itemize}
\item {} 
Mixin model is built to be extended from the start, allowing you to
extract useful data from incoming e-mails more easily than with a mail
gateway.

\end{itemize}

\end{description}

\end{itemize}


\paragraph{Alias support integration}
\label{\detokenize{reference/mixins:alias-support-integration}}
Aliases are usually configured on a parent model which will then create specific
record when contacted by e-mail. For example, Project have aliases to create tasks
or issues, Sales Channel have aliases to generate Leads.

\begin{sphinxadmonition}{note}{Note:}
The model that will be created by the alias \sphinxstylestrong{must} inherit the
\sphinxcode{\sphinxupquote{mail\_thread}} model.
\end{sphinxadmonition}

Alias support is added by inheriting \sphinxcode{\sphinxupquote{mail.alias.mixin}}; this mixin will
generate a new \sphinxcode{\sphinxupquote{mail.alias}} record for each record of the parent class that
gets created (for example, every \sphinxcode{\sphinxupquote{project.project}} record having its \sphinxcode{\sphinxupquote{mail.alias}}
record initialized on creation).

\begin{sphinxadmonition}{note}{Note:}
Aliases can also be created manually and supported by a simple
{\hyperref[\detokenize{reference/orm:odoo.fields.Many2one}]{\sphinxcrossref{\sphinxcode{\sphinxupquote{Many2one}}}}} field. This guide assumes you wish a
more complete integration with automatic creation of the alias, record-specific
default values, etc.
\end{sphinxadmonition}

Unlike \sphinxcode{\sphinxupquote{mail.thread}} inheritance, the \sphinxcode{\sphinxupquote{mail.alias.mixin}} \sphinxstylestrong{requires} some
specific overrides to work correctly. These overrides will specify the values
of the created alias, like the kind of record it must create and possibly
some default values these records may have depending on the parent object:
\index{get\_alias\_model\_name()}

\begin{fulllineitems}
\phantomsection\label{\detokenize{reference/mixins:get_alias_model_name}}\pysiglinewithargsret{\sphinxbfcode{\sphinxupquote{get\_alias\_model\_name}}}{\emph{vals}}{}
Return the model name for the alias. Incoming emails that are not
replies to existing records will cause the creation of a new record
of this alias model. The value may depend on \sphinxcode{\sphinxupquote{vals}}, the dict of
values passed to \sphinxcode{\sphinxupquote{create}} when a record of this model is created.
\begin{quote}\begin{description}
\item[{Parameters}] \leavevmode
\sphinxstyleliteralstrong{\sphinxupquote{dict}} (\sphinxstyleliteralemphasis{\sphinxupquote{vals}}) \textendash{} values of the newly created record that will holding
the alias

\item[{Returns}] \leavevmode
model name

\item[{Return type}] \leavevmode
\sphinxhref{https://docs.python.org/3/library/stdtypes.html\#str}{str}

\end{description}\end{quote}

\end{fulllineitems}

\index{get\_alias\_values()}

\begin{fulllineitems}
\phantomsection\label{\detokenize{reference/mixins:get_alias_values}}\pysiglinewithargsret{\sphinxbfcode{\sphinxupquote{get\_alias\_values}}}{}{}
Return values to create an alias, or to write on the alias after its
creation. While not completely mandatory, it is usually required to make
sure that newly created records will be linked to the alias’ parent (i.e.
tasks getting created in the right project) by setting a dictionary of
default values in the alias’ \sphinxcode{\sphinxupquote{alias\_defaults}} field.
\begin{quote}\begin{description}
\item[{Returns}] \leavevmode
dictionnary of values that will be written to the new alias

\item[{Return type}] \leavevmode
\sphinxhref{https://docs.python.org/3/library/stdtypes.html\#dict}{dict}

\end{description}\end{quote}

\end{fulllineitems}


The \sphinxcode{\sphinxupquote{get\_alias\_values()}} override is particularly interesting as it allows you
to modify the behaviour of your aliases easily. Among the fields that can be set
on the alias, the following are of particular interest:
\begin{description}
\item[{\sphinxcode{\sphinxupquote{alias\_name}} - {\hyperref[\detokenize{reference/orm:odoo.fields.Char}]{\sphinxcrossref{\sphinxcode{\sphinxupquote{Char}}}}}}] \leavevmode
name of the email alias, e.g. ‘jobs’ if you want to catch emails for
\textless{}\sphinxhref{mailto:jobs@example.odoo.com}{jobs@example.odoo.com}\textgreater{}

\item[{\sphinxcode{\sphinxupquote{alias\_user\_id}} - {\hyperref[\detokenize{reference/orm:odoo.fields.Many2one}]{\sphinxcrossref{\sphinxcode{\sphinxupquote{Many2one}}}}} (\sphinxcode{\sphinxupquote{res.users}})}] \leavevmode
owner of records created upon receiving emails on this alias;
if this field is not set the system will attempt to find the right owner
based on the sender (From) address, or will use the Administrator account
if no system user is found for that address

\item[{\sphinxcode{\sphinxupquote{alias\_defaults}} - {\hyperref[\detokenize{reference/orm:odoo.fields.Text}]{\sphinxcrossref{\sphinxcode{\sphinxupquote{Text}}}}}}] \leavevmode
Python dictionary that will be evaluated to provide
default values when creating new records for this alias

\item[{\sphinxcode{\sphinxupquote{alias\_force\_thread\_id}} - {\hyperref[\detokenize{reference/orm:odoo.fields.Integer}]{\sphinxcrossref{\sphinxcode{\sphinxupquote{Integer}}}}}}] \leavevmode
optional ID of a thread (record) to which all incoming messages will be
attached, even if they did not reply to it; if set, this will disable the
creation of new records completely

\item[{\sphinxcode{\sphinxupquote{alias\_contact}} - {\hyperref[\detokenize{reference/orm:odoo.fields.Selection}]{\sphinxcrossref{\sphinxcode{\sphinxupquote{Selection}}}}}}] \leavevmode
Policy to post a message on the document using the mailgateway
\begin{itemize}
\item {} 
\sphinxstyleemphasis{everyone}: everyone can post

\item {} 
\sphinxstyleemphasis{partners}: only authenticated partners

\item {} 
\sphinxstyleemphasis{followers}: only followers of the related document or members of following channels

\end{itemize}

\end{description}

Note that aliases make use of {\hyperref[\detokenize{reference/orm:reference-orm-inheritance}]{\sphinxcrossref{\DUrole{std,std-ref}{delegation inheritance}}}},
which means that while the alias is stored in another table, you have
access to all these fields directly from your parent object. This allows
you to make your alias easily configurable from the record’s form view.

\begin{sphinxadmonition}{note}{Example}

Let’s add aliases on our business trip class to create expenses on the fly via
e-mail.

\fvset{hllines={, ,}}%
\begin{sphinxVerbatim}[commandchars=\\\{\}]
\PYG{k}{class} \PYG{n+nc}{BusinessTrip}\PYG{p}{(}\PYG{n}{models}\PYG{o}{.}\PYG{n}{Model}\PYG{p}{)}\PYG{p}{:}
    \PYG{n}{\PYGZus{}name} \PYG{o}{=} \PYG{l+s+s1}{\PYGZsq{}}\PYG{l+s+s1}{business.trip}\PYG{l+s+s1}{\PYGZsq{}}
    \PYG{n}{\PYGZus{}inherit} \PYG{o}{=} \PYG{p}{[}\PYG{l+s+s1}{\PYGZsq{}}\PYG{l+s+s1}{mail.thread}\PYG{l+s+s1}{\PYGZsq{}}\PYG{p}{,} \PYG{l+s+s1}{\PYGZsq{}}\PYG{l+s+s1}{mail.alias.mixin}\PYG{l+s+s1}{\PYGZsq{}}\PYG{p}{]}
    \PYG{n}{\PYGZus{}description} \PYG{o}{=} \PYG{l+s+s1}{\PYGZsq{}}\PYG{l+s+s1}{Business Trip}\PYG{l+s+s1}{\PYGZsq{}}

    \PYG{n}{name} \PYG{o}{=} \PYG{n}{fields}\PYG{o}{.}\PYG{n}{Char}\PYG{p}{(}\PYG{n}{track\PYGZus{}visibility}\PYG{o}{=}\PYG{l+s+s1}{\PYGZsq{}}\PYG{l+s+s1}{onchange}\PYG{l+s+s1}{\PYGZsq{}}\PYG{p}{)}
    \PYG{n}{partner\PYGZus{}id} \PYG{o}{=} \PYG{n}{fields}\PYG{o}{.}\PYG{n}{Many2one}\PYG{p}{(}\PYG{l+s+s1}{\PYGZsq{}}\PYG{l+s+s1}{res.partner}\PYG{l+s+s1}{\PYGZsq{}}\PYG{p}{,} \PYG{l+s+s1}{\PYGZsq{}}\PYG{l+s+s1}{Responsible}\PYG{l+s+s1}{\PYGZsq{}}\PYG{p}{,}
                                 \PYG{n}{track\PYGZus{}visibility}\PYG{o}{=}\PYG{l+s+s1}{\PYGZsq{}}\PYG{l+s+s1}{onchange}\PYG{l+s+s1}{\PYGZsq{}}\PYG{p}{)}
    \PYG{n}{guest\PYGZus{}ids} \PYG{o}{=} \PYG{n}{fields}\PYG{o}{.}\PYG{n}{Many2many}\PYG{p}{(}\PYG{l+s+s1}{\PYGZsq{}}\PYG{l+s+s1}{res.partner}\PYG{l+s+s1}{\PYGZsq{}}\PYG{p}{,} \PYG{l+s+s1}{\PYGZsq{}}\PYG{l+s+s1}{Participants}\PYG{l+s+s1}{\PYGZsq{}}\PYG{p}{)}
    \PYG{n}{state} \PYG{o}{=} \PYG{n}{fields}\PYG{o}{.}\PYG{n}{Selection}\PYG{p}{(}\PYG{p}{[}\PYG{p}{(}\PYG{l+s+s1}{\PYGZsq{}}\PYG{l+s+s1}{draft}\PYG{l+s+s1}{\PYGZsq{}}\PYG{p}{,} \PYG{l+s+s1}{\PYGZsq{}}\PYG{l+s+s1}{New}\PYG{l+s+s1}{\PYGZsq{}}\PYG{p}{)}\PYG{p}{,} \PYG{p}{(}\PYG{l+s+s1}{\PYGZsq{}}\PYG{l+s+s1}{confirmed}\PYG{l+s+s1}{\PYGZsq{}}\PYG{p}{,} \PYG{l+s+s1}{\PYGZsq{}}\PYG{l+s+s1}{Confirmed}\PYG{l+s+s1}{\PYGZsq{}}\PYG{p}{)}\PYG{p}{]}\PYG{p}{,}
                             \PYG{n}{track\PYGZus{}visibility}\PYG{o}{=}\PYG{l+s+s1}{\PYGZsq{}}\PYG{l+s+s1}{onchange}\PYG{l+s+s1}{\PYGZsq{}}\PYG{p}{)}
    \PYG{n}{expense\PYGZus{}ids} \PYG{o}{=} \PYG{n}{fields}\PYG{o}{.}\PYG{n}{One2many}\PYG{p}{(}\PYG{l+s+s1}{\PYGZsq{}}\PYG{l+s+s1}{business.expense}\PYG{l+s+s1}{\PYGZsq{}}\PYG{p}{,} \PYG{l+s+s1}{\PYGZsq{}}\PYG{l+s+s1}{trip\PYGZus{}id}\PYG{l+s+s1}{\PYGZsq{}}\PYG{p}{,} \PYG{l+s+s1}{\PYGZsq{}}\PYG{l+s+s1}{Expenses}\PYG{l+s+s1}{\PYGZsq{}}\PYG{p}{)}
    \PYG{n}{alias\PYGZus{}id} \PYG{o}{=} \PYG{n}{fields}\PYG{o}{.}\PYG{n}{Many2one}\PYG{p}{(}\PYG{l+s+s1}{\PYGZsq{}}\PYG{l+s+s1}{mail.alias}\PYG{l+s+s1}{\PYGZsq{}}\PYG{p}{,} \PYG{n}{string}\PYG{o}{=}\PYG{l+s+s1}{\PYGZsq{}}\PYG{l+s+s1}{Alias}\PYG{l+s+s1}{\PYGZsq{}}\PYG{p}{,} \PYG{n}{ondelete}\PYG{o}{=}\PYG{l+s+s2}{\PYGZdq{}}\PYG{l+s+s2}{restrict}\PYG{l+s+s2}{\PYGZdq{}}\PYG{p}{,}
                               \PYG{n}{required}\PYG{o}{=}\PYG{n+nb+bp}{True}\PYG{p}{)}

    \PYG{k}{def} \PYG{n+nf}{get\PYGZus{}alias\PYGZus{}model\PYGZus{}name}\PYG{p}{(}\PYG{n+nb+bp}{self}\PYG{p}{,} \PYG{n}{vals}\PYG{p}{)}\PYG{p}{:}
    \PYG{l+s+sd}{\PYGZdq{}\PYGZdq{}\PYGZdq{} Specify the model that will get created when the alias receives a message \PYGZdq{}\PYGZdq{}\PYGZdq{}}
        \PYG{k}{return} \PYG{l+s+s1}{\PYGZsq{}}\PYG{l+s+s1}{business.expense}\PYG{l+s+s1}{\PYGZsq{}}

    \PYG{k}{def} \PYG{n+nf}{get\PYGZus{}alias\PYGZus{}values}\PYG{p}{(}\PYG{n+nb+bp}{self}\PYG{p}{)}\PYG{p}{:}
    \PYG{l+s+sd}{\PYGZdq{}\PYGZdq{}\PYGZdq{} Specify some default values that will be set in the alias at its creation \PYGZdq{}\PYGZdq{}\PYGZdq{}}
        \PYG{n}{values} \PYG{o}{=} \PYG{n+nb}{super}\PYG{p}{(}\PYG{n}{BusinessTrip}\PYG{p}{,} \PYG{n+nb+bp}{self}\PYG{p}{)}\PYG{o}{.}\PYG{n}{get\PYGZus{}alias\PYGZus{}values}\PYG{p}{(}\PYG{p}{)}
        \PYG{c+c1}{\PYGZsh{} alias\PYGZus{}defaults holds a dictionnary that will be written}
        \PYG{c+c1}{\PYGZsh{} to all records created by this alias}
        \PYG{c+c1}{\PYGZsh{}}
        \PYG{c+c1}{\PYGZsh{} in this case, we want all expense records sent to a trip alias}
        \PYG{c+c1}{\PYGZsh{} to be linked to the corresponding business trip}
        \PYG{n}{values}\PYG{p}{[}\PYG{l+s+s1}{\PYGZsq{}}\PYG{l+s+s1}{alias\PYGZus{}defaults}\PYG{l+s+s1}{\PYGZsq{}}\PYG{p}{]} \PYG{o}{=} \PYG{p}{\PYGZob{}}\PYG{l+s+s1}{\PYGZsq{}}\PYG{l+s+s1}{trip\PYGZus{}id}\PYG{l+s+s1}{\PYGZsq{}}\PYG{p}{:} \PYG{n+nb+bp}{self}\PYG{o}{.}\PYG{n}{id}\PYG{p}{\PYGZcb{}}
        \PYG{c+c1}{\PYGZsh{} we only want followers of the trip to be able to post expenses}
        \PYG{c+c1}{\PYGZsh{} by default}
        \PYG{n}{values}\PYG{p}{[}\PYG{l+s+s1}{\PYGZsq{}}\PYG{l+s+s1}{alias\PYGZus{}contact}\PYG{l+s+s1}{\PYGZsq{}}\PYG{p}{]} \PYG{o}{=} \PYG{l+s+s1}{\PYGZsq{}}\PYG{l+s+s1}{followers}\PYG{l+s+s1}{\PYGZsq{}}
        \PYG{k}{return} \PYG{n}{values}

\PYG{k}{class} \PYG{n+nc}{BusinessExpense}\PYG{p}{(}\PYG{n}{models}\PYG{o}{.}\PYG{n}{Model}\PYG{p}{)}\PYG{p}{:}
    \PYG{n}{\PYGZus{}name} \PYG{o}{=} \PYG{l+s+s1}{\PYGZsq{}}\PYG{l+s+s1}{business.expense}\PYG{l+s+s1}{\PYGZsq{}}
    \PYG{n}{\PYGZus{}inherit} \PYG{o}{=} \PYG{p}{[}\PYG{l+s+s1}{\PYGZsq{}}\PYG{l+s+s1}{mail.thread}\PYG{l+s+s1}{\PYGZsq{}}\PYG{p}{]}
    \PYG{n}{\PYGZus{}description} \PYG{o}{=} \PYG{l+s+s1}{\PYGZsq{}}\PYG{l+s+s1}{Business Expense}\PYG{l+s+s1}{\PYGZsq{}}

    \PYG{n}{name} \PYG{o}{=} \PYG{n}{fields}\PYG{o}{.}\PYG{n}{Char}\PYG{p}{(}\PYG{p}{)}
    \PYG{n}{amount} \PYG{o}{=} \PYG{n}{fields}\PYG{o}{.}\PYG{n}{Float}\PYG{p}{(}\PYG{l+s+s1}{\PYGZsq{}}\PYG{l+s+s1}{Amount}\PYG{l+s+s1}{\PYGZsq{}}\PYG{p}{)}
    \PYG{n}{trip\PYGZus{}id} \PYG{o}{=} \PYG{n}{fields}\PYG{o}{.}\PYG{n}{Many2one}\PYG{p}{(}\PYG{l+s+s1}{\PYGZsq{}}\PYG{l+s+s1}{business.trip}\PYG{l+s+s1}{\PYGZsq{}}\PYG{p}{,} \PYG{l+s+s1}{\PYGZsq{}}\PYG{l+s+s1}{Business Trip}\PYG{l+s+s1}{\PYGZsq{}}\PYG{p}{)}
    \PYG{n}{partner\PYGZus{}id} \PYG{o}{=} \PYG{n}{fields}\PYG{o}{.}\PYG{n}{Many2one}\PYG{p}{(}\PYG{l+s+s1}{\PYGZsq{}}\PYG{l+s+s1}{res.partner}\PYG{l+s+s1}{\PYGZsq{}}\PYG{p}{,} \PYG{l+s+s1}{\PYGZsq{}}\PYG{l+s+s1}{Created by}\PYG{l+s+s1}{\PYGZsq{}}\PYG{p}{)}
\end{sphinxVerbatim}

We would like our alias to be easily configurable from the form view of our
business trips, so let’s add the following to our form view:

\fvset{hllines={, ,}}%
\begin{sphinxVerbatim}[commandchars=\\\{\}]
\PYG{n+nt}{\PYGZlt{}page} \PYG{n+na}{string=}\PYG{l+s}{\PYGZdq{}Emails\PYGZdq{}}\PYG{n+nt}{\PYGZgt{}}
    \PYG{n+nt}{\PYGZlt{}group} \PYG{n+na}{name=}\PYG{l+s}{\PYGZdq{}group\PYGZus{}alias\PYGZdq{}}\PYG{n+nt}{\PYGZgt{}}
        \PYG{n+nt}{\PYGZlt{}label} \PYG{n+na}{for=}\PYG{l+s}{\PYGZdq{}alias\PYGZus{}name\PYGZdq{}} \PYG{n+na}{string=}\PYG{l+s}{\PYGZdq{}Email Alias\PYGZdq{}}\PYG{n+nt}{/\PYGZgt{}}
        \PYG{n+nt}{\PYGZlt{}div} \PYG{n+na}{name=}\PYG{l+s}{\PYGZdq{}alias\PYGZus{}def\PYGZdq{}}\PYG{n+nt}{\PYGZgt{}}
            \PYG{c}{\PYGZlt{}!\PYGZhy{}\PYGZhy{}}\PYG{c}{ display a link while in view mode and a configurable field}
\PYG{c}{            while in edit mode }\PYG{c}{\PYGZhy{}\PYGZhy{}\PYGZgt{}}
            \PYG{n+nt}{\PYGZlt{}field} \PYG{n+na}{name=}\PYG{l+s}{\PYGZdq{}alias\PYGZus{}id\PYGZdq{}} \PYG{n+na}{class=}\PYG{l+s}{\PYGZdq{}oe\PYGZus{}read\PYGZus{}only oe\PYGZus{}inline\PYGZdq{}}
                    \PYG{n+na}{string=}\PYG{l+s}{\PYGZdq{}Email Alias\PYGZdq{}} \PYG{n+na}{required=}\PYG{l+s}{\PYGZdq{}0\PYGZdq{}}\PYG{n+nt}{/\PYGZgt{}}
            \PYG{n+nt}{\PYGZlt{}div} \PYG{n+na}{class=}\PYG{l+s}{\PYGZdq{}oe\PYGZus{}edit\PYGZus{}only oe\PYGZus{}inline\PYGZdq{}} \PYG{n+na}{name=}\PYG{l+s}{\PYGZdq{}edit\PYGZus{}alias\PYGZdq{}}
                 \PYG{n+na}{style=}\PYG{l+s}{\PYGZdq{}display: inline;\PYGZdq{}} \PYG{n+nt}{\PYGZgt{}}
                \PYG{n+nt}{\PYGZlt{}field} \PYG{n+na}{name=}\PYG{l+s}{\PYGZdq{}alias\PYGZus{}name\PYGZdq{}} \PYG{n+na}{class=}\PYG{l+s}{\PYGZdq{}oe\PYGZus{}inline\PYGZdq{}}\PYG{n+nt}{/\PYGZgt{}}
                @
                \PYG{n+nt}{\PYGZlt{}field} \PYG{n+na}{name=}\PYG{l+s}{\PYGZdq{}alias\PYGZus{}domain\PYGZdq{}} \PYG{n+na}{class=}\PYG{l+s}{\PYGZdq{}oe\PYGZus{}inline\PYGZdq{}} \PYG{n+na}{readonly=}\PYG{l+s}{\PYGZdq{}1\PYGZdq{}}\PYG{n+nt}{/\PYGZgt{}}
            \PYG{n+nt}{\PYGZlt{}/div\PYGZgt{}}
        \PYG{n+nt}{\PYGZlt{}/div\PYGZgt{}}
        \PYG{n+nt}{\PYGZlt{}field} \PYG{n+na}{name=}\PYG{l+s}{\PYGZdq{}alias\PYGZus{}contact\PYGZdq{}} \PYG{n+na}{class=}\PYG{l+s}{\PYGZdq{}oe\PYGZus{}inline\PYGZdq{}}
                \PYG{n+na}{string=}\PYG{l+s}{\PYGZdq{}Accept Emails From\PYGZdq{}}\PYG{n+nt}{/\PYGZgt{}}
    \PYG{n+nt}{\PYGZlt{}/group\PYGZgt{}}
\PYG{n+nt}{\PYGZlt{}/page\PYGZgt{}}
\end{sphinxVerbatim}

Now we can change the alias address directly from the form view and change
who can send e-mails to the alias.

We can then override \sphinxcode{\sphinxupquote{message\_new()}} on our expense model to fetch the values
from our email when the expense will be created:

\fvset{hllines={, ,}}%
\begin{sphinxVerbatim}[commandchars=\\\{\}]
\PYG{k}{class} \PYG{n+nc}{BusinessExpense}\PYG{p}{(}\PYG{n}{models}\PYG{o}{.}\PYG{n}{Model}\PYG{p}{)}\PYG{p}{:}
    \PYG{c+c1}{\PYGZsh{} Previous code goes here}
    \PYG{c+c1}{\PYGZsh{} ...}

    \PYG{k}{def} \PYG{n+nf}{message\PYGZus{}new}\PYG{p}{(}\PYG{n+nb+bp}{self}\PYG{p}{,} \PYG{n}{msg}\PYG{p}{,} \PYG{n}{custom\PYGZus{}values}\PYG{o}{=}\PYG{n+nb+bp}{None}\PYG{p}{)}\PYG{p}{:}
        \PYG{l+s+sd}{\PYGZdq{}\PYGZdq{}\PYGZdq{} Override to set values according to the email.}

\PYG{l+s+sd}{        In this simple example, we simply use the email title as the name}
\PYG{l+s+sd}{        of the expense, try to find a partner with this email address and}
\PYG{l+s+sd}{        do a regex match to find the amount of the expense.\PYGZdq{}\PYGZdq{}\PYGZdq{}}
        \PYG{n}{name} \PYG{o}{=} \PYG{n}{msg\PYGZus{}dict}\PYG{o}{.}\PYG{n}{get}\PYG{p}{(}\PYG{l+s+s1}{\PYGZsq{}}\PYG{l+s+s1}{subject}\PYG{l+s+s1}{\PYGZsq{}}\PYG{p}{,} \PYG{l+s+s1}{\PYGZsq{}}\PYG{l+s+s1}{New Expense}\PYG{l+s+s1}{\PYGZsq{}}\PYG{p}{)}
        \PYG{c+c1}{\PYGZsh{} Match the last occurence of a float in the string}
        \PYG{c+c1}{\PYGZsh{} Example: \PYGZsq{}50.3 bar 34.5\PYGZsq{} becomes \PYGZsq{}34.5\PYGZsq{}. This is potentially the price}
        \PYG{c+c1}{\PYGZsh{} to encode on the expense. If not, take 1.0 instead}
        \PYG{n}{amount\PYGZus{}pattern} \PYG{o}{=} \PYG{l+s+s1}{\PYGZsq{}}\PYG{l+s+s1}{(}\PYG{l+s+s1}{\PYGZbs{}}\PYG{l+s+s1}{d+(}\PYG{l+s+s1}{\PYGZbs{}}\PYG{l+s+s1}{.}\PYG{l+s+s1}{\PYGZbs{}}\PYG{l+s+s1}{d*)?\textbar{}}\PYG{l+s+s1}{\PYGZbs{}}\PYG{l+s+s1}{.}\PYG{l+s+s1}{\PYGZbs{}}\PYG{l+s+s1}{d+)}\PYG{l+s+s1}{\PYGZsq{}}
        \PYG{n}{expense\PYGZus{}price} \PYG{o}{=} \PYG{n}{re}\PYG{o}{.}\PYG{n}{findall}\PYG{p}{(}\PYG{n}{amount\PYGZus{}pattern}\PYG{p}{,} \PYG{n}{name}\PYG{p}{)}
        \PYG{n}{price} \PYG{o}{=} \PYG{n}{expense\PYGZus{}price} \PYG{o+ow}{and} \PYG{n+nb}{float}\PYG{p}{(}\PYG{n}{expense\PYGZus{}price}\PYG{p}{[}\PYG{o}{\PYGZhy{}}\PYG{l+m+mi}{1}\PYG{p}{]}\PYG{p}{[}\PYG{l+m+mi}{0}\PYG{p}{]}\PYG{p}{)} \PYG{o+ow}{or} \PYG{l+m+mf}{1.0}
        \PYG{c+c1}{\PYGZsh{} find the partner by looking for it\PYGZsq{}s email}
        \PYG{n}{partner} \PYG{o}{=} \PYG{n+nb+bp}{self}\PYG{o}{.}\PYG{n}{env}\PYG{p}{[}\PYG{l+s+s1}{\PYGZsq{}}\PYG{l+s+s1}{res.partner}\PYG{l+s+s1}{\PYGZsq{}}\PYG{p}{]}\PYG{o}{.}\PYG{n}{search}\PYG{p}{(}\PYG{p}{[}\PYG{p}{(}\PYG{l+s+s1}{\PYGZsq{}}\PYG{l+s+s1}{email}\PYG{l+s+s1}{\PYGZsq{}}\PYG{p}{,} \PYG{l+s+s1}{\PYGZsq{}}\PYG{l+s+s1}{ilike}\PYG{l+s+s1}{\PYGZsq{}}\PYG{p}{,} \PYG{n}{email\PYGZus{}address}\PYG{p}{)}\PYG{p}{]}\PYG{p}{,}
                                                 \PYG{n}{limit}\PYG{o}{=}\PYG{l+m+mi}{1}\PYG{p}{)}
        \PYG{n}{defaults} \PYG{o}{=} \PYG{p}{\PYGZob{}}
            \PYG{l+s+s1}{\PYGZsq{}}\PYG{l+s+s1}{name}\PYG{l+s+s1}{\PYGZsq{}}\PYG{p}{:} \PYG{n}{name}\PYG{p}{,}
            \PYG{l+s+s1}{\PYGZsq{}}\PYG{l+s+s1}{amount}\PYG{l+s+s1}{\PYGZsq{}}\PYG{p}{:} \PYG{n}{price}\PYG{p}{,}
            \PYG{l+s+s1}{\PYGZsq{}}\PYG{l+s+s1}{partner\PYGZus{}id}\PYG{l+s+s1}{\PYGZsq{}}\PYG{p}{:} \PYG{n}{partner}\PYG{o}{.}\PYG{n}{id}
        \PYG{p}{\PYGZcb{}}
        \PYG{n}{defaults}\PYG{o}{.}\PYG{n}{update}\PYG{p}{(}\PYG{n}{custom\PYGZus{}values} \PYG{o+ow}{or} \PYG{p}{\PYGZob{}}\PYG{p}{\PYGZcb{}}\PYG{p}{)}
        \PYG{n}{res} \PYG{o}{=} \PYG{n+nb}{super}\PYG{p}{(}\PYG{n}{BusinessExpense}\PYG{p}{,} \PYG{n+nb+bp}{self}\PYG{p}{)}\PYG{o}{.}\PYG{n}{message\PYGZus{}new}\PYG{p}{(}\PYG{n}{msg}\PYG{p}{,} \PYG{n}{custom\PYGZus{}values}\PYG{o}{=}\PYG{n}{defaults}\PYG{p}{)}
        \PYG{k}{return} \PYG{n}{res}
\end{sphinxVerbatim}
\end{sphinxadmonition}


\subsubsection{Activities tracking}
\label{\detokenize{reference/mixins:reference-mixins-mail-activities}}\label{\detokenize{reference/mixins:activities-tracking}}
Activities are actions users have to take on a document like making a phone call
or organizing a meeting. Activities come with the mail module as they are
integrated in the Chatter but are \sphinxstyleemphasis{not bundled with mail.thread}. Activities
are records of the \sphinxcode{\sphinxupquote{mail.activity}} class, which have a type (\sphinxcode{\sphinxupquote{mail.activity.type}}),
name, description, scheduled time (among others). Pending activities are visible
above the message history in the chatter widget.

You can integrate activities using the \sphinxcode{\sphinxupquote{mail.activity.mixin}} class on your object
and the specific widgets to display them (via the field \sphinxcode{\sphinxupquote{activity\_ids}}) in the form
view and kanban view of your records (\sphinxcode{\sphinxupquote{mail\_activity}} and \sphinxcode{\sphinxupquote{kanban\_activity}}
widgets, respectively).

\begin{sphinxadmonition}{note}{Example}

Organizing a business trip is a tedious process and tracking needed activities
like ordering plane tickets or a cab for the airport could be useful. To do so,
we will add the activities mixin on our model and display the next planned activities
in the message history of our trip.

\fvset{hllines={, ,}}%
\begin{sphinxVerbatim}[commandchars=\\\{\}]
\PYG{k}{class} \PYG{n+nc}{BusinessTrip}\PYG{p}{(}\PYG{n}{models}\PYG{o}{.}\PYG{n}{Model}\PYG{p}{)}\PYG{p}{:}
    \PYG{n}{\PYGZus{}name} \PYG{o}{=} \PYG{l+s+s1}{\PYGZsq{}}\PYG{l+s+s1}{business.trip}\PYG{l+s+s1}{\PYGZsq{}}
    \PYG{n}{\PYGZus{}inherit} \PYG{o}{=} \PYG{p}{[}\PYG{l+s+s1}{\PYGZsq{}}\PYG{l+s+s1}{mail.thread}\PYG{l+s+s1}{\PYGZsq{}}\PYG{p}{,} \PYG{l+s+s1}{\PYGZsq{}}\PYG{l+s+s1}{mail.activity.mixin}\PYG{l+s+s1}{\PYGZsq{}}\PYG{p}{]}
    \PYG{n}{\PYGZus{}description} \PYG{o}{=} \PYG{l+s+s1}{\PYGZsq{}}\PYG{l+s+s1}{Business Trip}\PYG{l+s+s1}{\PYGZsq{}}

    \PYG{n}{name} \PYG{o}{=} \PYG{n}{fields}\PYG{o}{.}\PYG{n}{Char}\PYG{p}{(}\PYG{p}{)}
    \PYG{c+c1}{\PYGZsh{} [...]}
\end{sphinxVerbatim}

We modify the form view of our trips to display their next activites:

\fvset{hllines={, ,}}%
\begin{sphinxVerbatim}[commandchars=\\\{\}]
\PYG{n+nt}{\PYGZlt{}record} \PYG{n+na}{id=}\PYG{l+s}{\PYGZdq{}businness\PYGZus{}trip\PYGZus{}form\PYGZdq{}} \PYG{n+na}{model=}\PYG{l+s}{\PYGZdq{}ir.ui.view\PYGZdq{}}\PYG{n+nt}{\PYGZgt{}}
    \PYG{n+nt}{\PYGZlt{}field} \PYG{n+na}{name=}\PYG{l+s}{\PYGZdq{}name\PYGZdq{}}\PYG{n+nt}{\PYGZgt{}}business.trip.form\PYG{n+nt}{\PYGZlt{}/field\PYGZgt{}}
    \PYG{n+nt}{\PYGZlt{}field} \PYG{n+na}{name=}\PYG{l+s}{\PYGZdq{}model\PYGZdq{}}\PYG{n+nt}{\PYGZgt{}}business.trip\PYG{n+nt}{\PYGZlt{}/field\PYGZgt{}}
    \PYG{n+nt}{\PYGZlt{}field} \PYG{n+na}{name=}\PYG{l+s}{\PYGZdq{}arch\PYGZdq{}} \PYG{n+na}{type=}\PYG{l+s}{\PYGZdq{}xml\PYGZdq{}}\PYG{n+nt}{\PYGZgt{}}
        \PYG{n+nt}{\PYGZlt{}form} \PYG{n+na}{string=}\PYG{l+s}{\PYGZdq{}Business Trip\PYGZdq{}}\PYG{n+nt}{\PYGZgt{}}
            \PYG{c}{\PYGZlt{}!\PYGZhy{}\PYGZhy{}}\PYG{c}{ Your usual form view goes here }\PYG{c}{\PYGZhy{}\PYGZhy{}\PYGZgt{}}
            \PYG{n+nt}{\PYGZlt{}div} \PYG{n+na}{class=}\PYG{l+s}{\PYGZdq{}oe\PYGZus{}chatter\PYGZdq{}}\PYG{n+nt}{\PYGZgt{}}
                \PYG{n+nt}{\PYGZlt{}field} \PYG{n+na}{name=}\PYG{l+s}{\PYGZdq{}message\PYGZus{}follower\PYGZus{}ids\PYGZdq{}} \PYG{n+na}{widget=}\PYG{l+s}{\PYGZdq{}mail\PYGZus{}followers\PYGZdq{}}\PYG{n+nt}{/\PYGZgt{}}
                \PYG{n+nt}{\PYGZlt{}field} \PYG{n+na}{name=}\PYG{l+s}{\PYGZdq{}activity\PYGZus{}ids\PYGZdq{}} \PYG{n+na}{widget=}\PYG{l+s}{\PYGZdq{}mail\PYGZus{}activity\PYGZdq{}}\PYG{n+nt}{/\PYGZgt{}}
                \PYG{n+nt}{\PYGZlt{}field} \PYG{n+na}{name=}\PYG{l+s}{\PYGZdq{}message\PYGZus{}ids\PYGZdq{}} \PYG{n+na}{widget=}\PYG{l+s}{\PYGZdq{}mail\PYGZus{}thread\PYGZdq{}}\PYG{n+nt}{/\PYGZgt{}}
            \PYG{n+nt}{\PYGZlt{}/div\PYGZgt{}}
        \PYG{n+nt}{\PYGZlt{}/form\PYGZgt{}}
    \PYG{n+nt}{\PYGZlt{}/field\PYGZgt{}}
\PYG{n+nt}{\PYGZlt{}/record\PYGZgt{}}
\end{sphinxVerbatim}
\end{sphinxadmonition}

You can find concrete examples of integration in the following models:
\begin{itemize}
\item {} 
\sphinxcode{\sphinxupquote{crm.lead}} in the CRM (\sphinxstyleemphasis{crm}) Application

\item {} 
\sphinxcode{\sphinxupquote{sale.order}} in the Sales (\sphinxstyleemphasis{sale}) Application

\item {} 
\sphinxcode{\sphinxupquote{project.task}} in the Project (\sphinxstyleemphasis{poject}) Application

\end{itemize}


\subsection{Website features}
\label{\detokenize{reference/mixins:reference-mixins-website}}\label{\detokenize{reference/mixins:website-features}}

\subsubsection{Visitor tracking}
\label{\detokenize{reference/mixins:reference-mixins-website-utm}}\label{\detokenize{reference/mixins:visitor-tracking}}
The \sphinxcode{\sphinxupquote{utm.mixin}} class can be used to track online marketing/communication
campaigns through arguments in links to specified resources. The mixin adds
3 fields to your model:
\begin{itemize}
\item {} 
\sphinxcode{\sphinxupquote{campaign\_id}}: {\hyperref[\detokenize{reference/orm:odoo.fields.Many2one}]{\sphinxcrossref{\sphinxcode{\sphinxupquote{Many2one}}}}} field to a \sphinxcode{\sphinxupquote{utm.campaign}}
object (i.e. Christmas\_Special, Fall\_Collection, etc.)

\item {} 
\sphinxcode{\sphinxupquote{source\_id}}: {\hyperref[\detokenize{reference/orm:odoo.fields.Many2one}]{\sphinxcrossref{\sphinxcode{\sphinxupquote{Many2one}}}}} field to a \sphinxcode{\sphinxupquote{utm.source}}
object (i.e. Search Engine, mailing list, etc.)

\item {} 
\sphinxcode{\sphinxupquote{medium\_id}}: {\hyperref[\detokenize{reference/orm:odoo.fields.Many2one}]{\sphinxcrossref{\sphinxcode{\sphinxupquote{Many2one}}}}} field to a \sphinxcode{\sphinxupquote{utm.medium}}
object (i.e. Snail Mail, e-Mail, social network update, etc.)

\end{itemize}

These models have a single field \sphinxcode{\sphinxupquote{name}} (i.e. they are simply there to
distinguish campaigns but don’t have any specific behaviour).

Once a customer visits your website with these parameters set in the url
(i.e. \sphinxurl{http://www.odoo.com/?campaign\_id=mixin\_talk\&source\_id=www.odoo.com\&medium\_id=website}),
three cookies are set in the visitor’s website for these parameters.
Once a object that inherits the utm.mixin is created from the website (i.e. lead
form, job application, etc.), the utm.mixin code kicks in and fetches the values
from the cookies to set them in the new record. Once this is done, you can then
use the campaign/source/medium fields as any other field when defining reports
and views (group by, etc.).

To extend this behaviour, simply add a relational field to a simple model (the
model should support the \sphinxstyleemphasis{quick create} (i.e. call to \sphinxcode{\sphinxupquote{create()}} with a single
\sphinxcode{\sphinxupquote{name}} value) and extend the function \sphinxcode{\sphinxupquote{tracking\_fields()}}:

\fvset{hllines={, ,}}%
\begin{sphinxVerbatim}[commandchars=\\\{\}]
\PYG{k}{class} \PYG{n+nc}{UtmMyTrack}\PYG{p}{(}\PYG{n}{models}\PYG{o}{.}\PYG{n}{Model}\PYG{p}{)}\PYG{p}{:}
    \PYG{n}{\PYGZus{}name} \PYG{o}{=} \PYG{l+s+s1}{\PYGZsq{}}\PYG{l+s+s1}{my\PYGZus{}module.my\PYGZus{}track}\PYG{l+s+s1}{\PYGZsq{}}
    \PYG{n}{\PYGZus{}description} \PYG{o}{=} \PYG{l+s+s1}{\PYGZsq{}}\PYG{l+s+s1}{My Tracking Object}\PYG{l+s+s1}{\PYGZsq{}}

    \PYG{n}{name} \PYG{o}{=} \PYG{n}{fields}\PYG{o}{.}\PYG{n}{Char}\PYG{p}{(}\PYG{n}{string}\PYG{o}{=}\PYG{l+s+s1}{\PYGZsq{}}\PYG{l+s+s1}{Name}\PYG{l+s+s1}{\PYGZsq{}}\PYG{p}{,} \PYG{n}{required}\PYG{o}{=}\PYG{n+nb+bp}{True}\PYG{p}{)}


\PYG{k}{class} \PYG{n+nc}{MyModel}\PYG{p}{(}\PYG{n}{models}\PYG{o}{.}\PYG{n}{Models}\PYG{p}{)}\PYG{p}{:}
    \PYG{n}{\PYGZus{}name} \PYG{o}{=} \PYG{l+s+s1}{\PYGZsq{}}\PYG{l+s+s1}{my\PYGZus{}module.my\PYGZus{}model}\PYG{l+s+s1}{\PYGZsq{}}
    \PYG{n}{\PYGZus{}inherit} \PYG{o}{=} \PYG{p}{[}\PYG{l+s+s1}{\PYGZsq{}}\PYG{l+s+s1}{utm.mixin}\PYG{l+s+s1}{\PYGZsq{}}\PYG{p}{]}
    \PYG{n}{\PYGZus{}description} \PYG{o}{=} \PYG{l+s+s1}{\PYGZsq{}}\PYG{l+s+s1}{My Tracked Object}\PYG{l+s+s1}{\PYGZsq{}}

    \PYG{n}{my\PYGZus{}field} \PYG{o}{=} \PYG{n}{fields}\PYG{o}{.}\PYG{n}{Many2one}\PYG{p}{(}\PYG{l+s+s1}{\PYGZsq{}}\PYG{l+s+s1}{my\PYGZus{}module.my\PYGZus{}track}\PYG{l+s+s1}{\PYGZsq{}}\PYG{p}{,} \PYG{l+s+s1}{\PYGZsq{}}\PYG{l+s+s1}{My Field}\PYG{l+s+s1}{\PYGZsq{}}\PYG{p}{)}

    \PYG{n+nd}{@api.model}
    \PYG{k}{def} \PYG{n+nf}{tracking\PYGZus{}fields}\PYG{p}{(}\PYG{n+nb+bp}{self}\PYG{p}{)}\PYG{p}{:}
        \PYG{n}{result} \PYG{o}{=} \PYG{n+nb}{super}\PYG{p}{(}\PYG{n}{MyModel}\PYG{p}{,} \PYG{n+nb+bp}{self}\PYG{p}{)}\PYG{o}{.}\PYG{n}{tracking\PYGZus{}fields}\PYG{p}{(}\PYG{p}{)}
        \PYG{n}{result}\PYG{o}{.}\PYG{n}{append}\PYG{p}{(}\PYG{p}{[}
        \PYG{c+c1}{\PYGZsh{} (\PYGZdq{}URL\PYGZus{}PARAMETER\PYGZdq{}, \PYGZdq{}FIELD\PYGZus{}NAME\PYGZus{}MIXIN\PYGZdq{}, \PYGZdq{}NAME\PYGZus{}IN\PYGZus{}COOKIES\PYGZdq{})}
            \PYG{p}{(}\PYG{l+s+s1}{\PYGZsq{}}\PYG{l+s+s1}{my\PYGZus{}field}\PYG{l+s+s1}{\PYGZsq{}}\PYG{p}{,} \PYG{l+s+s1}{\PYGZsq{}}\PYG{l+s+s1}{my\PYGZus{}field}\PYG{l+s+s1}{\PYGZsq{}}\PYG{p}{,} \PYG{l+s+s1}{\PYGZsq{}}\PYG{l+s+s1}{odoo\PYGZus{}utm\PYGZus{}my\PYGZus{}field}\PYG{l+s+s1}{\PYGZsq{}}\PYG{p}{)}
        \PYG{p}{]}\PYG{p}{)}
        \PYG{k}{return} \PYG{n}{result}
\end{sphinxVerbatim}

This will tell the system to create a cookie named \sphinxstyleemphasis{odoo\_utm\_my\_field} with the
value found in the url parameter \sphinxcode{\sphinxupquote{my\_field}}; once a new record of this model is
created by a call from a website form, the generic override of the \sphinxcode{\sphinxupquote{create()}}
method of \sphinxcode{\sphinxupquote{utm.mixin}} will fetch the default values for this field from the
cookie (and the \sphinxcode{\sphinxupquote{my\_module.my\_track}} record will be creatwed on the fly if it
does not exist yet).

You can find concrete examples of integration in the following models:
\begin{itemize}
\item {} 
\sphinxcode{\sphinxupquote{crm.lead}} in the CRM (\sphinxstyleemphasis{crm}) Application

\item {} 
\sphinxcode{\sphinxupquote{hr.applicant}} in the Recruitment Process (\sphinxstyleemphasis{hr\_recruitment}) Application

\item {} 
\sphinxcode{\sphinxupquote{helpdesk.ticket}} in the Helpdesk (\sphinxstyleemphasis{helpdesk} - Odoo Enterprise only) Application

\end{itemize}


\subsubsection{Website visibility}
\label{\detokenize{reference/mixins:reference-mixins-website-published}}\label{\detokenize{reference/mixins:website-visibility}}
You can quite easily add a website visibility toggle on any of your record. While
this mixin is quite easy to implement manually, it is the most often-used after
the \sphinxcode{\sphinxupquote{mail.thread}} inheritance; a testament to its usefulness. The typical use
case for this mixin is any object that has a frontend-page; being able to control
the visibility of the page allows you to take your time while editing the page
and only publish it when you’re satisfied.

To include the functionnality, you only need to inherit \sphinxcode{\sphinxupquote{website.published.mixin}}:

\fvset{hllines={, ,}}%
\begin{sphinxVerbatim}[commandchars=\\\{\}]
\PYG{k}{class} \PYG{n+nc}{BlogPost}\PYG{p}{(}\PYG{n}{models}\PYG{o}{.}\PYG{n}{Model}\PYG{p}{)}\PYG{p}{:}
    \PYG{n}{\PYGZus{}name} \PYG{o}{=} \PYG{l+s+s2}{\PYGZdq{}}\PYG{l+s+s2}{blog.post}\PYG{l+s+s2}{\PYGZdq{}}
    \PYG{n}{\PYGZus{}description} \PYG{o}{=} \PYG{l+s+s2}{\PYGZdq{}}\PYG{l+s+s2}{Blog Post}\PYG{l+s+s2}{\PYGZdq{}}
    \PYG{n}{\PYGZus{}inherit} \PYG{o}{=} \PYG{p}{[}\PYG{l+s+s1}{\PYGZsq{}}\PYG{l+s+s1}{website.published.mixin}\PYG{l+s+s1}{\PYGZsq{}}\PYG{p}{]}
\end{sphinxVerbatim}

This mixin adds 2 fields on your model:
\begin{itemize}
\item {} 
\sphinxcode{\sphinxupquote{website\_published}}: {\hyperref[\detokenize{reference/orm:odoo.fields.Boolean}]{\sphinxcrossref{\sphinxcode{\sphinxupquote{Boolean}}}}} field which represents
the status of the publication

\item {} 
\sphinxcode{\sphinxupquote{website\_url}}: {\hyperref[\detokenize{reference/orm:odoo.fields.Char}]{\sphinxcrossref{\sphinxcode{\sphinxupquote{Char}}}}} field which represents
the URL through which the object is accessed

\end{itemize}

Note that this last field is a computed field and must be implemented for your class:

\fvset{hllines={, ,}}%
\begin{sphinxVerbatim}[commandchars=\\\{\}]
\PYG{k}{def} \PYG{n+nf}{\PYGZus{}compute\PYGZus{}website\PYGZus{}url}\PYG{p}{(}\PYG{n+nb+bp}{self}\PYG{p}{)}\PYG{p}{:}
    \PYG{k}{for} \PYG{n}{blog\PYGZus{}post} \PYG{o+ow}{in} \PYG{n+nb+bp}{self}\PYG{p}{:}
        \PYG{n}{blog\PYGZus{}post}\PYG{o}{.}\PYG{n}{website\PYGZus{}url} \PYG{o}{=} \PYG{l+s+s2}{\PYGZdq{}}\PYG{l+s+s2}{/blog/}\PYG{l+s+si}{\PYGZpc{}s}\PYG{l+s+s2}{\PYGZdq{}} \PYG{o}{\PYGZpc{}} \PYG{p}{(}\PYG{n}{log\PYGZus{}post}\PYG{o}{.}\PYG{n}{blog\PYGZus{}id}\PYG{p}{)}
\end{sphinxVerbatim}

Once the mechanism is in place, you just have to adapt your frontend and backend
views to make it accessible. In the backend, adding a button in the button box is
usually the way to go:

\fvset{hllines={, ,}}%
\begin{sphinxVerbatim}[commandchars=\\\{\}]
\PYG{n+nt}{\PYGZlt{}button} \PYG{n+na}{class=}\PYG{l+s}{\PYGZdq{}oe\PYGZus{}stat\PYGZus{}button\PYGZdq{}} \PYG{n+na}{name=}\PYG{l+s}{\PYGZdq{}website\PYGZus{}publish\PYGZus{}button\PYGZdq{}}
    \PYG{n+na}{type=}\PYG{l+s}{\PYGZdq{}object\PYGZdq{}} \PYG{n+na}{icon=}\PYG{l+s}{\PYGZdq{}fa\PYGZhy{}globe\PYGZdq{}}\PYG{n+nt}{\PYGZgt{}}
    \PYG{n+nt}{\PYGZlt{}field} \PYG{n+na}{name=}\PYG{l+s}{\PYGZdq{}website\PYGZus{}published\PYGZdq{}} \PYG{n+na}{widget=}\PYG{l+s}{\PYGZdq{}website\PYGZus{}button\PYGZdq{}}\PYG{n+nt}{/\PYGZgt{}}
\PYG{n+nt}{\PYGZlt{}/button\PYGZgt{}}
\end{sphinxVerbatim}

In the frontend, some security checks are needed to avoid showing ‘Edition’
buttons to website visitors:

\fvset{hllines={, ,}}%
\begin{sphinxVerbatim}[commandchars=\\\{\}]
\PYG{n+nt}{\PYGZlt{}div} \PYG{n+na}{id=}\PYG{l+s}{\PYGZdq{}website\PYGZus{}published\PYGZus{}button\PYGZdq{}} \PYG{n+na}{class=}\PYG{l+s}{\PYGZdq{}pull\PYGZhy{}right\PYGZdq{}}
     \PYG{n+na}{groups=}\PYG{l+s}{\PYGZdq{}base.group\PYGZus{}website\PYGZus{}publisher\PYGZdq{}}\PYG{n+nt}{\PYGZgt{}} \PYG{c}{\PYGZlt{}!\PYGZhy{}\PYGZhy{}}\PYG{c}{ or any other meaningful group }\PYG{c}{\PYGZhy{}\PYGZhy{}\PYGZgt{}}
    \PYG{n+nt}{\PYGZlt{}t} \PYG{n+na}{t\PYGZhy{}call=}\PYG{l+s}{\PYGZdq{}website.publish\PYGZus{}management\PYGZdq{}}\PYG{n+nt}{\PYGZgt{}}
      \PYG{n+nt}{\PYGZlt{}t} \PYG{n+na}{t\PYGZhy{}set=}\PYG{l+s}{\PYGZdq{}object\PYGZdq{}} \PYG{n+na}{t\PYGZhy{}value=}\PYG{l+s}{\PYGZdq{}blog\PYGZus{}post\PYGZdq{}}\PYG{n+nt}{/\PYGZgt{}}
      \PYG{n+nt}{\PYGZlt{}t} \PYG{n+na}{t\PYGZhy{}set=}\PYG{l+s}{\PYGZdq{}publish\PYGZus{}edit\PYGZdq{}} \PYG{n+na}{t\PYGZhy{}value=}\PYG{l+s}{\PYGZdq{}True\PYGZdq{}}\PYG{n+nt}{/\PYGZgt{}}
      \PYG{n+nt}{\PYGZlt{}t} \PYG{n+na}{t\PYGZhy{}set=}\PYG{l+s}{\PYGZdq{}action\PYGZdq{}} \PYG{n+na}{t\PYGZhy{}value=}\PYG{l+s}{\PYGZdq{}\PYGZsq{}blog.blog\PYGZus{}post\PYGZus{}action\PYGZsq{}\PYGZdq{}}\PYG{n+nt}{/\PYGZgt{}}
    \PYG{n+nt}{\PYGZlt{}/t\PYGZgt{}}
\PYG{n+nt}{\PYGZlt{}/div\PYGZgt{}}
\end{sphinxVerbatim}

Note that you must pass your object as the variable \sphinxcode{\sphinxupquote{object}} to the template;
in this example, the \sphinxcode{\sphinxupquote{blog.post}} record was passed as the \sphinxcode{\sphinxupquote{blog\_post}} variable
to the \sphinxcode{\sphinxupquote{qweb}} rendering engine, it is necessary to specify this to the publish
management template. The \sphinxcode{\sphinxupquote{publish\_edit}} variable allow the frontend
button to link to the backend (allowing you to switch from frontend to backend
and vice-versa easily); if set, you must specify the full external id of the action
you want to call in the backend in the \sphinxcode{\sphinxupquote{action}} variable (note that a Form View
must exist for the model).

The action \sphinxcode{\sphinxupquote{website\_publish\_button}} is defined in the mixin and adapts its
behaviour to your object: if the class has a valid \sphinxcode{\sphinxupquote{website\_url}} compute function,
the user is redirected to the frontend when he clicks on the button; the user
can then publish the page directly from the frontend. This ensures
that no online publication can happen by accident. If there is not compute function,
the boolean \sphinxcode{\sphinxupquote{website\_published}} is simply triggered.


\subsubsection{Website metadata}
\label{\detokenize{reference/mixins:reference-mixins-website-seo}}\label{\detokenize{reference/mixins:website-metadata}}
This simple mixin simply allows you to easily inject metadata in your frontend
pages.

\fvset{hllines={, ,}}%
\begin{sphinxVerbatim}[commandchars=\\\{\}]
\PYG{k}{class} \PYG{n+nc}{BlogPost}\PYG{p}{(}\PYG{n}{models}\PYG{o}{.}\PYG{n}{Model}\PYG{p}{)}\PYG{p}{:}
    \PYG{n}{\PYGZus{}name} \PYG{o}{=} \PYG{l+s+s2}{\PYGZdq{}}\PYG{l+s+s2}{blog.post}\PYG{l+s+s2}{\PYGZdq{}}
    \PYG{n}{\PYGZus{}description} \PYG{o}{=} \PYG{l+s+s2}{\PYGZdq{}}\PYG{l+s+s2}{Blog Post}\PYG{l+s+s2}{\PYGZdq{}}
    \PYG{n}{\PYGZus{}inherit} \PYG{o}{=} \PYG{p}{[}\PYG{l+s+s1}{\PYGZsq{}}\PYG{l+s+s1}{website.seo.metadata}\PYG{l+s+s1}{\PYGZsq{}}\PYG{p}{,} \PYG{l+s+s1}{\PYGZsq{}}\PYG{l+s+s1}{website.published.mixin}\PYG{l+s+s1}{\PYGZsq{}}\PYG{p}{]}
\end{sphinxVerbatim}

This mixin adds 3 fields on your model:
\begin{itemize}
\item {} 
\sphinxcode{\sphinxupquote{website\_meta\_title}}: {\hyperref[\detokenize{reference/orm:odoo.fields.Char}]{\sphinxcrossref{\sphinxcode{\sphinxupquote{Char}}}}} field that allow you to set
an additional title to your page

\item {} 
\sphinxcode{\sphinxupquote{website\_meta\_description}}: {\hyperref[\detokenize{reference/orm:odoo.fields.Char}]{\sphinxcrossref{\sphinxcode{\sphinxupquote{Char}}}}} field that contains a
short description of the page (sometimes used in search engines results)

\item {} 
\sphinxcode{\sphinxupquote{website\_meta\_keywords}}: {\hyperref[\detokenize{reference/orm:odoo.fields.Char}]{\sphinxcrossref{\sphinxcode{\sphinxupquote{Char}}}}} field that contains some
keywords to help your page to be classified more precisely by search engines; the
“Promote” tool will help you select lexically-related keywords easily

\end{itemize}

These fields are editable in the frontend using the “Promote” tool from the Editor
toolbar. Setting these fields can help search engines to better index your pages.
Note that search engines do not base their results only on these metadata; the
best SEO practice should still be to get referenced by reliable sources.


\subsection{Others}
\label{\detokenize{reference/mixins:reference-mixins-misc}}\label{\detokenize{reference/mixins:others}}

\subsubsection{Customer Rating}
\label{\detokenize{reference/mixins:reference-mixins-misc-rating}}\label{\detokenize{reference/mixins:customer-rating}}
The rating mixin allows sending email to ask for customer rating, automatic
transitioning in a kanban processes and aggregating statistics on your ratings.


\paragraph{Adding rating on your model}
\label{\detokenize{reference/mixins:adding-rating-on-your-model}}
To add rating support, simply inherit the \sphinxcode{\sphinxupquote{rating.mixin}} model:

\fvset{hllines={, ,}}%
\begin{sphinxVerbatim}[commandchars=\\\{\}]
\PYG{k}{class} \PYG{n+nc}{MyModel}\PYG{p}{(}\PYG{n}{models}\PYG{o}{.}\PYG{n}{Models}\PYG{p}{)}\PYG{p}{:}
    \PYG{n}{\PYGZus{}name} \PYG{o}{=} \PYG{l+s+s1}{\PYGZsq{}}\PYG{l+s+s1}{my\PYGZus{}module.my\PYGZus{}model}\PYG{l+s+s1}{\PYGZsq{}}
    \PYG{n}{\PYGZus{}inherit} \PYG{o}{=} \PYG{p}{[}\PYG{l+s+s1}{\PYGZsq{}}\PYG{l+s+s1}{rating.mixin}\PYG{l+s+s1}{\PYGZsq{}}\PYG{p}{,} \PYG{l+s+s1}{\PYGZsq{}}\PYG{l+s+s1}{mail.thread}\PYG{l+s+s1}{\PYGZsq{}}\PYG{p}{]}

    \PYG{n}{user\PYGZus{}id} \PYG{o}{=} \PYG{n}{fields}\PYG{o}{.}\PYG{n}{Many2one}\PYG{p}{(}\PYG{l+s+s1}{\PYGZsq{}}\PYG{l+s+s1}{res.users}\PYG{l+s+s1}{\PYGZsq{}}\PYG{p}{,} \PYG{l+s+s1}{\PYGZsq{}}\PYG{l+s+s1}{Responsible}\PYG{l+s+s1}{\PYGZsq{}}\PYG{p}{)}
    \PYG{n}{partner\PYGZus{}id} \PYG{o}{=} \PYG{n}{fields}\PYG{o}{.}\PYG{n}{Many2one}\PYG{p}{(}\PYG{l+s+s1}{\PYGZsq{}}\PYG{l+s+s1}{res.partner}\PYG{l+s+s1}{\PYGZsq{}}\PYG{p}{,} \PYG{l+s+s1}{\PYGZsq{}}\PYG{l+s+s1}{Customer}\PYG{l+s+s1}{\PYGZsq{}}\PYG{p}{)}
\end{sphinxVerbatim}

The behaviour of the mixin adapts to your model:
\begin{itemize}
\item {} 
The \sphinxcode{\sphinxupquote{rating.rating}} record will be linked to the \sphinxcode{\sphinxupquote{partner\_id}} field of your
model (if the field is present).
\begin{itemize}
\item {} 
this behaviour can be overriden with the function \sphinxcode{\sphinxupquote{rating\_get\_partner\_id()}}
if you use another field than \sphinxcode{\sphinxupquote{partner\_id}}

\end{itemize}

\item {} 
The \sphinxcode{\sphinxupquote{rating.rating}} record will be linked to the partner of the \sphinxcode{\sphinxupquote{user\_id}}
field of your model (if the field is present) (i.e. the partner who is rated)
\begin{itemize}
\item {} 
this behaviour can be overriden with the function \sphinxcode{\sphinxupquote{rating\_get\_rated\_partner\_id()}}
if you use another field than \sphinxcode{\sphinxupquote{user\_id}} (note that the function must return a
\sphinxcode{\sphinxupquote{res.partner}}, for \sphinxcode{\sphinxupquote{user\_id}} the system automatically fetches the partner
of the user)

\end{itemize}

\item {} 
The chatter history will display the rating event (if your model inherits from
\sphinxcode{\sphinxupquote{mail.thread}})

\end{itemize}


\paragraph{Send rating requests by e-mail}
\label{\detokenize{reference/mixins:send-rating-requests-by-e-mail}}
If you wish to send emails to request a rating, simply generate an e-mail with
links to the rating object. A very basic email template could look like this:

\fvset{hllines={, ,}}%
\begin{sphinxVerbatim}[commandchars=\\\{\}]
\PYG{n+nt}{\PYGZlt{}record} \PYG{n+na}{id=}\PYG{l+s}{\PYGZdq{}rating\PYGZus{}my\PYGZus{}model\PYGZus{}email\PYGZus{}template\PYGZdq{}} \PYG{n+na}{model=}\PYG{l+s}{\PYGZdq{}mail.template\PYGZdq{}}\PYG{n+nt}{\PYGZgt{}}
            \PYG{n+nt}{\PYGZlt{}field} \PYG{n+na}{name=}\PYG{l+s}{\PYGZdq{}name\PYGZdq{}}\PYG{n+nt}{\PYGZgt{}}My Model: Rating Request\PYG{n+nt}{\PYGZlt{}/field\PYGZgt{}}
            \PYG{n+nt}{\PYGZlt{}field} \PYG{n+na}{name=}\PYG{l+s}{\PYGZdq{}email\PYGZus{}from\PYGZdq{}}\PYG{n+nt}{\PYGZgt{}}\PYGZdl{}\PYGZob{}object.rating\PYGZus{}get\PYGZus{}rated\PYGZus{}partner\PYGZus{}id().email or \PYGZsq{}\PYGZsq{} \textbar{} safe\PYGZcb{}\PYG{n+nt}{\PYGZlt{}/field\PYGZgt{}}
            \PYG{n+nt}{\PYGZlt{}field} \PYG{n+na}{name=}\PYG{l+s}{\PYGZdq{}subject\PYGZdq{}}\PYG{n+nt}{\PYGZgt{}}Service Rating Request\PYG{n+nt}{\PYGZlt{}/field\PYGZgt{}}
            \PYG{n+nt}{\PYGZlt{}field} \PYG{n+na}{name=}\PYG{l+s}{\PYGZdq{}model\PYGZus{}id\PYGZdq{}} \PYG{n+na}{ref=}\PYG{l+s}{\PYGZdq{}my\PYGZus{}module.model\PYGZus{}my\PYGZus{}model\PYGZdq{}}\PYG{n+nt}{/\PYGZgt{}}
            \PYG{n+nt}{\PYGZlt{}field} \PYG{n+na}{name=}\PYG{l+s}{\PYGZdq{}partner\PYGZus{}to\PYGZdq{}} \PYG{n+nt}{\PYGZgt{}}\PYGZdl{}\PYGZob{}object.rating\PYGZus{}get\PYGZus{}partner\PYGZus{}id().id\PYGZcb{}\PYG{n+nt}{\PYGZlt{}/field\PYGZgt{}}
            \PYG{n+nt}{\PYGZlt{}field} \PYG{n+na}{name=}\PYG{l+s}{\PYGZdq{}auto\PYGZus{}delete\PYGZdq{}} \PYG{n+na}{eval=}\PYG{l+s}{\PYGZdq{}True\PYGZdq{}}\PYG{n+nt}{/\PYGZgt{}}
            \PYG{n+nt}{\PYGZlt{}field} \PYG{n+na}{name=}\PYG{l+s}{\PYGZdq{}body\PYGZus{}html\PYGZdq{}}\PYG{n+nt}{\PYGZgt{}}\PYG{c+cp}{\PYGZlt{}![CDATA[}
\PYG{c+cp}{\PYGZpc{} set access\PYGZus{}token = object.rating\PYGZus{}get\PYGZus{}access\PYGZus{}token()}
\PYG{c+cp}{\PYGZlt{}p\PYGZgt{}Hi,\PYGZlt{}/p\PYGZgt{}}
\PYG{c+cp}{\PYGZlt{}p\PYGZgt{}How satsified are you?\PYGZlt{}/p\PYGZgt{}}
\PYG{c+cp}{\PYGZlt{}ul\PYGZgt{}}
\PYG{c+cp}{    \PYGZlt{}li\PYGZgt{}\PYGZlt{}a href=\PYGZdq{}/rating/\PYGZdl{}\PYGZob{}access\PYGZus{}token\PYGZcb{}/10\PYGZdq{}\PYGZgt{}Satisfied\PYGZlt{}/a\PYGZgt{}\PYGZlt{}/li\PYGZgt{}}
\PYG{c+cp}{    \PYGZlt{}li\PYGZgt{}\PYGZlt{}a href=\PYGZdq{}/rating/\PYGZdl{}\PYGZob{}access\PYGZus{}token\PYGZcb{}/5\PYGZdq{}\PYGZgt{}Not satisfied\PYGZlt{}/a\PYGZgt{}\PYGZlt{}/li\PYGZgt{}}
\PYG{c+cp}{    \PYGZlt{}li\PYGZgt{}\PYGZlt{}a href=\PYGZdq{}/rating/\PYGZdl{}\PYGZob{}access\PYGZus{}token\PYGZcb{}/1\PYGZdq{}\PYGZgt{}Very unsatisfied\PYGZlt{}/a\PYGZgt{}\PYGZlt{}/li\PYGZgt{}}
\PYG{c+cp}{\PYGZlt{}/ul\PYGZgt{}}
\PYG{c+cp}{]]\PYGZgt{}}\PYG{n+nt}{\PYGZlt{}/field\PYGZgt{}}
\PYG{n+nt}{\PYGZlt{}/record\PYGZgt{}}
\end{sphinxVerbatim}

Your customer will then receive an e-mail with links to a simple webpage allowing
them to provide a feedback on their interaction with your users (including a free-text
feedback message).

You can then quite easily integrate your ratings with your form view by defining
an action for the ratings:

\fvset{hllines={, ,}}%
\begin{sphinxVerbatim}[commandchars=\\\{\}]
\PYG{n+nt}{\PYGZlt{}record} \PYG{n+na}{id=}\PYG{l+s}{\PYGZdq{}rating\PYGZus{}rating\PYGZus{}action\PYGZus{}my\PYGZus{}model\PYGZdq{}} \PYG{n+na}{model=}\PYG{l+s}{\PYGZdq{}ir.actions.act\PYGZus{}window\PYGZdq{}}\PYG{n+nt}{\PYGZgt{}}
    \PYG{n+nt}{\PYGZlt{}field} \PYG{n+na}{name=}\PYG{l+s}{\PYGZdq{}name\PYGZdq{}}\PYG{n+nt}{\PYGZgt{}}Customer Ratings\PYG{n+nt}{\PYGZlt{}/field\PYGZgt{}}
    \PYG{n+nt}{\PYGZlt{}field} \PYG{n+na}{name=}\PYG{l+s}{\PYGZdq{}res\PYGZus{}model\PYGZdq{}}\PYG{n+nt}{\PYGZgt{}}rating.rating\PYG{n+nt}{\PYGZlt{}/field\PYGZgt{}}
    \PYG{n+nt}{\PYGZlt{}field} \PYG{n+na}{name=}\PYG{l+s}{\PYGZdq{}view\PYGZus{}mode\PYGZdq{}}\PYG{n+nt}{\PYGZgt{}}kanban,pivot,graph\PYG{n+nt}{\PYGZlt{}/field\PYGZgt{}}
    \PYG{n+nt}{\PYGZlt{}field} \PYG{n+na}{name=}\PYG{l+s}{\PYGZdq{}domain\PYGZdq{}}\PYG{n+nt}{\PYGZgt{}}[(\PYGZsq{}res\PYGZus{}model\PYGZsq{}, \PYGZsq{}=\PYGZsq{}, \PYGZsq{}my\PYGZus{}module.my\PYGZus{}model\PYGZsq{}), (\PYGZsq{}res\PYGZus{}id\PYGZsq{}, \PYGZsq{}=\PYGZsq{}, active\PYGZus{}id), (\PYGZsq{}consumed\PYGZsq{}, \PYGZsq{}=\PYGZsq{}, True)]\PYG{n+nt}{\PYGZlt{}/field\PYGZgt{}}
\PYG{n+nt}{\PYGZlt{}/record\PYGZgt{}}

\PYG{n+nt}{\PYGZlt{}record} \PYG{n+na}{id=}\PYG{l+s}{\PYGZdq{}my\PYGZus{}module\PYGZus{}my\PYGZus{}model\PYGZus{}view\PYGZus{}form\PYGZus{}inherit\PYGZus{}rating\PYGZdq{}} \PYG{n+na}{model=}\PYG{l+s}{\PYGZdq{}ir.ui.view\PYGZdq{}}\PYG{n+nt}{\PYGZgt{}}
    \PYG{n+nt}{\PYGZlt{}field} \PYG{n+na}{name=}\PYG{l+s}{\PYGZdq{}name\PYGZdq{}}\PYG{n+nt}{\PYGZgt{}}my\PYGZus{}module.my\PYGZus{}model.view.form.inherit.rating\PYG{n+nt}{\PYGZlt{}/field\PYGZgt{}}
    \PYG{n+nt}{\PYGZlt{}field} \PYG{n+na}{name=}\PYG{l+s}{\PYGZdq{}model\PYGZdq{}}\PYG{n+nt}{\PYGZgt{}}my\PYGZus{}module.my\PYGZus{}model\PYG{n+nt}{\PYGZlt{}/field\PYGZgt{}}
    \PYG{n+nt}{\PYGZlt{}field} \PYG{n+na}{name=}\PYG{l+s}{\PYGZdq{}inherit\PYGZus{}id\PYGZdq{}} \PYG{n+na}{ref=}\PYG{l+s}{\PYGZdq{}my\PYGZus{}module.my\PYGZus{}model\PYGZus{}view\PYGZus{}form\PYGZdq{}}\PYG{n+nt}{/\PYGZgt{}}
    \PYG{n+nt}{\PYGZlt{}field} \PYG{n+na}{name=}\PYG{l+s}{\PYGZdq{}arch\PYGZdq{}} \PYG{n+na}{type=}\PYG{l+s}{\PYGZdq{}xml\PYGZdq{}}\PYG{n+nt}{\PYGZgt{}}
        \PYG{n+nt}{\PYGZlt{}xpath} \PYG{n+na}{expr=}\PYG{l+s}{\PYGZdq{}//div[@name=\PYGZsq{}button\PYGZus{}box\PYGZsq{}]\PYGZdq{}} \PYG{n+na}{position=}\PYG{l+s}{\PYGZdq{}inside\PYGZdq{}}\PYG{n+nt}{\PYGZgt{}}
            \PYG{n+nt}{\PYGZlt{}button} \PYG{n+na}{name=}\PYG{l+s}{\PYGZdq{}\PYGZpc{}(rating\PYGZus{}rating\PYGZus{}action\PYGZus{}my\PYGZus{}model)d\PYGZdq{}} \PYG{n+na}{type=}\PYG{l+s}{\PYGZdq{}action\PYGZdq{}}
                    \PYG{n+na}{class=}\PYG{l+s}{\PYGZdq{}oe\PYGZus{}stat\PYGZus{}button\PYGZdq{}} \PYG{n+na}{icon=}\PYG{l+s}{\PYGZdq{}fa\PYGZhy{}smile\PYGZhy{}o\PYGZdq{}}\PYG{n+nt}{\PYGZgt{}}
                \PYG{n+nt}{\PYGZlt{}field} \PYG{n+na}{name=}\PYG{l+s}{\PYGZdq{}rating\PYGZus{}count\PYGZdq{}} \PYG{n+na}{string=}\PYG{l+s}{\PYGZdq{}Rating\PYGZdq{}} \PYG{n+na}{widget=}\PYG{l+s}{\PYGZdq{}statinfo\PYGZdq{}}\PYG{n+nt}{/\PYGZgt{}}
            \PYG{n+nt}{\PYGZlt{}/button\PYGZgt{}}
        \PYG{n+nt}{\PYGZlt{}/xpath\PYGZgt{}}
    \PYG{n+nt}{\PYGZlt{}/field\PYGZgt{}}
\PYG{n+nt}{\PYGZlt{}/record\PYGZgt{}}
\end{sphinxVerbatim}

Note that there are default views (kanban,pivot,graph) for ratings which allow
you a quick bird’s eye view of your customer ratings.

You can find concrete examples of integration in the following models:
\begin{itemize}
\item {} 
\sphinxcode{\sphinxupquote{project.task}} in the Project (\sphinxstyleemphasis{rating\_project}) Application

\item {} 
\sphinxcode{\sphinxupquote{helpdesk.ticket}} in the Helpdesk (\sphinxstyleemphasis{helpdesk} - Odoo Enterprise only) Application

\end{itemize}


\section{Odoo Guidelines}
\label{\detokenize{reference/guidelines:odoo-guidelines}}\label{\detokenize{reference/guidelines::doc}}
This page introduces the Odoo Coding Guidelines. Those aim to improve the
quality of Odoo Apps code. Indeed proper code improves readability, eases
maintenance, helps debugging, lowers complexity and promotes reliability.
These guidelines should be applied to every new module and to all new development.

\begin{sphinxadmonition}{warning}{Warning:}
When modifying existing files in \sphinxstylestrong{stable version} the original file style
strictly supersedes any other style guidelines. In other words please never
modify existing files in order to apply these guidelines. It avoids disrupting
the revision history of code lines. Diff should be kept minimal. For more
details, see our \sphinxhref{https://odoo.com/submit-pr}{pull request guide}.
\end{sphinxadmonition}

\begin{sphinxadmonition}{warning}{Warning:}
When modifying existing files in \sphinxstylestrong{master (development) version} apply those
guidelines to existing code only for modified code or if most of the file is
under revision. In other words modify existing files structure only if it is
going under major changes. In that case first do a \sphinxstylestrong{move} commit then apply
the changes related to the feature.
\end{sphinxadmonition}


\subsection{Module structure}
\label{\detokenize{reference/guidelines:module-structure}}

\subsubsection{Directories}
\label{\detokenize{reference/guidelines:directories}}
A module is organised in important directories. Those contain the business logic; having a look at them should make you understand the purpose of the module.
\begin{itemize}
\item {} 
\sphinxstyleemphasis{data/} : demo and data xml

\item {} 
\sphinxstyleemphasis{models/} : models definition

\item {} 
\sphinxstyleemphasis{controllers/} : contains controllers (HTTP routes).

\item {} 
\sphinxstyleemphasis{views/} : contains the views and templates

\item {} 
\sphinxstyleemphasis{static/} : contains the web assets, separated into \sphinxstyleemphasis{css/, js/, img/, lib/, …}

\end{itemize}

Other optional directories compose the module.
\begin{itemize}
\item {} 
\sphinxstyleemphasis{wizard/} : regroups the transient models (formerly \sphinxstyleemphasis{osv\_memory}) and their views.

\item {} 
\sphinxstyleemphasis{report/} : contains the reports (RML report \sphinxstylestrong{{[}deprecated{]}}, models based on SQL views (for reporting) and other complex reports). Python objects and XML views are included in this directory.

\item {} 
\sphinxstyleemphasis{tests/} : contains the Python/YML tests

\end{itemize}


\subsubsection{File naming}
\label{\detokenize{reference/guidelines:file-naming}}
For \sphinxstyleemphasis{views} declarations, split backend views from (frontend)
templates in 2 differents files.

For \sphinxstyleemphasis{models}, split the business logic by sets of models, in each set
select a main model, this model gives its name to the set. If there is
only one model, its name is the same as the module name. For
each set named \textless{}main\_model\textgreater{} the following files may be created:
\begin{itemize}
\item {} 
\sphinxcode{\sphinxupquote{models/\sphinxstyleemphasis{\textless{}main\_model\textgreater{}}.py}}

\item {} 
\sphinxcode{\sphinxupquote{models/\sphinxstyleemphasis{\textless{}inherited\_main\_model\textgreater{}}.py}}

\item {} 
\sphinxcode{\sphinxupquote{views/\sphinxstyleemphasis{\textless{}main\_model\textgreater{}}\_templates.xml}}

\item {} 
\sphinxcode{\sphinxupquote{views/\sphinxstyleemphasis{\textless{}main\_model\textgreater{}}\_views.xml}}

\end{itemize}

For instance, \sphinxstyleemphasis{sale} module introduces \sphinxcode{\sphinxupquote{sale\_order}} and
\sphinxcode{\sphinxupquote{sale\_order\_line}} where \sphinxcode{\sphinxupquote{sale\_order}} is dominant. So the
\sphinxcode{\sphinxupquote{\textless{}main\_model\textgreater{}}} files will be named \sphinxcode{\sphinxupquote{models/sale\_order.py}} and
\sphinxcode{\sphinxupquote{views/sale\_order\_views.py}}.

For \sphinxstyleemphasis{data}, split them by purpose : demo or data. The filename will be
the main\_model name, suffixed by \sphinxstyleemphasis{\_demo.xml} or \sphinxstyleemphasis{\_data.xml}.

For \sphinxstyleemphasis{controllers}, the only file should be named \sphinxstyleemphasis{main.py}. Otherwise, if you need to inherit an existing controller from another module, its name will be \sphinxstyleemphasis{\textless{}module\_name\textgreater{}.py}. Unlike \sphinxstyleemphasis{models}, each controller class should be contained in a separated file.

For \sphinxstyleemphasis{static files}, since the resources can be used in different contexts (frontend, backend, both), they will be included in only one bundle. So, CSS/Less, JavaScript and XML files should be suffixed with the name of the bundle type. i.e.: \sphinxstyleemphasis{im\_chat\_common.css}, \sphinxstyleemphasis{im\_chat\_common.js} for ‘assets\_common’ bundle, and \sphinxstyleemphasis{im\_chat\_backend.css}, \sphinxstyleemphasis{im\_chat\_backend.js} for ‘assets\_backend’ bundle.
If the module owns only one file, the convention will be \sphinxstyleemphasis{\textless{}module\_name\textgreater{}.ext} (i.e.: \sphinxstyleemphasis{project.js}).
Don’t link data (image, libraries) outside Odoo: do not use an
URL to an image but copy it in our codebase instead.

Regarding \sphinxstyleemphasis{data}, split them by purpose: data or demo. The filename will be
the \sphinxstyleemphasis{main\_model} name, suffixed by \sphinxstyleemphasis{\_data.xml} or \sphinxstyleemphasis{\_demo.xml}.

Regarding \sphinxstyleemphasis{wizards}, naming convention is :
\begin{itemize}
\item {} 
\sphinxcode{\sphinxupquote{\sphinxstyleemphasis{\textless{}main\_transient\textgreater{}}.py}}

\item {} 
\sphinxcode{\sphinxupquote{\sphinxstyleemphasis{\textless{}main\_transient\textgreater{}}\_views.xml}}

\end{itemize}

Where \sphinxstyleemphasis{\textless{}main\_transient\textgreater{}} is the name of the dominant transient model, just like for \sphinxstyleemphasis{models}. \textless{}main\_transient\textgreater{}.py can contains the models ‘model.action’ and ‘model.action.line’.

For \sphinxstyleemphasis{statistics reports}, their names should look like :
\begin{itemize}
\item {} 
\sphinxcode{\sphinxupquote{\sphinxstyleemphasis{\textless{}report\_name\_A\textgreater{}}\_report.py}}

\item {} 
\sphinxcode{\sphinxupquote{\sphinxstyleemphasis{\textless{}report\_name\_A\textgreater{}}\_report\_views.py}} (often pivot and graph views)

\end{itemize}

For \sphinxstyleemphasis{printable reports}, you should have :
\begin{itemize}
\item {} 
\sphinxcode{\sphinxupquote{\sphinxstyleemphasis{\textless{}print\_report\_name\textgreater{}}\_reports.py}} (report actions, paperformat definition, …)

\item {} 
\sphinxcode{\sphinxupquote{\sphinxstyleemphasis{\textless{}print\_report\_name\textgreater{}}\_templates.xml}} (xml report templates)

\end{itemize}

The complete tree should look like

\fvset{hllines={, ,}}%
\begin{sphinxVerbatim}[commandchars=\\\{\}]
addons/\PYGZlt{}my\PYGZus{}module\PYGZus{}name\PYGZgt{}/
\textbar{}\PYGZhy{}\PYGZhy{} \PYGZus{}\PYGZus{}init\PYGZus{}\PYGZus{}.py
\textbar{}\PYGZhy{}\PYGZhy{} \PYGZus{}\PYGZus{}manifest\PYGZus{}\PYGZus{}.py
\textbar{}\PYGZhy{}\PYGZhy{} controllers/
\textbar{}   \textbar{}\PYGZhy{}\PYGZhy{} \PYGZus{}\PYGZus{}init\PYGZus{}\PYGZus{}.py
\textbar{}   \textbar{}\PYGZhy{}\PYGZhy{} \PYGZlt{}inherited\PYGZus{}module\PYGZus{}name\PYGZgt{}.py
\textbar{}   {}`\PYGZhy{}\PYGZhy{} main.py
\textbar{}\PYGZhy{}\PYGZhy{} data/
\textbar{}   \textbar{}\PYGZhy{}\PYGZhy{} \PYGZlt{}main\PYGZus{}model\PYGZgt{}\PYGZus{}data.xml
\textbar{}   {}`\PYGZhy{}\PYGZhy{} \PYGZlt{}inherited\PYGZus{}main\PYGZus{}model\PYGZgt{}\PYGZus{}demo.xml
\textbar{}\PYGZhy{}\PYGZhy{} models/
\textbar{}   \textbar{}\PYGZhy{}\PYGZhy{} \PYGZus{}\PYGZus{}init\PYGZus{}\PYGZus{}.py
\textbar{}   \textbar{}\PYGZhy{}\PYGZhy{} \PYGZlt{}main\PYGZus{}model\PYGZgt{}.py
\textbar{}   {}`\PYGZhy{}\PYGZhy{} \PYGZlt{}inherited\PYGZus{}main\PYGZus{}model\PYGZgt{}.py
\textbar{}\PYGZhy{}\PYGZhy{} report/
\textbar{}   \textbar{}\PYGZhy{}\PYGZhy{} \PYGZus{}\PYGZus{}init\PYGZus{}\PYGZus{}.py
\textbar{}   \textbar{}\PYGZhy{}\PYGZhy{} \PYGZlt{}main\PYGZus{}stat\PYGZus{}report\PYGZus{}model\PYGZgt{}.py
\textbar{}   \textbar{}\PYGZhy{}\PYGZhy{} \PYGZlt{}main\PYGZus{}stat\PYGZus{}report\PYGZus{}model\PYGZgt{}\PYGZus{}views.xml
\textbar{}   \textbar{}\PYGZhy{}\PYGZhy{} \PYGZlt{}main\PYGZus{}print\PYGZus{}report\PYGZgt{}\PYGZus{}reports.xml
\textbar{}   {}`\PYGZhy{}\PYGZhy{} \PYGZlt{}main\PYGZus{}print\PYGZus{}report\PYGZgt{}\PYGZus{}templates.xml
\textbar{}\PYGZhy{}\PYGZhy{} security/
\textbar{}   \textbar{}\PYGZhy{}\PYGZhy{} ir.model.access.csv
\textbar{}   {}`\PYGZhy{}\PYGZhy{} \PYGZlt{}main\PYGZus{}model\PYGZgt{}\PYGZus{}security.xml
\textbar{}\PYGZhy{}\PYGZhy{} static/
\textbar{}   \textbar{}\PYGZhy{}\PYGZhy{} img/
\textbar{}   \textbar{}   \textbar{}\PYGZhy{}\PYGZhy{} my\PYGZus{}little\PYGZus{}kitten.png
\textbar{}   \textbar{}   {}`\PYGZhy{}\PYGZhy{} troll.jpg
\textbar{}   \textbar{}\PYGZhy{}\PYGZhy{} lib/
\textbar{}   \textbar{}   {}`\PYGZhy{}\PYGZhy{} external\PYGZus{}lib/
\textbar{}   {}`\PYGZhy{}\PYGZhy{} src/
\textbar{}       \textbar{}\PYGZhy{}\PYGZhy{} js/
\textbar{}       \textbar{}   {}`\PYGZhy{}\PYGZhy{} \PYGZlt{}my\PYGZus{}module\PYGZus{}name\PYGZgt{}.js
\textbar{}       \textbar{}\PYGZhy{}\PYGZhy{} css/
\textbar{}       \textbar{}   {}`\PYGZhy{}\PYGZhy{} \PYGZlt{}my\PYGZus{}module\PYGZus{}name\PYGZgt{}.css
\textbar{}       \textbar{}\PYGZhy{}\PYGZhy{} less/
\textbar{}       \textbar{}   {}`\PYGZhy{}\PYGZhy{} \PYGZlt{}my\PYGZus{}module\PYGZus{}name\PYGZgt{}.less
\textbar{}       {}`\PYGZhy{}\PYGZhy{} xml/
\textbar{}           {}`\PYGZhy{}\PYGZhy{} \PYGZlt{}my\PYGZus{}module\PYGZus{}name\PYGZgt{}.xml
\textbar{}\PYGZhy{}\PYGZhy{} views/
\textbar{}   \textbar{}\PYGZhy{}\PYGZhy{} \PYGZlt{}main\PYGZus{}model\PYGZgt{}\PYGZus{}templates.xml
\textbar{}   \textbar{}\PYGZhy{}\PYGZhy{} \PYGZlt{}main\PYGZus{}model\PYGZgt{}\PYGZus{}views.xml
\textbar{}   \textbar{}\PYGZhy{}\PYGZhy{} \PYGZlt{}inherited\PYGZus{}main\PYGZus{}model\PYGZgt{}\PYGZus{}templates.xml
\textbar{}   {}`\PYGZhy{}\PYGZhy{} \PYGZlt{}inherited\PYGZus{}main\PYGZus{}model\PYGZgt{}\PYGZus{}views.xml
{}`\PYGZhy{}\PYGZhy{} wizard/
    \textbar{}\PYGZhy{}\PYGZhy{} \PYGZlt{}main\PYGZus{}transient\PYGZus{}A\PYGZgt{}.py
    \textbar{}\PYGZhy{}\PYGZhy{} \PYGZlt{}main\PYGZus{}transient\PYGZus{}A\PYGZgt{}\PYGZus{}views.xml
    \textbar{}\PYGZhy{}\PYGZhy{} \PYGZlt{}main\PYGZus{}transient\PYGZus{}B\PYGZgt{}.py
    {}`\PYGZhy{}\PYGZhy{} \PYGZlt{}main\PYGZus{}transient\PYGZus{}B\PYGZgt{}\PYGZus{}views.xml
\end{sphinxVerbatim}

\begin{sphinxadmonition}{note}{Note:}
File names should only contain \sphinxcode{\sphinxupquote{{[}a-z0-9\_{]}}} (lowercase
alphanumerics and \sphinxcode{\sphinxupquote{\_}})
\end{sphinxadmonition}

\begin{sphinxadmonition}{warning}{Warning:}
Use correct file permissions : folder 755 and file 644.
\end{sphinxadmonition}


\subsection{XML files}
\label{\detokenize{reference/guidelines:xml-files}}

\subsubsection{Format}
\label{\detokenize{reference/guidelines:format}}
To declare a record in XML, the \sphinxstylestrong{record} notation (using \sphinxstyleemphasis{\textless{}record\textgreater{}}) is recommended:
\begin{itemize}
\item {} 
Place \sphinxcode{\sphinxupquote{id}} attribute before \sphinxcode{\sphinxupquote{model}}

\item {} 
For field declaration, \sphinxcode{\sphinxupquote{name}} attribute is first. Then place the
\sphinxstyleemphasis{value} either in the \sphinxcode{\sphinxupquote{field}} tag, either in the \sphinxcode{\sphinxupquote{eval}}
attribute, and finally other attributes (widget, options, …)
ordered by importance.

\item {} 
Try to group the record by model. In case of dependencies between
action/menu/views, this convention may not be applicable.

\item {} 
Use naming convention defined at the next point

\item {} 
The tag \sphinxstyleemphasis{\textless{}data\textgreater{}} is only used to set not-updatable data with \sphinxcode{\sphinxupquote{noupdate=1}}.
If there is only not-updatable data in the file, the \sphinxcode{\sphinxupquote{noupdate=1}} can be
set on the \sphinxcode{\sphinxupquote{\textless{}odoo\textgreater{}}} tag and do not set a \sphinxcode{\sphinxupquote{\textless{}data\textgreater{}}} tag.

\end{itemize}

\fvset{hllines={, ,}}%
\begin{sphinxVerbatim}[commandchars=\\\{\}]
\PYG{n+nt}{\PYGZlt{}record} \PYG{n+na}{id=}\PYG{l+s}{\PYGZdq{}view\PYGZus{}id\PYGZdq{}} \PYG{n+na}{model=}\PYG{l+s}{\PYGZdq{}ir.ui.view\PYGZdq{}}\PYG{n+nt}{\PYGZgt{}}
    \PYG{n+nt}{\PYGZlt{}field} \PYG{n+na}{name=}\PYG{l+s}{\PYGZdq{}name\PYGZdq{}}\PYG{n+nt}{\PYGZgt{}}view.name\PYG{n+nt}{\PYGZlt{}/field\PYGZgt{}}
    \PYG{n+nt}{\PYGZlt{}field} \PYG{n+na}{name=}\PYG{l+s}{\PYGZdq{}model\PYGZdq{}}\PYG{n+nt}{\PYGZgt{}}object\PYGZus{}name\PYG{n+nt}{\PYGZlt{}/field\PYGZgt{}}
    \PYG{n+nt}{\PYGZlt{}field} \PYG{n+na}{name=}\PYG{l+s}{\PYGZdq{}priority\PYGZdq{}} \PYG{n+na}{eval=}\PYG{l+s}{\PYGZdq{}16\PYGZdq{}}\PYG{n+nt}{/\PYGZgt{}}
    \PYG{n+nt}{\PYGZlt{}field} \PYG{n+na}{name=}\PYG{l+s}{\PYGZdq{}arch\PYGZdq{}} \PYG{n+na}{type=}\PYG{l+s}{\PYGZdq{}xml\PYGZdq{}}\PYG{n+nt}{\PYGZgt{}}
        \PYG{n+nt}{\PYGZlt{}tree}\PYG{n+nt}{\PYGZgt{}}
            \PYG{n+nt}{\PYGZlt{}field} \PYG{n+na}{name=}\PYG{l+s}{\PYGZdq{}my\PYGZus{}field\PYGZus{}1\PYGZdq{}}\PYG{n+nt}{/\PYGZgt{}}
            \PYG{n+nt}{\PYGZlt{}field} \PYG{n+na}{name=}\PYG{l+s}{\PYGZdq{}my\PYGZus{}field\PYGZus{}2\PYGZdq{}} \PYG{n+na}{string=}\PYG{l+s}{\PYGZdq{}My Label\PYGZdq{}} \PYG{n+na}{widget=}\PYG{l+s}{\PYGZdq{}statusbar\PYGZdq{}} \PYG{n+na}{statusbar\PYGZus{}visible=}\PYG{l+s}{\PYGZdq{}draft,sent,progress,done\PYGZdq{}} \PYG{n+nt}{/\PYGZgt{}}
        \PYG{n+nt}{\PYGZlt{}/tree\PYGZgt{}}
    \PYG{n+nt}{\PYGZlt{}/field\PYGZgt{}}
\PYG{n+nt}{\PYGZlt{}/record\PYGZgt{}}
\end{sphinxVerbatim}

Odoo supports custom tags acting as syntactic sugar:
\begin{itemize}
\item {} 
menuitem: use it as a shortcut to declare a \sphinxcode{\sphinxupquote{ir.ui.menu}}

\item {} 
template: use it to declare a QWeb View requiring only the \sphinxcode{\sphinxupquote{arch}} section of the view.

\item {} 
report: use to declare a {\hyperref[\detokenize{reference/actions:reference-actions-report}]{\sphinxcrossref{\DUrole{std,std-ref}{report action}}}}

\item {} 
act\_window: use it if the record notation can’t do what you want

\end{itemize}

The 4 first tags are prefered over the \sphinxstyleemphasis{record} notation.


\subsubsection{Naming xml\_id}
\label{\detokenize{reference/guidelines:naming-xml-id}}

\paragraph{Security, View and Action}
\label{\detokenize{reference/guidelines:security-view-and-action}}
Use the following pattern :
\begin{itemize}
\item {} 
For a menu: \sphinxcode{\sphinxupquote{\sphinxstyleemphasis{\textless{}model\_name\textgreater{}}\_menu}}

\item {} 
For a view: \sphinxcode{\sphinxupquote{\sphinxstyleemphasis{\textless{}model\_name\textgreater{}}\_view\_\sphinxstyleemphasis{\textless{}view\_type\textgreater{}}}}, where \sphinxstyleemphasis{view\_type} is
\sphinxcode{\sphinxupquote{kanban}}, \sphinxcode{\sphinxupquote{form}}, \sphinxcode{\sphinxupquote{tree}}, \sphinxcode{\sphinxupquote{search}}, …

\item {} 
For an action: the main action respects \sphinxcode{\sphinxupquote{\sphinxstyleemphasis{\textless{}model\_name\textgreater{}}\_action}}.
Others are suffixed with \sphinxcode{\sphinxupquote{\_\sphinxstyleemphasis{\textless{}detail\textgreater{}}}}, where \sphinxstyleemphasis{detail} is a
lowercase string briefly explaining the action.
This is used only if multiple actions are declared for the
model.

\item {} 
For a group: \sphinxcode{\sphinxupquote{\sphinxstyleemphasis{\textless{}model\_name\textgreater{}}\_group\_\sphinxstyleemphasis{\textless{}group\_name\textgreater{}}}} where \sphinxstyleemphasis{group\_name}
is the name of the group, generally ‘user’, ‘manager’, …

\item {} 
For a rule: \sphinxcode{\sphinxupquote{\sphinxstyleemphasis{\textless{}model\_name\textgreater{}}\_rule\_\sphinxstyleemphasis{\textless{}concerned\_group\textgreater{}}}} where
\sphinxstyleemphasis{concerned\_group} is the short name of the concerned group (‘user’
for the ‘model\_name\_group\_user’, ‘public’ for public user, ‘company’
for multi-company rules, …).

\item {} 
For a group : \sphinxcode{\sphinxupquote{\sphinxstyleemphasis{\textless{}model\_name\textgreater{}}\_group\_\sphinxstyleemphasis{\textless{}group\_name\textgreater{}}}} where \sphinxstyleemphasis{group\_name} is the name of the group, generally ‘user’, ‘manager’, …

\end{itemize}

\fvset{hllines={, ,}}%
\begin{sphinxVerbatim}[commandchars=\\\{\}]
\PYG{c}{\PYGZlt{}!\PYGZhy{}\PYGZhy{}}\PYG{c}{ views and menus }\PYG{c}{\PYGZhy{}\PYGZhy{}\PYGZgt{}}
\PYG{n+nt}{\PYGZlt{}record} \PYG{n+na}{id=}\PYG{l+s}{\PYGZdq{}model\PYGZus{}name\PYGZus{}view\PYGZus{}form\PYGZdq{}} \PYG{n+na}{model=}\PYG{l+s}{\PYGZdq{}ir.ui.view\PYGZdq{}}\PYG{n+nt}{\PYGZgt{}}
    ...
\PYG{n+nt}{\PYGZlt{}/record\PYGZgt{}}

\PYG{n+nt}{\PYGZlt{}record} \PYG{n+na}{id=}\PYG{l+s}{\PYGZdq{}model\PYGZus{}name\PYGZus{}view\PYGZus{}kanban\PYGZdq{}} \PYG{n+na}{model=}\PYG{l+s}{\PYGZdq{}ir.ui.view\PYGZdq{}}\PYG{n+nt}{\PYGZgt{}}
    ...
\PYG{n+nt}{\PYGZlt{}/record\PYGZgt{}}

\PYG{n+nt}{\PYGZlt{}menuitem}
    \PYG{n+na}{id=}\PYG{l+s}{\PYGZdq{}model\PYGZus{}name\PYGZus{}menu\PYGZus{}root\PYGZdq{}}
    \PYG{n+na}{name=}\PYG{l+s}{\PYGZdq{}Main Menu\PYGZdq{}}
    \PYG{n+na}{sequence=}\PYG{l+s}{\PYGZdq{}5\PYGZdq{}}
\PYG{n+nt}{/\PYGZgt{}}
\PYG{n+nt}{\PYGZlt{}menuitem}
    \PYG{n+na}{id=}\PYG{l+s}{\PYGZdq{}model\PYGZus{}name\PYGZus{}menu\PYGZus{}action\PYGZdq{}}
    \PYG{n+na}{name=}\PYG{l+s}{\PYGZdq{}Sub Menu 1\PYGZdq{}}
    \PYG{n+na}{parent=}\PYG{l+s}{\PYGZdq{}module\PYGZus{}name.module\PYGZus{}name\PYGZus{}menu\PYGZus{}root\PYGZdq{}}
    \PYG{n+na}{action=}\PYG{l+s}{\PYGZdq{}model\PYGZus{}name\PYGZus{}action\PYGZdq{}}
    \PYG{n+na}{sequence=}\PYG{l+s}{\PYGZdq{}10\PYGZdq{}}
\PYG{n+nt}{/\PYGZgt{}}

\PYG{c}{\PYGZlt{}!\PYGZhy{}\PYGZhy{}}\PYG{c}{ actions }\PYG{c}{\PYGZhy{}\PYGZhy{}\PYGZgt{}}
\PYG{n+nt}{\PYGZlt{}record} \PYG{n+na}{id=}\PYG{l+s}{\PYGZdq{}model\PYGZus{}name\PYGZus{}action\PYGZdq{}} \PYG{n+na}{model=}\PYG{l+s}{\PYGZdq{}ir.actions.act\PYGZus{}window\PYGZdq{}}\PYG{n+nt}{\PYGZgt{}}
    ...
\PYG{n+nt}{\PYGZlt{}/record\PYGZgt{}}

\PYG{n+nt}{\PYGZlt{}record} \PYG{n+na}{id=}\PYG{l+s}{\PYGZdq{}model\PYGZus{}name\PYGZus{}action\PYGZus{}child\PYGZus{}list\PYGZdq{}} \PYG{n+na}{model=}\PYG{l+s}{\PYGZdq{}ir.actions.act\PYGZus{}window\PYGZdq{}}\PYG{n+nt}{\PYGZgt{}}
    ...
\PYG{n+nt}{\PYGZlt{}/record\PYGZgt{}}

\PYG{c}{\PYGZlt{}!\PYGZhy{}\PYGZhy{}}\PYG{c}{ security }\PYG{c}{\PYGZhy{}\PYGZhy{}\PYGZgt{}}
\PYG{n+nt}{\PYGZlt{}record} \PYG{n+na}{id=}\PYG{l+s}{\PYGZdq{}module\PYGZus{}name\PYGZus{}group\PYGZus{}user\PYGZdq{}} \PYG{n+na}{model=}\PYG{l+s}{\PYGZdq{}res.groups\PYGZdq{}}\PYG{n+nt}{\PYGZgt{}}
    ...
\PYG{n+nt}{\PYGZlt{}/record\PYGZgt{}}

\PYG{n+nt}{\PYGZlt{}record} \PYG{n+na}{id=}\PYG{l+s}{\PYGZdq{}model\PYGZus{}name\PYGZus{}rule\PYGZus{}public\PYGZdq{}} \PYG{n+na}{model=}\PYG{l+s}{\PYGZdq{}ir.rule\PYGZdq{}}\PYG{n+nt}{\PYGZgt{}}
    ...
\PYG{n+nt}{\PYGZlt{}/record\PYGZgt{}}

\PYG{n+nt}{\PYGZlt{}record} \PYG{n+na}{id=}\PYG{l+s}{\PYGZdq{}model\PYGZus{}name\PYGZus{}rule\PYGZus{}company\PYGZdq{}} \PYG{n+na}{model=}\PYG{l+s}{\PYGZdq{}ir.rule\PYGZdq{}}\PYG{n+nt}{\PYGZgt{}}
    ...
\PYG{n+nt}{\PYGZlt{}/record\PYGZgt{}}
\end{sphinxVerbatim}

\begin{sphinxadmonition}{note}{Note:}
View names use dot notation \sphinxcode{\sphinxupquote{my.model.view\_type}} or
\sphinxcode{\sphinxupquote{my.model.view\_type.inherit}} instead of \sphinxstyleemphasis{“This is the form view of
My Model”}.
\end{sphinxadmonition}


\paragraph{Inherited XML}
\label{\detokenize{reference/guidelines:inherited-xml}}
The naming pattern of inherited view is
\sphinxcode{\sphinxupquote{\sphinxstyleemphasis{\textless{}base\_view\textgreater{}}\_inherit\_\sphinxstyleemphasis{\textless{}current\_module\_name\textgreater{}}}}. A module may only
extend a view once.  Suffix the orginal name with
\sphinxcode{\sphinxupquote{\_inherit\_\sphinxstyleemphasis{\textless{}current\_module\_name\textgreater{}}}} where \sphinxstyleemphasis{current\_module\_name} is the
technical name of the module extending the view.

\fvset{hllines={, ,}}%
\begin{sphinxVerbatim}[commandchars=\\\{\}]
\PYG{n+nt}{\PYGZlt{}record} \PYG{n+na}{id=}\PYG{l+s}{\PYGZdq{}inherited\PYGZus{}model\PYGZus{}view\PYGZus{}form\PYGZus{}inherit\PYGZus{}my\PYGZus{}module\PYGZdq{}} \PYG{n+na}{model=}\PYG{l+s}{\PYGZdq{}ir.ui.view\PYGZdq{}}\PYG{n+nt}{\PYGZgt{}}
    ...
\PYG{n+nt}{\PYGZlt{}/record\PYGZgt{}}
\end{sphinxVerbatim}


\subsection{Python}
\label{\detokenize{reference/guidelines:python}}

\subsubsection{PEP8 options}
\label{\detokenize{reference/guidelines:pep8-options}}
Using a linter can help show syntax and semantic warnings or errors. Odoo
source code tries to respect Python standard, but some of them can be ignored.
\begin{itemize}
\item {} 
E501: line too long

\item {} 
E301: expected 1 blank line, found 0

\item {} 
E302: expected 2 blank lines, found 1

\item {} 
E126: continuation line over-indented for hanging indent

\item {} 
E123: closing bracket does not match indentation of opening bracket’s line

\item {} 
E127: continuation line over-indented for visual indent

\item {} 
E128: continuation line under-indented for visual indent

\item {} 
E265: block comment should start with ‘\# ‘

\end{itemize}


\subsubsection{Imports}
\label{\detokenize{reference/guidelines:imports}}
The imports are ordered as
\begin{enumerate}
\item {} 
External libraries (one per line sorted and split in python stdlib)

\item {} 
Imports of \sphinxcode{\sphinxupquote{odoo}}

\item {} 
Imports from Odoo modules (rarely, and only if necessary)

\end{enumerate}

Inside these 3 groups, the imported lines are alphabetically sorted.

\fvset{hllines={, ,}}%
\begin{sphinxVerbatim}[commandchars=\\\{\}]
\PYG{c+c1}{\PYGZsh{} 1 : imports of python lib}
\PYG{k+kn}{import} \PYG{n+nn}{base64}
\PYG{k+kn}{import} \PYG{n+nn}{re}
\PYG{k+kn}{import} \PYG{n+nn}{time}
\PYG{k+kn}{from} \PYG{n+nn}{datetime} \PYG{k+kn}{import} \PYG{n}{datetime}
\PYG{c+c1}{\PYGZsh{} 2 :  imports of odoo}
\PYG{k+kn}{import} \PYG{n+nn}{odoo}
\PYG{k+kn}{from} \PYG{n+nn}{odoo} \PYG{k+kn}{import} \PYG{n}{api}\PYG{p}{,} \PYG{n}{fields}\PYG{p}{,} \PYG{n}{models} \PYG{c+c1}{\PYGZsh{} alphabetically ordered}
\PYG{k+kn}{from} \PYG{n+nn}{odoo.tools.safe\PYGZus{}eval} \PYG{k+kn}{import} \PYG{n}{safe\PYGZus{}eval} \PYG{k}{as} \PYG{n+nb}{eval}
\PYG{k+kn}{from} \PYG{n+nn}{odoo.tools.translate} \PYG{k+kn}{import} \PYG{n}{\PYGZus{}}
\PYG{c+c1}{\PYGZsh{} 3 :  imports from odoo modules}
\PYG{k+kn}{from} \PYG{n+nn}{odoo.addons.website.models.website} \PYG{k+kn}{import} \PYG{n}{slug}
\PYG{k+kn}{from} \PYG{n+nn}{odoo.addons.web.controllers.main} \PYG{k+kn}{import} \PYG{n}{login\PYGZus{}redirect}
\end{sphinxVerbatim}


\subsubsection{Idiomatics Python Programming}
\label{\detokenize{reference/guidelines:idiomatics-python-programming}}\begin{itemize}
\item {} 
Each python file should have \sphinxcode{\sphinxupquote{\# -*- coding: utf-8 -*-}} as first line.

\item {} 
Always favor \sphinxstyleemphasis{readability} over \sphinxstyleemphasis{conciseness} or using the language features or idioms.

\item {} 
Don’t use \sphinxcode{\sphinxupquote{.clone()}}

\end{itemize}

\fvset{hllines={, ,}}%
\begin{sphinxVerbatim}[commandchars=\\\{\}]
\PYG{c+c1}{\PYGZsh{} bad}
\PYG{n}{new\PYGZus{}dict} \PYG{o}{=} \PYG{n}{my\PYGZus{}dict}\PYG{o}{.}\PYG{n}{clone}\PYG{p}{(}\PYG{p}{)}
\PYG{n}{new\PYGZus{}list} \PYG{o}{=} \PYG{n}{old\PYGZus{}list}\PYG{o}{.}\PYG{n}{clone}\PYG{p}{(}\PYG{p}{)}
\PYG{c+c1}{\PYGZsh{} good}
\PYG{n}{new\PYGZus{}dict} \PYG{o}{=} \PYG{n+nb}{dict}\PYG{p}{(}\PYG{n}{my\PYGZus{}dict}\PYG{p}{)}
\PYG{n}{new\PYGZus{}list} \PYG{o}{=} \PYG{n+nb}{list}\PYG{p}{(}\PYG{n}{old\PYGZus{}list}\PYG{p}{)}
\end{sphinxVerbatim}
\begin{itemize}
\item {} 
Python dictionary : creation and update

\end{itemize}

\fvset{hllines={, ,}}%
\begin{sphinxVerbatim}[commandchars=\\\{\}]
\PYG{c+c1}{\PYGZsh{} \PYGZhy{}\PYGZhy{} creation empty dict}
\PYG{n}{my\PYGZus{}dict} \PYG{o}{=} \PYG{p}{\PYGZob{}}\PYG{p}{\PYGZcb{}}
\PYG{n}{my\PYGZus{}dict2} \PYG{o}{=} \PYG{n+nb}{dict}\PYG{p}{(}\PYG{p}{)}

\PYG{c+c1}{\PYGZsh{} \PYGZhy{}\PYGZhy{} creation with values}
\PYG{c+c1}{\PYGZsh{} bad}
\PYG{n}{my\PYGZus{}dict} \PYG{o}{=} \PYG{p}{\PYGZob{}}\PYG{p}{\PYGZcb{}}
\PYG{n}{my\PYGZus{}dict}\PYG{p}{[}\PYG{l+s+s1}{\PYGZsq{}}\PYG{l+s+s1}{foo}\PYG{l+s+s1}{\PYGZsq{}}\PYG{p}{]} \PYG{o}{=} \PYG{l+m+mi}{3}
\PYG{n}{my\PYGZus{}dict}\PYG{p}{[}\PYG{l+s+s1}{\PYGZsq{}}\PYG{l+s+s1}{bar}\PYG{l+s+s1}{\PYGZsq{}}\PYG{p}{]} \PYG{o}{=} \PYG{l+m+mi}{4}
\PYG{c+c1}{\PYGZsh{} good}
\PYG{n}{my\PYGZus{}dict} \PYG{o}{=} \PYG{p}{\PYGZob{}}\PYG{l+s+s1}{\PYGZsq{}}\PYG{l+s+s1}{foo}\PYG{l+s+s1}{\PYGZsq{}}\PYG{p}{:} \PYG{l+m+mi}{3}\PYG{p}{,} \PYG{l+s+s1}{\PYGZsq{}}\PYG{l+s+s1}{bar}\PYG{l+s+s1}{\PYGZsq{}}\PYG{p}{:} \PYG{l+m+mi}{4}\PYG{p}{\PYGZcb{}}

\PYG{c+c1}{\PYGZsh{} \PYGZhy{}\PYGZhy{} update dict}
\PYG{c+c1}{\PYGZsh{} bad}
\PYG{n}{my\PYGZus{}dict}\PYG{p}{[}\PYG{l+s+s1}{\PYGZsq{}}\PYG{l+s+s1}{foo}\PYG{l+s+s1}{\PYGZsq{}}\PYG{p}{]} \PYG{o}{=} \PYG{l+m+mi}{3}
\PYG{n}{my\PYGZus{}dict}\PYG{p}{[}\PYG{l+s+s1}{\PYGZsq{}}\PYG{l+s+s1}{bar}\PYG{l+s+s1}{\PYGZsq{}}\PYG{p}{]} \PYG{o}{=} \PYG{l+m+mi}{4}
\PYG{n}{my\PYGZus{}dict}\PYG{p}{[}\PYG{l+s+s1}{\PYGZsq{}}\PYG{l+s+s1}{baz}\PYG{l+s+s1}{\PYGZsq{}}\PYG{p}{]} \PYG{o}{=} \PYG{l+m+mi}{5}
\PYG{c+c1}{\PYGZsh{} good}
\PYG{n}{my\PYGZus{}dict}\PYG{o}{.}\PYG{n}{update}\PYG{p}{(}\PYG{n}{foo}\PYG{o}{=}\PYG{l+m+mi}{3}\PYG{p}{,} \PYG{n}{bar}\PYG{o}{=}\PYG{l+m+mi}{4}\PYG{p}{,} \PYG{n}{baz}\PYG{o}{=}\PYG{l+m+mi}{5}\PYG{p}{)}
\PYG{n}{my\PYGZus{}dict} \PYG{o}{=} \PYG{n+nb}{dict}\PYG{p}{(}\PYG{n}{my\PYGZus{}dict}\PYG{p}{,} \PYG{o}{*}\PYG{o}{*}\PYG{n}{my\PYGZus{}dict2}\PYG{p}{)}
\end{sphinxVerbatim}
\begin{itemize}
\item {} 
Use meaningful variable/class/method names

\item {} 
Useless variable : Temporary variables can make the code clearer by giving
names to objects, but that doesn’t mean you should create temporary variables
all the time:

\end{itemize}

\fvset{hllines={, ,}}%
\begin{sphinxVerbatim}[commandchars=\\\{\}]
\PYG{c+c1}{\PYGZsh{} pointless}
\PYG{n}{schema} \PYG{o}{=} \PYG{n}{kw}\PYG{p}{[}\PYG{l+s+s1}{\PYGZsq{}}\PYG{l+s+s1}{schema}\PYG{l+s+s1}{\PYGZsq{}}\PYG{p}{]}
\PYG{n}{params} \PYG{o}{=} \PYG{p}{\PYGZob{}}\PYG{l+s+s1}{\PYGZsq{}}\PYG{l+s+s1}{schema}\PYG{l+s+s1}{\PYGZsq{}}\PYG{p}{:} \PYG{n}{schema}\PYG{p}{\PYGZcb{}}
\PYG{c+c1}{\PYGZsh{} simpler}
\PYG{n}{params} \PYG{o}{=} \PYG{p}{\PYGZob{}}\PYG{l+s+s1}{\PYGZsq{}}\PYG{l+s+s1}{schema}\PYG{l+s+s1}{\PYGZsq{}}\PYG{p}{:} \PYG{n}{kw}\PYG{p}{[}\PYG{l+s+s1}{\PYGZsq{}}\PYG{l+s+s1}{schema}\PYG{l+s+s1}{\PYGZsq{}}\PYG{p}{]}\PYG{p}{\PYGZcb{}}
\end{sphinxVerbatim}
\begin{itemize}
\item {} 
Multiple return points are OK, when they’re simpler

\end{itemize}

\fvset{hllines={, ,}}%
\begin{sphinxVerbatim}[commandchars=\\\{\}]
\PYG{c+c1}{\PYGZsh{} a bit complex and with a redundant temp variable}
\PYG{k}{def} \PYG{n+nf}{axes}\PYG{p}{(}\PYG{n+nb+bp}{self}\PYG{p}{,} \PYG{n}{axis}\PYG{p}{)}\PYG{p}{:}
        \PYG{n}{axes} \PYG{o}{=} \PYG{p}{[}\PYG{p}{]}
        \PYG{k}{if} \PYG{n+nb}{type}\PYG{p}{(}\PYG{n}{axis}\PYG{p}{)} \PYG{o}{==} \PYG{n+nb}{type}\PYG{p}{(}\PYG{p}{[}\PYG{p}{]}\PYG{p}{)}\PYG{p}{:}
                \PYG{n}{axes}\PYG{o}{.}\PYG{n}{extend}\PYG{p}{(}\PYG{n}{axis}\PYG{p}{)}
        \PYG{k}{else}\PYG{p}{:}
                \PYG{n}{axes}\PYG{o}{.}\PYG{n}{append}\PYG{p}{(}\PYG{n}{axis}\PYG{p}{)}
        \PYG{k}{return} \PYG{n}{axes}

 \PYG{c+c1}{\PYGZsh{} clearer}
\PYG{k}{def} \PYG{n+nf}{axes}\PYG{p}{(}\PYG{n+nb+bp}{self}\PYG{p}{,} \PYG{n}{axis}\PYG{p}{)}\PYG{p}{:}
        \PYG{k}{if} \PYG{n+nb}{type}\PYG{p}{(}\PYG{n}{axis}\PYG{p}{)} \PYG{o}{==} \PYG{n+nb}{type}\PYG{p}{(}\PYG{p}{[}\PYG{p}{]}\PYG{p}{)}\PYG{p}{:}
                \PYG{k}{return} \PYG{n+nb}{list}\PYG{p}{(}\PYG{n}{axis}\PYG{p}{)} \PYG{c+c1}{\PYGZsh{} clone the axis}
        \PYG{k}{else}\PYG{p}{:}
                \PYG{k}{return} \PYG{p}{[}\PYG{n}{axis}\PYG{p}{]} \PYG{c+c1}{\PYGZsh{} single\PYGZhy{}element list}
\end{sphinxVerbatim}
\begin{itemize}
\item {} 
Know your builtins : You should at least have a basic understanding of all
the Python builtins (\sphinxurl{http://docs.python.org/library/functions.html})

\end{itemize}

\fvset{hllines={, ,}}%
\begin{sphinxVerbatim}[commandchars=\\\{\}]
\PYG{n}{value} \PYG{o}{=} \PYG{n}{my\PYGZus{}dict}\PYG{o}{.}\PYG{n}{get}\PYG{p}{(}\PYG{l+s+s1}{\PYGZsq{}}\PYG{l+s+s1}{key}\PYG{l+s+s1}{\PYGZsq{}}\PYG{p}{,} \PYG{n+nb+bp}{None}\PYG{p}{)} \PYG{c+c1}{\PYGZsh{} very very redundant}
\PYG{n}{value} \PYG{o}{=} \PYG{n}{my\PYGZus{}dict}\PYG{o}{.}\PYG{n}{get}\PYG{p}{(}\PYG{l+s+s1}{\PYGZsq{}}\PYG{l+s+s1}{key}\PYG{l+s+s1}{\PYGZsq{}}\PYG{p}{)} \PYG{c+c1}{\PYGZsh{} good}
\end{sphinxVerbatim}

Also, \sphinxcode{\sphinxupquote{if 'key' in my\_dict}} and \sphinxcode{\sphinxupquote{if my\_dict.get('key')}} have very different
meaning, be sure that you’re using the right one.
\begin{itemize}
\item {} 
Learn list comprehensions : Use list comprehension, dict comprehension, and
basic manipulation using \sphinxcode{\sphinxupquote{map}}, \sphinxcode{\sphinxupquote{filter}}, \sphinxcode{\sphinxupquote{sum}}, … They make the code
easier to read.

\end{itemize}

\fvset{hllines={, ,}}%
\begin{sphinxVerbatim}[commandchars=\\\{\}]
\PYG{c+c1}{\PYGZsh{} not very good}
\PYG{n}{cube} \PYG{o}{=} \PYG{p}{[}\PYG{p}{]}
\PYG{k}{for} \PYG{n}{i} \PYG{o+ow}{in} \PYG{n}{res}\PYG{p}{:}
        \PYG{n}{cube}\PYG{o}{.}\PYG{n}{append}\PYG{p}{(}\PYG{p}{(}\PYG{n}{i}\PYG{p}{[}\PYG{l+s+s1}{\PYGZsq{}}\PYG{l+s+s1}{id}\PYG{l+s+s1}{\PYGZsq{}}\PYG{p}{]}\PYG{p}{,}\PYG{n}{i}\PYG{p}{[}\PYG{l+s+s1}{\PYGZsq{}}\PYG{l+s+s1}{name}\PYG{l+s+s1}{\PYGZsq{}}\PYG{p}{]}\PYG{p}{)}\PYG{p}{)}
\PYG{c+c1}{\PYGZsh{} better}
\PYG{n}{cube} \PYG{o}{=} \PYG{p}{[}\PYG{p}{(}\PYG{n}{i}\PYG{p}{[}\PYG{l+s+s1}{\PYGZsq{}}\PYG{l+s+s1}{id}\PYG{l+s+s1}{\PYGZsq{}}\PYG{p}{]}\PYG{p}{,} \PYG{n}{i}\PYG{p}{[}\PYG{l+s+s1}{\PYGZsq{}}\PYG{l+s+s1}{name}\PYG{l+s+s1}{\PYGZsq{}}\PYG{p}{]}\PYG{p}{)} \PYG{k}{for} \PYG{n}{i} \PYG{o+ow}{in} \PYG{n}{res}\PYG{p}{]}
\end{sphinxVerbatim}
\begin{itemize}
\item {} 
Collections are booleans too : In python, many objects have “boolean-ish” value
when evaluated in a boolean context (such as an if). Among these are collections
(lists, dicts, sets, …) which are “falsy” when empty and “truthy” when containing
items:

\end{itemize}

\fvset{hllines={, ,}}%
\begin{sphinxVerbatim}[commandchars=\\\{\}]
\PYG{n+nb}{bool}\PYG{p}{(}\PYG{p}{[}\PYG{p}{]}\PYG{p}{)} \PYG{o+ow}{is} \PYG{n+nb+bp}{False}
\PYG{n+nb}{bool}\PYG{p}{(}\PYG{p}{[}\PYG{l+m+mi}{1}\PYG{p}{]}\PYG{p}{)} \PYG{o+ow}{is} \PYG{n+nb+bp}{True}
\PYG{n+nb}{bool}\PYG{p}{(}\PYG{p}{[}\PYG{n+nb+bp}{False}\PYG{p}{]}\PYG{p}{)} \PYG{o+ow}{is} \PYG{n+nb+bp}{True}
\end{sphinxVerbatim}

So, you can write \sphinxcode{\sphinxupquote{if some\_collection:}} instead of \sphinxcode{\sphinxupquote{if len(some\_collection):}}.
\begin{itemize}
\item {} 
Iterate on iterables

\end{itemize}

\fvset{hllines={, ,}}%
\begin{sphinxVerbatim}[commandchars=\\\{\}]
\PYG{c+c1}{\PYGZsh{} creates a temporary list and looks bar}
\PYG{k}{for} \PYG{n}{key} \PYG{o+ow}{in} \PYG{n}{my\PYGZus{}dict}\PYG{o}{.}\PYG{n}{keys}\PYG{p}{(}\PYG{p}{)}\PYG{p}{:}
        \PYG{l+s+s2}{\PYGZdq{}}\PYG{l+s+s2}{do something...}\PYG{l+s+s2}{\PYGZdq{}}
\PYG{c+c1}{\PYGZsh{} better}
\PYG{k}{for} \PYG{n}{key} \PYG{o+ow}{in} \PYG{n}{my\PYGZus{}dict}\PYG{p}{:}
        \PYG{l+s+s2}{\PYGZdq{}}\PYG{l+s+s2}{do something...}\PYG{l+s+s2}{\PYGZdq{}}
\PYG{c+c1}{\PYGZsh{} accessing the key,value pair}
\PYG{k}{for} \PYG{n}{key}\PYG{p}{,} \PYG{n}{value} \PYG{o+ow}{in} \PYG{n}{my\PYGZus{}dict}\PYG{o}{.}\PYG{n}{items}\PYG{p}{(}\PYG{p}{)}\PYG{p}{:}
        \PYG{l+s+s2}{\PYGZdq{}}\PYG{l+s+s2}{do something...}\PYG{l+s+s2}{\PYGZdq{}}
\end{sphinxVerbatim}
\begin{itemize}
\item {} 
Use dict.setdefault

\end{itemize}

\fvset{hllines={, ,}}%
\begin{sphinxVerbatim}[commandchars=\\\{\}]
\PYG{c+c1}{\PYGZsh{} longer.. harder to read}
\PYG{n}{values} \PYG{o}{=} \PYG{p}{\PYGZob{}}\PYG{p}{\PYGZcb{}}
\PYG{k}{for} \PYG{n}{element} \PYG{o+ow}{in} \PYG{n}{iterable}\PYG{p}{:}
    \PYG{k}{if} \PYG{n}{element} \PYG{o+ow}{not} \PYG{o+ow}{in} \PYG{n}{values}\PYG{p}{:}
        \PYG{n}{values}\PYG{p}{[}\PYG{n}{element}\PYG{p}{]} \PYG{o}{=} \PYG{p}{[}\PYG{p}{]}
    \PYG{n}{values}\PYG{p}{[}\PYG{n}{element}\PYG{p}{]}\PYG{o}{.}\PYG{n}{append}\PYG{p}{(}\PYG{n}{other\PYGZus{}value}\PYG{p}{)}

\PYG{c+c1}{\PYGZsh{} better.. use dict.setdefault method}
\PYG{n}{values} \PYG{o}{=} \PYG{p}{\PYGZob{}}\PYG{p}{\PYGZcb{}}
\PYG{k}{for} \PYG{n}{element} \PYG{o+ow}{in} \PYG{n}{iterable}\PYG{p}{:}
    \PYG{n}{values}\PYG{o}{.}\PYG{n}{setdefault}\PYG{p}{(}\PYG{n}{element}\PYG{p}{,} \PYG{p}{[}\PYG{p}{]}\PYG{p}{)}\PYG{o}{.}\PYG{n}{append}\PYG{p}{(}\PYG{n}{other\PYGZus{}value}\PYG{p}{)}
\end{sphinxVerbatim}
\begin{itemize}
\item {} 
As a good developper, document your code (docstring on methods, simple
comments for tricky part of code)

\item {} 
In additions to these guidelines, you may also find the following link
interesting: \sphinxurl{http://python.net/~goodger/projects/pycon/2007/idiomatic/handout.html}
(a little bit outdated, but quite relevant)

\end{itemize}


\subsubsection{Programming in Odoo}
\label{\detokenize{reference/guidelines:programming-in-odoo}}\begin{itemize}
\item {} 
Avoid to create generators and decorators: only use the ones provided by
the Odoo API.

\item {} 
As in python, use \sphinxcode{\sphinxupquote{filtered}}, \sphinxcode{\sphinxupquote{mapped}}, \sphinxcode{\sphinxupquote{sorted}}, … methods to
ease code reading and performance.

\end{itemize}


\paragraph{Make your method work in batch}
\label{\detokenize{reference/guidelines:make-your-method-work-in-batch}}
When adding a function, make sure it can process multiple records. Typically,
such methods are decorated with the \sphinxcode{\sphinxupquote{api.multi}} decorator. Then you will have
to iterate on \sphinxcode{\sphinxupquote{self}} to treat each record.

\fvset{hllines={, ,}}%
\begin{sphinxVerbatim}[commandchars=\\\{\}]
\PYG{n+nd}{@api.multi}
\PYG{k}{def} \PYG{n+nf}{my\PYGZus{}method}\PYG{p}{(}\PYG{n+nb+bp}{self}\PYG{p}{)}
    \PYG{k}{for} \PYG{n}{record} \PYG{o+ow}{in} \PYG{n+nb+bp}{self}\PYG{p}{:}
        \PYG{n}{record}\PYG{o}{.}\PYG{n}{do\PYGZus{}cool\PYGZus{}stuff}\PYG{p}{(}\PYG{p}{)}
\end{sphinxVerbatim}

Avoid to use \sphinxcode{\sphinxupquote{api.one}}  decorator : this will probably not do what you expected,
and extending a such method is not as easy than a \sphinxstyleemphasis{api.multi} method, since it
returns a list of result (ordered by recordset ids).

For performance issue, when developping a ‘stat button’ (for instance), do not
perform a \sphinxcode{\sphinxupquote{search}} or a \sphinxcode{\sphinxupquote{search\_count}} in a loop in a \sphinxcode{\sphinxupquote{api.multi}} method. It
is recommended to use \sphinxcode{\sphinxupquote{read\_group}} method, to compute all value in only one request.

\fvset{hllines={, ,}}%
\begin{sphinxVerbatim}[commandchars=\\\{\}]
\PYG{n+nd}{@api.multi}
\PYG{k}{def} \PYG{n+nf}{\PYGZus{}compute\PYGZus{}equipment\PYGZus{}count}\PYG{p}{(}\PYG{n+nb+bp}{self}\PYG{p}{)}\PYG{p}{:}
\PYG{l+s+sd}{\PYGZdq{}\PYGZdq{}\PYGZdq{} Count the number of equipement per category \PYGZdq{}\PYGZdq{}\PYGZdq{}}
    \PYG{n}{equipment\PYGZus{}data} \PYG{o}{=} \PYG{n+nb+bp}{self}\PYG{o}{.}\PYG{n}{env}\PYG{p}{[}\PYG{l+s+s1}{\PYGZsq{}}\PYG{l+s+s1}{hr.equipment}\PYG{l+s+s1}{\PYGZsq{}}\PYG{p}{]}\PYG{o}{.}\PYG{n}{read\PYGZus{}group}\PYG{p}{(}\PYG{p}{[}\PYG{p}{(}\PYG{l+s+s1}{\PYGZsq{}}\PYG{l+s+s1}{category\PYGZus{}id}\PYG{l+s+s1}{\PYGZsq{}}\PYG{p}{,} \PYG{l+s+s1}{\PYGZsq{}}\PYG{l+s+s1}{in}\PYG{l+s+s1}{\PYGZsq{}}\PYG{p}{,} \PYG{n+nb+bp}{self}\PYG{o}{.}\PYG{n}{ids}\PYG{p}{)}\PYG{p}{]}\PYG{p}{,} \PYG{p}{[}\PYG{l+s+s1}{\PYGZsq{}}\PYG{l+s+s1}{category\PYGZus{}id}\PYG{l+s+s1}{\PYGZsq{}}\PYG{p}{]}\PYG{p}{,} \PYG{p}{[}\PYG{l+s+s1}{\PYGZsq{}}\PYG{l+s+s1}{category\PYGZus{}id}\PYG{l+s+s1}{\PYGZsq{}}\PYG{p}{]}\PYG{p}{)}
    \PYG{n}{mapped\PYGZus{}data} \PYG{o}{=} \PYG{n+nb}{dict}\PYG{p}{(}\PYG{p}{[}\PYG{p}{(}\PYG{n}{m}\PYG{p}{[}\PYG{l+s+s1}{\PYGZsq{}}\PYG{l+s+s1}{category\PYGZus{}id}\PYG{l+s+s1}{\PYGZsq{}}\PYG{p}{]}\PYG{p}{[}\PYG{l+m+mi}{0}\PYG{p}{]}\PYG{p}{,} \PYG{n}{m}\PYG{p}{[}\PYG{l+s+s1}{\PYGZsq{}}\PYG{l+s+s1}{category\PYGZus{}id\PYGZus{}count}\PYG{l+s+s1}{\PYGZsq{}}\PYG{p}{]}\PYG{p}{)} \PYG{k}{for} \PYG{n}{m} \PYG{o+ow}{in} \PYG{n}{equipment\PYGZus{}data}\PYG{p}{]}\PYG{p}{)}
    \PYG{k}{for} \PYG{n}{category} \PYG{o+ow}{in} \PYG{n+nb+bp}{self}\PYG{p}{:}
        \PYG{n}{category}\PYG{o}{.}\PYG{n}{equipment\PYGZus{}count} \PYG{o}{=} \PYG{n}{mapped\PYGZus{}data}\PYG{o}{.}\PYG{n}{get}\PYG{p}{(}\PYG{n}{category}\PYG{o}{.}\PYG{n}{id}\PYG{p}{,} \PYG{l+m+mi}{0}\PYG{p}{)}
\end{sphinxVerbatim}


\paragraph{Propagate the context}
\label{\detokenize{reference/guidelines:propagate-the-context}}
The context is a \sphinxcode{\sphinxupquote{frozendict}} that cannot be modified. To call a method with
a different context, the \sphinxcode{\sphinxupquote{with\_context}} method should be used :

\fvset{hllines={, ,}}%
\begin{sphinxVerbatim}[commandchars=\\\{\}]
\PYG{n}{records}\PYG{o}{.}\PYG{n}{with\PYGZus{}context}\PYG{p}{(}\PYG{n}{new\PYGZus{}context}\PYG{p}{)}\PYG{o}{.}\PYG{n}{do\PYGZus{}stuff}\PYG{p}{(}\PYG{p}{)} \PYG{c+c1}{\PYGZsh{} all the context is replaced}
\PYG{n}{records}\PYG{o}{.}\PYG{n}{with\PYGZus{}context}\PYG{p}{(}\PYG{o}{*}\PYG{o}{*}\PYG{n}{additionnal\PYGZus{}context}\PYG{p}{)}\PYG{o}{.}\PYG{n}{do\PYGZus{}other\PYGZus{}stuff}\PYG{p}{(}\PYG{p}{)} \PYG{c+c1}{\PYGZsh{} additionnal\PYGZus{}context values override native context ones}
\end{sphinxVerbatim}

Passing parameter in context can have dangerous side-effects. Since the values
are propagated automatically, some behavior can appears. Calling \sphinxcode{\sphinxupquote{create()}}
method of a model with \sphinxstyleemphasis{default\_my\_field} key in context will set the default
value of \sphinxstyleemphasis{my\_field} for the concerned model. But if curing this creation, other
object (such as sale.order.line, on sale.order creation) having a field
name \sphinxstyleemphasis{my\_field}, their default value will be set too.

If you need to create a key context influencing the behavior of some object,
choice a good name, and eventually prefix it by the name of the module to
isolate its impact. A good example are the keys of \sphinxcode{\sphinxupquote{mail}} module :
\sphinxstyleemphasis{mail\_create\_nosubscribe}, \sphinxstyleemphasis{mail\_notrack}, \sphinxstyleemphasis{mail\_notify\_user\_signature}, …


\paragraph{Do not bypass the ORM}
\label{\detokenize{reference/guidelines:do-not-bypass-the-orm}}
You should never use the database cursor directly when the ORM can do the same
thing! By doing so you are bypassing all the ORM features, possibly the
transactions, access rights and so on.

And chances are that you are also making the code harder to read and probably
less secure.

\fvset{hllines={, ,}}%
\begin{sphinxVerbatim}[commandchars=\\\{\}]
\PYG{c+c1}{\PYGZsh{} very very wrong}
\PYG{n+nb+bp}{self}\PYG{o}{.}\PYG{n}{env}\PYG{o}{.}\PYG{n}{cr}\PYG{o}{.}\PYG{n}{execute}\PYG{p}{(}\PYG{l+s+s1}{\PYGZsq{}}\PYG{l+s+s1}{SELECT id FROM auction\PYGZus{}lots WHERE auction\PYGZus{}id in (}\PYG{l+s+s1}{\PYGZsq{}} \PYG{o}{+} \PYG{l+s+s1}{\PYGZsq{}}\PYG{l+s+s1}{,}\PYG{l+s+s1}{\PYGZsq{}}\PYG{o}{.}\PYG{n}{join}\PYG{p}{(}\PYG{n+nb}{map}\PYG{p}{(}\PYG{n+nb}{str}\PYG{p}{,} \PYG{n}{ids}\PYG{p}{)}\PYG{p}{)}\PYG{o}{+}\PYG{l+s+s1}{\PYGZsq{}}\PYG{l+s+s1}{) AND state=}\PYG{l+s+si}{\PYGZpc{}s}\PYG{l+s+s1}{ AND obj\PYGZus{}price \PYGZgt{} 0}\PYG{l+s+s1}{\PYGZsq{}}\PYG{p}{,} \PYG{p}{(}\PYG{l+s+s1}{\PYGZsq{}}\PYG{l+s+s1}{draft}\PYG{l+s+s1}{\PYGZsq{}}\PYG{p}{,}\PYG{p}{)}\PYG{p}{)}
\PYG{n}{auction\PYGZus{}lots\PYGZus{}ids} \PYG{o}{=} \PYG{p}{[}\PYG{n}{x}\PYG{p}{[}\PYG{l+m+mi}{0}\PYG{p}{]} \PYG{k}{for} \PYG{n}{x} \PYG{o+ow}{in} \PYG{n+nb+bp}{self}\PYG{o}{.}\PYG{n}{env}\PYG{o}{.}\PYG{n}{cr}\PYG{o}{.}\PYG{n}{fetchall}\PYG{p}{(}\PYG{p}{)}\PYG{p}{]}

\PYG{c+c1}{\PYGZsh{} no injection, but still wrong}
\PYG{n+nb+bp}{self}\PYG{o}{.}\PYG{n}{env}\PYG{o}{.}\PYG{n}{cr}\PYG{o}{.}\PYG{n}{execute}\PYG{p}{(}\PYG{l+s+s1}{\PYGZsq{}}\PYG{l+s+s1}{SELECT id FROM auction\PYGZus{}lots WHERE auction\PYGZus{}id in }\PYG{l+s+si}{\PYGZpc{}s}\PYG{l+s+s1}{ }\PYG{l+s+s1}{\PYGZsq{}}\PYGZbs{}
           \PYG{l+s+s1}{\PYGZsq{}}\PYG{l+s+s1}{AND state=}\PYG{l+s+si}{\PYGZpc{}s}\PYG{l+s+s1}{ AND obj\PYGZus{}price \PYGZgt{} 0}\PYG{l+s+s1}{\PYGZsq{}}\PYG{p}{,} \PYG{p}{(}\PYG{n+nb}{tuple}\PYG{p}{(}\PYG{n}{ids}\PYG{p}{)}\PYG{p}{,} \PYG{l+s+s1}{\PYGZsq{}}\PYG{l+s+s1}{draft}\PYG{l+s+s1}{\PYGZsq{}}\PYG{p}{,}\PYG{p}{)}\PYG{p}{)}
\PYG{n}{auction\PYGZus{}lots\PYGZus{}ids} \PYG{o}{=} \PYG{p}{[}\PYG{n}{x}\PYG{p}{[}\PYG{l+m+mi}{0}\PYG{p}{]} \PYG{k}{for} \PYG{n}{x} \PYG{o+ow}{in} \PYG{n+nb+bp}{self}\PYG{o}{.}\PYG{n}{env}\PYG{o}{.}\PYG{n}{cr}\PYG{o}{.}\PYG{n}{fetchall}\PYG{p}{(}\PYG{p}{)}\PYG{p}{]}

\PYG{c+c1}{\PYGZsh{} better}
\PYG{n}{auction\PYGZus{}lots\PYGZus{}ids} \PYG{o}{=} \PYG{n+nb+bp}{self}\PYG{o}{.}\PYG{n}{search}\PYG{p}{(}\PYG{p}{[}\PYG{p}{(}\PYG{l+s+s1}{\PYGZsq{}}\PYG{l+s+s1}{auction\PYGZus{}id}\PYG{l+s+s1}{\PYGZsq{}}\PYG{p}{,}\PYG{l+s+s1}{\PYGZsq{}}\PYG{l+s+s1}{in}\PYG{l+s+s1}{\PYGZsq{}}\PYG{p}{,}\PYG{n}{ids}\PYG{p}{)}\PYG{p}{,} \PYG{p}{(}\PYG{l+s+s1}{\PYGZsq{}}\PYG{l+s+s1}{state}\PYG{l+s+s1}{\PYGZsq{}}\PYG{p}{,}\PYG{l+s+s1}{\PYGZsq{}}\PYG{l+s+s1}{=}\PYG{l+s+s1}{\PYGZsq{}}\PYG{p}{,}\PYG{l+s+s1}{\PYGZsq{}}\PYG{l+s+s1}{draft}\PYG{l+s+s1}{\PYGZsq{}}\PYG{p}{)}\PYG{p}{,} \PYG{p}{(}\PYG{l+s+s1}{\PYGZsq{}}\PYG{l+s+s1}{obj\PYGZus{}price}\PYG{l+s+s1}{\PYGZsq{}}\PYG{p}{,}\PYG{l+s+s1}{\PYGZsq{}}\PYG{l+s+s1}{\PYGZgt{}}\PYG{l+s+s1}{\PYGZsq{}}\PYG{p}{,}\PYG{l+m+mi}{0}\PYG{p}{)}\PYG{p}{]}\PYG{p}{)}
\end{sphinxVerbatim}


\paragraph{No SQL injections, please !}
\label{\detokenize{reference/guidelines:no-sql-injections-please}}
Care must be taken not to introduce SQL injections vulnerabilities when using
manual SQL queries. The vulnerability is present when user input is either
incorrectly filtered or badly quoted, allowing an attacker to introduce
undesirable clauses to a SQL query (such as circumventing filters or
executing UPDATE or DELETE commands).

The best way to be safe is to never, NEVER use Python string concatenation (+)
or string parameters interpolation (\%) to pass variables to a SQL query string.

The second reason, which is almost as important, is that it is the job of the
database abstraction layer (psycopg2) to decide how to format query parameters,
not your job! For example psycopg2 knows that when you pass a list of values
it needs to format them as a comma-separated list, enclosed in parentheses !

\fvset{hllines={, ,}}%
\begin{sphinxVerbatim}[commandchars=\\\{\}]
\PYG{c+c1}{\PYGZsh{} the following is very bad:}
\PYG{c+c1}{\PYGZsh{}   \PYGZhy{} it\PYGZsq{}s a SQL injection vulnerability}
\PYG{c+c1}{\PYGZsh{}   \PYGZhy{} it\PYGZsq{}s unreadable}
\PYG{c+c1}{\PYGZsh{}   \PYGZhy{} it\PYGZsq{}s not your job to format the list of ids}
\PYG{n+nb+bp}{self}\PYG{o}{.}\PYG{n}{env}\PYG{o}{.}\PYG{n}{cr}\PYG{o}{.}\PYG{n}{execute}\PYG{p}{(}\PYG{l+s+s1}{\PYGZsq{}}\PYG{l+s+s1}{SELECT distinct child\PYGZus{}id FROM account\PYGZus{}account\PYGZus{}consol\PYGZus{}rel }\PYG{l+s+s1}{\PYGZsq{}} \PYG{o}{+}
           \PYG{l+s+s1}{\PYGZsq{}}\PYG{l+s+s1}{WHERE parent\PYGZus{}id IN (}\PYG{l+s+s1}{\PYGZsq{}}\PYG{o}{+}\PYG{l+s+s1}{\PYGZsq{}}\PYG{l+s+s1}{,}\PYG{l+s+s1}{\PYGZsq{}}\PYG{o}{.}\PYG{n}{join}\PYG{p}{(}\PYG{n+nb}{map}\PYG{p}{(}\PYG{n+nb}{str}\PYG{p}{,} \PYG{n}{ids}\PYG{p}{)}\PYG{p}{)}\PYG{o}{+}\PYG{l+s+s1}{\PYGZsq{}}\PYG{l+s+s1}{)}\PYG{l+s+s1}{\PYGZsq{}}\PYG{p}{)}

\PYG{c+c1}{\PYGZsh{} better}
\PYG{n+nb+bp}{self}\PYG{o}{.}\PYG{n}{env}\PYG{o}{.}\PYG{n}{cr}\PYG{o}{.}\PYG{n}{execute}\PYG{p}{(}\PYG{l+s+s1}{\PYGZsq{}}\PYG{l+s+s1}{SELECT DISTINCT child\PYGZus{}id }\PYG{l+s+s1}{\PYGZsq{}}\PYGZbs{}
           \PYG{l+s+s1}{\PYGZsq{}}\PYG{l+s+s1}{FROM account\PYGZus{}account\PYGZus{}consol\PYGZus{}rel }\PYG{l+s+s1}{\PYGZsq{}}\PYGZbs{}
           \PYG{l+s+s1}{\PYGZsq{}}\PYG{l+s+s1}{WHERE parent\PYGZus{}id IN }\PYG{l+s+si}{\PYGZpc{}s}\PYG{l+s+s1}{\PYGZsq{}}\PYG{p}{,}
           \PYG{p}{(}\PYG{n+nb}{tuple}\PYG{p}{(}\PYG{n}{ids}\PYG{p}{)}\PYG{p}{,}\PYG{p}{)}\PYG{p}{)}
\end{sphinxVerbatim}

This is very important, so please be careful also when refactoring, and most
importantly do not copy these patterns!

Here is a memorable example to help you remember what the issue is about (but
do not copy the code there). Before continuing, please be sure to read the
online documentation of pyscopg2 to learn of to use it properly:
\begin{itemize}
\item {} 
The problem with query parameters (\sphinxurl{http://initd.org/psycopg/docs/usage.html\#the-problem-with-the-query-parameters})

\item {} 
How to pass parameters with psycopg2 (\sphinxurl{http://initd.org/psycopg/docs/usage.html\#passing-parameters-to-sql-queries})

\item {} 
Advanced parameter types (\sphinxurl{http://initd.org/psycopg/docs/usage.html\#adaptation-of-python-values-to-sql-types})

\end{itemize}


\paragraph{Keep your methods short/simple when possible}
\label{\detokenize{reference/guidelines:keep-your-methods-short-simple-when-possible}}
Functions and methods should not contain too much logic: having a lot of small
and simple methods is more advisable than having few large and complex methods.
A good rule of thumb is to split a method as soon as:
- it has more than one responsibility (see \sphinxurl{http://en.wikipedia.org/wiki/Single\_responsibility\_principle})
- it is too big to fit on one screen.

Also, name your functions accordingly: small and properly named functions are the starting point of readable/maintainable code and tighter documentation.

This recommendation is also relevant for classes, files, modules and packages. (See also \sphinxurl{http://en.wikipedia.org/wiki/Cyclomatic\_complexity})


\paragraph{Never commit the transaction}
\label{\detokenize{reference/guidelines:never-commit-the-transaction}}
The Odoo framework is in charge of providing the transactional context for
all RPC calls. The principle is that a new database cursor is opened at the
beginning of each RPC call, and committed when the call has returned, just
before transmitting the answer to the RPC client, approximately like this:

\fvset{hllines={, ,}}%
\begin{sphinxVerbatim}[commandchars=\\\{\}]
\PYG{k}{def} \PYG{n+nf}{execute}\PYG{p}{(}\PYG{n+nb+bp}{self}\PYG{p}{,} \PYG{n}{db\PYGZus{}name}\PYG{p}{,} \PYG{n}{uid}\PYG{p}{,} \PYG{n}{obj}\PYG{p}{,} \PYG{n}{method}\PYG{p}{,} \PYG{o}{*}\PYG{n}{args}\PYG{p}{,} \PYG{o}{*}\PYG{o}{*}\PYG{n}{kw}\PYG{p}{)}\PYG{p}{:}
    \PYG{n}{db}\PYG{p}{,} \PYG{n}{pool} \PYG{o}{=} \PYG{n}{pooler}\PYG{o}{.}\PYG{n}{get\PYGZus{}db\PYGZus{}and\PYGZus{}pool}\PYG{p}{(}\PYG{n}{db\PYGZus{}name}\PYG{p}{)}
    \PYG{c+c1}{\PYGZsh{} create transaction cursor}
    \PYG{n}{cr} \PYG{o}{=} \PYG{n}{db}\PYG{o}{.}\PYG{n}{cursor}\PYG{p}{(}\PYG{p}{)}
    \PYG{k}{try}\PYG{p}{:}
        \PYG{n}{res} \PYG{o}{=} \PYG{n}{pool}\PYG{o}{.}\PYG{n}{execute\PYGZus{}cr}\PYG{p}{(}\PYG{n}{cr}\PYG{p}{,} \PYG{n}{uid}\PYG{p}{,} \PYG{n}{obj}\PYG{p}{,} \PYG{n}{method}\PYG{p}{,} \PYG{o}{*}\PYG{n}{args}\PYG{p}{,} \PYG{o}{*}\PYG{o}{*}\PYG{n}{kw}\PYG{p}{)}
        \PYG{n}{cr}\PYG{o}{.}\PYG{n}{commit}\PYG{p}{(}\PYG{p}{)} \PYG{c+c1}{\PYGZsh{} all good, we commit}
    \PYG{k}{except} \PYG{n+ne}{Exception}\PYG{p}{:}
        \PYG{n}{cr}\PYG{o}{.}\PYG{n}{rollback}\PYG{p}{(}\PYG{p}{)} \PYG{c+c1}{\PYGZsh{} error, rollback everything atomically}
        \PYG{k}{raise}
    \PYG{k}{finally}\PYG{p}{:}
        \PYG{n}{cr}\PYG{o}{.}\PYG{n}{close}\PYG{p}{(}\PYG{p}{)} \PYG{c+c1}{\PYGZsh{} always close cursor opened manually}
    \PYG{k}{return} \PYG{n}{res}
\end{sphinxVerbatim}

If any error occurs during the execution of the RPC call, the transaction is
rolled back atomically, preserving the state of the system.

Similarly, the system also provides a dedicated transaction during the execution
of tests suites, so it can be rolled back or not depending on the server
startup options.

The consequence is that if you manually call \sphinxcode{\sphinxupquote{cr.commit()}} anywhere there is
a very high chance that you will break the system in various ways, because you
will cause partial commits, and thus partial and unclean rollbacks, causing
among others:
\begin{enumerate}
\item {} 
inconsistent business data, usually data loss

\item {} 
workflow desynchronization, documents stuck permanently

\item {} 
tests that can’t be rolled back cleanly, and will start polluting the
database, and triggering error (this is true even if no error occurs
during the transaction)

\end{enumerate}
\begin{description}
\item[{Here is the very simple rule:}] \leavevmode
You should \sphinxstylestrong{NEVER} call \sphinxcode{\sphinxupquote{cr.commit()}} yourself, \sphinxstylestrong{UNLESS} you have
created your own database cursor explicitly! And the situations where you
need to do that are exceptional!

And by the way if you did create your own cursor, then you need to handle
error cases and proper rollback, as well as properly close the cursor when
you’re done with it.

\end{description}

And contrary to popular belief, you do not even need to call \sphinxcode{\sphinxupquote{cr.commit()}}
in the following situations:
- in the \sphinxcode{\sphinxupquote{\_auto\_init()}} method of an \sphinxstyleemphasis{models.Model} object: this is taken
care of by the addons initialization method, or by the ORM transaction when
creating custom models
- in reports: the \sphinxcode{\sphinxupquote{commit()}} is handled by the framework too, so you can
update the database even from within a report
- within \sphinxstyleemphasis{models.Transient} methods: these methods are called exactly like
regular \sphinxstyleemphasis{models.Model} ones, within a transaction and with the corresponding
\sphinxcode{\sphinxupquote{cr.commit()/rollback()}} at the end
- etc. (see general rule above if you have in doubt!)

All \sphinxcode{\sphinxupquote{cr.commit()}} calls outside of the server framework from now on must
have an \sphinxstylestrong{explicit comment} explaining why they are absolutely necessary, why
they are indeed correct, and why they do not break the transactions. Otherwise
they can and will be removed !


\paragraph{Use translation method correctly}
\label{\detokenize{reference/guidelines:use-translation-method-correctly}}
Odoo uses a GetText-like method named “underscore” \sphinxcode{\sphinxupquote{\_( )}} to indicate that
a static string used in the code needs to be translated at runtime using the
language of the context. This pseudo-method is accessed within your code by
importing as follows:

\fvset{hllines={, ,}}%
\begin{sphinxVerbatim}[commandchars=\\\{\}]
\PYG{k+kn}{from} \PYG{n+nn}{odoo.tools.translate} \PYG{k+kn}{import} \PYG{n}{\PYGZus{}}
\end{sphinxVerbatim}

A few very important rules must be followed when using it, in order for it to
work and to avoid filling the translations with useless junk.

Basically, this method should only be used for static strings written manually
in the code, it will not work to translate field values, such as Product names,
etc. This must be done instead using the translate flag on the corresponding
field.

The rule is very simple: calls to the underscore method should always be in
the form \sphinxcode{\sphinxupquote{\_('literal string')}} and nothing else:

\fvset{hllines={, ,}}%
\begin{sphinxVerbatim}[commandchars=\\\{\}]
\PYG{c+c1}{\PYGZsh{} good: plain strings}
\PYG{n}{error} \PYG{o}{=} \PYG{n}{\PYGZus{}}\PYG{p}{(}\PYG{l+s+s1}{\PYGZsq{}}\PYG{l+s+s1}{This record is locked!}\PYG{l+s+s1}{\PYGZsq{}}\PYG{p}{)}

\PYG{c+c1}{\PYGZsh{} good: strings with formatting patterns included}
\PYG{n}{error} \PYG{o}{=} \PYG{n}{\PYGZus{}}\PYG{p}{(}\PYG{l+s+s1}{\PYGZsq{}}\PYG{l+s+s1}{Record }\PYG{l+s+si}{\PYGZpc{}s}\PYG{l+s+s1}{ cannot be modified!}\PYG{l+s+s1}{\PYGZsq{}}\PYG{p}{)} \PYG{o}{\PYGZpc{}} \PYG{n}{record}

\PYG{c+c1}{\PYGZsh{} ok too: multi\PYGZhy{}line literal strings}
\PYG{n}{error} \PYG{o}{=} \PYG{n}{\PYGZus{}}\PYG{p}{(}\PYG{l+s+s2}{\PYGZdq{}\PYGZdq{}\PYGZdq{}}\PYG{l+s+s2}{This is a bad multiline example}
\PYG{l+s+s2}{             about record }\PYG{l+s+si}{\PYGZpc{}s}\PYG{l+s+s2}{!}\PYG{l+s+s2}{\PYGZdq{}\PYGZdq{}\PYGZdq{}}\PYG{p}{)} \PYG{o}{\PYGZpc{}} \PYG{n}{record}
\PYG{n}{error} \PYG{o}{=} \PYG{n}{\PYGZus{}}\PYG{p}{(}\PYG{l+s+s1}{\PYGZsq{}}\PYG{l+s+s1}{Record }\PYG{l+s+si}{\PYGZpc{}s}\PYG{l+s+s1}{ cannot be modified}\PYG{l+s+s1}{\PYGZsq{}} \PYGZbs{}
          \PYG{l+s+s1}{\PYGZsq{}}\PYG{l+s+s1}{after being validated!}\PYG{l+s+s1}{\PYGZsq{}}\PYG{p}{)} \PYG{o}{\PYGZpc{}} \PYG{n}{record}

\PYG{c+c1}{\PYGZsh{} bad: tries to translate after string formatting}
\PYG{c+c1}{\PYGZsh{}      (pay attention to brackets!)}
\PYG{c+c1}{\PYGZsh{} This does NOT work and messes up the translations!}
\PYG{n}{error} \PYG{o}{=} \PYG{n}{\PYGZus{}}\PYG{p}{(}\PYG{l+s+s1}{\PYGZsq{}}\PYG{l+s+s1}{Record }\PYG{l+s+si}{\PYGZpc{}s}\PYG{l+s+s1}{ cannot be modified!}\PYG{l+s+s1}{\PYGZsq{}} \PYG{o}{\PYGZpc{}} \PYG{n}{record}\PYG{p}{)}

\PYG{c+c1}{\PYGZsh{} bad: dynamic string, string concatenation, etc are forbidden!}
\PYG{c+c1}{\PYGZsh{} This does NOT work and messes up the translations!}
\PYG{n}{error} \PYG{o}{=} \PYG{n}{\PYGZus{}}\PYG{p}{(}\PYG{l+s+s2}{\PYGZdq{}}\PYG{l+s+s2}{\PYGZsq{}}\PYG{l+s+s2}{\PYGZdq{}} \PYG{o}{+} \PYG{n}{que\PYGZus{}rec}\PYG{p}{[}\PYG{l+s+s1}{\PYGZsq{}}\PYG{l+s+s1}{question}\PYG{l+s+s1}{\PYGZsq{}}\PYG{p}{]} \PYG{o}{+} \PYG{l+s+s2}{\PYGZdq{}}\PYG{l+s+s2}{\PYGZsq{}}\PYG{l+s+s2}{ }\PYG{l+s+se}{\PYGZbs{}n}\PYG{l+s+s2}{\PYGZdq{}}\PYG{p}{)}

\PYG{c+c1}{\PYGZsh{} bad: field values are automatically translated by the framework}
\PYG{c+c1}{\PYGZsh{} This is useless and will not work the way you think:}
\PYG{n}{error} \PYG{o}{=} \PYG{n}{\PYGZus{}}\PYG{p}{(}\PYG{l+s+s2}{\PYGZdq{}}\PYG{l+s+s2}{Product }\PYG{l+s+si}{\PYGZpc{}s}\PYG{l+s+s2}{ is out of stock!}\PYG{l+s+s2}{\PYGZdq{}}\PYG{p}{)} \PYG{o}{\PYGZpc{}} \PYG{n}{\PYGZus{}}\PYG{p}{(}\PYG{n}{product}\PYG{o}{.}\PYG{n}{name}\PYG{p}{)}
\PYG{c+c1}{\PYGZsh{} and the following will of course not work as already explained:}
\PYG{n}{error} \PYG{o}{=} \PYG{n}{\PYGZus{}}\PYG{p}{(}\PYG{l+s+s2}{\PYGZdq{}}\PYG{l+s+s2}{Product }\PYG{l+s+si}{\PYGZpc{}s}\PYG{l+s+s2}{ is out of stock!}\PYG{l+s+s2}{\PYGZdq{}} \PYG{o}{\PYGZpc{}} \PYG{n}{product}\PYG{o}{.}\PYG{n}{name}\PYG{p}{)}

\PYG{c+c1}{\PYGZsh{} bad: field values are automatically translated by the framework}
\PYG{c+c1}{\PYGZsh{} This is useless and will not work the way you think:}
\PYG{n}{error} \PYG{o}{=} \PYG{n}{\PYGZus{}}\PYG{p}{(}\PYG{l+s+s2}{\PYGZdq{}}\PYG{l+s+s2}{Product }\PYG{l+s+si}{\PYGZpc{}s}\PYG{l+s+s2}{ is not available!}\PYG{l+s+s2}{\PYGZdq{}}\PYG{p}{)} \PYG{o}{\PYGZpc{}} \PYG{n}{\PYGZus{}}\PYG{p}{(}\PYG{n}{product}\PYG{o}{.}\PYG{n}{name}\PYG{p}{)}
\PYG{c+c1}{\PYGZsh{} and the following will of course not work as already explained:}
\PYG{n}{error} \PYG{o}{=} \PYG{n}{\PYGZus{}}\PYG{p}{(}\PYG{l+s+s2}{\PYGZdq{}}\PYG{l+s+s2}{Product }\PYG{l+s+si}{\PYGZpc{}s}\PYG{l+s+s2}{ is not available!}\PYG{l+s+s2}{\PYGZdq{}} \PYG{o}{\PYGZpc{}} \PYG{n}{product}\PYG{o}{.}\PYG{n}{name}\PYG{p}{)}

\PYG{c+c1}{\PYGZsh{} Instead you can do the following and everything will be translated,}
\PYG{c+c1}{\PYGZsh{} including the product name if its field definition has the}
\PYG{c+c1}{\PYGZsh{} translate flag properly set:}
\PYG{n}{error} \PYG{o}{=} \PYG{n}{\PYGZus{}}\PYG{p}{(}\PYG{l+s+s2}{\PYGZdq{}}\PYG{l+s+s2}{Product }\PYG{l+s+si}{\PYGZpc{}s}\PYG{l+s+s2}{ is not available!}\PYG{l+s+s2}{\PYGZdq{}}\PYG{p}{)} \PYG{o}{\PYGZpc{}} \PYG{n}{product}\PYG{o}{.}\PYG{n}{name}
\end{sphinxVerbatim}

Also, keep in mind that translators will have to work with the literal values
that are passed to the underscore function, so please try to make them easy to
understand and keep spurious characters and formatting to a minimum. Translators
must be aware that formatting patterns such as \%s or \%d, newlines, etc. need
to be preserved, but it’s important to use these in a sensible and obvious manner:

\fvset{hllines={, ,}}%
\begin{sphinxVerbatim}[commandchars=\\\{\}]
\PYG{c+c1}{\PYGZsh{} Bad: makes the translations hard to work with}
\PYG{n}{error} \PYG{o}{=} \PYG{l+s+s2}{\PYGZdq{}}\PYG{l+s+s2}{\PYGZsq{}}\PYG{l+s+s2}{\PYGZdq{}} \PYG{o}{+} \PYG{n}{question} \PYG{o}{+} \PYG{n}{\PYGZus{}}\PYG{p}{(}\PYG{l+s+s2}{\PYGZdq{}}\PYG{l+s+s2}{\PYGZsq{}}\PYG{l+s+s2}{ }\PYG{l+s+se}{\PYGZbs{}n}\PYG{l+s+s2}{Please enter an integer value }\PYG{l+s+s2}{\PYGZdq{}}\PYG{p}{)}

\PYG{c+c1}{\PYGZsh{} Better (pay attention to position of the brackets too!)}
\PYG{n}{error} \PYG{o}{=} \PYG{n}{\PYGZus{}}\PYG{p}{(}\PYG{l+s+s2}{\PYGZdq{}}\PYG{l+s+s2}{Answer to question }\PYG{l+s+si}{\PYGZpc{}s}\PYG{l+s+s2}{ is not valid.}\PYG{l+s+se}{\PYGZbs{}n}\PYG{l+s+s2}{\PYGZdq{}} \PYGZbs{}
          \PYG{l+s+s2}{\PYGZdq{}}\PYG{l+s+s2}{Please enter an integer value.}\PYG{l+s+s2}{\PYGZdq{}}\PYG{p}{)} \PYG{o}{\PYGZpc{}} \PYG{n}{question}
\end{sphinxVerbatim}

In general in Odoo, when manipulating strings, prefer \sphinxcode{\sphinxupquote{\%}} over \sphinxcode{\sphinxupquote{.format()}}
(when only one variable to replace in a string), and prefer \sphinxcode{\sphinxupquote{\%(varname)}} instead
of position (when multiple variables have to be replaced). This makes the
translation easier for the community translators.


\subsubsection{Symbols and Conventions}
\label{\detokenize{reference/guidelines:symbols-and-conventions}}\begin{itemize}
\item {} \begin{description}
\item[{Model name (using the dot notation, prefix by the module name) :}] \leavevmode\begin{itemize}
\item {} 
When defining an Odoo Model : use singular form of the name (\sphinxstyleemphasis{res.partner}
and \sphinxstyleemphasis{sale.order} instead of \sphinxstyleemphasis{res.partnerS} and \sphinxstyleemphasis{saleS.orderS})

\item {} 
When defining an Odoo Transient (wizard) : use \sphinxcode{\sphinxupquote{\textless{}related\_base\_model\textgreater{}.\textless{}action\textgreater{}}}
where \sphinxstyleemphasis{related\_base\_model} is the base model (defined in \sphinxstyleemphasis{models/}) related
to the transient, and \sphinxstyleemphasis{action} is the short name of what the transient do.
For instance : \sphinxcode{\sphinxupquote{account.invoice.make}}, \sphinxcode{\sphinxupquote{project.task.delegate.batch}}, …

\item {} 
When defining \sphinxstyleemphasis{report} model (SQL views e.i.) : use
\sphinxcode{\sphinxupquote{\textless{}related\_base\_model\textgreater{}.report.\textless{}action\textgreater{}}}, based on the Transient convention.

\end{itemize}

\end{description}

\item {} 
Odoo Python Class : use camelcase for code.

\end{itemize}

\fvset{hllines={, ,}}%
\begin{sphinxVerbatim}[commandchars=\\\{\}]
\PYG{k}{class} \PYG{n+nc}{AccountInvoice}\PYG{p}{(}\PYG{n}{models}\PYG{o}{.}\PYG{n}{Model}\PYG{p}{)}\PYG{p}{:}
    \PYG{o}{.}\PYG{o}{.}\PYG{o}{.}
\end{sphinxVerbatim}
\begin{itemize}
\item {} \begin{description}
\item[{Variable name :}] \leavevmode\begin{itemize}
\item {} 
use camelcase for model variable

\item {} 
use underscore lowercase notation for common variable.

\item {} 
Odoo works with a record or a recordset, don’t suffix variable names with
\sphinxstyleemphasis{\_id} or \sphinxstyleemphasis{\_ids} if they don’t contain an id or a list of ids.

\end{itemize}

\end{description}

\end{itemize}

\fvset{hllines={, ,}}%
\begin{sphinxVerbatim}[commandchars=\\\{\}]
\PYG{n}{ResPartner} \PYG{o}{=} \PYG{n+nb+bp}{self}\PYG{o}{.}\PYG{n}{env}\PYG{p}{[}\PYG{l+s+s1}{\PYGZsq{}}\PYG{l+s+s1}{res.partner}\PYG{l+s+s1}{\PYGZsq{}}\PYG{p}{]}
\PYG{n}{partners} \PYG{o}{=} \PYG{n}{ResPartner}\PYG{o}{.}\PYG{n}{browse}\PYG{p}{(}\PYG{n}{ids}\PYG{p}{)}
\PYG{n}{partner\PYGZus{}id} \PYG{o}{=} \PYG{n}{partners}\PYG{p}{[}\PYG{l+m+mi}{0}\PYG{p}{]}\PYG{o}{.}\PYG{n}{id}
\end{sphinxVerbatim}
\begin{itemize}
\item {} 
\sphinxcode{\sphinxupquote{One2Many}} and \sphinxcode{\sphinxupquote{Many2Many}} fields should always have \sphinxstyleemphasis{\_ids} as suffix (example: sale\_order\_line\_ids)

\item {} 
\sphinxcode{\sphinxupquote{Many2One}} fields should have \sphinxstyleemphasis{\_id} as suffix (example : partner\_id, user\_id, …)

\item {} \begin{description}
\item[{Method conventions}] \leavevmode\begin{itemize}
\item {} 
Compute Field : the compute method pattern is \sphinxstyleemphasis{\_compute\_\textless{}field\_name\textgreater{}}

\item {} 
Search method : the search method pattern is \sphinxstyleemphasis{\_search\_\textless{}field\_name\textgreater{}}

\item {} 
Default method : the default method pattern is \sphinxstyleemphasis{\_default\_\textless{}field\_name\textgreater{}}

\item {} 
Onchange method : the onchange method pattern is \sphinxstyleemphasis{\_onchange\_\textless{}field\_name\textgreater{}}

\item {} 
Constraint method : the constraint method pattern is \sphinxstyleemphasis{\_check\_\textless{}constraint\_name\textgreater{}}

\item {} 
Action method : an object action method is prefix with \sphinxstyleemphasis{action\_}. Its decorator is
\sphinxcode{\sphinxupquote{@api.multi}}, but since it use only one record, add \sphinxcode{\sphinxupquote{self.ensure\_one()}}
at the beginning of the method.

\end{itemize}

\end{description}

\item {} \begin{description}
\item[{In a Model attribute order should be}] \leavevmode\begin{enumerate}
\item {} 
Private attributes (\sphinxcode{\sphinxupquote{\_name}}, \sphinxcode{\sphinxupquote{\_description}}, \sphinxcode{\sphinxupquote{\_inherit}}, …)

\item {} 
Default method and \sphinxcode{\sphinxupquote{\_default\_get}}

\item {} 
Field declarations

\item {} 
Compute and search methods in the same order as field declaration

\item {} 
Constrains methods (\sphinxcode{\sphinxupquote{@api.constrains}}) and onchange methods (\sphinxcode{\sphinxupquote{@api.onchange}})

\item {} 
CRUD methods (ORM overrides)

\item {} 
Action methods

\item {} 
And finally, other business methods.

\end{enumerate}

\end{description}

\end{itemize}

\fvset{hllines={, ,}}%
\begin{sphinxVerbatim}[commandchars=\\\{\}]
\PYG{k}{class} \PYG{n+nc}{Event}\PYG{p}{(}\PYG{n}{models}\PYG{o}{.}\PYG{n}{Model}\PYG{p}{)}\PYG{p}{:}
    \PYG{c+c1}{\PYGZsh{} Private attributes}
    \PYG{n}{\PYGZus{}name} \PYG{o}{=} \PYG{l+s+s1}{\PYGZsq{}}\PYG{l+s+s1}{event.event}\PYG{l+s+s1}{\PYGZsq{}}
    \PYG{n}{\PYGZus{}description} \PYG{o}{=} \PYG{l+s+s1}{\PYGZsq{}}\PYG{l+s+s1}{Event}\PYG{l+s+s1}{\PYGZsq{}}

    \PYG{c+c1}{\PYGZsh{} Default methods}
    \PYG{k}{def} \PYG{n+nf}{\PYGZus{}default\PYGZus{}name}\PYG{p}{(}\PYG{n+nb+bp}{self}\PYG{p}{)}\PYG{p}{:}
        \PYG{o}{.}\PYG{o}{.}\PYG{o}{.}

    \PYG{c+c1}{\PYGZsh{} Fields declaration}
    \PYG{n}{name} \PYG{o}{=} \PYG{n}{fields}\PYG{o}{.}\PYG{n}{Char}\PYG{p}{(}\PYG{n}{string}\PYG{o}{=}\PYG{l+s+s1}{\PYGZsq{}}\PYG{l+s+s1}{Name}\PYG{l+s+s1}{\PYGZsq{}}\PYG{p}{,} \PYG{n}{default}\PYG{o}{=}\PYG{n}{\PYGZus{}default\PYGZus{}name}\PYG{p}{)}
    \PYG{n}{seats\PYGZus{}reserved} \PYG{o}{=} \PYG{n}{fields}\PYG{o}{.}\PYG{n}{Integer}\PYG{p}{(}\PYG{n}{oldname}\PYG{o}{=}\PYG{l+s+s1}{\PYGZsq{}}\PYG{l+s+s1}{register\PYGZus{}current}\PYG{l+s+s1}{\PYGZsq{}}\PYG{p}{,} \PYG{n}{string}\PYG{o}{=}\PYG{l+s+s1}{\PYGZsq{}}\PYG{l+s+s1}{Reserved Seats}\PYG{l+s+s1}{\PYGZsq{}}\PYG{p}{,}
        \PYG{n}{store}\PYG{o}{=}\PYG{n+nb+bp}{True}\PYG{p}{,} \PYG{n}{readonly}\PYG{o}{=}\PYG{n+nb+bp}{True}\PYG{p}{,} \PYG{n}{compute}\PYG{o}{=}\PYG{l+s+s1}{\PYGZsq{}}\PYG{l+s+s1}{\PYGZus{}compute\PYGZus{}seats}\PYG{l+s+s1}{\PYGZsq{}}\PYG{p}{)}
    \PYG{n}{seats\PYGZus{}available} \PYG{o}{=} \PYG{n}{fields}\PYG{o}{.}\PYG{n}{Integer}\PYG{p}{(}\PYG{n}{oldname}\PYG{o}{=}\PYG{l+s+s1}{\PYGZsq{}}\PYG{l+s+s1}{register\PYGZus{}avail}\PYG{l+s+s1}{\PYGZsq{}}\PYG{p}{,} \PYG{n}{string}\PYG{o}{=}\PYG{l+s+s1}{\PYGZsq{}}\PYG{l+s+s1}{Available Seats}\PYG{l+s+s1}{\PYGZsq{}}\PYG{p}{,}
        \PYG{n}{store}\PYG{o}{=}\PYG{n+nb+bp}{True}\PYG{p}{,} \PYG{n}{readonly}\PYG{o}{=}\PYG{n+nb+bp}{True}\PYG{p}{,} \PYG{n}{compute}\PYG{o}{=}\PYG{l+s+s1}{\PYGZsq{}}\PYG{l+s+s1}{\PYGZus{}compute\PYGZus{}seats}\PYG{l+s+s1}{\PYGZsq{}}\PYG{p}{)}
    \PYG{n}{price} \PYG{o}{=} \PYG{n}{fields}\PYG{o}{.}\PYG{n}{Integer}\PYG{p}{(}\PYG{n}{string}\PYG{o}{=}\PYG{l+s+s1}{\PYGZsq{}}\PYG{l+s+s1}{Price}\PYG{l+s+s1}{\PYGZsq{}}\PYG{p}{)}

    \PYG{c+c1}{\PYGZsh{} compute and search fields, in the same order of fields declaration}
    \PYG{n+nd}{@api.multi}
    \PYG{n+nd}{@api.depends}\PYG{p}{(}\PYG{l+s+s1}{\PYGZsq{}}\PYG{l+s+s1}{seats\PYGZus{}max}\PYG{l+s+s1}{\PYGZsq{}}\PYG{p}{,} \PYG{l+s+s1}{\PYGZsq{}}\PYG{l+s+s1}{registration\PYGZus{}ids.state}\PYG{l+s+s1}{\PYGZsq{}}\PYG{p}{,} \PYG{l+s+s1}{\PYGZsq{}}\PYG{l+s+s1}{registration\PYGZus{}ids.nb\PYGZus{}register}\PYG{l+s+s1}{\PYGZsq{}}\PYG{p}{)}
    \PYG{k}{def} \PYG{n+nf}{\PYGZus{}compute\PYGZus{}seats}\PYG{p}{(}\PYG{n+nb+bp}{self}\PYG{p}{)}\PYG{p}{:}
        \PYG{o}{.}\PYG{o}{.}\PYG{o}{.}

    \PYG{c+c1}{\PYGZsh{} Constraints and onchanges}
    \PYG{n+nd}{@api.constrains}\PYG{p}{(}\PYG{l+s+s1}{\PYGZsq{}}\PYG{l+s+s1}{seats\PYGZus{}max}\PYG{l+s+s1}{\PYGZsq{}}\PYG{p}{,} \PYG{l+s+s1}{\PYGZsq{}}\PYG{l+s+s1}{seats\PYGZus{}available}\PYG{l+s+s1}{\PYGZsq{}}\PYG{p}{)}
    \PYG{k}{def} \PYG{n+nf}{\PYGZus{}check\PYGZus{}seats\PYGZus{}limit}\PYG{p}{(}\PYG{n+nb+bp}{self}\PYG{p}{)}\PYG{p}{:}
        \PYG{o}{.}\PYG{o}{.}\PYG{o}{.}

    \PYG{n+nd}{@api.onchange}\PYG{p}{(}\PYG{l+s+s1}{\PYGZsq{}}\PYG{l+s+s1}{date\PYGZus{}begin}\PYG{l+s+s1}{\PYGZsq{}}\PYG{p}{)}
    \PYG{k}{def} \PYG{n+nf}{\PYGZus{}onchange\PYGZus{}date\PYGZus{}begin}\PYG{p}{(}\PYG{n+nb+bp}{self}\PYG{p}{)}\PYG{p}{:}
        \PYG{o}{.}\PYG{o}{.}\PYG{o}{.}

    \PYG{c+c1}{\PYGZsh{} CRUD methods (and name\PYGZus{}get, name\PYGZus{}search, ...) overrides}
    \PYG{k}{def} \PYG{n+nf}{create}\PYG{p}{(}\PYG{n+nb+bp}{self}\PYG{p}{,} \PYG{n}{values}\PYG{p}{)}\PYG{p}{:}
        \PYG{o}{.}\PYG{o}{.}\PYG{o}{.}

    \PYG{c+c1}{\PYGZsh{} Action methods}
    \PYG{n+nd}{@api.multi}
    \PYG{k}{def} \PYG{n+nf}{action\PYGZus{}validate}\PYG{p}{(}\PYG{n+nb+bp}{self}\PYG{p}{)}\PYG{p}{:}
        \PYG{n+nb+bp}{self}\PYG{o}{.}\PYG{n}{ensure\PYGZus{}one}\PYG{p}{(}\PYG{p}{)}
        \PYG{o}{.}\PYG{o}{.}\PYG{o}{.}

    \PYG{c+c1}{\PYGZsh{} Business methods}
    \PYG{k}{def} \PYG{n+nf}{mail\PYGZus{}user\PYGZus{}confirm}\PYG{p}{(}\PYG{n+nb+bp}{self}\PYG{p}{)}\PYG{p}{:}
        \PYG{o}{.}\PYG{o}{.}\PYG{o}{.}
\end{sphinxVerbatim}


\subsection{Javascript and CSS}
\label{\detokenize{reference/guidelines:javascript-and-css}}

\subsubsection{Static files organization}
\label{\detokenize{reference/guidelines:static-files-organization}}
Odoo addons have some conventions on how to structure various files. We explain
here in more details how web assets are supposed to be organized.

The first thing to know is that the Odoo server will serve (statically) all files
located in a \sphinxstyleemphasis{static/} folder, but prefixed with the addon name. So, for example,
if a file is located in \sphinxstyleemphasis{addons/web/static/src/js/some\_file.js}, then it will be
statically available at the url \sphinxstyleemphasis{your-odoo-server.com/web/static/src/js/some\_file.js}

The convention is to organize the code according to the following structure:
\begin{itemize}
\item {} 
\sphinxstyleemphasis{static}: all static files in general

\item {} 
\sphinxstyleemphasis{static/lib}: this is the place where js libs should be located, in a sub folder.
So, for example, all files from the \sphinxstyleemphasis{jquery} library are in \sphinxstyleemphasis{addons/web/static/lib/jquery}

\item {} 
\sphinxstyleemphasis{static/src}: the generic static source code folder

\item {} 
\sphinxstyleemphasis{static/src/css}: all css files

\item {} 
\sphinxstyleemphasis{static/src/fonts}

\item {} 
\sphinxstyleemphasis{static/src/img}

\item {} 
\sphinxstyleemphasis{static/src/js}

\item {} 
\sphinxstyleemphasis{static/src/less}: less files

\item {} 
\sphinxstyleemphasis{static/src/xml}: all qweb templates that will be rendered in JS

\item {} 
\sphinxstyleemphasis{static/tests}: this is where we put all test related files.

\end{itemize}


\subsubsection{Javascript coding guidelines}
\label{\detokenize{reference/guidelines:javascript-coding-guidelines}}\begin{itemize}
\item {} 
\sphinxcode{\sphinxupquote{use strict;}} is recommended for all javascript files

\item {} 
Use a linter (jshint, …)

\item {} 
Never add minified Javascript Libraries

\item {} 
Use camelcase for class declaration

\item {} 
Unless your code is supposed to run on every page, target specific pages
using the \sphinxcode{\sphinxupquote{if\_dom\_contains}} function of website module. Target an
element which is specific to the pages your code needs to run on
using JQuery.

\end{itemize}

\fvset{hllines={, ,}}%
\begin{sphinxVerbatim}[commandchars=\\\{\}]
\PYG{n+nx}{odoo}\PYG{p}{.}\PYG{n+nx}{website}\PYG{p}{.}\PYG{n+nx}{if\PYGZus{}dom\PYGZus{}contains}\PYG{p}{(}\PYG{l+s+s1}{\PYGZsq{}.jquery\PYGZus{}class\PYGZus{}selector\PYGZsq{}}\PYG{p}{,} \PYG{k+kd}{function} \PYG{p}{(}\PYG{p}{)} \PYG{p}{\PYGZob{}}
    \PYG{c+cm}{/*your code here*/}
\PYG{p}{\PYGZcb{}}\PYG{p}{)}\PYG{p}{;}
\end{sphinxVerbatim}


\subsubsection{CSS coding guidelines}
\label{\detokenize{reference/guidelines:css-coding-guidelines}}\begin{itemize}
\item {} 
Prefix all your classes with \sphinxstyleemphasis{o\_\textless{}module\_name\textgreater{}} where \sphinxstyleemphasis{module\_name} is the
technical name of the module (‘sale’, ‘im\_chat’, …) or the main route
reserved by the module (for website module mainly, i.e. : ‘o\_forum’ for
\sphinxstyleemphasis{website\_forum} module). The only exception for this rule is the
webclient: it simply uses \sphinxstyleemphasis{o\_} prefix.

\item {} 
Avoid using id

\item {} 
Use Bootstrap native classes

\item {} 
Use underscore lowercase notation to name class

\end{itemize}


\subsection{Git}
\label{\detokenize{reference/guidelines:git}}

\subsubsection{Configure your git}
\label{\detokenize{reference/guidelines:configure-your-git}}
Based on ancestral experience and oral tradition, the following things go a long
way towards making your commits more helpful:
\begin{itemize}
\item {} 
Be sure to define both the user.email and user.name in your local git config

\fvset{hllines={, ,}}%
\begin{sphinxVerbatim}[commandchars=\\\{\}]
git config \PYGZhy{}\PYGZhy{}global \PYGZlt{}var\PYGZgt{} \PYGZlt{}value\PYGZgt{}
\end{sphinxVerbatim}

\item {} 
Be sure to add your full name to your Github profile here. Please feel fancy
and add your team, avatar, your favorite quote, and whatnot ;-)

\end{itemize}


\subsubsection{Commit message structure}
\label{\detokenize{reference/guidelines:commit-message-structure}}
Commit message has four parts: tag, module, short description and full
description. Try to follow the preferred structure for your commit messages

\fvset{hllines={, ,}}%
\begin{sphinxVerbatim}[commandchars=\\\{\}]
[TAG] module: describe your change in a short sentence (ideally \PYGZlt{} 50 chars)

Long version of the change description, including the rationale for the change,
or a summary of the feature being introduced.

Please spend a lot more time describing WHY the change is being done rather
than WHAT is being changed. This is usually easy to grasp by actually reading
the diff. WHAT should be explained only if there are technical choices
or decision involved. In that case explain WHY this decision was taken.

End the message with references, such as task or bug numbers, PR numbers, and
OPW tickets, following the suggested format:
Related to task \PYGZsh{}taskId
Fixes \PYGZsh{}12345  (link and close issue on Github)
Closes \PYGZsh{}7865  (link and close PR on Github)
OPW\PYGZhy{}112233
\end{sphinxVerbatim}


\subsubsection{Tag and module name}
\label{\detokenize{reference/guidelines:tag-and-module-name}}
Tags are used to prefix your commit. They should be one of the following
\begin{itemize}
\item {} 
\sphinxstylestrong{{[}FIX{]}} for bug fixes: mostly used in stable version but also valid if you
are fixing a recent bug in development version;

\item {} 
\sphinxstylestrong{{[}REF{]}} for refactoring: when a feature is heavily rewritten;

\item {} 
\sphinxstylestrong{{[}ADD{]}} for adding new modules;

\item {} 
\sphinxstylestrong{{[}REM{]}} for removing resources: removing dead code, removing views,
removing modules, …;

\item {} 
\sphinxstylestrong{{[}REV{]}} for reverting commits: if a commit causes issues or is not wanted
reverting it is done using this tag;

\item {} 
\sphinxstylestrong{{[}MOV{]}} for moving files: use git move and do not change content of moved file
otherwise Git may loose track and history of the file; also used when moving
code from one file to another;

\item {} 
\sphinxstylestrong{{[}REL{]}} for release commits: new major or minor stable versions;

\item {} 
\sphinxstylestrong{{[}IMP{]}} for improvements: most of the changes done in development version
are incremental improvements not related to another tag;

\item {} 
\sphinxstylestrong{{[}MERGE{]}} for merge commits: used in forward port of bug fixes but also as
main commit for feature involving several separated commits;

\item {} 
\sphinxstylestrong{{[}CLA{]}} for signing the Odoo Individual Contributor License;

\item {} 
\sphinxstylestrong{{[}I18N{]}} for changes in translation files;

\end{itemize}

After tag comes the modified module name. Use the technical name as functional
name may change with time. If several modules are modified, list them or use
various to tell it is cross-modules. Unless really required or easier avoid
modifying code across several modules in the same commit. Understanding module
history may become difficult.


\subsubsection{Commit message header}
\label{\detokenize{reference/guidelines:commit-message-header}}
After tag and module name comes a meaningful commit message header. It should be
self explanatory and include the reason behind the change. Do not use single words
like “bugfix” or “improvements”. Try to limit the header length to about 50 characters
for readability.

Commit message header should make a valid sentence once concatenated with
\sphinxcode{\sphinxupquote{if applied, this commit will \textless{}header\textgreater{}}}. For example \sphinxcode{\sphinxupquote{{[}IMP{]} base: prevent to
archive users linked to active partners}} is correct as it makes a valid sentence
\sphinxcode{\sphinxupquote{if applied, this commit will prevent users to archive...}}.


\subsubsection{Commit message full description}
\label{\detokenize{reference/guidelines:commit-message-full-description}}
In the message description specify the part of the code impacted by your changes
(module name, lib, transversal object, …) and a description of the changes.

First explain WHY you are modifying code. What is important if someone goes back
to your commit in about 4 decades (or 3 days) is why you did it. It is the
purpose of the change.

What you did can be found in the commit itself. If there was some technical choices
involved it is a good idea to explain it also in the commit message after the why.
For Odoo R\&D developers “PO team asked me to do it” is not a valid why, by the way.

Please avoid commits which simultaneously impact multiple modules. Try to split
into different commits where impacted modules are different. It will be helpful
if we need to revert changes in a given module separately.

Don’t hesitate to be a bit verbose. Most people will only see your commit message
and judge everything you did in your life just based on those few sentences.
No pressure at all.

\sphinxstylestrong{You spend several hours, days or weeks working on meaningful features. Take
some time to calm down and write clear and understandable commit messages.}

If you are an Odoo R\&D developer the WHY should be the purpose of the task you
are working on. Full specifications make the core of the commit message.
\sphinxstylestrong{If you are working on a task that lacks purpose and specifications please
consider making them clear before continuing.}

Finally here are some examples of correct commit messages :

\fvset{hllines={, ,}}%
\begin{sphinxVerbatim}[commandchars=\\\{\}]
[REF] models: use {}`parent\PYGZus{}path{}` to implement parent\PYGZus{}store

 This replaces the former modified preorder tree traversal (MPTT) with the
 fields {}`parent\PYGZus{}left{}`/{}`parent\PYGZus{}right{}`[...]

[FIX] account: remove frenglish

 [...]

 Closes \PYGZsh{}22793
 Fixes \PYGZsh{}22769

[FIX] website: remove unused alert div, fixes look of input\PYGZhy{}group\PYGZhy{}btn

 Bootstrap\PYGZsq{}s CSS depends on the input\PYGZhy{}group\PYGZhy{}btn
 element being the first/last child of its parent.
 This was not the case because of the invisible
 and useless alert.
\end{sphinxVerbatim}

\begin{sphinxadmonition}{note}{Note:}
Use the long description to explain the \sphinxstyleemphasis{why} not the
\sphinxstyleemphasis{what}, the \sphinxstyleemphasis{what} can be seen in the diff
\end{sphinxadmonition}


\section{Mobile JavaScript}
\label{\detokenize{reference/mobile::doc}}\label{\detokenize{reference/mobile:reference-mobile}}\label{\detokenize{reference/mobile:mobile-javascript}}

\subsection{Introduction}
\label{\detokenize{reference/mobile:introduction}}
In Odoo 10.0 we released a mobile app which allows you to access all \sphinxstylestrong{Odoo apps}
(even your customized modules).

The application is a combination of \sphinxstylestrong{Odoo Web} and \sphinxstylestrong{Native Mobile
components}. In other words it is a Odoo Web instance loaded inside a native, mobile, WebView container.

This page documents how you can access mobile native components like Camera,
Vibration, Notification and Toast through Odoo Web (via JavaScript). For this, you
do not need to be a mobile developer, if you know Odoo JavaScript API you can
access all available mobile features.

\begin{sphinxadmonition}{warning}{Warning:}
These features work with \sphinxstylestrong{Odoo Enterprise 10.0+} only
\end{sphinxadmonition}


\subsection{How does it work?}
\label{\detokenize{reference/mobile:how-does-it-work}}
Internal workings of the mobile application:

\noindent\sphinxincludegraphics{{mobile_working}.jpg}

Of course, it is a web page that loads on a Mobile Native Web container. But it
is integrated in such a way that you can access native resources from your web
JavaScript.

WebPages (Odoo Web) is on the top of each layer, where the second layer is a Bridge
between Odoo Web (JS) and the native mobile components.

When any call from JavaScript is triggered it passes through Bridge and Bridge
passes it to the native invoker to perform that action.

When the native component has done its work, it is passed to the Bridge again and
you get the output in JavaScript.

Process time taken by the Native component depends on what you are requesting
from the Native resources. For example the Camera or GPS Location.


\subsection{How to use it?}
\label{\detokenize{reference/mobile:how-to-use-it}}
Just like the Odoo Web Framework, the Mobile API can be used anywhere by getting the object from
\sphinxstylestrong{web\_mobile.rpc}

\noindent\sphinxincludegraphics{{odoo_mobile_api}.png}

The mobile RPC object provides a list of methods that are available (this only works with the mobile
app).

Check if the method is available and then execute it.


\subsubsection{Methods}
\label{\detokenize{reference/mobile:methods}}
\begin{sphinxadmonition}{note}{Note:}
Each of the methods returns a JQuery Deffered object which returns
a data JSON dictionary
\end{sphinxadmonition}


\paragraph{Show Toast in device}
\label{\detokenize{reference/mobile:show-toast-in-device}}\index{showToast() (built-in function)}

\begin{fulllineitems}
\phantomsection\label{\detokenize{reference/mobile:showToast}}\pysiglinewithargsret{\sphinxbfcode{\sphinxupquote{showToast}}}{}{}~\begin{quote}\begin{description}
\item[{Arguments}] \leavevmode\begin{itemize}
\item {} 
\sphinxstyleliteralstrong{\sphinxupquote{args}} (\sphinxstyleliteralemphasis{\sphinxupquote{object}}) \textendash{} \sphinxstylestrong{message} text to display

\end{itemize}

\end{description}\end{quote}

\end{fulllineitems}


A toast provides simple feedback about an operation in a small popup. It only
fills the amount of space required for the message and the current activity
remains visible and interactive.

\fvset{hllines={, ,}}%
\begin{sphinxVerbatim}[commandchars=\\\{\}]
\PYG{n+nx}{mobile}\PYG{p}{.}\PYG{n+nx}{methods}\PYG{p}{.}\PYG{n+nx}{showToast}\PYG{p}{(}\PYG{p}{\PYGZob{}}\PYG{l+s+s1}{\PYGZsq{}message\PYGZsq{}}\PYG{o}{:} \PYG{l+s+s1}{\PYGZsq{}Message sent\PYGZsq{}}\PYG{p}{\PYGZcb{}}\PYG{p}{)}\PYG{p}{;}
\end{sphinxVerbatim}

\noindent\sphinxincludegraphics{{toast}.png}


\paragraph{Vibrating device}
\label{\detokenize{reference/mobile:vibrating-device}}\index{vibrate() (built-in function)}

\begin{fulllineitems}
\phantomsection\label{\detokenize{reference/mobile:vibrate}}\pysiglinewithargsret{\sphinxbfcode{\sphinxupquote{vibrate}}}{}{}~\begin{quote}\begin{description}
\item[{Arguments}] \leavevmode\begin{itemize}
\item {} 
\sphinxstyleliteralstrong{\sphinxupquote{args}} (\sphinxstyleliteralemphasis{\sphinxupquote{object}}) \textendash{} Vibrates constantly for the specified period of time
(in milliseconds).

\end{itemize}

\end{description}\end{quote}

\end{fulllineitems}


Vibrate mobile device with given duration.

\fvset{hllines={, ,}}%
\begin{sphinxVerbatim}[commandchars=\\\{\}]
\PYG{n+nx}{mobile}\PYG{p}{.}\PYG{n+nx}{methods}\PYG{p}{.}\PYG{n+nx}{vibrate}\PYG{p}{(}\PYG{p}{\PYGZob{}}\PYG{l+s+s1}{\PYGZsq{}duration\PYGZsq{}}\PYG{o}{:} \PYG{l+m+mi}{100}\PYG{p}{\PYGZcb{}}\PYG{p}{)}\PYG{p}{;}
\end{sphinxVerbatim}


\paragraph{Show snackbar with action}
\label{\detokenize{reference/mobile:show-snackbar-with-action}}\index{showSnackBar() (built-in function)}

\begin{fulllineitems}
\phantomsection\label{\detokenize{reference/mobile:showSnackBar}}\pysiglinewithargsret{\sphinxbfcode{\sphinxupquote{showSnackBar}}}{}{}~\begin{quote}\begin{description}
\item[{Arguments}] \leavevmode\begin{itemize}
\item {} 
\sphinxstyleliteralstrong{\sphinxupquote{args}} (\sphinxstyleliteralemphasis{\sphinxupquote{object}}) \textendash{} (\sphinxstyleemphasis{required}) \sphinxstylestrong{Message} to show in snackbar and action \sphinxstylestrong{button label} in Snackbar (optional)

\end{itemize}

\item[{Returns}] \leavevmode
\sphinxcode{\sphinxupquote{True}} if the user clicks on the Action button, \sphinxcode{\sphinxupquote{False}} if SnackBar auto dismissed after some time.

\end{description}\end{quote}

\end{fulllineitems}


Snackbars provide lightweight feedback about an operation. They show a brief
message at the bottom of the screen on mobile or in the lower left corner on larger devices.
Snackbars appear above all the other elements on the screen and only one can be
displayed at a time.

\fvset{hllines={, ,}}%
\begin{sphinxVerbatim}[commandchars=\\\{\}]
\PYG{n+nx}{mobile}\PYG{p}{.}\PYG{n+nx}{methods}\PYG{p}{.}\PYG{n+nx}{showSnackBar}\PYG{p}{(}\PYG{p}{\PYGZob{}}\PYG{l+s+s1}{\PYGZsq{}message\PYGZsq{}}\PYG{o}{:} \PYG{l+s+s1}{\PYGZsq{}Message is deleted\PYGZsq{}}\PYG{p}{,} \PYG{l+s+s1}{\PYGZsq{}btn\PYGZus{}text\PYGZsq{}}\PYG{o}{:} \PYG{l+s+s1}{\PYGZsq{}Undo\PYGZsq{}}\PYG{p}{\PYGZcb{}}\PYG{p}{)}\PYG{p}{.}\PYG{n+nx}{then}\PYG{p}{(}\PYG{k+kd}{function}\PYG{p}{(}\PYG{n+nx}{result}\PYG{p}{)}\PYG{p}{\PYGZob{}}
        \PYG{k}{if}\PYG{p}{(}\PYG{n+nx}{result}\PYG{p}{)}\PYG{p}{\PYGZob{}}
                \PYG{c+c1}{// Do undo operation}
        \PYG{p}{\PYGZcb{}}\PYG{k}{else}\PYG{p}{\PYGZob{}}
                \PYG{c+c1}{// Snack Bar dismissed}
        \PYG{p}{\PYGZcb{}}
\PYG{p}{\PYGZcb{}}\PYG{p}{)}\PYG{p}{;}
\end{sphinxVerbatim}

\noindent\sphinxincludegraphics{{snackbar}.png}


\paragraph{Showing notification}
\label{\detokenize{reference/mobile:showing-notification}}\index{showNotification() (built-in function)}

\begin{fulllineitems}
\phantomsection\label{\detokenize{reference/mobile:showNotification}}\pysiglinewithargsret{\sphinxbfcode{\sphinxupquote{showNotification}}}{}{}~\begin{quote}\begin{description}
\item[{Arguments}] \leavevmode\begin{itemize}
\item {} 
\sphinxstyleliteralstrong{\sphinxupquote{args}} (\sphinxstyleliteralemphasis{\sphinxupquote{object}}) \textendash{} \sphinxstylestrong{title} (first row) of the notification, \sphinxstylestrong{message} (second row) of the notification, in a standard notification.

\end{itemize}

\end{description}\end{quote}

\end{fulllineitems}


A notification is a message you can display to the user outside of your
application’s normal UI. When you tell the system to issue a notification, it
first appears as an icon in the notification area. To see the details of the
notification, the user opens the notification drawer. Both the notification
area and the notification drawer are system-controlled areas that the user can
view at any time.

\fvset{hllines={, ,}}%
\begin{sphinxVerbatim}[commandchars=\\\{\}]
\PYG{n+nx}{mobile}\PYG{p}{.}\PYG{n+nx}{showNotification}\PYG{p}{(}\PYG{p}{\PYGZob{}}\PYG{l+s+s1}{\PYGZsq{}title\PYGZsq{}}\PYG{o}{:} \PYG{l+s+s1}{\PYGZsq{}Simple Notification\PYGZsq{}}\PYG{p}{,} \PYG{l+s+s1}{\PYGZsq{}message\PYGZsq{}}\PYG{o}{:} \PYG{l+s+s1}{\PYGZsq{}This is a test for a simple notification\PYGZsq{}}\PYG{p}{\PYGZcb{}}\PYG{p}{)}
\end{sphinxVerbatim}

\noindent\sphinxincludegraphics{{mobile_notification}.png}


\paragraph{Create contact in device}
\label{\detokenize{reference/mobile:create-contact-in-device}}\index{addContact() (built-in function)}

\begin{fulllineitems}
\phantomsection\label{\detokenize{reference/mobile:addContact}}\pysiglinewithargsret{\sphinxbfcode{\sphinxupquote{addContact}}}{}{}~\begin{quote}\begin{description}
\item[{Arguments}] \leavevmode\begin{itemize}
\item {} 
\sphinxstyleliteralstrong{\sphinxupquote{args}} (\sphinxstyleliteralemphasis{\sphinxupquote{object}}) \textendash{} Dictionary with contact details. Possible keys (name, mobile, phone, fax, email, website, street, street2, country\_id, state\_id, city, zip, parent\_id, function and image)

\end{itemize}

\end{description}\end{quote}

\end{fulllineitems}


Create a new device contact with the given contact details.

\fvset{hllines={, ,}}%
\begin{sphinxVerbatim}[commandchars=\\\{\}]
\PYG{k+kd}{var} \PYG{n+nx}{contact} \PYG{o}{=} \PYG{p}{\PYGZob{}}
        \PYG{l+s+s1}{\PYGZsq{}name\PYGZsq{}}\PYG{o}{:} \PYG{l+s+s1}{\PYGZsq{}Michel Fletcher\PYGZsq{}}\PYG{p}{,}
        \PYG{l+s+s1}{\PYGZsq{}mobile\PYGZsq{}}\PYG{o}{:} \PYG{l+s+s1}{\PYGZsq{}9999999999\PYGZsq{}}\PYG{p}{,}
        \PYG{l+s+s1}{\PYGZsq{}phone\PYGZsq{}}\PYG{o}{:} \PYG{l+s+s1}{\PYGZsq{}7954856587\PYGZsq{}}\PYG{p}{,}
        \PYG{l+s+s1}{\PYGZsq{}fax\PYGZsq{}}\PYG{o}{:} \PYG{l+s+s1}{\PYGZsq{}765898745\PYGZsq{}}\PYG{p}{,}
        \PYG{l+s+s1}{\PYGZsq{}email\PYGZsq{}}\PYG{o}{:} \PYG{l+s+s1}{\PYGZsq{}michel.fletcher@agrolait.example.com\PYGZsq{}}\PYG{p}{,}
        \PYG{l+s+s1}{\PYGZsq{}website\PYGZsq{}}\PYG{o}{:} \PYG{l+s+s1}{\PYGZsq{}http://www.agrolait.com\PYGZsq{}}\PYG{p}{,}
        \PYG{l+s+s1}{\PYGZsq{}street\PYGZsq{}}\PYG{o}{:} \PYG{l+s+s1}{\PYGZsq{}69 rue de Namur\PYGZsq{}}\PYG{p}{,}
        \PYG{l+s+s1}{\PYGZsq{}street2\PYGZsq{}}\PYG{o}{:} \PYG{k+kc}{false}\PYG{p}{,}
        \PYG{l+s+s1}{\PYGZsq{}country\PYGZus{}id\PYGZsq{}}\PYG{o}{:} \PYG{p}{[}\PYG{l+m+mi}{21}\PYG{p}{,} \PYG{l+s+s1}{\PYGZsq{}Belgium\PYGZsq{}}\PYG{p}{]}\PYG{p}{,}
        \PYG{l+s+s1}{\PYGZsq{}state\PYGZus{}id\PYGZsq{}}\PYG{o}{:} \PYG{k+kc}{false}\PYG{p}{,}
        \PYG{l+s+s1}{\PYGZsq{}city\PYGZsq{}}\PYG{o}{:} \PYG{l+s+s1}{\PYGZsq{}Wavre\PYGZsq{}}\PYG{p}{,}
        \PYG{l+s+s1}{\PYGZsq{}zip\PYGZsq{}}\PYG{o}{:} \PYG{l+s+s1}{\PYGZsq{}1300\PYGZsq{}}\PYG{p}{,}
        \PYG{l+s+s1}{\PYGZsq{}parent\PYGZus{}id\PYGZsq{}}\PYG{o}{:} \PYG{p}{[}\PYG{l+m+mi}{8}\PYG{p}{,} \PYG{l+s+s1}{\PYGZsq{}Agrolait\PYGZsq{}}\PYG{p}{]}\PYG{p}{,}
        \PYG{l+s+s1}{\PYGZsq{}function\PYGZsq{}}\PYG{o}{:} \PYG{l+s+s1}{\PYGZsq{}Analyst\PYGZsq{}}\PYG{p}{,}
        \PYG{l+s+s1}{\PYGZsq{}image\PYGZsq{}}\PYG{o}{:} \PYG{l+s+s1}{\PYGZsq{}\PYGZlt{}\PYGZlt{}BASE 64 Image Data\PYGZgt{}\PYGZgt{}\PYGZsq{}}
\PYG{p}{\PYGZcb{}}

\PYG{n+nx}{mobile}\PYG{p}{.}\PYG{n+nx}{methods}\PYG{p}{.}\PYG{n+nx}{addContact}\PYG{p}{(}\PYG{n+nx}{contact}\PYG{p}{)}\PYG{p}{;}
\end{sphinxVerbatim}

\noindent\sphinxincludegraphics{{mobile_contact_create}.png}


\paragraph{Scanning barcodes}
\label{\detokenize{reference/mobile:scanning-barcodes}}\index{scanBarcode() (built-in function)}

\begin{fulllineitems}
\phantomsection\label{\detokenize{reference/mobile:scanBarcode}}\pysiglinewithargsret{\sphinxbfcode{\sphinxupquote{scanBarcode}}}{}{}~\begin{quote}\begin{description}
\item[{Returns}] \leavevmode
Scanned \sphinxcode{\sphinxupquote{code}} from any barcode

\end{description}\end{quote}

\end{fulllineitems}


The barcode API detects barcodes in real-time, on the device, in any orientation.

The barcode API can read the following barcode formats:
\begin{itemize}
\item {} 
1D barcodes: EAN-13, EAN-8, UPC-A, UPC-E, Code-39, Code-93, Code-128, ITF, Codabar

\item {} 
2D barcodes: QR Code, Data Matrix, PDF-417, AZTEC

\end{itemize}

\fvset{hllines={, ,}}%
\begin{sphinxVerbatim}[commandchars=\\\{\}]
\PYG{n+nx}{mobile}\PYG{p}{.}\PYG{n+nx}{methods}\PYG{p}{.}\PYG{n+nx}{scanBarcode}\PYG{p}{(}\PYG{p}{)}\PYG{p}{.}\PYG{n+nx}{then}\PYG{p}{(}\PYG{k+kd}{function}\PYG{p}{(}\PYG{n+nx}{code}\PYG{p}{)}\PYG{p}{\PYGZob{}}
        \PYG{k}{if}\PYG{p}{(}\PYG{n+nx}{code}\PYG{p}{)}\PYG{p}{\PYGZob{}}
                \PYG{c+c1}{// Perform operation with the scanned code}
        \PYG{p}{\PYGZcb{}}
\PYG{p}{\PYGZcb{}}\PYG{p}{)}\PYG{p}{;}
\end{sphinxVerbatim}


\paragraph{Switching account in device}
\label{\detokenize{reference/mobile:switching-account-in-device}}\index{switchAccount() (built-in function)}

\begin{fulllineitems}
\phantomsection\label{\detokenize{reference/mobile:switchAccount}}\pysiglinewithargsret{\sphinxbfcode{\sphinxupquote{switchAccount}}}{}{}
\end{fulllineitems}


Use switchAccount to switch from one account to another on the device.

\fvset{hllines={, ,}}%
\begin{sphinxVerbatim}[commandchars=\\\{\}]
\PYG{n+nx}{mobile}\PYG{p}{.}\PYG{n+nx}{methods}\PYG{p}{.}\PYG{n+nx}{switchAccount}\PYG{p}{(}\PYG{p}{)}\PYG{p}{;}
\end{sphinxVerbatim}

\noindent\sphinxincludegraphics{{mobile_switch_account}.png}


\renewcommand{\indexname}{Python Module Index}
\begin{sphinxtheindex}
\def\bigletter#1{{\Large\sffamily#1}\nopagebreak\vspace{1mm}}
\bigletter{o}
\item {\sphinxstyleindexentry{odoo.api}}\sphinxstyleindexpageref{reference/orm:\detokenize{module-odoo.api}}
\end{sphinxtheindex}

\renewcommand{\indexname}{Index}
\printindex
\end{document}